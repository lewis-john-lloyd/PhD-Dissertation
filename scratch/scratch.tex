\documentclass[12pt,letterpaper]{article}
\usepackage[latin1]{inputenc}
\usepackage{amsmath}
\usepackage{amsfonts}
\usepackage{amssymb}
\usepackage[left=0.5in,right=0.5in,top=1in,bottom=1in]{geometry}
\usepackage{ifthen}
\usepackage{IEEEtrantools}
\DeclareRobustCommand{\Eqn}[1]{Equation #1}
\DeclareRobustCommand{\Eqns}[2]{Equations #1 and #2}
\DeclareRobustCommand{\Eqnsthree}[3]{Equations #1, #2, and #3}
\DeclareRobustCommand{\Eqnatob}[2]{Equations #1 - #2}

\begin{document}
The analytic wall and form loss frictional pressure drops for single phase flow are expressed in \Eqn{\ref{velocity_based-pressure_drop}}.

\begin{IEEEeqnarray}{rCL}
\nonumber
\frac{d P}{d z} & = & \frac{d P}{d z}\bigg\vert_{fric}+\frac{d P}{d z}
\bigg\vert_{form} \\
\nonumber
& = & \frac{f(Re)}{D_{H}}\frac{\rho U^2}{2} + \frac{K}{\Delta z}\frac
{\rho U^2}{2} \\
\label{velocity_based-pressure_drop}
& = & \left(\frac{f(Re)}{D_{H}} + \frac{K}{\Delta z}\right)\frac{\rho U^2}{2}
\end{IEEEeqnarray}

An alternative formulation utilizes the mass flow rate as an independent variable instead of the velocity.
The mass flow rate is related to the velocity via \Eqn{\ref{mass_flow_rate_definition}}.

\begin{IEEEeqnarray}{rCL}
\label{mass_flow_rate_definition}
\dot{m} & = & \rho U A
\end{IEEEeqnarray}

Using \Eqns{\ref{velocity_based-pressure_drop}}{\ref{mass_flow_rate_definition}}, the combined pressure drop can be expressed as in \Eqn{\ref{mass_flow_rate_based_pressure_drop}}.

\begin{IEEEeqnarray}{rCL}
\label{mass_flow_rate_based_pressure_drop}
\frac{d p}{d z} & = & \left(\frac{f(Re)}{D_{H}} + \frac{K}{\Delta z}\right)\frac{\dot{m}^2}{2\rho A^2_{mom}}
\end{IEEEeqnarray}

This relationship was developed for single phase flow.
In multi-phase flow, the frictional and form loss pressure drops are treated differently.
The phasic wall friction factor is subject to a two-phase friction multiplier, $\Phi^2_k$, which is flow-regime dependent.

\begin{IEEEeqnarray}{rCL}
\frac{d P}{d z}\bigg\vert_{k,fric} & = & \frac{f(Re_k)\dot{m}_k^2}{2 \rho_k D_{H} A^2_{mom}}\Phi ^2_k
\end{IEEEeqnarray}

The phasic form loss pressure drop is proportional to the volume fraction of that phase.

\begin{IEEEeqnarray}{rCL}
\nonumber
\frac{d P}{d z}\bigg\vert_{k,form} & = & \alpha_k \frac{d P}{d z}\bigg\vert_{form} \\
& = & \alpha_k \frac{ K \dot{m}_k^2}{2 \rho_k \Delta z A^2_{mom}}
\end{IEEEeqnarray}

Phasic velocities are defined as follows, where $\dot{M}_k$ is the phasic mass flow rate.

\begin{IEEEeqnarray}{rCL}
\dot{M}_{k} & \equiv & \alpha_k \rho_k U_k A \\
U_k & = & \frac{\dot{M}_{k}}{\alpha_k \rho_k A}
\end{IEEEeqnarray}

The combined pressure loss for friction and form is given in \Eqn{\ref{two_phase_pressure_drop}}.

\begin{IEEEeqnarray}{rCL}
\nonumber
\frac{d P}{d z}\bigg\vert_{k} & = & \frac{d P}{d z}\bigg\vert_{k,form} + \frac{d P}{d z}\bigg\vert_{k, fric} \\
\nonumber
& = & \alpha_k \frac{ K \dot{m}_k^2}{2 \rho_k \Delta z A^2_{mom}} + \frac{f(Re_k)\dot{m}_k^2}{2 \rho_k D_{H} A^2_{mom}}\Phi^2_k \\
\label{two_phase_pressure_drop}
& = & \left(\alpha_k \frac{K}{\Delta z} + \Phi^2_k \frac{f(Re_k)}{D_{H}}\right)\frac{\dot{m}_k^2}{2 \rho_k A^2_{mom}}
\end{IEEEeqnarray}

In COBRA, we want the following product:
\begin{IEEEeqnarray} {rCL}
\label{cobraploss}
V \frac{d P}{d z}\bigg\vert_{k} & = & A \Delta z \frac{d p}{d z}\bigg\vert_{k}
\end{IEEEeqnarray}

Combing \Eqns{\ref{two_phase_pressure_drop}}{\ref{cobraploss}} produces the following expression.

\begin{IEEEeqnarray}{rCL}
\label{yarble}
V \frac{d P}{d z}\bigg\vert_{k} & = & \left(\alpha_k K + \Phi^2_k f(Re_k)\frac{\Delta z}{D_{H} }\right)\frac{\dot{m}_k^2}{2 \rho_k A_{mom}}
\end{IEEEeqnarray}

From \Eqn{\ref{yarble}}, we will then move to a discrete formulation.
Since COBRA uses a staggered grid, not all terms are collocated spatially, this requires an approximation to be made.
For a given variable $x$, we define the $\left< \; \right>_{ave}$ operator as shown in \Eqn{\ref{ave_operator}}.

\begin{IEEEeqnarray}{rCL}
\label{ave_operator}
\left< x_k \right>_{ave} & = & \frac{x_{k,j} + x_{k,j+1}}{2}
\end{IEEEeqnarray}

In COBRA, there is a distinction made between mass flow rates, $\dot{m}$ and momentum flow rates, $\dot{M}$.
This distinction is as follows:

\begin{IEEEeqnarray}{lCCCL}
\textbf{Mass Flow Rate} & \equiv & \dot{m}^{n+1}_k & = & U^{n+1}_k \left<\alpha^n_k \rho^n_k\right>^{n+1}_{don} A_{mom}\\
\textbf{Momentum Flow Rate} & \equiv & \dot{M}^{n+1}_k & \equiv & U^{n+1}_k \left<\alpha^n_k \rho^n_k\right>_{ave} A_{mom}
\end{IEEEeqnarray}

Where the donored macroscopic density is defined as follows:
\begin{equation}
\label{donor_eqn}
\left<\alpha^n_k \rho^n_k\right>^{n+1}_{don} = \left\{
\begin{array}{rl}
\alpha^{n}_{k,j}\rho^{n}_{k,j} & \text{if } U^{n+1}_k >= 0,\\
\alpha^{n}_{k,j+1}\rho^{n}_{k,j+1} & \text{if } U^{n+1}_k < 0
\end{array} \right.
\end{equation}

\Eqnatob{\ref{yarble}}{\ref{donor_eqn}} yield the following definition:

\begin{IEEEeqnarray} {rCL}
\label{good_presure_drop_00}
V \frac{d P}{d z}\bigg\vert_{k} & = & \left(\left<\alpha_k\right>_{ave} K + \Phi^2_k f(Re_k) \frac{\Delta z}{D_{H}}\right)\frac{\dot{M}_k^2}{2\left<\rho_k\right>_{ave} A_{mom}}
\end{IEEEeqnarray}

The time discretization of this term is as follows:

\begin{IEEEeqnarray}{rCL}
\label{good_presure_drop_01}
V \frac{d P}{d z}\bigg\vert^{n+1}_{k} & = & \left( \left< \alpha^n_k \right>_{ave} K^{n} + (\Phi^2_k)^{n}f(Re^{n}_{k})\frac{\Delta z}{D_{H}}\right)\frac{(\dot{M}^{n+1}_k)^2}{2 \left< \rho^{n}_{k} \right>_{ave} A_{mom}}
\end{IEEEeqnarray}

Currently, COBRA does not give the same definition as \Eqn{\ref{good_presure_drop_01}}.
To obtain the formulation used in COBRA, a limiter function needs to be defined.

\begin{IEEEeqnarray}{rCL}
\label{limiter_func}
\Psi(x, y) & = & \mathbf{max} [\mathbf{abs}(x), U_{min} y]
\end{IEEEeqnarray}

\Eqn{\ref{limiter_func}} allows for the relationship in \Eqn{\ref{relate_limiters}}.

\begin{IEEEeqnarray} {rCL}
\Psi(\dot{M}, \left< \alpha_k \rho_k \right>_{ave} A_{mom}) & = & \mathbf{max}[\mathbf{abs}(U_k \left< \alpha_k \rho_k \right>_{ave} A_{mom}), U_{min} \left< \alpha_k \rho_k \right>_{ave} A_{mom}]\\
& = & \left< \alpha_k \rho_k \right>_{ave} A_{mom} \mathbf{max}[\mathbf{abs}(U_k),U_{min}]\\
\label{relate_limiters}
& = & \left< \alpha_k \rho_k \right>_{ave} A_{mom} \Psi(U_k)
\end{IEEEeqnarray}

At the current moment, COBRA provides the following:

\begin{IEEEeqnarray} {rCL}
\label{step_01}
V_{mom} \frac{d P}{d z}\bigg\vert^{n+1}_{k}& = & \left[\frac{K^{n}}{2}\Psi(U^{n+1}_k) + \frac{(\Phi^2_k)^{n} f(Re^{n}_k)\Delta z }{2 D_{H}\left< \rho^n_k \right>_{ave} A_{mom}} \frac{\Psi(\dot{m}^{n+1}_{k},\left< \alpha^n_k \rho^n_k \right>_{ave} , A_{mom})^2}{\Psi(\dot{M}^{n+1}_{k} ,\left< \alpha^n_k \rho^n_k \right>_{ave}, A_{mom})}\right]\dot{M}^{n+1}_{k}
\end{IEEEeqnarray}

We will define the following:

\begin{IEEEeqnarray}{rCL}
\label{limit_ratio}
\gamma^{n+1}_{k} & \equiv & \frac{\Psi(\dot{m}^{n+1}_k, \left< \alpha^n_k \rho^n_k \right>_{ave} , A_{mom})^2}{\Psi(\dot{M}^{n+1}_k, \left<\alpha^n_k \rho^n_k \right>_{ave},A_{mom})}
\end{IEEEeqnarray}

\Eqnsthree{\ref{relate_limiters}}{\ref{step_01}}{\ref{limit_ratio}} yield \Eqn{\ref{cobra_pressure_drop}}.

\begin{IEEEeqnarray}{rCL}
\label{cobra_pressure_drop}
V_{mom} \frac{d P}{d z}\bigg\vert^{n+1}_{k} & = & \left[ \frac{K^{n}}{\left< \alpha^n_k \rho^n_k \right>_{ave}}\Psi(M^{n+1}_k, \left< \alpha^n_k \rho^n_k \right>_{ave}, A_{mom}) + (\Phi^{2}_k)^{n}f(Re^{n}_k)\frac{\Delta z}{D_{H} \left< \rho^n_k \right>_{ave}} \gamma^{n+1}_{k}\right]\frac{\dot{M}^{n+1}_{k}}{2 A_{mom}}
\end{IEEEeqnarray}

Some things to note:

\Eqns{\ref{good_presure_drop_01}} {\ref{cobra_pressure_drop}} are different.
\Eqn{\ref{good_presure_drop_01}} can be stated as follows:

\begin{IEEEeqnarray}{rCL}
\label{pressure_drop_coefficient}
V \frac{d P}{d z}\bigg\vert^{n+1}_{k} & = & \beta_k^{n}(\dot{M}^{n+1}_k)^2 \\
\label{dpdz_coefficient}
\beta_k^{n} & = & \left( \left< \alpha^n_k \right>_{ave} K^{n} + (\Phi^2_k)^{n} f(Re^n_k)\frac{\Delta z}{D_{H}} \right)\frac{1}{2 \left< \rho^{n}_k \right>_{ave} A_{mom}}
\end{IEEEeqnarray}

Since we are currently using $\gamma^{n+1}_k$ for stability, and we have agreed to keep using it, I propose that we modify COBRA to calculate \Eqn{\ref{new_dpdz_01}} instead of \Eqn{\ref{cobra_pressure_drop}}.

\begin{IEEEeqnarray}{rCL}
\label{new_dpdz_01}
V_{mom} \frac{d P}{d z}\bigg\vert^{n+1}_{k} & = & \left[ \left< \alpha^n_k \right>_{ave} K^{n} + (\Phi^{2}_k)^{n} f(Re^{n}_k) \frac{\Delta z }{D_{H}} \right] \gamma^{n+1}_k \frac{\dot{M}^{n+1}_{k}}{2 \left<\rho^n_k\right>_{ave} A_{mom}}
\end{IEEEeqnarray}

By using \Eqn{\ref{new_dpdz_01}} instead of \Eqn{\ref{cobra_pressure_drop}}, we can use \Eqn{\ref{dpdz_coefficient}} as follows:

\begin{IEEEeqnarray}{rCL}
\label{new_dpdz_02}
V_{mom} \frac{d P}{d z}\bigg\vert^{n+1}_{k} & = & \beta^{n}_k \gamma^{n+1}_{k}\dot{M}^{n+1}_{k}
\end{IEEEeqnarray}

Should it be decided that the limiter defined in \Eqn{\ref{limiter_func}} no longer be needed in an iterative solution algorithm,
we could transition from \Eqn{\ref{new_dpdz_02}} to \Eqn{\ref{pressure_drop_coefficient}}.

\begin{IEEEeqnarray}{rCCCL}
V_{mom} \frac{d P}{d z}\bigg\vert^{n+1}_{k} & = & \beta^{n}_k \gamma^{n+1}_k \dot{M}^{n+1}_{k} & \Rightarrow & \beta_k^{n} (\dot{M}^{n+1}_k)^2
\end{IEEEeqnarray}

Also note that the wall drag portion of \Eqn{\ref{cobra_pressure_drop}} is related to the proposed form in \Eqn{\ref{new_dpdz_02}} as follows:

\begin{IEEEeqnarray}{rCL}
\nonumber
\underbrace{V_{mom} \frac{d P}{d z}\bigg\vert^{n+1}_{k,form}}_{\textbf{COBRA}} & = & \frac{K^{n}}{\left<\alpha^n_k\rho^n_k\right>_{ave}} \\
\nonumber
& = & \frac{\left< \rho^n_k \right>_{ave}}{\left< \alpha^n_k \right>_{ave}\left< \alpha^n_k \rho^n_k \right>_{ave} }\frac{ \left< \alpha^n_k \right>_{ave} K^{n}}{\left< \rho^n_k \right>_{ave}} \\
\label{relation}
& = & \underbrace{V_{mom} \frac{d P}{d z}\bigg\vert^{n+1}_{k,form}}_{\textbf{PROPOSED}}\frac{ \left< \rho^n_k \right>_{ave}}{\left< \alpha^n_k \right>_{ave}\left< \alpha^n_k \rho^n_k \right>_{ave}}
\end{IEEEeqnarray}

An interesting observation is that if $\left< \alpha_k \rho_k \right>_{ave} = \left<\alpha_k\right>_{ave}\left< \rho_k \right>_{ave}$, then \Eqn{\ref{relation}} is as follows:

\begin{IEEEeqnarray}{rCCCL}
\underbrace{V_{mom} \frac{d P}{d z}\bigg\vert ^{n+1}_{k,form}}_{\textbf{COBRA}} & = & \underbrace{V_{mom} \frac{d p}{d z}\bigg\vert^{n+1}_{k,form}}_{\textbf{PROPOSED}} \frac{1}{ \left<\alpha^n_k\right>^2_{ave}}
\end{IEEEeqnarray}

So, if somewhere in the past, someone decided that the two-phase multiplier should be applied to every phase of the analytic form of the form-loss pressure drop terms and that the average of the products is the product of the averages, then they would have ended up with what we have in COBRA today.
\end{document}