\chapter{Domain Decomposition}
\label{chap:domDecomposition}
As discussed in \sect{sect:code_coupling}, domain coupling in thermal-hydraulic, safety-analysis software is an established method to allow different physics to be represented in different domains.
This research extends the domain-coupling framework to allow different domains to be consistently coupled while subject to different mathematical treatments.
The following chapter describes the mathematical formulation of this novel domain decomposition algorithm, several architectural changes that were made to \cobra{} to allow for the algorithm to be implemented properly, and a description of its implementation in the software.

%-------------------------------------------------------------------------------
%-------------------------------------------------------------------------------
%-------------------------------------------------------------------------------
\section{Mathematical Formulation}
\label{sec:domDecompMath}

The governing partial differential equations described in \sect{subsect:governing_equations} contain nonlinearities that are distributed in both space and time.
In the absence of nonlinear physics, a single Newton step is adequate to accurately solve the governing set of discrete algebraic equations.
However, if the number of Newton steps is limited to one for a given spatial mesh when nonlinearities are present, then the only way to resolve those nonlinearities is to refine the temporal discretization.
If the number of Newton steps is not limited to one, then the nonlinearities may be resolved for a fixed temporal discretization.
When the nonlinearities are isolated to a given spatial portion of the domain, the additional Newton steps do not improve the solution in the remainder of the domain.
The ability to resolve the nonlinear physics only where they occur could potentially provide a way to reduce the computational cost associated with resolving the nonlinear error in a problem at a given timestep.

With the selective nonlinear refinement algorithm, the domain is split into two segments, a linear domain and a nonlinear domain.
There is no requirement that these two domains be contiguous.
Each continuity volume and momentum flow path is characterized as either existing in the linear domain or the nonlinear domain.
Each momentum flow path is associated with a single domain; the momentum equations are not modified, regardless of the domain in which they exist.
The coupling between the two domains occurs at the continuity volume's boundaries connecting the linear domain  to the nonlinear domain.
Those continuity volumes that are in the linear domain, but have flow paths connecting them to the nonlinear domain, are considered nonlinear boundary continuity (NBC) volumes.
All other continuity volumes have associated with them the nonlinear continuity equations, \eqref{eqn:nlnNcgMassEquation} -- \eqref{eqn:nlnVapMassEquation}, regardless of the domain in which they exist.
The NBC volumes have a modified set of conservation equations that will now be discussed.

The starting place for the formulation of the NBC volumes' continuity equations will be \eqref{eqn:nlnNcgMassEquation} -- \eqref{eqn:nlnVapMassEquation}.
In these normal nonlinear continuity equations the advection terms are formulated in terms of new-time velocities and donored quantities evaluated using the new-time velocities.
However, in an NBC volume, the flow paths that connect linear continuity volumes to nonlinear continuity volumes are formulated in terms of phasic mass and energy flow rates.
This formulation introduces six new unknowns per domain connection, the flow rates.
The index $i$ represents the spatial coordinate of the flow rate.
The definition of the six unknown flow rates for a given location is shown in \eqref{eqn:nbcFluxDefinition}.

\begin{equation}
\label{eqn:nbcFluxDefinition}
\vec{\Psi}^{n+1, k}_{i} = \begin{bmatrix}
\Psi_{m, n} \\
\Psi_{m, l} \\
\Psi_{h, g} \\
\Psi_{h, l} \\
\Psi_{m, e} \\
\Psi_{m, v} \\
\end{bmatrix}^{n+1, k}_{i} = \begin{bmatrix}
\frac{\don{\alpha^{n}_{g} \rho^{n}_{n}}^{n+1,k}_{d}}{\ave{\alpha_{g} \rho_{g}}^{n}_{a}}\dot{m}^{n+1,k}_{g} \\
%
\frac{\don{\alpha^{n}_{l}\rho^{n}_{l}}^{n+1,k}_{d}}{\ave{\alpha_{l} \rho_{l}}^{n}_{a}}\dot{m}^{n+1,k}_{l}\\
%
\frac{\don{\alpha^{n}_{g} \rho^{n}_{g} h^{n}_{g}}^{n+1,k}_{d}}{\ave{\alpha_{g} \rho_{g}}^{n}_{a}}\dot{m}^{n+1, k}_{g}\\
%
\frac{\don{\alpha^{n}_{l}\rho^{n}_{l} h^{n}_{l}}^{n+1,k}_{d}}{\ave{\alpha_{l} \rho_{l}}^{n}_{a}}\dot{m}^{n+1, k}_{l} +\frac{\don{\alpha^{n}_{e} \rho^{n}_{l} h^{n}_{l}}^{n+1, k}_{d}}{\ave{\alpha_{e} \rho_{l}}^{n}_{a}}\dot{m}^{n+1, k}_{e} \\
%
\frac{ \don{\alpha^{n}_{e} \rho^{n}_{l}}^{n+1, k}_{d}}{ \ave{\alpha_{e} \rho_{l}}^{n}_{a}} \dot{m}^{n+1, k}_{e} \\
%
\frac{ \don{\alpha^{n}_{g} \rho^{n}_{v}}^{n+1, k}_{d}}{ \ave{\alpha_{g} \rho_{g}}^{n}_{a}} \dot{m}^{n+1, k}_{g}
\end{bmatrix}_{i}
\end{equation}

Using \eqref{eqn:momentumToFlowRates} and \eqref{eqn:momentumVector}, the flow rates, \eqref{eqn:nbcFluxDefinition}, can be expressed as \eqref{eqn:nbcFluxMatrixStyle}.

\begin{equation}
\label{eqn:nbcFluxMatrixStyle}
\vec{\Psi}^{n+1, k}_{i} = \vec{\Xi}^{n+1, k}_{i} \cdot \vec{\dot{m}}^{n+1, k}_{i}
\end{equation}

The corresponding continuity equations for a given NBC volume in residual form are given by \eqref{eqn:nbcNcgMassEquation} -- \eqref{eqn:nbcVapMassEquation}.
In these equations, the set of flow paths connecting an NBC volume to other continuity volumes within the linear domain is given by $N_{f}$.
The set of inter-domain flow paths connecting an NBC volume to the nonlinear domain is given by $N_{\text{NBC}}$.

\begin{IEEEeqnarray}{rCl}
\label{eqn:nbcNcgMassEquation}
F^{k}_{m, n} & = & V_c\left[ (\alpha_g \rho_{n})^{n+1, k} -(\alpha_g \rho_{n})^{n}\right] +\dt{} \sum_{i\,\in \, N_{f} }\left( \don{ \alpha^{n}_g \rho^{n}_{n} }^{n+1,k}_{d} u^{n+1, k}_{g}  A_{p} \right)_{i} \nonumber \\
& + & \dt{} \sum_{i\, \in \, N_{\text{NBC}}} \left( \Psi^{n+1, k}_{m,n} \right)_{i} \\
\label{eqn:nbcLiqMassEquation}
F^{k}_{m, l} & = & V_c \left(\alpha_l \rho_l \right)^{n+1,k} - V_c \left(\alpha_l \rho_l \right)^{n} + \dt{} \sum_{i\,\in \,N_{f}} \left(\don{\alpha^n_l \rho^n_l}^{n+1,k}_{d} u^{n+1, k}_l A_{p}  \right)_{i} \nonumber \\
&+& \left[(1-\eta)\Gamma + \Upsilon \right]^{n+1, k} +  \dt{} \sum_{i\,\in \, N_{\text{NBC}} } \left( \Psi^{n+1, k}_{m,l} \right)_{i}  \\
\label{eqn:nbcGasEnergyEquation}
F^{k}_{h, g} & = & V_c \left[\left( \alpha_g \rho_g h_g \right)^{n+1, k} - \left( \alpha_g \rho_g h_g \right)^{n} - \alpha^{n}_{g} ( P^{\,n+1, k} - P^{\,n} ) \right] - \dt{} q_{wg}^{n} \nonumber \\
& - & \dt{} \left[q_{i,v} + \dot{\Gamma} h^{'}_v + q_{gl}\right]^{n+1, k} + \dt{} \sum_{i \, \in \, N_{f} } \left( \don{ \alpha^{n}_g \rho^{n}_g h_g^{n} }^{n+1,k}_{d} u^{n+1, k}_g  A_{p}  \right)_{i} \nonumber \\
& + &  \dt{} \sum_{i\, \in \, N_{\text{NBC}} } \left( \Psi^{n+1, k}_{h,g} \right)_{i} \\
\label{eqn:nbcLiqEnergyEquation}
F^{k}_{h, l} & = & V_c\left[\left( \alpha_l \rho_l h_l \right)^{n+1,k} - \left( \alpha_l \rho_l h_l \right)^{n} - \alpha^{n}_l (P^{\,n+1,k} - P^{\,n})\right] - \dt{} \left[q_{i,l} -\dot{\Gamma} h^{'}_l - q_{gl}\right]^{n+1,k}    \nonumber \\
& +& \dt{} \sum_{i\,\in \, N_{f} } \left( \don{ \alpha^{n}_l \rho^{n}_l h^{n}_l }^{n+1,k}_{d} u^{n+1,k}_l A_{p} + \don{ \alpha^{n}_e \rho^{n}_l h^{n}_l }^{n+1,k}_{d} u^{n+1,k}_e A_{p} \right)_{i} \nonumber \\
& + &  \dt{} \sum_{i \,\in \, N_{\text{NBC}} } \left( \Psi^{n+1, k}_{h,l} \right)_{i} \\
\label{eqn:nbcEntMassEquation}
F^{k}_{m, e} & = & V_c \left(\alpha_e \rho_l \right)^{n+1,k} - V_c \left(\alpha_e \rho_l \right)^{n} + \dt{} \sum_{i \, \in \, N_{f} } \left( \don{ \alpha^{n}_e \rho^{n}_l }^{n+1, k}_{d} u^{n+1,k}_e  A_{p} \right)_{i} \nonumber \\
&-& \left[\Upsilon -\eta\Gamma\right]^{n+1,k} +  \dt{} \sum_{i \, \in \, N_{\text{NBC}} } \left( \Psi^{n+1, k}_{m,e} \right)_{i}\\
\label{eqn:nbcVapMassEquation}
F^{k}_{m, v} & = & V_c \left[\left(\alpha_g \rho_v \right)^{n+1, k} - \left(\alpha_g \rho_v \right)^{n}\right] + \dt{} \sum_{i\, \in \, N_{f} } \left( \don{ \alpha^{n}_g \rho^{n}_v }^{n+1,k}_{d} u^{n+1, k}_{g}  A_{p} \right)_{i} - \Gamma^{n+1, k} \nonumber \\
& + & \dt{} \sum_{i \, \in \, N_{\text{NBC}} } \left( \Psi^{n+1, k}_{m,v} \right)_{i}
\end{IEEEeqnarray}

By treating the boundary flow rates as independent parameters, the linearized system given in \eqref{eqn:nlnContinuitySystem} is modified to be \eqref{eqn:nbcContinuitySystem}.

\begin{IEEEeqnarray}{rcl}
\label{eqn:nbcContinuitySystem}
\frac{\partial \vec{F}^{k}_{c}}{\partial (\alpha_{g} P_{n} )} \delta (\alpha_{g} P_{n})^{k} + \frac{\partial \vec{F}^{k}_{c}}{\partial \alpha_{g}} \delta \alpha^{k}_{g} + \frac{\partial \vec{F}^{k}_{c}}{\partial (\alpha_{g} h_{v} )} \delta (\alpha_{g} h_{v})^{k} + \frac{\partial \vec{F}^{k}_{c}}{\partial ((1 - \alpha_{g}) h_{l} )} \delta ((1 - \alpha_{g}) h_{l})^{k} & + &  \nonumber \\
\frac{\partial \vec{F}^{k}_{c}}{\partial \alpha_{e}} \delta \alpha_{e}^{k} + \frac{\partial \vec{F}^{k}_{c}}{\partial P } \delta P^{k} + \sum_{i \, \in \, N_{f} } \frac{\partial \vec{F}^{k}_{c}}{\partial \momVec{}_{i} } \delta \momVec{}_{i}^{k} + \sum_{i\,\in \, N_{\text{NBC}}} \frac{\partial \vec{F}^{k}_{c}}{\partial \vec{\Psi}_{i} } \delta \vec{\Psi}_{i}^{k}  = - \vec{F}^{k}_{c} & & 
\end{IEEEeqnarray}

Since the continuity equations are linear in the boundary flow rates, the matrix of derivatives of the NBC volumes' continuity equations with respect to a given boundary flow rate vector, $ \frac{\partial \vec{F}^{k}_{c}}{\partial \vec{\Psi}_{i} } $, is the identity matrix times the timestep size, $\dt{} \vec{I}$.
For algorithmic convenience, the flow rates in the continuity equations' residuals are kept as unknown parameters when solving the local linear system.
However, the residual used for convergence includes this term.
Utilizing \eqref{eqn:momentumToFlowRates} and \eqref{eqn:momentumUpdate}, the linear system for the NBC volume, \eqref{eqn:nbcContinuitySystem}, can be expressed as \eqref{eqn:nbcLinearSystem} in a manner analogous to that outlined in \sect{sect:nlnCobraSolver}.

\begin{IEEEeqnarray}{rcl}
\label{eqn:nbcLinearSystem}
\frac{\partial \vec{F}^{k}_{c}}{\partial (\alpha_{g} P_{n} )} \delta (\alpha_{g} P_{n})^{k} + \frac{\partial \vec{F}^{k}_{c}}{\partial \alpha_{g}} \delta \alpha^{k}_{g} + \frac{\partial \vec{F}^{k}_{c}}{\partial (\alpha_{g} h_{v} )} \delta (\alpha_{g} h_{v})^{k} + \frac{\partial \vec{F}^{k}_{c}}{\partial ((1 - \alpha_{g}) h_{l} )} \delta ((1 - \alpha_{g}) h_{l})^{k} & + & \nonumber \\
\frac{\partial \vec{F}^{k}_{c}}{\partial \alpha_{e}} \delta \alpha_{e}^{k} + \left( \frac{\partial \vec{F}^{k}_{c}}{\partial P } + \dt{} \sum_{i \, \in \, N_{f} } \vec{\Xi}^{k}_{i}\frac{\partial \momVec{}^{k}_{i}}{\partial P}\right) \delta P^{k} + \dt{} \sum_{i \, \in \, N_{f} } \vec{\Xi}^{k}_{i} \frac{\partial \momVec{}_{i}^{k}}{\partial P_{o(i)}} \delta P_{o(i)}^{k} & + &\nonumber \\
\dt{} \sum_{i \, \in \, N_{\text{NBC}}} \delta \vec{\Psi}^{k}_{i} = - \left( \vec{F}^{k}_{c} + \dt{} \sum_{i \, \in \, N_{f} } \vec{\Xi}^{k}_{i} \delta \momVec{}_{i}^{*} - \dt{} \sum_{i \, \in \, N_{\text{NBC}}} \vec{\Psi}^{k}_{i} \right) - \dt{} \sum_{i \, \in \, N_{\text{NBC}}} \vec{\Psi}^{k}_{i} & &
\end{IEEEeqnarray}

As outlined in \sect{sect:nlnCobraSolver}, the advection terms in the residuals of \eqref{eqn:nbcLinearSystem} are evaluated using $\momVec{}^{n+1, k + \onehalf}$.
The first six columns of \eqref{eqn:nbcLinearSystem} will be represented by $\vec{J}_{n}$.
The next $N_{f}$ inter-continuity volume coupling columns multiplying their respective pressure changes, $\delta P_{o(i)}$, will be represented by $\vec{K}_{c}$.
The final $6 * N_{\text{NBC}}$ columns representing the inter-domain coupling coefficients, which are multiplied by the change in flow rates, $\delta \vec{\Psi}_{i}$, will be collectively referred to as $\vec{Q}_{c}$.
This matrix, $\vec{Q}_{c}$, is also present with the opposite sign on the right-hand side of \eqref{eqn:nbcLinearSystem}.
Using this matrix notation, \eqref{eqn:nbcLinearSystem} can be represented as \eqref{eqn:nbcLinSystem}.

\begin{equation}
\label{eqn:nbcLinSystem}
\left[ \vec{J}_{c} \vert \vec{K}_{c} \vert \vec{Q}_{c} \right] \delta \vec{C}_{c} = \vec{r}_{c} - \vec{Q}_{c} \vec{\Psi}_{c}
\end{equation}

The vector of unknowns, $\vec{C}_{c}$, is defined in \eqref{eqn:nbcUpdate}.

\begin{equation}
\label{eqn:nbcUpdate}
\delta \vec{C}_{c} \equiv 
\begin{bmatrix}
\delta ( \alpha_{g} P_{n} ) \\
\delta \alpha_{g} \\
\delta ( \alpha_{g} h_v ) \\
\delta ( (1 - \alpha_{g} ) h_l ) \\
\delta \alpha_{e} \\
\delta P \\ 
\delta P_{o(1)} \\
\vdots \\
\delta P_{o(N_{f})} \\
\delta \vec{\Psi}_{1} \\
\vdots \\
\delta \vec{\Psi}_{N_{\text{NBC}}}
\end{bmatrix}
=
\begin{bmatrix}
( \alpha_{g} P_{n})^{n+1} - (\alpha_{g} P_{g} )^{n} \\
\alpha^{n+1}_{g} - \alpha^{n}_{g} \\
( \alpha_{g} h_{v} )^{n+1} - ( \alpha_{g} h_{v} )^{n} \\
( ( 1 - \alpha_{g} ) h_{l} )^{n+1} - ( ( 1 - \alpha_{g} ) h_{l} )^{n} \\
\alpha^{n+1}_{e,j} - \alpha^{n}_{e} \\
 P^{n+1} - P^{n} \\
 P_{o(1)}^{n+1} - P_{o(1)}^{n} \\
 \vdots \\
 P_{o(N_{f})}^{n+1} - P_{o(N_{f})}^{n} \\
 \vec{\Psi}_{1}^{n+1} - \vec{\Psi}_{1}^{n} \\
 \vdots \\
 \vec{\Psi}_{N_{\text{NBC}}}^{n+1} - \vec{\Psi}_{N_{\text{NBC}}}^{n}
\end{bmatrix}
\end{equation}

\eqref{eqn:nbcLinSystem} is then subjected to partial $\vec{LU}$ decomposition without pivoting, producing \eqref{eqn:nbcLUSystem}.

\begin{equation}
\label{eqn:nbcLUSystem}
\left[ \vec{U}_{c} \vert \vec{L}^{-1}_{c}\vec{K}_{c} \vert \vec{L}^{-1}_{c}\vec{Q}_{c} \right] \delta \vec{C}_{c} = \vec{L}^{-1}_{c}\vec{r}_{c}  -\vec{L}^{-1}_{c}\vec{Q}_{c}\vec{\Psi}_{c}
\end{equation}

However, since the matrix product $\vec{L}_{c}^{-1}\vec{Q}_{c}$ occurs twice in \eqref{eqn:nbcLUSystem}, only a single instance is included in the solution of this system.
The $\vec{L}^{-1}_{c}\vec{Q}_{c}$ is omitted from the right-hand side of \eqref{eqn:nbcLinSystem} during the $\vec{LU}$ decomposition.
This $\vec{LU}$ decomposition is such that the lower triangular matrix has ones along the diagonal.
The $N_{\text{NBC}}$ identity matrices that comprise $\vec{Q}_{c}$ are now themselves the inverse of the lower triangular matrix multiplied by \dt{}.
The final row of the linear system in \eqref{eqn:nbcLUSystem} is then scaled by the sixth row's sixth column entry, $\vec{U}_{c}[6,6]$.

The linear pressure matrix, $\vec{A}_{\text{lin}}$, is assembled from the sixth row of each continuity volume's linear system in the linear domain including the NBC volumes.
There will be $N_{\text{lin}}$ rows in the system, which correspond to each of the linear pressure updates.
Let $N_{n}$ represent the total number of interfaces between the linear and the nonlinear domains.
Since there can be multiple interfaces from a single NBC volume to the nonlinear domain, $N_{n}$ is greater than or equal to the number of NBC volumes.
There will be an additional $6 * N_{n}$ right-hand sides for the linear pressure matrix that correspond to the coefficient matrices, $\vec{L}^{-1}_{c}\vec{Q}_{c}$, for the unknown flow rates in each of the NBC volumes, represented as $\vec{B}_{\text{lin}}$.
The portion the pressure system that is from $\vec{L}^{-1}_{c} \vec{r}_{c}$ will be referred to as $\vec{res}_{\text{lin}}$.
The resulting system is shown in \eqref{eqn:linearPressureMatrix}.

\begin{equation}
\label{eqn:linearPressureMatrix}
\vec{A}_{\text{lin}} \delta \vec{P}_{\text{lin}} + \vec{B}_{\text{lin}} \delta \vec{\Psi}_{\text{lin}} = \vec{res}_{\text{lin}} - \vec{B}_{\text{lin}} \vec{\Psi}_{\text{lin}}
\end{equation}

The linear system in \eqref{eqn:linearPressureMatrix} is then multiplied by $\vec{A}^{-1}_{\text{lin}}$, resulting in \eqref{eqn:invertedLinearSystem}.

\begin{IEEEeqnarray}{rcl}
\delta \vec{P}_{\text{lin}} + \vec{A}^{-1}_{\text{lin}}\vec{B}_{\text{lin}} \delta \vec{\Psi}_{\text{lin}} & = & \vec{A}^{-1}_{\text{lin}}\vec{res}_{\text{lin}} - \vec{A}^{-1}_{\text{lin}}\vec{B}_{\text{lin}} \vec{\Psi}_{\text{lin}} \nonumber \\
\label{eqn:invertedLinearSystem}
\delta \vec{P}_{\text{lin}} + \vec{W}_{\text{lin}} \delta \vec{\Psi}_{\text{lin}} & = & \delta \vec{P}^{*}_{\text{lin}} - \vec{W}_{\text{lin}} \vec{\Psi}_{\text{lin}}
\end{IEEEeqnarray}

Once \eqref{eqn:linearPressureMatrix} has been inverted, the resulting pressure updates, $\delta \vec{P}_{\text{lin}}$, for the linear domain are functions of all of the unknown flow rates between the two domains and their updates.
The constant portion of \eqref{eqn:invertedLinearSystem} is shown as $\delta \vec{P}_{\text{lin}}^{*}$, representing the portion of the change in $\delta \vec{P}_{\text{lin}}$ that comes from the linear domain.
The matrix, $\vec{W}_{\text{lin}}$, represents the matrix of all inter-domain pressure coupling coefficients.
A representative row of \eqref{eqn:invertedLinearSystem} is shown in \eqref{eqn:linearDomainPressureUpdates}.
The vectors $\vec{w}_{j, i}$ represent the coupling coefficients for each NBC volume's pressure update.

\begin{equation}
\label{eqn:linearDomainPressureUpdates}
\delta P_{j} + \sum_{i\, \in \, N_{n}} \vec{w}^{T}_{j, i} \cdot \delta \vec{\Psi}_{i} = \delta P_{j}^{*} - \sum_{i\, \in \, N_{n}} \vec{w}^{T}_{j, i} \cdot{} \vec{\Psi}^{k}_{i}
\end{equation}

In the linear domain a single Newton step is taken to be the change from the old-time solution to the new-time solution, not an iterate.
This approach means that \eqref{eqn:linearDomainPressureUpdates} is interpreted not as an iterative solution but as a change from old-time to new-time values.
As a result, the pressure update is taken to be $\delta P_{j} = P_{j}^{n+1} - P_{j}^{n}$, the flow rates on the right-hand side are evaluated as $\vec{\Psi}^{n}$, and the flow rate updates are defined as $\delta \vec{\Psi} = \vec{\Psi}^{n+1} - \vec{\Psi}^{n}$.

Once the linear system has been inverted, the nonlinear pressure matrix is finalized.
Recall that the nonlinear continuity volumes' governing equations are not modified by the domain decomposition.
The nonlinear domain is composed of $N_{\text{nln}}$ continuity volumes.
Additionally, for each NBC volume there will be an extra row in the nonlinear pressure matrix, $\vec{A}_{\text{nln}}$, that corresponds to \eqref{eqn:linearDomainPressureUpdates}.
However, within the nonlinear domain, the flow rates are not kept as unknowns; instead, the flow rates in \eqref{eqn:linearDomainPressureUpdates} are expressed in terms of unknown pressures by using \eqref{eqn:momentumUpdate}, \eqref{eqn:nbcFluxMatrixStyle}, and \eqref{eqn:nbcFluxDerivative}.

\begin{equation}
\label{eqn:nbcFluxDerivative}
\delta \vec{\Psi} = \vec{\Xi}^{k} \cdot \delta \momVec{}
\end{equation}

The resulting equations for the pressure updates of the NBC volumes as viewed by the nonlinear domain are given by \eqref{eqn:nbcPressureEquation} 

\begin{IEEEeqnarray}{rcl}
\delta P_{j} + \sum_{i\,\in \, N_{n}} \vec{w}^{T}_{j,i} \vec{\Xi}^{k}_{i} \delta \vec{m}^{k}_{i} & = & \delta P^{*}_{j} - \sum_{i\,\in \, N_{n}} \vec{w}^{T}_{j,i} \vec{\Xi}_{i}^{k} \momVec{}^{k}_{i}  \nonumber \\
\delta P_{j} + \sum_{i\,\in \, N_{n}} \vec{w}^{T}_{j,i} \left[\vec{\Xi}_{i}^{k} \frac{\partial \momVec{}_{i}^{k}}{\partial P_{s(i)}} \delta P_{s(i)} + \vec{\Xi}_{i}^{k} \frac{\partial \momVec{}_{i}^{k}}{\partial P_{o(i)}} \delta P_{o(i)}\right] & = & \delta P^{*}_{j} - \sum_{i\,\in \, N_{n}} \vec{w}^{T}_{j,i} \left[ \vec{\Xi}_{i}^{k} \momVec{}^{k}_{i} + \vec{\Xi}_{i}^{k}\delta \momVec{}^{*}_{i} \right] \nonumber \\
\label{eqn:nbcPressureEquation}
\delta P_{j} + \sum_{i\,\in \, N_{n}} \vec{w}^{T}_{j,i} \left[\vec{\Xi}_{i}^{k} \frac{\partial \momVec{}_{i}^{k}}{\partial P_{s(i)}} \delta P_{s(i)} + \vec{\Xi}_{i}^{k} \frac{\partial \momVec{}_{i}^{k}}{\partial P_{o(i)}} \delta P_{o(i)}\right] & = & \delta P^{*}_{j} - \sum_{i\,\in \, N_{n}} \vec{w}^{T}_{j,i} \vec{\Xi}_{i}^{k} \momVec{}^{k + \onehalf}_{i}
\end{IEEEeqnarray}

The flux terms on the right-hand side of \eqref{eqn:nbcPressureEquation} are evaluated using $\momVec{}^{n+1,k+\onehalf}$.
The subscript $s(i)$ represents the NBC volume to which the flow path $i$ is connected.
The subscript $o(i)$ represents the nonlinear continuity volume to which the flow path $i$ is connected.
\eqref{eqn:nbcPressureEquation} shows that every NBC volume has a functional dependence upon the change in pressure of not only every other NBC volume but also every nonlinear continuity volume that is connected to an NBC volume.  
The final nonlinear pressure matrix will have $N_{\text{nln}} + N_{n}$ equations and unknowns and is represented by \eqref{eqn:nlnPressureMatrix}.

\begin{equation}
\label{eqn:nlnPressureMatrix}
\vec{A}^{k}_{\text{nln}} \delta \vec{P}^{k}_{\text{nln}} = \vec{res}^{k}_{\text{nln}}
\end{equation}

This system is then solved to obtain the pressure updates for the nonlinear domain.
The continuity volumes and flow paths in the nonlinear domain are updated using $\delta \vec{P}^{k}_{\text{nln}}$.
The NBC volumes' pressures are not updated because the old iterate, $k$, of a given NBC volume is always the old-time value to maintain consistency with the linear domain.
The nonlinear domain is subjected to multiple Newton iterates as outlined in \sect{sect:nlnCobraSolver}.
A deviation from the convergence procedure for a single Newton domain is that the norms used for convergence determination are formulated using the residuals and the updates from only that portion of the domain that is subject to the nonlinear solver.
These two vectors, the residual and the update, do not include terms from the NBC volumes.

Upon termination of the Newton iterations for the nonlinear domain, all of the inter-domain flow rates are now known.
These flow rates are used to obtain the linear domain's pressure updates.
The linear domain's pressure updates, $\delta P_{j}$, have their functional dependence resolved by using \eqref{eqn:linearPressureResolution}.

\begin{equation}
\label{eqn:linearPressureResolution}
\delta P_{j} = \delta P^{*}_{j} - \sum_{i\,\in \, N_{n}} \vec{w}_{j,i}^{T} \vec{\Xi}^{k}_{i} \left[ \momVec{}^{k}_{i} + \delta \momVec{}_{i}\right]
\end{equation}

Once the pressure update vector for the linear domain is obtained, all of linear domain's continuity volumes and flow paths are looped over.
During this loop, the new-time linear variables are calculated.
The completion of the linear update marks the completion of a single timestep.

%-------------------------------------------------------------------------------
%-------------------------------------------------------------------------------
%-------------------------------------------------------------------------------
\section{Implementation in \cobra{}}
\label{sec:dd_algo}

In this section, several architectural changes that were necessary to integrate the domain decomposition algorithm from \sect{sec:domDecompMath} into the \cobra{} software are addressed.
First, the modification of the data structures for the pressure matrices and their solvers is discussed.
Second, the object-oriented, volume data structures for domain decomposition are detailed.
Next, the implementation of domain decomposition algorithm will be shown in detail.
Lastly, the dual domain input file is described.

%-------------------------------------------------------------------------------
%-------------------------------------------------------------------------------
%-------------------------------------------------------------------------------
\subsection{Pressure Matrix and Solver Data Structures}
\label{subsect:domDecompSolverStructs}

As seen in \sect{sect:linCobraAlg}, the largest linear system that is solved during a Newton step is the pressure matrix that is used to obtain the pressure update vector.
\cobra{} provides the user with three options for the linear algebra method to be used when solving for the pressure updates: the Gauss elimination routine, the SuperLU solver \cite{Li1999}, and the Pardiso solver \cite{Schenk2006, Schenk2007}.
The SuperLU and the Pardiso solvers utilize sparse matrix storage and the Gauss elimination routine utilizes full matrix storage.
The sparsity of the matrices typically encountered is such that use of the Gauss elimination routine is discouraged.

The method that the user selects via the \cobra{} input file, ``\classname{deck.inp}," determines the data storage structure of the pressure matrix.
Each of the three methods utilizes a different memory format for matrix storage.
Given that the software as obtained was based upon procedural programming practices and written in a mixed Fortran dialect (FORTRAN 77, Fortran 90, and Fortran 95), the static memory structures for each of the three possible pressure matrices were predefined in different modules.
However, this procedural paradigm precluded the existence of two separate pressure matrices, one for the linear domain and one for the nonlinear domain.
This procedural approach for the pressure matrices was deemed inappropriate for the work being done, and an object-based approach was introduced instead.
To implement this switch, the way in which the solver was stored and accessed was modified to both enable multiple solvers and reduce the logical complexity of accessing the proper matrix.

\begin{figure}[ht!]
\singlespace\centering
\tikzsetnextfilename{images/matrixClassDiagram_eps}
\begin{tikzpicture}
\begin{abstractclass}[text width=6cm]{matrix}{0,0}
	\attribute{+ n : int}
	\attribute{+ nrhs : int}
	\operation[0]{+ put(i : int, j : int, val : real ) }
	\operation[0]{+ get(i : int, j : int ) : real }
	\operation[0]{+ rescale(i : int, j : int )}
	\operation[0]{+ reset()}
	\operation[0]{+ invert(b : real(:,:))}
\end{abstractclass}

\begin{abstractclass}[text width=6cm]{fullMatrix}{0,-6}
	\inherit{matrix}
	\attribute{- A : real(:,:) }
	\operation{+ put(i : int, j : int, val : real ) }
	\operation{+ get(i : int, j : int ) : real }
	\operation{+ rescale(i : int, j : int )}
	\operation{+ reset()}
\end{abstractclass}

\begin{class}[text width=4cm]{dsolveMatrix}{0,-11}
	\inherit{fullMatrix}
	\attribute{- ipiv : int(:) }
	\operation{+ invert(b : real(:,:))}
\end{class}

\begin{abstractclass}[text width=6cm]{sparseMatrix}{8,0}
	\inherit{matrix}
	\attribute{- num\_coeff: int }
	\attribute{- acoeff: real(:) }
	\attribute{- ij\_element: int(:,:) }
	\operation{+ put(i : int, j : int, val : real ) }
	\operation{+ get(i : int, j : int ) : real }
	\operation{+ rescale(i : int, j : int )}
	\operation{+ reset()}
\end{abstractclass}

\begin{class}[text width=4cm]{superluMatrix}{6,-6}
	\inherit{sparseMatrix}
	\attribute{- asub : int(:) }
	\attribute{- col\_beg : int(:) }
	\attribute{- ncpu : int }
	\operation{+ invert(b : real(:,:))}
\end{class}

\begin{class}[text width=4cm]{pardisoMatrix}{11,-6}
	\inherit{sparseMatrix}
	\attribute{- ia : int(:) }
	\attribute{- ja : int(:) }
	\attribute{- pt : int(:) }
	\attribute{- perm : int }
	\attribute{- mnum : int }
	\attribute{- mtype : int }
	\attribute{- maxfct : int }
	\operation{+ invert(b : real(:,:))}
\end{class}

\end{tikzpicture}
\caption{Matrix Class Diagram}
\label{fig:matrixClassDiagram}
\end{figure}

The object-oriented features in the Fortran 2003 and Fortran 2008 standards provided an alternative method of handling the pressure matrix.
The matrix storage structure was first replaced with an abstract class, \classname{matrix}.
\fig{fig:matrixClassDiagram} shows the class diagram for the \classname{matrix} hierarchy.
The \classname{matrix} class contains the number of right-hand sides anticipated, \classname{nrhs}, as well as the total number of pressure updates for a given domain, \classname{n}.
Additionally, this class would provide interfaces for the following procedures:

\begin{itemize}
\item{\classname{put}: a procedure that sets $\vec{A}_{i,j}$ to a given value.}
\item{\classname{get}: a procedure that returns $\vec{A}_{i,j}$.}
\item{\classname{rescale}: a procedure for scaling $\vec{A}_{i, :}$ by a given value.}
\item{\classname{reset}: a procedure for setting $\vec{A} = \vec{0}$.}
\item{\classname{invert}: a procedure that returns $\vec{x}$ from $\vec{A}\vec{x} = \vec{b}$.}
\end{itemize}

Given that the SuperLU and the Pardiso solvers utilize sparse matrices and the Gauss elimination routine uses a full matrix, there are two further abstract classes that inherit from \classname{matrix}; they are \classname{fullMatrix} and \classname{sparseMatrix}.
The \classname{fullMatrix} class contains an array representing $\vec{A}$.
The \classname{sparseMatrix} class contains the information required to construct a sparse matrix; however, the exact storage format is not specified in \classname{sparseMatrix}.
Both \classname{fullMatrix} and \classname{sparseMatrix} contain procedural implementations of the interfaces defined in \classname{matrix}.
There are three concrete classes that correspond to the three supported solvers: \classname{dsolveMatrix}, \classname{superluMatrix}, and \classname{pardisoMatrix}; all three implement their own linear system solving routines under the invert interface.
\classname{dsolveMatrix} is a concrete class inheriting from \classname{fullMatrix} that includes pivoting information.
\classname{superluMatrix} and \classname{pardisoMatrix} are concrete classes inheriting from \classname{sparseMatrix} that implement the matrix storage schemes for their respective solvers.

\begin{figure}[ht!]
\singlespace\centering
\begin{tikzpicture}
\begin{abstractclass}[text width=6cm]{solver}{0,0}
	\attribute{+ size : int}
	\attribute{+ A : matrix, ptr }
	\attribute{+ res : real(:,:)}
	\operation{+ insert(i : int, j : int) }
	\operation{+ solve() }
	\operation{+ rescale(i : int, j : int) }
	\operation{+ reset()}
	\operation[0]{+ init()}
\end{abstractclass}

\begin{class}[text width = 4cm]{dsolveSolver}{-5,-7}
	\inherit{solver}
	\attribute{+ A $=>$ dsolveMatrix }
	\operation{+ init()}
\end{class}

\begin{class}[text width = 4cm]{superluSolver}{0,-7}
	\inherit{solver}
	\attribute{+ A $=>$ superluMatrix }
	\operation{+ init()}
\end{class}

\begin{class}[text width = 4cm]{pardisoSolver}{5,-7}
	\inherit{solver}
	\attribute{+ A $=>$ pardisoMatrix }
	\operation{+ init()}
\end{class}

\end{tikzpicture}
\caption{Solver Class Diagram}
\label{fig:solverClassDiagram}
\end{figure}

Once the \classname{matrix} hierarchy was developed, an abstract \classname{solver} class was designed.
\fig{fig:solverClassDiagram} shows the class diagram for the \classname{solver} hierarchy.
The \classname{solver} class contains the size of the system, \classname{size}; a pointer to a \classname{matrix} object, \classname{A}; and the right-hand side of the linear system, \classname{res}.
Within the \classname{solver} class there is an interface for the initialization routine, \classname{init}.
Additionally, there is a procedure named \classname{insert} that will take the sixth row of a continuity volume's linear system and properly insert it into the correct row of the domain's pressure matrix, $\vec{A}_{j,:}$, as outlined in \sect{sec:domDecompMath}.
The following procedures for manipulating the linear system are also defined within the \classname{solver} object:

\begin{itemize}
\item{ \classname{solve}: an interface to call the solver associated with the matrix.}
\item{ \classname{rescale}: a procedure for scaling both $\vec{A}_{i, :}$ and $\vec{res}_{i}$.}
\item{ \classname{reset}: a procedure for setting both $\vec{A} = \vec{0}$ and $\vec{res} = \vec{0}$.}
\end{itemize}

There are three concrete classes that inherit from the abstract solver class: \classname{dsolveSolver}, \classname{superluSolver}, and \classname{pardisoSolver}.
Each of these concrete classes specifies an initialization routine that takes an adjacency list data structure, as discussed in \sect{subsect:nonlinearDiscretization}, and instantiates the \classname{matrix} pointer to the appropriate concrete subtype, and sets the appropriate parameters.
By abstracting the matrix storage strategies and the solvers in this way, a consistent interface is provided to the rest of the software.

%-------------------------------------------------------------------------------
%-------------------------------------------------------------------------------
%-------------------------------------------------------------------------------
\subsection{Volume Data Structures}
\label{subsect:domDecompVolumeStructs}

A difficulty that had to be overcome during this work was the proper mapping of the continuity volumes' memory indices to the proper pressure matrix indices.
The memory storage format for many variables was based upon a universal ordinal system for the continuity volumes.
This ordinal system was based upon the premise that there was a single pressure matrix.
It was determined that the programmatic complexity of trying to maintain a single ordinal system for the dual domains was greater than that of redesigning the memory architecture.
As part of the redesign, the software was partially transitioned to an object-based mesh to allow for multiple pressure matrices with different ordinal systems to co-exist.

A continuity volume was chosen as the basic unit for mesh representation.
Given that the mesh is decomposed into dual domains, there were two types of continuity volumes that needed to be distinguished.
The first type of volume, \classname{baseVolume}, encompasses all continuity volumes in the nonlinear domain and those continuity volumes in the linear domain that do not have a flow path connecting them to the nonlinear domain.
The second type of volume, \classname{nbcVolume}, is an extension of the \classname{baseVolume} type.
The set of \classname{nbcVolumes} includes those continuity volumes in the linear domain that do have a flow path connecting them to the nonlinear domain, the NBC volumes.
The volume class diagram for \cobra{} is outlined in \fig{fig:volumeClassDiagram}.

\begin{figure}[ht!]
\singlespace\centering
\begin{tikzpicture}

\begin{class}[text width=6cm]{baseVolume}{-4,0}
	\attribute{+ solver : ptr, solver}
	\attribute{+ ord : int}
	\attribute{+ num\_cons : int}
	\attribute{+ cons : int}
	\attribute{+ con\_type : int(:)}
	\attribute{+ con\_ord : int(:)}
	\attribute{+ gap\_ord : int(:)}
	\attribute{+ con\_coords : int(:,:)}
	\attribute{+ fp\_coords : int(:,:)}
	\attribute{+ con\_sig : real(:)}
	\attribute{+ b : int(:,:)}
	\attribute{+ b\_ords : int(:)}
	\attribute{+ jac : int(:,:)}
\end{class}

\begin{class}[text width=7cm]{nbcVolume}{4,0}
	\inherit{baseVolume}
	\attribute{+ nln\_solver : ptr, solver }
	\attribute{+ dp : real}
	\attribute{+ coef\_mat : real(:,:)}
	\attribute{+ flux : real(:,:)}
	\attribute{+ dflux\_dp : real(:,:)}
	\attribute{+ donored\_values : real(:,:)}
	\attribute{+ nbc\_cons : int}
	\attribute{+ nln\_ord\_by\_local\_nbc\_ord : int(:)}
	\attribute{+ nbc\_con\_ord : int(:)}
	\attribute{+ nbc\_con\_map : int(:)}
	\attribute{+ nbc\_ords : int(:)}
	\attribute{+ coords : int(:)}
	\attribute{+ nbc\_coords : int(:,:)}
\end{class}

\end{tikzpicture}
\caption{Volume Class Diagram}
\label{fig:volumeClassDiagram}
\end{figure}

The \classname{baseVolume} class contains information for the most basic type of continuity volumes.
Each \classname{baseVolume} has a pointer to the \classname{solver} object associated with its domain.
A \classname{baseVolume} that corresponds to a continuity volume in the linear domain has a \classname{solver} pointer to the linear pressure matrix solver.
The nonlinear domain's continuity volumes have a \classname{solver} pointer to the nonlinear pressure matrix solver.
This allows each continuity volume to be able to access the correct pressure matrix without having to reference a logical map or perform a calculation.
This class also stores the volume's ordinal, \classname{ord}.
This variable provides indexing information for the volume's pressure matrix.

Additionally, information regarding the connectivity of the volume to other volumes is stored in several variables: \classname{num\_cons}, \classname{cons}, \classname{con\_type}, \classname{con\_ord}, \classname{gap\_ord}, \classname{con\_coords}, \classname{fp\_coords}, and \classname{con\_sig}.
These variables represent the information necessary to properly reference those flow paths to which the continuity volume is connected, as well as the continuity volumes on the other end of those flow paths.
This data structure has the drawback of storing redundant flow path information.
Each of the continuity volumes connected by a particular flow path contains all of the information about that flow path in separate memory locations.

The \classname{baseVolume} class also contains volume's linear system from \sect{sect:linCobraAlg}.
The volume's linear system is stored as \classname{jac}.
The dimensions of \classname{jac} are dictated by the connectivity of the volume and its volume type, both of which are determined during input processing.
The right-hand side of the continuity volume's linear system is stored as \classname{b}.
Once again, the dimensions of \classname{b} are dictated by the type of the volume; an \classname{nbcVolume} will have more columns in \classname{jac} and \classname{b} than a comparable \classname{baseVolume}.  
The variable \classname{b\_ords} provides a mapping between the columns of \classname{b} and the columns of the pressure matrix of the domain to which the volume belongs.

Those linear continuity volumes that are connected to the nonlinear domain via a flow path have their own \classname{nbcVolume} class.
This class, as shown in \fig{fig:volumeClassDiagram}, is a derived class and contains additional information required to perform the domain decomposition.
First, a second \classname{solver} pointer is present, \classname{nln\_solver}.
This allows for proper indexing into the nonlinear pressure matrix when needed.
Second, the \classname{nbcVolume} also contains a separate collection of variables describing those flow paths that connect it to the nonlinear domain: \classname{nbc\_cons}, \classname{nbc\_con\_ord}, \classname{nbc\_con\_map}, \classname{nbc\_ords}, \classname{nbc\_coords}, \classname{coords}, and \classname{nln\_ord\_by\_local\_nbc\_ord}.
Lastly, the \classname{nbcVolume} class contains the pressure coefficients, \classname{coef\_mat}; the flow rate associated with the boundary, \classname{flux}; the derivative of the flow rate with respect to the pressure on either side of the flow path, \classname{dflux\_dp}; the matrix $\vec{\Xi}$ in compressed storage, \classname{donored\_values}; and the right-hand side of the linear domain's pressure solution, \classname{dp}.

Each continuity volume is represented by one of the above described volume classes.
This polymorphic approach allows for domain-agnostic procedures and references within the software.

%-------------------------------------------------------------------------------
%-------------------------------------------------------------------------------
%-------------------------------------------------------------------------------
\subsection{Domain Decomposition Algorithm}
\label{subsect:domDecompAlgorithm}

The implementation of the domain decomposition's mathematical algorithm provided in \sect{sec:domDecompMath} will now be covered.
\alg{alg:domDecompAlgorithm} provides a high-level overview of the software as implemented.
The steps in this algorithm will be detailed in the following section.

\begin{algo}[ht!]
\setlength{\baselineskip}{0.625\baselineskip}
\begin{algorithmic}[1]
\Require Input Processing
\Set $n = 0$
\Loop \; Transient Loop
    \Set $t^{n+1} : = t^{n} + \dt{}$
	\Algorithm Assemble Nonlinear and Linear Pressure Matrices	 \Comment{\alg{alg:xschem}}
	\LineIf{ \classname{LinearSolver} }{\textbf{solve} $\vec{A}_{\text{lin}} \vec{\delta P}_{\text{lin}} = \vec{res}_{\text{lin}}$}
	\If{ \classname{NonlinearSolver} }
	    \Set $k = 0$
		\Algorithm Get Nonlinear Coefficients \Comment{\alg{alg:domDecompGetCoef}}
		\Algorithm Set Nonlinear Boundary Values \Comment{\alg{alg:domDecompSetMat}}
		\Solve $\vec{A}^{k}_{\text{nln}} \vec{\delta P}_{\text{nln}}^{k} = \vec{res}_{\text{nln}}^{k}$	
		\Algorithm Update Nonlinear Variables \Comment{\alg{alg:updateVariables}} 
	    \Loop \; Newton Loop
			\Algorithm Assemble Nonlinear Pressure Matrix \Comment{\alg{alg:xschem}}
			\Algorithm Nonlinear Convergence Determination \Comment{\alg{alg:nlnConvergence}}
			\If{ \textbf{end} Newton loop}
				\State \textbf{break} Newton Loop 
			\EndIf		
			\Set $k \pluseq 1$
			\Algorithm Set Nonlinear Boundary Values \Comment{\alg{alg:domDecompSetMat}}
			\Solve $\vec{A}_{\text{nln}}^{k} \vec{\delta P}_{\text{nln}}^{k} = \vec{res}_{\text{nln}}^{k}$
			\Algorithm Update Nonlinear Variables \Comment{\alg{alg:updateVariables}}
		\EndLoop
	\EndIf
	\LineIf{ \classname{LinearSolver} }{\textbf{algorithm} Update Linear Variables} \Comment{\alg{alg:updateVariables}}
	\Set $n \pluseq 1$
\EndLoop
\end{algorithmic}
\caption{Dual domain \cobra{} algorithm.}
\label{alg:domDecompAlgorithm}
\end{algo}

First, the nonlinear domain decomposition input file described in \sect{subsect:domDecompInputFile} is parsed to determine to which domain each subchannel belongs.
Second, each continuity volume is designated as either a \classname{baseVolume} or an \classname{nbcVolume}.
Then the \classname{solver} objects are instantiated according to the \cobra{} input file.
Lastly, after the input processing and state initialization, the temporal integration loop begins.

In a dual domain problem, a single timestep is now broken into two parts: the linear and the nonlinear portions.
At the beginning of a timestep, the entire domain is traversed to assemble the pressure matrices for both the linear domain and the first iterate of the nonlinear domain.
This process is similar to the algorithm outlined in \ref{sect:linCobraAlg} for assembling the pressure matrix.
However, there are differences when an \classname{nbcVolume} is encountered.

%
%\comment{Is the following paragraph necessary?}
%For each \classname{baseVolume} the following procedure is followed.
%Each continuity volume's linearized system of equations, \eqref{eqn:linSystem}, is subjected to partial \vec{LU} decomposition without pivoting.
%The lower triangular matrix has a unit diagonal.
%The sixth equation in the upper-triangular system is then scaled by its diagonal.
%This upper-triangular, rectangular system for each continuity volume is then stored for later back-substitution.
%Since the pressure update, $\delta P$, corresponds to the last row in the linear system, the reduction of the linear system to its upper-triangular form allows for the isolation of the pressure update.
%This isolated equation represents the relationship of a given continuity volume's pressure change to the pressure change in the other continuity volumes to which it is attached.
%The last row of each continuity volume's \eqref{eqn:linSystem} is then formed into the linear pressure matrix, $\vec{A}_{\text{lin}}$, and its associated right-hand side, $\vec{res}_{\text{lin}}$.
%
%The two loops, one over the momentum flow paths and one over the continuity volumes, form a single group of operations that act upon a given domain.
%This grouping will be known hereafter as assembling the pressure matrix for a given domain.
%\alg{alg:xschem} shows the two loops and their associated actions.

\begin{algo}[ht!]
\setlength{\baselineskip}{0.625\baselineskip}
\begin{algorithmic}[1]
\For{i = 1, $N_{\text{nln}}$}
	\Set $\text{b} \Rightarrow \text{nbcVolume[i].solver.res(nbcVolume[i].ord,:)}$
	\For{j = 1, $N_{\text{nln}}$}
		\For{ k = 1, nbcVolume[j].nbc\_cons}
			\Set g\_ord = nbcVolume[j].nbc\_con\_ord(k)
			\Set s\_ord = 2 + 6 $*$ (g\_ord - 1)
			\Set e\_ord = s\_ord + 5
			\Set nbcVolume[i].coef\_mat(1:6, g\_ord) = b(s\_ord:e\_ord)
		\EndFor
	\EndFor
\EndFor
\end{algorithmic}
\caption{Obtain NBC Volume Coefficients.}
\label{alg:domDecompGetCoef}
\end{algo}

The \classname{nbcVolume} objects are treated slightly different from the regular \classname{baseVolume} objects.
The flow rate terms from the equations are not included in the residual at this point, but are instead treated as unknowns.
These flow rates are treated as independent parameters, which decreases the number of entries in the $\vec{K}_{c}$ matrix.
However, the derivatives of the residuals with respect these flow rates, $\vec{Q}_{c}$, are added to the linear system.
If a linear continuity volume is attached through more than one surface to the nonlinear domain, then $\vec{Q}_{c}$ will contain multiple identity matrices.
The resulting system will be subjected to the same \vec{LU} decomposition as outlined for a regular volume.
The last equation with the additional right-hand sides will be assembled into the linear domain's pressure matrix.
The linear system for the linear domain is then inverted by the chosen solver.
The additional coefficients, $\vec{W}_{\text{lin}}$, represent the coefficients for the boundary flow rate between the linear and the nonlinear domain.
After the inversion of the pressure matrix, these coefficients for the NBC volumes are collected into the \classname{nbcVolume} objects for ease of manipulation later.
This is shown in \alg{alg:domDecompGetCoef}.

\begin{algo}[ht!]
\setlength{\baselineskip}{0.625\baselineskip}
\begin{algorithmic}[1]
\For{i = 1, $N_{\text{nln}}$}
	\Set A $\Rightarrow$ nbcVolume[i].nln\_solver.A
	\Set i\_ord = nbcVolume[i].nln\_ord
	\For{j = 1, $N_{\text{nln}}$}
		\Set j\_ord = nbcVolume[j].nln\_ord
		\For{ k = 1, nbcVolume[i].nbc\_cons}
			\Set g\_ord = nbcVolume[j].nbc\_con\_ord(k)
			\Set k\_ord = nbcVolume[j].nln\_ord\_by\_local\_nbc\_ord(k)
			\Set dp = nbcVolume[i].coef\_mat(:, g\_ord) $\cdot$ nbcVolume[j].dflux\_dp(:, k)
			\Set A(i\_ord, k\_ord) $\pluseq$ dp
			\Set A(i\_ord, j\_ord) $\minuseq$ dp
		\EndFor
	\EndFor
	\Set res = nbcVolume[i].solver.res(nbcVolume[i].ord,:) $\cdot$ nbc\_flux
	\Set nbcVolume[i].nln\_solver.res(i\_ord, 1) = res
	\Set A(i\_ord, i\_ord) $\pluseq$ 1.0
	\State \textbf{call:} nbcVolume[i].nln\_solver.rescale(i\_ord)
\EndFor
\end{algorithmic}
\caption{Set NBC Volume Pressure Equations Into Nonlinear Pressure Matrix.}
\label{alg:domDecompSetMat}
\end{algo}

Once the pressure coefficients have been collected, and the linear domain has been inverted, the additional entries in the nonlinear pressure matrix that represent the NBC volume's pressure equations are populated.
This procedure is shown in detail in \alg{alg:domDecompSetMat}.
This process is a system of three nested loops.
The first, outer, loop is over all the NBC volumes in the domain, $N_{\text{nln}}$.
From \sect{sec:domDecompMath} and \eqref{eqn:linearDomainPressureUpdates}, each of these outer NBC volumes has an associated pressure equation, \eqref{eqn:nbcPressureEquation}.
This equation represents the inter-dependency of all of the NBC volumes upon each other.
Evaluating \eqref{eqn:nbcPressureEquation} requires a second, inner loop over all of the NBC volumes for each of the outer NBC volumes.
For each of these inner loop NBC volumes, their $N_{\text{NBC}}$ flow paths that connect them to nonlinear domain are then traversed, creating a third and final nested loop.
For each connecting flow path, the contribution is calculated and stored in the row of $\vec{A}_{\text{nln}}$ associated with the outer NBC volume.

Once the pressure equations for the NBC volumes have been inserted into the nonlinear pressure matrix, this matrix is solved.
The resulting pressure update vector is then used to update the nonlinear continuity volumes' variables according to \alg{alg:updateVariables}.
This inversion and update represents the first iterate for the nonlinear domain.
The Newton loop is now entered.
This loop is similar to that discussed in \sect{sect:nlnCobraSolver}, with the addition of a step to set the NBC volumes' equations prior to inverting $\vec{A}_{\text{nln}}$.
The loop termination criteria for this Newton loop are similar to those outlined in \sect{sect:nlnCobraSolver}.
The difference is that these convergence metrics depend only upon the nonlinear subdomain.
Upon termination of the Newton loop, the linear domain's variables are then updated.

%-------------------------------------------------------------------------------
%-------------------------------------------------------------------------------
%-------------------------------------------------------------------------------
\subsection{Dual Domain Input File}
\label{subsect:domDecompInputFile}

Since the domain decomposition algorithm uses the nonlinear solver, the nonlinear convergence tolerances and the iteration limit are input as described in \sect{subsect:nlnCobraInputFile}.
These parameters only apply to the nonlinear domain.
Additional information is appended to the end of the basic nonlinear input file to activate the domain decomposition algorithm.
The nonlinear input file needs to have the IDs of the subchannels that are going to be subjected to additional Newton steps specified.
The presence of the character string ``\classname{'begin-nln-channels'}'' after the nonlinear input signals that there will be two domains.
After that string, the IDs of the subchannels are listed, one per line, in no particular order.
Once all subchannels to be in the nonlinear domain have been listed, the character string ``\classname{'end-nln-channels'}'' needs to be on a separate line.
The following is an example of the dual domain input segment of the ``\classname{nwt.cob}'' input file.

{
\singlespace
\begin{verbatim}
'begin-nln-channels'
<chanID_1>
<chanID_2>
<chanID_3>
...
<chanID_N>
'end-nln-channels'
\end{verbatim}
}

In the above example, <chanID\_N> is the subchannel ID that will be included in the nonlinear domain.
The input processing routine has extensive error checking and user feedback for proper formatting of this input.
Note that there is no requirement that the subchannels specified to be in the nonlinear domain form a contiguous domain.


