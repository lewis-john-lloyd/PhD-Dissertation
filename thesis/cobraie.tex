\chapter{COBRA-IE}
\label{chap:cobra}

\section{Overview}
\label{sect:cobra_overview}
The ability to simulate the thermal-hydraulic behavior of a nuclear reactor core during postulated accidents and their follow-on mitigation/recovery measures is at the heart of safety analysis.
To be able to model the complex phenomena involved, many software codes have been developed since the 1960s.
In general, the these safety codes can be divided into several large categories. There are those codes that have extensive models available for large steam system components: RELAP \cite{RELAP}, TRACE \cite{TRACE}, and MELCOR are a few.
There are also those that have extensive modeling capabilities for in-core physics: COBRA \cite{Thurgood1983c} and VIPRE are two examples.
While there is definitely overlap between the capabilities of the various codes, each has its own particular strength and weaknesses.
This work deals with a variant of the COBRA family of codes, known as COBRA-IE.

\section{Physics}
\label{sect:cobra_physics}
COBRA-IE stands for COlant Boiling in Rod Arrays - Integrated Environment.
The Integrated Environment suffix indicated that this version of COBRA has, among other differences, been outfitted to couple with a various other codes in a systematic way using a generic semi-implicit coupling technique \cite{Weaver2002}.

\subsection{Conservation Equations}
One of the primary uses of COBRA is to simulate conditions where dispersed liquid within the flow could have a large impact upon the thermal-hydraulics performance of the nuclear reactor core.
The reason for this is because COBRA models the two-phase flow behavior using two discrete phases, liquid and gas, but allows for multiple fields within a given phase.
The ability to track different fields within a phase manifests itself in two important primary ways: the ability to account for the effects of con-condensable gases on condensation, and the ability to model the impact upon heat transfer by dispersed liquid droplets.


There are four equations representing the conservation of mass of the non-condensable gases, the continuous liquid field, the dispersed liquid field, and the total gaseous phase.
The general conservation of mass equation \eqref{eqn:conservation_of_mass} will yield each of the four conservation equations given appropriate assumptions and definitions for certain terms.

\begin{equation}
\label{eqn:conservation_of_mass}
\frac{\partial \alpha_k \rho_k }{\partial t } + \nabla \cdot \left( \alpha_k \rho_k \vec{U}_k \right) = \Gamma^{'''}_k + S^{'''}_k + s_k
\end{equation}

The left hand side of \eqref{eqn:conservation_of_mass} represents the Lagrangian derivative of the a given field $_k$.
The three terms on the right hand side represent inter-field, $S^{'''}_k$, inter-phase, $\Gamma_k$,  and external, $s_k$, sources or sinks of mass.
Since there are two liquid fields, the net phasic mass transfer, $\Gamma$, is apportioned between the continuous liquid field and the dispersed liquid field.
This division is given by \eqref{eqn:apportionment_of_mass_transfer}, where $\eta$ is a state dependent apportionment factor.

\begin{equation}
\label{eqn:apportionment_of_mass_transfer}
\Gamma = \Gamma_g = -( \Gamma_e + \Gamma_l ) =  \eta \Gamma + (1 - \eta)\Gamma
\end{equation}

The inter-field transfer of mass occurs only between the continuous and dispersed liquids, \eqref{eqn:entrainment_deentrainment}.

\begin{equation}
\label{eqn:entrainment_deentrainment}
S^{'''}_l + S^{'''}_e = 0
\end{equation}

Assumption that reduces the complexity of the conservation equations include the assumptions that mechanical equilibrium between the non-condensable gas field and the total gaseous field, $\vec{U}_{ncg} = \vec{U}_g$, thermal-equilibrium between different fields with the same phase, $ $, Dalton's law is applicable between the two gaseous fields, and that, most importantly, all fields share an equilibrium pressure.

Note that the momentum conservation equations are formulated in conservative form, this is a stark contrast between COBRA and some other common system analysis codes \cite{TRACE, RELAP} that use non-conservative forms for their momentum conservation laws.

COBRA uses two equations for phasic conservation of energy: the conservation of energy for the total liquid phase, \eqref{eqn:conservation_of_liq_energy}; and the conservation of energy for the total gaseous phase, \eqref{eqn:conservation_of_gas_energy}.
Several assumptions underlie these two equations.
The continuous liquid film is considered to be thermodynamic equilibrium with the dispersed liquid at a given point.

\begin{equation}
\label{eqn:conservation_of_liq_energy}
2 + 2 = 3
\end{equation} 

\begin{equation}
\label{eqn:conservation_of_gas_energy}
2 + 4  + 3 = 2
\end{equation}

Being that COBRA was developed primarily as a sub-channel analysis tool, there is an assumption that the primary flow path is defined as inline with the gravity vector, which will be referred to as axial flow.
However, there is also the ability to model "gaps," flow orthogonal to the axial direction.
Another assumption is that the non-condensable gases are in mechanical equilibrium with the vapor field.
The conservation of momentum is applied in each flow direction to the three-fields: continuous liquid films, entrained liquid droplets, and total gaseous flow.
For the conservation of energy, COBRA-IE tracks the combined fluid energy and the combined gas energy.
In total, there are twelve physical conservation equations that are used for the fluid-mechanic solution within COBRA.


In addition, there is an advection equation for Interfacial Area Transport.

In a one-dimensional simulation, there will be nine governing conservation equations for fluid-mechanics.


\section{Topology}
\label{sect:cobra_topology}

COBRA utilizes a staggered mesh strategy for generating the domain of the problem. 
With the staggered mesh, there are "continuity" meshes and there are "momentum" meshes; see \fig{fig:staggered_mesh}.
The conservation equations for mass and energy are solved at the center of the continuity cells.
The conservation equations for momentum are solved at the center of the momentum cells.
The edges of the continuity cells align with the center of the momentum cells and vice-verca.

\begin{figure}[ht]
\caption{COBRA-IE Staggered Mesh}
\label{fig:staggered_mesh}
\begin{center}
\begin{tikzpicture}
\draw (-3,0) rectangle +(1,5);
\draw (0,-1) rectangle +(1,1) (0,0) rectangle +(1,1) (0,1) rectangle +(1,1) (0, 2) rectangle +(1,1) (0,3) rectangle +(1,1) (0,4) rectangle +(1,1) (0, 5) rectangle +(1,1);
\draw[dashed] (3,-0.5) rectangle +(1,1) (3,0.5) rectangle +(1,1) (3,1.5) rectangle +(1,1) (3, 2.5) rectangle +(1,1) (3,3.5) rectangle +(1,1) (3,3.5) rectangle +(1,1) (3, 4.5) rectangle +(1,1) ;
\draw[dashed] (-3,0) -- (4,0);
\draw[dashed] (-3,5) -- (4,5);
\end{tikzpicture}
\end{center}
\end{figure}
 

\section{Numeric Method}
\label{sect:cobra_numeric_method}
The partial differential equations that represent the conservation laws discussed in \sect{sect:cobra_physics} are discretized using a technique known as the semi-implicit methodology \cite{Liles1978}.
The primary purpose of the semi-implicit methodology is to eliminate the sonic courant limitation placed upon time step size that exists in purely explicit discretization \cite{someone}. \ref{algo:semi_implicit}.

\section{Solution Methodology}
\label{sect:cobra_solution_methodology}

\begin{algo}[H]
\caption{Semi-Implicit Linear Solution Algorithm}
\label{algo:semi_implicit}
\setlength{\baselineskip}{0.625\baselineskip}
\begin{algorithmic}[1]
\Require $\Vec{x}^{0}$ and $t^{0}$
\Set $n = 0$
\Loop \; Take a Time Step
    \Set $\vec{x}^{n}$        
    \Calculate $\Delta t$ 
    \State $t^{n+1} : = t^{n} + \Delta t$
    \BlackBox Solve for $\vec{x}^{n+1}$ 
    \Test CCFL \Comment{Time-step Failure Mechanism (ccfl\_fail) }
    \BlackBox Interfacial Area Transport Equation
    \Calculate Courant Numbers 
\EndLoop{\;$n = n+1$}
\end{algorithmic}
\end{algo}

\subsection{Linearized Mode}
\label{subsect:solution_linear}


\subsection{Nonlinear Mode}
\label{subsect:solution_nonlinear}

\section{Time Step Selection}
\label{sect:time_step_selection}
