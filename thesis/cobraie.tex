\chapter{COBRA-IE}
\label{chap:cobra}

\section{Overview}
\label{sect:cobra_overview}
The ability to simulate the thermal-hydraulic behavior of a nuclear reactor core during postulated accidents and their follow-on mitigation/recovery measures is at the heart of safety analysis.
To be able to model the complex phenomena involved, many software codes have been developed since the 1960s.
In general, the these safety codes can be divided into several large categories. There are those codes that have extensive models available for large steam system components: RELAP \cite{RELAP}, TRACE \cite{TRACE}, and MELCOR are a few.
There are also those that have extensive modeling capabilities for in-core physics: COBRA \cite{Thurgood1983c} and VIPRE are two examples.
While there is definitely overlap between the capabilities of the various codes, each has its own particular strength and weaknesses.
This work deals with a variant of the COBRA family of codes, known as COBRA-IE.

\section{Physics}
\label{sect:cobra_physics}
COBRA-IE stands for COlant Boiling in Rod Arrays - Integrated Environment.
The Integrated Environment suffix indicated that this version of COBRA has, among other differences, been outfitted to couple with a various other codes in a systematic way using a generic semi-implicit coupling technique \cite{Weaver2002}.

COBRA-IE solves the two-phase, three-field conservation equations.
These equations include the following conservation of mass equations: non-condensable gas, continuous liquid film, entrained liquid droplets, and total gas mass.
Being that COBRA-IE was developed primarily as a sub-channel analysis tool, there is an assumption that the primary flow path is axial, where axial is defined as inline with the gravity vector.
There is the ability to model "gap" flow orthogonal to the primary axial flow direction.
The conservation of momentum is applied in each flow direction to the three-fields: continuous liquid films, entrained liquid droplets, and total gaseous flow.
For the conservation of energy, COBRA-IE tracks the combined fluid energy and the combined gas energy.
In total, there are twelve physical conservation equations that are used for the fluid-mechanic solution within COBRA.


In addition, there is an advection equation for Interfacial Area Transport.

\section{Topology}
\label{sect:cobra_topology}

COBRA utilizes a staggered mesh strategy for generating the domain of the problem. 
With the staggered mesh, there are "continuity" meshes and there are "momentum" meshes; see \fig{fig:staggered_mesh}.
The conservation equations for mass and energy are solved at the center of the continuity cells.
The conservation equations for momentum are solved at the center of the momentum cells.
The edges of the continuity cells align with the center of the momentum cells and vice-verca.

\begin{figure}[h]
\caption{COBRA-IE Staggered Mesh}
\label{fig:staggered_mesh}
\begin{center}
\begin{tikzpicture}
\draw (-3,0) rectangle +(1,5);
\draw (0,-1) rectangle +(1,1) (0,0) rectangle +(1,1) (0,1) rectangle +(1,1) (0, 2) rectangle +(1,1) (0,3) rectangle +(1,1) (0,4) rectangle +(1,1) (0, 5) rectangle +(1,1);
\draw[dashed] (3,-0.5) rectangle +(1,1) (3,0.5) rectangle +(1,1) (3,1.5) rectangle +(1,1) (3, 2.5) rectangle +(1,1) (3,3.5) rectangle +(1,1) (3,3.5) rectangle +(1,1) (3, 4.5) rectangle +(1,1) ;
\draw[dashed] (-3,0) -- (4,0);
\draw[dashed] (-3,5) -- (4,5);
\end{tikzpicture}
\end{center}
\end{figure}
 

\section{Numeric Method}
\label{sect:cobra_numeric_method}
The partial differential equations that represent the conservation laws discussed in \sect{sect:cobra_physics} are discretized using a technique known as the semi-implicit methodology \cite{Liles1978}.
The primary purpose of the semi-implicit methodology is to eliminate the sonic courant limitation placed upon time step size that exists in purely explicit discretization \cite{someone}. \ref{algo:semi_implicit}.

\section{Solution Methodology}
\label{sect:cobra_solution_methodology}

\begin{algo}[H]
\caption{Semi-Implicit Linear Solution Algorithm}
\label{algo:semi_implicit}
\setlength{\baselineskip}{0.625\baselineskip}
\begin{algorithmic}[1]
\Require $\Vec{x}^{0}$ and $t^{0}$
\Set $n = 0$
\Loop \; Take a Time Step
    \Set $\vec{x}^{n}$        
    \Calculate $\Delta t$ 
    \State $t^{n+1} : = t^{n} + \Delta t$
    \BlackBox Solve for $\vec{x}^{n+1}$ 
    \Test CCFL \Comment{Time-step Failure Mechanism (ccfl\_fail) }
    \BlackBox Interfacial Area Transport Equation
    \Calculate Courant Numbers 
\EndLoop{\;$n = n+1$}
\end{algorithmic}
\end{algo}

\subsection{Linearized Mode}
\label{subsect:solution_linear}


\subsection{Nonlinear Mode}
\label{subsect:solution_nonlinear}

\section{Time Step Selection}
\label{sect:time_step_selection}
