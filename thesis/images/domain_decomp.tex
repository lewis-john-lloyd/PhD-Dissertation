\begin{equation}
\label{eqn:si_solve}
\begin{bmatrix} 
	\vec{J}_{c_1}     	& \vec{J}_{c_1,m_2} & \vec{0} 			\\
	\vec{J}_{m_2,c_1} 	& \vec{J}_{m_2} 	& \vec{J}_{m_2,c_2} \\
	\vec{0} 			& \vec{J}_{c_2,m_2} & \vec{J}_{c_2}
 \end{bmatrix} \begin{bmatrix}
 \vec{\delta c}_{1} \\
 \vec{\delta m}_{2} \\
 \vec{\delta c}_{2}
 \end{bmatrix}  = -\begin{bmatrix}
 \vec{F}_{c_1} \\
 \vec{F}_{m_2} \\
 \vec{F}_{c_2}
 \end{bmatrix}
 \end{equation}
 
 The semi-implicit solution framework is composed of five steps.
 First, the momentum volume equations, rows one, three, and five of the linear system \eqref{eqn:si_solve}, are multiplied by their inverse diagonal Jacobian entries, $\vec{J}_{m_k}$.
 The off-diagonal entries for the momentum equations are now $\vec{J}^{-1}_{m_k}\vec{J}_{m_k,c_j}$ where $j$ is the index of the connected continuity volumes.
 Since the $J_{m_k,c_j}$ terms only contain derivatives with respect to the pressure for the adjoining continuity volumes, the [3 x 6] matrix product $\vec{J}^{-1}_{m_k}\vec{J}_{m_k,c_j}$ will be referred to as $\frac{\partial \vec{m}_k}{\partial P_j}$.
 This product matrix only contains entries in the sixth column.
 Using the above notation, the resulting system is given in \eqref{eqn:si_solve_1}.
 
  \begin{equation}
 \label{eqn:si_solve_1}
 \begin{bmatrix} 
\vec{J}_{c_1} 							& \vec{J}_{c_1,m_2} & \vec{0} 									\\
\frac{\partial \vec{m}_2}{\partial P_1} & \vec{I} 			& \frac{\partial \vec{m}_2}{\partial P_2} 	\\
\vec{0} 								& \vec{J}_{c_2,m_2} & \vec{J}_{c_2} 							
\end{bmatrix} \begin{bmatrix}
 \vec{\delta c}_{1} \\
 \vec{\delta m}_{2} \\
 \vec{\delta c}_{2}
\end{bmatrix}  = -\begin{bmatrix}
 \vec{F}_{c_1} \\
 \vec{J}^{-1}_{m_2}\vec{F}_{m_2} \\
 \vec{F}_{c_2}
 \end{bmatrix}
 \end{equation}

Second, the momentum equations are used to eliminate the $\vec{J}_{c_k,m_k}$ entries from the continuity volumes.
The resulting equation set, \eqref{eqn:si_solve_2}, no longer has equations for the momentum volumes.

\begin{IEEEeqnarray}{rl}
\label{eqn:si_solve_2}
\begin{bmatrix} 
\vec{J}_{c_1} - \vec{J}_{c_1,m_2}\frac{\partial \vec{m}_2}{\partial P_1} & -\vec{J}_{c_1,m_2}\frac{\partial \vec{m}_2}{\partial P_2} \\
-\vec{J}_{c_2,m_2}\frac{\partial \vec{m}_2}{\partial P_1} & \vec{J}_{c_2} - \vec{J}_{c_2,m_2}\frac{\partial \vec{m}_2}{\partial P_2}
\end{bmatrix} &\begin{bmatrix}
\vec{\delta c}_{1} \\
\vec{\delta c}_{2}
\end{bmatrix} \nonumber \\
 = -\begin{bmatrix}
\vec{F}_{c_1} - \vec{J}_{c_1,m_2}\vec{J}^{-1}_{m_2}\vec{F}_{m_2} \\
\vec{F}_{c_2} - \vec{J}_{c_2,m_2}\vec{J}^{-1}_{m_2}\vec{F}_{m_2}
\end{bmatrix} &
 \end{IEEEeqnarray}

The off-diagonal entries of the reduced Jacobian matrix in \eqref{eqn:si_solve_2}, $\vec{J}_{c_i,m_j}\frac{\partial \vec{m}_j}{\partial P_k}$ will be denoted by $\vec{C}_{c_i,P_j}$ as they represent the change in the continuity variables in volume $i$ due to the change in the pressure of continuity volume $j$.
The [6 x 6] matrix, $\vec{C}_{c_i,P_j}$, contains only entries in the last column, which correspond to the inter-continuity pressure coupling terms.
The diagonal entries of the reduced Jacobian and the right-hand side entries will be referred to as $\vec{J}^{*}_{c_i}$ and $\vec{F}^{*}_{c_i}$.
With this new nomenclature, \eqref{eqn:si_solve_2} can be restated as \eqref{eqn:si_solve_3}.

  \begin{equation}
 \label{eqn:si_solve_3}
 \begin{bmatrix} 
 \vec{J}^{*}_{c_1} & \vec{C}_{c_1,P_2} \\
 \vec{C}_{c_2,P_1} & \vec{J}^{*}_{c_2} 
 \end{bmatrix} \begin{bmatrix}
 \vec{\delta c}_{1} \\
 \vec{\delta c}_{2}
\end{bmatrix}  = -\begin{bmatrix}
 \vec{F}^{*}_{c_1} \\
 \vec{F}^{*}_{c_2}
\end{bmatrix}
 \end{equation}

Step three is to multiply each row by the $\vec{L}^{-1}_{c_i}$ portion of the LU decomposition without pivoting of $\vec{J}^{*}_{c_i}$, \eqref{eqn:si_solve_4}

  \begin{equation}
 \label{eqn:si_solve_4}
 \begin{bmatrix} 
 \vec{U}^{*}_{c_1} & \vec{L}^{-1}_{c_1}\vec{C}_{c_1,P_2} \\
 \vec{L}^{-1}_{c_2}\vec{C}_{c_2,P_1} & \vec{U}^{*}_{c_2}
 \end{bmatrix} \begin{bmatrix}
 \vec{\delta c}_{1} \\
 \vec{\delta c}_{2}
\end{bmatrix}  = -\begin{bmatrix}
 \vec{L}^{-1}_{c_1}\vec{F}^{*}_{c_1} \\
 \vec{L}^{-1}_{c_2}\vec{F}^{*}_{c_2}
\end{bmatrix}
 \end{equation}
 
By isolating the last row from each continuity block and dividing that row by its diagonal entry, a resulting matrix for the pressure update in every continuity volume is obtained, \eqref{eqn:si_pressure_matrix}. 
For a problem with $N$ continuity volumes, the resulting pressure matrix, \eqref{eqn:si_pressure_matrix}, will be $N$ x $N$.
For the simple geometry considered in this example, the resultant pressure matrix is a tridiagonal matrix.
However, this is not the case for more complex geometries.

  \begin{equation}
 \label{eqn:si_pressure_matrix}
 \begin{bmatrix} 
 1 & \frac{\vec{L}^{-1}_{c_1}\vec{C}_{c_1,P_2}[6]}{\vec{U}^{*}_{c_1}[6]} \\
 \frac{\vec{L}^{-1}_{c_2}\vec{C}_{c_2,P_1}[6]}{\vec{U}^{*}_{c_2}[6]} & 1
 \end{bmatrix} \begin{bmatrix}
 \vec{\delta P}_{1} \\
 \vec{\delta P}_{2}
\end{bmatrix}  = -\begin{bmatrix}
 \frac{\vec{L}^{-1}_{c_1}\vec{F}^{*}_{c_1}[6]}{\vec{U}^{*}_{c_1}[6]} \\
 \frac{\vec{L}^{-1}_{c_2}\vec{F}^{*}_{c_2}[6]}{\vec{U}^{*}_{c_2}[6]}
\end{bmatrix}
 \end{equation}
