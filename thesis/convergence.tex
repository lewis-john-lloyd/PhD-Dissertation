\chapter{Scaling and Convergence}
\label{chap:scaling_and_convergence}
As discussed in \chap{chap:intro}, COBRA-IE uses a single-shot linearization to solve its set of semi-implicit conservation equations.
The impact of this single-shot linearization upon temporal convergence will now be investigated.
However, prior to discussing the numerical experiments that were conducted to ferret out the impact of not resolving the nonlinearities, the scaling used to both calculate and determine nonlinear convergence needs to be addressed.

\section{Physics Based Scaling}
\label{sect:scaling}
For a given grid location, the nonlinear residuals for mass and energy, \eqref{eqn:cont_residual}, as formulated in section \ref{subsect:numerics_semi_implicit} will have six components.
These residuals will have the units of the conserved quantities for their corresponding PDEs.
Table \ref{tab:scaling_units_scales} shows some typical values of these conserved quantities during normal operations of typical PWR simulation.

\begin{table}[ht]
\centering
\begin{tabular}{|l|}
... your table ...
\end{tabular}
\caption{Typical scale for reactor simulations}
\label{tab:scaling_units_scales}
\end{table}

However, during accident scenarios, the range of these can values can dramatically change.
Once of the challenges that has been addressed during this work is coming up with a methodology for scaling that will provide a meaningful metric for convergence.
The scaling chosen should have the following characteristics:
\begin{itemize}
\item{$S^{-1}_k F(x_k)_i \approx 1$ when $x_k$ is a "poor" solution.}
\item{$S^{-1}_k F(x_k)_i \rightarrow 0$ when a phase disappears.}
\item{$0 \leq S^{-1}_k F(x_k)_i \leq 0 $ for all values of $F(x_k)_i$.}
\end{itemize}

The heart of the scaling used in this work is a physics-based methodology.
To illustrate the scaling procedure, we shall consider once again equation\ref{eqn:conservation_of_mass} for a simply connected continuity cell without mass transfer.
Assume that the entire channel single phase and in thermodynamics equilibrium such that the macroscopic densities on either side of this single cell are the same.

\begin{equation}
F = \left(\alpha_k \rho_k\right)^{n+1} - \left( \alpha_k \rho_k \right)^n - \frac{\Delta t}{V} \left( \frac{\alpha_k \rho_k }{<\alpha_k \rho_k>^n} V^{n+1}_{j-1} \right)
\end{equation}

In this equation there are three physically meaningful quantities: the temporal difference, the mass flowing into the cell, and mass flowing out of the cell.

\section{Nonlinear Convergence vs. Temporal Convergence}
\label{sect:nonlinear_temporal_convergence}

Three test problems were examined to determine the effect of temporal refinement on nonlinear convergence.
In each of the examples, an initial simulation is run subject to the time-step selection algorithm in COBRA, see \sect{subsect:time_step_selection}.
After the initial simulation is completed, the simulation is run again with a maximum time-step size set to one-half of the smallest time-step size of the previous run.
This procedure is iterated until the maximum time-step size reaches 1.0E-6.
This cut-off was chosen for practical constraints.

\section{Single Phase Flow in a Pipe}
\subsection{Model}
\subsubsection{Input Deck Description}
\subsection{Results}

\section{Flashing in a Pipe}
\subsection{Model}
\subsubsection{Input Deck Description}
\subsection{Results}

\section{GE LOFT}
\subsection{Experimental Setup}
\subsection{COBRA-IE Input Deck Description}
\subsection{Results}


\section{Review}
