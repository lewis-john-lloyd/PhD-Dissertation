\chapter{Scaling and Convergence}
\label{chap:scaling_and_convergence}
To evaluate the interplay between temporal convergence and nonlinear convergence, the following test procedure was proposed and executed.

\section{Scaling}
\label{sect:scaling}
Until this point, the definition of convergence, in the previously mentioned nonlinear sense, has been left purposely vague.
To determine if a simulation has converged, traditionally a delta value is used.



\section{Nonlinear Convergence vs. Temporal Convergence}
\label{sect:nonlinear_temporal_convergence}

Three test problems were examined to determine the effect of temporal refinement on nonlinear convergence.
In each of the examples, an initial simulation is run subject to the time-step selection algorithm in COBRA, see \sect{subsect:time_step_selection}.
After the initial simulation is completed, the simulation is run again with a maximum time-step size set to one-half of the smallest time-step size of the previous run.
This procedure is iterated until the maximum time-step size reaches 1.0E-6.
This cut-off was chosen for practical constraints.

\section{Single Phase Flow in a Pipe}
\subsection{Model}
\subsubsection{Input Deck Description}
\subsection{Results}

\section{Flashing in a Pipe}
\subsection{Model}
\subsubsection{Input Deck Description}
\subsection{Results}

\section{GE LOFT}
\subsection{Experimental Setup}
\subsection{COBRA-IE Input Deck Description}
\subsection{Results}


\section{Review}
