\chapter{Introduction}
\label{chap:intro}
In the United States, the goal of Nuclear Power Plant (NPP) designers, builders, operators, and regulators is to ensure the safety of the public during both normal operations and during severe accidents.
It is the responsibility of the Nuclear Regulatory Commission to issue licenses for the construction and operation of nuclear reactors.
Chapter 1 of Title 10 of the Code of Federal Regulations (10CFR) details the process by which an applicant can obtain an operating licesce for the construction of a NPP.
One of the requirements laid out in Part 50 of 10CFR (10CFR50) is the requirement for a safety analysis report (SAR).
Depending upon the license being pursued by the applicant, either a Preliminary Safety Analysis Report (PSAR) or a Final Safety Analysis Report (FSAR).
The PSAR is required for the issuance of a site construction license, while the FSAR is required for the issuance of either an operating license or a combined license.
For reactors that use light water, $H_2 O$, as a primary coolant, Light Water Reactors (LWR), both types of SARs require that the applicant provide information an evaluation of their Emergency Core Cooling System (ECCS) during postulated loss-of-coolant accidents (LOCA).
This evaluation must conform with section 46 of 10CFR50, which requires that the applicant perform analyses for "a number of postulated loss-of-coolant accidents of different sizes, locations, and other properties sufficient to provide assurance that the most severe postulated loss-of-coolant accidents are calculated."
This requirement necessitates that the designers and operators of NPP possess the ability to model the thermal-hydraulic behavior within the core of a reactor during severe accidents.  
The diverse physical conditions experienced by the reactor during severe accidents necessitates the inclusion of a wide array of physics during safety analyses.
Among the physics of interest are fluid-mechanics, neutron transport, structural mechanics, and radio-chemistry.
For each of these disciplines there are dedicated pieces of software under continual development to improve their predictive capabilities.
The work that follows is concerned with the mathematical formulation and solution of the equations governing the thermal-hydraulic behavior of the reactor core.

Within the United States, there are several codes that are widely available for simulating the thermal-hydraulic response of a nuclear power plant.
These safety codes can be divided into two large categories: system analysis codes, and sub-channel analysis codes.
While there is a swath of overlap between the capabilities of these two categories, each has its own particular strengths and weaknesses.
The system analysis code often have extensive models available for large components such as steam generators, pumps, valves, containment, etc.
RELAP \cite{RELAP}, TRACE \cite{TRACE}, and MELCOR \cite{Summers1994} are three of the more well known of these system level safety analysis codes.
Other codes have extensive modeling capabilities for in-core physics: COBRA \cite{Thurgood1983c} and VIPRE are two examples of these sub-channel codes.
For a complete analysis both codes are traditionally used in concert to provide an integrated overview of the system response.
The differences between these various codes will be reviewed in the following section to compare current practices, examine recent research, and illustrate where future research could be directed.

\section{Two Phase Flow}
\label{sect:two_phase_flow}
Within the United States, all operating commercial reactors are of a light water reactor (LWR) design.
There are two types of LWR designs within the US, pressurized water reactors (PWRs) and boiling water reactors (BWRs).
In both cases, the safety analysis requires the modeling of water as both a fluid and as a gas.
This fact has driven the development of safety software that can model the behavior of water under a extensive range of thermodynamic states, including multiple-phases.
There are several formulations of the governing conservation equations used to predict the thermal-hydraulic response of the nuclear reactor core to transient plant conditions. 

\subsection{General Assumptions}
\label{subsect:assumptions}
Consider \fig{fig:pipe_with_bubbles}; this is a stylized cross-sectional diagram of a flowing steam-water mixture within a pipe.

\begin{figure}[ht]
\caption{Two phase flow within a pipe.}
\label{fig:pipe_with_bubbles}
\begin{center}
\begin{tikzpicture}
\draw (-4,4) -- +(4,4);
\end{tikzpicture}
\end{center}
\end{figure}

The level of resolution sought for safety analysis codes does not include the resolution of exact interfaces between phases.
Of interest is the average behavior of the flow in a given location.
The definition of average in the above sentence can have several meanings.
There are spatial, temporal, and ensemble averaging techniques, each of which has their own physical interpretations\cite{Drew1998, Todreas2011}.
The following general equations result from the averaging processes discussed above with several assumptions thrown in for good luck.

\begin{equation}
\label{eqn:conservation_of_mass}
\frac{\partial \alpha_k \rho_k }{\partial t } + \nabla \cdot \left( \alpha_k \rho_k \vec{U}_k \right) = \Gamma^{'''}_k + S^{'''}_k + s_k
\end{equation}
\begin{equation}
\label{eqn:conservation_of_momentum}
\frac{\partial \alpha_k \rho_k \vec{U} }{\partial t } + \nabla \cdot \left( \alpha_k \rho_k \vec{U}_k \vec{U}_k \right) = -\alpha_k \nabla P + \alpha_k \vec{g} \Gamma^{'''}_k + S^{'''}_k + s_k
\end{equation}
\begin{equation}
\label{eqn:conservation_of_energy}
\frac{\partial \alpha_k \rho_k }{\partial t } + \nabla \cdot \left( \alpha_k \rho_k \vec{U}_k \right) = \Gamma^{'''}_k + S^{'''}_k + s_k
\end{equation}

\subsection{Conservation Equations}
One of the primary uses of COBRA is to simulate conditions where dispersed liquid within the flow could have a large impact upon the thermal-hydraulics performance of the nuclear reactor core.
The reason for this is because COBRA models the two-phase flow behavior using two discrete phases, liquid and gas, but allows for multiple fields within a given phase.
The ability to track different fields within a phase manifests itself in two important primary ways: the ability to account for the effects of con-condensable gases on condensation, and the ability to model the impact upon heat transfer by dispersed liquid droplets.


There are four equations representing the conservation of mass of the non-condensable gases, the continuous liquid field, the dispersed liquid field, and the total gaseous phase.
The general conservation of mass equation \eqref{eqn:conservation_of_mass} will yield each of the four conservation equations given appropriate assumptions and definitions for certain terms.

The left hand side of \eqref{eqn:conservation_of_mass} represents the Lagrangian derivative of the a given field $_k$.
The three terms on the right hand side represent inter-field, $S^{'''}_k$, inter-phase, $\Gamma_k$,  and external, $s_k$, sources or sinks of mass.
Since there are two liquid fields, the net phasic mass transfer, $\Gamma$, is apportioned between the continuous liquid field and the dispersed liquid field.
This division is given by \eqref{eqn:apportionment_of_mass_transfer}, where $\eta$ is a state dependent apportionment factor.

\begin{equation}
\label{eqn:apportionment_of_mass_transfer}
\Gamma = \Gamma_g = -( \Gamma_e + \Gamma_l ) =  \eta \Gamma + (1 - \eta)\Gamma
\end{equation}

The inter-field transfer of mass occurs only between the continuous and dispersed liquids, \eqref{eqn:entrainment_deentrainment}.

\begin{equation}
\label{eqn:entrainment_deentrainment}
S^{'''}_l + S^{'''}_e = 0
\end{equation}

Assumption that reduces the complexity of the conservation equations include the assumptions that mechanical equilibrium between the non-condensable gas field and the total gaseous field, $\vec{U}_{ncg} = \vec{U}_g$, thermal-equilibrium between different fields with the same phase, $ $, Dalton's law is applicable between the two gaseous fields, and that, most importantly, all fields share an equilibrium pressure.

Note that the momentum conservation equations are formulated in conservative form, this is a stark contrast between COBRA and some other common system analysis codes \cite{TRACE, RELAP} that use non-conservative forms for their momentum conservation laws.

COBRA uses two equations for phasic conservation of energy: the conservation of energy for the total liquid phase, \eqref{eqn:conservation_of_liq_energy}; and the conservation of energy for the total gaseous phase, \eqref{eqn:conservation_of_gas_energy}.
Several assumptions underlie these two equations.
The continuous liquid film is considered to be thermodynamic equilibrium with the dispersed liquid at a given point.

\begin{equation}
\label{eqn:conservation_of_liq_energy}
2 + 2 = 3
\end{equation}

\begin{equation}
\label{eqn:conservation_of_gas_energy}
2 + 4  + 3 = 2
\end{equation}

Being that COBRA was developed primarily as a sub-channel analysis tool, there is an assumption that the primary flow path is defined as inline with the gravity vector, which will be referred to as axial flow.
However, there is also the ability to model "gaps," flow orthogonal to the axial direction.
Another assumption is that the non-condensable gases are in mechanical equilibrium with the vapor field.
The conservation of momentum is applied in each flow direction to the three-fields: continuous liquid films, entrained liquid droplets, and total gaseous flow.
For the conservation of energy, COBRA-IE tracks the combined fluid energy and the combined gas energy.
In total, there are twelve physical conservation equations that are used for the fluid-mechanic solution within COBRA.


In addition, there is an advection equation for Interfacial Area Transport.

In a one-dimensional simulation, there will be nine governing conservation equations for fluid-mechanics.




\section{Numeric Discretization}
\label{sect:numeric_methods}

\subsection{Staggered Grid}
\label{subsect:topology}

COBRA utilizes a staggered mesh strategy for generating the domain of the problem.
With the staggered mesh, there are "continuity" meshes and there are "momentum" meshes; see \fig{fig:staggered_mesh}.
The conservation equations for mass and energy are solved at the center of the continuity cells.
The conservation equations for momentum are solved at the center of the momentum cells.
The edges of the continuity cells align with the center of the momentum cells and vice-verca.

\begin{figure}[ht]
\caption{A staggered mesh.}
\label{fig:staggered_mesh}
\begin{center}
\begin{tikzpicture}
\draw (-3,0) rectangle +(1,5);
\draw (0,-1) rectangle +(1,1) (0,0) rectangle +(1,1) (0,1) rectangle +(1,1) (0, 2) rectangle +(1,1) (0,3) rectangle +(1,1) (0,4) rectangle +(1,1) (0, 5) rectangle +(1,1);
\draw[dashed] (3,-0.5) rectangle +(1,1) (3,0.5) rectangle +(1,1) (3,1.5) rectangle +(1,1) (3, 2.5) rectangle +(1,1) (3,3.5) rectangle +(1,1) (3,3.5) rectangle +(1,1) (3, 4.5) rectangle +(1,1) ;
\draw[dashed] (-3,0) -- (4,0);
\draw[dashed] (-3,5) -- (4,5);
\end{tikzpicture}
\end{center}
\end{figure}

\subsection{Explicit}
\label{subsect:numerics_explicit}
Given the physical layout discussed in section \ref{subsect:topology}, the next choice is the numerical approximation to the partial differential equations from section \ref{sect:two_phase_flow}.
In particular, each of the

\subsection{Semi-Implicit}
\label{subsect:numerics_semi_implicit}

The partial differential equations that represent the conservation laws discussed in \sect{sect:two_phase_flow} are discretized using a technique known as the semi-implicit methodology \cite{Liles1978}.
A primary reason for using the semi-implicit methodology, as stated by \citet{Liles1978}, is to eliminate the sonic Courant limit of the purely explicit scheme discussed in \ref{subsect:numerics_explicit}
A generic algorithm used for is \ref{algo:semi_implicit}.

\begin{algo}[H]
\caption{Semi-Implicit Linear Solution Algorithm}
\label{algo:semi_implicit}
\setlength{\baselineskip}{0.625\baselineskip}
\begin{algorithmic}[1]
\Require $\Vec{x}^{0}$ and $t^{0}$
\Set $n = 0$
\Loop \; Take a Time Step
    \Set $\vec{x}^{n}$
    \Calculate $\Delta t$
    \State $t^{n+1} : = t^{n} + \Delta t$
    \BlackBox Solve for $\vec{x}^{n+1}$
    \Test CCFL \Comment{Time-step Failure Mechanism (ccfl\_fail) }
    \BlackBox Interfacial Area Transport Equation
    \Calculate Courant Numbers
\EndLoop{\;$n = n+1$}
\end{algorithmic}
\end{algo}


\subsection{SETS}
\label{subsect:numerics_sets}
To overcome the material courant limit exhibited by the semi-implicit method, several alternatives have been developed.
One alternative, known as the the stability-enhancing two-step method (SETS) \cite{Mahaffy1982}, eliminates the material courant limit by introducing multiple iterative refinement of the solution within a given time-step.

\subsection{Nearly Implicit}
\label{subsect:numerics_nearly_implicit}

\subsection{Fully Implicit}
\label{subsect:numerics_fully_implicit}

\section{Code Coupling}
\label{sect:code_coupling}

\subsection{Explicit}
\label{subsect:coupling_explicit}

\subsection{Semi-Implicit}
\label{subsect:coupling_semi_implicit}

\subsection{Implicit}
\label{subsect:coupling_implicit}

\section{Nonlinearities}
\label{sect:nonlinearities}
Building upon the previous research, 


\subsection{Single-shot Linearization}
\label{subsect:single_shot}

\subsection{Newton's Method}
\label{subsect:newtons_method}


\section{Temporal Convergence}
\label{sect:temporal_convergence}
Of primary use in the field of nuclear reactor safety analysis is simulation.
The ability to predict the behavior of reactors during off-normal events is the key to the licensing and the operation of nuclear power plants.
Within the United States, this simulation capacity is provided by a relatively small number of main stream software suites, among which are the RELAP variants, COBRA variants, and MELCOR.
While each of these software products varies in their models and implementations, the underlying numeric techniques and capabilities are similar.

\subsection{Time Step Selection}
\label{subsect:time_step_selection}


\subsection{Time Step Failure}
\label{subsect:time_step_failure}
There are two common methods for dealing with a potentially poor time-step in sub-channel codes: prediction and mitigation.
The predictive methodology is the less commonly applied of the two methods.
It uses information about the explicit portion of the nonlinear residual to identify situations where the linearization point may be poor.
In COBRA, this particular method is used only for the prediction of non-condenable gas appearance.

Mitigation, the second method, is the more common and actively used strategy.
This strategy can be further subdivided into two catagories: limiting and failure. 
There is an extensive set of limiting criteria.


\section{Review}
\label{sect:review}

\section{Research Objective}
The objective of this dissertation is the design, implementation, and evaluation of a practical non-linear solution framework for reactor safety systems codes.
Specifically, an efficient and reliable solution methodology to the two-phase, three-field, fluid-dynamics and the solid-structure heat transfer system of coupled non-linear partial differential equations is sought.
The specific methodology should be capable of obtaining a consistent solution to the system of PDEs while also possessing. \cite{Aktas1996}
The limiting behavior of nuclear reactors occurs during the end portion of a severe accident.
The primary point of investigation is the use of domain decomposition for the purpose of refining the solution to the nonlinear conservation within COBRA-IE.
The domain decomposition will take place utilizing a generic semi-implicit coupling methodology developed by \citet{Weaver2002}.
The ability to isolate a subdomain of the problem and increase ability of the software to resolve the complex nonlinear physics that occur near the quench front of a reflood transient.
This methodology will hopefully increase the predictive capabilities of safety analysis codes while keeping the computational overhead as low as reasonable possible. 
