\chapter{Introduction}
\label{chap:intro}
In the United States, the goal of Nuclear Power Plant (NPP) designers, builders, operators, and regulators is to ensure the safety of the public during both normal operations and postulated accident scenarios.
It is the responsibility of the Nuclear Regulatory Commission (NRC) to issue licenses for the construction and operation of nuclear reactors.
Chapter 1 of Title 10 of the Code of Federal Regulations (10CFR) details the regulatory procedures that govern the NRC.
Part 50 of 10CFR (10CFR50) lays out the process by which an applicant can obtain both construction and operating licenses for NPPs.
One of the documents required prior to the issuance of any license under 10CFR50 is a safety analysis report (SAR).
In part, SARs for Light Water Reactors (LWRs) require the applicant to provide an evaluation of their emergency core cooling systems during postulated loss-of-coolant accidents (LOCA).
This evaluation must conform with section 46 of 10CFR50, which requires the applicant to perform analyses for ``a number of postulated loss-of-coolant accidents of different sizes, locations, and other properties sufficient to provide assurance that the most severe postulated loss-of-coolant accidents are calculated" \cite{CFR10}.
This requires designers and operators of NPPs to model the thermal-hydraulic behavior within the core of a reactor during postulated LOCAs.  
The diverse physical conditions experienced by the reactor during postulated accidents necessitate the inclusion of a range of complex physics during safety analyses.
Each of these physics has dedicated pieces of software that are under continual development to improve their predictive capabilities.
The work that follows is concerned with the mathematical formulation and solution of the equations governing the thermal-hydraulic behavior within the reactor core.

%-------------------------------------------------------------------------------
%-------------------------------------------------------------------------------
%-------------------------------------------------------------------------------
\section{Design Motivation}
\label{sect:motivation}

All commercial reactors operating within the United States are of an LWR design.
The safety analyses required for licensing necessitate the modeling of water in both liquid and gaseous phases.
This has driven the development of safety software that can model the behavior of water under an extensive range of thermodynamic states, including multiple phases.
Several codes are widely available for simulating the thermal-hydraulic response of an NPP during postulated accidents.
These safety codes can be divided into two large categories: system analysis codes and subchannel analysis codes.
While there is considerable overlap between the capabilities of these two categories, each has its own particular strengths and weaknesses.
Three well-known system-level safety analysis software packages are \relap53d{} \cite{RELAP}, TRACE \cite{TRACE}, and MELCOR \cite{Summers1994}.
Two well-known subchannel analysis software packages are COBRA \cite{Thurgood1983c} and VIPRE.
A brief description of these two types of software is provided below.

The system analysis software has historically been used for complete plant analysis.
These software systems have extensive collections of empirical models for pumps, valves, accumulators, etc.
These models allow for a macroscopic description of the balance of plant and its interaction with the
reactor core.
The reactor core itself is normally a low-fidelity representation that operates primarily as a heat source for the balance of plant model.
While the system software can be successfully used to model the balance of plant, more detailed physical models and hydrodynamics are required to accurately provide details of in-core behavior.

To address the need for accurate simulations of accident scenarios, software capable of modeling the dominant physics of interest within a reactor core was developed.
This is the second type of software: subchannel analysis.
The use of subchannel analysis software stems from physical arguments about dominant flow within a LWR.
To motivate these arguments, a brief discussion of the geometry within the nuclear reactor is presented.
The basic geometry of commercial nuclear reactor cores in the United States is a vertical, cylindrical array of fuel bundles, control rods, and instrumentation.
Within a fuel bundle cylindrical fuel rods are arranged into a lattice.
The presence of concentric geometric structures of axially aligned units creates natural subchannels for coolant flow.
This is the basis for one of the underlying assumptions of subchannel analysis software: the axial flow is the dominant flow.
This assumption allows for reduced complexity when modeling the inter-subchannel transverse flow.
In the work that follows the governing physics and the computational framework of interest will be taken from a variant of the aforementioned COBRA subchannel analysis code, which shall be referred to generically as \cobra{}.

%-------------------------------------------------------------------------------
%-------------------------------------------------------------------------------
%-------------------------------------------------------------------------------
\section{Research Objectives}
\label{sect:researchObjectives}

The methods used to simulate thermal-hydraulic behavior in NPPs are characterized by the manner in which the governing conservation equations are integrated over space and time and how their nonlinearities are resolved.
Each of the methods outlined in \sect{sect:solution_techniques} has one thing in common: the governing equations are solved over the global domain, either through a single linear approximation or through iterative, nonlinear convergence.

\begin{figure}[ht!]
\centering
\tikzsetnextfilename{images/my_diagram_eps}
\begin{tikzpicture}
\draw[pattern=north west lines, pattern color=blue] [thick] (-3,0) rectangle (0,3);
\draw (-1.5,3) node[anchor=south] {Single-Shot Linearization};
\draw (-3,2) node[anchor=east] {RELAP5-3D};
\draw (-3,1.5) node[anchor=east] {COBRA};
\draw (-3,1) node[anchor=east] {MARS};
\draw[pattern=north east lines, pattern color=red] [thick] (3,0) rectangle (6,3);
\draw (4.5,3) node[anchor=south] {Iterative Convergence};
\draw (6,2) node[anchor=west] {CATHARE};
\draw (6,1.5) node[anchor=west] {TRACE};
\draw (0,-4) rectangle +(3,3);
\path[pattern=north west lines, pattern color=blue] (0,-4) rectangle +(0.75,3);
\path[pattern=north west lines, pattern color=blue] (2.25,-4) rectangle +(0.75,3);
\path[pattern=north west lines, pattern color=blue] (0.75,-1.75) rectangle +(1.5,0.75);
\path[pattern=north east lines, pattern color=red] (0.75,-3.25) rectangle +(1.5,1.5);
\path[pattern=north west lines, pattern color=blue] (0.75,-4) rectangle +(1.5,0.75);
\draw (1.5,-1) node[anchor=south] {Proposed Method};
\end{tikzpicture}
\caption{Current and proposed paradigms for thermal-hydraulic safety analysis.}
\label{fig:my_diagram}
\end{figure}

The advantage of the linear approximation is the reduction in computational costs; however, the accuracy of this approximation at large timestep sizes in regions of highly nonlinear physics is suspect.
This limitation has traditionally been mitigated by restrictions placed upon the maximum permissible change in the independent parameters during a given timestep, which can in turn lead to excessively small timestep sizes.
The benefit of an iterative solver is that the nonlinear physics are resolved at each timestep by using multiple iterations.
In trade, the computational cost of those multiple iterations is high.
This research seeks to find a balance between the competing incentives of computational cost and accuracy by creating a hybrid method, as illustrated in \fig{fig:my_diagram}.
Specifically, if there are parts of the domain with a high degree of nonlinear physics that are spatially isolable, then the computational cost of solving the entire domain's governing equations iteratively may be unnecessary.
Instead, by restricting the multiple iterations to only those portions of the global domain where they are necessary, the computational cost will be reduced while maintaining accuracy. 

The objectives of this research were the design, implementation, and evaluation of a novel, spatially selective, nonlinear solution method for nuclear thermal-hydraulic safety analysis.
Isolation of a specified subdomain where nonlinearities are high was achieved by developing a new variant of the code-coupling algorithm discussed in \sect{sect:code_coupling}.
For this work, the subdomain to be isolated was comprised of geometric components predetermined by the user.
Upon isolation, the nonlinear subdomain was subjected to Newton's method to resolve any nonlinearities at every timestep.
This converged solution was then communicated via coupling coefficients to the remainder of the domain for use in finishing its single Newton step.
This unique use of domain decomposition for selective nonlinear-refinement via semi-implicit coupling provides a route for obtaining nonlinearly converged, timestep-size insensitive solutions for traditional two-phase flow methods with a lower computational cost than that of a globally iterative Newton's method.
The following chapters will provide an overview of: relevant background material and research, \chap{chap:background}; the work that was done to allow this research to be carried out, \chap{chap:nonlinear}; the domain decomposition algorithm, \chap{chap:domDecomposition}; and the results of numerical experiments, \chap{chap:results}.
