\chapter{Concluding Remarks}
\label{chap:end}
I am a conclusion.

\section{Summary of Findings}
\label{sect:end:summary}


\section{Topics for Future Study}
\label{sect:end:future}

In review, the single-shot semi-implicit \cobra{} software was converted to a nonlinearly convergent, semi-implicit algorithm.
An operator-based scaling was introduced and implemented into both the legacy and the nonlinear \cobra{} solvers.
The nonlinear solver produced results that were qualitatively different than those obtained from the legacy solver for the problem with high nonlinearities.
Tests indicated that a timestep size insensitive solution may not be nonlinearly converged.
It was shown that the nonlinear solver produces results that are indistinguishable from those produced by the legacy solver for the single-phase problem, which has relatively mild nonlinearities.
A metric to quantify the nonlinear convergence of a timestep size insensitive simulation was developed and implemented.
This non-convergence of the discrete nonlinear system in the legacy solver mode produced a qualitatively different solution from the nonlinear solver.

