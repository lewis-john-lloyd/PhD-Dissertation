%=====================================================================
% main.tex
%=====================================================================
% This file contains:
%	- Document Class
%	- Packages
%	- Format Information
%	- Custom Commands
%	- Chapters
%	- Bibliography
%	- Appendices
%	- Curriculum Vita

%=====================================================================
% Document Style
%=====================================================================
% The margincheck option flags lines which overflow their hbox with a black
%  box at the end of the line.  This usually (but not always) indicates a
%  margin violation on the right margin.  Left margin violations aren't
%  indicated and if the margin violation is large enough, there isn't room
%  for the black box to be visiable.  

% A 12 Point UW PhD Thesis.
\documentclass[margincheck]{withesis}

%=====================================================================
% Packageges
%=====================================================================
% 	- inputenc
%		Purpos: fonts
%\usepackage[latin1]{inputenc}

% ifthen:
\usepackage{ifthen}
\newcommand{\StandAlone}{false}

% epsfig: The package epsfig is used to bring in the Encapsulated PostScript figures into the document.
\usepackage{epsfig}

% verbatim:
\usepackage{verbatim}

% pfg/TikZ: The package tikz is used to create graphics.
\usepackage{tikz}
\usetikzlibrary{shapes,arrows}

% times: The package times is used to change the fonts to Times Roman; however because the times typewriter font looks odd, the original LaTeX Computer Modern font is kept for the typewriter font using \renewcommand{\ttdefault}{cmtt}.
%	Times Roman is a PostScript font and therefore, the document cannot be correctly viewed from the *.dvi file. 
%	It should be converted to a *.ps file first and then viewed with a PostScript previewer.
\usepackage{times}
\renewcommand{\ttdefault}{cmtt}

% syntonly: The package syntonly checks the syntax of document without actually generating files.
\usepackage{syntonly}
% Activated by the \syntaxonly flag.
%\syntaxonly

% amsmath: The package amsmath includes most useful math stuff.
\usepackage{amsmath}

% IEEEtrantools: This package provides the IEEEeqnarray environment.
\usepackage{IEEEtrantools}

% algorithm: This package provides the algorithm environment.
% \usepackage[ruled,chapter]{algorithm}

% algpseudocode: This package provides the pseudocode algorithm environment.
\usepackage{algpseudocode}

% pdflscape: This package allows for pages to be rotated within the document.
\usepackage{pdflscape}

% hyperref: The package hyperref allows for hyperlinks within the document.
\usepackage[bookmarks,pdfauthor={Lewis John Lloyd},colorlinks,citecolor=black,filecolor=black,linkcolor=black,urlcolor=black]{hyperref}

%========================================================================
%  Draft Control Commands
%========================================================================
%	- \psdraft
%		Purpose: 	Causes the \psfig or \epsfig commands to draw a box and label the box with the postscript file name instead of reading in the full postscript figure.  
%					This can save time and toner when printing drafts.
%\psdraft


% 	- \psfull
%		Purpose: Causes the inclusion of the postscript figures
%\psfull


%	- \pagestyle{thesisdraft}:
%		Purpose: Changes the header and footer
%		Options:
%			{thesisdraft} 	Casues the header and footer to be: DRAFT: Do Not Distribute        <time><Date>        <input file name>
%			{thesis}		Casues the header and footer to be the correct format
\pagestyle{thesis}

%  	- \draftmargins:
%		Purpose:	The page margins can be marked with a post-script box using the \draftmargins command.
%					This command uses dvips's end-of-page hook.
%					This is only visible in the *.ps file (NOT the *.dvi file)!
%\draftmargins


%	- \draftscreen:
%		Purpose:	The word ``DRAFT'' can be diagonally printed across the page using the \draftscreen command.
%					This command uses dvip's beginning-of-page hook.
%					This is only visible in the *.ps file (NOT the *.dvi file)!
%\draftscreen

%=======================================================================
% Include custom math commands
%=======================================================================
%============================================================================
% commands.tex
%============================================================================
% This file contains:
% 	- Defined Variables
%	- Redefined math shorthand
%	- Defined math shorthand


%============================================================================
% Redefined Math Commands
%============================================================================
% 	- \Vec{1} or \vec{1}
%		Long Name: Vector
%		Arguements[1]: bold and overbar arg1	
\DeclareRobustCommand{\Vec}[1]{%
    \ifmmode
        \mathbf{#1}\,%
    \else
        $\displaystyle \mathbf{#1}\,$%
    \fi
}
\DeclareRobustCommand{\vec}[1]{\Vec{#1}}

\DeclareRobustCommand{\lbm}{%
    \ifmmode
        \text{lb}_{\text{m}}
    \else
        $\displaystyle \text{lb}_{\text{m}}$%
    \fi
}
\DeclareRobustCommand{\lbf}{%
    \ifmmode
        \text{lb}_{\text{f}}
    \else
        $\displaystyle \text{lb}_{\text{f}}$
    \fi
}
\DeclareRobustCommand{\dt}{%
	\ifmmode
		\Delta t
	\else
		$\Delta t$
	\fi
}
\DeclareRobustCommand{\dtmax}{%
	\ifmmode
		\Delta t_{\text{MAX}}
	\else
		$\Delta t_{\text{MAX}}$
	\fi
}
\DeclareRobustCommand{\dx}{%
	\ifmmode
		\Delta x
	\else
		$\Delta x$
	\fi
}

\delimitershortfall-1sp
\newcommand\abs[1]{\left|#1\right|}


%=======================================================================
% Include custom tikz commands
%=======================================================================
\tikzstyle{Decision} = [diamond, draw, text width=4.5em, text badly centered, node distance=3cm, inner sep=0pt]
\tikzstyle{Action} = [rectangle, draw,text width=5em, text centered, node distance=3cm, rounded corners, minimum height=0em]
\tikzstyle{NodePoint} = [circle, draw, minimum height = 0 em, node distance = 3 cm]
\tikzstyle{BlackBox} = [rectangle, draw, text centered, node distance=1cm, fill=black!10]
\tikzstyle{line} = [draw, -latex']
    

%=======================================================================
% Include custom variables based on ifthen commands
%=======================================================================
\DeclareRobustCommand{\BlackBox}{\State \textbf{Black Box: }}
\DeclareRobustCommand{\Test}{\State \textbf{Test: }}
\DeclareRobustCommand{\Define}{\State \textbf{Define: }}
\DeclareRobustCommand{\Update}{\State \textbf{Update: }}
\DeclareRobustCommand{\Set}{\State \textbf{Set: }}
\DeclareRobustCommand{\Calculate}{\State \textbf{Calculate: }}
%\newcommand{\algorithmicset}{\textbf{Set:}}
%\algnewcommand\Solve{\item[\algorithmicset]}

%=======================================================================
% Start Document
%=======================================================================
\begin{document}

%=======================================================================
% Bibliography Style
%=======================================================================
\bibliographystyle{plain}

%=======================================================================
% Chapters
%=======================================================================
%============================================================================
% prelude.tex
%============================================================================
% This files contains:
% Title Page
% Abstract
% Table of Contents
% List of Tables
% List of Figures
% List of Algorithms
% Nomenclature
% Dedication
% Acknowledgments

%============================================================================
% Make the page numbers Roman (i, ii, etc)
%============================================================================
\clearpage\pagenumbering{roman}  

% ============================================================================
% Title Page
% ============================================================================
% \degree{Degree Title Here}: The default degree is ``Doctor of Philosophy.''

% \dissertation: The default is dissertation, unless the document style msthesis was specified in which case it becomes thesis.

% \title{}:
\title{Selective Spatial-Temporal Nonlinear Refinement for Thermal-Hydraulic Safety Analysis Codes}

% \author{}:
\author{Lewis John Lloyd}

% \date{}:
\date{2012}

% \prelim: For a preliminary report, specify \prelim.
\prelim
% \department{}: The department can be changed using the command \department{}.
\department{Nuclear Engineering and Engineering Physics}

% \advisorname{}: Name of primary thesis advisor.
\advisorname{Michael Corradini}

% \advisortitle{}:
\advisortitle{Professor}

% Once the above are defined, use \maketitle to generate the titlepage.
\maketitle

%============================================================================
% Copyright Page
%============================================================================
\copyrightpage

%============================================================================
% Abstract
%============================================================================
\begin{\abstractType}
  %============================================================================
% abstract.tex
%============================================================================
The methods used to simulate the thermal-hydraulic behavior in the core of a nuclear power plant during postulated accidents are characterized by the manner in which the temporal-integration of the governing conservation equations and the manner in which the nonlinearities in those fully discrete equations are resolved.
Each of the methods used in has one thing in common: the discrete nonlinear problem is solved globally at each timestep.
This is done either through a single Newton step or through an iterative Newton procedure.

The primary advantage of the single Newton step approach is the low computational costs; however, the accuracy of this method in regions of highly non-linear physics is suspect.
This has traditionally been mitigated by limitations placed upon the maximum change in independent parameters during a timestep.
The resolution of the nonlinearities within a timestep requires the use of an iterative method.
The benefits of an iterative Newton solver is that the nonlinearities are resolved at each timestep; however, the computational cost of a global Newton method is high.

For spatially isolable nonlinearities the computational expenditure of iteratively solving the global nonlinear problem may be unnecessary.
The objectives of this research are the design, implementation, and evaluation of a novel, spatially selective, nonlinear solution method for nuclear thermal-hydraulic safety analysis.
Isolation of those subdomains where nonlinearities are high will be achieved by domain decomposition.
The method of decomposition chosen enables feedback across the subdomain boundaries. 
Upon isolation, the nonlinear subdomain will be subjected to a globalized Newton method to resolve the local nonlinearities.
The nonlinearly converged solution from the subdomain will then be communicated via coupling coefficients to the rest of the problem domain for use in calculating its single Newton step.
This unique use of selective nonlinear refinement via domain coupling may provide a route to nonlinearly converged timestep size insensitive solutions for traditional two-phase flow methods.
\end{\abstractType}

%============================================================================
% Acknowledgement Page
%============================================================================
\begin{acknowledgments}
  DOE and MUSC encourage fellows to publish reports and articles in scientific and engineering 
journals.  The fellow must submit any articles, reports, or thesis to MUSC prior to submission for 
publication.  All publications will show the joint affiliation of the fellow with the university and, if 
appropriate, with the laboratory in which the research was conducted, and should acknowledge 
fellowship support. 
 
Fellowship support should be acknowledged in the following manner: 
 
This research was performed under appointment to the Rickover Fellowship 
Program in Nuclear Engineering sponsored by Naval Reactors Division of the 
U.S. Department of Energy. 
\end{acknowledgments}

%============================================================================
% Auto-Generated Pages
%============================================================================
\tableofcontents
\listoftables
\listoffigures
\listofalgorithm

% ============================================================================
% Nomenclature Page
%============================================================================
%============================================================================
% nomenclature.tex
%============================================================================
% This file contains:
% 	- List of defined nomenclature items

\printnomenclature[6em]

\nomtypeA{NRC}{Nuclear Regulatory Commision}
\nomtypeA{NPP}{Nuclear Power Plant}
\nomtypeA{10CFR}{Title 10 of the Code of Federal Regulations}
\nomtypeA{10CFR50}{Part 50 of 10CFR}
\nomtypeA{SAR}{Safety Analysis Report}
\nomtypeA{LWR}{Light Water Reactor}
\nomtypeA{ECCS}{Emergency Core Cooling System}
\nomtypeA{LOCA}{Loss-of-Coolant Accident}
\nomtypeA{PWR}{Pressurized Water Reactor}
\nomtypeA{BWR}{Boiling Water Reactors}
\nomtypeA{CFL}{Courant-Friedrichs-Lewy}
\nomtypeA{NBC}{Nonlinear-Boundary Continuity}
\nomtypeA{PDE}{Partial Differential Equation}
\nomtypeA{SETS}{Stability Enhancing Two-Step}
\nomtypeA{PVM}{Parallel Virtual Machine}

\nomtypeR{$w^{n+1}_{k, j}$}{Volumetric flow for phase k in \relap53d{}.}{$ \text{ft}^{3}\,\text{s}^{-1}$}
\nomtypeR{$u^{n+1}_{k, j}$}{Flow of internal energy for phase k in \relap53d{}.}{$\text{BTU}\,\text{s}^{-1} $}
\nomtypeR{$n^{n+1}_{g,j}$}{Flow of \ncg{} mass in \relap53d{}.}{$ \lbm{} \, \text{s}^{-1}$}
\nomtypeR{$m^{n+1}_{k,j}$}{Flow of mass for phase k in \relap53d{}.}{$ \lbm{}\, \text{s}^{-1} $}
\nomtypeR{$s_{p}$}{Source or sink of external momentum.}{$ \lbf{} \text{s}$}
\nomtypeR{$s_{e}$}{Source or sink of external energy.}{$ \text{BTU} $}
\nomtypeR{$s_{m}$}{Source or sink of external mass.}{$\lbm{} $}
\nomtypeR{$\vec{a}$}{Vector of constants from right-hand-side of momentum system.}{}
\nomtypeR{$\vec{b}$}{Vector of pressure-update coefficients from momentum system.}{}
\nomtypeR{$\vec{u}$}{Vector of phasic velocities for a given flow path.}{$ \text{ft}\, \text{s}^{-1}$}
\nomtypeR{$\vec{x}$}{Vector of nine independent variables used by \cobra{}.}{}
\nomtypeR{$\momVec{}$}{Vector of three momenta fields.}{$ \lbm{}\, \text{s}^{-1}$}
\nomtypeR{$\vec{E}^{*}$}{Approximation of temporal integral of $\vec{E}(\vec{y}(\vec{x}))$.}{}
\nomtypeR{$\vec{S}$}{Vector of operator based scale factors.}{}
\nomtypeR{$\vec{F}$}{Vector of nonlinear residuals.}{}
\nomtypeR{$\vec{E}$}{Vector of spatially discrete conservation equations.}{}
\nomtypeR{$\vec{e}$}{Vector of conservation equations, except the temporal derivatives.}{}
\nomtypeR{$\vec{y}$}{Vector of conserved quantities.}{}
\nomtypeR{$\vec{g}$}{Gravitational acceleration vector.}{$ \text{ft} \, \text{s}^{-2} $}
\nomtypeR{$\vec{J}$}{A given Jacobian matrix.}{}
\nomtypeR{$\vec{I}$}{Identity matrix.}{}
\nomtypeR{$\vec{K}$}{Matrix representing inter-continuity volume pressure coupling coefficients.}{}
\nomtypeR{$\vec{C}$}{Vector of unknowns for a given continuity volume.}{}
\nomtypeR{$\vec{r}$}{Vector of right-hand sides for a given continuity volume.}{}
\nomtypeR{$\vec{U}$}{Upper triangular matrix from \vec{LU} decomposition.}{}
\nomtypeR{$\vec{L}$}{Lower triangular matrix from \vec{LU} decomposition.}{}
\nomtypeR{$\vec{A}$}{Pressure matrix.}{}
\nomtypeR{$\vec{res}$}{Right hand side of pressure system.}{}
\nomtypeR{$\vec{Q}$}{Matrix representing inter-domain flow rate coefficients.}{}
\nomtypeR{$\vec{B}$}{Matrix representing pressure matrix inter-domain coupling coefficients.}{}
\nomtypeR{$\vec{Z}$}{Matrix representing a linear domain continuity volume's linear system.}{}
\nomtypeR{$\vec{W}$}{Matrix representing global inter-domain coupling coefficients.}{}
\nomtypeR{$\vec{w}$}{Vector representing inter-domain coupling coefficients for a given NBC volume.}{}
\nomtypeR{$V$}{Volume of a given continuity volume.}{$ \text{ft}^{3} $}
\nomtypeR{$x$}{Spatial location.}{$\text{ft}$}
\nomtypeR[A]{$A$}{Cross-sectional area.}{$ \text{ft}^{2} $}
\nomtypeR[A]{$\tilde{A}$}{Effective cross-sectional area between two flow paths.}{$ \text{ft}^{2} $}
\nomtypeR[A]{$A(t)_{r}$}{Time dependent area ratio in valve problem.}{}
\nomtypeR{$K(t)$}{Effective time dependent loss coefficient in valve problem.}{$\lbf{}\,\text{s}\,\text{ft}^{-1}$}
\nomtypeR{$v$}{Specific volume.}{$\text{ft}^{3} \, \lbm{}$}
\nomtypeR{$K_{o}$}{Base loss coefficient in valve problem.}{$\lbf{}\,\text{s}\,\text{ft}^{-1}$}
\nomtypeR{$T_{\text{CPU}}$}{CPU time taken for a simulation to run.}{$\text{s}$}
\nomtypeR{$R$}{Integral of nonlinear residuals.}{}
\nomtypeR{$\tilde{R}$}{Time-averaged integral of nonlinear residuals.}{}
\nomtypeR{$\tilde{R}_{M}$}{Time-moment integral of nonlinear residuals.}{}
\nomtypeR{$q_{n,l}$}{Energy transferred between the \ncg{} field and the liquid water phase.}{$ \text{BTU}$}
\nomtypeR{$q_{w,\phi}$}{Energy transferred between a solid structure and a fluid $\phi$.}{$\text{BTU}$}
\nomtypeR{$q_{i,\phi}$}{Energy transferred between the saturated interface and the fluid $\phi$.}{$ \text{BTU}$}
\nomtypeR{$c$}{Local speed of sound.}{$ \text{ft}\, \text{s}^{-1} $}
\nomtypeR{$\mathcal{L}_{\infty}$}{Infinity norm of a given vector.}{}
\nomtypeR{$\mathcal{L}_{2}$}{Euclidean norm of a given vector.}{}
\nomtypeR{$N_{\text{CPL}}$}{Number of \cobra{} -- \relap53d{} coupling interfaces.}{}
\nomtypeR{$N_{f}$}{Set of flow paths connected to a given continuity volume.}{}
\nomtypeR{$N_{s}$}{Number of stages in a multi-stage temporal integration scheme.}{}
\nomtypeR{$N_{t}$}{Number of successful timesteps in a simulated time span.}{}
\nomtypeR{$n_{t}$}{Number of time step refinements in timestep generation function.}{}
\nomtypeR{$N_{n}$}{Number of inter-domain connections in the global domain.}{}
\nomtypeR{$N_{u}$}{Number of variables in a residual or update vector.}{}
\nomtypeR{$N_{\text{lin}}$}{Number of continuity volumes in the linear domain.}{}
\nomtypeR{$N_{\text{nln}}$}{Number of continuity volumes in the nonlinear domain.}{}
\nomtypeR{$N_{\text{NBC}}$}{Set of inter-domain connection for a linear continuity volumes.}{}
\nomtypeR{$N_{c}$}{Set of continuity volumes connected to a given flow path.}{}
\nomtypeR{$S$}{Scale factor for an equation.}{}
\nomtypeR{$t$}{Time.}{$\text{s}$}
\nomtypeR{$C_{1}$}{Coefficient in phase transition function.}{}
\nomtypeR{$P$}{Pressure.}{$\text{psia}$}
\nomtypeR{$h$}{Specific enthalpy.}{$ \text{BTU}\, \lbm{}^{-1}$}
\nomtypeR{$f(x)$}{A generic function.}{}
\nomtypeR{$r_{f}$}{Refinement factor for timestep generation function.}{}
\nomtypeR{$\dot{m}$}{A generic momentum.}{$\lbf{}$}
\nomtypeR{$u$}{A generic velocity.}{$\text{ft}\, \text{s}^{-1}$}
\nomtypeR{$K_{i,\phi_1 \phi_2}$}{Effective coefficient for drag between fluids $\phi_1$ and $\phi_2$.}{$\lbf{}\,\text{s}\,\text{ft}^{-1}$}
\nomtypeR{$K_{w,\phi}$}{Effective coefficient for drag between wall and fluid $\phi$.}{$\lbf{}\,\text{s}\,\text{ft}^{-1}$}
\nomtypeR{$e_{p}$}{Error in calculated pressure}{$ \text{psia} $}
\nomtypeR{$T$}{End time of simulation.}{$\text{s} $}

\nomtypeG{$F_{\text{tol}}$}{Convergence tolerance with respect to the scaled residual vector.}{}
\nomtypeG{$\dtmax{}$}{Maximum timestep allowed for a simulation.}{$\text{s}$}
\nomtypeG{$\dt{}$}{Timestep between time $t^{n}$ and $t^{n+1}$.}{$\text{s}$}
\nomtypeG{$\dx{}$}{Axial length of a volume.}{$\text{ft}$}
\nomtypeG{$\eta$}{Apportionment factor for inter-phase mass transfer.}{}
\nomtypeG{$\Upsilon$}{Inter-field field source or sink of mass.}{$\lbm{}$}
\nomtypeG{$\rho$}{Density.}{$\lbm{} \, \text{ft}^{-3} $}
\nomtypeG{$\Gamma$}{Inter-phase source or sink of mass.}{$ \lbm{}$}
\nomtypeG{$\alpha$}{Volume fraction.}{}
\nomtypeG{$\delta \vec{x}$}{Vector of updates to the independent parameters.}{}
\nomtypeG{$\tau^{'}_{w,\phi}$}{Shear force from contact between the channel wall and fluid $\phi$.}{$\lbf{} \, \text{ft}^{-3}$}
\nomtypeG{$\tau^{'}_{i,\phi_{1} \phi_{2}}$}{Shear force from contact between the field or phase $\phi_{1}$ and $\phi_{2}$.}{$ \lbf{} \, \text{ft}^{-3} $}
\nomtypeG{$\Gamma \vec{u}^{'}$}{Momentum transfer due to the exchange of mass between the aqueous phases.}{$ \lbf{} $}
\nomtypeG{$\Upsilon \vec{u}^{'}$}{Momentum transfer due to the exchange of mass between the two liquid fields.}{$ \lbf{} $}
\nomtypeG{$\Gamma h^{'}_{\phi}$}{Energy transfer rate due to aqueous phase change.}{$ \text{BTU}$}
\nomtypeG{$\vec{\Xi}$}{Matrix of coefficients that converts the momentum vector to the flow rate vector.}{}
\nomtypeG{$\Psi$}{A given flow rate.}{}
\nomtypeG{$\vec{\Psi}$}{Vector of flow rates.}{}
\nomtypeG{$\Omega$}{The domain; the set of volumes that determines the domain.}{}
\nomtypeG{$\delta_{\text{tol}}$}{Convergence tolerance with respect to the relative update vector.}{}

\nomtypeT{k}{Newton iteration index.}
\nomtypeT{n}{Time index.}
\nomtypeT{s}{Temporal integral stage index.}
\nomtypeT{$'''$}{Per unit volume.}
\nomtypeT{b}{The time point at which the donored quantities are evaluated in the donoring operator.}
\nomtypeT{c}{The time point at which the donoring velocity is evaluated in the donoring operator.}
\nomtypeT{.}{Per second, a denotation of a rate.}
\nomtypeT{$C_{2}$}{Coefficient in phase transition function.}
\nomtypeT{p}{Generic power.}

\nomtypeS{$\phi$}{A generic field or phase depending upon context.}
\nomtypeS{n}{Indicating the \ncg{} component of the gaseous phase.}
\nomtypeS{g}{Indicating the gaseous phase.}
\nomtypeS{e}{Indicating the entrained liquid field.}
\nomtypeS{l}{Indicating either the continuous liquid field or the total liquid phase.}
\nomtypeS{m}{Relating to a momentum flow path.}
\nomtypeS{v}{Indicating the vapor component of the gaseous phase.}
\nomtypeS{c}{Relating to a continuity volume.}
\nomtypeS{e}{Relating to a continuity volume's energy equations.}
\nomtypeS{d}{Donored quantity.}
\nomtypeS{lin}{Relating to the linear domain or solver.}
\nomtypeS{nln}{Relating to the nonlinear domain or solver.}
\nomtypeS{a}{Averaged quantity.}
\nomtypeS{r}{Relative.}
\nomtypeS{dep}{Depleting phase.}
\nomtypeS{car}{Carrier phase.}
\nomtypeS{p}{Index representing a particular coordinate for a given inter-domain flow path.}
\nomtypeS{j}{The coordinate of a particular continuity volume.}
\nomtypeS{i}{Generic index representing a particular coordinate of a location.}
\nomtypeS{o(i)}{Coordinate of continuity volume connected via a flow path at coordinate i.}
\nomtypeS{s(i)}{Coordinate of continuity volume connected via a flow path at coordinate i.}


%============================================================================
% Make the page numbers Arabic (1, 2, etc)
%============================================================================
\clearpage\pagenumbering{arabic}

\group{Introduction}

\subgroup{Motivation}
Of primary use in the field of nuclear reactor safety analysis is simulation.
The ability to predict the behaviour of reactors during off-normal events is the key to the licensing and the operation of nuclear power plants.
Within the United States, this simulation capacity is provided by a relatively small number of main stream software suites, among which are the RELAP variants, COBRA variants, and MELCOR.
While each of these software products varies in their models and implementations, the underlying numeric techniques and capabilities are similar.
Traditionally, these system codes utilize a semi-implicit discritization scheme.
In order to solve the resulting system of equations, the most common methodology is to take a single netwon step. 
The underlying numeric methods are first order methods.

% New Paragraph
The ability to use higher order methods is important.
Since the development of the semi-implicit method, whose form was motivated by the limited computer resources available at the time, there have been great advances in both the methodology used to solve linear algebra problems and computer capabilities.

\subgroup{Objectives}
The objective of this dissertation is the design, implementation, and evaluation of a practical non-linear solution framework for reactor safety systems codes.
Specifically, an efficient and reliable solution methodology to the two-phase, three-field, fluid-dynamics and the solid-structure heat transfer system of coupled non-linear partial differential equations is sought.
The specific methodology should be capable of obtaining a consistent solution to the system of PDEs while also possessing.         % Chapter: Introduction
\group{Fluid Mechanics}

Of primary concern in reactor safety analysis is thermal-hydraulic behaviour of the nuclear core during off-normal conditions. In order to evaluate 

\subgroup{Two Phase Flow}
The physical behaviour of fluids within COBRA are represented by a set of differential equations.
These governing equations are a collection of balance laws.
Balance laws are statements of conservation that include external sources and sinks.
The physical quantities being tracked by COBRA are mass, momentum, and energy.
Since COBRA models two phase behaviour, the


  % Chapter: Background
%============================================================================
% mathematics.tex
%============================================================================
% This file contains:
% 	- Hydrodynamic Conservation Equations
%	- Structural Thermal Energy Equations
%	- Newton's Method
%	- Jacobian-Free Methodology

%============================================================================
\group{Mathematical Formulation}
% Begin Chapter
%============================================================================
As mentioned earlier, most reactor safety-analysis codes depend upon three discrete sets of physics.
These include, but are not necessarily limited to, two-phase hydrodynamics, heat-transfer between the fluid and a solid structure, and a nuclear power source.
The different physical phenomena are represented by a system of PDEs and ODEs that constitute a set of balance laws for mass, momentum, and energy.
This system encompasses different scales in both space and time.
\pagebreak
%============================================================================
\subgroup{Hydrodynamic Conservation Equations}
% Hydro. Cons. Eqns.
%============================================================================
\begin{minipage}{0.42\textwidth}
\begin{center}\textbf{Conserved Variables}\end{center}
\begin{align}
\Vec{q} & = 
\begin{bmatrix} 
\alpha \rho_g  \\
\alpha \left( \rho_v + \rho_g \right)\\
\left(1-\alpha\right) \rho_l \\
\alpha_e \rho_l \\
\alpha \left( \rho_v + \rho_g \right) \Vec{U}_v \\
\left( 1-\alpha \right) \rho_l \Vec{U}_l \\
\alpha_e \rho_l \Vec{U}_e \\
\alpha \left( \rho_v H_v+ \rho_g H_g \right)\\
\left( 1 -\alpha \right) \rho_l H_l
\end{bmatrix} 
\end{align}
\end{minipage}
\begin{minipage}{0.42\textwidth}
\begin{center}\textbf{Independent Variables}\end{center}
\begin{align}
\Vec{x} & = \begin{bmatrix}
\alpha \\
\alpha_e \\
P \\
\alpha P_g \\
\alpha \left( \rho_v + \rho_g \right) \Vec{U}_v \\
\left( 1-\alpha \right) \rho_l \Vec{U}_l \\
\alpha_e  \rho_l \Vec{U}_e \\
\alpha H_v \\
\left( 1-\alpha \right) H_l
\end{bmatrix}
\end{align}
\end{minipage}

Note that $\displaystyle \Vec{q}=f(\Vec{x})$.

\subsubgroup{Mass}
\begin{align}
\Ddt{\alpha   \rho_v}{\Vec{U}_v} & =  \Gamma   + \DivOne{\Vec{G}_v^{T}}\\
\Ddt{\alpha   \rho_g}{\Vec{U}_v} & =  \Gamma   + \DivOne{\Vec{G}_g^{T}}\\
\Ddt{\alpha_l \rho_l}{\Vec{U}_l} & = -\Gamma_l + \DivOne{\Vec{G}_l^{T}} - S''' \\
\Ddt{\alpha_e \rho_l}{\Vec{U}_e} & = -\Gamma_e + S'''
\end{align}

\subsubgroup{Momentum}

\begin{align}
\Ddt{\alpha \rho_g \Vec{U}_g }{\Vec{U}_g} & = \nonumber \\
-\alpha\;\nabla P + \alpha \rho_g g - \tau^{'''}_{wv}-\tau^{'''}_{I_{lv}}-\tau^{'''}_{I_{ev}}+\Gamma_e U^{'}+\Div{\alpha T_g^{T}} & \\
\Ddt{\alpha_e \rho_l \Vec{U}_e}{\Vec{U}_e} & = \nonumber \\
-\alpha_e\;\nabla P + \alpha_e \rho_l g - \tau^{'''}_{wl}+\tau^{'''}_{ev}+\Gamma_e U^{'}+S^{'''}U^{'} & \\
\Ddt{\alpha_l \rho_l \Vec{U}_l}{\Vec{U}_l} & = \nonumber \\
-\alpha_l\;\nabla P + \alpha_l \rho_l g - \tau^{'''}_{wl}+\tau^{'''}_{lv}-\Gamma_l U^{'}-S^{'''}U^{'}+\Div{\alpha_l T_l^{T}} &
\end{align}

\begin{align}
\DisDdtOne{\dot{\Vec{m}}_{i,j}} & = \min(A_{j},A_{j+1})\frac{1}{2}\left(\frac{\dot{\Vec{m}}_{i,j}^{n}}{A_{j}}+\frac{\dot{\Vec{m}}_{i,j+1}^{n}}{A_{j+1}}\right)\hat{\Vec{U}}_{i,j-\frac{1}{2}}^{n} 
\end{align}

\subsubgroup{Energy}

\begin{align}
\Ddt{\alpha \rho_g H_g}{\Vec{U}_g} & \nonumber \\
= \Gamma H^{'}_{v}+q_{iv}+q_{gl}+Q_g^{'''}-\Div{\alpha \Vec{q}_g^{T}} &\\
\DerivPar{\left(1-\alpha \right)\rho_l H_l}{t}+\Div{\alpha_l \rho_l H_l \Vec{U}_l} +\Div{\alpha_e \rho_l H_l \Vec{U}_e} & = \nonumber \\
-\Gamma H^{'}_l + q_{il} - q_{gl} + Q_l^{'''}-\Div{\alpha_l \Vec{q}_l^{T}} &
\end{align}
%============================================================================
\subgroup{Discrete Hydrodynamic Equations}
% Hydro. Cons. Eqns.
\pagebreak
%============================================================================
\subsubgroup{Axial Momentum}

\begin{IEEEeqnarray}{rCl}
F_g(\Vec{x}^{n+1}) & = & \overbrace{E_g({\Vec{x}^{n}})}^{\text{Purely explicit terms}}-\overbrace{A_{mom,j}\Delta z_j\left[ \alpha_g^n\frac{P_{J+1}^{n+1}-P_{J}^{n+1}}{\Delta z_j}\right]}^{\text{Semi-Implicit Pressure Term}}  \\
& - & \overbrace{A_{mom,j}\Delta z_j\left[ \frac{dP}{dz}\bigg|_{w,g}^{*}+\frac{dP}{dz}\bigg|_{i,lg}^{*}+\frac{dP}{dz}\bigg|_{i,eg}^{*}\right]}^{\text{Semi-Implicit Drag Terms}}\nonumber \\
& + & \overbrace{\mathcal{S}p^{n+1}_{g,j}}^{\text{Implicit Source Term}} -\overbrace{\frac{\left(M_{g,j}^{n+1}-M_{g,j}^{n}\right)}{\Delta t}\Delta z}^{\text{Time rate of change}}\nonumber \\
& = &  0 \nonumber
\end{IEEEeqnarray}

Current methodology in COBRA-IE for obtaining tentative newtime $\dot{\vec{M}}^{\widetilde{n+1}}$ and $\displaystyle \frac{d\dot{\vec{M}}}{d P}$.
This methodology is based on linearizing certain terms within the momentum equations.
These  equations are solved for the $\dot{\vec{M}}^{\widetilde{n+1}}$.
The derivative of these equations with respect to pressure is taken to find $\displaystyle \frac{d\dot{\vec{M}}}{d P}$.
\begin{IEEEeqnarray}{rCl}
\mat{J}_{0}\,\dot{\vec{M}}^{\widetilde{n+1}} & = & -\vec{E} - \vec{I}(\dot{\vec{M}}^{n}) + \dot{\vec{M}}^{n} \nonumber \\
\dot{\vec{M}}^{\widetilde{n+1}} & = & -\mat{J}^{-1}_{\,0}\,\vec{E} - \mat{J}^{-1}_{\,0}\,\vec{I}(\dot{\vec{M}}^{n}) + \mat{J}^{-1}_{\,0}\,\dot{\vec{M}}^{n} \nonumber
\end{IEEEeqnarray}

Proposed methodology for obtaining tentative newtime $\dot{\vec{M}}^{\widetilde{n+1}}$ and $\displaystyle \frac{d\dot{\vec{M}}}{d P}$.
This methodology is based on linearizing the the equations and taking a Newton Step.
\begin{IEEEeqnarray}{rCl}
\mat{J}_{0}\,\left[\dot{\vec{M}}^{n+1}_{k+1}-\dot{\vec{M}}^{n+1}_{k}\right] & = & -\vec{E} - \vec{I}(\dot{\vec{M}}^{n+1}_{k}) \nonumber \\
\dot{\vec{M}}^{n+1}_{k+1} & = & \dot{\vec{M}}^{n+1}_{k} - \mat{J}^{-1}_{0}\,\vec{E} - \mat{J}^{-1}_{0}\,\vec{I}(\dot{\vec{M}}^{n+1}_{k}) \nonumber
\end{IEEEeqnarray}

Difference between current process and new process.
\begin{IEEEeqnarray}{rCl}
\dot{\vec{M}}^{n+1}_{k+1} - \dot{\vec{M}}^{\widetilde{n+1}}_{k+1} & = & \dot{\vec{M}}^{n+1}_{k} - \mat{J}^{-1}_{0}\,\vec{I}(\dot{\vec{M}}^{n+1}_{k}) + \mat{J}^{-1}_{\,0}\,\vec{I}(\dot{\vec{M}}^{n}) - \mat{J}^{-1}_{\,0}\,\dot{\vec{M}}^{n}\nonumber \\
\dot{\vec{M}}^{n+1}_{1} - \dot{\vec{M}}^{\widetilde{n+1}}_{1} & = & \dot{\vec{M}}^{n} - \mat{J}^{-1}_{\,0}\,\dot{\vec{M}}^{n}\nonumber \\
\dot{\vec{M}}^{n+1}_{1} & = & \dot{\vec{M}}^{\widetilde{n+1}}_{1} + [\mat{I} - \mat{J}^{-1}_{\,0}]\,\dot{\vec{M}}^{n} \nonumber \\
\dot{\vec{M}}^{n+1}_{2} - \dot{\vec{M}}^{\widetilde{n+1}}_{2} & = & \dot{\vec{M}}^{n+1}_{1} - \mat{J}^{-1}_{0}\,\vec{I}(\dot{\vec{M}}^{n+1}_{1}) + \mat{J}^{-1}_{\,0}\,\vec{I}(\dot{\vec{M}}^{n+1}_{1}) - \mat{J}^{-1}_{\,0}\,\dot{\vec{M}}^{n+1}_{1}\nonumber \\
\dot{\vec{M}}^{n+1}_{2} & = & \dot{\vec{M}}^{\widetilde{n+1}}_{2} + [\mat{I} - \mat{J}^{-1}_{\,0}]\dot{\vec{M}}^{n+1}_{1}\nonumber \\
\dot{\vec{M}}^{n+1}_{k+1} & = & \dot{\vec{M}}^{\widetilde{n+1}}_{k+1} + [\mat{I} - \mat{J}^{-1}_{\,0}]\dot{\vec{M}}^{n+1}_{k} \nonumber
\end{IEEEeqnarray}

The nonlinear functional associated with this equation is as follows:

\begin{IEEEeqnarray}{rCl}
\overbrace{\FVS{F}{\Vec{M}^{n+1},\;P^{n+1}}}^{\text{Nonlinear Functional}} & = & 0 \nonumber \\
& = & \underbrace{\FVS{E}{\Vec{M}^{n}}}_{\text{Explict Terms}} +\underbrace{\FVS{I}{\Vec{M}^{n+1},\;P^{n+1}}}_{\text{Implicit Terms}}\nonumber
\end{IEEEeqnarray}
\begin{IEEEeqnarray}{rCl}
\Vec{x}^{n+1} & = & \begin{bmatrix} M_l^{n+1}\\M_g^{n+1}\\M_e^{n+1}\\P^{n+1}\end{bmatrix} \nonumber \\
\FVS{F}{\Vec{x}^{n+1}} & = & \FVS{E}{\Vec{x}^n}+\FVS{I}{\Vec{x}^{n+1}} \nonumber
\end{IEEEeqnarray}

Now, this nonlinear functional is solved using a Newton Step.

\begin{IEEEeqnarray}{rCl}
\Vec{F}(\Vec{x}_{k+1}^{n+1}) = \Vec{F}(\Vec{x}_{k}^{n+1}) + \Mat{J}\cdot\Vec{\delta x}_{k} & = & 0\nonumber \\
\Vec{\delta x}_k & =&  \Vec{x}^{n+1}_{k+1}-\Vec{x}^{n+1}_{k} \nonumber \\
\Vec{F}(\Vec{x}_{k}^{n+1}) + \Mat{J} \cdot\Vec{\delta x}_{k} & = & 0 \nonumber \\
\Mat{J} \cdot\Vec{\delta x}_{k} & = & -\Vec{F}(\Vec{x}_{k}^{n+1}) \nonumber
\end{IEEEeqnarray}

\begin{align}
\begin{bmatrix} 
\DerivParOne{\vec{I}_{l}}{\vec{\dot{m}}_{l}} & \DerivParOne{\vec{I}_{l}}{\vec{\dot{m}}_{g}}  & \DerivParOne{\vec{I}_{l}}{\vec{\dot{m}}_{e}} & \DerivParOne{\vec{I}_{l}}{\Delta P}\\
\DerivParOne{\vec{I}_{g}}{\vec{\dot{m}}_{l}} & \DerivParOne{\vec{I}_{g}}{\vec{\dot{m}}_{g}}  & \DerivParOne{\vec{I}_{g}}{\vec{\dot{m}}_{e}} & \DerivParOne{\vec{I}_{g}}{\Delta P}\\
\DerivParOne{\vec{I}_{e}}{\vec{\dot{m}}_{l}} & \DerivParOne{\vec{I}_{e}}{\vec{\dot{m}}_{g}}  & \DerivParOne{\vec{I}_{e}}{\vec{\dot{m}}_{e}} & \DerivParOne{\vec{I}_{e}}{\Delta P}
\end{bmatrix}_{0}
\cdot
\begin{bmatrix}
\Delta \vec{\dot{m}}_l \\
\Delta \vec{\dot{m}}_g \\
\Delta \vec{\dot{m}}_e \\
\Delta P
\end{bmatrix}_{k\rightarrow k+1} & =
-\begin{bmatrix}
\vec{E}_{l} \\
\vec{E}_{g} \\
\vec{E}_{e}
\end{bmatrix} -
\begin{bmatrix}
\vec{I}_{l} \\
\vec{I}_{g} \\
\vec{I}_{e}
\end{bmatrix}_{k} \\ 
% HERE IS A SEPERATE EQUATION
\begin{bmatrix} 
\DerivParOne{\vec{I}_{l}}{\vec{\dot{m}}_{l}} & \DerivParOne{\vec{I}_{l}}{\vec{\dot{m}}_{g}}  & \DerivParOne{\vec{I}_{l}}{\vec{\dot{m}}_{e}} \\
\DerivParOne{\vec{I}_{g}}{\vec{\dot{m}}_{l}} & \DerivParOne{\vec{I}_{g}}{\vec{\dot{m}}_{g}}  & \DerivParOne{\vec{I}_{g}}{\vec{\dot{m}}_{e}} \\
\DerivParOne{\vec{I}_{e}}{\vec{\dot{m}}_{l}} & \DerivParOne{\vec{I}_{e}}{\vec{\dot{m}}_{g}}  & \DerivParOne{\vec{I}_{e}}{\vec{\dot{m}}_{e}}
\end{bmatrix}_{0}
\cdot
\begin{bmatrix}
\Delta \vec{\dot{m}}_l \\
\Delta \vec{\dot{m}}_g \\
\Delta \vec{\dot{m}}_e
\end{bmatrix}_{k\rightarrow k+1} & =
-\vec{E} -
\vec{I}_{k} -
\begin{bmatrix}
\DerivParOne{\vec{I}_{l}}{\Delta P} \\
\DerivParOne{\vec{I}_{g}}{\Delta P} \\
\DerivParOne{\vec{I}_{e}}{\Delta P}
\end{bmatrix}_{0}
\Delta P_{k \rightarrow k+1} \\
% HERE IS A SEPERATE EQUATION
\mat{J}_{\,0}
\cdot
\begin{bmatrix}
\Delta \vec{\dot{m}}_l \\
\Delta \vec{\dot{m}}_g \\
\Delta \vec{\dot{m}}_e
\end{bmatrix}_{k\rightarrow k+1} & =
-\vec{E} -
\vec{I}_{k} -
\vec{p}_{0}
\Delta P_{k \rightarrow k+1} \\
% HERE IS A SEPERATE EQUATION
\begin{bmatrix}
\Delta \vec{\dot{m}}_l \\
\Delta \vec{\dot{m}}_g \\
\Delta \vec{\dot{m}}_e
\end{bmatrix}_{k\rightarrow k+1} & =
-\mat{J}_{\,0}^{-1}\left(\vec{E} + \vec{I}_{k}\right) -
\mat{J}_{\,0}^{-1} \cdot \vec{p}_{0} \Delta P_{k \rightarrow k+1} \\
% HERE IS A SEPERATE EQUATION
\vec{\dot{m}}^{n+1}_{k+1}- \vec{\dot{m}}^{n+1}_{k} & =
-\mat{J}_{\,0}^{-1}\left(\vec{E} + \vec{I}_{k}\right) -
\mat{J}_{\,0}^{-1} \cdot \vec{p}_{0} \Delta P_{k \rightarrow k+1}\\
% HERE IS A SEPERATE EQUATION
\vec{\dot{m}}^{n+1}_{k+1} & =
\underbrace{\vec{\dot{m}}^{n+1}_{k} -\mat{J}_{\,0}^{-1}\left(\vec{E} + \vec{I}_{k}\right)}_{\vec{\dot{m}}^{*}_{k}} - \underbrace{\mat{J}_{\,0}^{-1} \cdot \vec{p}_{0}}_{\frac{\partial \vec{\dot{m}}}{\partial \Delta P}} \Delta P_{k \rightarrow k+1} \\
% HERE IS A SEPERATE EQUATION
\vec{\dot{m}}^{n+1}_{k+1} & =
\underbrace{-\mat{J}_{\,0}^{-1}\cdot \vec{E} + \vec{\dot{m}}^{n+1}_{k} - \mat{J}^{-1}_{\,0}\cdot \vec{I}_{k}}_{\vec{\dot{m}}^{n+1}_{*}} - \underbrace{\mat{J}_{\,0}^{-1} \cdot \vec{p}_{0}}_{\frac{\partial \vec{\dot{m}}}{\partial \Delta P}} \Delta P_{k \rightarrow k+1}
\end{align}

\begin{align}
% HERE IS A SEPERATE EQUATION
\vec{\dot{m}}^{n+1}_{k+1} & =
\underbrace{-\mat{J}_{\,0}^{-1}\cdot \vec{E} + \vec{\dot{m}}^{n+1}_{k} - \mat{J}_{\,0}\cdot \vec{I}_{k}}_{\vec{\dot{m}}^{n+1}_{*}} - \underbrace{\mat{J}_{\,0}^{-1} \cdot \vec{p}_{0}}_{\frac{\partial \vec{\dot{m}}}{\partial \Delta P}} \Delta P_{k \rightarrow k+1} \\
% HERE IS A SEPERATE EQUATION
\vec{\dot{m}}^{n+1}_{k+1} & =
\underbrace{-\mat{J}_{\,0}^{-1}\cdot \vec{E} + \vec{\dot{m}}^{n+1}_{k} - \mat{J}_{\,0}\cdot \vec{I}_{k}}_{\vec{\dot{m}}^{n+1}_{*}} - \frac{\partial \vec{\dot{m}}}{\partial \Delta P}\bigg|_{0} \Delta P_{k \rightarrow k+1} \\
% HERE IS A SEPERATE EQUATION
\vec{\dot{m}}^{n+1}_{k+1} & =
\underbrace{-\mat{J}_{\,0}^{-1}\cdot \vec{E} + \left(\mat{I} - \mat{J}^{-1}_{\,0}\cdot \mat{\alpha}_{0}\right)\cdot\vec{x}^{n+1}_{k}}_{\vec{\dot{m}}^{n+1}_{*}} - \frac{\partial \vec{\dot{m}}}{\partial \Delta P}\bigg|_{0} \Delta P_{k \rightarrow k+1}
\end{align}


\begin{align}
\Mat{J}_{0} & \equiv -\frac{2 A_{mom}\Delta t}{\Delta z}\cdot \nonumber \\
&\cdot \begin{bmatrix} 
K^{n}_{w,l} + \frac{K^{n}_{i,lg}}{\Ave{\alpha\rho}^{n}_{l}} +\frac{1}{2} &  -\frac{K^{n}_{i,lg}}{\Ave{\alpha\rho}^{n}_{g}} & 0\\
-\frac{K^{n}_{i,lg}}{\Ave{\alpha\rho}^{n}_{l}} &  K^{n}_{w,g} + \frac{K^{n}_{i,lg}}{\Ave{\alpha\rho}^{n}_{g}}+\frac{K^{n}_{i,eg}}{\Ave{\alpha\rho}^{n}_{g}} +\frac{1}{2} & -\frac{K^{n}_{i,eg}}{\Ave{\alpha\rho}^{n}_{e}}\\
0 & -\frac{K^{n}_{i,eg}}{\Ave{\alpha\rho}^{n}_{g}} &  K^{k}_{w,e} + \frac{K^{n}_{i,eg}}{\Ave{\alpha\rho}^{n}_{e}}+\frac{1}{2}\\
\end{bmatrix}
\end{align}

\subsubgroup{Linearization}
The semi-implicit method, with a single Newton Step linearizes about the old time value and solves for a single delta. This has several ramifications.

\subsubsubgroup{Wall Drag}
The beginning equation to formulate the wall drag is shown in Eqn.

\begin{IEEEeqnarray}{rCl}
\frac{dP}{dz}\bigg\vert^{n+1}_{k} & \equiv & \left(\frac{f^{n}}{D_h}+\frac{K_{form}}{\Delta z}\right) \left(\frac{1}{2 \rho^{n}}\right) \left(\frac{\dot{\vec{m}}^{n+1}_{k}}{A_{mom}}\right)^{2}\\
 & = & \left[\left(\frac{f^{n}}{D_h}+\frac{K_{form}}{\Delta z}\right) \left(\frac{1}{2 A^2_{mom} \rho^{n}}\right)\right] \left[\dot{\vec{m}}^{n+1}_{k}\right]^2\\
  & = & \left[K^{n}\right] \left[\dot{\vec{m}}^{n+1}_{k}\right]^2\\
\frac{\partial }{\partial_{\dot{\vec{m}}^{n+1}_{k}}} \left(\frac{dP}{dz}\bigg\vert^{n+1}_{k}\right) & = & \left[K^{n}\right]2\left[ \dot{\vec{m}}_{k}^{n+1}\right]
\end{IEEEeqnarray}

The Momentum Equations, outlined in Section \ref{blarp} contain an implicit wall drag. This means that we need to evaluate the wall drag at the future time value.
To do this in the semi-implicit methodology requires that we linearize the future value using Newton's Method.
The current implementation uses the following derivation.

Modifying this derivation to take into account the multiple Newton Steps results in the equations shown below.
For brevity, the phase index, $_k$, is dropped from this derivation.
Once the linearization starts, the index, $_k$, will refer to Newton iteration.

%============================================================================
\subgroup{Structural Thermal Energy Equations}
% Heat Structures
%============================================================================

%============================================================================
\subgroup{Newton's Method}
% Newton's Method
%============================================================================

COBRA currently uses a single linearized Newton step to solve the hydrodynamic equations.
This method has been shown to be adequate \ref{blarp} given a small enough time step.
One potential improvement that can be made to the current method is multiple Newton steps being taken with a frozen Jacobian.
The Jacobian will be evaluated at the old time value and not updated during the Newton process.


%============================================================================
\subgroup{Krylov Solvers}
% Krylov Solvers
%============================================================================
To solve a give newton iteraete, a Krylov subspace based method is used.
This methodology eliminates the requirement of analytically forming the Jacobian.   % Chapter: Mathematical formulation
\include{jfnk}          % Chapter: Jacobian Free Newton Krylov Overview

%=======================================================================
% Bibliography
%=======================================================================
%\bibliography{../references}    

%=======================================================================
% Appendices
%=======================================================================
%\appendix
\begin{appendices}

% code.tex
% this file is part of the example UW-Madison Thesis document
% It demonstrates one method for incorporating program listings
% into a document.
\chapter{Model stuff} \label{merp} 

Test Test Test 

\begin{landscape}
\begin{IEEEeqnarray}{lCr}
\begin{bmatrix}
  \begin{matrix} 
    x & x & x & 0 & 0 & x \\ 0 & x & 0 & x & x & x \\ x & x & x & 0 & 0 & x \\ 0 & x & 0 & x & 0 & x \\ 0 & x & 0 & x & x & x \\ x & x & x & 0 & 0 & x \\
    0 & 0 & 0 & 0 & 0 & x \\ 0 & 0 & 0 & 0 & 0 & x \\ 0 & 0 & 0 & 0 & 0 & x \\
    0 & 0 & 0 & 0 & 0 & 0 \\ 0 & 0 & 0 & 0 & 0 & 0 \\ 0 & 0 & 0 & 0 & 0 & 0 \\ 0 & 0 & 0 & 0 & 0 & 0 \\ 0 & 0 & 0 & 0 & 0 & 0 \\ 0 & 0 & 0 & 0 & 0 & 0
  \end{matrix} 
& \begin{matrix} 
    0 & \frac{\partial F_{J  ,1}}{\partial M_{g}} & 0 	\\ \frac{\partial F_{J,2}}{\partial M_{l}} & 0 & 0 	\\ 0 & \frac{\partial F_{J,3}}{\partial M_{g}} 	 & 0 	\\ \frac{\partial F_{J,4}}{\partial M_{l}} & 0 & \frac{\partial F_{J,4}}{\partial M_{e}} 	\\ 0 & 0 & \frac{\partial F_{J,5}}{\partial M_{e}} 	\\ \frac{\partial F_{J,6}}{\partial M_{l}} & 0 & 0 \\
    x & x 					  & 0 	\\ x 					   & x & x	\\ 0 & x 					 & x	\\ 
    0 & \frac{\partial F_{J+1,1}}{\partial M_{g}} & 0 	\\ \frac{\partial F_{J+1,2}}{\partial M_{l}} & 0 & 0 	\\ 0 & \frac{\partial F_{J+1,3}}{\partial M_{g}} & 0 	\\ \frac{\partial F_{J+1,4}}{\partial M_{l}} & 0 & \frac{\partial F_{J+1,4}}{\partial M_{e}} 	\\ 0 & 0 & \frac{\partial F_{J+1,5}}{\partial M_{e}} 	\\ \frac{\partial F_{J+1,6}}{\partial M_{l}} & 0 & 0 
  \end{matrix} 
& \begin{matrix}
    0 & 0 & 0 & 0 & 0 & 0 \\ 0 & 0 & 0 & 0 & 0 & 0 \\ 0 & 0 & 0 & 0 & 0 & 0 \\ 0 & 0 & 0 & 0 & 0 & 0 \\ 0 & 0 & 0 & 0 & 0 & 0 \\ 0 & 0 & 0 & 0 & 0 & 0 \\
    0 & 0 & 0 & 0 & 0 & x \\ 0 & 0 & 0 & 0 & 0 & x \\ 0 & 0 & 0 & 0 & 0 & x \\
    x & x & x & 0 & 0 & x \\ 0 & x & 0 & x & x & x \\ x & x & x & 0 & 0 & x \\ 0 & x & 0 & x & 0 & x \\ 0 & x & 0 & x & x & x \\ x & x & x & 0 & 0 & x
  \end{matrix}\\
\end{bmatrix}
  \begin{bmatrix}
    \delta(\alpha_g P_{nc}) \\ \delta \alpha_{g} \\ \delta(\alpha_{g} H_v) \\ \delta((1-\alpha_{g}) H_l) \\ \delta \alpha_e \\ \delta P \\ \delta M_l \\ \delta M_g \\ \delta M_e \\ \delta (\alpha_g P_{nc}) \\ \delta \alpha_{g} \\ \delta(\alpha_{g} H_v) \\ \delta((1-\alpha_{g}) H_l) \\ \delta \alpha_e \\ \delta P
  \end{bmatrix}
  & = & 
 -\begin{bmatrix}
    F_{J,1} \\ F_{J,2} \\ F_{J,3} \\ F_{J,4} \\ F_{J,5} \\ F_{J,6} \\ F_{j,1} \\ F_{j,2} \\ F_{j,3} \\ F_{J+1,1} \\ F_{J+1,2} \\ F_{J+1,3} \\ F_{J+1,4} \\ F_{J+1,5} \\ F_{J+1,6}
  \end{bmatrix}
\end{IEEEeqnarray}

\begin{IEEEeqnarray}{lCr}
\begin{bmatrix}
%
   \begin{matrix} x & x & x & 0 & 0 & x \\ 0 & x & 0 & x & x & x \\ x & x & x & 0 & 0 & x \\ 0 & x & 0 & x & 0 & x \\ 0 & x & 0 & x & x & x \\ x & x & x & 0 & 0 & x \end{matrix}
 & \begin{matrix} 0 & x & 0             \\ x & 0 & 0             \\ 0 & x & 0             \\ x & 0 & x             \\ 0 & 0 & x             \\ x & 0 & 0             \end{matrix}
 & \mat{0} \\
%
   \begin{matrix} 0 & 0 & 0 & 0 & 0 & x \\ 0 & 0 & 0 & 0 & 0 & x \\ 0 & 0 & 0 & 0 & 0 & x \end{matrix}
 & \mat{J}^{*}
 & \begin{matrix} 0 & 0 & 0 & 0 & 0 & x \\ 0 & 0 & 0 & 0 & 0 & x \\ 0 & 0 & 0 & 0 & 0 & x \end{matrix}\\
%
   \mat{0}
 & \begin{matrix} 0 & x & 0             \\ x & 0 & 0             \\ 0 & x & 0             \\ x & 0 & x             \\ 0 & 0 & x             \\ x & 0 & 0             \end{matrix}
 & \begin{matrix} x & x & x & 0 & 0 & x \\ 0 & x & 0 & x & x & x \\ x & x & x & 0 & 0 & x \\ 0 & x & 0 & x & 0 & x \\ 0 & x & 0 & x & x & x \\ x & x & x & 0 & 0 & x \end{matrix}\\
\end{bmatrix}
\begin{bmatrix}\delta(\alpha_g P_{nc}) \\ \delta \alpha_{g} \\ \delta(\alpha_{g} H_v) \\ \delta((1-\alpha_{g}) H_l) \\ \delta \alpha_e \\ \delta P \\ \delta M_l \\ \delta M_g \\ \delta M_e \\ \delta (\alpha_g P_{nc}) \\ \delta \alpha_{g} \\ \delta(\alpha_{g} H_v) \\ \delta((1-\alpha_{g}) H_l) \\ \delta \alpha_e \\ \delta P\end{bmatrix}
& = &
-\begin{bmatrix} F_{J,1} \\ F_{J,2} \\ F_{J,3} \\ F_{J,4} \\ F_{J,5} \\ F_{J,6} \\ F_{j,1} \\ F_{j,2} \\ F_{j,3} \\ F_{J+1,1} \\ F_{J+1,2} \\ F_{J+1,3} \\ F_{J+1,4} \\ F_{J+1,5} \\ F_{J+1,6}\end{bmatrix}
\end{IEEEeqnarray}

\begin{IEEEeqnarray}{lCr}
\begin{bmatrix}
%
\begin{matrix} x & x & x & 0 & 0 & x \\ 0 & x & 0 & x & x & x \\ x & x & x & 0 & 0 & x \\ 0 & x & 0 & x & 0 & x \\ 0 & x & 0 & x & x & x \\ x & x & x & 0 & 0 & x \end{matrix}
 & \begin{matrix} 0 & x & 0 \\ x & 0 & 0 \\ 0 & x & 0 \\ x & 0 & x \\ 0 & 0 & x \\ x & 0 & 0 \end{matrix}
 & \begin{matrix} 0 \end{matrix}\\
%
\begin{matrix} 0 & 0 & 0 & 0 & 0 & x \\ 0 & 0 & 0 & 0 & 0 & x \\ 0 & 0 & 0 & 0 & 0 & x \end{matrix}
 & \mat{J}^{*}
 & \begin{matrix} 0 & 0 & 0 & 0 & 0 & x \\ 0 & 0 & 0 & 0 & 0 & x \\ 0 & 0 & 0 & 0 & 0 & x \end{matrix}\\
%
\begin{matrix} 0 \end{matrix}
 & \begin{matrix} 0 & x & 0 \\ x & 0 & 0 \\ 0 & x & 0 \\ x & 0 & x \\ 0 & 0 & x \\ x & 0 & 0 \end{matrix}
 & \begin{matrix} x & x & x & 0 & 0 & x \\ 0 & x & 0 & x & x & x \\ x & x & x & 0 & 0 & x \\ 0 & x & 0 & x & 0 & x \\ 0 & x & 0 & x & x & x \\ x & x & x & 0 & 0 & x \end{matrix}\\
\end{bmatrix} \begin{bmatrix}\delta(\alpha_g P_{nc}) \\ \delta \alpha_{g} \\ \delta(\alpha_{g} H_v) \\ \delta((1-\alpha_{g}) H_l) \\ \delta \alpha_e \\ \delta P \\ \delta M_l \\ \delta M_g \\ \delta M_e \\ \delta (\alpha_g P_{nc}) \\ \delta \alpha_{g} \\ \delta(\alpha_{g} H_v) \\ \delta((1-\alpha_{g}) H_l) \\ \delta \alpha_e \\ \delta P\end{bmatrix}& = & -\begin{bmatrix} F_{J,1} \\ F_{J,2} \\ F_{J,3} \\ F_{J,4} \\ F_{J,5} \\ F_{J,6} \\ F_{j,1} \\ F_{j,2} \\ F_{j,3} \\ F_{J+1,1} \\ F_{J+1,2} \\ F_{J+1,3} \\ F_{J+1,4} \\ F_{J+1,5} \\ F_{J+1,6}\end{bmatrix}
\end{IEEEeqnarray}

\begin{IEEEeqnarray}{lCr}
\begin{bmatrix}
\begin{matrix} x & x & x & 0 & 0 & x \\ 0 & x & 0 & x & x & x \\ x & x & x & 0 & 0 & x \\ 0 & x & 0 & x & 0 & x \\ 0 & x & 0 & x & x & x \\ x & x & x & 0 & 0 & x \end{matrix} & \begin{matrix} 0 & x & 0 \\ x & 0 & 0 \\ 0 & x & 0 \\ x & 0 & x \\ 0 & 0 & x \\ x & 0 & 0 \end{matrix} & \begin{matrix} 0 \end{matrix}\\
\begin{matrix} 0 & 0 & 0 & 0 & 0 & -\frac{d M_l}{d P} \\ 0 & 0 & 0 & 0 & 0 & -\frac{d M_g}{d P} \\ 0 & 0 & 0 & 0 & 0 & -\frac{d M_e}{d P} \end{matrix} & \mat{I} & \begin{matrix} 0 & 0 & 0 & 0 & 0 & \frac{d M_l}{d P} \\ 0 & 0 & 0 & 0 & 0 & \frac{d M_g}{d P} \\ 0 & 0 & 0 & 0 & 0 & \frac{d M_e}{d P} \end{matrix}\\
\begin{matrix} 0 \end{matrix} & \begin{matrix} 0 & x & 0 \\ x & 0 & 0 \\ 0 & x & 0 \\ x & 0 & x \\ 0 & 0 & x \\ x & 0 & 0 \end{matrix} & \begin{matrix} x & x & x & 0 & 0 & x \\ 0 & x & 0 & x & x & x \\ x & x & x & 0 & 0 & x \\ 0 & x & 0 & x & 0 & x \\ 0 & x & 0 & x & x & x \\ x & x & x & 0 & 0 & x \end{matrix}\\
\end{bmatrix} \begin{bmatrix}\delta(\alpha_g P_{nc}) \\ \delta \alpha_{g} \\ \delta(\alpha_{g} H_v) \\ \delta((1-\alpha_{g}) H_l) \\ \delta \alpha_e \\ \delta P \\ \delta M_l \\ \delta M_g \\ \delta M_e \\ \delta (\alpha_g P_{nc}) \\ \delta \alpha_{g} \\ \delta(\alpha_{g} H_v) \\ \delta((1-\alpha_{g}) H_l) \\ \delta \alpha_e \\ \delta P\end{bmatrix}& = & -\begin{bmatrix} F_{J,1} \\ F_{J,2} \\ F_{J,3} \\ F_{J,4} \\ F_{J,5} \\ F_{J,6} \\ F_{j,1} \\ F_{j,2} \\ F_{j,3} \\ F_{J+1,1} \\ F_{J+1,2} \\ F_{J+1,3} \\ F_{J+1,4} \\ F_{J+1,5} \\ F_{J+1,6}\end{bmatrix}
\end{IEEEeqnarray}

\begin{IEEEeqnarray}{lCr}
\begin{bmatrix}
\begin{matrix} x & x & x & 0 & 0 & x-\frac{d M_g}{d P} \\ 0 & x & 0 & x & x & x-\frac{d M_l}{d P} \\ x & x & x & 0 & 0 & x-\frac{d M_g}{d P} \\ 0 & x & 0 & x & 0 & x-\frac{d M_l}{d P}-\frac{d M_e}{d P} \\ 0 & x & 0 & x & x & x-\frac{d M_e}{d P} \\ x & x & x & 0 & 0 & x-\frac{d M_l}{d P} \\ 0 & 0 & 0 & 0 & 0 & \frac{d M_g}{d P} \\ 0 & 0 & 0 & 0 & 0 & \frac{d M_l}{d P} \\ 0 & 0 & 0 & 0 & 0 & \frac{d M_g}{d P} \\ 0 & 0 & 0 & 0 & 0 & \frac{d M_g}{d P} + \frac{d M_g}{d P} \\ 0 & 0 & 0 & 0 & 0 & \frac{d M_e}{d P} \\ 0 & 0 & 0 & 0 & 0 & \frac{d M_l}{d P}\end{matrix}  & \begin{matrix} 0 & 0 & 0 & 0 & 0 & \frac{d M_g}{d P} \\ 0 & 0 & 0 & 0 & 0 & \frac{d M_l}{d P} \\ 0 & 0 & 0 & 0 & 0 & \frac{d M_g}{d P} \\ 0 & 0 & 0 & 0 & 0 & \frac{d M_g}{d P} + \frac{d M_g}{d P} \\ 0 & 0 & 0 & 0 & 0 & \frac{d M_e}{d P} \\ 0 & 0 & 0 & 0 & 0 & \frac{d M_l}{d P} \\ x & x & x & 0 & 0 & x-\frac{d M_g}{d P} \\ 0 & x & 0 & x & x & x-\frac{d M_l}{d P} \\ x & x & x & 0 & 0 & x-\frac{d M_g}{d P} \\ 0 & x & 0 & x & 0 & x-\frac{d M_l}{d P}-\frac{d M_e}{d P} \\ 0 & x & 0 & x & x & x-\frac{d M_e}{d P} \\ x & x & x & 0 & 0 & x-\frac{d M_l}{d P}\end{matrix}\\
\end{bmatrix}\begin{bmatrix}\delta(\alpha_g P_{nc}) \\ \delta \alpha_{g} \\ \delta(\alpha_{g} H_v) \\ \delta((1-\alpha_{g}) H_l) \\ \delta \alpha_e \\ \delta P \\ \delta (\alpha_g P_{nc}) \\ \delta \alpha_{g} \\ \delta(\alpha_{g} H_v) \\ \delta((1-\alpha_{g}) H_l) \\ \delta \alpha_e \\ \delta P\end{bmatrix}
\end{IEEEeqnarray}
\end{landscape}
\pagebreak

				% Appendix: Verification Work
\include{models2}
% code.tex
% this file is part of the example UW-Madison Thesis document
% It demonstrates one method for incorporating program listings
% into a document.
\chapter{Model stuff}

Test Test Test 

\pagebreak



\end{appendices}

%=======================================================================
% End Document
%=======================================================================
\end{document}