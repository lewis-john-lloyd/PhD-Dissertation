%=====================================================================
% main.tex
%=====================================================================
% This file contains:
%	- Document Class
%	- Packages
%	- Format Information
%	- Custom Commands
%	- Chapters
%	- Bibliography
%	- Appendices	
%	- Curriculum Vita

%=====================================================================
% Document Style
%=====================================================================
% The margincheck option flags lines which overflow their hbox with a black
%  box at the end of the line.  This usually (but not always) indicates a
%  margin violation on the right margin.  Left margin violations aren't
%  indicated and if the margin violation is large enough, there isn't room
%  for the black box to be visiable.  

% A 12 Point UW PhD Thesis.
\listfiles
\documentclass[margincheck]{withesis}

%=====================================================================
% Packageges
%=====================================================================
\usepackage[T1]{fontenc}

\usepackage{times}
\renewcommand{\ttdefault}{cmtt}

\usepackage[usenames,dvipsnames,svgnames,table]{xcolor}

\usepackage{etoolbox}
\usepackage{ragged2e}

\usepackage[normalem]{ulem}

\usepackage{amssymb}
\usepackage{amsmath}
\allowdisplaybreaks[1]

\usepackage{verbatim}

\usepackage{tikz}
\usetikzlibrary{shapes}
\usetikzlibrary{arrows}
\usetikzlibrary{patterns}

\usetikzlibrary{external}
\tikzexternalize[shell escape=-enable-write18]
\tikzset{external/system call={lualatex \tikzexternalcheckshellescape -halt-on-error -interaction=batchmode -jobname "\image" "\texsource"}}
%\tikzset{external/force remake}

\usepackage{pgfplots}
\pgfplotsset{compat=newest}
\pgfplotsset{plot coordinates/math parser=false}

\usepackage{pgfplotstable}
\pgfplotstableset{set thousands separator ={}}
\usepackage{booktabs}

\usepackage[simplified,nocolor]{pgf-umlcd}

\usepackage{multirow}

\usepackage{IEEEtrantools}

\usepackage{algpseudocode}

\usepackage[numbers, sort&compress]{natbib}
\renewcommand{\bibname}{List of References}
\let\oldbibsection\bibsection
\renewcommand{\bibsection}{\oldbibsection\addcontentsline{toc}{chapter}{List of References}\singlespace\raggedright}

\usepackage[bookmarks,
	bookmarksdepth=2,
	pdfauthor={Lewis John Lloyd}, %
	colorlinks, %
	citecolor=black, %
	filecolor=black, %
	linkcolor=black, %
	urlcolor=black]{hyperref}

\usepackage[intoc]{nomencl}
\makenomenclature

\pagestyle{thesis}
\setcounter{errorcontextlines}{999}

%============================================================================
% commands.tex
%============================================================================
% This file contains:
% 	- Defined Variables
%	- Redefined math shorthand
%	- Defined math shorthand

%============================================================================
% Defined Variables
%============================================================================
% 	- \abstractType:
%		Use: Toggles the type of abstract to be used
%		Default Value: abstract
%		Options: abstract, umiabstract
\newcommand{\abstractType}{abstract}

%============================================================================
% Redefined Math Commands
%============================================================================
% 	- \Vec{1} or \vec{1}
%		Long Name: Vector
%		Arguements[1]: bold and overbar arg1	
\DeclareRobustCommand{\Vec}[1]{%
    \ifmmode
        \mathbf{#1}\,%
    \else
        $\displaystyle \mathbf{#1}\,$%
    \fi
}
\DeclareRobustCommand{\vec}[1]{\Vec{#1}}

\DeclareRobustCommand{\lbm}{%
    \ifmmode
        \text{lb}_{\text{m}}
    \else
        $\displaystyle \text{lb}_{\text{m}}$%
    \fi
}
\DeclareRobustCommand{\lbf}{%
    \ifmmode
        \text{lb}_{\text{f}}
    \else
        $\displaystyle \text{lb}_{\text{f}}$
    \fi
}
\DeclareRobustCommand{\dt}{%
	\ifmmode
		\Delta t
	\else
		$Delta t$
	\fi
}
\DeclareRobustCommand{\dtmax}{%
	\ifmmode
		\Delta t_{\text{MAX}}
	\else
		$\Delta t_{\text{MAX}}$
	\fi
}
\DeclareRobustCommand{\dx}{%
	\ifmmode
		\Delta x
	\else
		$\Delta x$
	\fi
}

\delimitershortfall-1sp
\newcommand\abs[1]{\left|#1\right|}

\tikzstyle{Decision} = [diamond, draw, text width=4.5em, text badly centered, node distance=3cm, inner sep=0pt]
\tikzstyle{Action} = [rectangle, draw,text width=5em, text centered, node distance=3cm, rounded corners, minimum height=0em]
\tikzstyle{NodePoint} = [circle, draw, minimum height = 0 em, node distance = 3 cm]
\tikzstyle{BlackBox} = [rectangle, draw, text centered, node distance=1cm, fill=black!10]
\tikzstyle{line} = [draw, -latex']
    
\DeclareRobustCommand{\BlackBox}{\State \textbf{Black Box: }}
\DeclareRobustCommand{\Test}{\State \textbf{Test: }}
\DeclareRobustCommand{\Define}{\State \textbf{Define: }}
\DeclareRobustCommand{\Update}{\State \textbf{Update: }}
\DeclareRobustCommand{\Set}{\State \textbf{Set: }}
\DeclareRobustCommand{\Calculate}{\State \textbf{Calculate: }}
%\newcommand{\algorithmicset}{\textbf{Set:}}
%\algnewcommand\Solve{\item[\algorithmicset]}

\newcommand{\UnitsCol}[1]{\hfill\makebox[4em]{#1\hfill}\ignorespaces}

\newcommand{\nomsubtitle}[1]{\item[\large\bfseries #1]}

\renewcommand\nomgroup[1]{\def\nomtemp{\csname nomstart#1\endcsname}\nomtemp}

\newcommand{\nomstartR}{\nomsubtitle{Roman Symbols}%
	\item[\bfseries Symbol]%
	\textbf{Description}%
	\UnitsCol{\textbf{Units}}}

\newcommand{\nomstartG}{\nomsubtitle{Greek Symbols}%
	\item[\bfseries Symbol]%
	\textbf{Description}%
	\UnitsCol{\textbf{Units}}}

\newcommand{\nomstartA}{\nomsubtitle{Acronyms}}

\renewcommand*{\nompreamble}{\markboth{\nomname}{\nomname}}

\newcommand{\nomdescr}[1]{\parbox[t]{11cm}{\singlespace \RaggedRight #1}}
\newcommand{\nomwithunits}[4]{\nomenclature[#1]{#2}{\nomdescr{#3}\UnitsCol{#4}}}

\newcommand{\nomtypeR}[4][]{\nomwithunits{R#1}{#2}{#3}{#4}}
%\nomtypeR[A]{\(A\)}{a dimensionless coefficient}{--}{1}
\newcommand{\nomtypeG}[4][]{\nomwithunits{G#1}{#2}{#3}{#4}}

\newcommand{\nomtypeA}[3][]{\nomenclature[A#1]{#2}{\nomdescr{#3}}}
%\newcommand{\nomtypeD}[4][]{\nomenclature[D#1]{#2}{\nomdescr{#3}\DefinitionCol{#4}}}

\setlength{\nomitemsep}{-\parsep}



\numberwithin{figure}{chapter}
%=======================================================================
% Start Document
%=======================================================================
%\includeonly{chap_introduction,chap_nonlinear_solver,chap_dom_decomp,chap_final_thoughts}
%=======================================================================
% Start Document
%=======================================================================
\begin{document}
%=======================================================================
% Preface
%=======================================================================
% Make the page numbers Roman (i, ii, etc)
\clearpage\pagenumbering{roman}  

% ============================================================================
% Title Page
% ============================================================================
\title{Investigation of a Spatially Selective Nonlinear Solver for Thermal-Hydraulic Safety Analysis Codes}
\author{Lewis John Lloyd}
\date{2014}
\department{Nuclear Engineering and Engineering Physics}
\advisorname{Michael Corradini}
\advisortitle{Professor}
%\maketitle

%============================================================================
% Copyright Page
%============================================================================
%\copyrightpage

%============================================================================
% Abstract
%============================================================================
%		Options: abstract, umiabstract
%\newcommand{\abstractType}{abstract}
%\begin{\abstractType}
%	%============================================================================
% abstract.tex
%============================================================================
The methods used to simulate the thermal-hydraulic behavior in the core of a nuclear power plant during postulated accidents can be characterized by the manner in which the governing conservation equations are discretized and the manner in which any nonlinearities present in those fully discrete equations are resolved.
While each method has a different way of approximating the governing equations, they all require that a discrete nonlinear problem be approximately solved at each timestep.
This is done either through a single Newton step or through an iterative Newton procedure.

The primary advantage of using a single Newton step is the low computational cost; however, the accuracy of a linear approximation in regions of highly non-linear physics may be suspect.
This has traditionally been mitigated by limitations placed upon the maximum change in independent parameters within a timestep.
Alternatively, by resolving the nonlinearities within a timestep through an iterative Newton solver, the errors from the linear approximation are reduced; however, the computational cost of global Newton methods is high.

For spatially isolable nonlinearities the computational expenditure of iteratively solving the global nonlinear problem may be unnecessary.
The objectives of this research include the design, implementation, and evaluation of a novel, spatially selective, nonlinear solution method for nuclear thermal-hydraulic safety analysis.
Isolation of subdomains where nonlinearities are high will be achieved by domain decomposition.
The method of decomposition chosen enables feedback across the subdomain boundaries. 
Upon isolation, the nonlinear subdomain will be subjected to a globalized Newton method to resolve the local nonlinearities.
The nonlinearly converged solution from the subdomain will then be communicated via coupling coefficients to the rest of the problem domain for use in calculating its single Newton step.
This unique use of selective nonlinear refinement via domain coupling may provide a route to nonlinearly converged timestep size insensitive solutions for traditional two-phase flow methods at a lower computational cost.
%\end{\abstractType}

%============================================================================
% Acknowledgement Page
%============================================================================
%\begin{acknowledgments}
%  This research was performed under appointment to the Rickover Fellowship 
Program in Nuclear Engineering sponsored by Naval Reactors Division of the 
U.S. Department of Energy. 
%\end{acknowledgments}

%============================================================================
% Auto-generated Tables
%============================================================================
\tableofcontents  % Table of contents
\listoftables     % List of tables
\listoffigures    % List of figures
\listofalgorithms % List of algorithms

%============================================================================
% Nomenclature Page
%============================================================================
%============================================================================
% nomenclature.tex
%============================================================================
% This file contains:
% 	- List of defined nomenclature items

%============================================================================
% Nomenclature List
%============================================================================

\begin{nomenclature}
\begin{longtable}{p{.13\textwidth} p{.67\textwidth} c} 
\NomItem{NPP}{Nuclear Power Plan}{}
%
\NomItem{NRC}{Nuclear Regulatory Commission}{}
%
\NomItem{10CFR}{Title 10 of the Code of Federal Regulations}{}
%
\NomItem{10CFR50}{Part 50 of 10CFR}{}
%
\NomItem{SAR}{Safety Analysis Report}{}
%
\NomItem{LWR}{Light Water Reactor}{}
%
\NomItem{ECCS}{Emergency Core Cooling System}{}
%
\NomItem{LOCA}{Loss-of-Coolant Accident}{}
%
\NomItem{PWR}{Pressurized Water Reactor}{}
%
\NomItem{BWR}{Boiling Water Reactors}{}
%
\NomItem{$L$}{Axial length of the problem}{
$\displaystyle \left[\, \text{ft} \,\right]$
}
%
\NomItem{$M$}{Number of sections in problem}{}
%
\NomItem{$S_{i}$}{Indexed section}{}
%
\NomItem{$J_{i}$}{Number of axial continuity volumes in section $i$}{}
%
\NomItem{$\Delta x_{i,j}$}{Indexed axial continuity volume length}{
$\displaystyle \left[\, \text{ft} \,\right]$
}
%
\NomItem{$T(i)$}{Number of channels in section $i$}{}
%
\NomItem{K}{Total number of channels in a problem}{}
%
\NomItem{$A_{c_{k,j}}$}{Cross-sectional area of indexed continuity volume}{
$\displaystyle \left[\, \text{ft}^2 \,\right]$
}
%
\NomItem{$A_{m_{k,j}}$}{Cross-sectional area of indexed momentum volume}{
$\displaystyle \left[\, \text{ft}^2 \,\right]$
}
%
\NomItem{$\omega_{k}$}{Volume of a geometric sub-component}{
$\displaystyle \left[\, \text{ft}^3 \,\right]$
}
%
\NomItem{$\Omega$}{Volume of whole domain}{
$\displaystyle \left[\, \text{ft}^3 \,\right]$
}
%
\NomItem{$\tilde{a}$}{Cross-sectionally averaged variable}{}
%
\NomItem{A}{Cross-sectional flow area}{
$\displaystyle \left[\, \text{ft}^2 \,\right]$
}
%
\NomItem{$S^{'''}$}{Volumetric inter-field source or sink of mass}{
$\displaystyle \left[\, \frac{ \lbm{} }{ \text{ft}^3 \text{s} } \,\right]$
}
%
\NomItem{$\Gamma^{'''}$}{Volumetric inter-phase source or sink of mass}{
$\displaystyle \left[\, \frac{ \lbm{} }{\text{ft}^3 \text{s} } \,\right]$
}
%
\NomItem{$s_{m,k}$}{Volumetric source or sink of mass for a given field or phase k}{
$\displaystyle \left[\, \frac{ \lbm{} }{\text{ft}^3 \text{s} } \,\right]$
}
%
\NomItem{$\eta$}{Apportionment factor}{
$\displaystyle \left[\, - \,\right]$
}
%
\NomItem{$\vec{u}_{k}$}{Velocity vector for field or phase k}{
$\displaystyle \left[\, \frac{\text{ft}}{\text{s}} \,\right]$
}
%
\NomItem{$\rho_{k}$}{Density of field or phase k}{
$\displaystyle \left[\, \frac{ \lbm{} }{\text{ft}^3} \,\right]$
}
%
\NomItem{$\alpha_{k}$}{Volume fraction of field or phase k}{
$\displaystyle \left[\, - \,\right]$
}
%
\NomItem{P}{Pressure}{
$\displaystyle \left[\, \text{psia} \, \right]$
}
%
\NomItem{$h_{k}$}{Enthalpy of field or phase k}{
$\displaystyle \left[\, \frac{\text{BTU}}{ \lbm } \,\right]$
}
%
\NomItem{$\Gamma^{'''} h^{'}_{k}$}{Volumetric energy transfer due to aqueous phase change}{
$\displaystyle \left[\, \frac{\text{BTU}}{\text{ft}^3 \text{s}} \,\right]$
}
%
\NomItem{$q^{'''}_{i,k}$}{Volumetric energy transfer between the saturated interface and any given field or phase k}{
$\displaystyle \left[\, \frac{\text{BTU}}{\text{ft}^3 \text{s}} \,\right]$
}
%
\NomItem{$q^{'''}_{g,l}$}{Volumetric energy transfer between the \ncg{} field and the liquid water phase}{
$\displaystyle \left[\, \frac{\text{BTU}}{\text{ft}^3 \text{s}} \,\right]$
}
%
\NomItem{$q^{'''}_{w,k}$}{Volumetric energy transfer between a solid structure and any given field or phase k}{
$\displaystyle \left[\, \frac{\text{BTU}}{\text{ft}^3 \text{s}} \,\right]$
}
%
\NomItem{$\alpha_k \frac{\delta P}{\delta t}$}{Pressure work done by a given field or phase k}{
$\displaystyle \left[\, \frac{\text{BTU}}{\text{ft}^3 \text{s}} \,\right]$
}
%
\NomItem{$s_{e,k}$}{External source or sink of volumetric energy for a given field or phase k}{
$\displaystyle \left[\, \frac{\text{BTU}}{\text{ft}^3 \text{s}} \,\right]$
}
%
\NomItem{$\vec{g}$}{Gravitational acceleration vector}{
$\displaystyle \left[\, \frac{\text{ft}}{\text{s}^2} \,\right]$
}
%
\NomItem{$\tau^{'}_{w,k}$}{Shear force from contact between the channel wall and field or phase k}{
$\displaystyle \left[\, \frac{ \lbf{} }{\text{ft}^3} \,\right]$
}
%
\NomItem{$\tau^{'}_{i,k_{1} k_{2}}$}{Shear force from contact between the field or phase $k_{1}$ and $k_{2}$}{
$\displaystyle \left[\, \frac{ \lbf{} }{ \text{ft}^3 } \,\right]$
}
%
\NomItem{$\Gamma^{'''} \vec{u}^{'}$}{Momentum transfer due to the exchange of mass between the aqueous phases}{
$\displaystyle \left[\, \frac{ \lbf{} }{ \text{ft}^3 } \,\right] $
}
%
\NomItem{$S^{'''} \vec{u}^{'}$}{Momentum transfer due to the exchange of mass between the two liquid fields}{
$\displaystyle \left[\, \frac{ \lbf{} }{ \text{ft}^3 } \,\right] $
}
%
\NomItem{$s_{p,k}$}{External source or sink of momentum for a given field or phase k}{
$\displaystyle \left[\, \frac{ \lbf{} }{ \text{ft}^3 } \,\right] $
}
%
\NomItem{$\vec{e}$}{Vector of conservation equations, except the temporal derivatives of the conserved quantities}{

}
%
\NomItem{$\vec{y}$}{Vector of conserved quantities}{

}
%
\NomItem{$V_{j}$}{Indexed volume}{
$\displaystyle \left[\, \text{ft}^3 \,\right]$
}
%
\NomItem{$\don{a}_{d,j \pm \onehalf }$}{Donored quantity}{

}
%
\NomItem{$\ave{a}_{a,j \pm \onehalf }$}{Average quantity}{

}
%
\NomItem{$\vec{E}$}{Vector of spatially discrete conservation equations}{

}
%
\NomItem{$\vec{x}$}{Vector of nine independent variables used by \cobra{}}{

}
%
\NomItem{$\dot{m}_{k}$}{Momentum of a given phase or field flowing through a cross-sectional area}{
$\displaystyle \left[\, \frac{ \lbm }{\text{s}} \,\right]$
}
%
\NomItem{$\vec{E}^{*}$}{Approximation of temporal integral of $\vec{E}(\vec{y}(\vec{x}))$}{

}
%
\NomItem{$\delta \vec{x}^{k}$}{Newton update vector}{

}
%
\NomItem{$\lambda_j $}{Linesearch scaling parameter}{
$\displaystyle \left[\, - \,\right]$
}
%
\NomItem{$\alpha $}{Linesearch termination parameter}{
$\displaystyle \left[\, - \,\right]$
}
%
\NomItem{$\vec{S} $}{Vector of scaling parameters}{

}
%
\NomItem{$\vec{F} $}{Vector of nonlinear residuals}{

}
\NomItem{$R $}{Integral of nonlinear residuals}{
$\displaystyle \left[\, - \,\right]$
}
\NomItem{$\tilde{R} $}{Time-averaged integral of nonlinear residuals}{
$\displaystyle \left[\, - \,\right]$
}
%
\NomItem{$\tilde{R}_{M} $}{Time-moment integral of nonlinear residuals}{
$\displaystyle \left[\, - \,\right]$
}
%
\NomItem{\dtmax{}}{Maximum timestep allowed for a simulation}{
$\displaystyle \left[\, \text{s} \,\right]$
}
%
\NomItem{\dt{}}{Timestep between time $t^{n}$ and $t^{n+1}$ for a simulation}{
$\displaystyle \left[\, \text{s} \,\right]$
}
%
\NomItem{$n^{n+1}_{g,j}$}{Flux of \ncg{} mass in RELAP5-3D}{
$\displaystyle \left[\, \frac{ \lbm{} }{\text{s}} \,\right]$
}
%
\NomItem{$m^{n+1}_{k,j}$}{Flux of mass for phase k in RELAP5-3D}{
$\displaystyle \left[\, \frac{ \lbm{} }{\text{s}} \,\right]$
}
%
\NomItem{$w^{n+1}_{k,j}$}{Volumetric flux for phase k in RELAP5-3D}{
$\displaystyle \left[\, \frac{ \text{ft}^3 }{\text{s}} \,\right]$
}
%
\NomItem{$u^{n+1}_{k,j}$}{Flux of internal energy for phase k in RELAP5-3D}{
$\displaystyle \left[\, \frac{ \text{BTU} }{\text{s}} \,\right]$
}
%
\LastNomItem{$t^{i}$}{Discrete points in time}{
$\displaystyle \left[\, \text{s} \,\right]$
}
\end{longtable}
\end{nomenclature}

%=======================================================================
% Chapters
%=======================================================================
% Make the page numbers Arabic (1, 2, etc)
\clearpage\pagenumbering{arabic}

\chapter{Introduction}
\label{chap:intro}
In the United States, the goal of Nuclear Power Plant (NPP) designers, builders, operators, and regulators is to ensure the safety of the public during both normal operations and postulated accident scenarios.
It is the responsibility of the Nuclear Regulatory Commission (NRC) to issue licenses for the construction and operation of nuclear reactors.
Chapter 1 of Title 10 of the Code of Federal Regulations (10CFR) details the regulatory procedures that govern the NRC.
Part 50 of 10CFR (10CFR50) lays out the process by which an applicant can obtain both construction and operating licenses for NPPs.
One of the documents required prior to the issuance of any license under 10CFR50 is a safety analysis report (SAR).
In part, SARs for Light Water Reactors (LWRs) require the applicant to provide an evaluation of their emergency core cooling systems during postulated loss-of-coolant accidents (LOCA).
This evaluation must conform with section 46 of 10CFR50, which requires the applicant to perform analyses for ``a number of postulated loss-of-coolant accidents of different sizes, locations, and other properties sufficient to provide assurance that the most severe postulated loss-of-coolant accidents are calculated" \cite{CFR10}.
This requires designers and operators of NPPs to model the thermal-hydraulic behavior within the core of a reactor during postulated LOCAs.  
The diverse physical conditions experienced by the reactor during postulated accidents necessitate the inclusion of a range of complex physics during safety analyses.
Each of these physics has dedicated pieces of software that are under continual development to improve their predictive capabilities.
The work that follows is concerned with the mathematical formulation and solution of the equations governing the thermal-hydraulic behavior within the reactor core.

%-------------------------------------------------------------------------------
%-------------------------------------------------------------------------------
%-------------------------------------------------------------------------------
\section{Design Motivation}
\label{sect:motivation}

All commercial reactors operating within the United States are of an LWR design.
The safety analyses required for licensing necessitate the modeling of water in both liquid and gaseous phases.
This has driven the development of safety software that can model the behavior of water under an extensive range of thermodynamic states, including multiple phases.
Several codes are widely available for simulating the thermal-hydraulic response of an NPP during postulated accidents.
These safety codes can be divided into two large categories: system analysis codes and subchannel analysis codes.
While there is considerable overlap between the capabilities of these two categories, each has its own particular strengths and weaknesses.
Three well-known system-level safety analysis software packages are \relap53d{} \cite{RELAP}, TRACE \cite{TRACE}, and MELCOR \cite{Summers1994}.
Two well-known subchannel analysis software packages are COBRA \cite{Thurgood1983c} and VIPRE.
A brief description of these two types of software is provided below.

The system analysis software has historically been used for complete plant analysis.
These software systems have extensive collections of empirical models for pumps, valves, accumulators, etc.
These models allow for a macroscopic description of the balance of plant and its interaction with the
reactor core.
The reactor core itself is normally a low-fidelity representation that operates primarily as a heat source for the balance of plant model.
While the system software can be successfully used to model the balance of plant, more detailed physical models and hydrodynamics are required to accurately provide details of in-core behavior.

To address the need for accurate simulations of accident scenarios, software capable of modeling the dominant physics of interest within a reactor core was developed.
This is the second type of software: subchannel analysis.
The use of subchannel analysis software stems from physical arguments about dominant flow within a LWR.
To motivate these arguments, a brief discussion of the geometry within the nuclear reactor is presented.
The basic geometry of commercial nuclear reactor cores in the United States is a vertical, cylindrical array of fuel bundles, control rods, and instrumentation.
Within a fuel bundle cylindrical fuel rods are arranged into a lattice.
The presence of concentric geometric structures of axially aligned units creates natural subchannels for coolant flow.
This is the basis for one of the underlying assumptions of subchannel analysis software: the axial flow is the dominant flow.
This assumption allows for reduced complexity when modeling the inter-subchannel transverse flow.
In the work that follows the governing physics and the computational framework of interest will be taken from a variant of the aforementioned COBRA subchannel analysis code, which shall be referred to generically as \cobra{}.

%-------------------------------------------------------------------------------
%-------------------------------------------------------------------------------
%-------------------------------------------------------------------------------
\section{Research Objectives}
\label{sect:researchObjectives}

The methods used to simulate thermal-hydraulic behavior in NPPs are characterized by the manner in which the governing conservation equations are integrated over space and time and how their nonlinearities are resolved.
Each of the methods outlined in \sect{sect:solution_techniques} has one thing in common: the governing equations are solved over the global domain, either through a single linear approximation or through iterative, nonlinear convergence.

\begin{figure}[ht!]
\centering
\tikzsetnextfilename{images/my_diagram_eps}
\begin{tikzpicture}
\draw[pattern=north west lines, pattern color=blue] [thick] (-3,0) rectangle (0,3);
\draw (-1.5,3) node[anchor=south] {Single-Shot Linearization};
\draw (-3,2) node[anchor=east] {RELAP5-3D};
\draw (-3,1.5) node[anchor=east] {COBRA};
\draw (-3,1) node[anchor=east] {MARS};
\draw[pattern=north east lines, pattern color=red] [thick] (3,0) rectangle (6,3);
\draw (4.5,3) node[anchor=south] {Iterative Convergence};
\draw (6,2) node[anchor=west] {CATHARE};
\draw (6,1.5) node[anchor=west] {TRACE};
\draw (0,-4) rectangle +(3,3);
\path[pattern=north west lines, pattern color=blue] (0,-4) rectangle +(0.75,3);
\path[pattern=north west lines, pattern color=blue] (2.25,-4) rectangle +(0.75,3);
\path[pattern=north west lines, pattern color=blue] (0.75,-1.75) rectangle +(1.5,0.75);
\path[pattern=north east lines, pattern color=red] (0.75,-3.25) rectangle +(1.5,1.5);
\path[pattern=north west lines, pattern color=blue] (0.75,-4) rectangle +(1.5,0.75);
\draw (1.5,-1) node[anchor=south] {Proposed Method};
\end{tikzpicture}
\caption{Current and proposed paradigms for thermal-hydraulic safety analysis.}
\label{fig:my_diagram}
\end{figure}

The advantage of the linear approximation is the reduction in computational costs; however, the accuracy of this approximation at large timestep sizes in regions of highly nonlinear physics is suspect.
This limitation has traditionally been mitigated by restrictions placed upon the maximum permissible change in the independent parameters during a given timestep, which can in turn lead to excessively small timestep sizes.
The benefit of an iterative solver is that the nonlinear physics are resolved at each timestep by using multiple iterations.
In trade, the computational cost of those multiple iterations is high.
This research seeks to find a balance between the competing incentives of computational cost and accuracy by creating a hybrid method, as illustrated in \fig{fig:my_diagram}.
Specifically, if there are parts of the domain with a high degree of nonlinear physics that are spatially isolable, then the computational cost of solving the entire domain's governing equations iteratively may be unnecessary.
Instead, by restricting the multiple iterations to only those portions of the global domain where they are necessary, the computational cost will be reduced while maintaining accuracy. 

The objectives of this research were the design, implementation, and evaluation of a novel, spatially selective, nonlinear solution method for nuclear thermal-hydraulic safety analysis.
Isolation of a specified subdomain where nonlinearities are high was achieved by developing a new variant of the code-coupling algorithm discussed in \sect{sect:code_coupling}.
For this work, the subdomain to be isolated was comprised of geometric components predetermined by the user.
Upon isolation, the nonlinear subdomain was subjected to Newton's method to resolve any nonlinearities at every timestep.
This converged solution was then communicated via coupling coefficients to the remainder of the domain for use in finishing its single Newton step.
This unique use of domain decomposition for selective nonlinear-refinement via semi-implicit coupling provides a route for obtaining nonlinearly converged, timestep-size insensitive solutions for traditional two-phase flow methods with a lower computational cost than that of a globally iterative Newton's method.
The following chapters will provide an overview of: relevant background material and research, \chap{chap:background}; the work that was done to allow this research to be carried out, \chap{chap:nonlinear}; the domain decomposition algorithm, \chap{chap:domDecomposition}; and the results of numerical experiments, \chap{chap:results}.
     % Chapter: Introduction
\chapter{Background Material}
\label{chap:background}
This chapter provides a review of mathematical and theoretical material pertinent to this research.
First, the geometric discretization used in this work is discussed.
Second, the two-phase flow equations are detailed.
Third, the numerical approximations with respect to space, time, and nonlinearities are covered.
Next, the various solution methods currently used to solve the system of discrete governing equations are presented.
And last, an outline of the various means by which domain coupling of thermal-hydraulic safety analysis software has traditionally been achieved is provided.

%-------------------------------------------------------------------------------
%-------------------------------------------------------------------------------
%-------------------------------------------------------------------------------
\section{Geometric Discretization}
\label{sect:geometry}
The first step in any simulation of the thermal-hydraulic behavior of the core of a nuclear reactor is to create a discrete computational representation of the domain.
For this work the computational grid used to represent the physical model is a staggered mesh.
With a staggered mesh the domain is divided into two overlapping meshes.
There is one mesh comprised of continuity volumes and another comprised of momentum flow paths, as shown in \fig{fig:staggered_mesh}.
Thermodynamic variables are defined as constant over the continuity volumes, while momenta are defined as constant over momentum flow paths.
By definition the momentum flow paths do not contain mass or energy.
The boundaries of the continuity volumes align with the center of the momentum flow paths.

\begin{figure}[ht!]
\centering
\tikzsetnextfilename{images/staggered_mesh_eps}
\begin{tikzpicture}
\draw (-3,0) rectangle +(1,5);
\draw (0,0) rectangle +(1,1) (0,1) rectangle +(1,1) (0, 2) rectangle +(1,1) (0,3) rectangle +(1,1) (0,4) rectangle +(1,1);
\draw[dashed] (3,-0.5) rectangle +(1,1) (3,0.5) rectangle +(1,1) (3,1.5) rectangle +(1,1) (3, 2.5) rectangle +(1,1) (3,3.5) rectangle +(1,1) (3,3.5) rectangle +(1,1) (3, 4.5) rectangle +(1,1) ;
\draw[dashed] (-3,0) -- (4,0);
\draw[dashed] (-3,5) -- (4,5);
\draw (-2.5,-1) node {Channel};
\draw (0.5,-1) node {Continuity};
\draw (0.5,-1.5) node {Volume};
\draw (3.5,-1) node {Momentum};
\draw (3.5,-1.5) node {Flow Path};
\end{tikzpicture}
\caption{A representation of the staggered mesh.}
\label{fig:staggered_mesh}
\end{figure}

The geometric modeling framework in this work involves two primary components: sections and subchannels.
The total axial height of the problem is divided into sections.
Each section is defined by two bounding elevations and a spatial discretization of that span into discrete non-overlapping axial segments that represent the axial spacing of the continuity volumes.
The total number of continuity volumes in a given subchannel is one less than the number of momentum flow paths in that subchannel.
The particular axial discretization of a given section is independent from that of the other sections.

Within a given section, subchannels are defined.
A subchannel inherits the axial discretization of its parent section.
The number of subchannels in a given section is independent of the number of subchannels in other the sections.
Each continuity volume and momentum flow path has an associated cross-sectional area, $A_{c}$ and $A_{p}$, respectively. 
These cross-sectional areas can vary from volume to volume and from flow path to flow path.

It is also possible to model multi-dimensional flow through the definition of transverse flow paths that connect two subchannels within a given section.
The transverse flow paths are orthogonal to the axial flow direction.
Additionally, it is permissible to have the axial flow from a single subchannel split into multiple subchannels in an adjacent section.
This flow splitting is only permitted at section boundaries.
By using these features geometrically complex models can be created.
An example of a model that includes these different geometric characteristics is given in \fig{fig:complex_geometry}.

\begin{figure}[ht!]
\centering
\tikzsetnextfilename{images/complex_geometry_pdf}
\begin{tikzpicture}
%\draw [thick] (-2,-2) rectangle (-1,2);
%\draw [thick] (1,-2) rectangle (2,2);
%\draw [thick] (-0.5,3) rectangle (0.5,7);
%\draw [thick] (-0.5,-7) rectangle (0.5,-3);

%Section 1
\draw [thick] (-0.25,-6) rectangle (0.25,-5.25);
\draw [thick] (-0.4,-5.25) rectangle (0.4,-4.5);
\draw [thick] (-0.5,-4.5) rectangle (0.5,-3.75);
\draw [thick] (-0.3,-3.75) rectangle (0.3,-3);

%Section 2: left then right
\draw [thick] (-1.75,-2) rectangle (-1.25,-1);
\draw [thick] (-1.6,-1) rectangle (-1.4,0);
\draw [thick] (-2,-0) rectangle (-1,1);
\draw [thick] (-1.75,1) rectangle (-1.25,2);

\draw [thick] (1.1,-2) rectangle (1.9,-1);
\draw [thick] (1.1,-1) rectangle (1.9,0);
\draw [thick] (1.25,0) rectangle (1.75,1);
\draw [thick] (1.3,1) rectangle (1.7,2);

%Section 3
\draw [thick] (-0.25,3) rectangle (0.25,3.8);
\draw [thick] (-0.5,3.8) rectangle (0.5,4.6);
\draw [thick] (-0.4,4.6) rectangle (0.4,5.4);
\draw [thick] (-0.5,5.4) rectangle (0.5,6.2);
\draw [thick] (-0.25,6.2) rectangle (0.25,7);
\draw [thick] (-0.25,7) rectangle (0.25,7.8);

%Flow lines
\draw [dashed] (-1.5,2.5) -- (1.5,2.5);
\draw [dashed](-1.5,-2.5) -- (1.5,-2.5);
\draw [dashed,<-] (0,3) -- (0,2.5);
\draw [dashed,->] (-1.5,2.5) -- (-1.5,2);
\draw [dashed,->] (1.5,2.5) -- (1.5,2);
\draw [dashed,->] (0,-2.5) -- (0,-3);
\draw [dashed,<-] (-1.5,-2) -- (-1.5,-2.5);
\draw [dashed,<-] (1.5,-2) -- (1.5,-2.5);

\draw [dashed, <->] (-1.25,-1.5) -- (1.1,-1.5);
\draw [dashed, <->] (-1.4,-0.5) -- (1.1,-0.5);
\draw [dashed, <->] (-1,0.5) -- (1.25,0.5);
\draw [dashed, <->] (-1.25,1.5) -- (1.3,1.5);	
\foreach \y/\ytext in {-4.5/ 1,0/ 2,5.5/ 3}
	\draw (2,\y) node [anchor=west] {Section $\ytext$};
\end{tikzpicture}
\caption{A model with transverse flow paths and complex geometric characteristics.}
\label{fig:complex_geometry}
\end{figure}

%-------------------------------------------------------------------------------
%-------------------------------------------------------------------------------
%-------------------------------------------------------------------------------
\section{Two-Phase Flow Equations}
\label{sect:two_phase_flow}
The primary purpose of subchannel analysis codes is to determine fuel integrity via evaluation of effective core cooling during postulated NPP accidents, such as a LOCA.
This requires accurate modeling the heat transfer between the coolant and the fuel. 
During postulated accidents, the coolant, H$_2$O, can undergo phase-change.
There are several formulations of the governing conservation equations of fluid mechanics used to predict the thermal-hydraulic response of the nuclear reactor core to transient plant conditions.
To model the complex phenomenon of phase-change, the governing equations for the fluid mechanics within the core will be those of a multicomponent fluid \cite{Drew1998}.
In particular, they are a subcategory of two-phase flow \cite{Todreas2011, Stewart1984, Ishii1984}.
This section will discuss both the assumptions used and the final governing equations obtained for modeling two-phase flow.

%-------------------------------------------------------------------------------
%-------------------------------------------------------------------------------
%-------------------------------------------------------------------------------
\subsection{Assumptions}
\label{subsect:assumptions}

The governing equations of fluid-mechanics provide extensive information about fluid behavior.
However, one of the basic assumptions of subchannel analysis is that the exact spatial behavior of the coolant is not of direct interest; as such, the average behavior of the fluid is what is modeled.
Given the multi-phase nature of the coolant, there are various ways in which the interactions between the various phases can be modeled.
For the conditions encountered in a commercial nuclear reactor core, certain assumptions can be made to reduce the complexity of the physics involved.
This section will discuss: the technique used to obtain equations for averaged quantities, the manner in which the multi-phase nature of the coolant is dealt with, and the general assumptions used reduce the complexity of the physics.

The first simplification is that the exact topology of the interface between the coolant phases need not be known to model the fluid mechanics.
Since the exact deterministic behavior of the phasic-interface is no longer necessary, the governing equations can be subjected to averaging procedures to produce conservation laws for averaged quantities.
There are several averaging techniques that have been used to produce the conservation laws for two-phase flow: spatial, temporal, and ensemble averaging \cite{Drew1998, Todreas2011}.
Each of these techniques has its own physical interpretation and mathematical formulation.
The formulation used in this work is area averaging, a particular form of spatial averaging.
In the area-averaged formulation the governing equations are averaged over the cross-sectional area.

The second simplification regards the manner in which the multi-phase nature of the coolant is treated.
The particular formulation of the two-phase flow equations used in this work is commonly referred to as a ``two-phase, three-field formulation."
While this designation comes from the three water fields that are modeled by the software, the software also models a \ncg{} field that is mixed with the steam.
To simulate the behavior of in-core fluid dynamics during accident scenarios more accurately, the liquid and the gaseous phases are each divided into two distinct fields.
The two fields of the liquid phase are a continuous liquid field and an entrained liquid droplet field.
The gaseous phase is composed of a mixture of a \ncg{} field and a water-vapor field. 
This ability to track the different fields within a phase provides two important benefits to safety analysis: the ability to account for the effects of \ncgs{} on condensation and the ability to model the effects of the entrained liquid droplets on heat transfer.
The benefits of modeling the entrained liquid field have inspired the developers of the French CATHARE software to migrate to a similar three-field formulation in the next version of their software \cite{Emonot2011}.

Given the four fields of interest, there are twelve conservation equations in axial flow, one each for the mass, momentum, and energy of each of the four fields.
Additional closure relationships are necessary to describe the interactions between the various fields and their interfacial transfer terms.
These equations allow for a description of the time-dependent behavior of the in-core fluid.
However, assumptions are made to reduce the number of required conservation laws and the number of corresponding closure relationships.
The following is a list of the assumptions that are made to reduce the complexity of the governing equations.

\begin{itemize}
\item{
Thermodynamic and pressure equilibrium exists between the continuous liquid field and the entrained liquid field.
The basis for this assumption is that, while the entrained droplets are being modeled by a separate set of governing equations than those of the continuous liquid field, the droplets are constantly entraining from and depositing to the continuous liquid field. 
}
\item{
The liquid and gaseous phases are assumed to be in pressure equilibrium with the interface between the phases.
The basis for this assumption is that the inter-phase dynamics at the interface are negligible when compared to the bulk flow dynamics.
}
\item{The \ncg{} and steam components of the gaseous phase are in mechanical equilibrium.}
\item{The two components of the gaseous phase obey Dalton's Law.}
\item{The two components of the gaseous phase are in thermal equilibrium.}
\item{
The viscous dissipation of momentum in the axial flow direction and the associated generation of energy are neglected.
This assumption is made because the simulations of interest are dominated by inertial, not viscous, forces so that the neglected terms would be small compared to other terms in the calculations.
}
\item{
The wall-shear effects of viscosity are accounted for via empirically based friction correlations.
}
\item{
The primary positive flow direction is counter to the gravity vector, which will be referred to as the axial flow direction. This vertically oriented coordinate system was chosen to accommodate the vertical core design of NPPs in the United States.}
\item{
The mechanical energy of the phases is neglected in the conservation of energy. 
The assumption is that in the simulations of interest the mechanical energy term is negligible.
}
\item{
The effects of turbulence are only incorporated through the closure relationships used for wall friction and wall heat transfer.
}
\end{itemize}

%-------------------------------------------------------------------------------
%-------------------------------------------------------------------------------
%-------------------------------------------------------------------------------
\subsection{Governing Equations}
\label{subsect:governing_equations}

Following the application of the above assumptions, nine governing partial differential equations (PDEs) for axial flow remain: four for mass conservation, two for energy conservation, and three for momentum conservation.
The details of this system of PDEs are discussed below.

%-------------------------------------------------------------------------------
%-------------------------------------------------------------------------------
%-------------------------------------------------------------------------------
\subsubsection{Conservation of Mass Equations}
\label{subsubsect:mass_equations}

There are four equations that represent the conservation of mass, one each for the \ncg{} \eqref{eqn:conservation_of_ncg}, continuous liquid \eqref{eqn:conservation_of_liq}, entrained liquid  \eqref{eqn:conservation_of_ent}, and vapor \eqref{eqn:conservation_of_vap} fields.

\begin{IEEEeqnarray}{rCl}
\label{eqn:conservation_of_ncg}
\frac{\partial \left(\alpha_g \rho_{n}\right) }{\partial t } + \nabla \cdot \left( \alpha_g \rho_{n} \vec{u}_g \right) & = & \dot{s}^{'''}_{m,n} \\
\label{eqn:conservation_of_liq}
\frac{\partial \left(\alpha_l \rho_l \right)}{\partial t } + \nabla \cdot \left( \alpha_l \rho_l \vec{u}_l \right)         & = & -(1-\eta)\dot{\Gamma}^{'''} - \dot{\Upsilon}^{'''} + \dot{s}^{'''}_{m,l} \\
\label{eqn:conservation_of_ent}
\frac{\partial \left(\alpha_e \rho_l \right)}{\partial t } + \nabla \cdot \left( \alpha_e \rho_l \vec{u}_e \right)         & = & -\eta\dot{\Gamma}^{'''} + \dot{\Upsilon}^{'''}+ \dot{s}^{'''}_{m,e} \\
\label{eqn:conservation_of_vap}
\frac{\partial \left(\alpha_g \rho_v \right)}{\partial t } + \nabla \cdot \left( \alpha_g \rho_v \vec{u}_g \right)         & = & \dot{\Gamma}^{'''} + \dot{s}^{'''}_{m,v} 
\end{IEEEeqnarray}

The left-hand sides of \eqref{eqn:conservation_of_ncg} -- \eqref{eqn:conservation_of_ent} represent the Lagrangian derivative for the given field.
The terms on the right-hand side represent the volumetric rate of the inter-field ($\dot{\Upsilon}^{'''}$), inter-phase ($\dot{\Gamma}^{'''}$), and external ($\dot{s}^{'''}_{m,\phi}$) mass transfer, where $\phi$ represents the relevant phase or field.
Since there are two liquid fields, the rate of mass transfer between the water-vapor field and the liquid fields, $\dot{\Gamma}^{'''}$, is apportioned between the continuous liquid field and the entrained liquid field.
The relationship between the phasic mass transfer terms is given by \eqref{eqn:apportionment_of_mass_transfer}, where $\eta$ is an apportionment factor. 

\begin{equation}
\label{eqn:apportionment_of_mass_transfer}
\dot{\Gamma}^{'''} = \eta \dot{\Gamma}^{'''} + (1 - \eta)\dot{\Gamma}^{'''}
\end{equation}

The inter-field transfer of mass occurs only between the continuous and entrained liquid fields, \eqref{eqn:entrainment_deentrainment}.

\begin{equation}
\label{eqn:entrainment_deentrainment}
\dot{\Upsilon}^{'''}_l + \dot{\Upsilon}^{'''}_e = 0
\end{equation}

Within the conservation of mass equations several assumptions from \sect{subsect:assumptions} are evident.
The mechanical equilibrium of the \ncg{} and the vapor field manifests itself in the singular velocity for the two gaseous fields: $\vec{u}_g$, where the $g$ subscript denotes the total gaseous phase.
Dalton's Law allows the two components of the gaseous phase to occupy the same volume, thus providing for a singular volume fraction, $\alpha_g$.
The thermodynamic equilibrium of the two liquid fields results in only one liquid density, $\rho_l$.

%-------------------------------------------------------------------------------
%-------------------------------------------------------------------------------
%-------------------------------------------------------------------------------
\subsubsection{Conservation of Energy Equations}
\label{subsubsect:energy_equations}

In addition to the conservation of mass equations there are conservation of energy equations for each of the two phases, \eqref{eqn:con_energy_gas} -- \eqref{eqn:con_energy_liq}.

\begin{IEEEeqnarray}{rCl}
\label{eqn:con_energy_gas}
\frac{\partial \left( \alpha_g \rho_g h_g \right)}{\partial t } + \nabla \cdot \left(  \alpha_g \rho_g h_g \vec{u}_g \right) & =& \nonumber \\
\dot{\Gamma}^{'''} h^{'}_v + \dot{q}^{'''}_{i,v} + \dot{q}^{'''}_{n,l}  + \dot{q}^{'''}_{w,g} + \alpha_g\frac{\partial P}{\partial t} + \dot{s}^{'''}_{h,g}  & &\\
\label{eqn:con_energy_liq}
\frac{\partial \left( (1 - \alpha_g) \rho_l h_l \right) }{\partial t } + \nabla \cdot \left( \alpha_l \rho_l h_l \vec{u}_l \right) + \nabla \cdot \left( \alpha_e \rho_l h_l \vec{u}_e \right)& = & \nonumber \\
-\dot{\Gamma}^{'''} h^{'}_l + \dot{q}^{'''}_{i,l} - \dot{q}^{'''}_{n,l}  + \dot{q}^{'''}_{w,l} + (1 - \alpha_g) \frac{\partial P}{\partial t} + \dot{s}^{'''}_{h,l}  & &
\end{IEEEeqnarray}

The conservation of energy equations used in this work are formulated such that the conserved quantities are the phasic enthalpies, $\alpha_{\phi} \rho_{\phi} h_{\phi}$.
Under the assumption of thermodynamic equilibrium for the two liquid fields, there is a single enthalpy for the two liquid fields.
The gaseous phasic enthalpy, however, is defined according to \eqref{eqn:gaseous_enthalpy}.

\begin{equation}
\label{eqn:gaseous_enthalpy}
\rho_g h_g = \rho_v h_v + \rho_n h_n
\end{equation}

The various terms on the right hand sides of \eqref{eqn:con_energy_gas} and \eqref{eqn:con_energy_liq} are defined as follows:

\begin{itemize}
\item{
$\dot{\Gamma}^{'''} h^{'}_{\phi}$:
 energy transfer rate due to the phase change of water.
 The effective enthalpies, $h^{'}_{\phi}$, are dependent upon the mechanism of phase change.
}
\item{
$\dot{q}^{'''}_{i,\phi}$:
energy transfer rate between the liquid and vapor phases and the saturated interface.
}
\item{
$\dot{q}^{'''}_{n,l}$:
energy transfer rate between the liquid phase and the \ncgs{}.
}
\item{
$\dot{q}^{'''}_{w,\phi}$:
 energy transfer rate between the solid-structures and a given phase.
}
\item{
$\alpha_{\phi} \frac{\partial P}{\partial t}$:
 pressure work done by a given phase $\phi$.
 The liquid volume fraction is the sum of the volume fractions of the continuous and the entrained liquid fields.
}
\item{
$\dot{s}^{'''}_{h,\phi}$:
 energy transfer rate between an external source and a given phase.
}
\end{itemize}

%-------------------------------------------------------------------------------
%-------------------------------------------------------------------------------
%-------------------------------------------------------------------------------
\subsubsection{Conservation of Momentum Equations}
\label{subsubsect:momentum_equations}

Finally, there are three governing equations for the conservation of momentum: the continuous liquid field \eqref{eqn:con_mom_liq}, the gaseous phase \eqref{eqn:con_mom_gas}, and the entrained liquid droplet field \eqref{eqn:con_mom_ent}.
These equations are expressed in vector notation; however, only axial flow will be detailed below.
The assumed mechanical equilibrium between the two gaseous fields enables the use of a single momentum conservation equation for the net gaseous phase.

\begin{IEEEeqnarray}{rCl}
\label{eqn:con_mom_liq}
\frac{\partial \left( \alpha_l \rho_l \vec{u}_l \right )}{\partial t } + \nabla \cdot \left( \alpha_l \rho_l \vec{u}_l \vec{u}_l \right) & = & \nonumber \\
 -\alpha_l \nabla P + \alpha_l \rho_l \vec{g} - \vec{\tau}^{'}_{w,l} + \vec{\tau}^{'}_{i,gl} - (1 - \eta)\dot{\Gamma}^{'''}\vec{u}^{'} - \dot{\Upsilon}^{'''}\vec{u}^{'} + \dot{s}^{'''}_{p,l} & & \\
\label{eqn:con_mom_gas}
\frac{\partial \left( \alpha_g \rho_g \vec{u}_g \right) }{\partial t } + \nabla \cdot \left( \alpha_g \rho_g \vec{u}_g \vec{u}_g \right) & = & \nonumber \\
 -\alpha_g \nabla P + \alpha_g \rho_g \vec{g} - \vec{\tau}^{'}_{w,g} - \vec{\tau}^{'}_{i,gl} - \vec{\tau}^{'}_{i,ge} + \dot{\Gamma}^{'''}\vec{u}^{'} + \dot{s}^{'''}_{p,g} & & \\
\label{eqn:con_mom_ent}
\frac{\partial \left( \alpha_e \rho_l \vec{u}_e \right) }{\partial t } + \nabla \cdot \left( \alpha_e \rho_l \vec{u}_e \vec{u}_e \right) & = & \nonumber \\
 -\alpha_e \nabla P + \alpha_e \rho_l \vec{g} - \vec{\tau}^{'}_{w,e} + \vec{\tau}^{'}_{i,ge} - \eta \dot{\Gamma}^{'''}\vec{u}^{'} + \dot{\Upsilon}^{'''}\vec{u}^{'} + \dot{s}^{'''}_{p,l} & &
\end{IEEEeqnarray}

The $\rho_g$ used in \eqref{eqn:con_mom_gas} is defined by \eqref{eqn:gaseous_density}.

\begin{equation}
\label{eqn:gaseous_density}
\rho_g = \rho_n + \rho_v
\end{equation}

The material derivative of the fluid momentum is represented by the left-hand sides of \eqref{eqn:con_mom_liq} -- \eqref{eqn:con_mom_ent}.
The right-hand sides represent various surface, boundary, and body forces that act upon the fluid.
The various terms in \eqref{eqn:con_mom_liq} -- \eqref{eqn:con_mom_ent} are described below.

\begin{itemize}
\item{
$\alpha_{\phi} \nabla P$:
pressure gradient acting on field or phase $\phi$.
}
\item{
$\alpha_{\phi} \rho_{\phi} \vec{g}$:
gravity body force acting upon field or phase $\phi$.
}
\item{
$\vec{\tau}^{'}_{w,\phi}$:
 shear forces from contact between field or phase $\phi$ and the subchannel walls. 
}
\item{
$\vec{\tau}^{'}_{i,\phi_1\,\phi_2}$:
 shear forces from the interface between one field or phase $\phi_1$ and another $\phi_2$. 
}
\item{
$\dot{\Gamma}^{'''}\vec{u}^{'}$:
 momentum contribution from the exchange of mass due to the phase change of water.
 The $\vec{u}^{'}$ term depends upon the net inter-phase transfer of mass.
}
\item{
$\dot{\Upsilon}^{'''}\vec{u}^{'}$:
 momentum contribution from the exchange of mass between the two liquid fields.
 The $\vec{u}^{'}$ term depends upon the net inter-field transfer of mass.
}
\item{
$\dot{s}^{'''}_{p,\phi}$:
 momentum transfer rate between an external source and a given field $\phi$.
}
\end{itemize}

Note that the momentum conservation equations are formulated such that the temporal derivatives are of the conserved quantities, which is referred to as a conservative formulation.
This formulation will be retained during the numerical discretization process.
Other common system analysis codes \cite{TRACE, RELAP} use different, non-conservative variants of the above equations for their momentum conservation laws.
Both formulations are known to have computational strengths and weaknesses \cite{Park2009a}.

%-------------------------------------------------------------------------------
%-------------------------------------------------------------------------------
%-------------------------------------------------------------------------------
\subsection{Summary}
\label{subsect:summary}

It is useful to refer to the collection of continuous conservation laws, \eqref{eqn:conservation_of_ncg} -- \eqref{eqn:conservation_of_ent}, \eqref{eqn:con_energy_gas} -- \eqref{eqn:con_energy_liq}, and \eqref{eqn:con_mom_liq} -- \eqref{eqn:con_mom_ent}, as a vector of equations.
To accomplish this, \eqref{eqn:conservation_equations} defines such a system.
The vector, $\vec{e}$, represents the conservation equations minus the temporal derivatives of the conserved quantities.

\begin{equation}
\label{eqn:conservation_equations}
\frac{\partial \vec{y} }{\partial t} = \vec{e}(\vec{y}(t))
\end{equation}

The vector, $\vec{y}$, of conserved quantities is defined in \eqref{eqn:conserved_variables}.

\begin{equation}
\label{eqn:conserved_variables}
\vec{y} = [\alpha_g \rho_n,\, \alpha_g \rho_v,\, \alpha_l \rho_l, \alpha_e \rho_l, \alpha_g \rho_g h_g, \alpha_l \rho_l h_l,\, \alpha_g \rho_g \vec{u}_g,\, \alpha_l \rho_l \vec{u}_l,\, \alpha_e \rho_l \vec{u}_e]^{T}
\end{equation}

%-------------------------------------------------------------------------------
%-------------------------------------------------------------------------------
%-------------------------------------------------------------------------------
\section{Numerical Approximations}
\label{sect:numeric_approximation}
The set of conservation laws, \eqref{eqn:conservation_equations}, governing the fluid mechanics within the geometry of interest does not have a general closed form solution.
As such, methods need to be selected to approximate the spatial, temporal, and nonlinear behavior of its numerical solution.

\subsection{Spatial Approximations}
\label{subsect:spatial_approx}
In the thermal-hydraulic safety analysis methods of interest the governing equation are discretized using the finite-volume method \cite{LeVeque2002}.
The discussion that follows is specific to \cobra{}, but other safety analysis software use similar procedures \cite{RELAP,TRACE}.

The labeling scheme for the finite volumes is shown in \fig{fig:vertical_pipe_with_cells}, which is a segment of a subchannel.
Two momentum flow paths, $j + \onehalf $ and $j - \onehalf$, spatially overlap a continuity volume, $j$.
Variables are indexed by the mesh on which their conservation equations are defined.
For example, the velocity $u_{j - \onehalf}$ would be spatially located at center of the momentum flow path $j - \onehalf $ and at the boundary between the continuity volumes $j$ and $j-1$.

\begin{figure}[ht!]
\centering
\begin{tikzpicture}
\draw (-2,-3) rectangle +(2,2);
\node[anchor=center] at (-1,-2) {j-1};
\draw (-2,-1) rectangle +(2,2);
\node[anchor=center] at (-1, 0) {j};
\draw (-2, 1) rectangle +(2,2);
\node[anchor=center] at (-1, 2) {j+1};
\draw[dashed] (-0,-2) rectangle +(2,2);
\node[anchor=center] at (1, -1) {j};
\draw[dashed] (-0, 0) rectangle +(2,2);
\node[anchor=center] at (1, 1) {j+1};
\node[anchor=north] at (-1, -3) {Continuity};
\node[anchor=north] at (1, -2) {Momentum};
\end{tikzpicture}
\caption{Illustration of indexing scheme.}
\label{fig:vertical_pipe_with_cells}
\end{figure}

Recalling the staggered mesh from \sect{sect:geometry}, the six scalar conservation laws, \eqref{eqn:conservation_of_ncg} -- \eqref{eqn:con_energy_liq}, are each integrated over the continuity volumes.
The assumption in these integrals is that the value of the conserved quantities and all thermodynamically-related variables are a constant, average value within a given continuity volume.
\fig{fig:constant_value} shows a graphical representation of this idea for a generic function $f(x)$ over several spatial continuity volumes. 

\begin{figure}[ht!]
\centering
\tikzsetnextfilename{images/constant_value_pdf}
\begin{tikzpicture}
\draw [->, thick] (-5,0) -- (6,0);
\draw [->, thick] (-5,0) -- (-5,4);
\draw (-4,1) -- (-1,1);
\draw (-1,2) -- (2,2);
\draw (2,3) -- (5,3);
\draw (-5.5,3) node {$f(x)$};
\draw [dashed] (-4,1) -- (-4,0);
\draw [dashed] (-1,2) -- (-1,0);
\draw [dashed] (2,3) -- (2,0);
\draw [dashed] (5,3) -- (5,0);
\foreach \x / \xtext in {-2.5/x_{j-1},-1/x_{j-\onehalf},0.5/x_j,2/x_{j+\onehalf},3.5/x_{j+1}}
	\draw [thick] (\x,-2pt) -- (\x,0pt) node [anchor=north] {$\xtext$};
\end{tikzpicture}
\caption{Constant variable values within computational volumes.}
\label{fig:constant_value}
\end{figure}

Within a subchannel, a given continuity volume of length $\Delta x $ has a constant cross-sectional area, $A_{c}$, over its length.
The cross-sectional area of the boundary between two continuity volumes is given by the cross-sectional area of the momentum path at that boundary, $A_{p}$.

\begin{figure}[ht!]
\centering
\tikzsetnextfilename{images/isoparametric_volume_pdf}
\begin{tikzpicture}
\draw [dotted] (2,0) arc (0:180:2 and 1);
\draw (-2,0) arc (180:360:2 and 1);
\draw [dashed] (0,4) circle (2 and 1);
%\filldraw [black] (0,4) circle (2pt);
\draw [pattern=dots] (0,8) circle (2 and 1);
\draw (-2,0) -- (-2,8);
\draw (2,0) -- (2,8);
\draw [<->] (2.75,0) -- (2.75,8);
\draw (2.5,0) -- (3,0);
\draw (2.5,8) -- (3,8);
\draw (3.25,4) node {$\Delta x_j$};
\filldraw [gray!10] (0,8) circle (0.5);
\draw (0,8) node {$A_{C,j}$};
\foreach \y/\ytext in {0/$x_{j-\frac{1}{2}}$,4/$x_j$,8/$x_{j+\frac{1}{2}}$}
	\draw (-2.25,\y) node [anchor=east] {\ytext};
\end{tikzpicture}
\caption{Single continuity volume over which the conservation equations are integrated.}
\label{fig:single_volume}
\end{figure}

For illustrative purposes \eqref{eqn:conservation_of_liq} will be integrated over a simplified volume, $V_j$, consisting of singly connected axial flow as shown in \fig{fig:single_volume}.
This volume integrated equation is given by \eqref{eqn:spatially_discrete_liq_m_con}.
The same general procedure is used for the other five scalar conservation equations.

\begin{IEEEeqnarray}{lcl}
\int_{V_j}\frac{\partial \left(\alpha_l \rho_l \right)}{\partial t } & + & \nabla \cdot \left( \alpha_l \rho_l u_l \right) \mathrm{d}V = \int_{V_j} \left(-(1-\eta)\dot{\Gamma}^{'''} - \dot{\Upsilon}^{'''} + \dot{s}^{'''}_{m,l}\right) \mathrm{d}V \nonumber \\
V_j \frac{\partial \left(\alpha_{l,j} \rho_{l,j} \right)}{\partial t } & = & -\int_{V_j}\nabla \cdot \left( \alpha_l \rho_l u_l \right) \mathrm{d}V -(1-\eta_j)\dot{\Gamma}_j - \dot{\Upsilon}_j + \dot{s}_{m,l,j} \nonumber \\
V_j \frac{\partial \left(\alpha_{l,j} \rho_{l,j} \right)}{\partial t } & = & -\left[\alpha_l \rho_l u_l A_{p}\right]_{x_{j-\onehalf}}^{x_{j+\onehalf}} -(1-\eta_j)\dot{\Gamma}_j - \dot{\Upsilon}_j + \dot{s}_{m,l,j} \nonumber \\
\label{eqn:spatially_discrete_liq_m_con}
V_j \frac{\partial \left(\alpha_{l,j} \rho_{l,j} \right)}{\partial t } & = & -\left( \don{\alpha_l \rho_l}_{\text{d},j+\onehalf} u_{l,j+\onehalf} A_{p,j+\onehalf} - \don{\alpha_l \rho_l}_{\text{d},j-\onehalf} u_{l,j - \onehalf} A_{p,j - \onehalf}\right) \nonumber \\
& & -(1-\eta)\dot{\Gamma}_j - \dot{\Upsilon}_j + \dot{s}_{m,l,j}
\end{IEEEeqnarray}

For the mass-flux terms evaluated on the continuity volume edge, the advected quantity is evaluated using a 1st order upwind method \cite{Tannehill1997}.
The velocity and cross-sectional area utilized in the continuity flux terms, $u_{j \pm \onehalf}$ and $A_{p,j \pm \onehalf}$, have the values defined in the momentum flow path that aligns with the boundary of continuity volumes.
The sign of the velocity at the volume boundary determines the value of the donored quantity, $\don{a}_{\text{d},j \pm \onehalf}$.
A more generic formulation for this scheme is given by \eqref{eqn:upwind_donoring}.
The superscripts, $b$ and $c$, denote the discrete points in time at which the variables are evaluated.
The particular time superscripts in \eqref{eqn:upwind_donoring} are for illustrative purposes and are not fixed.

\begin{equation}
\label{eqn:upwind_donoring}
\don{a^{b}}^{c}_{\text{d}, j - \onehalf} = \begin{cases} a^{b}_{j-1} &  u^{c}_{j - \onehalf} \geq 0 \\ a^{b}_{j} & u^{c}_{j - \onehalf} < 0 \end{cases}
\end{equation}

The three momentum conservation equations, \eqref{eqn:con_mom_liq} -- \eqref{eqn:con_mom_ent}, are integrated over their momentum flow path.
The cross-sectional area for a momentum path, $A_{p}$, can be defined independently of the two cross-sectional areas of the adjoining continuity volumes.
The momentum flux terms, \eqref{eqn:momentum_flux_terms}, are treated similarly to the flux terms in the continuity equations.

\begin{equation}
\label{eqn:momentum_flux_terms}
-\left[\don{\alpha_l \rho_l u_l}_{\text{d}} \ave{u_{l}}_{\text{a}} \tilde{A}\right]_{x_{j }}^{x_{j + 1}}
\end{equation}

There are two primary differences.
First, the area in the flux term, $\tilde{A}$, is taken as the minimum of two areas from the adjoining momentum flow paths, \eqref{eqn:area_def}.

\begin{equation}
\label{eqn:area_def}
\tilde{A}_{j} = \min\left(A_{p,j - \onehalf}, A_{p,j+\onehalf}\right)
\end{equation}

Second, the phasic velocity that is used to determine the donored quantities is the arithmetic mean of the velocities from the two adjacent momentum flow paths, \eqref{eqn:average_advecting_vel}.

\begin{equation}
\label{eqn:average_advecting_vel}
\ave{u}_{\text{a},\phi, j} = \frac{u_{\phi, j - \onehalf} + u_{\phi, j+ \onehalf}}{2}
\end{equation}

The ordering of the governing equations within a given continuity volumes is as follows:

\begin{enumerate}
\item{Conservation of the \ncg{} field mass.}
\item{Conservation of the continuous liquid field mass.}
\item{Conservation of the gaseous phase energy.}
\item{Conservation of the liquid phase energy.}
\item{Conservation of the entrained liquid field mass.}
\item{Conservation of the vapor field mass.}
\end{enumerate}

The conservation of momentum equations within a given momentum flow path are ordered as follows:

\begin{enumerate}
\item{Conservation of the continuous liquid field momentum.}
\item{Conservation of the gaseous phase momentum.}
\item{Conservation of the entrained liquid field momentum.}
\end{enumerate}

%-------------------------------------------------------------------------------
%-------------------------------------------------------------------------------
%-------------------------------------------------------------------------------
\subsection{Temporal Approximations}
\label{subsect:temporal_approx}

Once the governing conservation equations have been spatially discretized utilizing the method outlined above, the temporal derivatives need to be approximated numerically.
The continuous conservation equations, \eqref{eqn:conservation_equations}, are now a spatially-discrete, temporally-continuous set of equations given by \eqref{eqn:temporal_semi_discrete}, where $\vec{E}$ now represents the spatially discrete approximation of $\vec{e}$.

\begin{equation}
\label{eqn:temporal_semi_discrete}
\frac{\partial \,\vec{y} }{\partial t} = \vec{E}(\vec{y}(t))
\end{equation}

Given that there are nine conservation equations, nine independent parameters need to be chosen.
The choice of the nine variables that will be solved for is another distinguishing characteristic of safety analysis software.
This work uses the following set of nine independent variables \eqref{eqn:independentVariables}.

\begin{equation}
\label{eqn:independentVariables}
\vec{x} = [\alpha_{g}P_{n}, \alpha_g, \alpha_g h_v, (1 - \alpha_g) h_l, \alpha_e, P, \dot{m}_g, \dot{m}_l, \dot{m}_e]^{T}
\end{equation}

The definition of the conserved momentum quantity for a given phase, $\dot{m}_{\phi}$, is given by \eqref{eqn:mom_dot}.

\begin{equation}
\label{eqn:mom_dot}
\dot{m}_{\phi} = \ave{\alpha_{\phi} \rho_{\phi}}_{\text{a}} u_{\phi} A_{p}
\end{equation}

The averaging operator $\ave{a}_{\text{a}}$ provides the average of a quantity from the two adjoining continuity volumes, \eqref{eqn:average_val}.

\begin{equation}
\label{eqn:average_val}
\ave{a}_{\text{a},j + \onehalf} = \frac{a_{j} + a_{j+1}}{2}
\end{equation}

The use of the momenta and the product of phasic volume fractions and phasic enthalpies, such as $\alpha_g h_v$, as independent parameters represents a unique choice.
These independent parameters are not the only option for safety analysis software.
Other software, such as TRACE and \relap53d{}, use variants of these parameters.
These variations include the use of velocities instead of momenta, and temperatures or internal energies instead of enthalpies \cite{RELAP, TRACE}.

A distinguishing feature of the numerical method in \cobra{} is its treatment of the temporal derivative for the conservation of momentum.
In the work that follows the temporal derivative of the conserved momentum equations is directly discretized, which is known as the conservative form.
The other option, the non-conservative form, expands the temporal derivative analytically via the chain rule, \eqref{eqn:non_conservative}, and these equations are then temporally discretized.

\begin{equation}
\label{eqn:non_conservative}
\frac{\partial \alpha_{\phi} \rho_{\phi} u_{\phi}}{\partial t} = \alpha_{\phi} \rho_{\phi} \frac{\partial u_{\phi}}{\partial t} + u_{\phi} \frac{\partial \alpha_{\phi} \rho_{\phi}}{\partial t}
\end{equation}

In this work the temporal derivative is approximated by a one-step difference scheme where the continuous time variables are now evaluated at discrete points, $t^0, t^1, \ldots, t^{N_{t}}$.
The notations $t^0$ and $t^{N_{t}}$ represent the initial and final time, respectively.
The term ``one-step" refers to the fact that the temporal derivative involves only two consecutive points in time.
The integral over a time interval, $\dt{} = t^{n+1} - t^{n}$, is show in \eqref{eqn:simple_partial_t}.

\begin{IEEEeqnarray}{rcl}
\int^{t^{n+1}}_{t^n}\frac{\partial \vec{y}(\vec{x})}{\partial t}\mathrm{d}\tau & = & \int^{t^{n+1}}_{t^n}\vec{E}(\vec{y}(\vec{x}))\mathrm{d}\tau \nonumber \\
\vec{y}(\vec{x}^{n+1}) - \vec{y}(\vec{x}^{n}) & = & \int_{t^{n+1}}^{t^n}\vec{E}(\vec{y}(\vec{x}))\mathrm{d}\tau \nonumber  \\
\vec{y}(\vec{x}^{n+1}) - \vec{y}(\vec{x}^{n}) & = & \Delta t \vec{E}(\vec{y}(\vec{x}^{*})) \nonumber  \\
\label{eqn:simple_partial_t}
\frac{\vec{y}(\vec{x}^{n+1}) - \vec{y}(\vec{x}^{n})}{\Delta t} & = & \vec{E}(\vec{y}(\vec{x}^{*}))
\end{IEEEeqnarray}

The choice of how to approximate the temporal integral of the sources and sinks of the system, $\vec{E}(\vec{y}(\vec{x}^{*}))$, hereafter referred to as $\vec{E}^{*}$, is a factor that defines the eventual solution algorithm.
There are two subcategories for solving this one-step temporal-integration problem, single-stage and multi-stage \cite{Stewart1981,LeVeque2007}.
\alg{alg:single_stage_temporal} shows how a multi-stage temporal integration scheme would work.

\begin{algo}[ht!]
\setlength{\baselineskip}{0.625\baselineskip}
\begin{algorithmic}[1]
\Require $\vec{y}^{0}$ and $t^{0}$
\Set $n = 0$
\Loop \; Transient Loop
    \State $t^{n+1} : = t^{n} + \Delta t$
    \For{$s = 1 \to N_{s}$} \; Stage Loop
		\BlackBox Solve $\displaystyle \frac{\vec{y}^{s} - \vec{y}^{n}}{\Delta t} =  \vec{E}^{*,s}$ for $\vec{y}^{s}$.
	\EndFor
	\State $n = n + 1$
\EndLoop
\end{algorithmic}
\caption{Multi-stage temporal integration scheme.}
\label{alg:single_stage_temporal}
\end{algo}

The final-stage conserved variables will be the new-time variables, $\vec{y}^{N_{s}} = \vec{y}^{n+1}$. 
At each stage the choice of how to approximate the driving function, $\vec{E}^{*,s}$, can change in both its functional dependence upon the stage values of $\vec{y}^{s}$ and in which components of $\vec{E}^{*}$ are included.
By excluding certain portions of $\vec{E}^{*}$ at certain stages, a time-splitting algorithm is developed.
Time-splitting algorithms are also known as operator-splitting algorithms. 
By changing the functional dependencies of terms within $\vec{E}^{*}$ at different stages, predictor-corrector methods, also called stabilizing correction methods, are generated. 
A single-stage method is the degenerate multi-stage case of $N_{s} = 1$.

For this work the conserved quantities within a continuity volume (e.g., $\alpha_g \rho_g$, $\left(1-\alpha_g\right)\rho_l h_l$) are nonlinear functions of the chosen independent parameters, \eqref{eqn:nonlinear_functions}, regardless of the approximation chosen for $\vec{E}^{*}$.

\begin{equation}
\label{eqn:nonlinear_functions}
\vec{y}^{n+1} = \vec{y}(\vec{x}^{n+1})
\end{equation}

This nonlinearity necessitates the use of a nonlinear solver at every timestep.
Any dependence of $\vec{E}^{*}$ on the new-time parameters creates additional nonlinearities.
The discrete formulation, \eqref{eqn:simple_partial_t}, can be expressed as a nonlinear residual that is a function of $\vec{x}^{n+1}$, \eqref{eqn:nonlinear_residuals}.

\begin{equation}
\label{eqn:nonlinear_residuals}
\vec{F}(\vec{x}^{n+1}) = \vec{y}(\vec{x}^{n+1}) - \vec{y}(\vec{x}^n) -\Delta t \vec{E}^{*}
\end{equation}

The temporal integration method used has an associated temporal accuracy that depends upon the approximation of $\vec{E}^{*}$ and the number of stages. 
The temporal accuracy is a way of quantifying the behavior of the solution as the timestep is reduced.
Order of accuracy estimates for temporal integration techniques are proportionality statements between the numerical error and a power of the timestep, $\mathcal{O}(\dt{}^{p})$ \cite{LeVeque2007}. 
However, it has been shown that if the nonlinear problem, \eqref{eqn:nonlinear_residuals}, is not solved at every timestep, the temporal accuracy of a method can be degraded \cite{Knoll2001, Mahaffy1993}.

%-------------------------------------------------------------------------------
%-------------------------------------------------------------------------------
%-------------------------------------------------------------------------------
\subsection{Nonlinear Approximations}
\label{subsect:nonlinear_approximations}

The nonlinear residual is a function of the new-time unknowns, $\vec{x}^{n+1}$.
The method used to solve \eqref{eqn:nonlinear_residuals} for $\vec{x}^{n+1}$ in this work is Newton's method \cite{Deuflhard2004, Dennis1996}.
Newton's method is an iterative procedure to obtain $\vec{x}^{n+1,k}$ such that $\vec{F}(\vec{x}^{n+1,k}) = 0$, as such two superscripts are required.
The nonlinear iterate superscript will be $k$.
Successive linearization of the nonlinear problem, \eqref{eqn:newton_taylor}, generates an iterative solution method.

\begin{equation}
\label{eqn:newton_taylor}
0 = \vec{F}(\vec{x}^{n+1,k}+\vec{\delta x}^k) \approx \vec{F}(\vec{x}^{n+1,k}) + \vec{J}(\vec{x}^{n+1,k}) \cdot \vec{\delta x}^k
\end{equation}

The algorithm is then one of finding successive updates, $\vec{\delta x}^k = \vec{x}^{n+1,k+1} - \vec{x}^{n+1,k}$, by solving \eqref{eqn:newton}.

\begin{equation}
\label{eqn:newton}
\vec{J}(\vec{x}^{n+1,k})\cdot \vec{\delta x}^k = -\vec{F}(\vec{x}^{n+1,k})
\end{equation} 

Since the system of nonlinear equations being solved represents a transient simulation, the iterations at each timestep start with an initial guess for the new-time variables that is equal to the old-time variables, $\vec{x}^{n+1,0} = \vec{x}^{n}$.
The underlying assumption of this initial value is that the independent parameters will not change greatly over a timestep, and the old-time variables will provide an initial vector that is within the radius of convergence of Newton's method.
\alg{alg:local_newton} provides an overview of a transient simulation using Newton's method for a single-stage temporal integration scheme.
The algorithm generalizes to a multi-stage temporal integration method by enclosing the Newton loop within a stage loop.

\begin{algo}[ht!]
\setlength{\baselineskip}{0.625\baselineskip}
\begin{algorithmic}[1]
\Require $\vec{x}^{0}$ and $t^{0}$
\Set $n = 0$
\Loop \; Transient Loop
    \Set $t^{n+1} = t^{n} + \dt{}$
    \Set $k = 0$
    \Set $\vec{x}^{n+1,k} = \vec{x}^{n}$
    \Loop \; Newton Loop
		\Calculate $\vec{F}(\vec{x}^{n+1,k})$ and $\vec{J}(\vec{x}^{n+1,k})$
		\Calculate $\vec{\delta x}^k = - \vec{J}^{-1}\cdot\vec{F}$
		\Calculate $\vec{x}^{n+1,k+1} = \vec{x}^{n+1, k} + \vec{\delta x}^{k}$
		\Set $k \pluseq 1$
		\BlackBox Loop Termination Criteria
	\EndLoop
	\Set $n \pluseq 1$
\EndLoop
\end{algorithmic}
\caption{Local Newton's method for single-stage temporal integration.}
\label{alg:local_newton}
\end{algo}

In \alg{alg:local_newton}, there is a black box step, the calculation of the loop termination criteria.
In multi-stage temporal integration methods the loop termination criteria may vary between stages. 
In the case where the loop is terminated after only a single iterate the resultant method is labeled a single-shot linearization.

%-------------------------------------------------------------------------------
%-------------------------------------------------------------------------------
%-------------------------------------------------------------------------------
\section{Solution Methods}
\label{sect:solution_techniques}

\sect{sect:numeric_approximation} provided a framework for characterizing the different solution methods used in thermal-hydraulic safety codes. 
Each method can be defined by the manner in which the temporal integration is carried out and the manner in which the nonlinearities are resolved.
The methods described below form the core of available techniques that have been developed for two-phase safety analysis codes. 
While each of the following methods may have many subtly varying algorithmic implementations that have appeared in safety analysis software, the algorithms detailed below are general enough to encompass these variants.

%-------------------------------------------------------------------------------
%-------------------------------------------------------------------------------
%-------------------------------------------------------------------------------
\subsection{Fully Explicit Method}
\label{subsect:numerics_explicit}
The least computationally expensive method on a per timestep basis for temporally integrating \eqref{eqn:simple_partial_t} is a single-stage, fully explicit method.
In the fully explicit method, the driving function is approximated as $\vec{E}(\vec{x}^n)$.
The nonlinear vector notation formulation for this method is given by \eqref{eqn:explicit}.

\begin{equation}
\label{eqn:explicit}
\vec{F}(\vec{x}^{n+1}) = \vec{y}(\vec{x}^{n+1}) - \vec{y}^{n} - \Delta t \vec{E}(\vec{y}(\vec{x}^{n}))
\end{equation}

Given the choice of independent parameters in this work, there are nonlinearities present in \eqref{eqn:explicit}.
The algorithmic implementation shown in \alg{algo:explicit} assumes a linearized solution technique where only a single Newton step is taken.

\begin{algo}[ht!]
\setlength{\baselineskip}{0.625\baselineskip}
\begin{algorithmic}[1]
\Require $\vec{x}^{0}$ and $t^{0}$
\Set $n = 0$
\Loop \; Take a Timestep
    \State $t^{n+1} : = t^{n} + \Delta t$
    \Calculate $\vec{F}(\vec{x}^n)$ and $\vec{J}(\vec{x}^n)$
    \Calculate $\vec{\delta x} = -\vec{J}^{-1}\vec{F}$
    \Calculate $\vec{x}^{n+1} = \vec{x}^{n} + \vec{\delta x}$ 
\EndLoop{\;$n \pluseq n+1$}
\end{algorithmic}
\caption{Single-stage, fully explicit, linearized method.}
\label{algo:explicit}
\end{algo}

While this particular method is the least computationally expensive on a per timestep basis of those discussed, it has a severe weakness.
That weakness is the Courant-Friedrichs-Lewy (CFL) limit imposed upon the timestep size, $\dt{}$.
The CFL limit is a relationship between the spatial and temporal discretization and the characteristic velocities of information propagation in the problem of interest \cite{LeVeque2007, Tannehill1997}.
For the case of \eqref{eqn:explicit}, the CFL limit is given in \eqref{eqn:cfl_explicit}.

\begin{equation}
\label{eqn:cfl_explicit}
\dt{}_i \lesssim \frac{\dx{}_i}{|u_{\phi,i}|+|c_i|}
\end{equation}

In \eqref{eqn:cfl_explicit}, $c_i$ and $u_i$ are the speed of sound and the magnitude of a given phasic velocity, respectively, at a given location within the domain.
These two velocities, for most applications of interest, can be of drastically different magnitude.
For example, in single phase gaseous flow, the ratio of the local phasic velocity to the speed of sound is typically much less than one ($\approx 0.01$) for problems of interest.
Using the local fluid conditions, $\dt{}_i$ is calculated at every point within the domain, $\Omega$.
The $\dt{}$ chosen for the $t^{n} \rightarrow t^{n+1}$ timestep is the most restrictive calculated value over the domain, \eqref{eqn:global_cfl}.

\begin{equation}
\label{eqn:global_cfl}
\dt{} = \min_{i \in \Omega} \dt{}_i
\end{equation}

Since the CFL limit is based upon the local speed of sound it is referred to as the sonic Courant limit.
While the explicit method may provide the lowest computational cost on a per-timestep basis, the number of timesteps required for a given problem may be much greater than the following methods due to this restrictive CFL limitation.
This limitation of the explicit method prompted the development of alternative methods that were capable of exceeding this sonic Courant limit.

%-------------------------------------------------------------------------------
%-------------------------------------------------------------------------------
%-------------------------------------------------------------------------------
\subsection{Fully Implicit Method}
\label{subsect:numerics_fully_implicit}
One alternative method for integrating \eqref{eqn:temporal_semi_discrete} is to use a fully implicit discretization of $\vec{E}^{*}$ \cite{Frepoli2003, Barre1990}.
The fully implicit method temporally approximates $\vec{E}^{*}$ as a function  of new-time parameters only, \eqref{eqn:implicit}.

\begin{equation}
\label{eqn:implicit}
\vec{F}(\vec{x}^{n+1}) = \vec{y}(\vec{x}^{n+1}) - \vec{y}^{n} - \Delta t \vec{E}(\vec{y}(\vec{x}^{n+1}))
\end{equation}

This method has the advantage of not being limited by a CFL number.
While this allows for greatly increased timestep sizes, the numerical scheme introduces nonphysical diffusion into the solution \cite{Mahaffy1993}.
Additionally, the solution of \eqref{eqn:implicit} is the most computationally expensive on a per timestep basis of the methods considered.
This computational expense comes from the full inter-volume coupling of the Jacobian matrix during the nonlinear iterations.
The computational implementation of a fully implicit method is presented in \alg{algo:implicit}.
The black-box step in \alg{algo:implicit} will be discussed in \sect{subsect:nlnCobraAlgo}.

\begin{algo}[ht!]
\setlength{\baselineskip}{0.625\baselineskip}
\begin{algorithmic}[1]
\Require $\vec{x}^{0}$ and $t^{0}$
\Set $n = 0$
\Loop \; Transient Loop
    \Set $t^{n+1} : = t^{n} + \Delta t$
    \Set $k = 0$
    \Set $\vec{x}^{k} = \vec{x}^{n}$
    \Loop \; Newton Loop
		\Calculate $\vec{F}(\vec{x}^{k})$ and $\vec{J}(\vec{x}^{k})$
		\Calculate $\vec{\delta x}^k = - \vec{J}^{-1}\cdot\vec{F}$
		\Calculate $\vec{x}^{k+1} = \vec{x}^{k} + \vec{\delta x}^{k}$
		\Set $k \pluseq 1$
		\BlackBox Loop Termination Criteria
	\EndLoop	
	\Set $n \pluseq 1$
\EndLoop
\end{algorithmic}
\caption{Fully implicit method.}
\label{algo:implicit}
\end{algo}

%-------------------------------------------------------------------------------
%-------------------------------------------------------------------------------
%-------------------------------------------------------------------------------
\subsection{Semi-Implicit Method}
\label{subsect:semi_implicit}

Another alternative method for integrating \eqref{eqn:temporal_semi_discrete} is the semi-implicit method.
It was developed to overcome the sonic Courant limitations of the explicit method while avoiding both the high computational cost and excessive diffusivity of the fully implicit method \cite{Liles1978}.
The two distinguishing characteristics of the semi-implicit method are the use of new-time variables in the evaluation of both the pressure gradient in the momentum conservation equations and the advecting velocities in the mass and energy equations. 
The implicit evaluation of the pressure gradient in the momentum equations is what leads to a CFL limit known as the material Courant limit.
Similar to the sonic Courant limit discussed in \sect{subsect:numerics_explicit}, the material Courant limit dictates the largest $\Delta t$ that can be achieved while maintaining a stable solution algorithm.
In this method the Courant limit is given by \eqref{eqn:si_cfl}.

\begin{equation}
\label{eqn:si_cfl}
\dt{}_i \lesssim \frac{\dx{}_i}{|u_{\phi,i}|}
\end{equation}

The characteristic velocity used in the calculation of the material Courant limit is based on the phasic velocity only.
This allows for larger, but still limited, timestep sizes than the fully explicit method.
The limit on timestep size is due to the explicit evaluation of the donored quantities in the flux terms in the conservation equations.
As stated, the flux of mass and energy include new-time velocities, $u^{n+1}_{\phi}$.
Since this work uses momenta, $\dot{m}_{\phi}$, as independent parameters, these velocities are derived quantities whose functional form is given in \eqref{eqn:si_vel}.
The $\ave{\alpha_{\phi} \rho_{\phi}}^{n}_{\text{a}}$ represents the arithmetic average of macroscopic densities from adjoining continuity volumes.
The averaging operator $\ave{a}^{n}_{\text{a}}$ is an extension of \eqref{eqn:upwind_donoring}.
There is a mismatch between the averaged macroscopic density and the momentum in \eqref{eqn:si_vel}.
This mismatch will be discussed in more detail in \sect{subsect:nlnPhaseTransition}.

\begin{equation}
\label{eqn:si_vel}
u^{n+1}_{\phi, j \pm \onehalf} = \frac{\dot{m}^{n+1}_{\phi, j \pm \onehalf}}{A_{p, j \pm \onehalf} \ave{\alpha_{\phi} \rho_{\phi}}^{n}_{\text{a}, j \pm \onehalf}} 
\end{equation}

An implementation of the semi-implicit method that takes only a single Newton step is shown in \alg{alg:si_legacy}s.

\begin{algo}[ht!]
\setlength{\baselineskip}{0.625\baselineskip}
\begin{algorithmic}[1]
\Require $\vec{x}^{0}$ and $t^{0}$
\Set $n = 0$
\Loop \; Transient Loop
    \Set $t^{n+1} : = t^{n} + \Delta t$
	\Calculate $\vec{F}(\vec{x}^{n})$ and $\vec{J}(\vec{x}^{n})$
	\Calculate $\vec{\delta x} = - \vec{J}^{-1}\cdot\vec{F}$
	\Calculate $\vec{x}^{n+1} = \vec{x}^{n} + \vec{\delta x}$
	\Set $n \pluseq 1$
\EndLoop
\end{algorithmic}
\caption{Semi-implicit method.}
\label{alg:si_legacy}
\end{algo}

%-------------------------------------------------------------------------------
%-------------------------------------------------------------------------------
%-------------------------------------------------------------------------------
\subsection{Stability-Enhancing Two-Step Method} 
\label{subsect:numerics_sets}
In order to overcome the material Courant limit placed upon simulations by the semi-implicit method, the Stability Enhancing Two-Step (SETS) method was developed \cite{Mahaffy1982}.
While not as stable as the fully implicit method, the SETS method allows for timesteps to exceed the material Courant limit.
It overcomes the material Courant limit by taking a multi-stage approach to the temporal integration.
While there are variants of the SETS method \cite{TRACE}, \alg{alg:sets} reflects the original publication.
The outline below uses velocities as independent parameters, not momenta.

There are three stages in the original SETS method.
Stage one is a linearized solution of the momentum equations where only the velocity terms in the momentum equations of $\vec{E}^{*}$ are evaluated implicitly.
This first stage only involves the solution of the momentum equations; the mass and energy equations are not solved during this step.
The lack of implicit dependence upon any continuity variables allows for coupling only between momentum flow paths.
This step results in predicted $u^{*}_{\phi}$ phasic velocities.

In stage two, the momentum and continuity equations are solved using a traditional semi-implicit scheme with two exceptions.
The first is that the momentum flux terms are evaluated using the stage-one predicted velocities as the advecting velocities instead of explicitly evaluating them.
The second is that the nonlinearities in the momentum equations, arising from interfacial and wall drag, are subject to a single-shot linearization.
The single-shot nature of the momentum equations allows for the reduction of the system into coupled continuity volumes as in the semi-implicit method.
The nonlinear continuity equations are then solved via Newton's method.
Stated another way, the Jacobians for the momentum equations are fixed at their first Newton iterate value, while the Jacobians for the mass and energy equations are updated at each iterate.

In the third stage the mass and energy equations are solved such that all continuity variables are implicitly evaluated, but the velocities are from stage two.
These velocities, $\vec{u}^{2}$, are the only results from stage two that are used in stage three. 
The resulting system of coupled continuity equations is then solved.
This final stage allows timestep sizes that exceed the material Courant limit.
For each stage the residual and Jacobian matrix will be denoted by $\vec{F}^{s}$ and $\vec{J}^{s}$, with $s = 1 \to 3$.

\begin{algo}[ht!]
\setlength{\baselineskip}{0.625\baselineskip}
\begin{algorithmic}[1]
\Require $\vec{x}^{0}$ and $t^{0}$
\Set $n = 0$
\Loop \; Transient Loop
    \Set $t^{n+1} : = t^{n} + \Delta t$
	\Calculate $\vec{F}^{1}$ and $\vec{J}^{1}$
	\Calculate $\vec{u}^{*}$ from $\vec{\delta u}^{*} = -\left[\vec{J}^{1}\right]^{-1} \vec{F}^{1}$
	\Loop \; Newton Loop
		\Calculate $\vec{F}^{2}$ and $\vec{J}^{2}$
		\Calculate $\vec{\delta x} = - \left[\vec{J}^{2}\right]^{-1} \vec{F}^{2}$
		\Calculate $\vec{x}^{n+1} = \vec{x}^{n} + \vec{\delta x}$
		\BlackBox Loop Termination Criteria
	\EndLoop
	\Calculate $\vec{x}^{n+1}$ from $\vec{F}^{3}$.
	\Set $n \pluseq 1$
\EndLoop
\end{algorithmic}
\caption{SETS method.}
\label{alg:sets}
\end{algo}

This algorithm was designed to allow existing software utilizing the semi-implicit method to be embedded within a larger framework.
The black-box loop termination criteria is a choice that varies with implementation.
The TRACE code uses an $\mathcal{L}_{\infty}$ norm of the unscaled Newton updates for the continuity variables to determine convergence.

%-------------------------------------------------------------------------------
%-------------------------------------------------------------------------------
%-------------------------------------------------------------------------------
\subsection{Nearly-Implicit Method}
\label{subsect:numerics_nearly_implicit}
Another possible method used to overcome the material Courant limit without incurring the same computational cost as the fully implicit method is the nearly-implicit method \cite{Trapp1986, RELAP}.
The nearly-implicit method is a multistage temporal integration scheme.
In the first stage the approximation  of $\vec{E}^{*}$ is one in which the mass and energy terms are implicit except for the donored values in the flux terms. 
In addition, the the chain rule is applied to the continuous temporal derivatives, resulting in mass equations similar to \eqref{eqn:ni_dt}.
The energy temporal derivatives are similarly expanded.

\begin{equation}
\label{eqn:ni_dt}
\frac{\partial \alpha_{\phi} \rho_{\phi}}{\partial t} = \alpha_{\phi} \frac{\partial \rho_{\phi}}{\partial t} + \rho_{\phi} \frac{\partial \alpha_{\phi}}{\partial t}
\end{equation}

The temporal discretization of \eqref{eqn:ni_dt} is given by \eqref{eqn:ni_dis_dt}.

\begin{equation}
\label{eqn:ni_dis_dt}
\alpha_{\phi} \frac{\partial \rho_{\phi}}{\partial t} + \rho_{\phi} \frac{\partial \alpha_{\phi}}{\partial t} = \alpha^{n}_{\phi} \frac{ \rho^{n+1}_{\phi} - \rho^{n}_{\phi}}{\dt{}} + \rho^{n}_{\phi} \frac{\alpha^{n+1}_{\phi} - \alpha^{n}_{\phi} }{\dt}
\end{equation}

The momentum equations are implicit in their momentum flux, pressure, and interfacial exchange terms.

The nearly-implicit method is a three-stage solution process.
The first stage is similar to the semi-implicit method.
However, as opposed to the semi-implicit method where the momentum variables are eliminated from the equations being solved, the nearly-implicit method uses the mass and energy equations to eliminate the pressure terms from the momentum equations.
The resulting matrix is a linear system of coupled velocities.
The stage-one velocities are taken to be the new-time velocities.
Once the stage-one velocities are obtained, the corresponding stage-one continuity variables are obtained analogously to the new-time momentum in the semi-implicit method.
Stage one is a single linearization of the semi-implicit method.
In the second stage, only the continuity equations are solved and they are in conservative form.
The interfacial exchange terms in the mass and energy equations are evaluated using the stage-one continuity variables.
The resulting system of equations is linear in the conserved quantities.
This linear system is solved for the conserved quantities.
In stage three, the nonlinear relationship between the conserved quantities and the independent parameters is then approximated via a single linearization of the equations of state.
\alg{alg:ni} shows the three-stage process.

\begin{algo}[ht!]
\setlength{\baselineskip}{0.625\baselineskip}
\begin{algorithmic}[1]
\Require $\vec{x}^{0}$ and $t^{0}$
\Set $n = 0$
\Loop \; Transient Loop
    \Set $t^{n+1} : = t^{n} + \Delta t$
	\Calculate $\vec{F}^{1}$ and $\vec{J}^{1}$
	\Calculate $\vec{x}^{*}$ and $\vec{u}^{n+1}$ from $\vec{\delta x}^{*} = -\left[\vec{J}^{1}\right]^{-1} \vec{F}^{1}$
	\Calculate $\vec{F}^{2}$
	\Calculate $\vec{x}^{**}$ from $\vec{F}^{2}$
	\Calculate $\vec{x}^{n+1}$ from $\vec{F}^{3}$.
	\Set $n \pluseq 1$
\EndLoop
\end{algorithmic}
\caption{The nearly-implicit method.}
\label{alg:ni}
\end{algo}

%-------------------------------------------------------------------------------
%-------------------------------------------------------------------------------
%-------------------------------------------------------------------------------
\section{Domain Coupling}
\label{sect:code_coupling}

\cobra{} possesses models of the physics necessary to simulate the complicated in-core flow patterns and heat transfer encountered during a postulated LOCA.
However, it lacks the detailed models for NPP components (pumps, valves, accumulators, pressurizers) that are traditionally available from system analysis codes.
When modeling full NPP transients, it is advantageous to use sub-channel software for the core and system analysis software for the rest of the NPP.
The combination of different specialized software is a common practice.
This coupling of capabilities can be accomplished in two ways: the two pieces of software can be merged or they can be coupled via data exchange.
Examples of the first method are MARS \cite{Jeong2008}, COBRA/TRAC \cite{Thurgood1983c}, and TRACE \cite{TRACE}.
Coupling software via data exchange is a more common option \cite{Makihara2003, Aumiller2002, Aumiller2001, Avramova2006, Weaver2002, Rodriguez2002}, and it will be discussed first.

When coupling via data exchange, the use of explicit coupling can lead to a sonic Courant limit imposed at the boundary of the two coupled systems \cite{Ragusa2009, Aumiller2001}.
To circumvent the sonic Courant limit, a semi-implicit method for the coupling of software via data exchange has been developed for thermal-hydraulic simulations \cite{Weaver2002, Aumiller2002}.
This coupling technique allows for the material Courant limit to be uniformly applied across the two pieces of software.
This technique is of direct interest to the proposed research.

The coupling of the different software involves the use of a third-party message passing interface.
The program used for the software coupling in the papers of interest is the Parallel Virtual Machine (PVM) \cite{Geist1994}.
PVM allows the data exchange to occur between the coupled software.
As originally formulated, the software coupling technique involves breaking the coupled problem into two pieces, the master and the slave components.
In this respect, the problem can be thought of as being a single domain broken in two: one subdomain controlled by the master process, and  one subdomain being controlled by the slave process.
The presentation of this method is in terms of the \relap53d{} conservation equations, which are different from those in \cobra{}.
The coupling method first developed presented \relap53d{} as both the master and slave software \cite{Weaver2002}.
Later work showed how this base method could be extended to allow coupling between \relap53d{} and the \cobra{} subchannel analysis software \cite{Aumiller2002}.
This reformulation was designed to allow for a consistent transition between a two-phase, two-field based solver and and a two-phase, three-field based solver. 
A short outline of the method as formulated for \relap53d{} coupling follows.

The semi-implicit software coupling is accomplished via the staggered mesh formulation.
The master domain is truncated on a continuity volume.
The fluxes of conserved continuity quantities at the boundary of the coupled volumes are retained as unknowns during the formulation of the global pressure matrix.
Upon forming the pressure matrix, the Newton update for the pressures will be expressed as a linear combination of all of the unknown mass and energy fluxes through the coupled boundaries of the domain, \eqref{eqn:si_relap} \cite{Weaver2002}.

\begin{IEEEeqnarray}{rcl}
\label{eqn:si_relap}
\delta P^{n+1}_{j} = a_j & + & 
\sum_{i = 1}^{N_{\text{CPL}}} b_{j,i}n_{g,i}^{n+1} +
\sum_{i = 1}^{N_{\text{CPL}}} c_{j,i}u_{g,i}^{n+1} +
\sum_{i = 1}^{N_{\text{CPL}}} d_{j,i}u_{f,i}^{n+1} +
\sum_{i = 1}^{N_{\text{CPL}}} e_{j,i}m_{g,i}^{n+1} +
\sum_{i = 1}^{N_{\text{CPL}}} f_{j,i}m_{f,i}^{n+1} \nonumber \\
& + & \sum_{i = 1}^{N_{\text{CPL}}} g_{j,i}w_{g,i}^{n+1} +
\sum_{i = 1}^{N_{\text{CPL}}} h_{j,i}w_{f,i}^{n+1}
\end{IEEEeqnarray}

\eqref{eqn:si_relap} represents change in pressure in every volume in the computational domain as a linear combination of the fluxes of mass ($n_{g,j}^{n+1}$, $m_{g,j}^{n+1}$, and $m_{f,j}^{n+1}$), volume ($w_{f,j}^{n+1}$ and $w_{g,j}^{n+1}$), and internal energy ($u_{g,j}^{n+1}$ and $u_{f,j}^{n+1}$) through the $N_\text{CPL}$ coupled boundaries.

From the point-of-view of the slave process, the momentum flow path between the master domain and the slave domain is a traditional momentum flow path.
Since the interface boundary between the master and slave domain is composed entirely of momentum flow paths, the only implicit unknowns from the master domain used in the slave domain are the pressures from the coupled continuity volumes.
These implicit pressures are expressed in terms of the old-time pressures and the updates to those pressures from the old-time values and the new-time values.
The pressure update for a given continuity volume at the boundary of the master domain, $\delta P_{\text{cbv}}^{n+1}$, is expressed in terms of the unknown velocities in the slave momentum flow paths adjacent to the boundary of the master continuity volumes, \eqref{eqn:pressure_coupled} \cite{Weaver2002}.

\begin{IEEEeqnarray}{rcl}
\label{eqn:pressure_coupled}
\delta P^{n+1}_{\text{cbv}} = a_{\text{cbv}} & + & 
\sum_{i = 1}^{N_{\text{CPL}}} b_{\text{cbv},i}\don{\alpha^n_g \rho^n_n}^{n}_{\text{d}} A_i v_{g,i}^{n+1} +
\sum_{i = 1}^{N_{\text{CPL}}} c_{\text{cbv},i}\don{\alpha_g^n \rho_g^n u^n_g}_{\text{d}}^{n} A_i v_{g,i}^{n+1} \nonumber \\
& + & \sum_{i = 1}^{N_{\text{CPL}}} d_{\text{cbv},i} \don{\alpha_f^n \rho_f^n u_f^n}_{\text{d}}^{n} A_i v_{f,i}^{n+1} +
\sum_{i = 1}^{N_{\text{CPL}}} e_{\text{cbv},i} \don{\alpha_g^n \rho_g^n}_{\text{d}}^{n} A_i v_{g,i}^{n+1} \nonumber \\
& + & \sum_{i = 1}^{N_{\text{CPL}}} f_{\text{cbv},i} \don{\alpha_f^n \rho_f^n}_{\text{d}}^{n} A_i v_{f,i}^{n+1} +
\sum_{i = 1}^{N_{\text{CPL}}} g_{\text{cbv},i} \don{\alpha^n_g}_{\text{d}}^{n} A_i v_{g,i}^{n+1} \nonumber \\
&+ & \sum_{i = 1}^{N_{\text{CPL}}} h_{\text{cbv},i} \don{\alpha_f^n}_{\text{d}}^{n} A_i v_{f,i}^{n+1}
\end{IEEEeqnarray}

This formulation allows for the updated pressure in the master domain to be calculated within the slave domain without inter-software communication.
The intent of this formulation was to enable the semi-implicit method to be used for software coupling.

The coupling of system analysis codes and sub-channel codes can also be viewed through the following lens.
The global problem, comprised of the balance of plant (e.g., \relap53d{}) and the in-core region (e.g., \cobra{}), can be considered as part of a global domain.
Each piece of software can then be viewed as solving a subset of the global domain.
The use of more detailed physics for the in-core thermal-hydraulic behavior is a particular application of model reduction \cite{Paraschivoiu1999}.

If the semi-implicit coupling methodology is used to couple two pieces of software that use the semi-implicit method, then the boundaries between the two domains no longer represent a disparity in the global method used.
This fact enables the use of the material Courant limit at coupling boundaries.
Consequently, this software coupling method allows each software process to use consistent boundary information in its calculations.

Additionally, if the boundary values are explicitly evaluated, then the decomposition of the global problem could be viewed as an application of an additive Schwarz domain decomposition algorithm.
In an additive Schwarz algorithm the boundary values for each subdomain are functions only of the old-time values in the other domains.
From this view point, the sonic Courant limit observed at software coupling boundaries follows naturally \cite{Aumiller2001}.

The additive Schwarz method has seen extensive use as a nonlinear preconditioner for fully implicit computational fluid dynamics calculations \cite{Cai2009, Cai2002}.
In particular, a nonlinearly convergent solution within each subdomain is obtained at each timestep.
It has been shown that the treatment of localized nonlinearities via domain decomposition allows for globalization strategies to be used in the global Newton step that would otherwise exhibit stalled convergence if the full domain was initially subjected to a globalization strategy \cite{Cai2011}.
This work is based upon the concept of nonlinear domain decomposition and nonlinear elimination \cite{Lanzkron1996, Dryja1997}.
However, these applications do not worry about consistency at the boundaries of the subdomains since the obtained subdomain solutions are only used as a preconditioned initial guess for a global Newton-Krylov-Schwarz algorithm \cite{Chan1984}.

The two paradigms of discrete software coupling and domain decomposition are complementary views of the same process.
The software coupling is the algorithmic implementation of the domain decomposition.
Using the semi-implicit software coupling as a basis for domain decomposition within a single code provides a mean of isolating subdomains that provides a consistent formulation for the nonlinear problem.       % Chapter: Background
\chapter{Nonlinear \cobra{}}
\label{chap:nonlinear}
The original \cobra{} software utilized the single step of the traditional semi-implicit method.
In order to investigate the effects of nonlinear convergence upon the determination of the temporal convergence for a given solution, \cobra{} was modified to enable an iterative global Newton's method.
Then, for effective use of the nonlinear solver, an operator-based scaling method was developed to provide a physically meaningful measure of convergence.
A metric was then developed and tested to aid in the identification of solutions that may not be temporally converged.
This chapter will discuss each of these phases of development as well several implementation specific practicalities that were identified and resolved.

%-------------------------------------------------------------------------------
%-------------------------------------------------------------------------------
%-------------------------------------------------------------------------------
\section{Linear \cobra{}}
\label{sect:linCobraAlg}

\cobra{} utilizes the semi-implicit method as outlined in \sect{subsect:semi_implicit}.
The governing equations in the linearized semi-implicit method are not viewed through the lens of a single Newton step for a system of nonlinear equations.
Instead, the view that the linearization is from $\vec{x}^{n}$ to $\vec{x}^{n+1}$, as opposed to $\vec{x}^{n+1, k}$ to $\vec{x}^{n+1, k+1}$, is standard in the literature of linearized safety analysis software.
For that reason, the following discussion will present the algorithm from the point of view of a single linearization.
An outline of the solution algorithm is given in \alg{alg:linCobraAlgorithm}.
This section will outline the different steps within a single timestep of the linear solver.

\begin{algo}[ht!]
\setlength{\baselineskip}{0.625\baselineskip}
\begin{algorithmic}[1]
\Require $\vec{x}^{0}$ and $t^{0}$
\Set $n = 0$
\Loop \; Transient Loop
    \Set $t^{n+1} : = t^{n} + \dt{}$
	\Algorithm Assemble Linear Pressure Matrix \Comment{\alg{alg:xschem}}
   	\Solve $\vec{A}_{l} \delta \vec{P}_{l} = \vec{res}_{l}$
	\Algorithm Update Linear Variables \Comment{\alg{alg:updateVariables}}
	\Set $n \pluseq 1$
\EndLoop
\end{algorithmic}
\caption{Linear \cobra{} algorithm.}
\label{alg:linCobraAlgorithm}
\end{algo}

Following the initialization of data, the first step in a purely hydrodynamic simulation in \cobra{} is a loop over all momentum equations in the domain.
The nonlinear momentum equations, neglecting spatial discretization notation and external sources, are given in \eqref{eqn:nlnLiqMomentumEquation} -- \eqref{eqn:nlnEntMomentumEquation}.

\begin{IEEEeqnarray}{rCl}
\label{eqn:nlnLiqMomentumEquation}
\dot{m}_{l}^{n+1} - \dot{m}_{l}^{n} & = & \frac{\dt{}}{\dx{}}\left(- \sum_{i\,\in\, N_{c}} \left( \don{\alpha_l \rho_l u_l}_{\text{d}} \ave{u}_{\text{a,l}} \tilde{A} \right)_{i}^{n}
 - \ave{\alpha_{l}}_{\text{a}}^{n} \nabla P^{\,n+1} + g \ave{\alpha_l \rho_l}_{\text{a}}^{n} - K^{n}_{wl}(\dot{m}_l^{n+1})^2 \right. \nonumber \\
 & + & \left. K^{n}_{i,gl}(\dot{m}^{n+1}_l - \dot{m}_g^{n+1})^2 - \left[(1 - \eta)\dot{\Gamma} u^{'} + \dot{\Upsilon} u^{'}\right]^{n}\vphantom{\sum_{N_{k}}}\right) \\
\label{eqn:nlnGasMomentumEquation}
\dot{m}_{g}^{n+1} - \dot{m}_{g}^{n} & = & \frac{\dt{}}{\dx{}}\left(- \sum_{i\,\in \,N_{c}} \left( \don{\alpha_g \rho_g u_g}_{\text{d}} \ave{u}_{\text{a},g}  \tilde{A} \right)_{i}^{n} - \ave{\alpha_{g}}_{\text{a}}^{n} \nabla P^{\,n + 1} + g \ave{\alpha_g \rho_g}_{\text{a}}^{n} - K^{n}_{wg}(\dot{m}_g^{n+1})^2 \right.\nonumber \\
& - & \left. K^{n}_{i,gl}(\dot{m}^{n+1}_l - \dot{m}_g^{n+1})^2 -K^{n}_{i,ge}(\dot{m}^{n+1}_e - \dot{m}_g^{n+1})^2 + \left[\dot{\Gamma} u^{'}\right]^{n}\vphantom{\sum_{N_{k}}}\right) \\
\label{eqn:nlnEntMomentumEquation}
\dot{m}_{e}^{n+1} - \dot{m}_{e}^{n} & = & \frac{\dt{}}{\dx{}}\left(- \sum_{i\,\in \,N_{c}} \left( \don{\alpha_e \rho_l u_e}_{\text{d}} \ave{u}_{\text{a},e}  \tilde{A} \right)_{i}^n -\ave{\alpha_{e}}_{\text{a}}^{n} \nabla P^{\,n+1} + g \ave{\alpha_e \rho_l}_{\text{a}}^{n} - K^{n}_{we}(\dot{m}_e^{n+1})^2\right. \nonumber \\
&+& \left. K^{n}_{i,ge}(\dot{m}^{n+1}_e - \dot{m}_g^{n+1})^2 - \left[ \eta \dot{\Gamma} u^{'} - \dot{\Upsilon} u^{'}\right]^{n}\vphantom{\sum_{N_{k}}}\right)
\end{IEEEeqnarray}

In the above equations the flux terms are summed over $N_{c}$, which indicates the set of continuity volumes to which a given momentum flow path connects.
The $K$ terms represent effective wall and interfacial friction coefficients calculated using parameters only from time $n$.
The transfer of momentum associated with the interfacial transfer of mass is also explicitly evaluated.
The vector notation for the three momenta, \momVec{}, is defined by \eqref{eqn:momentumVector}.

\begin{equation}
\label{eqn:momentumVector}
\momVec{} = \begin{bmatrix}
\dot{m}_{l} \\
\dot{m}_{g} \\
\dot{m}_{e}
\end{bmatrix}
\end{equation}

The terms in the momentum equations that are evaluated at their new-time values are then linearized about their old-time values.
This includes the momenta, $\momVec{}^{n}$, and the pressure, $P^{n}$, giving \eqref{eqn:linLiqMomentumEquation} - \eqref{eqn:linEntMomentumEquation}.

\begin{IEEEeqnarray}{rCl}
\label{eqn:linLiqMomentumEquation}
\dot{m}_{l}^{n+1} - \dot{m}_{l}^{n} & = & \frac{\dt{}}{\dx{}}\left(- \sum_{i\, \in \, N_{c}} \left( \don{\alpha_l \rho_l u_l}_{\text{d}} \ave{u}_{\text{a},l} \tilde{A}\right)_{i}^{n}
 -\ave{\alpha_{l}}_{\text{a}}^{n} \nabla P^{\,n} -\ave{\alpha_{l}}_{\text{a}}^{n} \delta (\nabla P) + g \ave{\alpha_l \rho_l}_{\text{a}}^{n} \right. \nonumber \\
  & + & K^{n}_{wl}(\dot{m}_l^{n})^{2} - 2 K^{n}_{wl}\dot{m}_l^{n}\dot{m}_l^{n+1} - K^{n}_{i,gl}(\dot{m}^{n}_l - \dot{m}_g^{n})^2  \nonumber \\
 & + & \left. 2 K^{n}_{i,gl}(\dot{m}^{n}_l - \dot{m}_g^{n})(\dot{m}^{n+1}_l - \dot{m}_g^{n+1}) - \left[(1 - \eta)\dot{\Gamma} u^{'} + \dot{\Upsilon} u^{'}\right]^{n}\vphantom{\sum_{N_{k}}}\right) \\
\label{eqn:linGasMomentumEquation}
\dot{m}_{g}^{n+1} - \dot{m}_{g}^{n} & = & \frac{\dt{}}{\dx{}}\left(- \sum_{i\, \in \, N_{c}} \left( \don{\alpha_g \rho_g u_g}_{\text{d}} \ave{u}_{\text{a},g} \tilde{A}\right)_{i}^{n}
 -\ave{\alpha_{g}}_{\text{a}}^{n} \nabla P^{\,n} -\ave{\alpha_{g}}_{\text{a}}^{n} \delta (\nabla P) + g \ave{\alpha_g \rho_g}_{\text{a}}^{n}  \right. \nonumber \\
  & + & K^{n}_{wg}(\dot{m}_g^{n})^{2} + K^{n}_{i,gl}(\dot{m}^{n}_l - \dot{m}_g^{n})^2 + K^{n}_{i,ge}(\dot{m}^{n}_e - \dot{m}_g^{n})^2 + \left[\dot{\Gamma} u^{'}\right]^{n} \nonumber \\
 & - & 2 K^{n}_{i,gl}(\dot{m}^{n}_l - \dot{m}_g^{n})(\dot{m}^{n+1}_l - \dot{m}_g^{n+1}) - 2 K^{n}_{wg}\dot{m}_g^{n}\dot{m}_g^{n+1} \nonumber \\
 & - & \left. 2 K^{n}_{i,ge}(\dot{m}^{n}_e - \dot{m}_g^{n})(\dot{m}^{n+1}_e - \dot{m}_g^{n+1})\vphantom{\sum_{N_{k}}}\right) \\
\label{eqn:linEntMomentumEquation}
\dot{m}_{e}^{n+1} - \dot{m}_{e}^{n} & = & \frac{\dt{}}{\dx{}}\left(- \sum_{i\, \in \, N_{c}} \left( \don{\alpha_e \rho_e u_e}_{\text{d}} \ave{u}_{\text{a},e} \tilde{A}\right)_{i}^{n}
 -\ave{\alpha_{e}}_{\text{a}}^{n} \nabla P^{\,n} - \ave{\alpha_{e}}_{\text{a}}^{n} \delta (\nabla P ) + g \ave{\alpha_e \rho_e}_{\text{a}}^{n}  \right. \nonumber \\
  & + & K^{n}_{we}(\dot{m}_e^{n})^{2} - 2 K^{n}_{we}\dot{m}_e^{n}\dot{m}_e^{n+1} - K^{n}_{i,ge}(\dot{m}^{n}_e - \dot{m}_g^{n})^2  \nonumber \\
 & + & \left. 2 K^{n}_{i,ge}(\dot{m}^{n}_e - \dot{m}_g^{n})(\dot{m}^{n+1}_e - \dot{m}_g^{n+1}) -\left[ \eta \dot{\Gamma} u^{'} - \dot{\Upsilon} u^{'}\right]^{n}\vphantom{\sum_{N_{k}}}\right)
\end{IEEEeqnarray}

The momentum equations, \eqref{eqn:linLiqMomentumEquation} - \eqref{eqn:linEntMomentumEquation}, are now linear in the unknown variables $\momVec{}^{n+1}$.
These three equations are then simultaneously solved for the three unknown new time momenta, $\momVec{}^{n+1}$.
The differential of the pressure gradient , $\delta \left( \nabla P \right)$, should be read as the difference between the pressure updates, $\delta P_{a} - \delta P_{b}$, for the two continuity volumes attached to this momentum flow path.
The $\delta P$ terms are left as unknowns, creating additional unknowns on the right-hand side of the linear system.
The resulting system is of the form $\displaystyle \vec{A}_{j} \momVec{}_{j}^{n+1} = \vec{a}_{j} + \sum_{i\, \in \, N_{c}} \vec{b}_{j, i}\, \delta P_{i}$.
The solution of this linear system is expressed in \eqref{eqn:momStar}.

\begin{equation}
\label{eqn:momStar}
\momVec{}_{j}^{n+1} = \momVec{}^{*}_{j} + \sum_{i\, \in \, N_{c}} \frac{\partial \momVec{}_{j}}{\partial P_{i}} \delta P_{i}
\end{equation}

The definition for the new-time momenta is a function of the as-of-yet unknown changes in pressures that occur in the continuity volumes connected by the momentum flow path.
The $\momVec{}^{*}$ vector on the right-hand side of \eqref{eqn:momStar} contains the tentative new time momenta.
Once all of the momentum equations have been solved in the above manner, providing estimated new-time momenta for each momentum flow path, the domain's continuity volumes are then looped over.
The continuity equations used in this work, omitting external sources, are shown in \eqref{eqn:nlnNcgMassEquation} -- \eqref{eqn:nlnLiqMassEquation}. 
The new-time velocities used for the $N_{f}$ advection terms in the continuity equations are defined with the new-time momenta, \eqref{eqn:momStar}, in conjunction with \eqref{eqn:si_vel}.
$N_{f}$ represents the set of momentum flow paths that connect to a given continuity volume, and the summation index, $i$, represents the index of a given momentum flow path.

\begin{IEEEeqnarray}{rCl}
\label{eqn:nlnNcgMassEquation}
V_c \left[(\alpha_g \rho_{n})^{n+1}\right. & - & \left. (\alpha_g \rho_{n})^{n}\right] = -\dt{} \sum_{i\, \in \, N_{f}}\left( \don{\alpha^{n}_g \rho^{n}_{n}}^{n+1}_{d} u^{n+1}_g  A_{m} \right)_{i} \\
\label{eqn:nlnLiqMassEquation}
V_c \left[ \left(\alpha_l \rho_l \right)^{n+1} \right. & - & \left. \left(\alpha_l \rho_l \right)^{n} \right]=  -\dt{} \sum_{i\, \in \, N_{f} }\left( \don{\alpha^n_l \rho^n_l}^{n+1}_{d} u^{n+1}_l  A_{m} \right)_{i} \nonumber \\
&- &\left[(1-\eta)\Gamma + \Upsilon\right]^{n+1} \\
\label{eqn:nlnGasEnergyEquation}
V_c \left[ \left( \alpha_g \rho_g h_g \right)^{n+1} \right. & - & \left. \left( \alpha_g \rho_g h_g \right)^{n} - \alpha^{n}_g ( P^{\,n+1} - P^{n} ) \right] = \dt{} \left[ q_{i,v} + \dot{\Gamma} h^{'}_v + q_{gl} \right]^{n+1} \nonumber \\
& - &\dt{} \sum_{i\, \in \, N_{f}} \left(  \don{\alpha^{n}_g \rho^{n}_g h^{n}_{g}}^{n+1}_{d} u^{n+1}_g  A_{m} \right)_{i} + \dt{} q^{n}_{wg} \\ 
\label{eqn:nlnLiqEnergyEquation}
V_c\left[\left( \alpha_l \rho_l h_l \right)^{n+1} \right. & - & \left. \left( \alpha_l \rho_l h_l \right)^{n} - \alpha^{n}_l (P^{\,n+1} - P^{\,n}) \right] = \dt{} \left[q_{i,l} -\dot{\Gamma} h^{'}_l - q_{gl}\right]^{n+1}  + \dt{} q^{n}_{wl} \nonumber \\
& -& \dt{} \sum_{i\, \in \, N_{f} } \left( \don{\alpha^{n}_l \rho^{n}_l h^{n}_l}^{n+1}_{d} u^{n+1}_l A_{m} + \don{\alpha^{n}_e \rho^{n}_l h^{n}_l}^{n+1}_{d} u^{n+1}_e  A_{m} \right)_{i}\\
\label{eqn:nlnEntMassEquation}
V_c \left[ \left(\alpha_e \rho_l \right)^{n+1}\right. & - & \left. \left(\alpha_e \rho_l \right)^{n} \right]= -\dt{} \sum_{i\,\in \,N_{f}} \left( \don{\alpha^{n}_e \rho^{n}_l}^{n+1}_{d} u^{n+1}_e  A_{m} \right)_{i}+ \left[ \Upsilon -\eta\Gamma \right]^{n+1} \\
\label{eqn:nlnVapMassEquation}
V_c\left[ \left(\alpha_g \rho_v \right)^{n+1}\right. &-& \left. \left(\alpha_g \rho_v \right)^{n}\right] = - \dt{} \sum_{i \, \in \, N_{f}} \left( \don{\alpha^{n}_g \rho^{n}_v}^{n+1}_{d} u^{n+1}_g  A_{m} \right)_{i} + \Gamma^{n+1}
\end{IEEEeqnarray}

Unlike the momentum equations, the continuity equations are linearized with variables other than their conserved quantities, \eqref{eqn:independentVariables}.
This creates a more complicated linearization than that used for the momentum equations.
The new-time variables in \eqref{eqn:nlnNcgMassEquation} -- \eqref{eqn:nlnLiqMassEquation} are linearized about the old-time variables with respect to the continuity variables in \eqref{eqn:independentVariables}.
The use of \eqref{eqn:momStar} in defining the new-time velocities introduces inter-continuity volume coupling through the momentum equations' dependency upon the changes in pressure of both the continuity volume being solved, $\delta P$, and those of the continuity volumes to which the $N_{f}$ flow paths connect , $\delta P_{o(i)}$.

\begin{equation}
\label{eqn:linSystem}
\underbrace{\left[\vec{Z}_{c}\right]}_{\left[6\, \text{x}\, 6 + N_{f}\right]} \delta \vec{C}_{c} = \vec{r}_{c}
\end{equation}

With the addition of the unknown pressure updates from the connected flow paths, $\delta P_{o(i)}$,  the six linearized continuity equations now form the linear system \eqref{eqn:linSystem}.
Additionally, while the momentum equations were fully expanded to solve for the new-time conserved quantities directly, the continuity equations instead solve for their associated updates, \eqref{eqn:linearUpdate}.

\begin{equation}
\label{eqn:linearUpdate}
\delta \vec{C}_{c} \equiv 
\begin{bmatrix}
\delta ( \alpha_{g} P_{n} ) \\
\delta \alpha_{g} \\
\delta ( \alpha_{g} h_v ) \\
\delta ( (1 - \alpha_{g} ) h_l ) \\
\delta \alpha_{e} \\
\delta P \\ 
\delta P_{o(i)} \\
\vdots \\
\delta P_{o(N_{f})}
\end{bmatrix}
=
\begin{bmatrix}
( \alpha_{g} P_{n})^{n+1} - (\alpha_{g} P_{g} )^{n} \\
\alpha^{n+1}_{g} - \alpha^{n}_{g} \\
( \alpha_{g} h_{v} )^{n+1} - ( \alpha_{g} h_{v} )^{n} \\
( ( 1 - \alpha_{g} ) h_{l} )^{n+1} - ( ( 1 - \alpha_{g} ) h_{l} )^{n} \\
\alpha^{n+1}_{e} - \alpha^{n}_{e} \\
 P^{n+1} - P^{n} \\
 P_{i}^{n+1} - P_{i}^{n} \\
 \vdots \\
 P_{N_{f}}^{n+1} - P_{N_{f}}^{n}
\end{bmatrix}
\end{equation}

Each continuity volume's linearized system of equations, \eqref{eqn:linSystem}, is subjected to partial LU decomposition without pivoting.
The resulting lower-triangular matrix has a unit diagonal.
The sixth equation in the upper-triangular system is then scaled by its diagonal.
This upper-triangular, rectangular system for each continuity volume is then stored for later back-substitution.
Since the pressure update corresponds to the last row in the system of equations, this allows for the isolation of the pressure updates.
The last row of each continuity volume's \eqref{eqn:linSystem} is then formed into a global, linear pressure matrix, $\vec{A}_{\text{lin}}$, and its associated right-hand side, $\vec{res}_{\text{lin}}$.

The two loops, one over the momentum flow paths and one over the continuity volumes, form a single group of operations that act upon a given domain.
This grouping will be known hereafter as assembling the pressure matrix for a given domain.
\alg{alg:xschem} shows the two loops and their associated actions.
Note that a discussion of vectors $\vec{F}_{m}$ and $\vec{F}_{c}$ will be in \sect{subsect:nlnCobraAlgo}, and one of vectors $\vec{S}_{m}$ and $\vec{S}_{c}$ will be in \sect{sect:nlnScaling}.

\begin{algo}[ht!]
\setlength{\baselineskip}{0.625\baselineskip}
\begin{algorithmic}[1]
\Loop \; Momentum Flow Paths
	\Calculate $\vec{J}_{m, j}$, $\vec{a}_{j}$, and $\vec{b}_{j}$
	\Calculate $\momVec{}^{*}$ and $\frac{\partial \momVec{}}{\partial P}$
	\Calculate $\vec{F}_{m, j}$	
	\Calculate $\vec{S}_{m, j}$	
\EndLoop
\Loop \; Continuity Volumes
   	\Calculate $\vec{Z}_{c,j}$ and $\vec{r}_{c, j}$
 	\Calculate $\vec{F}_{c, j}$	
	\Calculate $\vec{S}_{c, j}$	
   	\Calculate $\vec{U}_{c, j} = \vec{L}_{c, j}^{-1} \vec{Z}_{c, j}$
   	\Calculate $\vec{r}^{*}_{c, j} = \vec{L}_{c, j}^{-1} \vec{r}_{c, j}$
   	\Set Scale $\vec{U}[6, :]_{c, j}$ and $\vec{r}[6]^{*}_{c, j}$ by $\vec{U}[6,6]_{c, j}$
   	\Set $\vec{A}_{\text{lin},i, :} = \vec{U}[6,:]_{c, j}$
   	\Set $\vec{res}_{\text{lin},i} = \vec{r}[6]^{*}_{c, j}$
\EndLoop
\end{algorithmic}
\caption{Assembling the Pressure Matrix}
\label{alg:xschem}
\end{algo}

\begin{equation}
\label{eqn:pressureUpdateSystem}
\vec{A}_{\text{lin}} \delta \vec{P}_{\text{lin}} = \vec{res}_{\text{lin}}
\end{equation}

The system, \eqref{eqn:pressureUpdateSystem}, is then solved to determine the pressure updates for the whole domain.
A loop over each continuity volume is then performed so that the pressure update can be used in solving its associated upper triangular system for the updates to the other five continuity variables in \eqref{eqn:linearUpdate}.
As the loop over the continuity volume is progressing, any momentum flow path connected to the given continuity volume will have its momenta updated according to \eqref{eqn:momStar} to include the contribution from that continuity volumes' change in pressure. 
After the continuity volume loop has been completed the $\momVec{}^{*}$ variables are assigned to $\momVec{}^{n+1}$. 
This update loop is shown in \alg{alg:updateVariables}.

\begin{algo}[ht!]
\setlength{\baselineskip}{0.625\baselineskip}
\begin{algorithmic}[1]
\Loop \; Continuity Volumes
	\Solve $\vec{U}_{c, j} \delta \vec{C}_{c, j} = \vec{r}^{*}_{c, j}$ using $\delta\vec{P}_{\text{lin}}$		
	\Loop \; Connected Flow Paths
		\Set $\momVec{}^{*}_{i} \pluseq  \frac{\partial \momVec{}_{i}}{\partial P_{j}} \delta P_{j}$
	\EndLoop
\EndLoop
\Set $\momVec{}^{n+1} \Leftarrow \momVec{}^{*}$
\end{algorithmic}
\caption{Updating Continuity and Momentum Variables}
\label{alg:updateVariables}
\end{algo}

Upon completion of this process, a single timestep is considered to have been taken.
The solution obtained via this process is subjected to several physical and computational limits discussed in \sect{sect:miscConcerns}.

%-------------------------------------------------------------------------------
%-------------------------------------------------------------------------------
%-------------------------------------------------------------------------------
\section{Nonlinear \cobra{}}
\label{sect:nlnCobraSolver}
As obtained, the \cobra{} software used the timestep of the semi-implicit method as outlined in \sect{sect:linCobraAlg}.
In order to evaluate the proposed domain decomposition algorithm, \cobra{} needed to be modified to be able to take multiple Newton steps within a given timestep.
To achieve this goal, several objectives were met.
First, the software routines that calculated portions of $\vec{F}(\vec{x}^{k})$ and $\vec{J}(\vec{x}^{k})$ were evaluated and transitioned to the proper nonlinear discretization.
Second, a method to identify the active portion of the domain was developed to correctly construct $\vec{x}^{k}$ and $\vec{F}^{k}$.
Lastly, a Newton loop algorithm was implemented.
This section will describe work done to meet the above objectives.

%-------------------------------------------------------------------------------
%-------------------------------------------------------------------------------
%-------------------------------------------------------------------------------
\subsection{Evaluation of Discretization}
\label{subsect:nonlinearDiscretization}

Given the formation of the governing equations from \sect{sect:linCobraAlg}, several modifications were required to implement the nonlinear solver.
Using an iterative nonlinear solver requires that the governing equations be linearized about a tentative new-time value $\vec{x}^{n+1, k}$; this differs from the linear case in which the linearization occurs about the old-time values $\vec{x}^{n}$.
First, the correct linearization about a new-time guess value as opposed to the old-time value needed to be implemented.
Each Newton step is an incremental change in the new-time variable, not a change in the independent parameters over a timestep.
Second, memory saving techniques had been employed in the construction of the linear solver that precluded more than one Newton step.
In particular, there had been the implicit assumption within the software that the first Newton iterate, $\vec{x}^{n+1, 0}$, was the old-time variable $\vec{x}^{n}$.
This design decision required that all subroutines involved with the evaluation of the components of both the nonlinear residual and its Jacobian be vetted for accurate usage of variables.
Third, the assumption that $\vec{x}^{n+1, k} = \vec{x}^{n}$ produced source code that was inconsistent with an iterative Newton method.
The source code was modified to reflect the intended discretization of the governing conservation laws.
To do this, areas had to be identified where: there were implicit cancellation of terms, the new-time variables were used in place of old-time variables, and the old-time variables were used in place of new-time variables.
This required each of the Fortran procedures involved with the calculation of the various physical parameters and operators to be evaluated to determine if the variables being used were correct.
Where appropriate, the software was changed to reflect the distinction between old-time and tentative new-time variables and to introduce terms that had been assumed to be equal to zero.
Additionally, upon verification of the proper discretization of the governing equations, several architectural changes were also made to expedite the implementation of the nonlinear solver.

%-------------------------------------------------------------------------------
%-------------------------------------------------------------------------------
%-------------------------------------------------------------------------------
\subsection{Active Domain Determination}
\label{subsect:activeDomainDetermination}
In \cobra{} the smallest geometric component that can be input is the channel.
As mentioned in \sect{sect:geometry} channels inherit their geometric properties from the section in which they reside.
To create geometrically complicated problems, it is possible to enforce zero momentum in a given momentum flow path, decoupling adjacent continuity volumes by eliminating the flow paths that connect them.
This constraint is known as a no-flow boundary condition, and it allows more complicated models to be created by isolating portions of the domain.
If an isolated domain is not connected to an external boundary condition, then it is marked as an inactive region.
The inactive regions are thermodynamically isolated and form a closed system.
These areas of the domain do not impact the solution in the portion of the domain that has boundary values, known as the active region.
After evaluating several prototypical models, it was determined that a non-trivial portion of the domain of these problems were thermodynamically isolable.
\fig{fig:activeDomain} shows an example of a domain with inactive regions.
The inactive portions of the domain are marked with grey, the active with white, and the external boundary conditions with black.
Inter-channel flow paths are marked with dashed lines.

\begin{figure}[ht!]
\centering
\tikzsetnextfilename{images/active_inactive_eps}
\begin{tikzpicture}
% \draw [thick] (-2,-3) rectangle (2,3);
% \draw [<->] (3.5,-3) -- (3.5,3);
% \draw (3.25,-3) -- (3.75,-3);
% \draw (3.25,3) -- (3.75,3);
% \draw [<->] (-2,-3.75) -- (2,-3.75);
% \draw (-2,-3.5) -- (-2,-4);
% \draw (2,-3.5) -- (2,-4);

% Middle rows of gray
\filldraw [gray] (-4.4,.5) rectangle (-3.6,2.5);
\filldraw [gray] (-.4,.5) rectangle (.4,2.5);
\filldraw [gray] (-4.4,-.5) rectangle (-3.6,-2.5);
\filldraw [gray] (-.4,-.5) rectangle (.4,-2.5);

% Top row of gray
\filldraw [gray] (-4.4,7.5) rectangle (-3.6,5);
\filldraw [gray] (-.4,7.5) rectangle (.4,5);
\filldraw [gray] (1.6,7.5) rectangle (2.4,5);
\filldraw [gray] (3.6,7.5) rectangle (4.4,5);

% Bottom row of gray
\filldraw [gray] (1.6,-7) rectangle (2.4,-4.5);

% Top and bottom row of rectangles
\foreach \x in {-4,-2,...,4}
\foreach \y in {-6,6}
{
\draw (\x,\y) +(-.4,-1.5) rectangle ++(.4,1.5);
}

% Middle rows of rectangles
\foreach \x in {-4,-2,0,4}
\foreach \y in {-2,2}
{
\draw (\x,\y) +(-.4,-1.5) rectangle ++(.4,1.5);
}

%Horizontal partitions
\foreach \y in {-7.5,-7,...,-4.5,-2.5,-1.5,1.5,2.5,4.5,5,...,7.5}
\draw (-1.6,\y) -- (-2.4,\y);

\foreach \y in {-7.5,-7,...,-4.5,-2.5,-1.5,1.5,2.5}
\draw (3.6,\y) -- (4.4,\y);

\foreach \y in {-7.5,-7,...,-4.5}
\draw (-3.6,\y) -- (-4.4,\y);

\foreach \y in {-7.5,-7,...,-4.5}
\draw (-.4,\y) -- (.4,\y);

% Vertical connections
\foreach \x in {-4,-2,0,4}
\draw [dashed] (\x,3.5) -- (\x,4.5);

\foreach \x in {-2,4}
\draw [dashed] (\x,.5) -- (\x,-.5);

\foreach \x in {-4,-2,0,4}
\draw [dashed] (\x,-3.5) -- (\x,-4.5);

% Black rectangles
\filldraw [black] (-1.6,7.5) rectangle (-2.4,8.5);

\filldraw [black] (-3.6,-7.5) rectangle (-4.4,-8.5);

%Horizontal connentions
\foreach \y in {4.75,3,-3,-4.75,-5.25,...,-7.25}
\draw [dashed] (-3.6,\y) -- (-2.4,\y);

\foreach \y in {4.75,3,-3,-4.75,-5.25,...,-7.25}
\draw [dashed] (-1.6,\y) -- (-.4,\y);

\foreach \y in {4.75,-7.25}
\draw [dashed] (1.6,\y) -- (.4,\y);

\foreach \y in {4.75,-7.25}
\draw [dashed] (3.6,\y) -- (2.4,\y);

\end{tikzpicture}
\caption{A Domain with Inactive Regions}
\label{fig:activeDomain}
\end{figure}

Originally, \cobra{} would solve the momentum and continuity equations at every defined location in the problem.
The pressure matrix, $\vec{A}_{N\,x\,N}$, was sized based upon the number of volumes in the domain, $N$, not the number of active volumes, $N_{a}$.
During development it was determined that those continuity volumes and momentum flow paths which were in the inactive region would be pruned from the the simulation.
While the additional costs of including these volumes in the simulation have historically been acceptable, the use of an iterative solver would amplify the computational time spent in those regions to an unacceptable degree.
Removing these inactive volumes saved computational time by precluding the evaluation of the residuals at those locations and by modifying the size of the pressure system \eqref{eqn:pressureUpdateSystem}.
Additionally, since the inactive regions of the problem are interpreted as not being part of the system being modeled, including their equations in the residual would not provide an accurate measure of convergence.

To determine which areas of the domain were active and which ones were inactive, a geometry traversing routine was written.
This routine parses the \cobra{} input file and creates an adjacency list data structure to represent the active portion of the domain.
The data structure allowed the active volumes to be treated independently from the memory structure of the software.
Once the active volumes were identified, the way in which the domain information was stored in memory was changed.

%-------------------------------------------------------------------------------
%-------------------------------------------------------------------------------
%-------------------------------------------------------------------------------
\subsection{Nonlinear Algorithm}
\label{subsect:nlnCobraAlgo}

During this research, the \cobra{} software was transitioned from the linear solver outlined in \sect{sect:linCobraAlg} to a nonlinearly-convergent solver. 
The solution method selected for the nonlinear system was the iterative Newton's method.
Newton's method uses successive linearizations to find a solution vector that adequately satisfies the system of discrete nonlinear equations.
The algorithmic outline for the nonlinear solver is presented in \alg{alg:nlnCobraAlgorithm}.

\begin{algo}[ht!]
\setlength{\baselineskip}{0.625\baselineskip}
\begin{algorithmic}[1]
\Require $\vec{x}^{0}$ and $t^{0}$
\Set $n = 0$
\Loop \; Transient Loop
    \Set $t^{n+1} : = t^{n} + \dt{}$
    \Set $k = 0$
	\Algorithm Assemble Nonlinear Pressure Matrix	 \Comment{\alg{alg:xschem}}
	\Solve $\vec{A}^{k} \vec{\delta P}^{k} = \vec{res}^{k}$
	\Algorithm Update Nonlinear Variables \Comment{\alg{alg:updateVariables}} 
    \Loop \; Newton Loop
		\Algorithm Assemble Nonlinear Pressure Matrix \Comment{\alg{alg:xschem}}
		\Algorithm Convergence Determination \Comment{\alg{alg:nlnConvergence}}
		\If{ \textbf{end} Newton loop}
			\State \textbf{break} Newton Loop 
		\EndIf		
		\Set $k \pluseq 1$
		\Solve $\vec{A}^{k} \vec{\delta P}^{k} = \vec{res}^{k}$
		\Algorithm Update Nonlinear Variables \Comment{\alg{alg:updateVariables}}
	\EndLoop
	\Set $n \pluseq 1$
\EndLoop
\end{algorithmic}
\caption{Nonlinear \cobra{} algorithm.}
\label{alg:nlnCobraAlgorithm}
\end{algo}

The following section will roughly parallel \alg{alg:nlnCobraAlgorithm}.
First, the method by which the nonlinear residuals are evaluated and calculated is presented.
This portion will be an extension of the procedure discussed in \sect{sect:linCobraAlg} and outlined in \alg{alg:xschem}.
Second, the process through which the Newton vector is updated, \alg{alg:updateVariables}, is expounded upon and  discussed.
Lastly, the method by which determination of convergence is discussed and is presented in \alg{alg:nlnConvergence}.

A detailed discussion of the process through which the nonlinear residuals are evaluated and the pressure matrix is assembled will now be presented.
Two quantities that are important to the iterative solver are the residual vector, $\vec{F}^{k}$, and the update vector, $\delta \vec{x}$.
The update vector represents the changes in the solution vector that result from a given Newton step.
The residual vector is composed of the residuals from the governing PDEs that are being solved.
This includes both the residuals from the continuity volumes and the residuals from the momentum flow paths.

Every momentum flow path has associated with it three momentum equation residuals \eqref{eqn:nlnLiqMomentumResidual} -- \eqref{eqn:nlnEntMomentumResidual}.
The vector notation for the three momentum equations is $\vec{F}_{m}$.

\begin{IEEEeqnarray}{lCl}
\label{eqn:nlnLiqMomentumResidual}
F^{k}_{m, l} & = & \dot{m}_{l}^{n+1, k} - \dot{m}_{l}^{n} + \frac{\dt{}}{\dx{}}\left(\sum_{i\,\in \, N_{c} } \left( \don{\alpha_l \rho_l u_l}_{\text{d}} \ave{u}_{\text{a},l} \tilde{A}\right)_{i}^{n}
 +\ave{\alpha_l}^{n}_{\text{a}} \nabla P^{\,n+1, k} - g \ave{\alpha_l \rho_l}_{\text{a}}^{n} \right. \nonumber \\
& + & \left. K^{n}_{wl}(\dot{m}_l^{n+1, k})^2 - K^{n}_{i,gl}(u^{n+1, k}_{l} - u^{n+1,k}_{g})^2 + \left[(1 - \eta)\dot{\Gamma} u^{'} + \dot{\Upsilon} u^{'}\right]^{n} \vphantom{\sum_{N_{k}}}\right) \\
\label{eqn:nlnGasMomentumResidual}
F^{k}_{m, g} & = & \dot{m}_{g}^{n+1,k} - \dot{m}_{g}^{n} + \frac{\dt{}}{\dx{}}\left(\sum_{i\, \in \, N_{c} } \left( \don{\alpha_g \rho_g u_g}_{\text{d}} \ave{u}_{\text{a},g}  \tilde{A}\right)_{i}^{n}  +\ave{\alpha_g}^{n}_{\text{a}} \nabla P^{\,n+1, k} - g \ave{\alpha_g \rho_g}_{\text{a}}^{n} \right.\nonumber \\
& + & \left. K^{n}_{wg}(\dot{m}_g^{n+1, k})^2 + K^{n}_{i,gl}(u^{n+1, k}_{l} - u^{n+1, k}_{g})^2 + K^{n}_{i,ge}(u^{n+1,k}_{e} - u^{n+1,k}_{g})^2 - (\dot{\Gamma} u^{'})^{n} \vphantom{\sum_{N_{k}}}\right) \\
\label{eqn:nlnEntMomentumResidual}
F^{k}_{m, e} & = & \dot{m}_{e}^{n+1, k} - \dot{m}_{e}^{n} + \frac{\dt{}}{\dx{}}\left(\sum_{i\, \in \, N_{c} } \left( \don{\alpha_e \rho_l u_e}_{\text{d}} \ave{u}_{\text{a},e}  \tilde{A}\right)_{i}^{n} + \ave{\alpha_{e}}^{n}_{\text{a}} \nabla P^{\,n+1, k} - g \ave{\alpha_e \rho_l}^{n}_{\text{a}} \right. \nonumber \\
& + & \left. K^{n}_{we}(\dot{m}_e^{n+1, k})^2 - K^{n}_{i,ge}(u^{n+1, k}_{e} - u^{n+1, k}_{g})^2 + \left[\eta \dot{\Gamma} u^{'} - \dot{\Upsilon} u^{'}\right]^n\vphantom{\sum_{N_{k}}}\right)
\end{IEEEeqnarray}

These three momentum equations are now linearized about their tentative new-time values.
The system of linear equations that results from this linearization is shown in \eqref{eqn:nlnMomentumSystem}.

\begin{equation}
\label{eqn:nlnMomentumSystem}
\frac{\partial\, \vec{F}^{k}_{m}}{\partial \dot{m}^{n+1}_{l}} \delta \dot{m}_{l}^{k} + \frac{\partial\, \vec{F}^{k}_{m}}{\partial \dot{m}^{n+1}_{g}} \delta \dot{m}_{g}^{k} + \frac{\partial\, \vec{F}^{k}_{m}}{\partial \dot{m}^{n+1}_{e}} \delta \dot{m}_{e}^{k} + \sum_{i\, \in \, N_{c}} \frac{\partial\, \vec{F}^{k}_{m}}{\partial P_{i}} \delta P_{i}^{k} = - \vec{F}^{k}_{m}
\end{equation}

The first three columns on the right-hand side of \eqref{eqn:nlnMomentumSystem} represent the derivatives of the momentum residual with respect to the momentum variables.
These three columns will be referred to as the Jacobian matrix for a given momentum flow path, $\vec{J}^{k}_{m}$, defined in \eqref{eqn:momentumJacobian}.

\begin{equation}
\label{eqn:momentumJacobian}
\vec{J}^{k}_{m} = 
\begin{bmatrix}
\frac{\partial F^{k}_{m,l}}{\partial \dot{m}^{n+1}_{l}} & \frac{\partial F^{k}_{m,l}}{\partial \dot{m}^{n+1}_{g}} & \frac{\partial F^{k}_{m,l}}{\partial \dot{m}^{n+1}_{e}} \\
\frac{\partial F^{k}_{m,g}}{\partial \dot{m}^{n+1}_{l}} & \frac{\partial F^{k}_{m,g}}{\partial \dot{m}^{n+1}_{g}} & \frac{\partial F^{k}_{m,g}}{\partial \dot{m}^{n+1}_{e}} \\
\frac{\partial F^{k}_{m,e}}{\partial \dot{m}^{n+1}_{l}} & \frac{\partial F^{k}_{m,e}}{\partial \dot{m}^{n+1}_{g}} & \frac{\partial F^{k}_{m,e}}{\partial \dot{m}^{n+1}_{e}} \\
\end{bmatrix}
\end{equation}

Using \eqref{eqn:momentumVector} and \eqref{eqn:momentumJacobian}, \eqref{eqn:nlnMomentumSystem} can be expressed as \eqref{eqn:momentumMatrixSystem}.

\begin{equation}
\label{eqn:momentumMatrixSystem}
\vec{J}^{k}_{m} \delta \momVec{}^{k}  = - \vec{F}^{k}_{m} - \sum_{i\, \in \, N_{c}} \frac{\partial\, \vec{F}^{k}_{m}}{\partial P_{i}} \delta P_{i}^{k}
\end{equation}

This system of equations, \eqref{eqn:momentumMatrixSystem}, is then solved for the updated new-time momentum vector via the procedure shown in \eqref{eqn:momentumSolution}.

\begin{IEEEeqnarray}{rcl}
\label{eqn:momentumSolution}
\vec{J}^{k}_{m} \delta \momVec{}^{k} & = & -\vec{F}_{m}^{k} - \sum_{i\,\in \, N_{c} } \frac{\partial \vec{F}^{k}_{m}}{\partial P_{i}} \delta P^{k}_{i} \nonumber \\
\vec{J}^{k}_{m} \left[\momVec{}^{n+1, k + 1} - \momVec{}^{n+1, k} \right] & = & -\vec{F}^{k}_{m} - \sum_{i \, \in \, N_{c} } \frac{\partial \vec{F}^{k}_{m}}{\partial P_{i}} \delta P^{k}_{i} \nonumber \\
\vec{J}^{k}_{m} \momVec{}^{n+1, k + 1} & = & - \vec{F}^{k}_{m} + \vec{J}^{k}_{m} \momVec{}^{n+1, k} - \sum_{i \, \in \, N_{c} } \frac{\partial \vec{F}^{k}_{m}}{\partial P_{i}} \delta P^{k}_{i} \nonumber \\
\momVec{}^{n+1, k + 1} & = & \left[\vec{J}_{m}^{k}\right]^{-1} \left[- \vec{F}^{k}_{m}  + \vec{J}^{k}_{m} \momVec{}^{n+1, k} \right] - \sum_{i \, \in \, N_{c} } \left[\vec{J}^{k}_{m}\right]^{-1} \left[\frac{\partial \vec{F}^{k}_{m}}{\partial P_{i}}\right] \delta P^{k}_{i} \nonumber \\
\momVec{}^{n+1, k + 1} & = & \momVec{}^{n+1, k + \onehalf} + \sum_{i \, \in \, N_{c} } \frac{\partial \momVec{}^{k}}{\partial P_{i}} \delta P^{k}_{i}
\end{IEEEeqnarray}

The variable $\momVec{}^{n+1, k + \onehalf}$, defined in \eqref{eqn:momHalf} and present in \eqref{eqn:momentumSolution}, is the iterative analog to $\momVec{}^{*}$ from \eqref{eqn:momStar}.

\begin{IEEEeqnarray}{rcl}
\label{eqn:momHalf}
\momVec{}^{n+1, k + \onehalf} & = & \left[\vec{J}^{k}_{m}\right]^{-1}\left[ \vec{J}^{k}_{m} \momVec{}^{n+1, k}  - \vec{F}^{k}_{m} \right] \nonumber \\
\momVec{}^{n+1, k + \onehalf} & = & \momVec{}^{n+1, k} - \left[\vec{J}^{k}_{m}\right]^{-1}\vec{F}^{k}_{m} \nonumber \\
\momVec{}^{n+1, k + \onehalf} & = & \momVec{}^{n+1, k} + \delta \momVec{}^{*}
\end{IEEEeqnarray}

For algebraic convenience, the update vector, $\delta \momVec{}^{k}$, is defined in \eqref{eqn:momentumUpdate}.

\begin{IEEEeqnarray}{rcl}
\label{eqn:momentumUpdate}
\delta \momVec{}^{k} & = & - \left[\vec{J}_{m}^{k}\right]^{-1} \vec{F}^{k}_{m} - \sum_{i\,\in \, N_{c}} \left[\vec{J}_{m}^{k}\right]^{-1}\left[\frac{\partial\, \vec{F}^{k}_{m}}{\partial P_{i}}\right] \delta P_{i}^{k} \nonumber \\
\delta \momVec{}^{k} & = & \delta \momVec{}^{*} + \sum_{i\,\in \, N_{c}} \frac{\partial \momVec{}}{\partial P_{i}} \delta P_{i}^{k}
\end{IEEEeqnarray}

Likewise, each continuity volume has associated with it six residuals representing the mass and energy equations \eqref{eqn:nlnNcgMassResidual} -- \eqref{eqn:nlnVapMassResidual}.
For a given continuity volume, these six equations are collectively referred to as $\vec{F}_{c}$.

\begin{IEEEeqnarray}{lCl}
\label{eqn:nlnNcgMassResidual}
F^{k}_{c, n} & = & V_c\left[ (\alpha_g \rho_{n})^{n+1, k} -(\alpha_g \rho_{n})^{n}\right] +\dt{} \sum_{i\, \in \, N_{f}}\left( \don{\alpha^{n}_g \rho^{n}_{n}}^{n+1,k}_{d} u^{n+1, k}_{g}  A_{m} \right)_{i} \\
\label{eqn:nlnLiqMassResidual}
F^{k}_{c, l} & = & V_c \left(\alpha_l \rho_l \right)^{n+1,k} - V_c \left(\alpha_l \rho_l \right)^{n} + \dt{} \sum_{i\,\in\,N_{f}} \left(\don{\alpha^n_l \rho^n_l}^{n+1,k}_{d} u^{n+1, k}_l A_{m} \right)_{i}   \nonumber \\
&+& \left[(1-\eta)\Gamma + \Upsilon\right]^{n+1, k} \\
\label{eqn:nlnGasEnergyResidual}
F^{k}_{e, g} & = & V_c \left[\left( \alpha_g \rho_g h_g \right)^{n+1, k} - \left( \alpha_g \rho_g h_g \right)^{n} - \alpha^{n}_{g} ( P^{\,n+1, k} - P^{\,n} ) \right] - \dt{} q^{n}_{wg} \nonumber \\
& - & \dt{} \left[q_{i,v} + \dot{\Gamma} h^{'}_v + q_{gl}\right]^{n+1, k} + \dt{} \sum_{i\,\in\,N_{f}} \left(\don{\alpha^{n}_g \rho^{n}_g h_g^{n}}^{n+1,k}_{d} u^{n+1, k}_g  A_{m} \right)_{i} \\
\label{eqn:nlnLiqEnergyResidual}
F^{k}_{e, l} & = & V_c\left[\left( \alpha_l \rho_l h_l \right)^{n+1,k} - \left( \alpha_l \rho_l h_l \right)^{n} - \alpha^{n}_l (P^{\,n+1,k} - P^{\,n})\right] - \dt{} \left[q_{i,l} - \dot{\Gamma} h^{'}_l - q_{gl}\right]^{n+1,k}    \nonumber \\
& +& \dt{} \sum_{i \, \in \, N_{f} } \left( \don{\alpha^{n}_l \rho^{n}_l h^{n}_l}^{n+1,k}_{d} u^{n+1,k}_l A_{m} + \don{\alpha^{n}_e \rho^{n}_l h^{n}_l}^{n+1,k}_{d} u^{n+1,k}_e  A_{m} \right)_{i} -\dt{} q^{n}_{wl}\\
\label{eqn:nlnEntMassResidual}
F^{k}_{c, e} & = & V_c \left(\alpha_e \rho_l \right)^{n+1,k} - V_c \left(\alpha_e \rho_l \right)^{n} + \dt{} \sum_{i \, \in \, N_{f} } \left( \don{\alpha^{n}_e \rho^{n}_l}^{n+1, k}_{d} u^{n+1,k}_e  A_{m} \right)_{i} \nonumber \\
&-& \left[\Upsilon - \eta \Gamma \right]^{n+1, k} \\
\label{eqn:nlnVapMassResidual}
F^{k}_{c, v} & = & V_c \left[\left(\alpha_g \rho_v \right)^{n+1, k} - \left(\alpha_g \rho_v \right)^{n}\right] + \dt{} \sum_{i \, \in \, N_{f} } \left( \don{\alpha^{n}_g \rho^{n}_v}^{n+1,k}_{d} u^{n+1, k}_{g} A_{m} \right)_{i} - \Gamma^{n+1, k}
\end{IEEEeqnarray}

For each continuity volume, the above six equations are linearized about the tentative new-time independent parameters.

\begin{IEEEeqnarray}{rcl}
\label{eqn:nlnContinuitySystem}
 - \vec{F}^{k}_{c} & = & \frac{\partial \vec{F}^{k}_{c}}{\partial (\alpha_{g} P_{n} )} \delta (\alpha_{g} P_{n})^{k} + \frac{\partial \vec{F}^{k}_{c}}{\partial \alpha_{g}} \delta \alpha^{k}_{g} + \frac{\partial \vec{F}^{k}_{c}}{\partial (\alpha_{g} h_{v} )} \delta (\alpha_{g} h_{v})^{k} + \frac{\partial \vec{F}^{k}_{c}}{\partial ((1 - \alpha_{g}) h_{l} )} \delta ((1 - \alpha_{g}) h_{l})^{k} \nonumber \\
& + & \frac{\partial \vec{F}^{k}_{c}}{\partial \alpha_{e}} \delta \alpha_{e}^{k} + \frac{\partial \vec{F}^{k}_{c}}{\partial P } \delta P^{k} + \sum_{i \, \in \, N_{f} } \frac{\partial \vec{F}^{k}_{c}}{\partial \momVec{}_{i} } \delta \momVec{}_{i}^{k} 
\end{IEEEeqnarray}

The matrix of the derivatives of the continuity residual vector with respect to the momentum vector at the boundaries of the continuity volume, $\displaystyle \frac{\partial\,\vec{F}_{c}}{\partial \momVec{}_{i}}$, will be known as $\vec{\Xi}_{i}$.
This matrix converts the momentum vector at a continuity volume's edge to mass and energy flows and is defined by \eqref{eqn:momentumToFlowRates}.

\begin{equation}
\label{eqn:momentumToFlowRates}
\vec{\Xi}^{n+1, k}_{i} = \dt{} \begin{bmatrix}
 0 & \frac{\don{\alpha^{n}_{g} \rho^{n}_{n}}^{n+1,k}_{d}}{\ave{\alpha_{g} \rho_{g}}^{n}_{a}} & 0 \\
\frac{\don{\alpha^{n}_{l}\rho^{n}_{l}}^{n+1,k}_{d}}{\ave{\alpha_{l} \rho_{l}}^{n}_{a}} & 0 & 0 \\
0 & \frac{\don{\alpha^{n}_{g} \rho^{n}_{g} h^{n}_{g}}^{n+1,k}_{d}}{\ave{\alpha_{g} \rho_{g}}^{n}_{a}} & 0 \\
\frac{\don{\alpha^{n}_{l}\rho^{n}_{l} h^{n}_{l}}^{n+1,k}_{d}}{\ave{\alpha_{l} \rho_{l}}^{n}_{a}} & 0 & \frac{\don{\alpha^{n}_{e} \rho^{n}_{l} h^{n}_{l}}^{n+1, k}_{d}}{\ave{\alpha_{e} \rho_{l}}^{n}_{a}} \\
0 & 0 & \frac{ \don{\alpha^{n}_{e} \rho^{n}_{l}}^{n+1, k}_{d}}{ \ave{\alpha_{e} \rho_{l}}^{n}_{a}} \\
0 & \frac{ \don{\alpha^{n}_{g} \rho^{n}_{v}}^{n+1, k}_{d}}{ \ave{\alpha_{g} \rho_{g}}^{n}_{a}} & 0
\end{bmatrix}_{i}
\end{equation}

The definition of the momentum update vector, \eqref{eqn:momentumUpdate}, is used to eliminate the momentum update vector from \eqref{eqn:nlnContinuitySystem}.
In the above equations, a given momentum flow path will be referenced by the index $i$.
Note that the summation over the number of continuity volumes to which a particular flow path is connected is at most two terms.
Since each momentum flow path can connect to only two continuity volumes, one of those two continuity volumes will be the current continuity volume, which will be marked by the index $s(i)$, and the other continuity volume will be indexed with $o(i)$.
The resulting system is shown in \eqref{eqn:continuitySystem}.

\begin{IEEEeqnarray}{rcl}
\label{eqn:continuitySystem}
\frac{\partial \vec{F}^{k}_{c}}{\partial (\alpha_{g} P_{n} )} \delta (\alpha_{g} P_{n})^{k} + \frac{\partial \vec{F}^{k}_{c}}{\partial \alpha_{g}} \delta \alpha^{k}_{g} + \frac{\partial \vec{F}^{k}_{c}}{\partial (\alpha_{g} h_{v} )} \delta (\alpha_{g} h_{v})^{k} + \frac{\partial \vec{F}^{k}_{c}}{\partial ((1 - \alpha_{g}) h_{l} )} \delta ((1 - \alpha_{g}) h_{l})^{k} & + & \nonumber \\
\frac{\partial \vec{F}^{k}_{c}}{\partial \alpha_{e}} \delta \alpha_{e}^{k} + \frac{\partial \vec{F}^{k}_{c}}{\partial P } \delta P^{k} + \sum_{i\, \in \, N_{f} } \vec{\Xi}^{k}_{i}\left( \delta \momVec{}_{i}^{*} + \frac{\partial \momVec{}_{i}^{k}}{\partial P_{s(i)}} \delta P_{s(i)}^{k} + \frac{\partial \momVec{}_{i}^{k}}{\partial P_{o(i)}} \delta P_{o(i)}^{k} \right) = \vec{F}^{k}_{m} & &
\end{IEEEeqnarray}

By rearranging \eqref{eqn:continuitySystem} and recognizing that the $\delta P_{s(i)}$ terms from the momentum equations correspond to the given volumes own $\delta P$, \eqref{eqn:continuitySystem02} is obtained.

\begin{IEEEeqnarray}{rcl}
\label{eqn:continuitySystem02}
\frac{\partial \vec{F}^{k}_{c}}{\partial (\alpha_{g} P_{n} )} \delta (\alpha_{g} P_{n})^{k} + \frac{\partial \vec{F}^{k}_{c}}{\partial \alpha_{g}} \delta \alpha^{k}_{g} + \frac{\partial \vec{F}^{k}_{c}}{\partial (\alpha_{g} h_{v} )} \delta (\alpha_{g} h_{v})^{k} + \frac{\partial \vec{F}^{k}_{c}}{\partial ((1 - \alpha_{g}) h_{l} )} \delta ((1 - \alpha_{g}) h_{l})^{k} & + & \nonumber \\
\frac{\partial \vec{F}^{k}_{c}}{\partial \alpha_{e}} \delta \alpha_{e}^{k} + \left( \frac{\partial \vec{F}^{k}_{c}}{\partial P } + \sum_{i\,\in \, N_{f} } \vec{\Xi}^{k}_{i}\frac{\partial \momVec{}^{k}_{i}}{\partial P_{s(i)}}\right) \delta P^{k} + \sum_{i \, \in \, N_{f} } \vec{\Xi}^{k}_{i} \frac{\partial \momVec{}_{i}^{k}}{\partial P_{o(i)}} \delta P_{o(i)}^{k} & = & \nonumber \\
- \vec{F}^{k}_{c} - \sum_{i \, \in \, N_{f} } \vec{\Xi}^{k}_{i} \delta \momVec{}_{i}^{*} & &
\end{IEEEeqnarray}

In \eqref{eqn:continuitySystem02} there are six equations and six plus the number of connecting flow paths, $N_{f}$, unknowns.
The first six columns of \eqref{eqn:continuitySystem02} will be represented by $\vec{J}^{k}_{c}$.
The next $N_{f}$ columns, representing the inter-continuity volume coupling, will be denoted  by $\vec{K}^{k}_{c}$.
Notice that the right-hand side of \eqref{eqn:continuitySystem02} is the residual evaluated at the tentative new-time values plus the change in the momentum vector multiplied by $\vec{\Xi}$.
Since the continuity equations are linear functions of their flow paths' momentum vectors, this combination is the equivalent of using $\momVec{}^{n+1, k + \onehalf}$ for the advecting velocity in the evaluation of the continuity residuals.
Due to the use of the tentative new-time momenta in the advection terms, there is now a discrepancy between the residuals, $\vec{F}^{k}_{c}$, for the continuity volume and the right-hand side of the linear system, $\vec{r}^{k}_{c}$.
Note that for convergence determination, the residual, $\vec{F}^{k}_{c}$, is used, not the modified right-hand side in \eqref{eqn:continuitySystem02}.
The vector of unknowns is $\delta \vec{C}^{k}_{c}$, which is the iterative analog of \eqref{eqn:linearUpdate}.
Using the above matrix notation, \eqref{eqn:continuitySystem02} can be represented as \eqref{eqn:nlnLinearSystem}.

\begin{equation}
\label{eqn:nlnLinearSystem}
\left[ \vec{J}^{k}_{c} \vert \vec{K}^{k}_{c} \right] \delta \vec{C}_{c}^{k} = \vec{r}^{k}_{c}
\end{equation}

This rectangular, linear system for the nonlinear boundary volume is then subjected to partial $\vec{LU}$ decomposition without pivoting.
This $\vec{LU}$ decomposition is such that the lower triangular matrix has ones along the diagonal.
The final row of the linear system in \eqref{eqn:nlnLUSystem} is then scaled by the sixth row's sixth column, $\vec{U}^{k}_{c}[6,6]$.

\begin{equation}
\label{eqn:nlnLUSystem}
\left[ \vec{U}^{k}_{c} \vert \left[\vec{L}^{k}_{c}\right]^{-1}\vec{K}^{k}_{c} \right] \delta \vec{C}^{k}_{c} = \left[\vec{L}^{k}_{c}\right]^{-1}\vec{r}^{k}_{c}
\end{equation}

This scaled row of \eqref{eqn:nlnLUSystem} is referred to as the pressure equation for a given continuity volume.
The pressure equations for all continuity volumes are collated into the global pressure matrix according to the continuity volumes' ordinal.
This pressure matrix is then inverted to solve for the pressure update for all continuity volumes.
At this point all continuity volumes are looped over to solve their linear systems, \eqref{eqn:nlnLUSystem}, for the non-pressure portion of their update vectors, $\delta \vec{C}_{c}^{k}$.

The above discussion expounds upon the algorithm for a single Newton step that is given by \alg{alg:xschem}.
It is during this process that both the residuals and the scale factors, $\vec{F}^{k}$ and $\vec{S}^{k}$, are evaluated and stored.
Concurrently, the pressure matrix is being both constructed and solved.
In order to obtain the new-time momentum variables, \eqref{eqn:momentumSolution} is used during \alg{alg:updateVariables}.
For the purpose of convergence determination, the change in the momentum vectors, $\delta \momVec{}$, is calculated for each flow path, \eqref{eqn:deltaMomenta}.
This is necessary since the momentum equations are solved for the new-time variables directly, not their updates.

\begin{equation}
\label{eqn:deltaMomenta}
\delta \momVec{}^{k} = \momVec{}^{n+1,k + \onehalf} + \sum_{i \, \in \, N_{c} } \frac{\partial \momVec{}^{k}}{\partial P_{i}} \delta P^{k}_{i} - \momVec{}^{n+1, k}
\end{equation}

The residuals from every volume in the active domain, both continuity and momentum, are assembled into the $\vec{F}^{k}$ vector.
The degree to which a given solution vector, $\vec{x}^{n+1, k}$, satisfies the governing differential equation can be measured by a norm of the residual vector of the nonlinear system.
If the exact solution for the nonlinear system were known, the residual vector would be identically $\vec{0}$; however, in practice the residual will never be identically zero.
As such it is necessary to discuss the various means by which a solution for a timestep is determined to be accurate enough.

While a Newton step is being performed, the residual is being evaluated with the current solution vector, $\vec{x}^{n+1,k}$.
Once the Newton update vector, $\delta \vec{x}^{k}$, has been calculated, the residual is reevaluated using the new solution vector, $\vec{x}^{n+1, k+1}$.
Based upon the discussion above, the process for evaluating the residual is the same as that of assembling the pressure matrix.
Using the two residuals, $\vec{F}^{k}$ and $\vec{F}^{k+1}$, the comparative accuracy of the two solution vectors can be determined.
These residual vectors are used in the evaluation of the Newton loop termination criteria.

There are three ways in which the iterative Newton loop can be terminated.
One way is to have the number of Newton iterates, $k$, exceed a specified threshold, $\kmax{}$.
Another way is to have the $\mathcal{L}_{2}$ norm of the relative update vector, \eqref{eqn:nlnUpdateVector}, divided by the square root of the number of entries in the update vector, $N_{u}$, drop below a specified threshold $\dtol{}$.

\begin{equation}
\label{eqn:nlnUpdateVector}
\delta \tilde{\vec{x}}^{k} = \frac{\delta \vec{x}^{k}}{\vec{x}^{n+1, k}} = \frac{ \vec{x}^{n+1, k+1} - \vec{x}^{n+1, k}}{\vec{x}^{n+1,k}}
\end{equation}

Finally, the Newton loop will terminate if the $\mathcal{L}_{2}$ norm of the scaled residual vector, \eqref{eqn:nlnScaledResidual}, divided by the square root of the number of unknowns, $N_{u}$, drops below a specified threshold, $\ftol{}$.

\begin{equation}
\label{eqn:nlnScaledResidual}
\tilde{\vec{F}}^{k} = \left[\vec{S}^{k}\right]^{-1}\vec{F}^{k}
\end{equation}

The scaling matrix, $\vec{S}^{k}$, will be discussed in detail in \sect{sect:nlnScaling}.
Note that it is assembled and stored during the process of assembling the pressure matrix, \alg{alg:xschem}.
The convergence determination logic for when to end the Newton loop is given in \alg{alg:nlnConvergence}.

\begin{algo}[ht!]
\setlength{\baselineskip}{0.625\baselineskip}
\begin{algorithmic}[1]
\If{ $\displaystyle \frac{||\tilde{\vec{F}}^{k+1}||_{2}}{\sqrt{N_{u}}} \leq \ftol{}$}
	\State \textbf{end} Newton Loop
\ElsIf{ $\displaystyle \frac{||\delta \tilde{\vec{x}}^{k}||_{2}}{\sqrt{N_{u}}} \leq \dtol{}$}
	\State \textbf{end} Newton Loop
\ElsIf{ $k > \kmax{}$}
	\State \textbf{end} Newton Loop
\EndIf
\end{algorithmic}
\caption{Convergence Determination of Newton Loop}
\label{alg:nlnConvergence}
\end{algo}

In order to determine convergence according to \alg{alg:nlnConvergence}, it is necessary to have the values of $\vec{F}^{k+1}$ and $\vec{S}^{k+1}$.
This is accomplished by looping over the domain and assembling the pressure matrix again.
As shown in \alg{alg:xschem}, the residual and scale factor are both evaluated during this process.
If \alg{alg:nlnConvergence} determines that the Newton loop should not end, then the pressure matrix has already been  assembled and can then be solved for the new update vector.

%-------------------------------------------------------------------------------
%-------------------------------------------------------------------------------
%-------------------------------------------------------------------------------
\section{Operator-Based Scaling}
\label{sect:nlnScaling}

An important aspect of the nonlinear solver outlined in \sect{sect:nlnCobraSolver} is the convergence criteria.
As part of this work a novel operator-based scaling has been developed to obtain meaningful convergence thresholds.
The iterative solver depends upon the scaled residual to help determine when convergence has been achieved.
In particular, various norms of the residual vector are evaluated to measure the degree to which the nonlinear system of algebraic equations is being satisfied.
The evaluation of residual norms necessitates the use of a scaling matrix for the residual vector.

Without proper scaling the residual has several negative characteristics \cite{Frepoli2003, McHugh1995}.
The different conservation equations will have different orders of magnitude at a given point in time because of the physical quantities that they represent.
Over the course of a transient the phasic composition of a volume can vary dramatically as phases appear and disappear.
When norms are taken of the unscaled residual those equations whose terms are the largest may have a residual that is orders of magnitude greater than those equations whose terms are smallest.
For convergence criteria that are based upon relative convergence, the reduction of those equations with larger magnitudes would result in apparent convergence without any consideration given to those equations whose scale is orders of magnitude smaller.
This would create a situation where any norm taken of the residual would be biased towards certain equations due to the their units or phasic composition.
To avoid this an operator-based scaling strategy was developed and implemented.

For a given continuity volume, the nonlinear residual will have six components: four for the conservation of mass and two for the conservation of energy.
For each momentum flow path, the three conservation of momentum equations will form the three components of the nonlinear residual.
These residuals have the units of the conserved quantities for their corresponding PDEs; \tab{tab:scaling_units_scales} shows the units for the different conservation equations.

\begin{table}[ht!]
\centering
\singlespace
\pgfplotstabletypeset[col sep=&,row sep=\\,
	columns/Residual/.style={string type, column type=l},
	columns/Units/.style={string type},
	every head row/.style={
		before row=\toprule,
		after row=\midrule
	},
	every last row/.style={
		after row=\bottomrule}]{
Residual & Units \\
Conservation of the \NCG{} Field Mass                  & [ \lbm ] \\
Conservation of the Continuous Liquid Water Field Mass & [ \lbm ] \\
Conservation of the Entrained Liquid Water Field Mass  & [ \lbm ] \\
Conservation of the Water Vapor Field Mass             & [ \lbm ] \\
Conservation of the Gaseous Phase Enthalpy             & [ BTU ]  \\
Conservation of the Liquid Phase Enthalpy              & [ BTU ]  \\
Conservation of the Continuous Liquid Field Momentum   & [ $\frac{ \lbm \text{ft} }{\text{s}}$ ] \\
Conservation of the Entrained Liquid Field Momentum    & [ $\frac{ \lbm \text{ft} }{\text{s}}$ ] \\
Conservation of the Gaseous Phase Momentum             & [ $\frac{ \lbm \text{ft} }{\text{s}}$ ] \\
}

\caption{Residuals and their units.}
\label{tab:scaling_units_scales}
\end{table}

For the aforementioned reasons, it is desirable to scale the residual.
A challenge that has been addressed in this work is the development of a method for scaling of these residuals that is based upon the local physics of interest at any given point in the transient.
In constructing this scaling factor it was determined that the following characteristics were desirable:

\begin{itemize}
\item{$(S_{i}^{-1} F_i)^{k} \approx 1$ when $\vec{x}^{k}$ is a "poor" solution.}
\item{$(S_{i}^{-1} F_i)^{k} \rightarrow 0$ when phase $i$ disappears.}
\item{$0 \leq \abs{S_{i}^{-1} F^{k}_{i}}^{k} \leq 1 $ for all values of $\vec{x}^{k}_i$.}
\end{itemize}

The dynamic behavior of the residual necessitates a method for scaling that adapts to relevant physical situations.
The scaling method developed during this work is an operator-based approach.
The governing PDEs can be viewed as a collection of operators, both linear and nonlinear, acting upon the vector of independent parameters.
The summation of these operators must balance to zero for the nonlinear equation to be satisfied.
The scaling factor developed uses the magnitudes of these operators to determine an absolute measure of the physics processes occurring in a given volume.
This is accomplished by summing the absolute value of the different discrete operators in the governing equations.
This scaling creates a relative measure of the nonlinear residual when compared to the magnitude of the physics involved in the process.

\begin{IEEEeqnarray}{lCl}
\label{eqn:nlnLiqMomentumScale}
S^{k}_{m, l} & = & \frac{\dt{}}{\dx{}} \left[\abs{\frac{\dx{} \left[\dot{m}_{l}^{n+1, k} - \dot{m}_{l}^{n}\right]}{\dt{}}} + \abs{\sum_{i\, \in \, N_{c} } \left( \don{\alpha_l \rho_l u_l}_{\text{d}} \ave{u}_{\text{a},l} \tilde{A}\right)^{n}_{i}} +\abs{\ave{\alpha_l}^{n}_{\text{a}} \nabla P^{\,n+1, k}} + \abs{g \ave{\alpha_l \rho_l}_{\text{a}}^{n}} \right.\nonumber \\
& + & \left. \abs{K^{n}_{wl}(\dot{m}_l^{n+1, k})^2} + \abs{K^{n}_{i,gl}(u^{n+1, k}_{l} - u^{n+1,k}_{g})^2} + \abs{\left[(1 - \eta)\dot{\Gamma} u^{'}\right]^{n}} + \abs{\left[\dot{\Upsilon} u^{'}\right]^{n}} \vphantom{\abs{\frac{\dx{} \left[\dot{m}_{l}^{n+1, k} - \dot{m}_{l}^{n}\right]}{\dt{}}}} \right] \\
\label{eqn:nlnGasMomentumScale}
S^{k}_{m, g} & = & \frac{\dt{}}{\dx{}} \left[\abs{\frac{\dx{} \left[ \dot{m}_{g}^{n+1,k} - \dot{m}_{g}^{n}\right]}{\dt{}}} + \abs{\sum_{i\, \in \, N_{c} } \left( \don{\alpha_g \rho_g u_g}_{\text{d}} \ave{u}_{\text{a},g}  \tilde{A}\right)_{i}^{n}}  +\abs{\ave{\alpha_g}^{n}_{\text{a}} \nabla P^{\,n+1, k}} + \abs{g \ave{\alpha_g \rho_g}_{\text{a}}^{n}} \right. \nonumber \\
& + & \abs{K^{n}_{wg}(\dot{m}_g^{n+1, k})^2} + \abs{K^{n}_{i,gl}(u^{n+1, k}_{l} - u^{n+1, k}_{g})^2} + \abs{K^{n}_{i,ge}(u^{n+1,k}_{e} - u^{n+1,k}_{g})^2} \nonumber \\
& + & \left. \abs{\left[\dot{\Gamma} u^{'}\right]^{n}} \vphantom{\abs{\frac{\dx{} \left[\dot{m}_{l}^{n+1, k} - \dot{m}_{l}^{n}\right]}{\dt{}}}} \right]\\
\label{eqn:nlnEntMomentumScale}
S^{k}_{m, e} & = & \frac{\dt{}}{\dx{}} \left[\abs{\frac{\dx{} \left[\dot{m}_{e}^{n+1, k} - \dot{m}_{e}^{n}\right]}{\dt{}}} + \abs{\sum_{i\, \in \, N_{c}} \left( \don{\alpha_e \rho_l u_e}_{\text{d}} \ave{u}_{\text{a},e}  \tilde{A}\right)_{i}^{n}} + \abs{\ave{\alpha_{e}}^{n}_{\text{a}} \nabla P^{\,n+1, k}} + \abs{g \ave{\alpha_e \rho_l}^{n}_{\text{a}}} \right. \nonumber \\
& + & \left. \abs{K^{n}_{we}(\dot{m}_e^{n+1, k})^2} + \abs{K^{n}_{i,ge}(u^{n+1, k}_{e} - u^{n+1, k}_{g})^2} + \abs{\left[\eta \dot{\Gamma} u^{'}\right]^{n}} +\abs{\left[\dot{\Upsilon} u^{'}\right]^{n}} \vphantom{\abs{\frac{\dx{} \left[\dot{m}_{l}^{n+1, k} - \dot{m}_{l}^{n}\right]}{\dt{}}}} \right]
\end{IEEEeqnarray}

For the momentum equations, \eqref{eqn:nlnLiqMomentumEquation} -- \eqref{eqn:nlnEntMomentumEquation}, the scaled residuals are shown in \eqref{eqn:nlnLiqMomentumScale} -- \eqref{eqn:nlnEntMomentumScale}.
Likewise, the scale factors for the continuity equations, \eqref{eqn:nlnNcgMassEquation} -- \eqref{eqn:nlnLiqMassEquation}, are given in  \eqref{eqn:nlnNcgMassScale} -- \eqref{eqn:nlnLiqMassScale}.

\begin{IEEEeqnarray}{lCl}
\label{eqn:nlnNcgMassScale}
S^{k}_{c, n} & = & \dt{}\left[\abs{\frac{V_c\left[(\alpha_g \rho_{n})^{n+1, k} -(\alpha_g \rho_{n})^{n}\right]}{\dt{}}} + \sum_{i\,\in \, N_{f}}\abs{ \don{\alpha^{n}_g \rho^{n}_{n}}^{n+1,k}_{d} u^{n+1, k}_{g}  A_{m} }_{i} \right]\\
\label{eqn:nlnLiqMassScale}
S^{k}_{m, l} & = & \dt{}\left[\abs{\frac{V_c \left[\left(\alpha_l \rho_l \right)^{n+1,k} - \left(\alpha_l \rho_l \right)^{n}\right]}{\dt{}}} + \sum_{i\, \in \, N_{f} } \abs{ \don{ \alpha^n_l \rho^n_l }^{n+1,k}_{d} u^{n+1, k}_l A_{m} }_{i} \right. \nonumber \\
& + & \left. \abs{\left[(1-\eta)\dot{\Gamma}\right]^{n+1,k}} + \abs{\dot{\Upsilon}^{n+1, k}} \vphantom{\abs{\frac{V_c \left[\left(\alpha_l \rho_l \right)^{n+1,k} - \left(\alpha_l \rho_l \right)^{n}\right]}{\dt{}}}} \right] \\
\label{eqn:nlnGasEnergyScale}
S^{k}_{e, g} & = & \dt{}\left[\abs{\frac{V_c \left[\left( \alpha_g \rho_g h_g \right)^{n+1, k} - \left( \alpha_g \rho_g h_g \right)^{n}\right]}{\dt{}}} +\abs{\frac{V_c \alpha^{n}_{g} ( P^{n+1, k} - P^{n} )}{\dt{}}} + \abs{q_{wg}^{n}} \right. \nonumber \\
& + & \left. \abs{\left[\dot{\Gamma} h^{'}_v\right]^{n+1,k}} + \abs{q_{i,v}^{n+1,k}} +\abs{q^{n+1,k}_{gl}} + \sum_{i\,\in \, N_{f} } \abs{\don{\alpha^{n}_g \rho^{n}_g h_g^{n}}^{n+1,k}_{d} u^{n+1, k}_g  A_{m}}_{i} \vphantom{\abs{\frac{V_c \left[\left( \alpha_g \rho_g h_g \right)^{n+1, k} - \left( \alpha_g \rho_g h_g \right)^{n}\right]}{\dt{}}}} \right]\\
\label{eqn:nlnLiqEnergyScale}
S^{k}_{e, l} & = & \dt{}\left[ \abs{\frac{V_c\left[\left( \alpha_l \rho_l h_l \right)^{n+1,k} - \left( \alpha_l \rho_l h_l \right)^{n}\right]}{\dt{}}} + \abs{\frac{V_c \alpha^{n}_l (P^{\,n+1,k} - P^{\,n})}{\dt{}}} + \abs{q_{wl}^{n}} \right . \nonumber \\
& + & \abs{\left[\dot{\Gamma} h^{'}_l\right]^{n+1}} + \abs{q^{n+1,k}_{i,l}} + \abs{q_{gl}^{n+1,k}} + \sum_{i\,\in \, N_{f} }\abs{ \don{ \alpha^{n}_l \rho^{n}_l h^{n}_l }^{n+1,k}_{d} u^{n+1,k}_l A_{m} }_{i} \nonumber \\
& + & \left. \sum_{i\,\in \, N_{f} } \abs{ \don{ \alpha^{n}_e \rho^{n}_l h^{n}_l }^{n+1,k}_{d} u^{n+1,k}_e  A_{m} }_{i} \vphantom{ \abs{\frac{V_c\left[\left( \alpha_l \rho_l h_l \right)^{n+1,k} - \left( \alpha_l \rho_l h_l \right)^{n}\right]}{\dt{}}}} \right] \\
\label{eqn:nlnEntMassScale}
S^{k}_{m, e} & = & \dt{} \left[ \abs{\frac{V_c \left[\left(\alpha_e \rho_l \right)^{n+1,k} - \left(\alpha_e \rho_l \right)^{n}\right]}{\dt{}}} + \sum^{N_{f}}_{i\,=\,1}\abs{\left( \don{\alpha^{n}_e \rho^{n}_l}^{n+1, k}_{d} u^{n+1,k}_e  \tilde{A}\right)}_{i} \right. \nonumber \\
& + & \left. \abs{\dot{\Upsilon}^{n+1, k}} + \abs{\left[\eta\dot{\Gamma} \right]^{n+1,k}} \vphantom{\abs{\frac{V_c \left[\left(\alpha_e \rho_l \right)^{n+1,k} - \left(\alpha_e \rho_l \right)^{n}\right]}{\dt{}}}} \right] \\
\label{eqn:nlnVapMassScale}
S^{k}_{c, v} & = & \dt{}\left[\abs{\frac{V_c \left[\left(\alpha_g \rho_v \right)^{n+1, k} - \left(\alpha_g \rho_v \right)^{n}\right]}{\dt{}}} + \sum_{i\,\in \, N_{f}} \abs{ \don{ \alpha^{n}_g \rho^{n}_v }^{n+1,k}_{d} u^{n+1, k}_{g}  A_{m} }_{i} \right. \nonumber \\
& + & \left.  \abs{\dot{\Gamma}^{n+1, k}} \vphantom{\abs{\frac{V_c \left[\left(\alpha_g \rho_v \right)^{n+1, k} - \left(\alpha_g \rho_v \right)^{n}\right]}{\dt{}}}}\right]
\end{IEEEeqnarray}

The issue of phase transition also needed to be considered during this work.
Since \cobra{} does not actually transition the governing equations to those for single-phase flow, there will always be a nonlinear residual for those phases that are nominally absent.
It was observed that the effects of maintaining a depleted phase in the system of equations when solving the nonlinear problem created spurious convergence issues when using these scale factors.
The unscaled residuals would be on the order of machine round-off.
The operator-based scaling factors for these residuals would also be within orders of magnitude of machine round-off.
However, the residual would be unable to decrease with additional Newton steps due to parametric constraints.
This created the situation where the scaled nonlinear residual for the depleted field would stagnate at approximately $\mathcal{O}$(1).
These depleted residuals would dominate the norms used to determine convergence.

To overcome this deficiency it was determined that when a phase or field began to deplete, the scaling factor itself would be scaled to create an artificial decrease in the scaled residual to counter the artificial presence of the depleted field.
This scaling function is shown by \eqref{eqn:scaling_factor_small}.

\begin{equation}
\label{eqn:scaling_factor_small}
S_{\phi} = \max\left[1.0, \left(C_1 \frac{\alpha_{\phi,\text{MIN}}}{\alpha_{\phi}}\right)^{C_2} \right] S_{\phi}
\end{equation}

For this work, the constant $C_1$ was set equal to 100, and the exponent $C_2$ was set equal to 10.
This particular phase-transition scaling produced the regular scaling factor when the volume fraction of a phase is at least two orders of magnitude greater than the minimum volume fraction for that phase or field, which is on the order of \expneg{1.0}{6}{[-]}.
This drove the scaled residuals for phases that were nominally not present to well below those residuals for equations of the phases that were present.

Another phase depletion limit was a floor placed upon the scale factor.
For each equation, the scale factor was limited to the minimum possible value of the conserved quantity based upon the volume fraction limits.
For the mass equations it would be the minimum macroscopic density of the given field based upon the minimum volume fraction for that field.
\tab{tab:minimumConservedValues} shows the floors for the various residuals.
These floors are updated during the iterative process.

\begin{table}[ht]
\centering
\singlespace
\pgfplotstabletypeset[col sep=&,row sep=\\,
	columns/Residual/.style={column name = Equation, string type, column type=l},
	columns/Min/.style={column name = Minimum  , string type},
	every head row/.style={
		before row=\toprule,
		after row=\midrule
	},
	every last row/.style={
		after row=\bottomrule}]{
Residual & Min \\
Conservation of the \NCG{} Field Mass                  &  $\alpha_{g, \text{min}} \rho_n $ \\
Conservation of the Continuous Liquid Water Field Mass &  $\alpha_{l, \text{min}} \rho_l $ \\
Conservation of the Entrained Liquid Water Field Mass  &  $\alpha_{e, \text{min}} \rho_l $ \\
Conservation of the Water Vapor Field Mass             &  $\alpha_{g, \text{min}} \rho_v $ \\
Conservation of the Gaseous Phase Enthalpy             &  $\alpha_{g, \text{min}} \rho_g h_g$   \\
Conservation of the Liquid Phase Enthalpy              &  $( 1.0 - \alpha_{g, \text{min}}) \rho_l h_l$   \\
Conservation of the Continuous Liquid Field Momentum   &  $ u_l \frac{\alpha_{l, \text{min}} \ave{\rho_l} }{\ave{\alpha_l \rho_l}} A_{mom}$  \\
Conservation of the Entrained Liquid Field Momentum    &  $ u_e \frac{\alpha_{e, \text{min}} \ave{\rho_l} }{\ave{\alpha_e \rho_l}} A_{mom}$  \\
Conservation of the Gaseous Phase Momentum             &  $ u_g \frac{\alpha_{g, \text{min}} \ave{\rho_g} }{\ave{\alpha_g \rho_g}} A_{mom}$ \\
}

\caption{Minimum conserved quantities for conservation equations.}
\label{tab:minimumConservedValues}
\end{table}



%-------------------------------------------------------------------------------
%-------------------------------------------------------------------------------
%-------------------------------------------------------------------------------
\section{Convergence Metric}
\label{sect:temporal_convergence}

An important factor in thermal-hydraulic safety analysis is the temporal convergence of the solution.
A definition for a temporally converged solution is required.
In theory, a temporally converged solution is one where the local truncation error due to the discrete approximation of the temporal integral is orders of magnitude below both the engineering scales of interest and precision of the physical models being used in the simulation.
Unfortunately, the precise measurement of the error in a simulation requires that an analytic solution be available for comparison.
During the simulation of physically realistic systems, there is rarely an analytic solution against which to compare.
This situation requires a slightly different definition of a temporally converged solution --- a definition that does not depend upon accurately measuring the local truncation error.

An alternative definition for temporal convergence could be ``as the timestep size is reduced, the change in the solution is small enough."
While commonly used, this definition is subjective.
Traditionally, this ``change in solution" is addressed in a very qualitative manner.
Engineering judgment of which parameters of the solution are of interest is required.
These parameters may include items of regulatory concern such as peak clad temperature or peak system pressure.
Examining only engineering parameters of interest is a weakness.
This locality means that the entire solution domain is not being considered.
Depending upon the context in which the work is being done, the degree of ``small enough" may be nothing more than looking at a graph of the parameter of interest and using engineering judgment to say that ``those two graphs look about the same."
In some cases, a more quantifiable measure may be used.
An example of a quantifiable metric would be if two simulations with different \dtmax{} are classified as dissimilar if the two solutions produce a ``calculated peak fuel cladding temperature different by more than $50\,^{\circ}\mathrm{F}$" \cite{CFR10}.

While it may be that the change in the chosen parameters of interest does not exceed the limits placed upon it as the timestep size is refined, that behavior does not imply that the solution obtained is the solution to the discrete nonlinear problem.
A metric that can quantify the degree to which the obtained solution satisfies the nonlinear system of equations would be of great value.
The previously mentioned work into nonlinear convergence shows that a solution may be timestep-size insensitive but not be the converged solution of the discretized problem \cite{Knoll2001}.
If the nonlinearities of the discrete governing equations are not resolved, then the temporal convergence rate can be degraded.
This degradation can produce results that qualitatively appear to be converged due to an almost zeroth order of temporal accuracy.
In practice, the timestep-size insensitivity of a solution is often interpreted as temporal convergence.
However, this perceived temporal convergence of the solution may not be a result of reaching the solution to the discretized nonlinear equations, but instead could be indicative of the degraded order of accuracy due to the failure to resolve the nonlinearities at each timestep.
To determine if the timestep-size insensitive transient solution is both timestep-size insensitive and an accurate solution to the nonlinear problem, it is necessary to examine the nonlinear convergence of the system as an issue separate from the temporal-convergence.

The norm of the scaled residual from \sect{sect:nlnScaling} provides a well-scaled metric for instantaneous nonlinear convergence at any given time in the simulation.
The residual vector norm is divided by the number of equations in the residual to provide an average residual value per equation.
This equation-averaged, scaled residual provides a metric for determining the degree of nonlinear convergence at any timestep in the simulation.
The natural extension of this metric to transient problems would be a temporal integral, \eqref{eqn:metricResidualIntegral}, of said norm.

\begin{equation}
\label{eqn:metricResidualIntegral}
R = \int_{t^{0}}^{t^{N}} ||\tilde{\vec{F}}(\tau)||_2 \,\mathrm{d} \tau
\end{equation}

Given the bounds of the scaled residual it was considered desirable to have a similarly scaled transient residual.
The transient residual in \eqref{eqn:metricResidualIntegral} possesses a dependence upon the number of timesteps taken.
To remove this dependence, a temporal average was instead investigated, \eqref{eqn:metricResidualAverage}.

\begin{equation}
\label{eqn:metricResidualAverage}
\tilde{R} = \frac{\int_{t^{0}}^{t^{N}} ||\tilde{\vec{F}}(\tau)||_2 \,\mathrm{d} \tau}{t^{N} - t^{0}}
\end{equation}

This metric possesses the desirable bounds $0 \leq R \leq 1$.
Other weighted temporal integrals were considered, such as a simple moment about $t^{0}$, \eqref{eqn:metricResidualMoment}.
The metric in \eqref{eqn:metricResidualMoment} has the disadvantage of arbitrarily setting the moment about the beginning of the transient, resulting in the end of the simulation being more important.
Since different simulations may different transient behavior, fixing the moment arm at $t^{0}$ is too strict.

\begin{equation}
\label{eqn:metricResidualMoment}
\tilde{R}_{\text{M}} = \frac{\int_{t^{0}}^{t^{N}} \,\tau\,||\tilde{\vec{F}}(\tau)||_2 \,\mathrm{d} \tau}{\int_{t^{0}}^{t^{N}} \,\tau \,\mathrm{d} \tau}
\end{equation}



%-------------------------------------------------------------------------------
%-------------------------------------------------------------------------------
%-------------------------------------------------------------------------------
\section{Additional Considerations}
\label{sect:miscConcerns}

There are several aspects of the solution algorithms outlined in \sect{sect:linCobraAlg} and \sect{sect:nlnCobraSolver} that require additional information.
First, the quality control used during development is discussed.
Second, the method for activating the nonlinear solver is provided.
Third, several necessary and practical considerations required to allow the software to simulation of the transition from single-phase to multi-phase flow are outlined.
Lastly, the timestep selection and acceptance criteria are detailed.

%-------------------------------------------------------------------------------
%-------------------------------------------------------------------------------
%-------------------------------------------------------------------------------
\subsection{Quality Control}
\label{subsect:nlnDevelopment}
The development of the nonlinear solver within \cobra{} took place under strict quality assurance guidelines.
These guidelines have been documented in extensive detail \cite{Aumiller2013} and represent the current best practices for thermal-hydraulic safety analysis. 
At every step of the development, the linear solver was required to maintain the same solution.
This verification was dependent upon a larger number of verification and assessment problems.
The output of the unmodified \cobra{} and the modified \cobra{} software was compared to machine precision.
It was required that either the results of the two simulations be identical or that the reason for the difference be identified and understood.
The \cobra{} software has the ability to repeat a timestep.
As such, it was required that the backup capabilities continued to work while nonlinear solver was being implemented.
Additionally, testing was done to ensure that the ability to restart the software mid-simulation was unaffected.
While adding time to the development cycle, the overhead of the quality assurance procedures ensured that the linear solver could continue to be used for design purposes.

%-------------------------------------------------------------------------------
%-------------------------------------------------------------------------------
%-------------------------------------------------------------------------------
\subsection{Nonlinear Input File}
\label{subsect:nlnCobraInputFile}
The primary input file for \cobra{} is fixed format, so it was decided that a separate input file for the nonlinear solver would provide the most flexibility when running simulations.
This decision allowed for existing models to be run in nonlinear mode without modification of the actual input file.
The presence of the file ``\classname{nwt.cob}" triggers the input processing routine for the nonlinear solver.
There are three parameters that need to be present in the input file; in order they are \kmax{},\ftol{}, and \dtol{}, each on a separate line. 
If the file is present but empty, then the default values are used; they are $35$, \expneg{1.0}{5}, and \expneg{1.0}{10}, respectively.

%-------------------------------------------------------------------------------
%-------------------------------------------------------------------------------
%-------------------------------------------------------------------------------
\subsection{Phase Transition}
\label{subsect:nlnPhaseTransition}
Given that the governing conservation laws in thermal-hydraulic analyses are those of two-phase flow, phase transition is integral to accurate simulations.
The ability to model a simulation as it moves from single-phase flow to multi-phase flow and back is the keystone of the software.
However, the methods for addressing phase transitions within two-phase analysis software are ad hoc procedures that vary between implementations.
While every software package addresses this issue out of necessity, there is no agreed upon systematic or optimal way of doing so.
However, there are certain physical limits that have been identified that need to be considered when dealing with phase transitions \cite{Bestion2000}.

When multi-phase flow within \cobra{} approaches that of single-phase flow, there is a lower limit imposed upon the volume fraction of the phase that is disappearing.
This means that the software does not transition between the governing equations for two-phase flow and those for single-phase flow when a phase depletes.
Even during simulated single-phase flow, the non-dominant phase is still present, but it is at a small volume fraction.
However, since the \ncgs{} and the vapor fields share a common volume fraction, the partial pressure of the \ncgs{} is allowed to go to zero. 
Since \cobra{} incorporates two liquid fields, the lower limit for the aggregate liquid volume fraction, $\alpha_e + \alpha_l$, is equally divided between the two liquid fields.
As a phase approaches depletion, $\alpha_k \rightarrow \alpha_{k,\text{min}}$, or starts to appear, several physical limits are imposed upon the depleted phase.

First, the velocity of the depleting phase is required to approach that of the carrier phase.
The following list shows how the velocities are required to behave as their respective volume fractions approach zero.
\begin{itemize}
\item{ $\alpha_e \rightarrow 0$ : $u_e \rightarrow u_g$}
\item{ $\alpha_g \rightarrow 0$ : $u_g \rightarrow u_l$}
\item{ $\alpha_l \rightarrow 0$ : $u_l \rightarrow u_g$}
\end{itemize}
These velocity equilibrium constraints are imposed by artificially increasing the interfacial drag between the two phases to force equilibrium between the new-time velocities.
However, there is a subtle issue created by this increase in the the interfacial drag.
If the new-time velocities are not equal, then the increased interfacial drag will dominate the residual for the momentum equation of the phase that is not disappearing.
For example, \tab{tab:gasMomentumResidual} contains the magnitudes of the various terms in the gaseous momentum equation from a nominally single-phase gaseous flow simulation.

\begin{table}[ht]
\centering
\singlespace
\pgfplotstabletypeset[sci zerofill, sci E, precision=2, col sep=&,row sep=\\,
	columns/0/.style={ column name=$\Delta m_{g}$},
	columns/1/.style={ column name=$\ave{\alpha_g}^{n}_{a}\nabla P$},
	columns/2/.style={ column name=$\tau_{g,l}$},
	columns/3/.style={ column name=$\tau_{g,e}$},
	columns/4/.style={ column name=$\tau_{w}$},
	columns/5/.style={ column name=$\nabla \cdot \left(\alpha_{g} \rho_{g} u_{g} u_{g}\right)$},
	columns/6/.style={ column name=$g \ave{\alpha_g \rho_g}^{n}_{a}$},
	every head row/.style={
		before row=\toprule,
		after row=\midrule
	},
	every last row/.style={
		after row=\bottomrule}]{
3.7616122e-02 & 1.238438e-01 & -2.71654356e-5 & -3.21746 & -7.22740678e-3 & -4.21611e-02 & -1.14449452e-1 \\
}
\caption{Relative magnitude of terms in a gaseous momentum equation.}
\label{tab:gasMomentumResidual}
\end{table}

The term of interest is the interfacial shear, $\tau_{g,e}$, between the gaseous phase and the entrained liquid phase.
This term is an order of magnitude greater than the other terms.
This is due to the combination of the artificially increased interfacial drag coefficient and definition of velocities in \cobra{}.
Since \cobra{} uses conservation of momentum equations with momentum as an independent parameter in a semi-implicit discretization, the velocities are poorly defined.
Compare the old-time velocity, as defined by \eqref{eqn:oldTimeVelocity}, with the new-time velocity as defined by \eqref{eqn:si_vel}.

\begin{equation}
\label{eqn:oldTimeVelocity}
u^{n}_{\phi, j \pm \onehalf} = \frac{\dot{m}^{n}_{\phi, j \pm \onehalf}}{A_{m, j \pm \onehalf} \ave{\alpha_{\phi} \rho_{\phi}}^{n}_{\text{a}, j \pm \onehalf}} 
\end{equation}

Since the new-time velocity is what is forced to be equal by the interfacial drag, at the beginning of the next timestep, the old-time velocity is not equal to the new-time velocity at the end of the previous timestep.
The discrepancy in the definition of the velocities, \eqref{eqn:compareVelocities}, created a non-zero relative velocity between the depleted phase and the present phase at the beginning of every timestep.

\begin{equation}
\label{eqn:compareVelocities}
\frac{\dot{m}^{n+1}_{\phi, j \pm \onehalf}}{A_{m, j \pm \onehalf} \ave{\alpha_{\phi} \rho_{\phi}}^{n}_{\text{a}, j \pm \onehalf}} \neq \frac{\dot{m}^{n+1}_{\phi, j \pm \onehalf}}{A_{m, j \pm \onehalf} \ave{\alpha_{\phi} \rho_{\phi}}^{n+1}_{\text{a}, j \pm \onehalf}}
\end{equation}

For the linear solver, this was acceptable; however, the iterative nature of the nonlinear solver exacerbated this problem.
This created a situation where multiple Newton steps might have been required to drive the new-time relative velocity to zero even though the previous timestep's new-time relative velocity was zero.
To remove this problem, if the new-time relative velocity of the previous timestep is below a certain threshold, $u_{r, min}$, then the momentum of the depleted phase, $\dot{m}_{d}$, is relinearized so that the new-time relative velocity is zero.
The linearization point is equal to the carrier momentum, $\dot{m}_{c}$ multiplied by the ratio depleted phase's to the carrier phase's macroscopic densities, \eqref{eqn:velRelinearization}.

\begin{equation}
\label{eqn:velRelinearization}
\dot{m}^{n+1}_{d, j \pm \onehalf} = \dot{m}^{n+1}_{c, j \pm \onehalf} \frac{\ave{\alpha_{d} \rho_{d}}^{n+1}_{\text{a}, j \pm \onehalf}}{\ave{\alpha_{c} \rho_{c}}^{n+1}_{\text{a}, j \pm \onehalf}}
\end{equation}

The second restriction is placed upon the thermodynamic state of the depleting phase.
Below a certain threshold, the thermodynamic state of the depleted phase is set to be equal to the saturation properties associated with the thermodynamic state of carrier phase.
Additionally, this issue applied to the presence or absence of the \ncg{} field.
Within \cobra{}, only the \ncg{} field is capable of being fully depleted; the partial pressure of the \ncg{} field is allowed to go to zero.
To determine if the initial linear point for $P^{n+1}_{n}$ is appropriate, information from the explicit portion of the continuity equations is used to identify situations where the linearization point may be poor.

\comment{Insert concluding paragraph.}

%-------------------------------------------------------------------------------
%-------------------------------------------------------------------------------
%-------------------------------------------------------------------------------
\subsection{Timestep Selection and Acceptance}
\label{subsect:nlnTimesteps}

Upon completion of a timestep within \cobra{}, there are certain constraints that are imposed upon the independent parameters.
These constraints are designed to deal with the possibility that the obtained solution may not be an accurate one.
After the single Newton step, the updated parameters are evaluated to determine their validity.
There are two ways of resolving potentially invalid solutions: parameter limiting and timestep failure.
Stated another way, the limiting procedure can either truncate the updated parameter so that it falls within a valid range, or the timestep can be considered a failure.
When a predicted volume fraction falls outside of its valid range, it is truncated to obey the constraint of equation \eqref{eqn:volume_fraction}.

\begin{equation}
\label{eqn:volume_fraction}
\alpha_{\phi,\text{min}} \leq \alpha_{\phi} \leq \alpha_{\phi,\text{max}} 
\end{equation}

There are also limits placed upon the changes of the thermodynamic parameters within a timestep in \cobra{}.
The constrained thermodynamic parameters and the limits imposed upon their per-timestep changes are listed below.

\begin{itemize}
\item{Change of phasic enthalpy cannot be greater than $45$ [$\frac{\text{BTU}}{\lbm{}}$].}
\item{Change in pressure cannot be greater than $20$ [psia].}
\item{Change in partial pressure of the \ncg{} field cannot be greater than $20$ [psia].}
\end{itemize}

These three limits are an attempt to mitigate an initial guess that may be outside of the Newton step's radius of convergence.
If any of the above limits are exceeded, the timestep is considered a failure and is repeated with a smaller timestep size.

To determine the \dt{} at each timestep \cobra{} uses an adaptive timestep selection algorithm.
This algorithm utilizes the maximum permissible timestep calculated from the material Courant limit, $\dtcrnt{}$.
The \dt{} for any timestep is given by \eqref{eqn:time_step}.

\begin{equation}
\label{eqn:time_step}
\dt{}^{n \rightarrow n+1} = \max\left[ \dt{}_{\text{MIN}}, \min\left[1.2 \dt{}^{n-1 \rightarrow n}, 0.85 \dtcrnt{}, \dtmax{}, \dt{}_{\text{vol}} \right]\right]
\end{equation}

\dtcrnt{} is the most restrictive timestep size as calculated by used the Courant number for both axial and transverse flows, and $\dt{}_{\text{vol}}$ is the most restrictive timestep based upon an estimation of the time required to empty a given continuity volume of all of the mass from any given field.
The most restrictive timestep is used as $\dt{}^{n \rightarrow n+1}$.
 % Chapter: Nonlinear Solver
\chapter{Domain Decomposition}
\label{chap:domDecomposition}
As discussed in \sect{sect:code_coupling}, domain coupling in thermal-hydraulic, safety-analysis software is an established method to allow for different physics to be represented in different domains.
This research extends the domain coupling framework to allow for different domains to be consistently coupled while subject to different mathematical treatments.
The following chapter describes the mathematical formulation of this novel domain decomposition algorithm, several architectural changes that were made to \cobra{} to allow for the algorithm to be implemented properly, and a description of its implementation in the software.

%-------------------------------------------------------------------------------
%-------------------------------------------------------------------------------
%-------------------------------------------------------------------------------
\section{Mathematical Formulation}
\label{sec:domDecompMath}

The governing partial differential equations described in \sect{subsect:governing_equations} contain nonlinearities that are distributed in both space and time.
In the absence of nonlinear physics a single Newton step is adequate to solve the governing set of discrete algebraic equations.
However, if the number of Newton steps is limited to one for a given spatial mesh when nonlinearities are present, then the only way to resolve those nonlinearities is to refine the temporal discretization.
If the number of Newton steps is not limited to one, then the nonlinearities may be resolved for a fixed temporal discretization.
When the nonlinearities are isolated to a given spatial portion of the domain, the additional Newton steps do not improve the solution in the remainder of the domain.
The ability to resolve the nonlinear physics only where they occur could potentially provide a way to reduce the computational cost associated with resolving the nonlinear error for a given timestep.

With the selective nonlinear refinement algorithm, the domain is split into two segments, a linear domain and a nonlinear domain.
There is no requirement that these two domains be contiguous.
Each continuity volume and momentum flow path is characterized as either existing in the linear domain or the nonlinear domain.
Each momentum flow path is associated with a single domain; the momentum equations are not modified, regardless of the domain in which they exists.
The coupling between the two domains occurs at the flow paths connecting the continuity volumes of the linear and the nonlinear domains.
Those continuity volumes which are in the linear domain, but have flow paths connecting them to the nonlinear domain, are considered to be nonlinear boundary continuity (NBC) volumes.
All other continuity volumes have associated with them the nonlinear continuity equations, \eqref{eqn:nlnNcgMassEquation} -- \eqref{eqn:nlnVapMassEquation}.
The NBC volumes have a modified set of conservation equations which will now be discussed.

The starting place for the formulation of the NBC volumes' continuity equations will be \eqref{eqn:nlnNcgMassEquation} -- \eqref{eqn:nlnVapMassEquation}.
In these normal nonlinear continuity equations the advection terms are formulated in terms of new-time velocities and donored quantities evaluated using the new-time velocities.
However, in an NBC volume, the flow paths that connect linear continuity volumes to nonlinear continuity volumes are formulated in terms of phasic mass and energy flow rates, \eqref{eqn:massAndEnergyFluxes}.
The index $i$ represents the spatial coordinate of the flow rate.


\begin{equation}
\label{eqn:massAndEnergyFluxes}
\vec{\Psi}^{n+1, k}_{i} = \begin{bmatrix}
\Psi_{c, n} \\
\Psi_{c, l} \\
\Psi_{e, g} \\
\Psi_{e, l} \\
\Psi_{c, e} \\
\Psi_{c, v} \\
\end{bmatrix}^{n+1, k}_{i}
\end{equation}

This formulation introduces six new unknowns per domain connection, the flow rates.
The definition of the six unknown flow rates for a given location are shown in \eqref{eqn:nbcFluxDefinition}.

\begin{equation}
\label{eqn:nbcFluxDefinition}
\vec{\Psi}^{n+1, k}_{i} = \begin{bmatrix}
\frac{\don{\alpha^{n}_{g} \rho^{n}_{n}}^{n+1,k}_{d}}{\ave{\alpha_{g} \rho_{g}}^{n}_{a}}\dot{m}^{n+1,k}_{g} \\
%
\frac{\don{\alpha^{n}_{l}\rho^{n}_{l}}^{n+1,k}_{d}}{\ave{\alpha_{l} \rho_{l}}^{n}_{a}}\dot{m}^{n+1,k}_{l}\\
%
\frac{\don{\alpha^{n}_{g} \rho^{n}_{g} h^{n}_{g}}^{n+1,k}_{d}}{\ave{\alpha_{g} \rho_{g}}^{n}_{a}}\dot{m}^{n+1, k}_{g}\\
%
\frac{\don{\alpha^{n}_{l}\rho^{n}_{l} h^{n}_{l}}^{n+1,k}_{d}}{\ave{\alpha_{l} \rho_{l}}^{n}_{a}}\dot{m}^{n+1, k}_{l} +\frac{\don{\alpha^{n}_{e} \rho^{n}_{l} h^{n}_{l}}^{n+1, k}_{d}}{\ave{\alpha_{e} \rho_{l}}^{n}_{a}}\dot{m}^{n+1, k}_{e} \\
%
\frac{ \don{\alpha^{n}_{e} \rho^{n}_{l}}^{n+1, k}_{d}}{ \ave{\alpha_{e} \rho_{l}}^{n}_{a}} \dot{m}^{n+1, k}_{e} \\
%
\frac{ \don{\alpha^{n}_{g} \rho^{n}_{v}}^{n+1, k}_{d}}{ \ave{\alpha_{g} \rho_{g}}^{n}_{a}} \dot{m}^{n+1, k}_{g}
\end{bmatrix}_{i}
\end{equation}

Using \eqref{eqn:momentumToFlowRates} and \eqref{eqn:momentumVector}, the flow rate vector, $\vec{\Psi}$, can be expressed in matrix form as \eqref{eqn:nbcFluxMatrixStyle}.

\begin{equation}
\label{eqn:nbcFluxMatrixStyle}
\vec{\Psi}^{n+1, k}_{i} = \vec{\Xi}^{n+1, k}_{i} \cdot \vec{\dot{m}}^{n+1, k}_{i}
\end{equation}

The corresponding continuity equations for a given NBC volume in residual formulation are given by \eqref{eqn:nbcNcgMassEquation} -- \eqref{eqn:nbcVapMassEquation}.
In these equations, the set of flow paths connecting an NBC volume to other continuity volumes within the linear domain is given by $N_{f}$.
The index $i$ corresponds to the coordinates of the flow paths in $N_{f}$.
The set of flow paths connecting an NBC volume to the nonlinear domain is given by $N_{\text{NBC}}$.
The index $p$ corresponds to the coordinate of a given flow path in the set of flow paths $N_{\text{NBC}}$.

\begin{IEEEeqnarray}{rCl}
\label{eqn:nbcNcgMassEquation}
F^{k}_{c, n} & = & V_c\left[ (\alpha_g \rho_{n})^{n+1, k} -(\alpha_g \rho_{n})^{n}\right] +\dt{} \sum_{i\,\in \, N_{f} }\left( \don{ \alpha^{n}_g \rho^{n}_{n} }^{n+1,k}_{d} u^{n+1, k}_{g}  A_{m} \right)_{i} \nonumber \\
& + & \dt{} \sum_{p\, \in \, N_{\text{NBC}}} \left( \Psi^{n+1, k}_{c,n} \right)_{p} \\
\label{eqn:nbcLiqMassEquation}
F^{k}_{c, l} & = & V_c \left(\alpha_l \rho_l \right)^{n+1,k} - V_c \left(\alpha_l \rho_l \right)^{n} + \dt{} \sum_{i\,\in \,N_{f}} \left(\don{\alpha^n_l \rho^n_l}^{n+1,k}_{d} u^{n+1, k}_l A_{m}  \right)_{i} \nonumber \\
&+& \left[(1-\eta)\Gamma + \Upsilon \right]^{n+1, k} +  \dt{} \sum_{p\,\in \, N_{\text{NBC}} } \left( \Psi^{n+1, k}_{c,l} \right)_{p}  \\
\label{eqn:nbcGasEnergyEquation}
F^{k}_{e, g} & = & V_c \left[\left( \alpha_g \rho_g h_g \right)^{n+1, k} - \left( \alpha_g \rho_g h_g \right)^{n} - \alpha^{n}_{g} ( P^{\,n+1, k} - P^{\,n} ) \right] - \dt{} q_{wg}^{n} \nonumber \\
& - & \dt{} \left[q_{i,v} + \dot{\Gamma} h^{'}_v + q_{gl}\right]^{n+1, k} + \dt{} \sum_{i \, \in \, N_{f} } \left( \don{ \alpha^{n}_g \rho^{n}_g h_g^{n} }^{n+1,k}_{d} u^{n+1, k}_g  A_{m}  \right)_{i} \nonumber \\
& + &  \dt{} \sum_{p\, \in \, N_{\text{NBC}} } \left( \Psi^{n+1, k}_{e,g} \right)_{p} \\
\label{eqn:nbcLiqEnergyEquation}
F^{k}_{e, l} & = & V_c\left[\left( \alpha_l \rho_l h_l \right)^{n+1,k} - \left( \alpha_l \rho_l h_l \right)^{n} - \alpha^{n}_l (P^{\,n+1,k} - P^{\,n})\right] - \dt{} \left[q_{i,l} -\dot{\Gamma} h^{'}_l - q_{gl}\right]^{n+1,k}    \nonumber \\
& +& \dt{} \sum_{i\,\in \, N_{f} } \left( \don{ \alpha^{n}_l \rho^{n}_l h^{n}_l }^{n+1,k}_{d} u^{n+1,k}_l A_{m} + \don{ \alpha^{n}_e \rho^{n}_l h^{n}_l }^{n+1,k}_{d} u^{n+1,k}_e A_{m} \right)_{i} \nonumber \\
& + &  \dt{} \sum_{p \,\in \, N_{\text{NBC}} } \left( \Psi^{n+1, k}_{e,l} \right)_{p} \\
\label{eqn:nbcEntMassEquation}
F^{k}_{c, e} & = & V_c \left(\alpha_e \rho_l \right)^{n+1,k} - V_c \left(\alpha_e \rho_l \right)^{n} + \dt{} \sum_{i \, \in \, N_{f} } \left( \don{ \alpha^{n}_e \rho^{n}_l }^{n+1, k}_{d} u^{n+1,k}_e  A_{m} \right)_{i} \nonumber \\
&-& \left[\Upsilon -\eta\Gamma\right]^{n+1,k} +  \dt{} \sum_{p \, \in \, N_{\text{NBC}} } \left( \Psi^{n+1, k}_{c,e} \right)_{p}\\
\label{eqn:nbcVapMassEquation}
F^{k}_{c, v} & = & V_c \left[\left(\alpha_g \rho_v \right)^{n+1, k} - \left(\alpha_g \rho_v \right)^{n}\right] + \dt{} \sum_{i\, \in \, N_{f} } \left( \don{ \alpha^{n}_g \rho^{n}_v }^{n+1,k}_{d} u^{n+1, k}_{g}  A_{m} \right)_{i} - \Gamma^{n+1, k} \nonumber \\
& + & \dt{} \sum_{p \, \in \, N_{\text{NBC}} } \left( \Psi^{n+1, k}_{c,v} \right)_{p}
\end{IEEEeqnarray}

By treating the boundary flow rates as independent parameters, the linearized system given in \eqref{eqn:nlnContinuitySystem} is modified to be \eqref{eqn:nbcContinuitySystem}.

\begin{IEEEeqnarray}{rcl}
\label{eqn:nbcContinuitySystem}
\frac{\partial \vec{F}^{k}_{c}}{\partial (\alpha_{g} P_{n} )} \delta (\alpha_{g} P_{n})^{k} + \frac{\partial \vec{F}^{k}_{c}}{\partial \alpha_{g}} \delta \alpha^{k}_{g} + \frac{\partial \vec{F}^{k}_{c}}{\partial (\alpha_{g} h_{v} )} \delta (\alpha_{g} h_{v})^{k} + \frac{\partial \vec{F}^{k}_{c}}{\partial ((1 - \alpha_{g}) h_{l} )} \delta ((1 - \alpha_{g}) h_{l})^{k} & + &  \nonumber \\
\frac{\partial \vec{F}^{k}_{c}}{\partial \alpha_{e}} \delta \alpha_{e}^{k} + \frac{\partial \vec{F}^{k}_{c}}{\partial P } \delta P^{k} + \sum_{i \, \in \, N_{f} } \frac{\partial \vec{F}^{k}_{c}}{\partial \momVec{}_{i} } \delta \momVec{}_{i}^{k} + \sum_{p\,\in \, N_{\text{NBC}}} \frac{\partial \vec{F}^{k}_{c}}{\partial \vec{\Psi}_{p} } \delta \vec{\Psi}_{p}^{k}  = - \vec{F}^{k}_{c} & & 
\end{IEEEeqnarray}

Since the continuity equations are linear in the boundary flow rates, the matrix of derivatives of the NBC volumes' continuity equations with respect to the boundary flow rates, $ \frac{\partial \vec{F}^{k}_{c}}{\partial \vec{\Psi}_{p} } $, is the identity matrix times the timestep size, $\dt{} \vec{I}$.
For algorithmic convenience, the unknown flow rates in the continuity equations' residuals are kept as unknown parameters when solving the local linear system.
However, the residual used for convergence does include this term.
Utilizing \eqref{eqn:momentumToFlowRates} and \eqref{eqn:momentumUpdate}, the linear system for the NBC volume, \eqref{eqn:nbcContinuitySystem}, can be expressed as \eqref{eqn:nbcLinearSystem} in a manner analogous to that outlined in \sect{sect:nlnCobraSolver}.

\begin{IEEEeqnarray}{rcl}
\label{eqn:nbcLinearSystem}
\frac{\partial \vec{F}^{k}_{c}}{\partial (\alpha_{g} P_{n} )} \delta (\alpha_{g} P_{n})^{k} + \frac{\partial \vec{F}^{k}_{c}}{\partial \alpha_{g}} \delta \alpha^{k}_{g} + \frac{\partial \vec{F}^{k}_{c}}{\partial (\alpha_{g} h_{v} )} \delta (\alpha_{g} h_{v})^{k} + \frac{\partial \vec{F}^{k}_{c}}{\partial ((1 - \alpha_{g}) h_{l} )} \delta ((1 - \alpha_{g}) h_{l})^{k} & + & \nonumber \\
\frac{\partial \vec{F}^{k}_{c}}{\partial \alpha_{e}} \delta \alpha_{e}^{k} + \left( \frac{\partial \vec{F}^{k}_{c}}{\partial P } + \sum_{i \, \in \, N_{f} } \vec{\Xi}^{k}_{i}\frac{\partial \momVec{}^{k}_{i}}{\partial P}\right) \delta P^{k} + \sum_{i \, \in \, N_{f} } \vec{\Xi}^{k}_{i} \frac{\partial \momVec{}_{i}^{k}}{\partial P_{o(i)}} \delta P_{o(i)}^{k} + \dt{} \sum_{p \, \in \, N_{\text{NBC}}} \delta \vec{\Psi}^{k}_{p} & = &\nonumber \\
- \left( \vec{F}^{k}_{c} + \sum_{i \, \in \, N_{f} } \vec{\Xi}^{k}_{i} \delta \momVec{}_{i}^{*} - \dt{} \sum_{p \, \in \, N_{\text{NBC}}} \vec{\Psi}^{k}_{p} \right) - \dt{} \sum_{p \, \in \, N_{\text{NBC}}} \vec{\Psi}^{k}_{p} & &
\end{IEEEeqnarray}

As outlined in \sect{sect:nlnCobraSolver}, the advection terms in the residuals on the right-hand side of \eqref{eqn:nbcLinearSystem} are evaluated using $\momVec{}^{n+1, k + \onehalf}$.
The first six columns of \eqref{eqn:nbcLinearSystem} will be represented by $\vec{J}_{n}$.
The next $N_{f}$ inter-continuity coupling columns multipiying their respective pressure changes, $\delta P_{o(i)}$, will be represented by $\vec{K}_{n}$.
The final $6 * N_{\text{NBC}}$ inter-domain coupling coefficient columns multiplied by the change in flow rates, $\delta \vec{\Psi}_{p}$, will be collectively referred to as $\vec{Q}_{n}$.
This matrix, $\vec{Q}_{n}$, is also present with the opposite sign on the right-hand side of \eqref{eqn:nbcLinearSystem}.
Using this matrix notation, \eqref{eqn:nbcLinearSystem} can be represented as \eqref{eqn:nbcLinSystem}.

\begin{equation}
\label{eqn:nbcLinSystem}
\left[ \vec{J}_{n} \vert \vec{K}_{n} \vert \vec{Q}_{n} \right] \delta \vec{C}_{n} = \vec{r}_{n} - \vec{Q}_{n} \vec{\Psi}_{n}
\end{equation}

The vector of unknowns, $\vec{C}_{n}$ is defined in \eqref{eqn:nbcUpdate}.

\begin{equation}
\label{eqn:nbcUpdate}
\delta \vec{C}_{n} \equiv 
\begin{bmatrix}
\delta ( \alpha_{g} P_{n} ) \\
\delta \alpha_{g} \\
\delta ( \alpha_{g} h_v ) \\
\delta ( (1 - \alpha_{g} ) h_l ) \\
\delta \alpha_{e} \\
\delta P \\ 
\delta P_i \\
\vdots \\
\delta P_{N_{f}} \\
\delta \vec{\Psi}_{p} \\
\vdots \\
\delta \vec{\Psi}_{N_{\text{NBC}}}
\end{bmatrix}
=
\begin{bmatrix}
( \alpha_{g} P_{n})^{n+1} - (\alpha_{g} P_{g} )^{n} \\
\alpha^{n+1}_{g} - \alpha^{n}_{g} \\
( \alpha_{g} h_{v} )^{n+1} - ( \alpha_{g} h_{v} )^{n} \\
( ( 1 - \alpha_{g} ) h_{l} )^{n+1} - ( ( 1 - \alpha_{g} ) h_{l} )^{n} \\
\alpha^{n+1}_{e,j} - \alpha^{n}_{e} \\
 P^{n+1} - P^{n} \\
 P_{o(i)}^{n+1} - P_{o(i)}^{n} \\
 \vdots \\
 P_{o(N_{f})}^{n+1} - P_{o(N_{f})}^{n} \\
 \vec{\Psi}_{p}^{n+1} - \vec{\Psi}_{p}^{n} \\
 \vdots \\
 \vec{\Psi}_{N_{\text{NBC}}}^{n+1} - \vec{\Psi}_{N_{\text{NBC}}}^{n}
\end{bmatrix}
\end{equation}

\eqref{eqn:nbcLinSystem} is then subjected to partial $\vec{LU}$ decomposition without pivoting, producing \eqref{eqn:nbcLUSystem}.

\begin{equation}
\label{eqn:nbcLUSystem}
\left[ \vec{U}_{n} \vert \vec{L}^{-1}_{n}\vec{K}_{n} \vert \vec{L}^{-1}_{n}\vec{Q}_{n} \right] \delta \vec{C}_{j} = \vec{L}^{-1}_{n}\vec{r}_{n}  -\vec{L}^{-1}_{n}\vec{Q}_{n}\vec{\Psi}_{n}
\end{equation}

However, since the matrix product $\vec{L}_{n}^{-1}\vec{Q}_{n}$ occurs twice in the resulting system, only a single instance is included in the solution of this system.
The $\vec{Q}_{n}$ is omitted from the right-hand side of \eqref{eqn:nbcLinSystem} during the $\vec{LU}$ decomposition.
This $\vec{LU}$ decomposition is such that the lower triangular matrix has ones along the diagonal.
The $N_{\text{NBC}}$ identity matrices that comprise $\vec{Q}_{n}$ are now themselves the inverse of the lower triangular matrix times \dt{}.
The final row of the linear system in \eqref{eqn:nbcLUSystem} is then scaled by the sixth row's sixth column entry, $\vec{U}_{n}[6,6]$.

The linear pressure matrix, $\vec{A}_{\text{lin}}$, is assembled from the sixth row of each continuity volume's linear system in the linear domain including the NBC volumes.
There will be $N_{\text{lin}}$ rows in the system that correspond to each of the linear pressure updates.
Let $N_{n}$ represent the total number of interfaces between the linear and the nonlinear domains.
Since there can be multiple interfaces from a single NBC volume to the nonlinear domain, the $N_{n}$ is greater than or equal to the number of NBC volumes.
There will be an additional $6 * N_{n}$ right-hand sides for the linear pressure matrix that correspond to the coefficient matrices, $\vec{L}^{-1}_{n}\vec{Q}_{n}$, for the unknown flow rates in each of the NBC volumes, represented as $\vec{B}_{\text{lin}}$.
The portion of the right-hand side of the linear system that is from $\vec{L}^{-1}_{n} \vec{r}_{n}$ will be referred to as $\vec{res}_{\text{lin}}$.
The resulting system is shown in \eqref{eqn:linearPressureMatrix}.

\begin{equation}
\label{eqn:linearPressureMatrix}
\vec{A}_{\text{lin}} \delta \vec{P}_{\text{lin}} + \vec{B}_{\text{lin}} \delta \vec{\Psi}_{\text{lin}} = \vec{res}_{\text{lin}} - \vec{B}_{\text{lin}} \vec{\Psi}_{\text{lin}}
\end{equation}

The linear system in \eqref{eqn:linearPressureMatrix} will be multiplied by $\vec{A}^{-1}_{\text{lin}}$, resulting in \eqref{eqn:invertedLinearSystem}.

\begin{IEEEeqnarray}{rcl}
\delta \vec{P}_{\text{lin}} + \vec{A}^{-1}_{\text{lin}}\vec{B}_{\text{lin}} \delta \vec{\Psi}_{\text{lin}} & = & \vec{A}^{-1}_{\text{lin}}\vec{res}_{\text{lin}} - \vec{A}^{-1}_{\text{lin}}\vec{B}_{\text{lin}} \vec{\Psi}_{\text{lin}} \nonumber \\
\label{eqn:invertedLinearSystem}
\delta \vec{P}_{\text{lin}} + \vec{W}_{\text{lin}} \delta \vec{\Psi}_{\text{lin}} & = & \delta \vec{P}^{*}_{\text{lin}} - \vec{W}_{\text{lin}} \vec{\Psi}_{\text{lin}}
\end{IEEEeqnarray}

Once \eqref{eqn:linearPressureMatrix} has been inverted, the resulting pressure updates, $\delta \vec{P}_{l}$, for the linear domain are functions of all of the unknown flow rates between the two domains and their updates.
The constant portion of \eqref{eqn:invertedLinearSystem} is shown as $\delta \vec{P}^{*}$, representing the portion of the change in $\delta \vec{P}$ that comes from the linear domain.
A representative row of \eqref{eqn:invertedLinearSystem} is shown in \eqref{eqn:linearDomainPressureUpdates}.
The vectors $\vec{w}_{j, i}$ represent the coupling coefficients for each NBC volume's pressure update.

\begin{equation}
\label{eqn:linearDomainPressureUpdates}
\delta P_{j} + \sum_{i\, \in \, N_{n}} \vec{w}^{T}_{j, i} \cdot \delta \vec{\Psi}_{i} = \delta P_{j}^{*} - \sum_{i\, \in \, N_{n}} \vec{w}^{T}_{j, i} \cdot{} \vec{\Psi}^{k}_{i}
\end{equation}

In the linear domain a single Newton step is taken to be the change from the old-time solution to the new-time solution, not an iterate.
This approach means that \eqref{eqn:linearDomainPressureUpdates} is interpreted not as an iterative solution but as a change from old-time to new-time values.
As a result, the pressure update is taken to be $\delta P_{j} = P_{j}^{n+1} - P_{j}^{n}$, the flow rates on the right-hand side are evaluated as $\vec{\Psi}^{n}$, and the flow rate updates are defined as $\delta \vec{\Psi} = \vec{\Psi}^{n+1} - \vec{\Psi}^{n}$.

Once the linear system has been inverted, the nonlinear pressure matrix is finalized.
Recall that the nonlinear continuity volumes' governing equations are not modified by the domain decomposition.
The nonlinear domain is composed of $N_{\text{nln}}$ continuity volumes.
Additionally, for each NBC volume there will be an extra row in the nonlinear pressure matrix, $\vec{A}_{n}$, that corresponds to \eqref{eqn:linearDomainPressureUpdates}.
However, within the nonlinear domain, the flow rates are not kept as unknowns; instead, the flow rates in \eqref{eqn:linearDomainPressureUpdates} are expressed in terms unknown pressures by using \eqref{eqn:momentumUpdate}, \eqref{eqn:nbcFluxMatrixStyle}, and \eqref{eqn:nbcFluxDerivative}.

\begin{equation}
\label{eqn:nbcFluxDerivative}
\delta \vec{\Psi} = \vec{\Xi}^{k} \cdot \delta \momVec{}
\end{equation}

The resulting equations for the pressure updates of the NBC volumes as viewed by the nonlinear domain are given by \eqref{eqn:nbcPressureEquation} 

\begin{IEEEeqnarray}{rcl}
\delta P_{j} + \sum_{i\,\in \, N_{n}} \vec{w}^{T}_{j,i} \vec{\Xi}^{k}_{i} \delta \vec{m}^{k}_{i} & = & \delta P^{*}_{j} - \sum_{i\,\in \, N_{n}} \vec{w}^{T}_{j,i} \vec{\Xi}_{i}^{k} \momVec{}^{k}_{i}  \nonumber \\
\delta P_{j} + \sum_{i\,\in \, N_{n}} \vec{w}^{T}_{j,i} \left[\vec{\Xi}_{i}^{k} \frac{\partial \momVec{}_{i}^{k}}{\partial P_{s(i)}} \delta P_{s(i)} + \vec{\Xi}_{i}^{k} \frac{\partial \momVec{}_{i}^{k}}{\partial P_{o(i)}} \delta P_{o(i)}\right] & = & \delta P^{*}_{j} - \sum_{i\,\in \, N_{n}} \vec{w}^{T}_{j,i} \left[ \vec{\Xi}_{i}^{k} \momVec{}^{k}_{i} + \vec{\Xi}_{i}^{k}\delta \momVec{}^{*}_{i} \right] \nonumber \\
\label{eqn:nbcPressureEquation}
\delta P_{j} + \sum_{i\,\in \, N_{n}} \vec{w}^{T}_{j,i} \left[\vec{\Xi}_{i}^{k} \frac{\partial \momVec{}_{i}^{k}}{\partial P_{s(i)}} \delta P_{s(i)} + \vec{\Xi}_{i}^{k} \frac{\partial \momVec{}_{i}^{k}}{\partial P_{o(i)}} \delta P_{o(i)}\right] & = & \delta P^{*}_{j} - \sum_{i\,\in \, N_{n}} \vec{w}^{T}_{j,i} \vec{\Xi}_{i}^{k} \momVec{}^{k + \onehalf}_{i}
\end{IEEEeqnarray}

The flux terms on the right-hand side of \eqref{eqn:nbcPressureEquation} are evaluated using $\momVec{}^{n+1,k+\onehalf}$.
The subscript $s(i)$ represents the NBC volume to which the flow path $i$ is connected.
The subscript $o(i)$ represents the nonlinear continuity volume to which the flow path $i$ is connected.
\eqref{eqn:nbcPressureEquation} shows that every NBC volume has a functional dependence upon the change in pressure of not only every other NBC volume but also every nonlinear continuity volume that is connected to an NBC volume.  
The final nonlinear pressure matrix will have $N_{\text{nln}} + N_{n}$ equations and unknowns and is represented by \eqref{eqn:nlnPressureMatrix}.

\begin{equation}
\label{eqn:nlnPressureMatrix}
\vec{A}^{k}_{n} \delta \vec{P}^{k}_{n} = \vec{res}^{k}_{n}
\end{equation}

This system is then solved to obtain the pressure updates for the nonlinear domain.
The continuity volumes and flow paths in the nonlinear domain are updated using $\delta \vec{P}^{k}_{n}$.
The NBC volumes' pressures are not updated because the old iterate, $k$, of a given NBC volume is always the old-time value to maintain consistency with the linear domain.
The nonlinear domain is subjected to multiple Newton iterates as outlined in \sect{sect:nlnCobraSolver}.
A deviation from the convergence procedure for a single Newton domain is that the norms used for convergence determination are formulated using the residuals and the updates from only that portion of the domain that is subject to the nonlinear solver.
These two vectors, the residual and the update, do not include terms from the NBC volumes.

Upon termination of the Newton iterations for the nonlinear domain, all of the inter-domain flow rates are now known.
These flow rates are used to obtain the linear domain's pressure updates.
The linear domain's pressure updates, $\delta P_{j}$, have their functional dependence resolved by using \eqref{eqn:linearPressureResolution}.

\begin{equation}
\label{eqn:linearPressureResolution}
\delta P_{j} = \delta P^{*}_{j} - \sum_{i\,\in \, N_{n}} \vec{w}_{j,i}^{T} \vec{\Xi}^{k}_{i} \left[ \momVec{}^{k}_{i} + \delta \momVec{}_{i}\right]
\end{equation}

Once the pressure update vector for the linear domain is obtained, all of linear domain's continuity volumes and flow paths are looped over.
During this loop, the new-time linear variables are calculated.
The completion of the linear update marks the completion of a single timestep.

%-------------------------------------------------------------------------------
%-------------------------------------------------------------------------------
%-------------------------------------------------------------------------------
\section{Implementation in \cobra{}}
\label{sec:dd_algo}

In this section, several architectural changes that were necessary to integrate the domain decomposition algorithm from \sect{sec:domDecompMath} are addressed.
First, the modification of the data structures for the pressure matrices and their solvers is discussed.
Second, the object-oriented, volume data structures for domain decomposition are detailed.
Next, the implementation of domain decomposition algorithm will be shown in detail.
Lastly, the dual domain input file is described.

%-------------------------------------------------------------------------------
%-------------------------------------------------------------------------------
%-------------------------------------------------------------------------------
\subsection{Pressure Matrix And Solver Data Structures}
\label{subsect:domDecompSolverStructs}

As seen in \sect{sect:linCobraAlg}, the largest linear system that is solved during a Newton step is the pressure matrix that is used to obtain the pressure update vector.
\cobra{} provides the user with three options for the linear algebra method to be used for solving for the pressure update: the Gauss elimination routine, the SuperLU solver \cite{Li1999}, and the Pardiso solver \cite{Schenk2006, Schenk2007}.
The SuperLU and the Pardiso solvers utilize sparse matrix storage and the Gauss elimination routine utilizes a full matrix.
The sparsity of the matrices typically encountered are such that use of the Gauss elimination routine is discouraged.

The method that the user selects via the \cobra{} input file, ``\classname{deck.inp}," determines the data storage structure of the pressure matrix.
Each of the three methods utilizes a different memory format for matrix storage.
Given that the original software was based upon procedural programming practices and written in a mixed Fortran dialect (FORTRAN 77, Fortran 90, and Fortran 95), the static memory structures for each of the three possible pressure matrices were predefined in different modules.
However, this procedural paradigm precluded the existence of two separate pressure matrices, one for the linear domain and one for the nonlinear domain.
This procedural approach for the pressure matrices was deemed to be inappropriate for the work being done, and an object-based approach was introduced instead.
To implement this switch, the way in which the solver was stored and accessed was modified to both enable multiple solvers and reduce the logical complexity of accessing the proper matrix.

\begin{figure}[ht!]
\singlespace\centering
\tikzsetnextfilename{images/matrixClassDiagram_eps}
\begin{tikzpicture}
\begin{abstractclass}[text width=6cm]{matrix}{0,0}
	\attribute{+ n : int}
	\attribute{+ nrhs : int}
	\operation[0]{+ put(i : int, j : int, val : real ) }
	\operation[0]{+ get(i : int, j : int ) : real }
	\operation[0]{+ rescale(i : int, j : int )}
	\operation[0]{+ reset()}
	\operation[0]{+ invert(b : real(:,:))}
\end{abstractclass}

\begin{abstractclass}[text width=6cm]{fullMatrix}{0,-6}
	\inherit{matrix}
	\attribute{- A : real(:,:) }
	\operation{+ put(i : int, j : int, val : real ) }
	\operation{+ get(i : int, j : int ) : real }
	\operation{+ rescale(i : int, j : int )}
	\operation{+ reset()}
\end{abstractclass}

\begin{class}[text width=4cm]{dsolveMatrix}{0,-11}
	\inherit{fullMatrix}
	\attribute{- ipiv : int(:) }
	\operation{+ invert(b : real(:,:))}
\end{class}

\begin{abstractclass}[text width=6cm]{sparseMatrix}{8,0}
	\inherit{matrix}
	\attribute{- num\_coeff: int }
	\attribute{- acoeff: real(:) }
	\attribute{- ij\_element: int(:,:) }
	\operation{+ put(i : int, j : int, val : real ) }
	\operation{+ get(i : int, j : int ) : real }
	\operation{+ rescale(i : int, j : int )}
	\operation{+ reset()}
\end{abstractclass}

\begin{class}[text width=4cm]{superluMatrix}{6,-6}
	\inherit{sparseMatrix}
	\attribute{- asub : int(:) }
	\attribute{- col\_beg : int(:) }
	\attribute{- ncpu : int }
	\operation{+ invert(b : real(:,:))}
\end{class}

\begin{class}[text width=4cm]{pardisoMatrix}{11,-6}
	\inherit{sparseMatrix}
	\attribute{- ia : int(:) }
	\attribute{- ja : int(:) }
	\attribute{- pt : int(:) }
	\attribute{- perm : int }
	\attribute{- mnum : int }
	\attribute{- mtype : int }
	\attribute{- maxfct : int }
	\operation{+ invert(b : real(:,:))}
\end{class}

\end{tikzpicture}
\caption{Matrix Class Diagram}
\label{fig:matrixClassDiagram}
\end{figure}

The object-oriented features in the Fortran 2003 and Fortran 2008 standards provided an alternative method of handling the pressure matrix.
The matrix storage structure was first replaced with an abstract class, \classname{matrix}.
The \classname{matrix} class contains the number of right-hand sides anticipated, as well as the total number of pressure updates for a given domain, $N$.
Additionally, this class would provide interfaces for the following procedures:

\begin{itemize}
\item{\classname{put} : a procedure that sets $\vec{A}_{i,j}$ to a given value.}
\item{\classname{get} : a procedure that returns $\vec{A}_{i,j}$.}
\item{\classname{scale} : a procedure for scaling $\vec{A}_{i, :}$ by a given value.}
\item{\classname{reset} : a procedure for setting $\vec{A} = \vec{0}$.}
\item{\classname{invert} : a procedure that returns $\vec{x}$ from $\vec{A}\vec{x} = \vec{b}$.}
\end{itemize}

Given that the SuperLU and the Pardiso solvers utilize sparse matrices and the Gauss elimination routine uses a full matrix, there are two further abstract classes that inherit from \classname{matrix}; they are \classname{fullMatrix} and \classname{sparseMatrix}.
The \classname{fullMatrix} class contains an array representing $\vec{A}$.
The \classname{sparseMatrix} class contains the information required to construct a sparse matrix; however, the exact storage format is not specified in \classname{sparseMatrix}.
Both \classname{fullMatrix} and \classname{sparseMatrix} contain procedural implementations of the interfaces defined in \classname{matrix}.
There are three concrete classes that correspond to the three supported solvers: \classname{dsolveMatrix}, \classname{superluMatrix}, and \classname{pardisoMatrix}; all three implement their own linear system solving routines under the invert interface.
\classname{dsolveMatrix} is a concrete class inheriting from \classname{fullMatrix} that includes pivoting information.
\classname{superluMatrix} and \classname{pardisoMatrix} are concrete classes inheriting from \classname{sparseMatrix} that implement the matrix storage schemes for their respective solvers.
\fig{fig:matrixClassDiagram} shows the class diagram for the \classname{matrix} hierarchy.

\begin{figure}[ht!]
\singlespace\centering
\begin{tikzpicture}
\begin{abstractclass}[text width=6cm]{solver}{0,0}
	\attribute{+ size : int}
	\attribute{+ A : matrix, ptr }
	\attribute{+ res : real(:,:)}
	\operation{+ insert(i : int, j : int) }
	\operation{+ solve() }
	\operation{+ rescale(i : int, j : int) }
	\operation{+ reset()}
	\operation[0]{+ init()}
\end{abstractclass}

\begin{class}[text width = 4cm]{dsolveSolver}{-5,-7}
	\inherit{solver}
	\attribute{+ A $=>$ dsolveMatrix }
	\operation{+ init()}
\end{class}

\begin{class}[text width = 4cm]{superluSolver}{0,-7}
	\inherit{solver}
	\attribute{+ A $=>$ superluMatrix }
	\operation{+ init()}
\end{class}

\begin{class}[text width = 4cm]{pardisoSolver}{5,-7}
	\inherit{solver}
	\attribute{+ A $=>$ pardisoMatrix }
	\operation{+ init()}
\end{class}

\end{tikzpicture}
\caption{Solver Class Diagram}
\label{fig:solverClassDiagram}
\end{figure}

Once the \classname{matrix} hierarchy was developed, an abstract \classname{solver} class was designed.
The \classname{solver} class contains the size of the system, $N_{a}$; a pointer to a \classname{matrix} object, $\vec{A}$; and the right-hand side of the linear system, \classname{res}.
Within the \classname{solver} class there is an interface for the initialization routine, \classname{init}.
Additionally, there is a procedure named \classname{insert} that will take the sixth row of a continuity volume's linear system and properly insert it into the the correct row of the domain's pressure matrix, $\vec{A}_{j,:}$, as outlined in \sect{sec:domDecompMath}.
The following procedures for manipulating the linear system are also defined within the \classname{solver} object:

\begin{itemize}
\item{ \classname{solve} : an interface to call the solver associated with the matrix.}
\item{ \classname{scale} : a procedure for scaling both $\vec{A}_{i, :}$ and $\vec{res}_{i}$.}
\item{ \classname{reset} : a procedure for setting both $\vec{A} = \vec{0}$ and $\vec{res} = \vec{0}$.}
\end{itemize}

There are three concrete classes that inherit from the abstract solver class: \classname{dsolveSolver}, \classname{superluSolver}, and \classname{pardisoSolver}.
Each of these concrete classes specifies an initialization routine that takes an adjacency list data structure, instantiates the \classname{matrix} pointer to the appropriate concrete subtype, and sets the appropriate parameters.
By abstracting the matrix storage strategies and the solvers in this way, a consistent interface is provided to the rest of the software.
\fig{fig:solverClassDiagram} shows the class diagram for the \classname{solver} hierarchy.

%-------------------------------------------------------------------------------
%-------------------------------------------------------------------------------
%-------------------------------------------------------------------------------
\subsection{Volume Data Structures}
\label{subsect:domDecompVolumeStructs}

A difficulty that had to be overcome during this work was the proper mapping of the continuity volumes memory indices to the proper pressure matrix indices.
The memory storage format for many variables was based upon a universal ordinal system for the continuity volumes.
This ordinal system was based upon the premise that there was a single pressure matrix.
It was determined that the programmatic complexity of trying to maintain a single ordinal system for the dual domains was greater than that of redesigning the memory architecture.
As part of the redesign, the software was partially transitioned to an object-based mesh to allow for multiple pressure matrices with different ordinal systems to co-exist.

A continuity volume was chosen as the basic unit for mesh representation.
Given that the mesh is decomposed into dual domains, there were two types of continuity volumes that needed to be distinguished.
The first type of volume, \classname{baseVolume}, encompasses all continuity volumes in the nonlinear domain and those continuity volumes in the linear domain that do not have a flow path connecting them to the nonlinear domain.
The second type of volume, \classname{nbcVolume}, includes those continuity volumes in the linear domain that have a flow path connecting them to the nonlinear domain, the NBC volumes.
The volume class diagram for \cobra{} is outlined in \fig{fig:volumeClassDiagram}.

\begin{figure}[ht!]
\singlespace\centering
\begin{tikzpicture}

\begin{class}[text width=6cm]{baseVolume}{-4,0}
	\attribute{+ solver : ptr, solver}
	\attribute{+ ord : int}
	\attribute{+ num\_cons : int}
	\attribute{+ cons : int}
	\attribute{+ con\_type : int(:)}
	\attribute{+ con\_ord : int(:)}
	\attribute{+ gap\_ord : int(:)}
	\attribute{+ con\_coords : int(:,:)}
	\attribute{+ fp\_coords : int(:,:)}
	\attribute{+ con\_sig : real(:)}
	\attribute{+ b : int(:,:)}
	\attribute{+ b\_ords : int(:)}
	\attribute{+ jac : int(:,:)}
\end{class}

\begin{class}[text width=7cm]{nbcVolume}{4,0}
	\inherit{baseVolume}
	\attribute{+ nln\_solver : ptr, solver }
	\attribute{+ dp : real}
	\attribute{+ coef\_mat : real(:,:)}
	\attribute{+ flux : real(:,:)}
	\attribute{+ dflux\_dp : real(:,:)}
	\attribute{+ donored\_values : real(:,:)}
	\attribute{+ nbc\_cons : int}
	\attribute{+ nln\_ord\_by\_local\_nbc\_ord : int(:)}
	\attribute{+ nbc\_con\_ord : int(:)}
	\attribute{+ nbc\_con\_map : int(:)}
	\attribute{+ nbc\_ords : int(:)}
	\attribute{+ coords : int(:)}
	\attribute{+ nbc\_coords : int(:,:)}
\end{class}

\end{tikzpicture}
\caption{Volume Class Diagram}
\label{fig:volumeClassDiagram}
\end{figure}

The \classname{baseVolume} class contains information for the most basic type of continuity volumes.
Each \classname{baseVolume} has a pointer to the \classname{solver} object associated with its domain.
A \classname{baseVolume} that corresponds to a continuity volume in the linear domain has a \classname{solver} pointer to the linear pressure matrix solver.
The nonlinear domain's continuity volumes have a \classname{solver} pointer to the nonlinear pressure matrix solver.
This allows each continuity volume to be able to access the correct pressure matrix without having to reference a logical map or perform a calculation.
This class also stores the volume's ordinal.
This variable provides indexing information for the volume's pressure matrix.

Additionally, information regarding the connectivity of the volume to other volumes is stored in several variables: \classname{num\_cons}, \classname{cons}, \classname{con\_type}, \classname{con\_ord}, \classname{gap\_ord}, \classname{con\_coords}, \classname{fp\_coords}, and \classname{con\_sig}.
These variables represent the information necessary to properly references those flow paths to which the continuity volume is connected as well as the continuity volumes on the other end of those flow paths.
This data structure has the drawback of storing redundant flow path information.
Each of the continuity volumes connected by a particular flow path contain all of the information about that flow path in separate memory locations.

The \classname{baseVolume} class also contains the Jacobian matrix and the right-hand sides from \sect{sect:linCobraAlg}.
The volume's linear system, $\vec{Z}$, is stored as \classname{jac}.
The dimensions of $\vec{Z}$ are dictated by the connectivity of the volume and its volume type, both of which are determined during input processing.
The right-hand side of the continuity volume's linear system, $\vec{b}$, is stored as \classname{b}.
Once again, the dimensions of \classname{b} are dictated by the type of the volume; an \classname{nbcVolume} will have more columns in \classname{jac} and \classname{b} than a comparable \classname{baseVolume}.  
The variable \classname{b\_ords} provides a mapping between the columns of \classname{b} and the columns of the pressure matrix of the domain to which the volume belongs.

Those linear continuity volumes that are connected to the nonlinear domain via a flow path have their own \classname{nbcVolume} class.
This class, as shown in \fig{fig:volumeClassDiagram}, is a derived class and contains additional information required to perform the domain decomposition.
First, a second \classname{solver} pointer is present, \classname{nln\_solver}.
This allows for proper indexing into the nonlinear pressure matrix when needed.
Second, the \classname{nbcVolume} also contains a separate collection of variables describing those flow paths that connect it to the nonlinear domain: \classname{nbc\_cons}, \classname{nbc\_con\_ord}, \classname{nbc\_con\_map}, \classname{nbc\_ords}, \classname{nbc\_coords}, \classname{coords}, and \classname{nln\_ord\_by\_local\_nbc\_ord}.
Lastly, the \classname{nbcVolume} class also contains the pressure coefficients, \classname{coef\_mat}; the flow rate associated with the boundary, \classname{flux}; the derivative of the flow rate with respect to the pressure on either side of the flow path, \classname{dflux\_dp}; the matrix $\vec{\Xi}$ in compressed storage, \classname{donored\_values}; and the right-hand side of the linear domain's pressure solution, \classname{dp}.

Each continuity volume is represented by one of the above described volume classes.
This polymorphic approach allows for domain-agnostic procedures and references within the software.

%-------------------------------------------------------------------------------
%-------------------------------------------------------------------------------
%-------------------------------------------------------------------------------
\subsection{Domain Decomposition Algorithm}
\label{subsect:domDecompAlgorithm}

The implementation of the domain decomposition's mathematical algorithm provided in \sect{sec:domDecompMath} will now be covered.
\alg{alg:domDecompAlgorithm} provides a high level overview of the software as implemented.
The steps in this algorithm will be detailed in the following section.

\begin{algo}[ht!]
\setlength{\baselineskip}{0.625\baselineskip}
\begin{algorithmic}[1]
\Require Input Processing
\Set $n = 0$
\Loop \; Transient Loop
    \Set $t^{n+1} : = t^{n} + \dt{}$
	\Algorithm Assemble Nonlinear and Linear Pressure Matrices	 \Comment{\alg{alg:xschem}}
	\LineIf{ \classname{LinearSolver} }{\textbf{solve} $\vec{A}_{l} \vec{\delta P}_{l} = \vec{res}_{l}$}
	\If{ \classname{NonlinearSolver} }
	    \Set $k = 0$
		\Algorithm Get Nonlinear Coefficients \Comment{\alg{alg:domDecompGetCoef}}
		\Algorithm Set Nonlinear Boundary Values \Comment{\alg{alg:domDecompSetMat}}
		\Solve $\vec{A}^{k}_{n} \vec{\delta P}_{n}^{k} = \vec{res}_{n}^{k}$	
		\Algorithm Update Nonlinear Variables \Comment{\alg{alg:updateVariables}} 
	    \Loop \; Newton Loop
			\Algorithm Assemble Nonlinear Pressure Matrix \Comment{\alg{alg:xschem}}
			\Algorithm Nonlinear Convergence Determination \Comment{\alg{alg:nlnConvergence}}
			\If{ \textbf{end} Newton loop}
				\State \textbf{break} Newton Loop 
			\EndIf		
			\Set $k \pluseq 1$
			\Algorithm Set Nonlinear Boundary Values \Comment{\alg{alg:domDecompSetMat}}
			\Solve $\vec{A}_{n}^{k} \vec{\delta P}_{n}^{k} = \vec{res}_{n}^{k}$
			\Algorithm Update Nonlinear Variables \Comment{\alg{alg:updateVariables}}
		\EndLoop
	\EndIf
	\LineIf{ \classname{LinearSolver} }{\textbf{algorithm} Update Linear Variables} \Comment{\alg{alg:updateVariables}}
	\Set $n \pluseq 1$
\EndLoop
\end{algorithmic}
\caption{Dual domain \cobra{} algorithm.}
\label{alg:domDecompAlgorithm}
\end{algo}

First, the nonlinear domain decomposition input file described in \sect{subsect:domDecompInputFile} is parsed to determine to which domain each channel belongs.
Second, each continuity volume is designated as either a \classname{baseVolume} or an \classname{nbcVolume}.
Then the \classname{solver} objects are instantiated according to the \cobra{} input file.
Lastly, after the input processing and state initialization, the temporal integration loop begins.

In a dual domain problem, a single timestep is now broken into two parts: the linear and the nonlinear portions.
At the beginning of a timestep, the entire domain is traversed to assemble the pressure matrices for both the linear domain and the first iterate of the nonlinear domain.
This process is similar to the algorithm outlined in \ref{sect:linCobraAlg} for assembling the pressure matrix.
However, there are differences when an \classname{nbcVolume} is encountered.

For each \classname{baseVolume} the following procedure is followed.
Each continuity volume's linearized system of equations, \eqref{eqn:linSystem}, is subjected to partial \vec{LU} decomposition without pivoting.
The lower triangular matrix has a unit diagonal.
The sixth equation in the upper-triangular system is then scaled by its diagonal.
This upper-triangular, rectangular system for each continuity volume is then stored for later back-substitution.
Since the pressure update, $\delta P$, corresponds to the last row in the linear system, the reduction of the linear system to its upper-triangular form allows for the isolation of the pressure update.
This isolated equation represents the relationship of a given continuity volume's pressure change to the pressure change in the other continuity volumes to which it is attached.
The last row of each continuity volume's \eqref{eqn:linSystem} is then formed into the linear pressure matrix, $\vec{A}_{l}$, and its associated right-hand side, $\vec{r}$.

The two loops, one over the momentum flow paths and one over the continuity volumes, form a single group of operations that act upon a given domain.
This grouping will be known hereafter as assembling the pressure matrix for a given domain.
\alg{alg:xschem} shows the two loops and their associated actions.

\begin{algo}[ht!]
\setlength{\baselineskip}{0.625\baselineskip}
\begin{algorithmic}[1]
\For{i = 1, $N_{\text{nln}}$}
	\Set $\text{b} \Rightarrow \text{nbcVolume[i].solver.res(nbcVolume[i].ord,:)}$
	\For{j = 1, $N_{\text{nln}}$}
		\For{ k = 1, nbcVolume[j].nbc\_cons}
			\Set g\_ord = nbcVolume[j].nbc\_con\_ord(k)
			\Set s\_ord = 2 + 6 $*$ (g\_ord - 1)
			\Set e\_ord = s\_ord + 5
			\Set nbcVolume[i].coef\_mat(1:6, g\_ord) = b(s\_ord:e\_ord)
		\EndFor
	\EndFor
\EndFor
\end{algorithmic}
\caption{Obtain NBC Volume Coefficients.}
\label{alg:domDecompGetCoef}
\end{algo}

Each \classname{nbcVolume} is treated slightly differently.
The flow rate terms from the equations are not included in the residual at this point, but are instead treated as unknowns.
These flow rates are treated as independent parameters, which decreases the number of entries in the $\vec{K}_{n}$ matrix.
However, the derivatives of the flow rate terms with respect to themselves, $\dt{} \vec{I}$, are added to the linear system.
If a linear continuity volume is attached through more than one surface to the nonlinear domain, then there will be multiple identity matrices in the continuity volume's linear system.
The resulting system will be subjected to the same \vec{LU} decomposition as outlined for a regular volume.
The last equation with the additional right-hand sides will be assembled into the linear domain's pressure matrix.
The linear system for the linear domain is then inverted by the chosen solver.
The additional coefficients represent the coefficients for the boundary flow rate between the linear and the nonlinear domain.
After the inversion of the pressure matrix, the coefficients for the NBC volumes are collected into the \classname{nbcVolume} objects for ease of manipulation later.
This is shown in \alg{alg:domDecompGetCoef}.

\begin{algo}[ht!]
\setlength{\baselineskip}{0.625\baselineskip}
\begin{algorithmic}[1]
\For{i = 1, $N_{\text{nln}}$}
	\Set A $\Rightarrow$ nbcVolume[i].nln\_solver.A
	\Set i\_ord = nbcVolume[i].nln\_ord
	\For{j = 1, $N_{\text{nln}}$}
		\Set j\_ord = nbcVolume[j].nln\_ord
		\For{ k = 1, nbcVolume[i].nbc\_cons}
			\Set g\_ord = nbcVolume[j].nbc\_con\_ord(k)
			\Set k\_ord = nbcVolume[j].nln\_ord\_by\_local\_nbc\_ord(k)
			\Set dp = nbcVolume[i].coef\_mat(:, g\_ord) $\cdot$ nbcVolume[j].dflux\_dp(:, k)
			\Set A(i\_ord, k\_ord) $\pluseq$ dp
			\Set A(i\_ord, j\_ord) $\minuseq$ dp
		\EndFor
	\EndFor
	\Set res = nbcVolume[i].solver.res(nbcVolume[i].ord,:) $\cdot$ nbc\_flux
	\Set nbcVolume[i].nln\_solver.res(i\_ord, 1) = res
	\Set A(i\_ord, i\_ord) $\pluseq$ 1.0
	\State \textbf{call:} nbcVolume[i].nln\_solver.rescale(i\_ord)
\EndFor
\end{algorithmic}
\caption{Set NBC Volume Pressure Equations Into Nonlinear Pressure Matrix.}
\label{alg:domDecompSetMat}
\end{algo}

Once the pressure coefficients have been collected, and the linear domain has been inverted, the additional entries in the nonlinear pressure matrix that represent the NBC volume's pressure equations are populated.
This procedure is shown in detail in \alg{alg:domDecompSetMat}.
This process is a system of three nested loops.
The first, outer, loop is over all the NBC volumes in the domain, $N_{\text{nln}}$.
From \sect{sec:domDecompMath} and \eqref{eqn:linearDomainPressureUpdates}, each of these outer NBC volumes has an associated pressure equation, \eqref{eqn:nbcPressureEquation}.
This equation represents the inter-dependency of all of the NBC volumes upon each other.
Evaluating \eqref{eqn:nbcPressureEquation} requires a second, inner, loop over all of the NBC volumes for each of the outer NBC volumes.
For each of these inner loop NBC volumes, their $N_{\text{NBC}}$ flow paths that connect them to nonlinear domain are then traversed, creating a third and final nested loop.
For each connecting flow path, the contribution is calculated and stored in the row of $\vec{A}_{n}$ associated with the outer NBC volume.

Once the pressure equations for the NBC volumes have been inserted into the nonlinear pressure matrix, this matrix is solved.
The resulting pressure update vector is then used to update the nonlinear continuity volumes' variables according to \alg{alg:updateVariables}.
This inversion and update represent the first iterate for the nonlinear domain.
The Newton loop is now entered.
This loop is similar to that discussed in \sect{sect:nlnCobraSolver}, with the addition of a step to set the NBC volumes' equations.
Upon termination of the Newton loop, the linear domain's variables are updated.

%-------------------------------------------------------------------------------
%-------------------------------------------------------------------------------
%-------------------------------------------------------------------------------
\subsection{Dual Domain Input File}
\label{subsect:domDecompInputFile}

Since the domain decomposition algorithm uses the nonlinear solver, the nonlinear convergence tolerances and the iteration limit are input as described in \sect{subsect:nlnCobraInputFile}.
These parameters only apply to the nonlinear domain.
Additional information is now appended to the end of the basic nonlinear input to active the domain decomposition algorithm.
The nonlinear solver input file needs to have the IDs of the channels that are going to be subjected to additional Newtons steps specified.
The presence of the character string ``\classname{'begin-nln-channels'}'' after the nonlinear input signals that there will be two domains.
After that string, the IDs of the channels are listed, one per line, in no particular order.
Once all channels to be in the nonlinear domain have been listed, the character string ``\classname{'end-nln-channels'}'' needs to be on a separate line.
The following is an example of the dual domain input segment of the ``\classname{nwt.cob}'' input file.

{
\singlespace
\begin{verbatim}
'begin-nln-channels'
<chanID_1>
<chanID_2>
<chanID_3>
...
<chanID_N>
'end-nln-channels'
\end{verbatim}
}

In the above example, <chanID\_N> is the channel ID that will be included in the nonlinear domain.
The input processing routine has extensive error checking and user feedback for proper formatting of this input.
Note that there is no requirement that the channels specified to be in the nonlinear domain form a contiguous domain.


       % Chapter: Domain Decomposition Algorithm
\chapter{Numerical Results}
\label{chap:results}

A total of six test problems were developed for evaluating the nonlinear solver, the operator-based scaling, the convergence metric, the implementation of the domain decomposition algorithm, and the efficacy of the selective nonlinear refinement.
\comment{Are all of these covered in this chapter?}
Two of these problems were developed to evaluate the nonlinear solver, the operator-based scale factor, and the temporal convergence metric.
A third problem with extensive geometric complexity was used to evaluate the correct implementation of the domain decomposition algorithm.
The fourth problem was designed to show that the selective nonlinear refinement can produce the nonlinear solver's solution with less computational effort.
The fifth problem shows that by resolving local nonlinearities, the global solution more accurately reflected the correct solution.
And, the last problem is a simple model of a core refill event.
This problem shows that use of the domain decomposition algorithm can obtain results that are more reflective of the nonlinear solver, with less computational time.
The goal of the tests, the methods used, and the models will be discussed in each section.

\section{Single-Phase Flow and Flashing Problems}
\label{sect:single_phase_and_flashing}

Two test problems were developed to illustrate the effectiveness of the nonlinear solver, the operator-based scaling, and the convergence metric.


\subsection{Model}
\label{subsect:single_model}
The two problem developed for these tests were thought problems with generic input conditions.
For both of these test problems, the same computational geometry was used;  \fig{fig:exp_geometry} represents the model geometry.
Each block represents a single continuity volume with a fixed \dx{} of 4 [in].
Each continuity cell has a cross-sectional area of 4 [in$^2$].
The red block at the top of the channel represents a boundary cell where the pressure and enthalpy are specified.
It represents an infinite reservoir filled with a fluid at a specified thermodynamic state.
The red triangle represents a specified flow at the bottom edge of the first continuity volume. 

\begin{figure}[h!tb]
\centering
\tikzsetnextfilename{images/scaling_test_problem_geometry_pdf}
\begin{tikzpicture}
\foreach \x in {1,..., 12} \draw(0, 0.5*\x-0.5) rectangle +(.5,.5);
\filldraw[fill=red] (0, 6) rectangle +(.5,.5); 
\filldraw[fill=red] (0, -0.5) -- (0.25, 0) -- (0.5, -0.5) -- cycle;
\draw[->] (2,0) -- (2, 1) node[anchor=west] {z};
\draw[->] (2,0) -- (3, 0) node[anchor=west] {$\perp$ z};
\end{tikzpicture}
\caption{Geometry for single-phase and flashing problems.}
\label{fig:exp_geometry}
\end{figure}

The two problems, while having the same geometry, are different in their dominant physics.
One problem was designed to simulate single-phase, single-field continuous liquid flow in a pipe.
This problem will be referred to as the single-phase problem.
The second problem was designed such that high-pressure liquid flashes into steam as it enters a pipe initially filled with saturated vapor at a much lower pressure, known hereafter as the flashing problem.

\tab{tab:dualInitialConditions} provides the initial conditions for the two problems.
The pressure, enthalpy, and volume-fractions for the different fields allow for a complete description of the continuity variables.
The initial velocities for both problems are set to zero.

\begin{table}[h!tb]
\centering
\singlespace
\pgfplotstabletypeset[fixed zerofill, col sep=comma,
	columns/problem/.style={column name=,string type, column type=l},
	columns/p/.style={ column name=Pressure,dec sep align, precision=1},
	columns/h/.style={ column name=Enthalpy,dec sep align, precision=1},
	columns/a_g/.style={ column name=$\alpha_g$, precision=1},
	columns/a_l/.style={ column name=$\alpha_l$, precision=1},
	columns/a_e/.style={ column name=$\alpha_e$, precision=1},
	every head row/.style={
		before row=\toprule,
		after row={& \multicolumn{2}{c}{$[\text{psia}]$} &\multicolumn{2}{c}{$[\frac{\text{BTU}}{\lbm{}}]$} &$[\text{-}]$ & $[\text{-}]$ & $[\text{-}]$\\ \midrule}},
	every last row/.style={
after row=\bottomrule}]{tables/dualInitialConditionsData.tex}

\caption{Initial conditions for the single-phase and flashing problems.}
\label{tab:dualInitialConditions}
\end{table}

Each of the problems has a specified pressure-enthalpy boundary condition at the outlet and a flow-enthalpy boundary condition at the inlet of the domain.
\tab{tab:dualOutlet} contains the pressure, enthalpy, and composition of the pressure-enthalpy reservoir. 

\begin{table}[h!tb]
\centering
\singlespace
\pgfplotstabletypeset[fixed zerofill, col sep=comma,
	columns/problem/.style={column name=,string type, column type=l},
	columns/p/.style={ column name=Pressure,dec sep align, precision=1},
	columns/h/.style={ column name=Enthalpy,dec sep align, precision=1},
	columns/a_g/.style={ column name=$\alpha_g$, precision=1},
	columns/a_l/.style={ column name=$\alpha_l$, precision=1},
	columns/a_e/.style={ column name=$\alpha_e$, precision=1},
	every head row/.style={
		before row=\toprule,
		after row={& \multicolumn{2}{c}{$[\text{psia}]$} &\multicolumn{2}{c}{$[\frac{\text{BTU}}{\lbm{}}]$} &$[\text{-}]$ & $[\text{-}]$ & $[\text{-}]$\\ \midrule}},
	every last row/.style={
after row=\bottomrule}]{tables/dualOutletData.tex}

\caption{The outlet boundary conditions for the single-phase and flashing problems.}
\label{tab:dualOutlet}
\end{table}

The flow-enthalpy boundary condition describes the thermodynamic state of the in-flowing fluid and its flow rate.
\tab{tab:dualInlet} describes the inlet boundary conditions for the two problems.
\comment{At some point, this value should be compared to hf at 200 [psia] to show the driving force.}

\begin{table}[h!tb]
\centering
\singlespace
\pgfplotstabletypeset[fixed zerofill, col sep=comma,
	columns/problem/.style={column name=,string type, column type=l},
	columns/p/.style={ column name=Pressure,dec sep align, precision=1},
	columns/h/.style={ column name=Enthalpy,dec sep align, precision=1},
	columns/a_g/.style={ column name=$\alpha_g$, precision=1},
	columns/a_l/.style={ column name=$\alpha_l$, precision=1},
	columns/a_e/.style={ column name=$\alpha_e$, precision=1},
	every head row/.style={
		before row=\toprule,
		after row={& \multicolumn{2}{c}{$[\text{psia}]$} &\multicolumn{2}{c}{$[\frac{\text{BTU}}{\lbm{}}]$} &$[\text{-}]$ & $[\text{-}]$ & $[\text{-}]$\\ \midrule}},
	every last row/.style={
after row=\bottomrule}]{tables/dualInletData.tex}

\caption{The flow-enthalpy boundary conditions for the single-phase and flashing problems.}
\label{tab:dualInlet}
\end{table}

The specified mass flow, $\dot{m}(t)$, at the bottom of the channels is the same for both problems. 
This time-dependent function is given by \eqref{eqn:bc_time_func_single}.

\begin{equation}
\label{eqn:bc_time_func_single}
\dot{m}(t) = \left\{
\begin{array}{cclrcll}
 0.0           & [\frac{ \lbm{} }{\text{s}}] & , &                & t & \leq 1 & [\text{s}] \\
 0.5 ( t - 1)  & [\frac{ \lbm{} }{\text{s}}] & , & 1\; [\text{s}] < & t & \leq 2 & [\text{s}] \\
 0.5           & [\frac{ \lbm{} }{\text{s}}] & , &                & t & > 2    & [\text{s}]
\end{array}\right.
\end{equation}

Both problems adjust their initial pressure distribution to account for hydrostatic head.

\subsection{Results}
\label{subsect:single_results}

The two models described above were subjcted to to timestep-sensitivity studies.
An automated procedure was used to conduct the timestep sensitivity study.
Given an initial \dtmax{}, $\dt{}_{0}$, a refinement factor, $r_{f}$, and the number of timestep size refinements to be taken, $n_{t}$, the simulation was run with both the linear solver and the nonlinear solver for each \dtmax{} in the set, calculated according to \eqref{eqn:timeStepAlgo}.

\begin{equation}
\label{eqn:timeStepAlgo}
\dtmax{} \in \bigcup^{n_{t}-1}_{i\, = 0} \frac{\dt{}_{0}}{r^{i}_{f}}
\end{equation}

For these two problems, the three parameters, $\dt{}_{0}$, $r_{f}$, and $n_{t}$, were 1.0, 10.0, and 6, respectively.
Each of the two problems were run with both the linear and the nonlinear solver at each of the six different \dtmax{}; in total there were 24 simulations run.
The nonlinear convergence criteria used for the nonlinear simulations were $\kmax$ = 35, $F_{\text{tol}}$ = \expneg{1.0}{6}, and $\delta_{\text{tol}}$ = \expneg{1.0}{8}.
\comment{Someone will ask how these were chosen.}

The results from the simulations will now be analyzed to determine the impact of nonlinear convergence upon time-step size sensitivity.
The solutions produced by both solvers will be compared to determine the efficacy of the convergence metric in determining the validity of the nonlinear solution.

\begin{figure}[h!tb]
\centering
% This file was created by matlab2tikz v0.4.3.
% Copyright (c) 2008--2013, Nico Schlömer <nico.schloemer@gmail.com>
% All rights reserved.
% 
\tikzsetnextfilename{plots/single1pt000em0_pdf}
\begin{tikzpicture}

\begin{axis}[%
width=\mytikzpicwidth,
height=\mytikzpicheight,
scale only axis,
xmin=0.0,
xmax=5.0,
xlabel={Time $[\text{s}]$},
ymin=-0.1,
ymax=0.6,
ylabel={$\dot{m}_{l} [\frac{\lbm{}}{\text{s}}]$ @ 4 [in] from Inlet},
legend style={at={(0.03,0.97)},anchor=north west,draw=black,fill=white,legend cell align=left}
]
\addplot [
color=black,
solid,
mark=o,
mark options={solid}
]
table[row sep=crcr]{
0.0 0.0\\
0.0118019115179777 -3.47342599127387e-08\\
0.0226854234933853 -2.17124599549701e-11\\
0.0340500771999359 1.13636625608503e-13\\
0.0491764321923256 -3.90975188329314e-17\\
0.0619497485458851 7.33235559715788e-18\\
0.0774054601788521 6.03480906903423e-18\\
0.0961068719625473 5.00410893856865e-18\\
0.106882445514202 4.55962054128114e-18\\
0.118735581636429 4.07895953580668e-18\\
0.13177402317524 3.73458439446616e-18\\
0.146116316318512 3.39633680030827e-18\\
0.161892831325531 3.13169835450025e-18\\
0.179247006773949 2.82074048272625e-18\\
0.198336601257324 2.55728904109725e-18\\
0.21933513879776 2.33917371448105e-18\\
0.242433547973633 2.11583805101903e-18\\
0.267841786146164 1.92573850456939e-18\\
0.29579085111618 1.73994236525656e-18\\
0.32653483748436 1.58567131985904e-18\\
0.360353201627731 1.44291204364885e-18\\
0.397553414106369 1.30928928749703e-18\\
0.438473641872406 1.19390296871986e-18\\
0.483485877513886 1.08049495577519e-18\\
0.532999336719513 9.78869923861191e-19\\
0.587464153766632 8.93386597818123e-19\\
0.647375464439392 8.12703711062431e-19\\
0.71327793598175 7.38948668732023e-19\\
0.7857705950737 6.68411307285658e-19\\
0.865512549877167 6.09004798354536e-19\\
0.953228712081909 5.51530583334339e-19\\
1.04971647262573 0.0248580612242222\\
1.1558530330658 0.0779263973236084\\
1.27260315418243 0.136301591992378\\
1.4010283946991 0.200514197349548\\
1.54229605197906 0.271148055791855\\
1.69769060611725 0.348845303058624\\
1.86862456798553 0.434312283992767\\
2.05665183067322 0.500000059604645\\
2.26348209381104 0.50000011920929\\
2.49099516868591 0.5\\
2.74125957489014 0.5\\
3.01655030250549 0.5\\
3.31937026977539 0.5\\
3.65247201919556 0.5\\
4.01888418197632 0.5\\
4.42193746566772 0.5\\
4.71096897125244 0.5\\
5 0.5\\
};
\addlegendentry{Linear Solver};

\addplot [
color=black,
solid
]
table[row sep=crcr]{
0 0\\
0.0118019115179777 -3.46389690264459e-08\\
0.0226854234933853 -2.15821874260946e-11\\
0.0340500771999359 1.12671292651703e-13\\
0.0491764321923256 -3.9315788635527e-17\\
0.0619497485458851 7.14522252718001e-18\\
0.0774054601788521 5.89121051469502e-18\\
0.0961068719625473 4.88778625133808e-18\\
0.106882445514202 4.45462816049101e-18\\
0.118735581636429 3.99217133234731e-18\\
0.13177402317524 3.64475216635259e-18\\
0.146116316318512 3.31211699265088e-18\\
0.161892831325531 3.05590731048962e-18\\
0.179247006773949 2.751287296569e-18\\
0.198336601257324 2.49874678080988e-18\\
0.21933513879776 2.28569855583103e-18\\
0.242433547973633 2.06815853333086e-18\\
0.267841786146164 1.88126803406762e-18\\
0.29579085111618 1.70056360501536e-18\\
0.32653483748436 1.54573121417465e-18\\
0.360353201627731 1.40811048732721e-18\\
0.397553414106369 1.27695955364085e-18\\
0.438473641872406 1.16573002423692e-18\\
0.483485877513886 1.05619993740148e-18\\
0.532999336719513 9.57275236597152e-19\\
0.587464153766632 8.73335015986802e-19\\
0.647375464439392 7.9452486613793e-19\\
0.71327793598175 7.21562573027077e-19\\
0.7857705950737 6.52017156231528e-19\\
0.865512549877167 5.95456975788202e-19\\
0.953228712081909 5.37766763230551e-19\\
1.04971647262573 0.0248580686748028\\
1.1558530330658 0.0779263898730278\\
1.27260315418243 0.136301591992378\\
1.4010283946991 0.200514197349548\\
1.54229605197906 0.271148055791855\\
1.69769060611725 0.348845303058624\\
1.86862456798553 0.434312283992767\\
2.05665183067322 0.500000059604645\\
2.26348209381104 0.50000011920929\\
2.49099516868591 0.5\\
2.74125957489014 0.5\\
3.01655030250549 0.5\\
3.31937026977539 0.5\\
3.65247201919556 0.5\\
4.01888418197632 0.5\\
4.42193746566772 0.5\\
4.71096897125244 0.5\\
5 0.5\\
};
\addlegendentry{Nonlinear Solver};

\end{axis}
\end{tikzpicture}%
\caption{Single-phase solution with \dtmax{} = 1.0 {[s]}.}
\label{fig:single1pt000em0}
\end{figure}

\begin{figure}[h!tb]
\centering
% This file was created by matlab2tikz v0.4.3.
% Copyright (c) 2008--2013, Nico Schlömer <nico.schloemer@gmail.com>
% All rights reserved.
% 
\tikzsetnextfilename{plots/single1pt000em5_eps}
\begin{tikzpicture}

\begin{axis}[%
width=\mytikzpicwidth,
height=\mytikzpicheight,
scale only axis,
xmin=0.0,
xmax=5.0,
xlabel={Time $[\text{s}]$},
ymin=-0.1,
ymax=0.6,
ylabel={$\dot{m}_{l} [\frac{\lbm{}}{\text{s}}]$ @ 4 [in] from Inlet},
legend style={at={(0.03,0.97)},anchor=north west,draw=black,fill=white,legend cell align=left}
]
\addplot [
color=black,
solid,
mark=o,
mark options={solid}
]
table[row sep=crcr]{
0.0 0.0\\
0.0100024435669184 2.84913312498247e-05\\
0.020002443343401 6.3776082242839e-05\\
0.0300024431198835 6.68911525281146e-05\\
0.0400024428963661 5.20597714057658e-05\\
0.0500024445354939 3.02229072985938e-05\\
0.0600024424493313 8.62097022036323e-06\\
0.070002444088459 -8.08286313258577e-06\\
0.0800024420022964 -1.77508427441353e-05\\
0.0900024399161339 -2.0457244318095e-05\\
0.100002445280552 -1.77929214260075e-05\\
0.110002443194389 -1.20413887998438e-05\\
0.120002441108227 -5.4509541769221e-06\\
0.130002439022064 2.57638987477549e-07\\
0.140002444386482 4.09452377425623e-06\\
0.1500024497509 5.80263849769835e-06\\
0.160002440214157 5.68671612199978e-06\\
0.170002445578575 4.3527188609005e-06\\
0.180002436041832 2.48281412495999e-06\\
0.19000244140625 6.6041320678778e-07\\
0.200002446770668 -7.27347639895015e-07\\
0.210002437233925 -1.51257336256094e-06\\
0.220002442598343 -1.71384147051867e-06\\
0.230002447962761 -1.47271020978224e-06\\
0.240002438426018 -9.83303834800608e-07\\
0.250002443790436 -4.32722686127818e-07\\
0.260002434253693 3.80195892546453e-08\\
0.270002454519272 3.50443571051073e-07\\
0.280002444982529 4.8601015123495e-07\\
0.290002435445786 4.71122689305048e-07\\
0.300002455711365 3.56985424332379e-07\\
0.310002446174622 2.00401871097711e-07\\
0.320002436637878 4.94536074313601e-08\\
0.330002456903458 -6.43351896201239e-08\\
0.340002447366714 -1.27682696415832e-07\\
0.350002437829971 -1.42671197522759e-07\\
0.360002458095551 -1.21394094776406e-07\\
0.370002448558807 -8.00980686221919e-08\\
0.380002439022064 -3.4274272309176e-08\\
0.390002429485321 4.52052972832462e-09\\
0.4000024497509 2.9961800152023e-08\\
0.410002440214157 4.06867926017185e-08\\
0.420002430677414 3.90151591034282e-08\\
0.430002450942993 2.9264075607216e-08\\
0.44000244140625 1.61591167113784e-08\\
0.450002431869507 3.66112762328896e-09\\
0.460002452135086 -5.66451019423653e-09\\
0.470002442598343 -1.07697006868079e-08\\
0.4800024330616 -1.18717347064035e-08\\
0.490002453327179 -1.00023243021496e-08\\
0.500002443790436 -6.52047882354623e-09\\
0.510002434253693 -2.70829381143756e-09\\
0.520002424716949 4.87552886951903e-10\\
0.530002415180206 2.55794807557663e-09\\
0.540002465248108 3.40432459999818e-09\\
0.550002455711365 3.22969895272252e-09\\
0.560002446174622 2.39778930044565e-09\\
0.570002436637878 1.30157373767759e-09\\
0.580002427101135 2.67199290471254e-10\\
0.590002417564392 -4.96727770027405e-10\\
0.600002467632294 -9.0770491123493e-10\\
0.610002458095551 -9.87435577748386e-10\\
0.620002448558807 -8.23812018602155e-10\\
0.630002439022064 -5.30460286807255e-10\\
0.640002429485321 -2.13447412522605e-10\\
0.650002419948578 4.97134937382793e-11\\
0.660002470016479 2.18082135683417e-10\\
0.670002460479736 2.84695933494561e-10\\
0.680002450942993 2.6725097135305e-10\\
0.69000244140625 1.96373237115743e-10\\
0.700002431869507 1.04723278659957e-10\\
0.710002422332764 1.91418270123478e-11\\
0.720002472400665 -4.3403267091513e-11\\
0.730002462863922 -7.64486390858465e-11\\
0.740002453327179 -8.20911394416868e-11\\
0.750002443790436 -6.78251413366304e-11\\
0.760002434253693 -4.3130308352568e-11\\
0.770002424716949 -1.67756832730737e-11\\
0.780002415180206 4.87781343094795e-12\\
0.790002465248108 1.85687975412518e-11\\
0.800002455711365 2.3792990147542e-11\\
0.810002446174622 2.20966994701755e-11\\
0.820002436637878 1.6075077033384e-11\\
0.830002427101135 8.41679590607436e-12\\
0.840002417564392 1.34093940378638e-12\\
0.850002467632294 -3.7796458671191e-12\\
0.860002458095551 -6.43508987416275e-12\\
0.870002448558807 -6.81968760976592e-12\\
0.880002439022064 -5.58371838349503e-12\\
0.890002429485321 -3.50762257780857e-12\\
0.900002419948578 -1.31035438576121e-12\\
0.910002470016479 4.69463335393827e-13\\
0.920002460479736 1.57579049635259e-12\\
0.930002450942993 1.98898341373377e-12\\
0.94000244140625 1.82955430319542e-12\\
0.950002431869507 1.31339459707308e-12\\
0.960002422332764 6.75954463843359e-13\\
0.970002472400665 8.3715132536636e-14\\
0.980002462863922 -3.15176785494961e-13\\
0.990002453327179 -5.37747147158485e-13\\
1.00000238418579 2.53402081540344e-08\\
1.01000249385834 0.00498206028714776\\
1.02000248432159 0.00997622683644295\\
1.03000247478485 0.0149767594411969\\
1.04000246524811 0.0199822336435318\\
1.05000245571136 0.0249901637434959\\
1.06000244617462 0.0299980640411377\\
1.07000243663788 0.035004198551178\\
1.08000242710114 0.0400077626109123\\
1.09000241756439 0.0450087711215019\\
1.10000240802765 0.050007801502943\\
1.11000239849091 0.0550056844949722\\
1.12000238895416 0.0600032433867455\\
1.13000249862671 0.0650011226534843\\
1.14000248908997 0.0699996799230576\\
1.15000247955322 0.074999026954174\\
1.16000247001648 0.079999066889286\\
1.17000246047974 0.0849995687603951\\
1.18000245094299 0.0900002717971802\\
1.19000244140625 0.0950009748339653\\
1.20000243186951 0.100001506507397\\
1.21000242233276 0.10500180721283\\
1.22000241279602 0.110001891851425\\
1.23000240325928 0.115001797676086\\
1.24000239372253 0.1200016066432\\
1.25000238418579 0.125001385807991\\
1.26000249385834 0.130001202225685\\
1.27000248432159 0.135001078248024\\
1.28000247478485 0.140001028776169\\
1.29000246524811 0.145001038908958\\
1.30000245571136 0.15000107884407\\
1.31000244617462 0.155001133680344\\
1.32000243663788 0.160001203417778\\
1.33000242710114 0.16500124335289\\
1.34000241756439 0.170001268386841\\
1.35000240802765 0.17500127851963\\
1.36000239849091 0.180001273751259\\
1.37000238895416 0.185001254081726\\
1.38000249862671 0.190001234412193\\
1.39000248908997 0.195001214742661\\
1.40000247955322 0.200001209974289\\
1.41000247001648 0.205001205205917\\
1.42000246047974 0.210001200437546\\
1.43000245094299 0.215001210570335\\
1.44000244140625 0.220001220703125\\
1.45000243186951 0.225001215934753\\
1.46000242233276 0.230001226067543\\
1.47000241279602 0.235001221299171\\
1.48000240325928 0.240001231431961\\
1.49000239372253 0.245001226663589\\
1.50000238418579 0.250001221895218\\
1.51000249385834 0.255001217126846\\
1.52000248432159 0.260001212358475\\
1.53000247478485 0.265001207590103\\
1.54000246524811 0.270001232624054\\
1.55000245571136 0.275001227855682\\
1.56000244617462 0.280001223087311\\
1.57000243663788 0.285001218318939\\
1.58000242710114 0.290001213550568\\
1.59000241756439 0.295001208782196\\
1.60000240802765 0.300001233816147\\
1.61000239849091 0.305001229047775\\
1.62000238895416 0.310001224279404\\
1.63000249862671 0.315001219511032\\
1.64000248908997 0.320001214742661\\
1.65000247955322 0.325001209974289\\
1.66000247001648 0.33000123500824\\
1.67000246047974 0.335001230239868\\
1.68000245094299 0.340001225471497\\
1.69000244140625 0.345001220703125\\
1.70000243186951 0.350001215934753\\
1.71000242233276 0.355001211166382\\
1.72000241279602 0.360001236200333\\
1.73000240325928 0.365001231431961\\
1.74000239372253 0.370001226663589\\
1.75000238418579 0.375001221895218\\
1.76000249385834 0.380001217126846\\
1.77000248432159 0.385001212358475\\
1.78000247478485 0.390001207590103\\
1.79000246524811 0.395001232624054\\
1.80000245571136 0.400001227855682\\
1.81000244617462 0.405001223087311\\
1.82000243663788 0.410001218318939\\
1.83000242710114 0.415001213550568\\
1.84000241756439 0.420001208782196\\
1.85000240802765 0.425001233816147\\
1.86000239849091 0.430001229047775\\
1.87000238895416 0.435001224279404\\
1.88000249862671 0.440001219511032\\
1.89000248908997 0.445001214742661\\
1.90000247955322 0.450001209974289\\
1.91000247001648 0.45500123500824\\
1.92000246047974 0.460001230239868\\
1.93000245094299 0.465001225471497\\
1.94000244140625 0.470001220703125\\
1.95000243186951 0.475001215934753\\
1.96000242233276 0.480001211166382\\
1.97000241279602 0.485001236200333\\
1.98000240325928 0.490001231431961\\
1.99000239372253 0.495001226663589\\
2.00000238418579 0.500001192092896\\
2.01000237464905 0.500019133090973\\
2.0200023651123 0.500024974346161\\
2.03000235557556 0.500024378299713\\
2.04000234603882 0.500018894672394\\
2.05000233650208 0.500011026859283\\
2.06000232696533 0.500003159046173\\
2.07000255584717 0.499997049570084\\
2.08000254631042 0.499993503093719\\
2.09000253677368 0.499992519617081\\
2.10000252723694 0.499993473291397\\
2.1100025177002 0.499995589256287\\
2.12000250816345 0.4999980032444\\
2.13000249862671 0.50000011920929\\
2.14000248908997 0.500001549720764\\
2.15000247955322 0.500002145767212\\
2.16000247001648 0.500002145767212\\
2.17000246047974 0.500001609325409\\
2.18000245094299 0.500000953674316\\
2.19000244140625 0.500000238418579\\
2.20000243186951 0.499999731779099\\
2.21000242233276 0.499999433755875\\
2.22000241279602 0.499999344348907\\
2.23000240325928 0.499999433755875\\
2.24000239372253 0.499999612569809\\
2.25000238418579 0.499999850988388\\
2.26000237464905 0.5\\
2.2700023651123 0.50000011920929\\
2.28000235557556 0.500000178813934\\
2.29000234603882 0.500000178813934\\
2.30000233650208 0.50000011920929\\
2.31000232696533 0.500000059604645\\
2.32000255584717 0.5\\
2.33000254631042 0.499999970197678\\
2.34000253677368 0.499999940395355\\
2.35000252723694 0.499999940395355\\
2.3600025177002 0.499999940395355\\
2.37000250816345 0.499999970197678\\
2.38000249862671 0.5\\
2.39000248908997 0.5\\
2.40000247955322 0.5\\
2.41000247001648 0.5\\
2.42000246047974 0.5\\
2.43000245094299 0.5\\
2.44000244140625 0.5\\
2.45000243186951 0.5\\
2.46000242233276 0.5\\
2.47000241279602 0.5\\
2.48000240325928 0.5\\
2.49000239372253 0.5\\
2.50000238418579 0.5\\
2.51000237464905 0.5\\
2.5200023651123 0.5\\
2.53000235557556 0.5\\
2.54000234603882 0.5\\
2.55000233650208 0.5\\
2.56000232696533 0.5\\
2.57000255584717 0.5\\
2.58000254631042 0.5\\
2.59000253677368 0.5\\
2.60000252723694 0.5\\
2.6100025177002 0.5\\
2.62000250816345 0.5\\
2.63000249862671 0.5\\
2.64000248908997 0.5\\
2.65000247955322 0.5\\
2.66000247001648 0.5\\
2.67000246047974 0.5\\
2.68000245094299 0.5\\
2.69000244140625 0.5\\
2.70000243186951 0.5\\
2.71000242233276 0.5\\
2.72000241279602 0.5\\
2.73000240325928 0.5\\
2.74000239372253 0.5\\
2.75000238418579 0.5\\
2.76000237464905 0.5\\
2.7700023651123 0.5\\
2.78000235557556 0.5\\
2.79000234603882 0.5\\
2.80000233650208 0.5\\
2.81000232696533 0.5\\
2.82000255584717 0.5\\
2.83000254631042 0.5\\
2.84000253677368 0.5\\
2.85000252723694 0.5\\
2.8600025177002 0.5\\
2.87000250816345 0.5\\
2.88000249862671 0.5\\
2.89000248908997 0.5\\
2.90000247955322 0.5\\
2.91000247001648 0.5\\
2.92000246047974 0.5\\
2.93000245094299 0.5\\
2.94000244140625 0.5\\
2.95000243186951 0.5\\
2.96000242233276 0.5\\
2.97000241279602 0.5\\
2.98000240325928 0.5\\
2.99000239372253 0.5\\
3.00000238418579 0.5\\
3.01000237464905 0.5\\
3.0200023651123 0.5\\
3.03000235557556 0.5\\
3.04000234603882 0.5\\
3.05000233650208 0.5\\
3.06000232696533 0.5\\
3.07000255584717 0.5\\
3.08000254631042 0.5\\
3.09000253677368 0.5\\
3.10000252723694 0.5\\
3.1100025177002 0.5\\
3.12000250816345 0.5\\
3.13000249862671 0.5\\
3.14000248908997 0.5\\
3.15000247955322 0.5\\
3.16000247001648 0.5\\
3.17000246047974 0.5\\
3.18000245094299 0.5\\
3.19000244140625 0.5\\
3.20000243186951 0.5\\
3.21000242233276 0.5\\
3.22000241279602 0.5\\
3.23000240325928 0.5\\
3.24000239372253 0.5\\
3.25000238418579 0.5\\
3.26000237464905 0.5\\
3.2700023651123 0.5\\
3.28000235557556 0.5\\
3.29000234603882 0.5\\
3.30000233650208 0.5\\
3.31000232696533 0.5\\
3.32000255584717 0.5\\
3.33000254631042 0.5\\
3.34000253677368 0.5\\
3.35000252723694 0.5\\
3.3600025177002 0.5\\
3.37000250816345 0.5\\
3.38000249862671 0.5\\
3.39000248908997 0.5\\
3.40000247955322 0.5\\
3.41000247001648 0.5\\
3.42000246047974 0.5\\
3.43000245094299 0.5\\
3.44000244140625 0.5\\
3.45000243186951 0.5\\
3.46000242233276 0.5\\
3.47000241279602 0.5\\
3.48000240325928 0.5\\
3.49000239372253 0.5\\
3.50000238418579 0.5\\
3.51000237464905 0.5\\
3.5200023651123 0.5\\
3.53000235557556 0.5\\
3.54000234603882 0.5\\
3.55000233650208 0.5\\
3.56000232696533 0.5\\
3.57000255584717 0.5\\
3.58000254631042 0.5\\
3.59000253677368 0.5\\
3.60000252723694 0.5\\
3.6100025177002 0.5\\
3.62000250816345 0.5\\
3.63000249862671 0.5\\
3.64000248908997 0.5\\
3.65000247955322 0.5\\
3.66000247001648 0.5\\
3.67000246047974 0.5\\
3.68000245094299 0.5\\
3.69000244140625 0.5\\
3.70000243186951 0.5\\
3.71000242233276 0.5\\
3.72000241279602 0.5\\
3.73000240325928 0.5\\
3.74000239372253 0.5\\
3.75000238418579 0.5\\
3.76000237464905 0.5\\
3.7700023651123 0.5\\
3.78000235557556 0.5\\
3.79000234603882 0.5\\
3.80000233650208 0.5\\
3.81000232696533 0.5\\
3.82000255584717 0.5\\
3.83000254631042 0.5\\
3.84000253677368 0.5\\
3.85000252723694 0.5\\
3.8600025177002 0.5\\
3.87000250816345 0.5\\
3.88000249862671 0.5\\
3.89000248908997 0.5\\
3.90000247955322 0.5\\
3.91000247001648 0.5\\
3.92000246047974 0.5\\
3.93000245094299 0.5\\
3.94000244140625 0.5\\
3.95000243186951 0.5\\
3.96000242233276 0.5\\
3.97000241279602 0.5\\
3.98000240325928 0.5\\
3.99000239372253 0.5\\
4.00000238418579 0.5\\
4.01000261306763 0.5\\
4.0200023651123 0.5\\
4.03000259399414 0.5\\
4.04000234603882 0.5\\
4.05000257492065 0.5\\
4.06000232696533 0.5\\
4.07000255584717 0.5\\
4.08000230789185 0.5\\
4.09000253677368 0.5\\
4.10000228881836 0.5\\
4.1100025177002 0.5\\
4.12000226974487 0.5\\
4.13000249862671 0.5\\
4.14000225067139 0.5\\
4.15000247955322 0.5\\
4.1600022315979 0.5\\
4.17000246047974 0.5\\
4.18000221252441 0.5\\
4.19000244140625 0.5\\
4.20000267028809 0.5\\
4.21000242233276 0.5\\
4.2200026512146 0.5\\
4.23000240325928 0.5\\
4.24000263214111 0.5\\
4.25000238418579 0.5\\
4.26000261306763 0.5\\
4.2700023651123 0.5\\
4.28000259399414 0.5\\
4.29000234603882 0.5\\
4.30000257492065 0.5\\
4.31000232696533 0.5\\
4.32000255584717 0.5\\
4.33000230789185 0.5\\
4.34000253677368 0.5\\
4.35000228881836 0.5\\
4.3600025177002 0.5\\
4.37000226974487 0.5\\
4.38000249862671 0.5\\
4.39000225067139 0.5\\
4.40000247955322 0.5\\
4.4100022315979 0.5\\
4.42000246047974 0.5\\
4.43000221252441 0.5\\
4.44000244140625 0.5\\
4.45000267028809 0.5\\
4.46000242233276 0.5\\
4.4700026512146 0.5\\
4.48000240325928 0.5\\
4.49000263214111 0.5\\
4.50000238418579 0.5\\
4.51000261306763 0.5\\
4.5200023651123 0.5\\
4.53000259399414 0.5\\
4.54000234603882 0.5\\
4.55000257492065 0.5\\
4.56000232696533 0.5\\
4.57000255584717 0.5\\
4.58000230789185 0.5\\
4.59000253677368 0.5\\
4.60000228881836 0.5\\
4.6100025177002 0.5\\
4.62000226974487 0.5\\
4.63000249862671 0.5\\
4.64000225067139 0.5\\
4.65000247955322 0.5\\
4.6600022315979 0.5\\
4.67000246047974 0.5\\
4.68000221252441 0.5\\
4.69000244140625 0.5\\
4.70000267028809 0.5\\
4.71000242233276 0.5\\
4.7200026512146 0.5\\
4.73000240325928 0.5\\
4.74000263214111 0.5\\
4.75000238418579 0.5\\
4.76000261306763 0.5\\
4.7700023651123 0.5\\
4.78000259399414 0.5\\
4.79000234603882 0.5\\
4.80000257492065 0.5\\
4.81000232696533 0.5\\
4.82000255584717 0.5\\
4.83000230789185 0.5\\
4.84000253677368 0.5\\
4.85000228881836 0.5\\
4.8600025177002 0.5\\
4.87000226974487 0.5\\
4.88000249862671 0.5\\
4.89000225067139 0.5\\
4.90000247955322 0.5\\
4.9100022315979 0.5\\
4.92000246047974 0.5\\
4.93000221252441 0.5\\
4.94000244140625 0.5\\
4.95000267028809 0.5\\
4.96000242233276 0.5\\
4.9700026512146 0.5\\
4.98000240325928 0.5\\
4.99000263214111 0.5\\
5 0.5\\
};
\addlegendentry{Linear Solver};

\addplot [
color=black,
solid
]
table[row sep=crcr]{
0 0\\
0.0100024435669184 2.84935958916321e-05\\
0.020002443343401 6.37793418718502e-05\\
0.0300024431198835 6.68920474709012e-05\\
0.0400024428963661 5.20567446073983e-05\\
0.0500024445354939 3.0215727747418e-05\\
0.0600024424493313 8.61130592966219e-06\\
0.070002444088459 -8.09265384305036e-06\\
0.0800024420022964 -1.7758486137609e-05\\
0.0900024399161339 -2.04607840714743e-05\\
0.100002445280552 -1.77914389496436e-05\\
0.110002443194389 -1.20357135529048e-05\\
0.120002441108227 -5.44300200999714e-06\\
0.130002439022064 2.65806392008017e-07\\
0.140002444386482 4.10107259085635e-06\\
0.1500024497509 5.80632058699848e-06\\
0.160002440214157 5.68720497540198e-06\\
0.170002445578575 4.3505142457434e-06\\
0.180002436041832 2.47896650762414e-06\\
0.19000244140625 6.56121926567721e-07\\
0.200002446770668 -7.31034162981814e-07\\
0.210002437233925 -1.51496624312131e-06\\
0.220002442598343 -1.71472174770315e-06\\
0.230002447962761 -1.47226819535717e-06\\
0.240002438426018 -9.82024175755214e-07\\
0.250002443790436 -4.31158440505897e-07\\
0.260002434253693 3.94325461172684e-08\\
0.270002454519272 3.51441514112594e-07\\
0.280002444982529 4.86504973196134e-07\\
0.290002435445786 4.71165435556031e-07\\
0.300002455711365 3.56710614823896e-07\\
0.310002446174622 1.99971310621549e-07\\
0.320002436637878 4.90119340668116e-08\\
0.330002456903458 -6.46869366960345e-08\\
0.340002447366714 -1.27895532386901e-07\\
0.350002437829971 -1.42742365483173e-07\\
0.360002458095551 -1.21353011195424e-07\\
0.370002448558807 -7.99895687464414e-08\\
0.380002439022064 -3.4143766924899e-08\\
0.390002429485321 4.63722615862139e-09\\
0.4000024497509 3.00433278255241e-08\\
0.410002440214157 4.0726362726673e-08\\
0.420002430677414 3.90173902076185e-08\\
0.430002450942993 2.9240371901551e-08\\
0.44000244140625 1.61229660733397e-08\\
0.450002431869507 3.62447449830938e-09\\
0.460002452135086 -5.69340752321068e-09\\
0.470002442598343 -1.07869384535775e-08\\
0.4800024330616 -1.18772280899293e-08\\
0.490002453327179 -9.99859217643007e-09\\
0.500002443790436 -6.51128662099154e-09\\
0.510002434253693 -2.69740541014585e-09\\
0.520002424716949 4.97196450677251e-10\\
0.530002415180206 2.56460697123373e-09\\
0.540002465248108 3.40747385862983e-09\\
0.550002455711365 3.22977200539754e-09\\
0.560002446174622 2.39574271532206e-09\\
0.570002436637878 1.29853805486135e-09\\
0.580002427101135 2.64148924955521e-10\\
0.590002417564392 -4.99100094586424e-10\\
0.600002467632294 -9.09093356149526e-10\\
0.610002458095551 -9.87850801159595e-10\\
0.620002448558807 -8.23476786759869e-10\\
0.630002439022064 -5.29681021266271e-10\\
0.640002429485321 -2.12546549804848e-10\\
0.650002419948578 5.05096554859197e-11\\
0.660002470016479 2.18632237314331e-10\\
0.670002460479736 2.84943763029233e-10\\
0.680002450942993 2.67249944396752e-10\\
0.69000244140625 1.96192978529908e-10\\
0.700002431869507 1.04465727734926e-10\\
0.710002422332764 1.88850532434337e-11\\
0.720002472400665 -4.36005884174584e-11\\
0.730002462863922 -7.65580862593929e-11\\
0.740002453327179 -8.21291368247046e-11\\
0.750002443790436 -6.77863390419198e-11\\
0.760002434253693 -4.30613010526937e-11\\
0.770002424716949 -1.66997422834658e-11\\
0.780002415180206 4.95288489041346e-12\\
0.790002465248108 1.8621935590768e-11\\
0.800002455711365 2.38086651088709e-11\\
0.810002446174622 2.21057408489322e-11\\
0.820002436637878 1.60564495726989e-11\\
0.830002427101135 8.39203793262522e-12\\
0.840002417564392 1.31374436069392e-12\\
0.850002467632294 -3.8065891584671e-12\\
0.860002458095551 -6.44024590601422e-12\\
0.870002448558807 -6.82633376908326e-12\\
0.880002439022064 -5.58278250017974e-12\\
0.890002429485321 -3.50029011851605e-12\\
0.900002419948578 -1.31448194343187e-12\\
0.910002470016479 4.77750163648677e-13\\
0.920002460479736 1.58609703062446e-12\\
0.930002450942993 1.99052710078695e-12\\
0.94000244140625 1.83220019017716e-12\\
0.950002431869507 1.31314945896188e-12\\
0.960002422332764 6.69692708406278e-13\\
0.970002472400665 9.29487606673995e-14\\
0.980002462863922 -3.24397301555701e-13\\
0.990002453327179 -5.36841567293916e-13\\
1.00000238418579 2.53402099303912e-08\\
1.01000249385834 0.00498206028714776\\
1.02000248432159 0.00997622683644295\\
1.03000247478485 0.0149767594411969\\
1.04000246524811 0.0199822336435318\\
1.05000245571136 0.0249901637434959\\
1.06000244617462 0.0299980640411377\\
1.07000243663788 0.035004198551178\\
1.08000242710114 0.0400077626109123\\
1.09000241756439 0.0450087711215019\\
1.10000240802765 0.050007801502943\\
1.11000239849091 0.0550056844949722\\
1.12000238895416 0.0600032433867455\\
1.13000249862671 0.0650011226534843\\
1.14000248908997 0.0699996799230576\\
1.15000247955322 0.074999026954174\\
1.16000247001648 0.079999066889286\\
1.17000246047974 0.0849995687603951\\
1.18000245094299 0.0900002717971802\\
1.19000244140625 0.0950009748339653\\
1.20000243186951 0.100001506507397\\
1.21000242233276 0.10500180721283\\
1.22000241279602 0.110001891851425\\
1.23000240325928 0.115001797676086\\
1.24000239372253 0.1200016066432\\
1.25000238418579 0.125001385807991\\
1.26000249385834 0.130001202225685\\
1.27000248432159 0.135001078248024\\
1.28000247478485 0.140001028776169\\
1.29000246524811 0.145001038908958\\
1.30000245571136 0.15000107884407\\
1.31000244617462 0.155001133680344\\
1.32000243663788 0.160001203417778\\
1.33000242710114 0.16500124335289\\
1.34000241756439 0.170001268386841\\
1.35000240802765 0.17500127851963\\
1.36000239849091 0.180001273751259\\
1.37000238895416 0.185001254081726\\
1.38000249862671 0.190001234412193\\
1.39000248908997 0.195001214742661\\
1.40000247955322 0.200001209974289\\
1.41000247001648 0.205001205205917\\
1.42000246047974 0.210001200437546\\
1.43000245094299 0.215001210570335\\
1.44000244140625 0.220001220703125\\
1.45000243186951 0.225001215934753\\
1.46000242233276 0.230001226067543\\
1.47000241279602 0.235001221299171\\
1.48000240325928 0.240001231431961\\
1.49000239372253 0.245001226663589\\
1.50000238418579 0.250001221895218\\
1.51000249385834 0.255001217126846\\
1.52000248432159 0.260001212358475\\
1.53000247478485 0.265001207590103\\
1.54000246524811 0.270001232624054\\
1.55000245571136 0.275001227855682\\
1.56000244617462 0.280001223087311\\
1.57000243663788 0.285001218318939\\
1.58000242710114 0.290001213550568\\
1.59000241756439 0.295001208782196\\
1.60000240802765 0.300001233816147\\
1.61000239849091 0.305001229047775\\
1.62000238895416 0.310001224279404\\
1.63000249862671 0.315001219511032\\
1.64000248908997 0.320001214742661\\
1.65000247955322 0.325001209974289\\
1.66000247001648 0.33000123500824\\
1.67000246047974 0.335001230239868\\
1.68000245094299 0.340001225471497\\
1.69000244140625 0.345001220703125\\
1.70000243186951 0.350001215934753\\
1.71000242233276 0.355001211166382\\
1.72000241279602 0.360001236200333\\
1.73000240325928 0.365001231431961\\
1.74000239372253 0.370001226663589\\
1.75000238418579 0.375001221895218\\
1.76000249385834 0.380001217126846\\
1.77000248432159 0.385001212358475\\
1.78000247478485 0.390001207590103\\
1.79000246524811 0.395001232624054\\
1.80000245571136 0.400001227855682\\
1.81000244617462 0.405001223087311\\
1.82000243663788 0.410001218318939\\
1.83000242710114 0.415001213550568\\
1.84000241756439 0.420001208782196\\
1.85000240802765 0.425001233816147\\
1.86000239849091 0.430001229047775\\
1.87000238895416 0.435001224279404\\
1.88000249862671 0.440001219511032\\
1.89000248908997 0.445001214742661\\
1.90000247955322 0.450001209974289\\
1.91000247001648 0.45500123500824\\
1.92000246047974 0.460001230239868\\
1.93000245094299 0.465001225471497\\
1.94000244140625 0.470001220703125\\
1.95000243186951 0.475001215934753\\
1.96000242233276 0.480001211166382\\
1.97000241279602 0.485001236200333\\
1.98000240325928 0.490001231431961\\
1.99000239372253 0.495001226663589\\
2.00000238418579 0.500001192092896\\
2.01000237464905 0.500019133090973\\
2.0200023651123 0.500024974346161\\
2.03000235557556 0.500024378299713\\
2.04000234603882 0.500018894672394\\
2.05000233650208 0.500011026859283\\
2.06000232696533 0.500003159046173\\
2.07000255584717 0.499997049570084\\
2.08000254631042 0.499993503093719\\
2.09000253677368 0.499992519617081\\
2.10000252723694 0.499993473291397\\
2.1100025177002 0.499995589256287\\
2.12000250816345 0.4999980032444\\
2.13000249862671 0.50000011920929\\
2.14000248908997 0.500001549720764\\
2.15000247955322 0.500002145767212\\
2.16000247001648 0.500002145767212\\
2.17000246047974 0.500001609325409\\
2.18000245094299 0.500000953674316\\
2.19000244140625 0.500000238418579\\
2.20000243186951 0.499999731779099\\
2.21000242233276 0.499999433755875\\
2.22000241279602 0.499999344348907\\
2.23000240325928 0.499999433755875\\
2.24000239372253 0.499999612569809\\
2.25000238418579 0.499999850988388\\
2.26000237464905 0.5\\
2.2700023651123 0.50000011920929\\
2.28000235557556 0.500000178813934\\
2.29000234603882 0.500000178813934\\
2.30000233650208 0.50000011920929\\
2.31000232696533 0.500000059604645\\
2.32000255584717 0.5\\
2.33000254631042 0.499999970197678\\
2.34000253677368 0.499999940395355\\
2.35000252723694 0.499999940395355\\
2.3600025177002 0.499999940395355\\
2.37000250816345 0.499999970197678\\
2.38000249862671 0.5\\
2.39000248908997 0.5\\
2.40000247955322 0.5\\
2.41000247001648 0.5\\
2.42000246047974 0.5\\
2.43000245094299 0.5\\
2.44000244140625 0.5\\
2.45000243186951 0.5\\
2.46000242233276 0.5\\
2.47000241279602 0.5\\
2.48000240325928 0.5\\
2.49000239372253 0.5\\
2.50000238418579 0.5\\
2.51000237464905 0.5\\
2.5200023651123 0.5\\
2.53000235557556 0.5\\
2.54000234603882 0.5\\
2.55000233650208 0.5\\
2.56000232696533 0.5\\
2.57000255584717 0.5\\
2.58000254631042 0.5\\
2.59000253677368 0.5\\
2.60000252723694 0.5\\
2.6100025177002 0.5\\
2.62000250816345 0.5\\
2.63000249862671 0.5\\
2.64000248908997 0.5\\
2.65000247955322 0.5\\
2.66000247001648 0.5\\
2.67000246047974 0.5\\
2.68000245094299 0.5\\
2.69000244140625 0.5\\
2.70000243186951 0.5\\
2.71000242233276 0.5\\
2.72000241279602 0.5\\
2.73000240325928 0.5\\
2.74000239372253 0.5\\
2.75000238418579 0.5\\
2.76000237464905 0.5\\
2.7700023651123 0.5\\
2.78000235557556 0.5\\
2.79000234603882 0.5\\
2.80000233650208 0.5\\
2.81000232696533 0.5\\
2.82000255584717 0.5\\
2.83000254631042 0.5\\
2.84000253677368 0.5\\
2.85000252723694 0.5\\
2.8600025177002 0.5\\
2.87000250816345 0.5\\
2.88000249862671 0.5\\
2.89000248908997 0.5\\
2.90000247955322 0.5\\
2.91000247001648 0.5\\
2.92000246047974 0.5\\
2.93000245094299 0.5\\
2.94000244140625 0.5\\
2.95000243186951 0.5\\
2.96000242233276 0.5\\
2.97000241279602 0.5\\
2.98000240325928 0.5\\
2.99000239372253 0.5\\
3.00000238418579 0.5\\
3.01000237464905 0.5\\
3.0200023651123 0.5\\
3.03000235557556 0.5\\
3.04000234603882 0.5\\
3.05000233650208 0.5\\
3.06000232696533 0.5\\
3.07000255584717 0.5\\
3.08000254631042 0.5\\
3.09000253677368 0.5\\
3.10000252723694 0.5\\
3.1100025177002 0.5\\
3.12000250816345 0.5\\
3.13000249862671 0.5\\
3.14000248908997 0.5\\
3.15000247955322 0.5\\
3.16000247001648 0.5\\
3.17000246047974 0.5\\
3.18000245094299 0.5\\
3.19000244140625 0.5\\
3.20000243186951 0.5\\
3.21000242233276 0.5\\
3.22000241279602 0.5\\
3.23000240325928 0.5\\
3.24000239372253 0.5\\
3.25000238418579 0.5\\
3.26000237464905 0.5\\
3.2700023651123 0.5\\
3.28000235557556 0.5\\
3.29000234603882 0.5\\
3.30000233650208 0.5\\
3.31000232696533 0.5\\
3.32000255584717 0.5\\
3.33000254631042 0.5\\
3.34000253677368 0.5\\
3.35000252723694 0.5\\
3.3600025177002 0.5\\
3.37000250816345 0.5\\
3.38000249862671 0.5\\
3.39000248908997 0.5\\
3.40000247955322 0.5\\
3.41000247001648 0.5\\
3.42000246047974 0.5\\
3.43000245094299 0.5\\
3.44000244140625 0.5\\
3.45000243186951 0.5\\
3.46000242233276 0.5\\
3.47000241279602 0.5\\
3.48000240325928 0.5\\
3.49000239372253 0.5\\
3.50000238418579 0.5\\
3.51000237464905 0.5\\
3.5200023651123 0.5\\
3.53000235557556 0.5\\
3.54000234603882 0.5\\
3.55000233650208 0.5\\
3.56000232696533 0.5\\
3.57000255584717 0.5\\
3.58000254631042 0.5\\
3.59000253677368 0.5\\
3.60000252723694 0.5\\
3.6100025177002 0.5\\
3.62000250816345 0.5\\
3.63000249862671 0.5\\
3.64000248908997 0.5\\
3.65000247955322 0.5\\
3.66000247001648 0.5\\
3.67000246047974 0.5\\
3.68000245094299 0.5\\
3.69000244140625 0.5\\
3.70000243186951 0.5\\
3.71000242233276 0.5\\
3.72000241279602 0.5\\
3.73000240325928 0.5\\
3.74000239372253 0.5\\
3.75000238418579 0.5\\
3.76000237464905 0.5\\
3.7700023651123 0.5\\
3.78000235557556 0.5\\
3.79000234603882 0.5\\
3.80000233650208 0.5\\
3.81000232696533 0.5\\
3.82000255584717 0.5\\
3.83000254631042 0.5\\
3.84000253677368 0.5\\
3.85000252723694 0.5\\
3.8600025177002 0.5\\
3.87000250816345 0.5\\
3.88000249862671 0.5\\
3.89000248908997 0.5\\
3.90000247955322 0.5\\
3.91000247001648 0.5\\
3.92000246047974 0.5\\
3.93000245094299 0.5\\
3.94000244140625 0.5\\
3.95000243186951 0.5\\
3.96000242233276 0.5\\
3.97000241279602 0.5\\
3.98000240325928 0.5\\
3.99000239372253 0.5\\
4.00000238418579 0.5\\
4.01000261306763 0.5\\
4.0200023651123 0.5\\
4.03000259399414 0.5\\
4.04000234603882 0.5\\
4.05000257492065 0.5\\
4.06000232696533 0.5\\
4.07000255584717 0.5\\
4.08000230789185 0.5\\
4.09000253677368 0.5\\
4.10000228881836 0.5\\
4.1100025177002 0.5\\
4.12000226974487 0.5\\
4.13000249862671 0.5\\
4.14000225067139 0.5\\
4.15000247955322 0.5\\
4.1600022315979 0.5\\
4.17000246047974 0.5\\
4.18000221252441 0.5\\
4.19000244140625 0.5\\
4.20000267028809 0.5\\
4.21000242233276 0.5\\
4.2200026512146 0.5\\
4.23000240325928 0.5\\
4.24000263214111 0.5\\
4.25000238418579 0.5\\
4.26000261306763 0.5\\
4.2700023651123 0.5\\
4.28000259399414 0.5\\
4.29000234603882 0.5\\
4.30000257492065 0.5\\
4.31000232696533 0.5\\
4.32000255584717 0.5\\
4.33000230789185 0.5\\
4.34000253677368 0.5\\
4.35000228881836 0.5\\
4.3600025177002 0.5\\
4.37000226974487 0.5\\
4.38000249862671 0.5\\
4.39000225067139 0.5\\
4.40000247955322 0.5\\
4.4100022315979 0.5\\
4.42000246047974 0.5\\
4.43000221252441 0.5\\
4.44000244140625 0.5\\
4.45000267028809 0.5\\
4.46000242233276 0.5\\
4.4700026512146 0.5\\
4.48000240325928 0.5\\
4.49000263214111 0.5\\
4.50000238418579 0.5\\
4.51000261306763 0.5\\
4.5200023651123 0.5\\
4.53000259399414 0.5\\
4.54000234603882 0.5\\
4.55000257492065 0.5\\
4.56000232696533 0.5\\
4.57000255584717 0.5\\
4.58000230789185 0.5\\
4.59000253677368 0.5\\
4.60000228881836 0.5\\
4.6100025177002 0.5\\
4.62000226974487 0.5\\
4.63000249862671 0.5\\
4.64000225067139 0.5\\
4.65000247955322 0.5\\
4.6600022315979 0.5\\
4.67000246047974 0.5\\
4.68000221252441 0.5\\
4.69000244140625 0.5\\
4.70000267028809 0.5\\
4.71000242233276 0.5\\
4.7200026512146 0.5\\
4.73000240325928 0.5\\
4.74000263214111 0.5\\
4.75000238418579 0.5\\
4.76000261306763 0.5\\
4.7700023651123 0.5\\
4.78000259399414 0.5\\
4.79000234603882 0.5\\
4.80000257492065 0.5\\
4.81000232696533 0.5\\
4.82000255584717 0.5\\
4.83000230789185 0.5\\
4.84000253677368 0.5\\
4.85000228881836 0.5\\
4.8600025177002 0.5\\
4.87000226974487 0.5\\
4.88000249862671 0.5\\
4.89000225067139 0.5\\
4.90000247955322 0.5\\
4.9100022315979 0.5\\
4.92000246047974 0.5\\
4.93000221252441 0.5\\
4.94000244140625 0.5\\
4.95000267028809 0.5\\
4.96000242233276 0.5\\
4.9700026512146 0.5\\
4.98000240325928 0.5\\
4.99000263214111 0.5\\
5 0.5\\
};
\addlegendentry{Nonlinear Solver};

\end{axis}
\end{tikzpicture}%
\caption{Single-phase solution with \dtmax{} = 5.0 {[s]}.}
\label{fig:single1pt000em5}
\end{figure}

The single-phase case was designed to test if the linear solver produced a simulation result that was equivalent to that produced by the nonlinear solver.
More specifically, it was designed to show that for problems where the physics of interest have relatively low nonlinearities, the linear solver provides as accurate a solution as the nonlinear solver.
\fig{fig:single1pt000em0} and \fig{fig:single1pt000em5} show the solution produced by both the nonlinear and linear solvers of \cobra{}.
The solutions produced by both solvers are qualitatively equivalent at \dtmax{} = 1.0 [s] and \dtmax{} = \expneg{1.0}{5} [s].
This indicates that the linear solver is adequate in regions where the solution is not highly nonlinear.

For the flashing problem, the parameter of interest for temporal convergence testing is $\alpha_g$ at 2 [in] from the inlet of the pipe.
This parameter is chosen because it represents the region where the nonlinearities should have the largest impact on the solution as this is the location where the flashing occurs.
As such, this location will be sensitive to the rate of vapor formation from the thermodynamically unstable liquid that enters the cell.
\fig{fig:flashing1pt0000em0} shows the gaseous volume fraction 2 [in] from the inlet of the pipe as a function of time for a \dtmax{} of \expneg{1.0}{1} [s] for both the linear and the nonlinear solvers.

\begin{figure}[h!tb]
\centering
% This file was created by matlab2tikz v0.4.3.
% Copyright (c) 2008--2013, Nico Schlömer <nico.schloemer@gmail.com>
% All rights reserved.
% 
\tikzsetnextfilename{plots/flashing1pt0000em0_eps}
\begin{tikzpicture}

\begin{axis}[%
width=\mytikzpicwidth,
height=\mytikzpicheight,
scale only axis,
xmin=0.0,
xmax=5.0,
xlabel={Time $[\text{s}]$},
ymin=0.5,
ymax=1.0,
ylabel={$\alpha_g$ [-] @ 2 [in] from Inlet},
legend style={draw=black,fill=white,legend cell align=left}
]
\addplot [
color=black,
solid
]
table[row sep=crcr]{
0.0 1.0\\
0.0118019115179777 1\\
0.0226854234933853 1\\
0.0340500771999359 1\\
0.0491764321923256 1\\
0.0619497485458851 1\\
0.0774054601788521 1\\
0.0961068719625473 1\\
0.106882445514202 1\\
0.118735581636429 1\\
0.13177402317524 1\\
0.146116316318512 1\\
0.161892831325531 1\\
0.179247006773949 1\\
0.198336601257324 1\\
0.21933513879776 1\\
0.242433547973633 1\\
0.267841786146164 1\\
0.29579085111618 1\\
0.32653483748436 1\\
0.360353201627731 1\\
0.397553414106369 1\\
0.438473641872406 1\\
0.483485877513886 1\\
0.532999336719513 1\\
0.587464153766632 1\\
0.647375464439392 1\\
0.71327793598175 1\\
0.7857705950737 1\\
0.865512549877167 1\\
0.953228712081909 1\\
1.04971647262573 0.997021615505219\\
1.1558530330658 0.985307455062866\\
1.2407523393631 0.970276594161987\\
1.29757130146027 0.957477152347565\\
1.34479975700378 0.945545017719269\\
1.38458728790283 0.934612452983856\\
1.41912090778351 0.924424767494202\\
1.45013403892517 0.914681613445282\\
1.47872579097748 0.90517920255661\\
1.5054817199707 0.895826578140259\\
1.53073275089264 0.88658881187439\\
1.55469310283661 0.877453088760376\\
1.57754015922546 0.868404984474182\\
1.59942138195038 0.859430253505707\\
1.62045061588287 0.850519120693207\\
1.64072275161743 0.841663300991058\\
1.6603125333786 0.832857310771942\\
1.67928779125214 0.824094474315643\\
1.69770324230194 0.815370202064514\\
1.71560847759247 0.806679904460907\\
1.73303949832916 0.79802268743515\\
1.75003552436829 0.789430022239685\\
1.76655149459839 0.781055510044098\\
1.78238117694855 0.772998511791229\\
1.79760921001434 0.765193462371826\\
1.81230998039246 0.757585883140564\\
1.82662653923035 0.750090539455414\\
1.84061861038208 0.742669582366943\\
1.85433006286621 0.735296964645386\\
1.86779499053955 0.72795307636261\\
1.88103151321411 0.720627963542938\\
1.89406371116638 0.71331000328064\\
1.90690815448761 0.70599102973938\\
1.91956806182861 0.698671102523804\\
1.93206679821014 0.691338658332825\\
1.94440972805023 0.683992326259613\\
1.95660436153412 0.676629960536957\\
1.96866297721863 0.671738088130951\\
1.98027718067169 0.672812759876251\\
1.99125838279724 0.677815139293671\\
2.00170540809631 0.684370279312134\\
2.01211428642273 0.691331386566162\\
2.02296447753906 0.698101699352264\\
2.03420996665955 0.704113781452179\\
2.04571580886841 0.709079086780548\\
2.05740904808044 0.712941884994507\\
2.06921482086182 0.71580046415329\\
2.08101534843445 0.717850506305695\\
2.09267902374268 0.719229459762573\\
2.10420393943787 0.720068991184235\\
2.1155960559845 0.720495879650116\\
2.12691164016724 0.720689058303833\\
2.13817572593689 0.721016407012939\\
2.14939737319946 0.722309648990631\\
2.16053771972656 0.725886106491089\\
2.17153453826904 0.732412099838257\\
2.18234181404114 0.740104675292969\\
2.19297122955322 0.747277319431305\\
2.2034854888916 0.753501296043396\\
2.21399807929993 0.758969664573669\\
2.22421646118164 0.76383113861084\\
2.24485516548157 0.773090600967407\\
2.2556414604187 0.777811586856842\\
2.26643514633179 0.78242814540863\\
2.27722930908203 0.786858320236206\\
2.28802180290222 0.791011035442352\\
2.2988224029541 0.794807195663452\\
2.30910420417786 0.798032104969025\\
2.32745814323425 0.80274897813797\\
2.34450769424438 0.805849254131317\\
2.36390542984009 0.807824075222015\\
2.37500143051147 0.808235883712769\\
2.38614940643311 0.808172166347504\\
2.39734673500061 0.807666897773743\\
2.40859293937683 0.806699872016907\\
2.41979336738586 0.805263638496399\\
2.43107962608337 0.802674949169159\\
2.44180870056152 0.79903507232666\\
2.4530987739563 0.793921113014221\\
2.46468305587769 0.787385225296021\\
2.47643661499023 0.779563665390015\\
2.48832750320435 0.77061265707016\\
2.50033926963806 0.760712027549744\\
2.5124671459198 0.750045478343964\\
2.5246946811676 0.738811135292053\\
2.53700971603394 0.727198243141174\\
2.54939413070679 0.71539968252182\\
2.56182551383972 0.703596353530884\\
2.57428097724915 0.691951513290405\\
2.58673787117004 0.680607497692108\\
2.59917783737183 0.669684410095215\\
2.61158609390259 0.659270703792572\\
2.62395215034485 0.649651288986206\\
2.6361358165741 0.641215026378632\\
2.64810156822205 0.635994970798492\\
2.6597592830658 0.635662436485291\\
2.67102980613709 0.639588057994843\\
2.68197822570801 0.64653742313385\\
2.69267916679382 0.655223429203033\\
2.70332479476929 0.665210664272308\\
2.71397757530212 0.676258027553558\\
2.72448706626892 0.687942206859589\\
2.74232935905457 0.708736419677734\\
2.75233578681946 0.723106145858765\\
2.76248455047607 0.746678531169891\\
2.77299618721008 0.767698109149933\\
2.78397274017334 0.786229491233826\\
2.79858231544495 0.805485188961029\\
2.81091928482056 0.817172288894653\\
2.82584691047668 0.826655030250549\\
2.84390950202942 0.832634270191193\\
2.85421967506409 0.83385705947876\\
2.86443328857422 0.83380115032196\\
2.87539649009705 0.832415580749512\\
2.88654828071594 0.829780161380768\\
2.89785838127136 0.825988829135895\\
2.90905976295471 0.821340024471283\\
2.92766308784485 0.812942206859589\\
2.9380214214325 0.808586359024048\\
2.94941568374634 0.80469822883606\\
2.96126461029053 0.801513075828552\\
2.97307300567627 0.798118710517883\\
2.98484778404236 0.793721735477448\\
2.99659848213196 0.787129402160645\\
3.00838160514832 0.778874039649963\\
3.02029156684875 0.769327104091644\\
3.03234505653381 0.758802771568298\\
3.04447245597839 0.747617602348328\\
3.05664157867432 0.736036121845245\\
3.06882643699646 0.724325954914093\\
3.08100342750549 0.712768971920013\\
3.09315085411072 0.701684951782227\\
3.10524773597717 0.691458225250244\\
3.11726975440979 0.682536721229553\\
3.12919759750366 0.675742566585541\\
3.14099335670471 0.672509431838989\\
3.15258574485779 0.673115789890289\\
3.16393399238586 0.677472770214081\\
3.1750283241272 0.685294687747955\\
3.18587756156921 0.696173489093781\\
3.19650077819824 0.709694087505341\\
3.20691919326782 0.725244760513306\\
3.21715044975281 0.742033123970032\\
3.22718858718872 0.758418679237366\\
3.2467474937439 0.784654200077057\\
3.26547813415527 0.801358640193939\\
3.28411507606506 0.811337649822235\\
3.29831433296204 0.815494418144226\\
3.3087158203125 0.816964566707611\\
3.32302951812744 0.816825926303864\\
3.33405566215515 0.815179109573364\\
3.3473973274231 0.811917245388031\\
3.36354088783264 0.807795643806458\\
3.38307428359985 0.80543839931488\\
3.3943293094635 0.804838478565216\\
3.40593409538269 0.803381621837616\\
3.41763758659363 0.800470113754272\\
3.42948269844055 0.796141386032104\\
3.44148325920105 0.790801346302032\\
3.45353007316589 0.784392714500427\\
3.4655749797821 0.777148246765137\\
3.48354387283325 0.765116512775421\\
3.4938976764679 0.757575452327728\\
3.50528645515442 0.748863220214844\\
3.51756191253662 0.739044904708862\\
3.52981090545654 0.728857576847076\\
3.54198026657104 0.718411505222321\\
3.55405426025391 0.707884550094604\\
3.56603574752808 0.697501063346863\\
3.57792735099792 0.687490999698639\\
3.58974385261536 0.678059160709381\\
3.60148811340332 0.669394671916962\\
3.61318278312683 0.661635458469391\\
3.62483644485474 0.656085550785065\\
3.63638854026794 0.653754055500031\\
3.64775395393372 0.654484272003174\\
3.65890121459961 0.657902538776398\\
3.66983270645142 0.663596034049988\\
3.68056321144104 0.671233534812927\\
3.69111466407776 0.680723905563354\\
3.70150089263916 0.691940426826477\\
3.71173763275146 0.704434335231781\\
3.72185039520264 0.717744290828705\\
3.73186469078064 0.731415927410126\\
3.75170636177063 0.758384227752686\\
3.7714056968689 0.78208589553833\\
3.7911114692688 0.798542380332947\\
3.8088846206665 0.807387351989746\\
3.82693791389465 0.81149297952652\\
3.84111881256104 0.812401294708252\\
3.85744023323059 0.812185168266296\\
3.87718939781189 0.813103318214417\\
3.88805747032166 0.814119160175323\\
3.89892983436584 0.814466416835785\\
3.90982556343079 0.813674628734589\\
3.92078495025635 0.811778545379639\\
3.9318573474884 0.808977901935577\\
3.94304871559143 0.805517554283142\\
3.95445036888123 0.801541984081268\\
3.96601796150208 0.797248244285584\\
3.97746133804321 0.792374908924103\\
3.98889112472534 0.786799848079681\\
4.00039577484131 0.780433714389801\\
4.01201629638672 0.773246467113495\\
4.02377510070801 0.765269994735718\\
4.03563594818115 0.756564557552338\\
4.04752445220947 0.747200310230255\\
4.05936098098755 0.737273216247559\\
4.0711030960083 0.726977527141571\\
4.08272695541382 0.716625571250916\\
4.09423494338989 0.706573486328125\\
4.10562992095947 0.697181880474091\\
4.11694383621216 0.688742518424988\\
4.12819576263428 0.681517958641052\\
4.13939952850342 0.677124381065369\\
4.1505651473999 0.676256716251373\\
4.16158246994019 0.678564786911011\\
4.17241191864014 0.683457136154175\\
4.18306303024292 0.690370500087738\\
4.1935658454895 0.698813796043396\\
4.2039475440979 0.708419322967529\\
4.21423053741455 0.718966424465179\\
4.22443199157715 0.730270206928253\\
4.2345666885376 0.742040574550629\\
4.24464702606201 0.754002392292023\\
4.25469064712524 0.765895545482636\\
4.26471757888794 0.776910960674286\\
4.27471828460693 0.78636509180069\\
4.2847638130188 0.794048726558685\\
4.29489898681641 0.799984812736511\\
4.30505514144897 0.804239630699158\\
4.31983995437622 0.807999908924103\\
4.33454179763794 0.808928072452545\\
4.34839296340942 0.807726681232452\\
4.36166000366211 0.805512011051178\\
4.37771320343018 0.802959859371185\\
4.39713716506958 0.802385091781616\\
4.40819406509399 0.802595138549805\\
4.41921901702881 0.801958739757538\\
4.43028879165649 0.800117373466492\\
4.44142246246338 0.797202825546265\\
4.45270013809204 0.793436765670776\\
4.46414184570313 0.789081275463104\\
4.47538042068481 0.784011363983154\\
4.48650979995728 0.778233647346497\\
4.49770545959473 0.771687626838684\\
4.50898551940918 0.764402031898499\\
4.52039957046509 0.756402134895325\\
4.5319881439209 0.747746407985687\\
4.54377889633179 0.73848021030426\\
4.55586624145508 0.728542983531952\\
4.56815242767334 0.718027353286743\\
4.58033037185669 0.707345902919769\\
4.59236669540405 0.696863949298859\\
4.60425090789795 0.687036275863647\\
4.6159782409668 0.678349554538727\\
4.62757444381714 0.671239972114563\\
4.63906097412109 0.668458938598633\\
4.65037202835083 0.669922590255737\\
4.66144800186157 0.675077080726624\\
4.67229557037354 0.683269858360291\\
4.68293762207031 0.693869650363922\\
4.69340085983276 0.706282436847687\\
4.70371246337891 0.719945728778839\\
4.71390056610107 0.734441220760345\\
4.72399282455444 0.749322056770325\\
4.73401403427124 0.7640261054039\\
4.75387620925903 0.788362562656403\\
4.77353286743164 0.804273962974548\\
4.79333353042603 0.813261687755585\\
4.80498886108398 0.816133856773376\\
4.81894111633301 0.817074656486511\\
4.83303546905518 0.815881550312042\\
4.84702253341675 0.813483476638794\\
4.86394691467285 0.810876548290253\\
4.8844256401062 0.810437440872192\\
4.89526796340942 0.810307621955872\\
4.9058313369751 0.809220314025879\\
4.91633558273315 0.807012498378754\\
4.92694330215454 0.803822040557861\\
4.93775272369385 0.799894034862518\\
4.94876766204834 0.795530676841736\\
4.96000909805298 0.790432095527649\\
4.97146701812744 0.784575462341309\\
4.98318338394165 0.777906775474548\\
5 0.767241597175598\\
};
\addlegendentry{Linear Mode};

\addplot [
color=black,
solid,
mark=o,
mark options={solid}
]
table[row sep=crcr]{
0 1\\
0.0118019115179777 1\\
0.0226854234933853 1\\
0.0340500771999359 1\\
0.0491764321923256 1\\
0.0619497485458851 1\\
0.0774054601788521 1\\
0.0961068719625473 1\\
0.106882445514202 1\\
0.118735581636429 1\\
0.13177402317524 1\\
0.146116316318512 1\\
0.161892831325531 1\\
0.179247006773949 1\\
0.198336601257324 1\\
0.21933513879776 1\\
0.242433547973633 1\\
0.267841786146164 1\\
0.29579085111618 1\\
0.32653483748436 1\\
0.360353201627731 1\\
0.397553414106369 1\\
0.438473641872406 1\\
0.483485877513886 1\\
0.532999336719513 1\\
0.587464153766632 1\\
0.647375464439392 1\\
0.71327793598175 1\\
0.7857705950737 1\\
0.865512549877167 1\\
0.953228712081909 1\\
1.04971647262573 0.995161294937134\\
1.1558530330658 0.979693293571472\\
1.27260315418243 0.957357287406921\\
1.32053518295288 0.947412252426147\\
1.35773289203644 0.93841552734375\\
1.38946354389191 0.929924786090851\\
1.42200791835785 0.920374691486359\\
1.45245742797852 0.910801649093628\\
1.48083031177521 0.901349127292633\\
1.50743794441223 0.892022550106049\\
1.53255617618561 0.882808029651642\\
1.55641222000122 0.873687744140625\\
1.57917487621307 0.864649832248688\\
1.60098683834076 0.855681836605072\\
1.62195324897766 0.846776962280273\\
1.64217150211334 0.837925493717194\\
1.66171324253082 0.829122722148895\\
1.68064773082733 0.820361614227295\\
1.69902181625366 0.811640918254852\\
1.71688628196716 0.802955090999603\\
1.73428571224213 0.794298350811005\\
1.75125479698181 0.785774171352386\\
1.76759457588196 0.777539014816284\\
1.78327512741089 0.769592463970184\\
1.79837548732758 0.761874079704285\\
1.81300187110901 0.754316449165344\\
1.82727634906769 0.746847033500671\\
1.84123587608337 0.739443123340607\\
1.85493552684784 0.732072591781616\\
1.86838006973267 0.724733471870422\\
1.88161540031433 0.71740061044693\\
1.89463996887207 0.710077345371246\\
1.90748333930969 0.702748000621796\\
1.92014336585999 0.695416688919067\\
1.93264579772949 0.688069760799408\\
1.94498836994171 0.683255314826965\\
1.95688509941101 0.681701004505157\\
1.96837878227234 0.682874798774719\\
1.97949528694153 0.685770511627197\\
1.99014973640442 0.689332842826843\\
2.00136232376099 0.693269908428192\\
2.01271271705627 0.697051227092743\\
2.02417707443237 0.700365483760834\\
2.0357928276062 0.703082680702209\\
2.04750537872314 0.705152451992035\\
2.05927753448486 0.706590950489044\\
2.07107305526733 0.707451343536377\\
2.08286380767822 0.707899689674377\\
2.09455490112305 0.708136200904846\\
2.10611391067505 0.708497524261475\\
2.11754536628723 0.709739506244659\\
2.12883257865906 0.713333308696747\\
2.13992094993591 0.720506727695465\\
2.15075325965881 0.729838073253632\\
2.16134643554688 0.738739728927612\\
2.1717312335968 0.746206641197205\\
2.18185663223267 0.752267837524414\\
2.20230388641357 0.762538492679596\\
2.21264386177063 0.767454922199249\\
2.22270774841309 0.772260308265686\\
2.23349833488464 0.777447164058685\\
2.24430298805237 0.782548725605011\\
2.25509786605835 0.787428200244904\\
2.26583743095398 0.791951775550842\\
2.27648019790649 0.7960165143013\\
2.29678177833557 0.802417457103729\\
2.30735301971436 0.804985463619232\\
2.32744646072388 0.808435916900635\\
2.34562826156616 0.810013592243195\\
2.36075878143311 0.810389518737793\\
2.3733606338501 0.810078859329224\\
2.38860893249512 0.809010863304138\\
2.40705943107605 0.806161463260651\\
2.41769051551819 0.803072869777679\\
2.42919039726257 0.798441052436829\\
2.4407947063446 0.792471528053284\\
2.45252275466919 0.78523051738739\\
2.46437191963196 0.776840925216675\\
2.47634053230286 0.767454028129578\\
2.48842525482178 0.757237017154694\\
2.50060844421387 0.746374726295471\\
2.51288247108459 0.73504239320755\\
2.52523136138916 0.723420560359955\\
2.53763508796692 0.711685955524445\\
2.55007195472717 0.699997961521149\\
2.56252241134644 0.688497424125671\\
2.57496857643127 0.677303493022919\\
2.58739495277405 0.666511058807373\\
2.59979104995728 0.656353831291199\\
2.61202311515808 0.647194385528564\\
2.62403297424316 0.639329731464386\\
2.63583588600159 0.636150002479553\\
2.64725995063782 0.638070404529572\\
2.65829944610596 0.643946290016174\\
2.66903948783875 0.652319014072418\\
2.68084859848022 0.662393510341644\\
2.69513750076294 0.675918996334076\\
2.71242713928223 0.693264961242676\\
2.7327516078949 0.713853061199188\\
2.74304962158203 0.723939299583435\\
2.75329804420471 0.733607172966003\\
2.76352214813232 0.742822408676147\\
2.77671360969543 0.753786623477936\\
2.79198384284973 0.765558302402496\\
2.81012392044067 0.778243958950043\\
2.82984495162964 0.7897989153862\\
2.84034872055054 0.795110523700714\\
2.85096287727356 0.800727307796478\\
2.86159515380859 0.807240605354309\\
2.87224507331848 0.813998699188232\\
2.88282418251038 0.819493353366852\\
2.89327502250671 0.82299679517746\\
2.90423226356506 0.824577033519745\\
2.91532301902771 0.824596226215363\\
2.92652130126953 0.823707103729248\\
2.93784737586975 0.822374880313873\\
2.94925856590271 0.820825397968292\\
2.96071529388428 0.819130122661591\\
2.97219562530518 0.817319929599762\\
2.98368406295776 0.815415620803833\\
2.99517321586609 0.813416659832001\\
3.00665736198425 0.811287820339203\\
3.018141746521 0.808829605579376\\
3.02963495254517 0.805378913879395\\
3.04114675521851 0.80073493719101\\
3.05272746086121 0.794805824756622\\
3.06440591812134 0.787599384784698\\
3.07619190216064 0.779191017150879\\
3.08806824684143 0.769720613956451\\
3.10001063346863 0.759366571903229\\
3.11200356483459 0.748313188552856\\
3.12402725219727 0.736752152442932\\
3.13606834411621 0.724875271320343\\
3.14812350273132 0.712887644767761\\
3.16019630432129 0.701007485389709\\
3.17221331596375 0.689547657966614\\
3.18425130844116 0.678698837757111\\
3.1963107585907 0.668757975101471\\
3.20831441879272 0.660067141056061\\
3.22024464607239 0.653569519519806\\
3.2320544719696 0.65085232257843\\
3.24364876747131 0.651964008808136\\
3.25497913360596 0.656622290611267\\
3.26603627204895 0.664417147636414\\
3.27683329582214 0.675070106983185\\
3.28738880157471 0.688274800777435\\
3.30090045928955 0.707143545150757\\
3.31724953651428 0.733064949512482\\
3.3364520072937 0.763977468013763\\
3.35590267181396 0.788724601268768\\
3.37547159194946 0.80456668138504\\
3.39097309112549 0.812018096446991\\
3.40369939804077 0.815524220466614\\
3.41909837722778 0.816869139671326\\
3.43773102760315 0.815859615802765\\
3.44846701622009 0.815338432788849\\
3.45963358879089 0.815738797187805\\
3.47084331512451 0.816300868988037\\
3.4820818901062 0.815788686275482\\
3.49336910247803 0.813811302185059\\
3.50474190711975 0.810532748699188\\
3.51621460914612 0.806376278400421\\
3.52777361869812 0.801736891269684\\
3.53940916061401 0.796741425991058\\
3.55113887786865 0.791134297847748\\
3.5629894733429 0.784699440002441\\
3.57491517066956 0.777324438095093\\
3.58686423301697 0.769021987915039\\
3.59883189201355 0.759876608848572\\
3.61085295677185 0.750011324882507\\
3.62294244766235 0.739583373069763\\
3.63507795333862 0.728764295578003\\
3.64724469184875 0.717696487903595\\
3.65939402580261 0.70654833316803\\
3.67149782180786 0.695481717586517\\
3.68352150917053 0.684714615345001\\
3.69545078277588 0.674504101276398\\
3.70728397369385 0.665115237236023\\
3.71901512145996 0.656813383102417\\
3.73064255714417 0.650577366352081\\
3.74213266372681 0.648110270500183\\
3.75340151786804 0.649445056915283\\
3.7644145488739 0.654268085956573\\
3.77517652511597 0.662119626998901\\
3.78571152687073 0.672529995441437\\
3.79605078697205 0.685197651386261\\
3.80622386932373 0.699702262878418\\
3.82096457481384 0.721896111965179\\
3.83880090713501 0.750025570392609\\
3.85830807685852 0.778035581111908\\
3.87783885002136 0.797684550285339\\
3.89773917198181 0.809440314769745\\
3.90787315368652 0.812820613384247\\
3.91816449165344 0.814779281616211\\
3.93745827674866 0.815654277801514\\
3.94821214675903 0.815431535243988\\
3.95906329154968 0.815935730934143\\
3.96999645233154 0.817408263683319\\
3.98096656799316 0.818963408470154\\
3.991947889328 0.819588601589203\\
4.00295209884644 0.818919479846954\\
4.01402473449707 0.817019820213318\\
4.02517938613892 0.81412672996521\\
4.03638935089111 0.810531377792358\\
4.04771900177002 0.806409060955048\\
4.05928993225098 0.801859438419342\\
4.07111120223999 0.796508491039276\\
4.08298397064209 0.790336966514587\\
4.09486865997314 0.783325910568237\\
4.10676383972168 0.775501728057861\\
4.11869525909424 0.766916155815125\\
4.13065099716187 0.757630288600922\\
4.14261865615845 0.747690081596375\\
4.15452194213867 0.737242519855499\\
4.16634511947632 0.726497709751129\\
4.17807626724243 0.715754806995392\\
4.18974590301514 0.705283641815186\\
4.20136594772339 0.695355176925659\\
4.21294355392456 0.686231434345245\\
4.22449111938477 0.678154289722443\\
4.23601484298706 0.671893894672394\\
4.24750518798828 0.668983280658722\\
4.25887632369995 0.669286847114563\\
4.27007293701172 0.672407865524292\\
4.28108787536621 0.677898943424225\\
4.29192972183228 0.685348749160767\\
4.30261182785034 0.694470524787903\\
4.31315040588379 0.70512866973877\\
4.32356023788452 0.716991424560547\\
4.33452129364014 0.729404628276825\\
4.34778356552124 0.745184481143951\\
4.36383104324341 0.764494061470032\\
4.38311052322388 0.784277379512787\\
4.3932032585144 0.79193127155304\\
4.4033842086792 0.797783195972443\\
4.41344928741455 0.801898837089539\\
4.42365694046021 0.804547786712646\\
4.43415594100952 0.805887162685394\\
4.44478559494019 0.806167900562286\\
4.4555401802063 0.805964827537537\\
4.46635627746582 0.806215047836304\\
4.47721815109253 0.807421147823334\\
4.48809051513672 0.808788061141968\\
4.4989447593689 0.80925053358078\\
4.50979852676392 0.808584213256836\\
4.52070713043213 0.806863725185394\\
4.53168821334839 0.80427610874176\\
4.54263353347778 0.801081836223602\\
4.55374670028687 0.797418534755707\\
4.56528568267822 0.793280065059662\\
4.57695484161377 0.78841096162796\\
4.58858823776245 0.782783329486847\\
4.6002082824707 0.776357889175415\\
4.61189317703247 0.769100785255432\\
4.62363004684448 0.761059939861298\\
4.63552665710449 0.75224769115448\\
4.64763736724854 0.742677927017212\\
4.65980005264282 0.732520461082459\\
4.67189693450928 0.721945226192474\\
4.68388748168945 0.711209774017334\\
4.69575357437134 0.700695872306824\\
4.70748949050903 0.690852165222168\\
4.71910238265991 0.682121574878693\\
4.73061180114746 0.674898386001587\\
4.74203300476074 0.671482026576996\\
4.75329446792603 0.672371327877045\\
4.7643346786499 0.676974236965179\\
4.77515077590942 0.684498250484467\\
4.78576755523682 0.694226086139679\\
4.79621171951294 0.705526888370514\\
4.8065128326416 0.717834830284119\\
4.81669855117798 0.730810105800629\\
4.82679224014282 0.744160354137421\\
4.8414249420166 0.76282000541687\\
4.859130859375 0.782685697078705\\
4.87904977798462 0.797951638698578\\
4.88912153244019 0.802902400493622\\
4.89928483963013 0.806248068809509\\
4.90958023071289 0.808123350143433\\
4.92001533508301 0.808634281158447\\
4.93059682846069 0.807890236377716\\
4.94941663742065 0.804432272911072\\
4.9602427482605 0.801936388015747\\
4.97117471694946 0.800463736057281\\
4.98216915130615 0.800458490848541\\
5 0.801366329193115\\
};
\addlegendentry{Nonlinear Mode};

\end{axis}
\end{tikzpicture}%
\caption{Flashing solution at \dtmax{} = \expneg{1.0}{1} {[s]}.}
\label{fig:flashing1pt0000em0}
\end{figure}

\begin{figure}[h!tb]
\centering
% This file was created by matlab2tikz v0.4.3.
% Copyright (c) 2008--2013, Nico Schlömer <nico.schloemer@gmail.com>
% All rights reserved.
% 
\tikzsetnextfilename{plots/flashing1pt0000em5_eps}
\begin{tikzpicture}

\begin{axis}[%
width=\mytikzpicwidth,
height=\mytikzpicheight,
scale only axis,
xmin=0.0,
xmax=5.0,
xlabel={Time $[\text{s}]$},
ymin=0.5,
ymax=1.0,
ylabel={$\alpha_g$ [-] @ 2 [in] from Inlet},
legend style={draw=black,fill=white,legend cell align=left}
]
\addplot [
color=black,
solid
]
table[row sep=crcr]{
0.0 1.0\\
0.0100024435669184 1\\
0.020002443343401 1\\
0.0300024431198835 1\\
0.0400024428963661 1\\
0.0500024445354939 1\\
0.0600024424493313 1\\
0.070002444088459 1\\
0.0800024420022964 1\\
0.0900024399161339 1\\
0.100002445280552 1\\
0.110002443194389 1\\
0.120002441108227 1\\
0.130002439022064 1\\
0.140002444386482 1\\
0.1500024497509 1\\
0.160002440214157 1\\
0.170002445578575 1\\
0.180002436041832 1\\
0.19000244140625 1\\
0.200002446770668 1\\
0.210002437233925 1\\
0.220002442598343 1\\
0.230002447962761 1\\
0.240002438426018 1\\
0.250002443790436 1\\
0.260002434253693 1\\
0.270002454519272 1\\
0.280002444982529 1\\
0.290002435445786 1\\
0.300002455711365 1\\
0.310002446174622 1\\
0.320002436637878 1\\
0.330002456903458 1\\
0.340002447366714 1\\
0.350002437829971 1\\
0.360002458095551 1\\
0.370002448558807 1\\
0.380002439022064 1\\
0.390002429485321 1\\
0.4000024497509 1\\
0.410002440214157 1\\
0.420002430677414 1\\
0.430002450942993 1\\
0.44000244140625 1\\
0.450002431869507 1\\
0.460002452135086 1\\
0.470002442598343 1\\
0.4800024330616 1\\
0.490002453327179 1\\
0.500002443790436 1\\
0.510002434253693 1\\
0.520002424716949 1\\
0.530002415180206 1\\
0.540002465248108 1\\
0.550002455711365 1\\
0.560002446174622 1\\
0.570002436637878 1\\
0.580002427101135 1\\
0.590002417564392 1\\
0.600002467632294 1\\
0.610002458095551 1\\
0.620002448558807 1\\
0.630002439022064 1\\
0.640002429485321 1\\
0.650002419948578 1\\
0.660002470016479 1\\
0.670002460479736 1\\
0.680002450942993 1\\
0.69000244140625 1\\
0.700002431869507 1\\
0.710002422332764 1\\
0.720002472400665 1\\
0.730002462863922 1\\
0.740002453327179 1\\
0.750002443790436 1\\
0.760002434253693 1\\
0.770002424716949 1\\
0.780002415180206 1\\
0.790002465248108 1\\
0.800002455711365 1\\
0.810002446174622 1\\
0.820002436637878 1\\
0.830002427101135 1\\
0.840002417564392 1\\
0.850002467632294 1\\
0.860002458095551 1\\
0.870002448558807 1\\
0.880002439022064 1\\
0.890002429485321 1\\
0.900002419948578 1\\
0.910002470016479 1\\
0.920002460479736 1\\
0.930002450942993 1\\
0.94000244140625 1\\
0.950002431869507 1\\
0.960002422332764 1\\
0.970002472400665 1\\
0.980002462863922 1\\
0.990002453327179 1\\
1.00000238418579 1\\
1.01000249385834 0.999940812587738\\
1.02000248432159 0.99977570772171\\
1.03000247478485 0.999614715576172\\
1.04000246524811 0.999387145042419\\
1.05000245571136 0.999091565608978\\
1.06000244617462 0.998726606369019\\
1.07000243663788 0.99829089641571\\
1.08000242710114 0.997783184051514\\
1.09000241756439 0.997202277183533\\
1.10000240802765 0.996546924114227\\
1.11000239849091 0.995815932750702\\
1.12000238895416 0.995008230209351\\
1.13000249862671 0.994122743606567\\
1.14000248908997 0.993158280849457\\
1.15000247955322 0.992113947868347\\
1.16000247001648 0.990988671779633\\
1.17000246047974 0.989781439304352\\
1.18000245094299 0.988491296768188\\
1.19000244140625 0.987117409706116\\
1.20000243186951 0.985658764839172\\
1.21000242233276 0.984114527702332\\
1.22000241279602 0.982483804225922\\
1.23000240325928 0.980765759944916\\
1.24000239372253 0.978960752487183\\
1.25000238418579 0.97708523273468\\
1.26000249385834 0.975153565406799\\
1.27000248432159 0.973167359828949\\
1.28000247478485 0.971124470233917\\
1.29000246524811 0.969022333621979\\
1.30000245571136 0.966858863830566\\
1.31000244617462 0.96463268995285\\
1.32000243663788 0.962342798709869\\
1.33000242710114 0.959988653659821\\
1.34000241756439 0.95756995677948\\
1.35000240802765 0.955086469650269\\
1.36000239849091 0.952538251876831\\
1.37000238895416 0.949925243854523\\
1.38000249862671 0.947247445583344\\
1.39000248908997 0.944505035877228\\
1.40000247955322 0.941698014736176\\
1.41000247001648 0.938826501369476\\
1.42000246047974 0.935890555381775\\
1.43000245094299 0.932890295982361\\
1.44000244140625 0.929825723171234\\
1.45000243186951 0.92669689655304\\
1.46000242233276 0.923503935337067\\
1.47000241279602 0.920246839523315\\
1.48000240325928 0.916925668716431\\
1.49000239372253 0.913540422916412\\
1.50000238418579 0.910091161727905\\
1.51000249385834 0.906577944755554\\
1.52000248432159 0.90300065279007\\
1.53000247478485 0.899359464645386\\
1.54000246524811 0.895654261112213\\
1.55000245571136 0.891885101795197\\
1.56000244617462 0.888051927089691\\
1.57000243663788 0.884154796600342\\
1.58000242710114 0.880193650722504\\
1.59000241756439 0.876168489456177\\
1.60000240802765 0.872079253196716\\
1.61000239849091 0.867926001548767\\
1.62000238895416 0.86370861530304\\
1.63000249862671 0.859427154064178\\
1.64000248908997 0.855081498622894\\
1.65000247955322 0.850671648979187\\
1.66000247001648 0.846197605133057\\
1.67000246047974 0.841659307479858\\
1.68000245094299 0.837056696414948\\
1.69000244140625 0.832389712333679\\
1.70000243186951 0.827658355236053\\
1.71000242233276 0.822862565517426\\
1.72000241279602 0.818002343177795\\
1.73000240325928 0.813077509403229\\
1.74000239372253 0.80808812379837\\
1.75000238418579 0.803034067153931\\
1.76000249385834 0.797916233539581\\
1.77000248432159 0.792760729789734\\
1.78000247478485 0.787597835063934\\
1.79000246524811 0.782431662082672\\
1.80000245571136 0.777256071567535\\
1.81000244617462 0.772062599658966\\
1.82000243663788 0.766842603683472\\
1.83000242710114 0.761588752269745\\
1.84000241756439 0.756294667720795\\
1.85000240802765 0.750954985618591\\
1.86000239849091 0.745565354824066\\
1.87000238895416 0.740122199058533\\
1.88000249862671 0.734622418880463\\
1.89000248908997 0.729063630104065\\
1.90000247955322 0.723443686962128\\
1.91000247001648 0.717760682106018\\
1.92000246047974 0.712013244628906\\
1.93000245094299 0.706199824810028\\
1.94000244140625 0.700319290161133\\
1.95000243186951 0.69437050819397\\
1.96000242233276 0.688352346420288\\
1.97000241279602 0.682263791561127\\
1.98000240325928 0.676115453243256\\
1.99000239372253 0.672087252140045\\
2.00000238418579 0.67232882976532\\
2.01000237464905 0.676584005355835\\
2.0200023651123 0.683436095714569\\
2.03000235557556 0.691209495067596\\
2.04000234603882 0.698723435401917\\
2.05000233650208 0.705377638339996\\
2.06000232696533 0.710965991020203\\
2.07000255584717 0.715481877326965\\
2.08000254631042 0.719024658203125\\
2.09000253677368 0.721766769886017\\
2.10000252723694 0.723845064640045\\
2.1100025177002 0.725369334220886\\
2.12000250816345 0.726444840431213\\
2.13000249862671 0.727207958698273\\
2.14000248908997 0.727903068065643\\
2.15000247955322 0.729034006595612\\
2.16000247001648 0.731527984142303\\
2.17000246047974 0.73650336265564\\
2.18000245094299 0.744256794452667\\
2.19000244140625 0.753634631633759\\
2.20000243186951 0.762977123260498\\
2.21000242233276 0.771209299564362\\
2.22000241279602 0.778065741062164\\
2.23000240325928 0.783720254898071\\
2.24000239372253 0.788461208343506\\
2.25000238418579 0.792538404464722\\
2.26000237464905 0.796130836009979\\
2.2700023651123 0.799349129199982\\
2.28000235557556 0.802254021167755\\
2.29000234603882 0.804873406887054\\
2.30000233650208 0.807216942310333\\
2.31000232696533 0.809285640716553\\
2.32000255584717 0.811077654361725\\
2.33000254631042 0.812591433525085\\
2.34000253677368 0.813827633857727\\
2.35000252723694 0.814789414405823\\
2.3600025177002 0.815482795238495\\
2.37000250816345 0.815915882587433\\
2.38000249862671 0.816098809242249\\
2.39000248908997 0.816043078899384\\
2.40000247955322 0.815761923789978\\
2.41000247001648 0.815302014350891\\
2.42000246047974 0.814667403697968\\
2.43000245094299 0.813839137554169\\
2.44000244140625 0.812807083129883\\
2.45000243186951 0.811569392681122\\
2.46000242233276 0.810128271579742\\
2.47000241279602 0.808315396308899\\
2.48000240325928 0.805980622768402\\
2.49000239372253 0.803062856197357\\
2.50000238418579 0.799528241157532\\
2.51000237464905 0.795372486114502\\
2.5200023651123 0.790614366531372\\
2.53000235557556 0.785287499427795\\
2.54000234603882 0.779434502124786\\
2.55000233650208 0.773102402687073\\
2.56000232696533 0.766340374946594\\
2.57000255584717 0.75919783115387\\
2.58000254631042 0.75172346830368\\
2.59000253677368 0.743964850902557\\
2.60000252723694 0.735968053340912\\
2.6100025177002 0.727777719497681\\
2.62000250816345 0.719438433647156\\
2.63000249862671 0.710996389389038\\
2.64000248908997 0.702500998973846\\
2.65000247955322 0.694006979465485\\
2.66000247001648 0.685588479042053\\
2.67000246047974 0.677358090877533\\
2.68000245094299 0.669408679008484\\
2.69000244140625 0.661824107170105\\
2.70000243186951 0.654695153236389\\
2.71000242233276 0.648134887218475\\
2.72000241279602 0.643493950366974\\
2.73000240325928 0.642441034317017\\
2.74000239372253 0.64506870508194\\
2.75000238418579 0.650754928588867\\
2.76000237464905 0.658449411392212\\
2.7700023651123 0.667110979557037\\
2.78000235557556 0.676054418087006\\
2.79000234603882 0.685020327568054\\
2.80000233650208 0.693921685218811\\
2.81000232696533 0.702739238739014\\
2.82000255584717 0.71136599779129\\
2.83000254631042 0.719673097133636\\
2.84000253677368 0.727609395980835\\
2.85000252723694 0.735151290893555\\
2.8600025177002 0.742286622524261\\
2.87000250816345 0.749027669429779\\
2.88000249862671 0.755425810813904\\
2.89000248908997 0.761612594127655\\
2.90000247955322 0.767895638942719\\
2.91000247001648 0.774844527244568\\
2.92000246047974 0.78313934803009\\
2.93000245094299 0.792882740497589\\
2.94000244140625 0.803150773048401\\
2.95000243186951 0.812635600566864\\
2.96000242233276 0.82045191526413\\
2.97000241279602 0.8263338804245\\
2.98000240325928 0.830405116081238\\
2.99000239372253 0.832934319972992\\
3.00000238418579 0.834196984767914\\
3.01000237464905 0.834425449371338\\
3.0200023651123 0.833804368972778\\
3.03000235557556 0.832489609718323\\
3.04000234603882 0.830633878707886\\
3.05000233650208 0.828379988670349\\
3.06000232696533 0.825823307037354\\
3.07000255584717 0.823015332221985\\
3.08000254631042 0.819951713085175\\
3.09000253677368 0.816335499286652\\
3.10000252723694 0.812011420726776\\
3.1100025177002 0.806896984577179\\
3.12000250816345 0.800952196121216\\
3.13000249862671 0.794185936450958\\
3.14000248908997 0.786645114421844\\
3.15000247955322 0.778406262397766\\
3.16000247001648 0.769563794136047\\
3.17000246047974 0.760220050811768\\
3.18000245094299 0.750483810901642\\
3.19000244140625 0.740464270114899\\
3.20000243186951 0.730268776416779\\
3.21000242233276 0.720010936260223\\
3.22000241279602 0.709814250469208\\
3.23000240325928 0.699811816215515\\
3.24000239372253 0.690146446228027\\
3.25000238418579 0.680972337722778\\
3.26000237464905 0.672465205192566\\
3.2700023651123 0.664815247058868\\
3.28000235557556 0.658277809619904\\
3.29000234603882 0.654153883457184\\
3.30000233650208 0.653075635433197\\
3.31000232696533 0.655135214328766\\
3.32000255584717 0.660297453403473\\
3.33000254631042 0.668414294719696\\
3.34000253677368 0.679233729839325\\
3.35000252723694 0.692402482032776\\
3.3600025177002 0.707357227802277\\
3.37000250816345 0.723401069641113\\
3.38000249862671 0.739875912666321\\
3.39000248908997 0.756195962429047\\
3.40000247955322 0.771478772163391\\
3.41000247001648 0.784752488136292\\
3.42000246047974 0.795693695545197\\
3.43000245094299 0.804405152797699\\
3.44000244140625 0.811127781867981\\
3.45000243186951 0.816102921962738\\
3.46000242233276 0.819564938545227\\
3.47000241279602 0.821808576583862\\
3.48000240325928 0.82325553894043\\
3.49000239372253 0.824495315551758\\
3.50000238418579 0.82609885931015\\
3.51000237464905 0.828206777572632\\
3.5200023651123 0.830396354198456\\
3.53000235557556 0.832020282745361\\
3.54000234603882 0.832619369029999\\
3.55000233650208 0.832059502601624\\
3.56000232696533 0.830418109893799\\
3.57000255584717 0.827859818935394\\
3.58000254631042 0.824575543403625\\
3.59000253677368 0.820736944675446\\
3.60000252723694 0.816327691078186\\
3.6100025177002 0.811276376247406\\
3.62000250816345 0.805596888065338\\
3.63000249862671 0.799300372600555\\
3.64000248908997 0.792412281036377\\
3.65000247955322 0.784977316856384\\
3.66000247001648 0.777066648006439\\
3.67000246047974 0.76876026391983\\
3.68000245094299 0.760121822357178\\
3.69000244140625 0.751200020313263\\
3.70000243186951 0.74203360080719\\
3.71000242233276 0.732673406600952\\
3.72000241279602 0.723198413848877\\
3.73000240325928 0.713714063167572\\
3.74000239372253 0.704339861869812\\
3.75000238418579 0.695197284221649\\
3.76000237464905 0.686403214931488\\
3.7700023651123 0.678071260452271\\
3.78000235557556 0.670320153236389\\
3.79000234603882 0.66326379776001\\
3.80000233650208 0.657258152961731\\
3.81000232696533 0.65373033285141\\
3.82000255584717 0.652927458286285\\
3.83000254631042 0.654780507087708\\
3.84000253677368 0.659134268760681\\
3.85000252723694 0.665776073932648\\
3.8600025177002 0.674457848072052\\
3.87000250816345 0.684912621974945\\
3.88000249862671 0.696814358234406\\
3.89000248908997 0.709755480289459\\
3.90000247955322 0.723380088806152\\
3.91000247001648 0.737419962882996\\
3.92000246047974 0.751447975635529\\
3.93000245094299 0.764988660812378\\
3.94000244140625 0.777369737625122\\
3.95000243186951 0.78799432516098\\
3.96000242233276 0.796693861484528\\
3.97000241279602 0.803554534912109\\
3.98000240325928 0.808800458908081\\
3.99000239372253 0.81277722120285\\
4.00000238418579 0.816004395484924\\
4.01000261306763 0.819147408008575\\
4.0200023651123 0.822688817977905\\
4.03000259399414 0.826514184474945\\
4.04000234603882 0.830002725124359\\
4.05000257492065 0.83250880241394\\
4.06000232696533 0.833692789077759\\
4.07000255584717 0.833533406257629\\
4.08000230789185 0.832201063632965\\
4.09000253677368 0.829906404018402\\
4.10000228881836 0.826833486557007\\
4.1100025177002 0.823146998882294\\
4.12000226974487 0.818972527980804\\
4.13000249862671 0.814260363578796\\
4.14000225067139 0.809014856815338\\
4.15000247955322 0.803245842456818\\
4.1600022315979 0.796961486339569\\
4.17000246047974 0.790179312229156\\
4.18000221252441 0.782933235168457\\
4.19000244140625 0.775281429290771\\
4.20000267028809 0.767274618148804\\
4.21000242233276 0.758937776088715\\
4.2200026512146 0.750287234783173\\
4.23000240325928 0.741360604763031\\
4.24000263214111 0.732235491275787\\
4.25000238418579 0.723028361797333\\
4.26000261306763 0.713881134986877\\
4.2700023651123 0.70494669675827\\
4.28000259399414 0.696373283863068\\
4.29000234603882 0.688297808170319\\
4.30000257492065 0.68085104227066\\
4.31000232696533 0.674146115779877\\
4.32000255584717 0.668870568275452\\
4.33000230789185 0.666284382343292\\
4.34000253677368 0.666319847106934\\
4.35000228881836 0.668751239776611\\
4.3600025177002 0.673304796218872\\
4.37000226974487 0.679687261581421\\
4.38000249862671 0.687608063220978\\
4.39000225067139 0.696780920028687\\
4.40000247955322 0.706885099411011\\
4.4100022315979 0.717602491378784\\
4.42000246047974 0.72868424654007\\
4.43000221252441 0.739975452423096\\
4.44000244140625 0.751364469528198\\
4.45000267028809 0.762536823749542\\
4.46000242233276 0.773160457611084\\
4.4700026512146 0.782671272754669\\
4.48000240325928 0.790699064731598\\
4.49000263214111 0.797164916992188\\
4.50000238418579 0.802182257175446\\
4.51000261306763 0.806034445762634\\
4.5200023651123 0.809218764305115\\
4.53000259399414 0.812403678894043\\
4.54000234603882 0.816087603569031\\
4.55000257492065 0.820129036903381\\
4.56000232696533 0.823852241039276\\
4.57000255584717 0.826563596725464\\
4.58000230789185 0.827903032302856\\
4.59000253677368 0.827838659286499\\
4.60000228881836 0.826531589031219\\
4.6100025177002 0.824223399162292\\
4.62000226974487 0.821124196052551\\
4.63000249862671 0.81739330291748\\
4.64000225067139 0.813075184822083\\
4.65000247955322 0.808099567890167\\
4.6600022315979 0.802511811256409\\
4.67000246047974 0.796341896057129\\
4.68000221252441 0.78961855173111\\
4.69000244140625 0.782401621341705\\
4.70000267028809 0.77474582195282\\
4.71000242233276 0.76672637462616\\
4.7200026512146 0.758427917957306\\
4.73000240325928 0.749910414218903\\
4.74000263214111 0.741192102432251\\
4.75000238418579 0.732276856899261\\
4.76000261306763 0.723193943500519\\
4.7700023651123 0.714022874832153\\
4.78000259399414 0.704891204833984\\
4.79000234603882 0.695958316326141\\
4.80000257492065 0.68740302324295\\
4.81000232696533 0.679418742656708\\
4.82000255584717 0.672211349010468\\
4.83000230789185 0.666107714176178\\
4.84000253677368 0.662941515445709\\
4.85000228881836 0.663268625736237\\
4.8600025177002 0.666871905326843\\
4.87000226974487 0.673372209072113\\
4.88000249862671 0.682315528392792\\
4.89000225067139 0.693206906318665\\
4.90000247955322 0.70544570684433\\
4.9100022315979 0.718441963195801\\
4.92000246047974 0.731704592704773\\
4.93000221252441 0.744857788085938\\
4.94000244140625 0.757653117179871\\
4.95000267028809 0.769903540611267\\
4.96000242233276 0.781043231487274\\
4.9700026512146 0.790530323982239\\
4.98000240325928 0.798227965831757\\
4.99000263214111 0.804194629192352\\
5 0.808577299118042\\
};
\addlegendentry{Linear Mode};

\addplot [
color=black,
solid,
mark=o,
mark options={solid}
]
table[row sep=crcr]{
0 1\\
0.0100024435669184 1\\
0.020002443343401 1\\
0.0300024431198835 1\\
0.0400024428963661 1\\
0.0500024445354939 1\\
0.0600024424493313 1\\
0.070002444088459 1\\
0.0800024420022964 1\\
0.0900024399161339 1\\
0.100002445280552 1\\
0.110002443194389 1\\
0.120002441108227 1\\
0.130002439022064 1\\
0.140002444386482 1\\
0.1500024497509 1\\
0.160002440214157 1\\
0.170002445578575 1\\
0.180002436041832 1\\
0.19000244140625 1\\
0.200002446770668 1\\
0.210002437233925 1\\
0.220002442598343 1\\
0.230002447962761 1\\
0.240002438426018 1\\
0.250002443790436 1\\
0.260002434253693 1\\
0.270002454519272 1\\
0.280002444982529 1\\
0.290002435445786 1\\
0.300002455711365 1\\
0.310002446174622 1\\
0.320002436637878 1\\
0.330002456903458 1\\
0.340002447366714 1\\
0.350002437829971 1\\
0.360002458095551 1\\
0.370002448558807 1\\
0.380002439022064 1\\
0.390002429485321 1\\
0.4000024497509 1\\
0.410002440214157 1\\
0.420002430677414 1\\
0.430002450942993 1\\
0.44000244140625 1\\
0.450002431869507 1\\
0.460002452135086 1\\
0.470002442598343 1\\
0.4800024330616 1\\
0.490002453327179 1\\
0.500002443790436 1\\
0.510002434253693 1\\
0.520002424716949 1\\
0.530002415180206 1\\
0.540002465248108 1\\
0.550002455711365 1\\
0.560002446174622 1\\
0.570002436637878 1\\
0.580002427101135 1\\
0.590002417564392 1\\
0.600002467632294 1\\
0.610002458095551 1\\
0.620002448558807 1\\
0.630002439022064 1\\
0.640002429485321 1\\
0.650002419948578 1\\
0.660002470016479 1\\
0.670002460479736 1\\
0.680002450942993 1\\
0.69000244140625 1\\
0.700002431869507 1\\
0.710002422332764 1\\
0.720002472400665 1\\
0.730002462863922 1\\
0.740002453327179 1\\
0.750002443790436 1\\
0.760002434253693 1\\
0.770002424716949 1\\
0.780002415180206 1\\
0.790002465248108 1\\
0.800002455711365 1\\
0.810002446174622 1\\
0.820002436637878 1\\
0.830002427101135 1\\
0.840002417564392 1\\
0.850002467632294 1\\
0.860002458095551 1\\
0.870002448558807 1\\
0.880002439022064 1\\
0.890002429485321 1\\
0.900002419948578 1\\
0.910002470016479 1\\
0.920002460479736 1\\
0.930002450942993 1\\
0.94000244140625 1\\
0.950002431869507 1\\
0.960002422332764 1\\
0.970002472400665 1\\
0.980002462863922 1\\
0.990002453327179 1\\
1.00000238418579 1\\
1.01000249385834 0.999940812587738\\
1.02000248432159 0.99977570772171\\
1.03000247478485 0.999614715576172\\
1.04000246524811 0.999387145042419\\
1.05000245571136 0.999091565608978\\
1.06000244617462 0.998726546764374\\
1.07000243663788 0.99829089641571\\
1.08000242710114 0.997783184051514\\
1.09000241756439 0.997202277183533\\
1.10000240802765 0.996546864509583\\
1.11000239849091 0.995815932750702\\
1.12000238895416 0.995008230209351\\
1.13000249862671 0.994122684001923\\
1.14000248908997 0.993158280849457\\
1.15000247955322 0.992113947868347\\
1.16000247001648 0.990988612174988\\
1.17000246047974 0.989781439304352\\
1.18000245094299 0.988491296768188\\
1.19000244140625 0.987117409706116\\
1.20000243186951 0.985658764839172\\
1.21000242233276 0.984114527702332\\
1.22000241279602 0.982483804225922\\
1.23000240325928 0.980765759944916\\
1.24000239372253 0.978960752487183\\
1.25000238418579 0.97708523273468\\
1.26000249385834 0.975153505802155\\
1.27000248432159 0.973167300224304\\
1.28000247478485 0.971124410629272\\
1.29000246524811 0.969022274017334\\
1.30000245571136 0.966858863830566\\
1.31000244617462 0.964632630348206\\
1.32000243663788 0.962342798709869\\
1.33000242710114 0.959988653659821\\
1.34000241756439 0.957569897174835\\
1.35000240802765 0.955086469650269\\
1.36000239849091 0.952538251876831\\
1.37000238895416 0.949925184249878\\
1.38000249862671 0.947247445583344\\
1.39000248908997 0.944505035877228\\
1.40000247955322 0.941698014736176\\
1.41000247001648 0.938826501369476\\
1.42000246047974 0.935890555381775\\
1.43000245094299 0.932890236377716\\
1.44000244140625 0.929825663566589\\
1.45000243186951 0.926696836948395\\
1.46000242233276 0.923503875732422\\
1.47000241279602 0.920246779918671\\
1.48000240325928 0.916925609111786\\
1.49000239372253 0.913540363311768\\
1.50000238418579 0.910091102123261\\
1.51000249385834 0.906577885150909\\
1.52000248432159 0.903000593185425\\
1.53000247478485 0.899359405040741\\
1.54000246524811 0.895654201507568\\
1.55000245571136 0.891884982585907\\
1.56000244617462 0.888051867485046\\
1.57000243663788 0.884154677391052\\
1.58000242710114 0.880193531513214\\
1.59000241756439 0.876168370246887\\
1.60000240802765 0.872079133987427\\
1.61000239849091 0.867925882339478\\
1.62000238895416 0.86370849609375\\
1.63000249862671 0.859426975250244\\
1.64000248908997 0.85508131980896\\
1.65000247955322 0.850671470165253\\
1.66000247001648 0.846197426319122\\
1.67000246047974 0.841659069061279\\
1.68000245094299 0.837056457996368\\
1.69000244140625 0.8323894739151\\
1.70000243186951 0.827658116817474\\
1.71000242233276 0.822862327098846\\
1.72000241279602 0.818002045154572\\
1.73000240325928 0.813077211380005\\
1.74000239372253 0.808087766170502\\
1.75000238418579 0.803033709526062\\
1.76000249385834 0.797915875911713\\
1.77000248432159 0.792760372161865\\
1.78000247478485 0.787597477436066\\
1.79000246524811 0.782431244850159\\
1.80000245571136 0.777255713939667\\
1.81000244617462 0.772062182426453\\
1.82000243663788 0.766842186450958\\
1.83000242710114 0.761588335037231\\
1.84000241756439 0.756294250488281\\
1.85000240802765 0.750954568386078\\
1.86000239849091 0.745564937591553\\
1.87000238895416 0.740121781826019\\
1.88000249862671 0.734622061252594\\
1.89000248908997 0.729063272476196\\
1.90000247955322 0.723443329334259\\
1.91000247001648 0.717760384082794\\
1.92000246047974 0.712012946605682\\
1.93000245094299 0.706199586391449\\
1.94000244140625 0.700319111347198\\
1.95000243186951 0.69437038898468\\
1.96000242233276 0.688745439052582\\
1.97000241279602 0.684689223766327\\
1.98000240325928 0.682687282562256\\
1.99000239372253 0.682839155197144\\
2.00000238418579 0.684832692146301\\
2.01000237464905 0.6881183385849\\
2.0200023651123 0.692070066928864\\
2.03000235557556 0.696122884750366\\
2.04000234603882 0.699903011322021\\
2.05000233650208 0.703207314014435\\
2.06000232696533 0.705953180789948\\
2.07000255584717 0.708130538463593\\
2.08000254631042 0.709769666194916\\
2.09000253677368 0.710934162139893\\
2.10000252723694 0.711767792701721\\
2.1100025177002 0.71250307559967\\
2.12000250816345 0.713563203811646\\
2.13000249862671 0.715750575065613\\
2.14000248908997 0.720194697380066\\
2.15000247955322 0.727541267871857\\
2.16000247001648 0.737007617950439\\
2.17000246047974 0.746980786323547\\
2.18000245094299 0.75606632232666\\
2.19000244140625 0.76367050409317\\
2.20000243186951 0.769839704036713\\
2.21000242233276 0.774884164333344\\
2.22000241279602 0.779128074645996\\
2.23000240325928 0.782818675041199\\
2.24000239372253 0.786113321781158\\
2.25000238418579 0.789097189903259\\
2.26000237464905 0.791806697845459\\
2.2700023651123 0.794249296188354\\
2.28000235557556 0.796417117118835\\
2.29000234603882 0.798294723033905\\
2.30000233650208 0.799862682819366\\
2.31000232696533 0.801101744174957\\
2.32000255584717 0.802003145217896\\
2.33000254631042 0.802580416202545\\
2.34000253677368 0.802861630916595\\
2.35000252723694 0.802873730659485\\
2.3600025177002 0.802634477615356\\
2.37000250816345 0.802153050899506\\
2.38000249862671 0.801429092884064\\
2.39000248908997 0.800221979618073\\
2.40000247955322 0.798298180103302\\
2.41000247001648 0.795549929141998\\
2.42000246047974 0.791922390460968\\
2.43000245094299 0.787411272525787\\
2.44000244140625 0.78205281496048\\
2.45000243186951 0.775913238525391\\
2.46000242233276 0.769079387187958\\
2.47000241279602 0.761649787425995\\
2.48000240325928 0.75372725725174\\
2.49000239372253 0.745413899421692\\
2.50000238418579 0.736807584762573\\
2.51000237464905 0.727999925613403\\
2.5200023651123 0.719075739383698\\
2.53000235557556 0.71011346578598\\
2.54000234603882 0.701183497905731\\
2.55000233650208 0.692348182201386\\
2.56000232696533 0.683664500713348\\
2.57000255584717 0.675186276435852\\
2.58000254631042 0.666962385177612\\
2.59000253677368 0.659052014350891\\
2.60000252723694 0.651571691036224\\
2.6100025177002 0.644659519195557\\
2.62000250816345 0.639001190662384\\
2.63000249862671 0.636265516281128\\
2.64000248908997 0.636875569820404\\
2.65000247955322 0.640672266483307\\
2.66000247001648 0.646990716457367\\
2.67000246047974 0.654961466789246\\
2.68000245094299 0.663949131965637\\
2.69000244140625 0.67361444234848\\
2.70000243186951 0.683757841587067\\
2.71000242233276 0.694252073764801\\
2.72000241279602 0.704960525035858\\
2.73000240325928 0.715604186058044\\
2.74000239372253 0.726011276245117\\
2.75000238418579 0.736059725284576\\
2.76000237464905 0.74566262960434\\
2.7700023651123 0.754777073860168\\
2.78000235557556 0.763424098491669\\
2.79000234603882 0.771661460399628\\
2.80000233650208 0.779595911502838\\
2.81000232696533 0.787380576133728\\
2.82000255584717 0.795373737812042\\
2.83000254631042 0.803945481777191\\
2.84000253677368 0.812890529632568\\
2.85000252723694 0.821369171142578\\
2.8600025177002 0.828521132469177\\
2.87000250816345 0.833931267261505\\
2.88000249862671 0.837590038776398\\
2.89000248908997 0.839699506759644\\
2.90000247955322 0.840515732765198\\
2.91000247001648 0.840286910533905\\
2.92000246047974 0.839248299598694\\
2.93000245094299 0.837615489959717\\
2.94000244140625 0.835552573204041\\
2.95000243186951 0.833161592483521\\
2.96000242233276 0.83050400018692\\
2.97000241279602 0.827616453170776\\
2.98000240325928 0.82451593875885\\
2.99000239372253 0.820978939533234\\
3.00000238418579 0.816808044910431\\
3.01000237464905 0.811919093132019\\
3.0200023651123 0.806261420249939\\
3.03000235557556 0.799826622009277\\
3.04000234603882 0.792640388011932\\
3.05000233650208 0.78475558757782\\
3.06000232696533 0.776243686676025\\
3.07000255584717 0.76718533039093\\
3.08000254631042 0.757668673992157\\
3.09000253677368 0.747788429260254\\
3.10000252723694 0.737645208835602\\
3.1100025177002 0.727348446846008\\
3.12000250816345 0.717015266418457\\
3.13000249862671 0.706768274307251\\
3.14000248908997 0.696734964847565\\
3.15000247955322 0.687047123908997\\
3.16000247001648 0.677841663360596\\
3.17000246047974 0.669269144535065\\
3.18000245094299 0.661491990089417\\
3.19000244140625 0.65468418598175\\
3.20000243186951 0.650017201900482\\
3.21000242233276 0.648434698581696\\
3.22000241279602 0.650020599365234\\
3.23000240325928 0.654721796512604\\
3.24000239372253 0.662368059158325\\
3.25000238418579 0.672687172889709\\
3.26000237464905 0.68532407283783\\
3.2700023651123 0.699792206287384\\
3.28000235557556 0.715433180332184\\
3.29000234603882 0.731622636318207\\
3.30000233650208 0.74785315990448\\
3.31000232696533 0.763584196567535\\
3.32000255584717 0.777789115905762\\
3.33000254631042 0.789840221405029\\
3.34000253677368 0.799637496471405\\
3.35000252723694 0.80733323097229\\
3.3600025177002 0.813177108764648\\
3.37000250816345 0.817418158054352\\
3.38000249862671 0.820353150367737\\
3.39000248908997 0.822400808334351\\
3.40000247955322 0.824148237705231\\
3.41000247001648 0.826191127300262\\
3.42000246047974 0.828715562820435\\
3.43000245094299 0.831327855587006\\
3.44000244140625 0.833372116088867\\
3.45000243186951 0.834353864192963\\
3.46000242233276 0.834100604057312\\
3.47000241279602 0.832691431045532\\
3.48000240325928 0.830315828323364\\
3.49000239372253 0.827174186706543\\
3.50000238418579 0.823444366455078\\
3.51000237464905 0.819157183170319\\
3.5200023651123 0.814174652099609\\
3.53000235557556 0.80850887298584\\
3.54000234603882 0.802171409130096\\
3.55000233650208 0.795189440250397\\
3.56000232696533 0.787611842155457\\
3.57000255584717 0.779510974884033\\
3.58000254631042 0.770982265472412\\
3.59000253677368 0.762116491794586\\
3.60000252723694 0.75298810005188\\
3.6100025177002 0.743649303913116\\
3.62000250816345 0.734138906002045\\
3.63000249862671 0.72450578212738\\
3.64000248908997 0.714825749397278\\
3.65000247955322 0.705201268196106\\
3.66000247001648 0.695751786231995\\
3.67000246047974 0.686601459980011\\
3.68000245094299 0.677873849868774\\
3.69000244140625 0.669694840908051\\
3.70000243186951 0.662191450595856\\
3.71000242233276 0.655487895011902\\
3.72000241279602 0.650723576545715\\
3.73000240325928 0.648885309696198\\
3.74000239372253 0.649973750114441\\
3.75000238418579 0.653858482837677\\
3.76000237464905 0.660316050052643\\
3.7700023651123 0.669060170650482\\
3.78000235557556 0.679768145084381\\
3.79000234603882 0.692050218582153\\
3.80000233650208 0.705434203147888\\
3.81000232696533 0.71948653459549\\
3.82000255584717 0.733878612518311\\
3.83000254631042 0.748284876346588\\
3.84000253677368 0.762206435203552\\
3.85000252723694 0.775071322917938\\
3.8600025177002 0.786237180233002\\
3.87000250816345 0.795463681221008\\
3.88000249862671 0.802794516086578\\
3.89000248908997 0.80842137336731\\
3.90000247955322 0.812635660171509\\
3.91000247001648 0.815868079662323\\
3.92000246047974 0.818732440471649\\
3.93000245094299 0.821850001811981\\
3.94000244140625 0.825400650501251\\
3.95000243186951 0.828947901725769\\
3.96000242233276 0.831803321838379\\
3.97000241279602 0.833474278450012\\
3.98000240325928 0.833809614181519\\
3.99000239372253 0.832912147045136\\
4.00000238418579 0.830981612205505\\
4.01000261306763 0.828209102153778\\
4.0200023651123 0.824769854545593\\
4.03000259399414 0.820816099643707\\
4.04000234603882 0.816367745399475\\
4.05000257492065 0.811372339725494\\
4.06000232696533 0.805842399597168\\
4.07000255584717 0.799781084060669\\
4.08000230789185 0.793201327323914\\
4.09000253677368 0.786133170127869\\
4.10000228881836 0.778631329536438\\
4.1100025177002 0.770748496055603\\
4.12000226974487 0.762511432170868\\
4.13000249862671 0.753932535648346\\
4.14000225067139 0.745044589042664\\
4.15000247955322 0.735920429229736\\
4.1600022315979 0.726675033569336\\
4.17000246047974 0.717452049255371\\
4.18000221252441 0.708409070968628\\
4.19000244140625 0.699700772762299\\
4.20000267028809 0.691461563110352\\
4.21000242233276 0.683844923973084\\
4.2200026512146 0.676974713802338\\
4.23000240325928 0.671293795108795\\
4.24000263214111 0.668199121952057\\
4.25000238418579 0.667755961418152\\
4.26000261306763 0.66975611448288\\
4.2700023651123 0.67393434047699\\
4.28000259399414 0.680002391338348\\
4.29000234603882 0.687669932842255\\
4.30000257492065 0.696648836135864\\
4.31000232696533 0.706620633602142\\
4.32000255584717 0.717259347438812\\
4.33000230789185 0.728299975395203\\
4.34000253677368 0.739564657211304\\
4.35000228881836 0.750948131084442\\
4.3600025177002 0.762172162532806\\
4.37000226974487 0.772863268852234\\
4.38000249862671 0.782439708709717\\
4.39000225067139 0.79052060842514\\
4.40000247955322 0.797021567821503\\
4.4100022315979 0.802044987678528\\
4.42000246047974 0.805847465991974\\
4.43000221252441 0.808882057666779\\
4.44000244140625 0.811804234981537\\
4.45000267028809 0.815179049968719\\
4.46000242233276 0.819006741046906\\
4.4700026512146 0.822681307792664\\
4.48000240325928 0.825474560260773\\
4.49000263214111 0.826951384544373\\
4.50000238418579 0.827024698257446\\
4.51000261306763 0.825829267501831\\
4.5200023651123 0.823601603507996\\
4.53000259399414 0.820559978485107\\
4.54000234603882 0.816870450973511\\
4.55000257492065 0.812584280967712\\
4.56000232696533 0.8076251745224\\
4.57000255584717 0.802040159702301\\
4.58000230789185 0.79586386680603\\
4.59000253677368 0.789134502410889\\
4.60000228881836 0.78190416097641\\
4.6100025177002 0.774235308170319\\
4.62000226974487 0.766205191612244\\
4.63000249862671 0.757900238037109\\
4.64000225067139 0.749383747577667\\
4.65000247955322 0.740675866603851\\
4.6600022315979 0.731779456138611\\
4.67000246047974 0.722719192504883\\
4.68000221252441 0.71356862783432\\
4.69000244140625 0.704450368881226\\
4.70000267028809 0.6955206990242\\
4.71000242233276 0.686956346035004\\
4.7200026512146 0.678950309753418\\
4.73000240325928 0.671709358692169\\
4.74000263214111 0.665537655353546\\
4.75000238418579 0.66224879026413\\
4.76000261306763 0.662468492984772\\
4.7700023651123 0.66599428653717\\
4.78000259399414 0.672457456588745\\
4.79000234603882 0.681407451629639\\
4.80000257492065 0.692347288131714\\
4.81000232696533 0.704669594764709\\
4.82000255584717 0.717771351337433\\
4.83000230789185 0.731148719787598\\
4.84000253677368 0.744415760040283\\
4.85000228881836 0.757317245006561\\
4.8600025177002 0.769666492938995\\
4.87000226974487 0.780902028083801\\
4.88000249862671 0.790477156639099\\
4.89000225067139 0.798252642154694\\
4.90000247955322 0.804286897182465\\
4.9100022315979 0.808729112148285\\
4.92000246047974 0.811772644519806\\
4.93000221252441 0.813669919967651\\
4.94000244140625 0.81481146812439\\
4.95000267028809 0.815780937671661\\
4.96000242233276 0.817187070846558\\
4.9700026512146 0.819230139255524\\
4.98000240325928 0.821508407592773\\
4.99000263214111 0.823348879814148\\
5 0.824254989624023\\
};
\addlegendentry{Nonlinear Mode};

\end{axis}
\end{tikzpicture}%
\caption{Flashing solution at \dtmax{} = \expneg{1.0}{5}{[s]}.}
\label{fig:flashing1pt0000em5}
\end{figure}

Note that the nonlinearly resolved solution is qualitatively different than the linear single-shot solution, \fig{fig:flashing1pt0000em0}.
Even as the \dtmax{} is reduced, this discrepancy does not disappear, \fig{fig:flashing1pt0000em5}.
The two solutions do not converge to the same solution as the timestep size is reduced.
However, the solution to the flashing problem produced by the nonlinear solver, \fig{fig:flashingAlphaNln}, qualitatively varies less as the \dtmax{} is reduced than that produced by the linear solver, \fig{fig:flashingAlphaLin}.
However, both \fig{fig:flashingAlphaLin} and \fig{fig:flashingAlphaNln} show that as the timestep size is reduced the two different solutions become more timestep-size insensitive.

\begin{figure}[h!tb]
\centering
% This file was created by matlab2tikz v0.4.3.
% Copyright (c) 2008--2013, Nico Schlömer <nico.schloemer@gmail.com>
% All rights reserved.
% 
\tikzsetnextfilename{plots/flashingAlphaLin_pdf}
\begin{tikzpicture}

\begin{axis}[%
width=\mytikzpicwidth,
height=\mytikzpicheight,
scale only axis,
xmin=2.0,
xmax=4.0,
xlabel={Time $[\text{s}]$},
ymin=0.5,
ymax=0.9,
ylabel={$\alpha_g$ [-] @ 2 [in] from Inlet},
legend columns=2,
legend style={at={(0.03,0.03)},%
	anchor=south west,%
	draw=black,%
	fill=white,%
	legend cell align=left,
	/tikz/column 2/.style={column sep=10pt,}}
]
\addplot [
color=black,
solid,
mark=o,
mark options={solid},
mark repeat={5}
]
table[row sep=crcr]{
0.0 1.0\\
0.0118019115179777 1\\
0.0226854234933853 1\\
0.0340500771999359 1\\
0.0491764321923256 1\\
0.0619497485458851 1\\
0.0774054601788521 1\\
0.0961068719625473 1\\
0.106882445514202 1\\
0.118735581636429 1\\
0.13177402317524 1\\
0.146116316318512 1\\
0.161892831325531 1\\
0.179247006773949 1\\
0.198336601257324 1\\
0.21933513879776 1\\
0.242433547973633 1\\
0.267841786146164 1\\
0.29579085111618 1\\
0.32653483748436 1\\
0.360353201627731 1\\
0.397553414106369 1\\
0.438473641872406 1\\
0.483485877513886 1\\
0.532999336719513 1\\
0.587464153766632 1\\
0.647375464439392 1\\
0.71327793598175 1\\
0.7857705950737 1\\
0.865512549877167 1\\
0.953228712081909 1\\
1.04971647262573 0.997021615505219\\
1.1558530330658 0.985307455062866\\
1.2407523393631 0.970276594161987\\
1.29757130146027 0.957477152347565\\
1.34479975700378 0.945545017719269\\
1.38458728790283 0.934612452983856\\
1.41912090778351 0.924424767494202\\
1.45013403892517 0.914681613445282\\
1.47872579097748 0.90517920255661\\
1.5054817199707 0.895826578140259\\
1.53073275089264 0.88658881187439\\
1.55469310283661 0.877453088760376\\
1.57754015922546 0.868404984474182\\
1.59942138195038 0.859430253505707\\
1.62045061588287 0.850519120693207\\
1.64072275161743 0.841663300991058\\
1.6603125333786 0.832857310771942\\
1.67928779125214 0.824094474315643\\
1.69770324230194 0.815370202064514\\
1.71560847759247 0.806679904460907\\
1.73303949832916 0.79802268743515\\
1.75003552436829 0.789430022239685\\
1.76655149459839 0.781055510044098\\
1.78238117694855 0.772998511791229\\
1.79760921001434 0.765193462371826\\
1.81230998039246 0.757585883140564\\
1.82662653923035 0.750090539455414\\
1.84061861038208 0.742669582366943\\
1.85433006286621 0.735296964645386\\
1.86779499053955 0.72795307636261\\
1.88103151321411 0.720627963542938\\
1.89406371116638 0.71331000328064\\
1.90690815448761 0.70599102973938\\
1.91956806182861 0.698671102523804\\
1.93206679821014 0.691338658332825\\
1.94440972805023 0.683992326259613\\
1.95660436153412 0.676629960536957\\
1.96866297721863 0.671738088130951\\
1.98027718067169 0.672812759876251\\
1.99125838279724 0.677815139293671\\
2.00170540809631 0.684370279312134\\
2.01211428642273 0.691331386566162\\
2.02296447753906 0.698101699352264\\
2.03420996665955 0.704113781452179\\
2.04571580886841 0.709079086780548\\
2.05740904808044 0.712941884994507\\
2.06921482086182 0.71580046415329\\
2.08101534843445 0.717850506305695\\
2.09267902374268 0.719229459762573\\
2.10420393943787 0.720068991184235\\
2.1155960559845 0.720495879650116\\
2.12691164016724 0.720689058303833\\
2.13817572593689 0.721016407012939\\
2.14939737319946 0.722309648990631\\
2.16053771972656 0.725886106491089\\
2.17153453826904 0.732412099838257\\
2.18234181404114 0.740104675292969\\
2.19297122955322 0.747277319431305\\
2.2034854888916 0.753501296043396\\
2.21399807929993 0.758969664573669\\
2.22421646118164 0.76383113861084\\
2.24485516548157 0.773090600967407\\
2.2556414604187 0.777811586856842\\
2.26643514633179 0.78242814540863\\
2.27722930908203 0.786858320236206\\
2.28802180290222 0.791011035442352\\
2.2988224029541 0.794807195663452\\
2.30910420417786 0.798032104969025\\
2.32745814323425 0.80274897813797\\
2.34450769424438 0.805849254131317\\
2.36390542984009 0.807824075222015\\
2.37500143051147 0.808235883712769\\
2.38614940643311 0.808172166347504\\
2.39734673500061 0.807666897773743\\
2.40859293937683 0.806699872016907\\
2.41979336738586 0.805263638496399\\
2.43107962608337 0.802674949169159\\
2.44180870056152 0.79903507232666\\
2.4530987739563 0.793921113014221\\
2.46468305587769 0.787385225296021\\
2.47643661499023 0.779563665390015\\
2.48832750320435 0.77061265707016\\
2.50033926963806 0.760712027549744\\
2.5124671459198 0.750045478343964\\
2.5246946811676 0.738811135292053\\
2.53700971603394 0.727198243141174\\
2.54939413070679 0.71539968252182\\
2.56182551383972 0.703596353530884\\
2.57428097724915 0.691951513290405\\
2.58673787117004 0.680607497692108\\
2.59917783737183 0.669684410095215\\
2.61158609390259 0.659270703792572\\
2.62395215034485 0.649651288986206\\
2.6361358165741 0.641215026378632\\
2.64810156822205 0.635994970798492\\
2.6597592830658 0.635662436485291\\
2.67102980613709 0.639588057994843\\
2.68197822570801 0.64653742313385\\
2.69267916679382 0.655223429203033\\
2.70332479476929 0.665210664272308\\
2.71397757530212 0.676258027553558\\
2.72448706626892 0.687942206859589\\
2.74232935905457 0.708736419677734\\
2.75233578681946 0.723106145858765\\
2.76248455047607 0.746678531169891\\
2.77299618721008 0.767698109149933\\
2.78397274017334 0.786229491233826\\
2.79858231544495 0.805485188961029\\
2.81091928482056 0.817172288894653\\
2.82584691047668 0.826655030250549\\
2.84390950202942 0.832634270191193\\
2.85421967506409 0.83385705947876\\
2.86443328857422 0.83380115032196\\
2.87539649009705 0.832415580749512\\
2.88654828071594 0.829780161380768\\
2.89785838127136 0.825988829135895\\
2.90905976295471 0.821340024471283\\
2.92766308784485 0.812942206859589\\
2.9380214214325 0.808586359024048\\
2.94941568374634 0.80469822883606\\
2.96126461029053 0.801513075828552\\
2.97307300567627 0.798118710517883\\
2.98484778404236 0.793721735477448\\
2.99659848213196 0.787129402160645\\
3.00838160514832 0.778874039649963\\
3.02029156684875 0.769327104091644\\
3.03234505653381 0.758802771568298\\
3.04447245597839 0.747617602348328\\
3.05664157867432 0.736036121845245\\
3.06882643699646 0.724325954914093\\
3.08100342750549 0.712768971920013\\
3.09315085411072 0.701684951782227\\
3.10524773597717 0.691458225250244\\
3.11726975440979 0.682536721229553\\
3.12919759750366 0.675742566585541\\
3.14099335670471 0.672509431838989\\
3.15258574485779 0.673115789890289\\
3.16393399238586 0.677472770214081\\
3.1750283241272 0.685294687747955\\
3.18587756156921 0.696173489093781\\
3.19650077819824 0.709694087505341\\
3.20691919326782 0.725244760513306\\
3.21715044975281 0.742033123970032\\
3.22718858718872 0.758418679237366\\
3.2467474937439 0.784654200077057\\
3.26547813415527 0.801358640193939\\
3.28411507606506 0.811337649822235\\
3.29831433296204 0.815494418144226\\
3.3087158203125 0.816964566707611\\
3.32302951812744 0.816825926303864\\
3.33405566215515 0.815179109573364\\
3.3473973274231 0.811917245388031\\
3.36354088783264 0.807795643806458\\
3.38307428359985 0.80543839931488\\
3.3943293094635 0.804838478565216\\
3.40593409538269 0.803381621837616\\
3.41763758659363 0.800470113754272\\
3.42948269844055 0.796141386032104\\
3.44148325920105 0.790801346302032\\
3.45353007316589 0.784392714500427\\
3.4655749797821 0.777148246765137\\
3.48354387283325 0.765116512775421\\
3.4938976764679 0.757575452327728\\
3.50528645515442 0.748863220214844\\
3.51756191253662 0.739044904708862\\
3.52981090545654 0.728857576847076\\
3.54198026657104 0.718411505222321\\
3.55405426025391 0.707884550094604\\
3.56603574752808 0.697501063346863\\
3.57792735099792 0.687490999698639\\
3.58974385261536 0.678059160709381\\
3.60148811340332 0.669394671916962\\
3.61318278312683 0.661635458469391\\
3.62483644485474 0.656085550785065\\
3.63638854026794 0.653754055500031\\
3.64775395393372 0.654484272003174\\
3.65890121459961 0.657902538776398\\
3.66983270645142 0.663596034049988\\
3.68056321144104 0.671233534812927\\
3.69111466407776 0.680723905563354\\
3.70150089263916 0.691940426826477\\
3.71173763275146 0.704434335231781\\
3.72185039520264 0.717744290828705\\
3.73186469078064 0.731415927410126\\
3.75170636177063 0.758384227752686\\
3.7714056968689 0.78208589553833\\
3.7911114692688 0.798542380332947\\
3.8088846206665 0.807387351989746\\
3.82693791389465 0.81149297952652\\
3.84111881256104 0.812401294708252\\
3.85744023323059 0.812185168266296\\
3.87718939781189 0.813103318214417\\
3.88805747032166 0.814119160175323\\
3.89892983436584 0.814466416835785\\
3.90982556343079 0.813674628734589\\
3.92078495025635 0.811778545379639\\
3.9318573474884 0.808977901935577\\
3.94304871559143 0.805517554283142\\
3.95445036888123 0.801541984081268\\
3.96601796150208 0.797248244285584\\
3.97746133804321 0.792374908924103\\
3.98889112472534 0.786799848079681\\
4.00039577484131 0.780433714389801\\
4.01201629638672 0.773246467113495\\
4.02377510070801 0.765269994735718\\
4.03563594818115 0.756564557552338\\
4.04752445220947 0.747200310230255\\
4.05936098098755 0.737273216247559\\
4.0711030960083 0.726977527141571\\
4.08272695541382 0.716625571250916\\
4.09423494338989 0.706573486328125\\
4.10562992095947 0.697181880474091\\
4.11694383621216 0.688742518424988\\
4.12819576263428 0.681517958641052\\
4.13939952850342 0.677124381065369\\
4.1505651473999 0.676256716251373\\
4.16158246994019 0.678564786911011\\
4.17241191864014 0.683457136154175\\
4.18306303024292 0.690370500087738\\
4.1935658454895 0.698813796043396\\
4.2039475440979 0.708419322967529\\
4.21423053741455 0.718966424465179\\
4.22443199157715 0.730270206928253\\
4.2345666885376 0.742040574550629\\
4.24464702606201 0.754002392292023\\
4.25469064712524 0.765895545482636\\
4.26471757888794 0.776910960674286\\
4.27471828460693 0.78636509180069\\
4.2847638130188 0.794048726558685\\
4.29489898681641 0.799984812736511\\
4.30505514144897 0.804239630699158\\
4.31983995437622 0.807999908924103\\
4.33454179763794 0.808928072452545\\
4.34839296340942 0.807726681232452\\
4.36166000366211 0.805512011051178\\
4.37771320343018 0.802959859371185\\
4.39713716506958 0.802385091781616\\
4.40819406509399 0.802595138549805\\
4.41921901702881 0.801958739757538\\
4.43028879165649 0.800117373466492\\
4.44142246246338 0.797202825546265\\
4.45270013809204 0.793436765670776\\
4.46414184570313 0.789081275463104\\
4.47538042068481 0.784011363983154\\
4.48650979995728 0.778233647346497\\
4.49770545959473 0.771687626838684\\
4.50898551940918 0.764402031898499\\
4.52039957046509 0.756402134895325\\
4.5319881439209 0.747746407985687\\
4.54377889633179 0.73848021030426\\
4.55586624145508 0.728542983531952\\
4.56815242767334 0.718027353286743\\
4.58033037185669 0.707345902919769\\
4.59236669540405 0.696863949298859\\
4.60425090789795 0.687036275863647\\
4.6159782409668 0.678349554538727\\
4.62757444381714 0.671239972114563\\
4.63906097412109 0.668458938598633\\
4.65037202835083 0.669922590255737\\
4.66144800186157 0.675077080726624\\
4.67229557037354 0.683269858360291\\
4.68293762207031 0.693869650363922\\
4.69340085983276 0.706282436847687\\
4.70371246337891 0.719945728778839\\
4.71390056610107 0.734441220760345\\
4.72399282455444 0.749322056770325\\
4.73401403427124 0.7640261054039\\
4.75387620925903 0.788362562656403\\
4.77353286743164 0.804273962974548\\
4.79333353042603 0.813261687755585\\
4.80498886108398 0.816133856773376\\
4.81894111633301 0.817074656486511\\
4.83303546905518 0.815881550312042\\
4.84702253341675 0.813483476638794\\
4.86394691467285 0.810876548290253\\
4.8844256401062 0.810437440872192\\
4.89526796340942 0.810307621955872\\
4.9058313369751 0.809220314025879\\
4.91633558273315 0.807012498378754\\
4.92694330215454 0.803822040557861\\
4.93775272369385 0.799894034862518\\
4.94876766204834 0.795530676841736\\
4.96000909805298 0.790432095527649\\
4.97146701812744 0.784575462341309\\
4.98318338394165 0.777906775474548\\
5 0.767241597175598\\
};
\addlegendentry{$\dtmax{} = \expneg{1.0}{0}$ [s]};

\addplot [
color=black,
solid,
mark=+,
mark options={solid},
mark repeat={7}
]
table[row sep=crcr]{
0 1\\
0.0118019115179777 1\\
0.0226854234933853 1\\
0.0340500771999359 1\\
0.0491764321923256 1\\
0.0619497485458851 1\\
0.0774054601788521 1\\
0.0961068719625473 1\\
0.106882445514202 1\\
0.118735581636429 1\\
0.13177402317524 1\\
0.146116316318512 1\\
0.161892831325531 1\\
0.179247006773949 1\\
0.198336601257324 1\\
0.21933513879776 1\\
0.242433547973633 1\\
0.267841786146164 1\\
0.29579085111618 1\\
0.32653483748436 1\\
0.360353201627731 1\\
0.397553414106369 1\\
0.438473641872406 1\\
0.483485877513886 1\\
0.532999336719513 1\\
0.587464153766632 1\\
0.647375464439392 1\\
0.71327793598175 1\\
0.7857705950737 1\\
0.865512549877167 1\\
0.953228712081909 1\\
1.04971647262573 0.997021615505219\\
1.14971649646759 0.986351609230042\\
1.2388744354248 0.970792293548584\\
1.29616057872772 0.957939326763153\\
1.34285283088684 0.946207404136658\\
1.38287508487701 0.935246229171753\\
1.41761469841003 0.925026714801788\\
1.44881391525269 0.915248930454254\\
1.47750616073608 0.905734837055206\\
1.50434100627899 0.896374046802521\\
1.52965128421783 0.887132167816162\\
1.55366480350494 0.87799197435379\\
1.57655787467957 0.868940055370331\\
1.59847927093506 0.859962105751038\\
1.61954414844513 0.851048231124878\\
1.63984787464142 0.842190027236938\\
1.65946626663208 0.833381831645966\\
1.67846739292145 0.824617147445679\\
1.69690656661987 0.815891146659851\\
1.7148334980011 0.807199358940125\\
1.73228430747986 0.798540771007538\\
1.74929881095886 0.789937436580658\\
1.76585221290588 0.781542837619781\\
1.78171455860138 0.773469805717468\\
1.79697132110596 0.765652060508728\\
1.81169188022614 0.758037328720093\\
1.82602417469025 0.750537514686584\\
1.84002900123596 0.743113934993744\\
1.85375189781189 0.735739588737488\\
1.86722791194916 0.728394269943237\\
1.88047194480896 0.721070051193237\\
1.89351260662079 0.713752150535584\\
1.90636217594147 0.706435024738312\\
1.91903162002563 0.699114441871643\\
1.93153655529022 0.691782832145691\\
1.94388437271118 0.684438705444336\\
1.95608687400818 0.67707633972168\\
1.96814596652985 0.671899914741516\\
1.9798002243042 0.672712087631226\\
1.99079525470734 0.677557289600372\\
2.00123715400696 0.68403571844101\\
2.01163697242737 0.690981984138489\\
2.02247047424316 0.697771728038788\\
2.03369832038879 0.703823328018188\\
2.04519176483154 0.708838582038879\\
2.05687618255615 0.712753057479858\\
2.06867718696594 0.715651094913483\\
2.08048534393311 0.717734813690186\\
2.09215831756592 0.719141185283661\\
2.10369086265564 0.720003366470337\\
2.11508893966675 0.720449388027191\\
2.12640833854675 0.720659196376801\\
2.13767504692078 0.721001625061035\\
2.1488983631134 0.722309768199921\\
2.1600399017334 0.725902020931244\\
2.17103719711304 0.732443273067474\\
2.18184471130371 0.740142703056335\\
2.19247317314148 0.747315406799316\\
2.2029869556427 0.753536224365234\\
2.21349930763245 0.758999824523926\\
2.223717212677 0.763855814933777\\
2.24435567855835 0.773103356361389\\
2.25514197349548 0.777817904949188\\
2.26593589782715 0.782427966594696\\
2.27673006057739 0.786851823329926\\
2.28752279281616 0.79099839925766\\
2.2983238697052 0.794788479804993\\
2.30863809585571 0.798012614250183\\
2.32703828811646 0.802706897258759\\
2.34415745735168 0.805782437324524\\
2.36360955238342 0.807716012001038\\
2.37471151351929 0.808102190494537\\
2.38586473464966 0.808014631271362\\
2.39706635475159 0.807488203048706\\
2.40831589698792 0.806493878364563\\
2.41951632499695 0.805034399032593\\
2.43080687522888 0.802421391010284\\
2.44148182868958 0.798784017562866\\
2.45279097557068 0.793648898601532\\
2.4643726348877 0.787106215953827\\
2.47612810134888 0.779280304908752\\
2.48802018165588 0.770330369472504\\
2.50003337860107 0.760434985160828\\
2.51216316223145 0.749777495861053\\
2.52439212799072 0.738555312156677\\
2.53670859336853 0.726956903934479\\
2.54909420013428 0.715174615383148\\
2.56152653694153 0.703388333320618\\
2.57398223876953 0.6917604804039\\
2.58643913269043 0.680432438850403\\
2.59887886047363 0.669523179531097\\
2.61128640174866 0.65911990404129\\
2.6236515045166 0.6495201587677\\
2.63582587242126 0.641115844249725\\
2.64778065681458 0.636001825332642\\
2.659423828125 0.635791957378387\\
2.67067933082581 0.639823436737061\\
2.68161129951477 0.646842300891876\\
2.69230103492737 0.65556663274765\\
2.70294141769409 0.66556191444397\\
2.71359300613403 0.676590740680695\\
2.72410321235657 0.688234090805054\\
2.74203324317932 0.709035813808441\\
2.7522988319397 0.724397957324982\\
2.76380586624146 0.752012193202972\\
2.77405524253845 0.772793173789978\\
2.78475761413574 0.790979564189911\\
2.7990026473999 0.809657752513886\\
2.81103157997131 0.820952892303467\\
2.82558655738831 0.830124080181122\\
2.84319829940796 0.835985720157623\\
2.85334610939026 0.837234139442444\\
2.86414122581482 0.837133586406708\\
2.87511372566223 0.835717022418976\\
2.88625264167786 0.833053946495056\\
2.89754724502563 0.829225063323975\\
2.90883326530457 0.824456036090851\\
2.92768001556396 0.815642714500427\\
2.93819761276245 0.810966014862061\\
2.94976687431335 0.806747138500214\\
2.96163415908813 0.803638696670532\\
2.97345376014709 0.800946414470673\\
2.98522281646729 0.797514140605927\\
2.9969425201416 0.791235864162445\\
3.00866484642029 0.78293764591217\\
3.02053761482239 0.77314168214798\\
3.03258919715881 0.7622349858284\\
3.04473614692688 0.750587403774261\\
3.0569326877594 0.738493502140045\\
3.0691511631012 0.726226091384888\\
3.08136558532715 0.714067995548248\\
3.09355187416077 0.702342867851257\\
3.10568857192993 0.691434383392334\\
3.11775040626526 0.681787312030792\\
3.12971878051758 0.673895657062531\\
3.14157199859619 0.669552743434906\\
3.15323138237 0.669406414031982\\
3.16463661193848 0.6733757853508\\
3.17577147483826 0.681038796901703\\
3.18664932250977 0.692003190517426\\
3.19729042053223 0.705828011035919\\
3.20771455764771 0.721758902072906\\
3.21794104576111 0.738930284976959\\
3.2279806137085 0.756408095359802\\
3.24761056900024 0.785510063171387\\
3.26646447181702 0.804471671581268\\
3.2851665019989 0.816074788570404\\
3.30015993118286 0.821442127227783\\
3.31109523773193 0.823484838008881\\
3.32280397415161 0.824106812477112\\
3.33651351928711 0.822816729545593\\
3.34809064865112 0.820433855056763\\
3.36209893226624 0.81698751449585\\
3.37904906272888 0.81411737203598\\
3.39955854415894 0.812461078166962\\
3.41116738319397 0.810398995876312\\
3.42286372184753 0.806844294071198\\
3.43470287322998 0.801888048648834\\
3.4465663433075 0.79605621099472\\
3.45787000656128 0.789692580699921\\
3.47379469871521 0.779717266559601\\
3.4930636882782 0.766135513782501\\
3.50416612625122 0.757685422897339\\
3.51637887954712 0.747951924800873\\
3.52864861488342 0.737764000892639\\
3.54084873199463 0.727272868156433\\
3.55296158790588 0.716613948345184\\
3.56498837471008 0.705977141857147\\
3.57693076133728 0.695570886135101\\
3.58879470825195 0.68558794260025\\
3.60058760643005 0.676199913024902\\
3.6123309135437 0.66754686832428\\
3.62403297424316 0.659753262996674\\
3.63569378852844 0.654087781906128\\
3.64725041389465 0.65157949924469\\
3.65861344337463 0.65206652879715\\
3.66975116729736 0.655189096927643\\
3.68066668510437 0.660557746887207\\
3.69137978553772 0.667939364910126\\
3.70191335678101 0.677314460277557\\
3.71228361129761 0.688492953777313\\
3.72250723838806 0.701058208942413\\
3.7325963973999 0.714500308036804\\
3.75248122215271 0.742160975933075\\
3.77218985557556 0.768642783164978\\
3.79181623458862 0.789814829826355\\
3.81167793273926 0.803487062454224\\
3.82728266716003 0.809676468372345\\
3.84524822235107 0.812438070774078\\
3.85556268692017 0.812902748584747\\
3.86804342269897 0.812742531299591\\
3.88314509391785 0.812954246997833\\
3.90141797065735 0.814873397350311\\
3.91194653511047 0.815695583820343\\
3.92283082008362 0.815373361110687\\
3.93378806114197 0.813795745372772\\
3.94480562210083 0.811182737350464\\
3.95595955848694 0.807796180248261\\
3.96731543540955 0.803848028182983\\
3.97884368896484 0.799512267112732\\
3.99042797088623 0.79450535774231\\
4.0019850730896 0.788779079914093\\
4.0135235786438 0.782292366027832\\
4.02511358261108 0.775009036064148\\
4.03680324554443 0.766947686672211\\
4.04859256744385 0.758141338825226\\
4.06042098999023 0.748647809028625\\
4.07218551635742 0.73859578371048\\
4.08385276794434 0.728193044662476\\
4.09540271759033 0.717761337757111\\
4.10684585571289 0.707655429840088\\
4.11820030212402 0.69822359085083\\
4.12947559356689 0.689776718616486\\
4.14068651199341 0.682577967643738\\
4.15185976028442 0.678209066390991\\
4.16300344467163 0.67734032869339\\
4.17401742935181 0.679639637470245\\
4.18484449386597 0.684535086154938\\
4.19550132751465 0.691486358642578\\
4.20601320266724 0.700023174285889\\
4.21640634536743 0.709792375564575\\
4.22670125961304 0.720569550991058\\
4.23691654205322 0.732152104377747\\
4.24706697463989 0.744234085083008\\
4.25716781616211 0.75652402639389\\
4.2672381401062 0.768650829792023\\
4.27728843688965 0.779620468616486\\
4.28730154037476 0.788829803466797\\
4.29739427566528 0.796225607395172\\
4.30756187438965 0.801840007305145\\
4.31767320632935 0.805773377418518\\
4.33207941055298 0.809107422828674\\
4.34596586227417 0.809810817241669\\
4.35955810546875 0.808497190475464\\
4.372718334198 0.806077480316162\\
4.38864183425903 0.803068578243256\\
4.40790939331055 0.801928520202637\\
4.41901111602783 0.802043914794922\\
4.43009757995605 0.801405787467957\\
4.4411883354187 0.79950225353241\\
4.45242166519165 0.79637348651886\\
4.46370792388916 0.792328178882599\\
4.47520685195923 0.787645101547241\\
4.48653984069824 0.782306373119354\\
4.49777412414551 0.776326954364777\\
4.5090594291687 0.769645631313324\\
4.52043962478638 0.762257635593414\\
4.53196239471436 0.754171311855316\\
4.54368162155151 0.745421946048737\\
4.55567598342896 0.73598438501358\\
4.56790256500244 0.725897014141083\\
4.58011674880981 0.715412199497223\\
4.59221315383911 0.704851686954498\\
4.60417032241821 0.694629490375519\\
4.61597204208374 0.68523496389389\\
4.62763738632202 0.67713737487793\\
4.63918876647949 0.671582758426666\\
4.65063285827637 0.670404672622681\\
4.66186189651489 0.673338711261749\\
4.67285776138306 0.67968761920929\\
4.68363618850708 0.688751339912415\\
4.69422245025635 0.699863612651825\\
4.7046422958374 0.712404668331146\\
4.71492433547974 0.725851655006409\\
4.72509813308716 0.73985880613327\\
4.73518991470337 0.754067301750183\\
4.74522352218628 0.767756819725037\\
4.7651481628418 0.789979934692383\\
4.78499126434326 0.80428010225296\\
4.80508518218994 0.811938107013702\\
4.81628894805908 0.814046561717987\\
4.8298454284668 0.814343512058258\\
4.84408235549927 0.812585413455963\\
4.85892581939697 0.809483766555786\\
4.87688589096069 0.806458413600922\\
4.88723421096802 0.805952787399292\\
4.89843463897705 0.805981874465942\\
4.90928411483765 0.805604934692383\\
4.91984224319458 0.804205298423767\\
4.93034839630127 0.801742315292358\\
4.94095993041992 0.798401355743408\\
4.95174980163574 0.794463753700256\\
4.96273040771484 0.789646208286285\\
4.97390985488892 0.783975720405579\\
4.9853401184082 0.777438700199127\\
5 0.768127679824829\\
};
\addlegendentry{$\dtmax{} = \expneg{1.0}{1}$ [s]};

\addplot [
color=black,
dash pattern=on 1pt off 3pt on 3pt off 3pt
]
table[row sep=crcr]{
0 1\\
0.0118019115179777 1\\
0.0226854234933853 1\\
0.0340500771999359 1\\
0.0491764321923256 1\\
0.0619497485458851 1\\
0.0774054601788521 1\\
0.0961068719625473 1\\
0.106106869876385 1\\
0.116106867790222 1\\
0.12610687315464 1\\
0.136106878519058 1\\
0.146106868982315 1\\
0.156106874346733 1\\
0.16610686480999 1\\
0.176106870174408 1\\
0.186106875538826 1\\
0.196106866002083 1\\
0.206106871366501 1\\
0.216106876730919 1\\
0.226106867194176 1\\
0.236106872558594 1\\
0.246106877923012 1\\
0.25610688328743 1\\
0.266106873750687 1\\
0.276106864213943 1\\
0.286106884479523 1\\
0.29610687494278 1\\
0.306106865406036 1\\
0.316106885671616 1\\
0.326106876134872 1\\
0.336106866598129 1\\
0.346106857061386 1\\
0.356106877326965 1\\
0.366106867790222 1\\
0.376106858253479 1\\
0.386106878519058 1\\
0.396106868982315 1\\
0.406106859445572 1\\
0.416106879711151 1\\
0.426106870174408 1\\
0.436106860637665 1\\
0.446106880903244 1\\
0.456106871366501 1\\
0.466106861829758 1\\
0.476106882095337 1\\
0.486106872558594 1\\
0.496106863021851 1\\
0.506106853485107 1\\
0.516106843948364 1\\
0.526106894016266 1\\
0.536106884479523 1\\
0.54610687494278 1\\
0.556106865406036 1\\
0.566106855869293 1\\
0.57610684633255 1\\
0.586106896400452 1\\
0.596106886863709 1\\
0.606106877326965 1\\
0.616106867790222 1\\
0.626106858253479 1\\
0.636106848716736 1\\
0.646106898784637 1\\
0.656106889247894 1\\
0.666106879711151 1\\
0.676106870174408 1\\
0.686106860637665 1\\
0.696106851100922 1\\
0.706106901168823 1\\
0.71610689163208 1\\
0.726106882095337 1\\
0.736106872558594 1\\
0.746106863021851 1\\
0.756106853485107 1\\
0.766106843948364 1\\
0.776106894016266 1\\
0.786106884479523 1\\
0.79610687494278 1\\
0.806106865406036 1\\
0.816106855869293 1\\
0.82610684633255 1\\
0.836106896400452 1\\
0.846106886863709 1\\
0.856106877326965 1\\
0.866106867790222 1\\
0.876106858253479 1\\
0.886106848716736 1\\
0.896106898784637 1\\
0.906106889247894 1\\
0.916106879711151 1\\
0.926106870174408 1\\
0.936106860637665 1\\
0.946106851100922 1\\
0.956106901168823 1\\
0.96610689163208 1\\
0.976106882095337 1\\
0.986106872558594 1\\
0.996106863021851 1\\
1.00610685348511 0.999962091445923\\
1.01610684394836 0.999858677387238\\
1.02610683441162 0.999682128429413\\
1.03610682487488 0.999420166015625\\
1.04610681533813 0.999130368232727\\
1.05610692501068 0.998766839504242\\
1.06610691547394 0.998329222202301\\
1.07610690593719 0.997817575931549\\
1.08610689640045 0.997231781482697\\
1.09610688686371 0.996571123600006\\
1.10610687732697 0.995834708213806\\
1.11610686779022 0.995021402835846\\
1.12610685825348 0.994130253791809\\
1.13610684871674 0.99316018819809\\
1.14610683917999 0.992110073566437\\
1.15610682964325 0.990979015827179\\
1.16610682010651 0.989766001701355\\
1.17610692977905 0.988470077514648\\
1.18610692024231 0.987090289592743\\
1.19610691070557 0.985625684261322\\
1.20610690116882 0.984075427055359\\
1.21610689163208 0.982438683509827\\
1.22610688209534 0.980714559555054\\
1.23610687255859 0.978902280330658\\
1.24610686302185 0.977015495300293\\
1.25610685348511 0.975069344043732\\
1.26610684394836 0.973066926002502\\
1.27610683441162 0.971007227897644\\
1.28610682487488 0.968888461589813\\
1.29610681533813 0.966709017753601\\
1.30610692501068 0.964467525482178\\
1.31610691547394 0.962163150310516\\
1.32610690593719 0.959795236587524\\
1.33610689640045 0.957363426685333\\
1.34610688686371 0.954867362976074\\
1.35610687732697 0.952307045459747\\
1.36610686779022 0.949682354927063\\
1.37610685825348 0.946993231773376\\
1.38610684871674 0.944239735603333\\
1.39610683917999 0.941421926021576\\
1.40610682964325 0.938539803028107\\
1.41610682010651 0.935593485832214\\
1.42610692977905 0.932582914829254\\
1.43610692024231 0.92950826883316\\
1.44610691070557 0.926369488239288\\
1.45610690116882 0.923166692256927\\
1.46610689163208 0.919899880886078\\
1.47610688209534 0.91656905412674\\
1.48610687255859 0.913174331188202\\
1.49610686302185 0.90971565246582\\
1.50610685348511 0.906193017959595\\
1.51610684394836 0.902606546878815\\
1.52610683441162 0.898956120014191\\
1.53610682487488 0.895241796970367\\
1.54610681533813 0.89146363735199\\
1.55610692501068 0.887621462345123\\
1.56610691547394 0.883715450763702\\
1.57610690593719 0.879745423793793\\
1.58610689640045 0.875711441040039\\
1.59610688686371 0.871613562107086\\
1.60610687732697 0.867451608181\\
1.61610686779022 0.863225638866425\\
1.62610685825348 0.858935594558716\\
1.63610684871674 0.854581415653229\\
1.64610683917999 0.850163161754608\\
1.65610682964325 0.845680713653564\\
1.66610682010651 0.841134011745453\\
1.67610692977905 0.836523115634918\\
1.68610692024231 0.831847906112671\\
1.69610691070557 0.827108323574066\\
1.70610690116882 0.822304368019104\\
1.71610689163208 0.81743597984314\\
1.72610688209534 0.812503039836884\\
1.73610687255859 0.80750560760498\\
1.74610686302185 0.802443623542786\\
1.75610685348511 0.797316908836365\\
1.76610684394836 0.792151093482971\\
1.77610683441162 0.78697544336319\\
1.78610682487488 0.781793296337128\\
1.79610681533813 0.776599764823914\\
1.80610692501068 0.771387100219727\\
1.81610691547394 0.766147792339325\\
1.82610690593719 0.76087474822998\\
1.83610689640045 0.755562543869019\\
1.84610688686371 0.750205993652344\\
1.85610687732697 0.744800865650177\\
1.86610686779022 0.739343285560608\\
1.87610685825348 0.733830749988556\\
1.88610684871674 0.728260457515717\\
1.89610683917999 0.722630143165588\\
1.90610682964325 0.716938018798828\\
1.91610682010651 0.711182296276093\\
1.92610692977905 0.705361664295197\\
1.93610692024231 0.699474811553955\\
1.94610691070557 0.693520188331604\\
1.95610690116882 0.687497079372406\\
1.96610689163208 0.681404232978821\\
1.97610688209534 0.675240397453308\\
1.98610687255859 0.673442900180817\\
1.99610686302185 0.675594449043274\\
2.00610685348511 0.680603623390198\\
2.01610684394836 0.68700110912323\\
2.02610683441162 0.693596124649048\\
2.03610682487488 0.699695527553558\\
2.04610681533813 0.704988956451416\\
2.05610680580139 0.709391117095947\\
2.06610679626465 0.712919175624847\\
2.07610678672791 0.715633451938629\\
2.08610677719116 0.71770191192627\\
2.09610676765442 0.719236552715302\\
2.10610675811768 0.720321118831635\\
2.11610698699951 0.721030235290527\\
2.12610697746277 0.721451163291931\\
2.13610696792603 0.721728563308716\\
2.14610695838928 0.722166061401367\\
2.15610694885254 0.723395168781281\\
2.1661069393158 0.726407766342163\\
2.17610692977905 0.731940031051636\\
2.18610692024231 0.739692091941834\\
2.19610691070557 0.747829377651215\\
2.20610690116882 0.755074679851532\\
2.21610689163208 0.761131048202515\\
2.23586487770081 0.770546674728394\\
2.24586486816406 0.774604558944702\\
2.25586485862732 0.778461694717407\\
2.26586484909058 0.782198548316956\\
2.27586483955383 0.785822153091431\\
2.28586506843567 0.78929740190506\\
2.29586505889893 0.792571008205414\\
2.30586504936218 0.795587480068207\\
2.31586503982544 0.798298001289368\\
2.33441281318665 0.802402794361115\\
2.35220170021057 0.805108547210693\\
2.37192106246948 0.806630849838257\\
2.38192105293274 0.806852400302887\\
2.391921043396 0.806747615337372\\
2.40192103385925 0.806344926357269\\
2.41192102432251 0.805608093738556\\
2.42192101478577 0.804523885250092\\
2.43192100524902 0.802556216716766\\
2.44192099571228 0.799595654010773\\
2.45192098617554 0.795610964298248\\
2.46192097663879 0.790633320808411\\
2.47192096710205 0.784735143184662\\
2.48192095756531 0.778014123439789\\
2.49192094802856 0.770580589771271\\
2.50192093849182 0.762549877166748\\
2.51192092895508 0.754036486148834\\
2.52192115783691 0.745150566101074\\
2.53192114830017 0.735996007919312\\
2.54192113876343 0.726669132709503\\
2.55192112922668 0.717261433601379\\
2.56192111968994 0.707854926586151\\
2.5719211101532 0.698522567749023\\
2.58192110061646 0.689328610897064\\
2.59192109107971 0.680329322814941\\
2.60192108154297 0.671574413776398\\
2.61192107200623 0.66310328245163\\
2.62192106246948 0.655068755149841\\
2.63192105293274 0.647618174552917\\
2.641921043396 0.640919923782349\\
2.65192103385925 0.636759400367737\\
2.66192102432251 0.636161267757416\\
2.67192101478577 0.639037311077118\\
2.68192100524902 0.644774317741394\\
2.69192099571228 0.652492821216583\\
2.70192098617554 0.66138631105423\\
2.71192097663879 0.671156525611877\\
2.72192096710205 0.681590616703033\\
2.73192095756531 0.692513167858124\\
2.74995803833008 0.712703168392181\\
2.76597833633423 0.733335316181183\\
2.77814722061157 0.748456358909607\\
2.79020619392395 0.761836171150208\\
2.80279803276062 0.774248242378235\\
2.81343126296997 0.78347510099411\\
2.82629728317261 0.792748868465424\\
2.8418653011322 0.800914645195007\\
2.86070275306702 0.807012021541595\\
2.87070274353027 0.809516072273254\\
2.88070273399353 0.812399804592133\\
2.89070272445679 0.815916955471039\\
2.90070271492004 0.819491028785706\\
2.9107027053833 0.82219135761261\\
2.92070269584656 0.823555707931519\\
2.93070268630981 0.823628425598145\\
2.94070267677307 0.822681427001953\\
2.95070266723633 0.821015298366547\\
2.96070265769959 0.818899154663086\\
2.97070264816284 0.816536068916321\\
2.9807026386261 0.814059555530548\\
2.99070262908936 0.811542391777039\\
3.00070261955261 0.808987438678741\\
3.01070261001587 0.806229531764984\\
3.02070260047913 0.802899062633514\\
3.03070259094238 0.798822820186615\\
3.04070281982422 0.793902933597565\\
3.05070281028748 0.788117468357086\\
3.06070280075073 0.781497478485107\\
3.07070279121399 0.774112224578857\\
3.08070278167725 0.766055643558502\\
3.0907027721405 0.757433950901031\\
3.10070276260376 0.748358130455017\\
3.11070275306702 0.738940536975861\\
3.12070274353027 0.729297876358032\\
3.13070273399353 0.719562590122223\\
3.14070272445679 0.70988392829895\\
3.15070271492004 0.70042610168457\\
3.1607027053833 0.691370069980621\\
3.17070269584656 0.68291175365448\\
3.18070268630981 0.675264537334442\\
3.19070267677307 0.668651342391968\\
3.20070266723633 0.664222478866577\\
3.21070265769959 0.662512719631195\\
3.22070264816284 0.663645923137665\\
3.2307026386261 0.667584538459778\\
3.24070262908936 0.674204111099243\\
3.25070261955261 0.683318674564362\\
3.26070261001587 0.694737732410431\\
3.27070260047913 0.708245992660522\\
3.28070259094238 0.723328649997711\\
3.29070281982422 0.739359438419342\\
3.30070281028748 0.755724430084229\\
3.32038044929504 0.783443987369537\\
3.3397479057312 0.801862239837646\\
3.35937190055847 0.812804460525513\\
3.37386155128479 0.817092716693878\\
3.38459610939026 0.818577587604523\\
3.39596796035767 0.818568587303162\\
3.40832281112671 0.816993534564972\\
3.42327213287354 0.813476026058197\\
3.44136095046997 0.809085369110107\\
3.45136094093323 0.807813167572021\\
3.46136093139648 0.807467699050903\\
3.47136092185974 0.807378172874451\\
3.481360912323 0.806684732437134\\
3.49136090278625 0.804951250553131\\
3.50136089324951 0.802177488803864\\
3.51136088371277 0.798493981361389\\
3.52136087417603 0.794176459312439\\
3.53136110305786 0.789161384105682\\
3.54136109352112 0.783513605594635\\
3.55136108398438 0.777254402637482\\
3.56136107444763 0.770410597324371\\
3.57136106491089 0.763030648231506\\
3.58136105537415 0.75519335269928\\
3.5913610458374 0.746980249881744\\
3.60136103630066 0.738459587097168\\
3.61136102676392 0.729691565036774\\
3.62136101722717 0.72072958946228\\
3.63136100769043 0.711642980575562\\
3.64136099815369 0.702532887458801\\
3.65136098861694 0.693526089191437\\
3.6613609790802 0.684766411781311\\
3.67136096954346 0.676409840583801\\
3.68136096000671 0.668611943721771\\
3.69136095046997 0.661519169807434\\
3.70136094093323 0.6552774310112\\
3.71136093139648 0.651764571666718\\
3.72136092185974 0.650922834873199\\
3.731360912323 0.652737319469452\\
3.74136090278625 0.657097995281219\\
3.75136089324951 0.663823068141937\\
3.76136088371277 0.672700762748718\\
3.77136087417603 0.683616518974304\\
3.78136110305786 0.696394681930542\\
3.79136109352112 0.710539758205414\\
3.81122899055481 0.740612089633942\\
3.83080196380615 0.769472599029541\\
3.85026788711548 0.791813433170319\\
3.86996960639954 0.806104421615601\\
3.88323998451233 0.812059164047241\\
3.89929723739624 0.815580487251282\\
3.9114396572113 0.816496014595032\\
3.92577981948853 0.816308498382568\\
3.94313144683838 0.816388368606567\\
3.96312928199768 0.817972362041473\\
3.97312927246094 0.818166613578796\\
3.98312926292419 0.817338883876801\\
3.99312925338745 0.815512001514435\\
4.00312948226929 0.812868058681488\\
4.01312923431396 0.809622764587402\\
4.0231294631958 0.805924236774445\\
4.03312921524048 0.801922619342804\\
4.04312944412231 0.797400236129761\\
4.05312919616699 0.792328238487244\\
4.06312942504883 0.786689460277557\\
4.07312917709351 0.780489146709442\\
4.08312940597534 0.773759484291077\\
4.09312915802002 0.766552627086639\\
4.10312938690186 0.758903980255127\\
4.11312913894653 0.75083327293396\\
4.12312936782837 0.742350339889526\\
4.13312911987305 0.733511328697205\\
4.14312934875488 0.724441051483154\\
4.15312910079956 0.715320289134979\\
4.1631293296814 0.706358075141907\\
4.17312908172607 0.697763442993164\\
4.18312931060791 0.689739644527435\\
4.19312953948975 0.682476758956909\\
4.20312929153442 0.676151871681213\\
4.21312952041626 0.672262012958527\\
4.22312927246094 0.671193778514862\\
4.23312950134277 0.67277979850769\\
4.24312925338745 0.676703035831451\\
4.25312948226929 0.682623744010925\\
4.26312923431396 0.690193653106689\\
4.2731294631958 0.699078261852264\\
4.28312921524048 0.709094703197479\\
4.29312944412231 0.720081329345703\\
4.30312919616699 0.731767892837524\\
4.31312942504883 0.743849396705627\\
4.32312917709351 0.756034851074219\\
4.33312940597534 0.768044412136078\\
4.34312915802002 0.779016733169556\\
4.35312938690186 0.788335084915161\\
4.36312913894653 0.795807898044586\\
4.37312936782837 0.801497578620911\\
4.38924884796143 0.807466745376587\\
4.40431070327759 0.809827208518982\\
4.41981267929077 0.80956894159317\\
4.43557548522949 0.807857990264893\\
4.4546480178833 0.806498110294342\\
4.46464824676514 0.806619703769684\\
4.47464799880981 0.806630849838257\\
4.48464822769165 0.805952847003937\\
4.49464845657349 0.804404735565186\\
4.50464820861816 0.802066564559937\\
4.5146484375 0.799104452133179\\
4.52464818954468 0.795631051063538\\
4.53464841842651 0.791801810264587\\
4.54464817047119 0.787368178367615\\
4.55464839935303 0.782315313816071\\
4.56464815139771 0.776647329330444\\
4.57464838027954 0.770392775535584\\
4.58464813232422 0.763600170612335\\
4.59464836120605 0.756330668926239\\
4.60464811325073 0.748667299747467\\
4.61464834213257 0.740674316883087\\
4.62464809417725 0.732382357120514\\
4.63464832305908 0.723809599876404\\
4.64464807510376 0.715000927448273\\
4.6546483039856 0.706070005893707\\
4.66464805603027 0.697206377983093\\
4.67464828491211 0.688659369945526\\
4.68464803695679 0.680716395378113\\
4.69464826583862 0.673687279224396\\
4.7046480178833 0.668332636356354\\
4.71464824676514 0.666353344917297\\
4.72464799880981 0.667799592018127\\
4.73464822769165 0.672410368919373\\
4.74464845657349 0.67981094121933\\
4.75464820861816 0.689598619937897\\
4.7646484375 0.701322555541992\\
4.77464818954468 0.714470207691193\\
4.78464841842651 0.728623747825623\\
4.79464817047119 0.743395566940308\\
4.81454849243164 0.772128939628601\\
4.83430004119873 0.793891966342926\\
4.85395812988281 0.807590425014496\\
4.8739070892334 0.814804553985596\\
4.8856954574585 0.816753149032593\\
4.89849042892456 0.816828906536102\\
4.91286373138428 0.814811944961548\\
4.92903614044189 0.810916543006897\\
4.94835424423218 0.807147085666656\\
4.95835447311401 0.806505858898163\\
4.96835422515869 0.806391656398773\\
4.97835445404053 0.805948376655579\\
4.98835420608521 0.804552257061005\\
5 0.801996648311615\\
};
\addlegendentry{$\dtmax{} = \expneg{1.0}{2}$ [s]};

\addplot [
color=black,
dotted
]
table[row sep=crcr]{
0 1\\
0.0109122171998024 1\\
0.020912216976285 1\\
0.0309122167527676 1\\
0.0409122183918953 1\\
0.0509122163057327 1\\
0.0609122179448605 1\\
0.0709122195839882 1\\
0.0809122174978256 1\\
0.0909122154116631 1\\
0.100912220776081 1\\
0.110912218689919 1\\
0.120912216603756 1\\
0.130912214517593 1\\
0.140912219882011 1\\
0.150912210345268 1\\
0.160912215709686 1\\
0.170912221074104 1\\
0.180912211537361 1\\
0.190912216901779 1\\
0.200912222266197 1\\
0.210912212729454 1\\
0.220912218093872 1\\
0.23091222345829 1\\
0.240912213921547 1\\
0.250912219285965 1\\
0.260912209749222 1\\
0.270912230014801 1\\
0.280912220478058 1\\
0.290912210941315 1\\
0.300912231206894 1\\
0.310912221670151 1\\
0.320912212133408 1\\
0.330912202596664 1\\
0.340912222862244 1\\
0.3509122133255 1\\
0.360912203788757 1\\
0.370912224054337 1\\
0.380912214517593 1\\
0.39091220498085 1\\
0.400912225246429 1\\
0.410912215709686 1\\
0.420912206172943 1\\
0.430912226438522 1\\
0.440912216901779 1\\
0.450912207365036 1\\
0.460912227630615 1\\
0.470912218093872 1\\
0.480912208557129 1\\
0.490912228822708 1\\
0.500912189483643 1\\
0.510912239551544 1\\
0.520912230014801 1\\
0.530912220478058 1\\
0.540912210941315 1\\
0.550912201404572 1\\
0.560912191867828 1\\
0.57091224193573 1\\
0.580912232398987 1\\
0.590912222862244 1\\
0.6009122133255 1\\
0.610912203788757 1\\
0.620912194252014 1\\
0.630912244319916 1\\
0.640912234783173 1\\
0.650912225246429 1\\
0.660912215709686 1\\
0.670912206172943 1\\
0.6809121966362 1\\
0.690912246704102 1\\
0.700912237167358 1\\
0.710912227630615 1\\
0.720912218093872 1\\
0.730912208557129 1\\
0.740912199020386 1\\
0.750912189483643 1\\
0.760912239551544 1\\
0.770912230014801 1\\
0.780912220478058 1\\
0.790912210941315 1\\
0.800912201404572 1\\
0.810912191867828 1\\
0.82091224193573 1\\
0.830912232398987 1\\
0.840912222862244 1\\
0.8509122133255 1\\
0.860912203788757 1\\
0.870912194252014 1\\
0.880912244319916 1\\
0.890912234783173 1\\
0.900912225246429 1\\
0.910912215709686 1\\
0.920912206172943 1\\
0.9309121966362 1\\
0.940912246704102 1\\
0.950912237167358 1\\
0.960912227630615 1\\
0.970912218093872 1\\
0.980912208557129 1\\
0.990912199020386 1\\
1.00091218948364 0.999999403953552\\
1.0109121799469 0.999952852725983\\
1.02091217041016 0.999807059764862\\
1.03091216087341 0.999626159667969\\
1.04091227054596 0.99938952922821\\
1.05091226100922 0.999084651470184\\
1.06091225147247 0.998710215091705\\
1.07091224193573 0.998264908790588\\
1.08091223239899 0.997747421264648\\
1.09091222286224 0.997156500816345\\
1.1009122133255 0.996491074562073\\
1.11091220378876 0.995749831199646\\
1.12091219425201 0.994931697845459\\
1.13091218471527 0.994035661220551\\
1.14091217517853 0.993060529232025\\
1.15091216564178 0.992005348205566\\
1.16091227531433 0.990869045257568\\
1.17091226577759 0.989650726318359\\
1.18091225624084 0.988349378108978\\
1.1909122467041 0.986964106559753\\
1.20091223716736 0.985493957996368\\
1.21091222763062 0.983938097953796\\
1.22091221809387 0.982295632362366\\
1.23091220855713 0.980565786361694\\
1.24091219902039 0.978749573230743\\
1.25091218948364 0.97686505317688\\
1.2609121799469 0.974924743175507\\
1.27091217041016 0.972929537296295\\
1.28091216087341 0.970877349376678\\
1.29091227054596 0.968765735626221\\
1.30091226100922 0.966592669487\\
1.31091225147247 0.964356780052185\\
1.32091224193573 0.96205723285675\\
1.33091223239899 0.959693431854248\\
1.34091222286224 0.957265079021454\\
1.3509122133255 0.954772055149078\\
1.36091220378876 0.952214300632477\\
1.37091219425201 0.949591755867004\\
1.38091218471527 0.946904540061951\\
1.39091217517853 0.94415271282196\\
1.40091216564178 0.941336274147034\\
1.41091227531433 0.938455402851105\\
1.42091226577759 0.935510158538818\\
1.43091225624084 0.932500541210175\\
1.4409122467041 0.929426670074463\\
1.45091223716736 0.926288545131683\\
1.46091222763062 0.92308634519577\\
1.47091221809387 0.919820010662079\\
1.48091220855713 0.916489601135254\\
1.49091219902039 0.913095116615295\\
1.50091218948364 0.909636676311493\\
1.5109121799469 0.906114220619202\\
1.52091217041016 0.902527809143066\\
1.53091216087341 0.898877441883087\\
1.54091227054596 0.895163059234619\\
1.55091226100922 0.891384720802307\\
1.56091225147247 0.887542426586151\\
1.57091224193573 0.883636116981506\\
1.58091223239899 0.879665791988373\\
1.59091222286224 0.87563145160675\\
1.6009122133255 0.871533095836639\\
1.61091220378876 0.86737072467804\\
1.62091219425201 0.863144159317017\\
1.63091218471527 0.858853578567505\\
1.64091217517853 0.854498744010925\\
1.65091216564178 0.850079834461212\\
1.66091227531433 0.845596611499786\\
1.67091226577759 0.841049134731293\\
1.68091225624084 0.836437404155731\\
1.6909122467041 0.831761300563812\\
1.70091223716736 0.827020823955536\\
1.71091222763062 0.822215855121613\\
1.72091221809387 0.817346453666687\\
1.73091220855713 0.81241250038147\\
1.74091219902039 0.807413935661316\\
1.75091218948364 0.802350699901581\\
1.7609121799469 0.797224819660187\\
1.77091217041016 0.792066514492035\\
1.78091216087341 0.786902129650116\\
1.79091227054596 0.781733572483063\\
1.80091226100922 0.77655440568924\\
1.81091225147247 0.771355986595154\\
1.82091224193573 0.766130089759827\\
1.83091223239899 0.760869443416595\\
1.84091222286224 0.755567908287048\\
1.8509122133255 0.75022029876709\\
1.86091220378876 0.744822382926941\\
1.87091219425201 0.73937064409256\\
1.88091218471527 0.733862102031708\\
1.89091217517853 0.728294372558594\\
1.90091216564178 0.722665250301361\\
1.91091227531433 0.716973125934601\\
1.92091226577759 0.711216390132904\\
1.93091225624084 0.705393612384796\\
1.9409122467041 0.699503660202026\\
1.95091223716736 0.693545401096344\\
1.96091222763062 0.687517642974854\\
1.97091221809387 0.681419491767883\\
1.98091220855713 0.675339877605438\\
1.99091219902039 0.671939432621002\\
2.00091218948364 0.672845304012299\\
2.0109121799469 0.67756325006485\\
2.02091217041016 0.684561610221863\\
2.03091216087341 0.692242205142975\\
2.04091215133667 0.69954389333725\\
2.05091214179993 0.70595246553421\\
2.06091213226318 0.711305320262909\\
2.07091212272644 0.715611636638641\\
2.0809121131897 0.718981444835663\\
2.09091210365295 0.721584022045136\\
2.10091233253479 0.723548352718353\\
2.11091232299805 0.72497832775116\\
2.1209123134613 0.72597324848175\\
2.13091230392456 0.726661384105682\\
2.14091229438782 0.72727245092392\\
2.15091228485107 0.728283703327179\\
2.16091227531433 0.730589926242828\\
2.17091226577759 0.735309839248657\\
2.18091225624084 0.742783904075623\\
2.1909122467041 0.751851081848145\\
2.20091223716736 0.76084440946579\\
2.21091222763062 0.768752753734589\\
2.22091221809387 0.775392532348633\\
2.23091220855713 0.780988156795502\\
2.24091219902039 0.785843372344971\\
2.25091218948364 0.79020357131958\\
2.2609121799469 0.794229090213776\\
2.27091217041016 0.79800546169281\\
2.28091216087341 0.801564812660217\\
2.29091215133667 0.804906666278839\\
2.30091214179993 0.808014094829559\\
2.31091213226318 0.810865163803101\\
2.32091212272644 0.813438892364502\\
2.3309121131897 0.815719187259674\\
2.34091210365295 0.817695617675781\\
2.35091233253479 0.819363832473755\\
2.36091232299805 0.82072502374649\\
2.3709123134613 0.821785032749176\\
2.38091230392456 0.822552978992462\\
2.39091229438782 0.823040843009949\\
2.40091228485107 0.823262214660645\\
2.41091227531433 0.823231875896454\\
2.42091226577759 0.822965145111084\\
2.43091225624084 0.822477757930756\\
2.4409122467041 0.821795642375946\\
2.45091223716736 0.820953845977783\\
2.46091222763062 0.819938063621521\\
2.47091221809387 0.81873881816864\\
2.48091220855713 0.817265868186951\\
2.49091219902039 0.815247058868408\\
2.50091218948364 0.812597692012787\\
2.5109121799469 0.80927437543869\\
2.52091217041016 0.805270433425903\\
2.53091216087341 0.800607085227966\\
2.54091215133667 0.795323967933655\\
2.55091214179993 0.789470911026001\\
2.56091213226318 0.783102810382843\\
2.57091212272644 0.776276230812073\\
2.5809121131897 0.769047498703003\\
2.59091210365295 0.761471211910248\\
2.60091233253479 0.753600120544434\\
2.61091232299805 0.745484530925751\\
2.6209123134613 0.737172305583954\\
2.63091230392456 0.72871059179306\\
2.64091229438782 0.720146477222443\\
2.65091228485107 0.711527168750763\\
2.66091227531433 0.702904999256134\\
2.67091226577759 0.694366931915283\\
2.68091225624084 0.686018884181976\\
2.6909122467041 0.677932322025299\\
2.70091223716736 0.670169293880463\\
2.71091222763062 0.662797391414642\\
2.72091221809387 0.655899703502655\\
2.73091220855713 0.64977103471756\\
2.74091219902039 0.646510541439056\\
2.75091218948364 0.647044837474823\\
2.7609121799469 0.651242554187775\\
2.77091217041016 0.658095061779022\\
2.78091216087341 0.66633677482605\\
2.79091215133667 0.674999475479126\\
2.80091214179993 0.683576941490173\\
2.81091213226318 0.691911041736603\\
2.82091212272644 0.699935555458069\\
2.8309121131897 0.707617700099945\\
2.84091210365295 0.714940965175629\\
2.85091233253479 0.721898794174194\\
2.86091232299805 0.728491544723511\\
2.8709123134613 0.734728813171387\\
2.88091230392456 0.740634500980377\\
2.89091229438782 0.746261894702911\\
2.90091228485107 0.751728773117065\\
2.91091227531433 0.757295548915863\\
2.92091226577759 0.763482868671417\\
2.93091225624084 0.771054744720459\\
2.9409122467041 0.780495941638947\\
2.95091223716736 0.791216492652893\\
2.96091222763062 0.801689028739929\\
2.97091221809387 0.81060254573822\\
2.98091220855713 0.817466020584106\\
2.99091219902039 0.822388470172882\\
3.00091218948364 0.825709521770477\\
3.0109121799469 0.827783286571503\\
3.02091217041016 0.828899741172791\\
3.03091216087341 0.829273998737335\\
3.04091215133667 0.829058825969696\\
3.05091214179993 0.828367590904236\\
3.06091213226318 0.827286183834076\\
3.07091212272644 0.825883269309998\\
3.0809121131897 0.824209868907928\\
3.09091210365295 0.822309374809265\\
3.10091233253479 0.820216357707977\\
3.11091232299805 0.817870676517487\\
3.1209123134613 0.814916551113129\\
3.13091230392456 0.811188697814941\\
3.14091229438782 0.806582510471344\\
3.15091228485107 0.801053822040558\\
3.16091227531433 0.794615924358368\\
3.17091226577759 0.78732568025589\\
3.18091225624084 0.779271304607391\\
3.1909122467041 0.77055823802948\\
3.20091223716736 0.761298298835754\\
3.21091222763062 0.751603901386261\\
3.22091221809387 0.741589844226837\\
3.23091220855713 0.731367170810699\\
3.24091219902039 0.721043705940247\\
3.25091218948364 0.710732221603394\\
3.2609121799469 0.700554370880127\\
3.27091217041016 0.690639972686768\\
3.28091216087341 0.681127607822418\\
3.29091215133667 0.672172605991364\\
3.30091214179993 0.663939714431763\\
3.31091213226318 0.656597554683685\\
3.32091212272644 0.65067994594574\\
3.3309121131897 0.647422075271606\\
3.34091210365295 0.647075235843658\\
3.35091233253479 0.649716019630432\\
3.36091232299805 0.655319929122925\\
3.3709123134613 0.663762748241425\\
3.38091230392456 0.674860596656799\\
3.39091229438782 0.688429832458496\\
3.40091228485107 0.703957259654999\\
3.41091227531433 0.720547795295715\\
3.42091226577759 0.737484812736511\\
3.43091225624084 0.754151344299316\\
3.4409122467041 0.769686877727509\\
3.45091223716736 0.78318327665329\\
3.46091222763062 0.794339239597321\\
3.47091221809387 0.803230047225952\\
3.48091220855713 0.810102581977844\\
3.49091219902039 0.815205156803131\\
3.50091218948364 0.81881046295166\\
3.5109121799469 0.821267664432526\\
3.52091217041016 0.823071300983429\\
3.53091216087341 0.824860811233521\\
3.54091215133667 0.827122986316681\\
3.55091214179993 0.829797685146332\\
3.56091213226318 0.832309544086456\\
3.57091212272644 0.834004402160645\\
3.5809121131897 0.834514856338501\\
3.59091210365295 0.83380651473999\\
3.60091233253479 0.83205384016037\\
3.61091232299805 0.829488635063171\\
3.6209123134613 0.826323986053467\\
3.63091230392456 0.822728455066681\\
3.64091229438782 0.818664371967316\\
3.65091228485107 0.8139768242836\\
3.66091227531433 0.808642268180847\\
3.67091226577759 0.802645087242126\\
3.68091225624084 0.795998632907867\\
3.6909122467041 0.788745880126953\\
3.70091223716736 0.780953586101532\\
3.71091222763062 0.772709548473358\\
3.72091221809387 0.764100432395935\\
3.73091220855713 0.755201518535614\\
3.74091219902039 0.74607390165329\\
3.75091218948364 0.736763179302216\\
3.7609121799469 0.72730964422226\\
3.77091217041016 0.717767834663391\\
3.78091216087341 0.708213984966278\\
3.79091215133667 0.698740541934967\\
3.80091214179993 0.689447343349457\\
3.81091213226318 0.680436968803406\\
3.82091212272644 0.671812653541565\\
3.8309121131897 0.663679361343384\\
3.84091210365295 0.656140565872192\\
3.85091233253479 0.649300396442413\\
3.86091232299805 0.644113719463348\\
3.8709123134613 0.641621291637421\\
3.88091230392456 0.641854405403137\\
3.89091229438782 0.644734978675842\\
3.90091228485107 0.650105714797974\\
3.91091227531433 0.657790899276733\\
3.92091226577759 0.667745530605316\\
3.93091225624084 0.679926037788391\\
3.9409122467041 0.693589627742767\\
3.95091223716736 0.708071172237396\\
3.96091222763062 0.722869753837585\\
3.97091221809387 0.737572908401489\\
3.98091220855713 0.751859486103058\\
3.99091219902039 0.765444397926331\\
4.00091218948364 0.777760922908783\\
4.01091241836548 0.78830361366272\\
4.02091217041016 0.796930611133575\\
4.03091239929199 0.803745746612549\\
4.04091215133667 0.80900377035141\\
4.05091238021851 0.813113152980804\\
4.06091213226318 0.816679358482361\\
4.07091236114502 0.820374727249146\\
4.0809121131897 0.824493527412415\\
4.09091234207153 0.828671276569366\\
4.10091209411621 0.832186818122864\\
4.11091232299805 0.834482789039612\\
4.12091207504272 0.835369169712067\\
4.13091230392456 0.834928333759308\\
4.14091205596924 0.83338063955307\\
4.15091228485107 0.830990672111511\\
4.16091203689575 0.827979862689972\\
4.17091226577759 0.824512362480164\\
4.18091201782227 0.820676684379578\\
4.1909122467041 0.816316545009613\\
4.20091199874878 0.811381101608276\\
4.21091222763062 0.805849194526672\\
4.22091197967529 0.799715459346771\\
4.23091220855713 0.792998731136322\\
4.24091243743896 0.785741984844208\\
4.25091218948364 0.778014183044434\\
4.26091241836548 0.769879996776581\\
4.27091217041016 0.761378347873688\\
4.28091239929199 0.752533137798309\\
4.29091215133667 0.743380427360535\\
4.30091238021851 0.733992457389832\\
4.31091213226318 0.724480271339417\\
4.32091236114502 0.714981496334076\\
4.3309121131897 0.70564740896225\\
4.34091234207153 0.696629405021667\\
4.35091209411621 0.68807178735733\\
4.36091232299805 0.680115759372711\\
4.37091207504272 0.672895967960358\\
4.38091230392456 0.666680812835693\\
4.39091205596924 0.662864446640015\\
4.40091228485107 0.66170346736908\\
4.41091203689575 0.663016438484192\\
4.42091226577759 0.666563928127289\\
4.43091201782227 0.672081470489502\\
4.4409122467041 0.679303050041199\\
4.45091199874878 0.688007295131683\\
4.46091222763062 0.698078036308289\\
4.47091197967529 0.709319770336151\\
4.48091220855713 0.72116231918335\\
4.49091243743896 0.733215630054474\\
4.50091218948364 0.745215177536011\\
4.51091241836548 0.756957590579987\\
4.52091217041016 0.768231749534607\\
4.53091239929199 0.77852338552475\\
4.54091215133667 0.787356317043304\\
4.55091238021851 0.794566750526428\\
4.56091213226318 0.800205826759338\\
4.57091236114502 0.804476082324982\\
4.5809121131897 0.80775111913681\\
4.59091234207153 0.810625851154327\\
4.60091209411621 0.81378048658371\\
4.61091232299805 0.81751012802124\\
4.62091207504272 0.821420311927795\\
4.63091230392456 0.824738562107086\\
4.64091205596924 0.826858282089233\\
4.65091228485107 0.82756108045578\\
4.66091203689575 0.826919853687286\\
4.67091226577759 0.825155735015869\\
4.68091201782227 0.822548687458038\\
4.6909122467041 0.819331645965576\\
4.70091199874878 0.815654814243317\\
4.71091222763062 0.811395406723022\\
4.72091197967529 0.806496143341064\\
4.73091220855713 0.800960004329681\\
4.74091243743896 0.79479992389679\\
4.75091218948364 0.788051009178162\\
4.76091241836548 0.780767917633057\\
4.77091217041016 0.773018836975098\\
4.78091239929199 0.764889299869537\\
4.79091215133667 0.756460309028625\\
4.80091238021851 0.747780621051788\\
4.81091213226318 0.73886501789093\\
4.82091236114502 0.729726612567902\\
4.8309121131897 0.720416605472565\\
4.84091234207153 0.711039304733276\\
4.85091209411621 0.701741516590118\\
4.86091232299805 0.692694783210754\\
4.87091207504272 0.684087634086609\\
4.88091230392456 0.676126956939697\\
4.89091205596924 0.669027149677277\\
4.90091228485107 0.663430571556091\\
4.91091203689575 0.661058127880096\\
4.92091226577759 0.661990106105804\\
4.93091201782227 0.665974736213684\\
4.9409122467041 0.672668695449829\\
4.95091199874878 0.681691586971283\\
4.96091222763062 0.692631900310516\\
4.97091197967529 0.704998373985291\\
4.98091220855713 0.718379437923431\\
4.99091243743896 0.732422590255737\\
5 0.745228171348572\\
};
\addlegendentry{$\dtmax{} = \expneg{1.0}{3}$ [s]};

\addplot [
color=black,
dashed
]
table[row sep=crcr]{
0 1\\
0.0100556658580899 1\\
0.0200556665658951 1\\
0.0300556663423777 1\\
0.0400556661188602 1\\
0.050055667757988 1\\
0.0600556656718254 1\\
0.0700556635856628 1\\
0.0800556689500809 1\\
0.0900556668639183 1\\
0.100055664777756 1\\
0.110055662691593 1\\
0.120055668056011 1\\
0.130055665969849 1\\
0.140055671334267 1\\
0.150055661797524 1\\
0.160055667161942 1\\
0.17005567252636 1\\
0.180055662989616 1\\
0.190055668354034 1\\
0.200055658817291 1\\
0.210055664181709 1\\
0.220055669546127 1\\
0.230055660009384 1\\
0.240055665373802 1\\
0.25005567073822 1\\
0.260055661201477 1\\
0.270055651664734 1\\
0.280055671930313 1\\
0.29005566239357 1\\
0.300055652856827 1\\
0.310055673122406 1\\
0.320055663585663 1\\
0.33005565404892 1\\
0.340055674314499 1\\
0.350055664777756 1\\
0.360055655241013 1\\
0.370055675506592 1\\
0.380055665969849 1\\
0.390055656433105 1\\
0.400055676698685 1\\
0.410055667161942 1\\
0.420055657625198 1\\
0.430055677890778 1\\
0.440055668354034 1\\
0.450055658817291 1\\
0.46005567908287 1\\
0.470055669546127 1\\
0.480055660009384 1\\
0.490055680274963 1\\
0.50005567073822 1\\
0.510055661201477 1\\
0.520055651664734 1\\
0.530055642127991 1\\
0.540055692195892 1\\
0.550055682659149 1\\
0.560055673122406 1\\
0.570055663585663 1\\
0.58005565404892 1\\
0.590055644512177 1\\
0.600055694580078 1\\
0.610055685043335 1\\
0.620055675506592 1\\
0.630055665969849 1\\
0.640055656433105 1\\
0.650055646896362 1\\
0.660055637359619 1\\
0.670055687427521 1\\
0.680055677890778 1\\
0.690055668354034 1\\
0.700055658817291 1\\
0.710055649280548 1\\
0.720055639743805 1\\
0.730055689811707 1\\
0.740055680274963 1\\
0.75005567073822 1\\
0.760055661201477 1\\
0.770055651664734 1\\
0.780055642127991 1\\
0.790055692195892 1\\
0.800055682659149 1\\
0.810055673122406 1\\
0.820055663585663 1\\
0.83005565404892 1\\
0.840055644512177 1\\
0.850055694580078 1\\
0.860055685043335 1\\
0.870055675506592 1\\
0.880055665969849 1\\
0.890055656433105 1\\
0.900055646896362 1\\
0.910055637359619 1\\
0.920055687427521 1\\
0.930055677890778 1\\
0.940055668354034 1\\
0.950055658817291 1\\
0.960055649280548 1\\
0.970055639743805 1\\
0.980055689811707 1\\
0.990055680274963 1\\
1.00005567073822 1\\
1.01005566120148 0.999947726726532\\
1.02005565166473 0.999781370162964\\
1.03005564212799 0.999619662761688\\
1.04005563259125 0.999391496181488\\
1.0500556230545 0.999095261096954\\
1.06005561351776 0.998729705810547\\
1.07005572319031 0.998293340206146\\
1.08005571365356 0.997784972190857\\
1.09005570411682 0.997203409671783\\
1.10005569458008 0.996547341346741\\
1.11005568504334 0.995815694332123\\
1.12005567550659 0.995007336139679\\
1.13005566596985 0.994121074676514\\
1.14005565643311 0.993155956268311\\
1.15005564689636 0.992110908031464\\
1.16005563735962 0.990984857082367\\
1.17005562782288 0.989776909351349\\
1.18005561828613 0.988485991954803\\
1.19005560874939 0.987111330032349\\
1.20005571842194 0.985651910305023\\
1.21005570888519 0.9841068983078\\
1.22005569934845 0.982475340366364\\
1.23005568981171 0.980756521224976\\
1.24005568027496 0.97895073890686\\
1.25005567073822 0.977074503898621\\
1.26005566120148 0.975142180919647\\
1.27005565166473 0.97315526008606\\
1.28005564212799 0.971111655235291\\
1.29005563259125 0.969008803367615\\
1.3000556230545 0.96684467792511\\
1.31005561351776 0.964617788791656\\
1.32005572319031 0.962327182292938\\
1.33005571365356 0.959972321987152\\
1.34005570411682 0.957552969455719\\
1.35005569458008 0.955068826675415\\
1.36005568504334 0.95251989364624\\
1.37005567550659 0.949906170368195\\
1.38005566596985 0.947227776050568\\
1.39005565643311 0.944484651088715\\
1.40005564689636 0.94167697429657\\
1.41005563735962 0.938804805278778\\
1.42005562782288 0.935868203639984\\
1.43005561828613 0.932867228984833\\
1.44005560874939 0.929802000522614\\
1.45005571842194 0.926672518253326\\
1.46005570888519 0.923478901386261\\
1.47005569934845 0.920221149921417\\
1.48005568981171 0.91689932346344\\
1.49005568027496 0.913513422012329\\
1.50005567073822 0.910063564777374\\
1.51005566120148 0.906549632549286\\
1.52005565166473 0.902971744537354\\
1.53005564212799 0.899329900741577\\
1.54005563259125 0.895624041557312\\
1.5500556230545 0.891854226589203\\
1.56005561351776 0.88802045583725\\
1.57005572319031 0.884122669696808\\
1.58005571365356 0.880160868167877\\
1.59005570411682 0.876135051250458\\
1.60005569458008 0.872045159339905\\
1.61005568504334 0.867891252040863\\
1.62005567550659 0.863673269748688\\
1.63005566596985 0.859391152858734\\
1.64005565643311 0.855044841766357\\
1.65005564689636 0.850634396076202\\
1.66005563735962 0.846159696578979\\
1.67005562782288 0.841620743274689\\
1.68005561828613 0.837017476558685\\
1.69005560874939 0.832349836826324\\
1.70005571842194 0.827617883682251\\
1.71005570888519 0.822821438312531\\
1.72005569934845 0.817960500717163\\
1.73005568981171 0.813035070896149\\
1.74005568027496 0.808045029640198\\
1.75005567073822 0.802990317344666\\
1.76005566120148 0.797871887683868\\
1.77005565166473 0.792716145515442\\
1.78005564212799 0.787553131580353\\
1.79005563259125 0.782386720180511\\
1.8000556230545 0.777210831642151\\
1.81005561351776 0.772016942501068\\
1.82005572319031 0.76679652929306\\
1.83005571365356 0.761542141437531\\
1.84005570411682 0.756247460842133\\
1.85005569458008 0.750907182693481\\
1.86005568504334 0.745516955852509\\
1.87005567550659 0.740073204040527\\
1.88005566596985 0.73457282781601\\
1.89005565643311 0.729013383388519\\
1.90005564689636 0.72339278459549\\
1.91005563735962 0.717709183692932\\
1.92005562782288 0.711961030960083\\
1.93005561828613 0.706147015094757\\
1.94005560874939 0.700265824794769\\
1.95005571842194 0.694316387176514\\
1.96005570888519 0.68829756975174\\
1.97005569934845 0.682208359241486\\
1.98005568981171 0.676063299179077\\
1.99005568027496 0.67209005355835\\
2.00005555152893 0.672386348247528\\
2.01005578041077 0.676676213741302\\
2.02005577087402 0.68353408575058\\
2.03005576133728 0.691292345523834\\
2.04005575180054 0.698781788349152\\
2.05005574226379 0.705409705638886\\
2.06005573272705 0.710973799228668\\
2.07005572319031 0.715468525886536\\
2.08005571365356 0.718993604183197\\
2.09005570411682 0.721721053123474\\
2.10005569458008 0.723787128925323\\
2.11005568504334 0.725301206111908\\
2.12005567550659 0.726367950439453\\
2.13005566596985 0.727122843265533\\
2.14005565643311 0.727808713912964\\
2.15005564689636 0.728926301002502\\
2.16005563735962 0.731399357318878\\
2.17005562782288 0.73634797334671\\
2.18005561828613 0.744077146053314\\
2.19005560874939 0.753436326980591\\
2.20005559921265 0.762768507003784\\
2.2100555896759 0.771009683609009\\
2.22005558013916 0.7779141664505\\
2.23005557060242 0.783676147460938\\
2.24005556106567 0.788599252700806\\
2.25005555152893 0.792941212654114\\
2.26005578041077 0.796880662441254\\
2.27005577087402 0.800520837306976\\
2.28005576133728 0.80391001701355\\
2.29005575180054 0.807060956954956\\
2.30005574226379 0.809967458248138\\
2.31005573272705 0.812615513801575\\
2.32005572319031 0.814990222454071\\
2.33005571365356 0.817079603672028\\
2.34005570411682 0.818875849246979\\
2.35005569458008 0.820376336574554\\
2.36005568504334 0.821582853794098\\
2.37005567550659 0.822501242160797\\
2.38005566596985 0.823140442371368\\
2.39005565643311 0.823511838912964\\
2.40005564689636 0.82362824678421\\
2.41005563735962 0.823503613471985\\
2.42005562782288 0.823152422904968\\
2.43005561828613 0.822600305080414\\
2.44005560874939 0.821887075901031\\
2.45005559921265 0.820997178554535\\
2.4600555896759 0.819917798042297\\
2.47005558013916 0.818645119667053\\
2.48005557060242 0.817046284675598\\
2.49005556106567 0.814895510673523\\
2.50005555152893 0.812122464179993\\
2.51005578041077 0.80868798494339\\
2.52005577087402 0.804587244987488\\
2.53005576133728 0.799841463565826\\
2.54005575180054 0.794488906860352\\
2.55005574226379 0.788577437400818\\
2.56005573272705 0.782159924507141\\
2.57005572319031 0.775290787220001\\
2.58005571365356 0.768024504184723\\
2.59005570411682 0.760414183139801\\
2.60005569458008 0.752511143684387\\
2.61005568504334 0.744364559650421\\
2.62005567550659 0.736021995544434\\
2.63005566596985 0.727531373500824\\
2.64005565643311 0.71893972158432\\
2.65005564689636 0.710294961929321\\
2.66005563735962 0.701651096343994\\
2.67005562782288 0.693099558353424\\
2.68005561828613 0.684744715690613\\
2.69005560874939 0.676657378673553\\
2.70005559921265 0.668899893760681\\
2.7100555896759 0.661541342735291\\
2.72005558013916 0.654667913913727\\
2.73005557060242 0.648640632629395\\
2.74005556106567 0.645641207695007\\
2.75005555152893 0.646485567092896\\
2.76005578041077 0.650976717472076\\
2.77005577087402 0.658193469047546\\
2.78005576133728 0.66678649187088\\
2.79005575180054 0.675744950771332\\
2.80005574226379 0.684580028057098\\
2.81005573272705 0.693134725093842\\
2.82005572319031 0.701343536376953\\
2.83005571365356 0.709175288677216\\
2.84005570411682 0.716615855693817\\
2.85005569458008 0.72366189956665\\
2.86005568504334 0.7303187251091\\
2.87005567550659 0.736603736877441\\
2.88005566596985 0.742558658123016\\
2.89005565643311 0.748276948928833\\
2.90005564689636 0.753968060016632\\
2.91005563735962 0.760070383548737\\
2.92005562782288 0.767307102680206\\
2.93005561828613 0.776335954666138\\
2.94005560874939 0.786964654922485\\
2.95005559921265 0.797959506511688\\
2.9600555896759 0.807880282402039\\
2.97005558013916 0.81595903635025\\
2.98005557060242 0.822075843811035\\
2.99005556106567 0.826438426971436\\
3.00005555152893 0.829345643520355\\
3.01005578041077 0.831075966358185\\
3.02005577087402 0.831853926181793\\
3.03005576133728 0.831849098205566\\
3.04005575180054 0.831187069416046\\
3.05005574226379 0.829966425895691\\
3.06005573272705 0.828271448612213\\
3.07005572319031 0.826188981533051\\
3.08005571365356 0.823799312114716\\
3.09005570411682 0.82111781835556\\
3.10005569458008 0.817826747894287\\
3.11005568504334 0.813765704631805\\
3.12005567550659 0.808844029903412\\
3.13005566596985 0.803015470504761\\
3.14005565643311 0.796288788318634\\
3.15005564689636 0.788721442222595\\
3.16005563735962 0.780405402183533\\
3.17005562782288 0.771453619003296\\
3.18005561828613 0.761985778808594\\
3.19005560874939 0.752123653888702\\
3.20005559921265 0.741985499858856\\
3.2100555896759 0.731679022312164\\
3.22005558013916 0.721302568912506\\
3.23005557060242 0.710951209068298\\
3.24005556106567 0.700726389884949\\
3.25005555152893 0.690740168094635\\
3.26005578041077 0.681114971637726\\
3.27005577087402 0.671987593173981\\
3.28005576133728 0.66350930929184\\
3.29005575180054 0.655835926532745\\
3.30005574226379 0.649209558963776\\
3.31005573272705 0.644904553890228\\
3.32005572319031 0.643498599529266\\
3.33005571365356 0.64510852098465\\
3.34005570411682 0.649744868278503\\
3.35005569458008 0.657317638397217\\
3.36005568504334 0.667640447616577\\
3.37005567550659 0.680481612682343\\
3.38005566596985 0.695534467697144\\
3.39005565643311 0.712074100971222\\
3.40005564689636 0.729236483573914\\
3.41005563735962 0.746363162994385\\
3.42005562782288 0.762783706188202\\
3.43005561828613 0.777498185634613\\
3.44005560874939 0.789928197860718\\
3.45005559921265 0.800009071826935\\
3.4600555896759 0.807941973209381\\
3.47005558013916 0.813996732234955\\
3.48005557060242 0.818458497524261\\
3.49005556106567 0.821704208850861\\
3.50005555152893 0.824261784553528\\
3.51005578041077 0.826777279376984\\
3.52005577087402 0.829691767692566\\
3.53005576133728 0.832885086536407\\
3.54005575180054 0.835778415203094\\
3.55005574226379 0.837765455245972\\
3.56005573272705 0.838518559932709\\
3.57005572319031 0.83800995349884\\
3.58005571365356 0.836407482624054\\
3.59005570411682 0.833948254585266\\
3.60005569458008 0.830848813056946\\
3.61005568504334 0.827280640602112\\
3.62005567550659 0.823312878608704\\
3.63005566596985 0.818763494491577\\
3.64005565643311 0.813595592975616\\
3.65005564689636 0.807788193225861\\
3.66005563735962 0.801341354846954\\
3.67005562782288 0.794283509254456\\
3.68005561828613 0.786669313907623\\
3.69005560874939 0.778579294681549\\
3.70005559921265 0.770098984241486\\
3.7100555896759 0.761303424835205\\
3.72005558013916 0.752253770828247\\
3.73005557060242 0.742995738983154\\
3.74005556106567 0.733568131923676\\
3.75005555152893 0.724023342132568\\
3.76005578041077 0.714435636997223\\
3.77005577087402 0.704895675182343\\
3.78005576133728 0.695501089096069\\
3.79005575180054 0.686350643634796\\
3.80005574226379 0.677541851997375\\
3.81005573272705 0.66917222738266\\
3.82005572319031 0.661339938640594\\
3.83005571365356 0.654139757156372\\
3.84005570411682 0.647838056087494\\
3.85005569458008 0.643849015235901\\
3.86005568504334 0.642520546913147\\
3.87005567550659 0.643815875053406\\
3.88005566596985 0.647614002227783\\
3.89005565643311 0.653734922409058\\
3.90005564689636 0.662006735801697\\
3.91005563735962 0.672401368618011\\
3.92005562782288 0.684789538383484\\
3.93005561828613 0.698384404182434\\
3.94005560874939 0.712630689144135\\
3.95005559921265 0.727082908153534\\
3.9600555896759 0.741376936435699\\
3.97005558013916 0.755227625370026\\
3.98005557060242 0.768339514732361\\
3.99005556106567 0.780121266841888\\
4.00005578994751 0.790145397186279\\
4.01005554199219 0.798339188098907\\
4.02005577087402 0.804884254932404\\
4.0300555229187 0.810170590877533\\
4.04005575180054 0.814807116985321\\
4.05005550384521 0.819474101066589\\
4.06005573272705 0.8244708776474\\
4.07005548477173 0.82944130897522\\
4.08005571365356 0.833689272403717\\
4.09005546569824 0.836669743061066\\
4.10005569458008 0.838184356689453\\
4.11005544662476 0.838298380374908\\
4.12005567550659 0.837203443050385\\
4.13005590438843 0.835140347480774\\
4.14005565643311 0.832354247570038\\
4.15005588531494 0.829033613204956\\
4.16005563735962 0.825311899185181\\
4.17005586624146 0.821146011352539\\
4.18005561828613 0.816400051116943\\
4.19005584716797 0.811061263084412\\
4.20005559921265 0.80511885881424\\
4.21005582809448 0.798581898212433\\
4.22005558013916 0.79148256778717\\
4.230055809021 0.783878564834595\\
4.24005556106567 0.775846481323242\\
4.25005578994751 0.767446100711823\\
4.26005554199219 0.758713245391846\\
4.27005577087402 0.749672114849091\\
4.2800555229187 0.740365564823151\\
4.29005575180054 0.730874121189117\\
4.30005550384521 0.721313059329987\\
4.31005573272705 0.711820244789124\\
4.32005548477173 0.702544331550598\\
4.33005571365356 0.693632245063782\\
4.34005546569824 0.685223162174225\\
4.35005569458008 0.677453458309174\\
4.36005544662476 0.670450210571289\\
4.37005567550659 0.664719581604004\\
4.38005590438843 0.661634206771851\\
4.39005565643311 0.661201000213623\\
4.40005588531494 0.663207650184631\\
4.41005563735962 0.667396366596222\\
4.42005586624146 0.673490226268768\\
4.43005561828613 0.681215643882751\\
4.44005584716797 0.690338432788849\\
4.45005559921265 0.700692176818848\\
4.46005582809448 0.712074756622314\\
4.47005558013916 0.723930776119232\\
4.480055809021 0.73590487241745\\
4.49005556106567 0.747757852077484\\
4.50005578994751 0.759305477142334\\
4.51005554199219 0.770326733589172\\
4.52005577087402 0.780300498008728\\
4.5300555229187 0.788813829421997\\
4.54005575180054 0.795752704143524\\
4.55005550384521 0.801227271556854\\
4.56005573272705 0.80554586648941\\
4.57005548477173 0.809248089790344\\
4.58005571365356 0.813024342060089\\
4.59005546569824 0.81730192899704\\
4.60005569458008 0.821843981742859\\
4.61005544662476 0.825940132141113\\
4.62005567550659 0.828926265239716\\
4.63005590438843 0.830485224723816\\
4.64005565643311 0.83061558008194\\
4.65005588531494 0.829485952854156\\
4.66005563735962 0.827337563037872\\
4.67005586624146 0.824433743953705\\
4.68005561828613 0.820978105068207\\
4.69005584716797 0.817043900489807\\
4.70005559921265 0.812465727329254\\
4.71005582809448 0.807241320610046\\
4.72005558013916 0.801378130912781\\
4.730055809021 0.794898331165314\\
4.74005556106567 0.787845134735107\\
4.75005578994751 0.780279695987701\\
4.76005554199219 0.772278010845184\\
4.77005577087402 0.763931691646576\\
4.7800555229187 0.755318582057953\\
4.79005575180054 0.746475577354431\\
4.80005550384521 0.737411618232727\\
4.81005573272705 0.728146433830261\\
4.82005548477173 0.718743741512299\\
4.83005571365356 0.709317624568939\\
4.84005546569824 0.700017869472504\\
4.85005569458008 0.691015005111694\\
4.86005544662476 0.682494878768921\\
4.87005567550659 0.674660682678223\\
4.88005590438843 0.667719841003418\\
4.89005565643311 0.662560164928436\\
4.90005588531494 0.66076785326004\\
4.91005563735962 0.662286579608917\\
4.92005586624146 0.666826069355011\\
4.93005561828613 0.67401248216629\\
4.94005584716797 0.683441042900085\\
4.95005559921265 0.694668292999268\\
4.96005582809448 0.707163274288177\\
4.97005558013916 0.720495522022247\\
4.980055809021 0.734345257282257\\
4.99005556106567 0.748146116733551\\
5 0.761430323123932\\
};
\addlegendentry{$\dtmax{} = \expneg{1.0}{4}$ [s]};

\addplot [
color=black,
solid
]
table[row sep=crcr]{
0 1\\
0.0100024435669184 1\\
0.020002443343401 1\\
0.0300024431198835 1\\
0.0400024428963661 1\\
0.0500024445354939 1\\
0.0600024424493313 1\\
0.070002444088459 1\\
0.0800024420022964 1\\
0.0900024399161339 1\\
0.100002445280552 1\\
0.110002443194389 1\\
0.120002441108227 1\\
0.130002439022064 1\\
0.140002444386482 1\\
0.1500024497509 1\\
0.160002440214157 1\\
0.170002445578575 1\\
0.180002436041832 1\\
0.19000244140625 1\\
0.200002446770668 1\\
0.210002437233925 1\\
0.220002442598343 1\\
0.230002447962761 1\\
0.240002438426018 1\\
0.250002443790436 1\\
0.260002434253693 1\\
0.270002454519272 1\\
0.280002444982529 1\\
0.290002435445786 1\\
0.300002455711365 1\\
0.310002446174622 1\\
0.320002436637878 1\\
0.330002456903458 1\\
0.340002447366714 1\\
0.350002437829971 1\\
0.360002458095551 1\\
0.370002448558807 1\\
0.380002439022064 1\\
0.390002429485321 1\\
0.4000024497509 1\\
0.410002440214157 1\\
0.420002430677414 1\\
0.430002450942993 1\\
0.44000244140625 1\\
0.450002431869507 1\\
0.460002452135086 1\\
0.470002442598343 1\\
0.4800024330616 1\\
0.490002453327179 1\\
0.500002443790436 1\\
0.510002434253693 1\\
0.520002424716949 1\\
0.530002415180206 1\\
0.540002465248108 1\\
0.550002455711365 1\\
0.560002446174622 1\\
0.570002436637878 1\\
0.580002427101135 1\\
0.590002417564392 1\\
0.600002467632294 1\\
0.610002458095551 1\\
0.620002448558807 1\\
0.630002439022064 1\\
0.640002429485321 1\\
0.650002419948578 1\\
0.660002470016479 1\\
0.670002460479736 1\\
0.680002450942993 1\\
0.69000244140625 1\\
0.700002431869507 1\\
0.710002422332764 1\\
0.720002472400665 1\\
0.730002462863922 1\\
0.740002453327179 1\\
0.750002443790436 1\\
0.760002434253693 1\\
0.770002424716949 1\\
0.780002415180206 1\\
0.790002465248108 1\\
0.800002455711365 1\\
0.810002446174622 1\\
0.820002436637878 1\\
0.830002427101135 1\\
0.840002417564392 1\\
0.850002467632294 1\\
0.860002458095551 1\\
0.870002448558807 1\\
0.880002439022064 1\\
0.890002429485321 1\\
0.900002419948578 1\\
0.910002470016479 1\\
0.920002460479736 1\\
0.930002450942993 1\\
0.94000244140625 1\\
0.950002431869507 1\\
0.960002422332764 1\\
0.970002472400665 1\\
0.980002462863922 1\\
0.990002453327179 1\\
1.00000238418579 1\\
1.01000249385834 0.999940812587738\\
1.02000248432159 0.99977570772171\\
1.03000247478485 0.999614715576172\\
1.04000246524811 0.999387145042419\\
1.05000245571136 0.999091565608978\\
1.06000244617462 0.998726606369019\\
1.07000243663788 0.99829089641571\\
1.08000242710114 0.997783184051514\\
1.09000241756439 0.997202277183533\\
1.10000240802765 0.996546924114227\\
1.11000239849091 0.995815932750702\\
1.12000238895416 0.995008230209351\\
1.13000249862671 0.994122743606567\\
1.14000248908997 0.993158280849457\\
1.15000247955322 0.992113947868347\\
1.16000247001648 0.990988671779633\\
1.17000246047974 0.989781439304352\\
1.18000245094299 0.988491296768188\\
1.19000244140625 0.987117409706116\\
1.20000243186951 0.985658764839172\\
1.21000242233276 0.984114527702332\\
1.22000241279602 0.982483804225922\\
1.23000240325928 0.980765759944916\\
1.24000239372253 0.978960752487183\\
1.25000238418579 0.97708523273468\\
1.26000249385834 0.975153565406799\\
1.27000248432159 0.973167359828949\\
1.28000247478485 0.971124470233917\\
1.29000246524811 0.969022333621979\\
1.30000245571136 0.966858863830566\\
1.31000244617462 0.96463268995285\\
1.32000243663788 0.962342798709869\\
1.33000242710114 0.959988653659821\\
1.34000241756439 0.95756995677948\\
1.35000240802765 0.955086469650269\\
1.36000239849091 0.952538251876831\\
1.37000238895416 0.949925243854523\\
1.38000249862671 0.947247445583344\\
1.39000248908997 0.944505035877228\\
1.40000247955322 0.941698014736176\\
1.41000247001648 0.938826501369476\\
1.42000246047974 0.935890555381775\\
1.43000245094299 0.932890295982361\\
1.44000244140625 0.929825723171234\\
1.45000243186951 0.92669689655304\\
1.46000242233276 0.923503935337067\\
1.47000241279602 0.920246839523315\\
1.48000240325928 0.916925668716431\\
1.49000239372253 0.913540422916412\\
1.50000238418579 0.910091161727905\\
1.51000249385834 0.906577944755554\\
1.52000248432159 0.90300065279007\\
1.53000247478485 0.899359464645386\\
1.54000246524811 0.895654261112213\\
1.55000245571136 0.891885101795197\\
1.56000244617462 0.888051927089691\\
1.57000243663788 0.884154796600342\\
1.58000242710114 0.880193650722504\\
1.59000241756439 0.876168489456177\\
1.60000240802765 0.872079253196716\\
1.61000239849091 0.867926001548767\\
1.62000238895416 0.86370861530304\\
1.63000249862671 0.859427154064178\\
1.64000248908997 0.855081498622894\\
1.65000247955322 0.850671648979187\\
1.66000247001648 0.846197605133057\\
1.67000246047974 0.841659307479858\\
1.68000245094299 0.837056696414948\\
1.69000244140625 0.832389712333679\\
1.70000243186951 0.827658355236053\\
1.71000242233276 0.822862565517426\\
1.72000241279602 0.818002343177795\\
1.73000240325928 0.813077509403229\\
1.74000239372253 0.80808812379837\\
1.75000238418579 0.803034067153931\\
1.76000249385834 0.797916233539581\\
1.77000248432159 0.792760729789734\\
1.78000247478485 0.787597835063934\\
1.79000246524811 0.782431662082672\\
1.80000245571136 0.777256071567535\\
1.81000244617462 0.772062599658966\\
1.82000243663788 0.766842603683472\\
1.83000242710114 0.761588752269745\\
1.84000241756439 0.756294667720795\\
1.85000240802765 0.750954985618591\\
1.86000239849091 0.745565354824066\\
1.87000238895416 0.740122199058533\\
1.88000249862671 0.734622418880463\\
1.89000248908997 0.729063630104065\\
1.90000247955322 0.723443686962128\\
1.91000247001648 0.717760682106018\\
1.92000246047974 0.712013244628906\\
1.93000245094299 0.706199824810028\\
1.94000244140625 0.700319290161133\\
1.95000243186951 0.69437050819397\\
1.96000242233276 0.688352346420288\\
1.97000241279602 0.682263791561127\\
1.98000240325928 0.676115453243256\\
1.99000239372253 0.672087252140045\\
2.00000238418579 0.67232882976532\\
2.01000237464905 0.676584005355835\\
2.0200023651123 0.683436095714569\\
2.03000235557556 0.691209495067596\\
2.04000234603882 0.698723435401917\\
2.05000233650208 0.705377638339996\\
2.06000232696533 0.710965991020203\\
2.07000255584717 0.715481877326965\\
2.08000254631042 0.719024658203125\\
2.09000253677368 0.721766769886017\\
2.10000252723694 0.723845064640045\\
2.1100025177002 0.725369334220886\\
2.12000250816345 0.726444840431213\\
2.13000249862671 0.727207958698273\\
2.14000248908997 0.727903068065643\\
2.15000247955322 0.729034006595612\\
2.16000247001648 0.731527984142303\\
2.17000246047974 0.73650336265564\\
2.18000245094299 0.744256794452667\\
2.19000244140625 0.753634631633759\\
2.20000243186951 0.762977123260498\\
2.21000242233276 0.771209299564362\\
2.22000241279602 0.778065741062164\\
2.23000240325928 0.783720254898071\\
2.24000239372253 0.788461208343506\\
2.25000238418579 0.792538404464722\\
2.26000237464905 0.796130836009979\\
2.2700023651123 0.799349129199982\\
2.28000235557556 0.802254021167755\\
2.29000234603882 0.804873406887054\\
2.30000233650208 0.807216942310333\\
2.31000232696533 0.809285640716553\\
2.32000255584717 0.811077654361725\\
2.33000254631042 0.812591433525085\\
2.34000253677368 0.813827633857727\\
2.35000252723694 0.814789414405823\\
2.3600025177002 0.815482795238495\\
2.37000250816345 0.815915882587433\\
2.38000249862671 0.816098809242249\\
2.39000248908997 0.816043078899384\\
2.40000247955322 0.815761923789978\\
2.41000247001648 0.815302014350891\\
2.42000246047974 0.814667403697968\\
2.43000245094299 0.813839137554169\\
2.44000244140625 0.812807083129883\\
2.45000243186951 0.811569392681122\\
2.46000242233276 0.810128271579742\\
2.47000241279602 0.808315396308899\\
2.48000240325928 0.805980622768402\\
2.49000239372253 0.803062856197357\\
2.50000238418579 0.799528241157532\\
2.51000237464905 0.795372486114502\\
2.5200023651123 0.790614366531372\\
2.53000235557556 0.785287499427795\\
2.54000234603882 0.779434502124786\\
2.55000233650208 0.773102402687073\\
2.56000232696533 0.766340374946594\\
2.57000255584717 0.75919783115387\\
2.58000254631042 0.75172346830368\\
2.59000253677368 0.743964850902557\\
2.60000252723694 0.735968053340912\\
2.6100025177002 0.727777719497681\\
2.62000250816345 0.719438433647156\\
2.63000249862671 0.710996389389038\\
2.64000248908997 0.702500998973846\\
2.65000247955322 0.694006979465485\\
2.66000247001648 0.685588479042053\\
2.67000246047974 0.677358090877533\\
2.68000245094299 0.669408679008484\\
2.69000244140625 0.661824107170105\\
2.70000243186951 0.654695153236389\\
2.71000242233276 0.648134887218475\\
2.72000241279602 0.643493950366974\\
2.73000240325928 0.642441034317017\\
2.74000239372253 0.64506870508194\\
2.75000238418579 0.650754928588867\\
2.76000237464905 0.658449411392212\\
2.7700023651123 0.667110979557037\\
2.78000235557556 0.676054418087006\\
2.79000234603882 0.685020327568054\\
2.80000233650208 0.693921685218811\\
2.81000232696533 0.702739238739014\\
2.82000255584717 0.71136599779129\\
2.83000254631042 0.719673097133636\\
2.84000253677368 0.727609395980835\\
2.85000252723694 0.735151290893555\\
2.8600025177002 0.742286622524261\\
2.87000250816345 0.749027669429779\\
2.88000249862671 0.755425810813904\\
2.89000248908997 0.761612594127655\\
2.90000247955322 0.767895638942719\\
2.91000247001648 0.774844527244568\\
2.92000246047974 0.78313934803009\\
2.93000245094299 0.792882740497589\\
2.94000244140625 0.803150773048401\\
2.95000243186951 0.812635600566864\\
2.96000242233276 0.82045191526413\\
2.97000241279602 0.8263338804245\\
2.98000240325928 0.830405116081238\\
2.99000239372253 0.832934319972992\\
3.00000238418579 0.834196984767914\\
3.01000237464905 0.834425449371338\\
3.0200023651123 0.833804368972778\\
3.03000235557556 0.832489609718323\\
3.04000234603882 0.830633878707886\\
3.05000233650208 0.828379988670349\\
3.06000232696533 0.825823307037354\\
3.07000255584717 0.823015332221985\\
3.08000254631042 0.819951713085175\\
3.09000253677368 0.816335499286652\\
3.10000252723694 0.812011420726776\\
3.1100025177002 0.806896984577179\\
3.12000250816345 0.800952196121216\\
3.13000249862671 0.794185936450958\\
3.14000248908997 0.786645114421844\\
3.15000247955322 0.778406262397766\\
3.16000247001648 0.769563794136047\\
3.17000246047974 0.760220050811768\\
3.18000245094299 0.750483810901642\\
3.19000244140625 0.740464270114899\\
3.20000243186951 0.730268776416779\\
3.21000242233276 0.720010936260223\\
3.22000241279602 0.709814250469208\\
3.23000240325928 0.699811816215515\\
3.24000239372253 0.690146446228027\\
3.25000238418579 0.680972337722778\\
3.26000237464905 0.672465205192566\\
3.2700023651123 0.664815247058868\\
3.28000235557556 0.658277809619904\\
3.29000234603882 0.654153883457184\\
3.30000233650208 0.653075635433197\\
3.31000232696533 0.655135214328766\\
3.32000255584717 0.660297453403473\\
3.33000254631042 0.668414294719696\\
3.34000253677368 0.679233729839325\\
3.35000252723694 0.692402482032776\\
3.3600025177002 0.707357227802277\\
3.37000250816345 0.723401069641113\\
3.38000249862671 0.739875912666321\\
3.39000248908997 0.756195962429047\\
3.40000247955322 0.771478772163391\\
3.41000247001648 0.784752488136292\\
3.42000246047974 0.795693695545197\\
3.43000245094299 0.804405152797699\\
3.44000244140625 0.811127781867981\\
3.45000243186951 0.816102921962738\\
3.46000242233276 0.819564938545227\\
3.47000241279602 0.821808576583862\\
3.48000240325928 0.82325553894043\\
3.49000239372253 0.824495315551758\\
3.50000238418579 0.82609885931015\\
3.51000237464905 0.828206777572632\\
3.5200023651123 0.830396354198456\\
3.53000235557556 0.832020282745361\\
3.54000234603882 0.832619369029999\\
3.55000233650208 0.832059502601624\\
3.56000232696533 0.830418109893799\\
3.57000255584717 0.827859818935394\\
3.58000254631042 0.824575543403625\\
3.59000253677368 0.820736944675446\\
3.60000252723694 0.816327691078186\\
3.6100025177002 0.811276376247406\\
3.62000250816345 0.805596888065338\\
3.63000249862671 0.799300372600555\\
3.64000248908997 0.792412281036377\\
3.65000247955322 0.784977316856384\\
3.66000247001648 0.777066648006439\\
3.67000246047974 0.76876026391983\\
3.68000245094299 0.760121822357178\\
3.69000244140625 0.751200020313263\\
3.70000243186951 0.74203360080719\\
3.71000242233276 0.732673406600952\\
3.72000241279602 0.723198413848877\\
3.73000240325928 0.713714063167572\\
3.74000239372253 0.704339861869812\\
3.75000238418579 0.695197284221649\\
3.76000237464905 0.686403214931488\\
3.7700023651123 0.678071260452271\\
3.78000235557556 0.670320153236389\\
3.79000234603882 0.66326379776001\\
3.80000233650208 0.657258152961731\\
3.81000232696533 0.65373033285141\\
3.82000255584717 0.652927458286285\\
3.83000254631042 0.654780507087708\\
3.84000253677368 0.659134268760681\\
3.85000252723694 0.665776073932648\\
3.8600025177002 0.674457848072052\\
3.87000250816345 0.684912621974945\\
3.88000249862671 0.696814358234406\\
3.89000248908997 0.709755480289459\\
3.90000247955322 0.723380088806152\\
3.91000247001648 0.737419962882996\\
3.92000246047974 0.751447975635529\\
3.93000245094299 0.764988660812378\\
3.94000244140625 0.777369737625122\\
3.95000243186951 0.78799432516098\\
3.96000242233276 0.796693861484528\\
3.97000241279602 0.803554534912109\\
3.98000240325928 0.808800458908081\\
3.99000239372253 0.81277722120285\\
4.00000238418579 0.816004395484924\\
4.01000261306763 0.819147408008575\\
4.0200023651123 0.822688817977905\\
4.03000259399414 0.826514184474945\\
4.04000234603882 0.830002725124359\\
4.05000257492065 0.83250880241394\\
4.06000232696533 0.833692789077759\\
4.07000255584717 0.833533406257629\\
4.08000230789185 0.832201063632965\\
4.09000253677368 0.829906404018402\\
4.10000228881836 0.826833486557007\\
4.1100025177002 0.823146998882294\\
4.12000226974487 0.818972527980804\\
4.13000249862671 0.814260363578796\\
4.14000225067139 0.809014856815338\\
4.15000247955322 0.803245842456818\\
4.1600022315979 0.796961486339569\\
4.17000246047974 0.790179312229156\\
4.18000221252441 0.782933235168457\\
4.19000244140625 0.775281429290771\\
4.20000267028809 0.767274618148804\\
4.21000242233276 0.758937776088715\\
4.2200026512146 0.750287234783173\\
4.23000240325928 0.741360604763031\\
4.24000263214111 0.732235491275787\\
4.25000238418579 0.723028361797333\\
4.26000261306763 0.713881134986877\\
4.2700023651123 0.70494669675827\\
4.28000259399414 0.696373283863068\\
4.29000234603882 0.688297808170319\\
4.30000257492065 0.68085104227066\\
4.31000232696533 0.674146115779877\\
4.32000255584717 0.668870568275452\\
4.33000230789185 0.666284382343292\\
4.34000253677368 0.666319847106934\\
4.35000228881836 0.668751239776611\\
4.3600025177002 0.673304796218872\\
4.37000226974487 0.679687261581421\\
4.38000249862671 0.687608063220978\\
4.39000225067139 0.696780920028687\\
4.40000247955322 0.706885099411011\\
4.4100022315979 0.717602491378784\\
4.42000246047974 0.72868424654007\\
4.43000221252441 0.739975452423096\\
4.44000244140625 0.751364469528198\\
4.45000267028809 0.762536823749542\\
4.46000242233276 0.773160457611084\\
4.4700026512146 0.782671272754669\\
4.48000240325928 0.790699064731598\\
4.49000263214111 0.797164916992188\\
4.50000238418579 0.802182257175446\\
4.51000261306763 0.806034445762634\\
4.5200023651123 0.809218764305115\\
4.53000259399414 0.812403678894043\\
4.54000234603882 0.816087603569031\\
4.55000257492065 0.820129036903381\\
4.56000232696533 0.823852241039276\\
4.57000255584717 0.826563596725464\\
4.58000230789185 0.827903032302856\\
4.59000253677368 0.827838659286499\\
4.60000228881836 0.826531589031219\\
4.6100025177002 0.824223399162292\\
4.62000226974487 0.821124196052551\\
4.63000249862671 0.81739330291748\\
4.64000225067139 0.813075184822083\\
4.65000247955322 0.808099567890167\\
4.6600022315979 0.802511811256409\\
4.67000246047974 0.796341896057129\\
4.68000221252441 0.78961855173111\\
4.69000244140625 0.782401621341705\\
4.70000267028809 0.77474582195282\\
4.71000242233276 0.76672637462616\\
4.7200026512146 0.758427917957306\\
4.73000240325928 0.749910414218903\\
4.74000263214111 0.741192102432251\\
4.75000238418579 0.732276856899261\\
4.76000261306763 0.723193943500519\\
4.7700023651123 0.714022874832153\\
4.78000259399414 0.704891204833984\\
4.79000234603882 0.695958316326141\\
4.80000257492065 0.68740302324295\\
4.81000232696533 0.679418742656708\\
4.82000255584717 0.672211349010468\\
4.83000230789185 0.666107714176178\\
4.84000253677368 0.662941515445709\\
4.85000228881836 0.663268625736237\\
4.8600025177002 0.666871905326843\\
4.87000226974487 0.673372209072113\\
4.88000249862671 0.682315528392792\\
4.89000225067139 0.693206906318665\\
4.90000247955322 0.70544570684433\\
4.9100022315979 0.718441963195801\\
4.92000246047974 0.731704592704773\\
4.93000221252441 0.744857788085938\\
4.94000244140625 0.757653117179871\\
4.95000267028809 0.769903540611267\\
4.96000242233276 0.781043231487274\\
4.9700026512146 0.790530323982239\\
4.98000240325928 0.798227965831757\\
4.99000263214111 0.804194629192352\\
5 0.808577299118042\\
};
\addlegendentry{$\dtmax{} = \expneg{1.0}{5}$ [s]};

\end{axis}
\end{tikzpicture}%
\caption{Linear solver flashing solutions.}
\label{fig:flashingAlphaLin}
\end{figure}

\begin{figure}[h!tb]
\centering
% This file was created by matlab2tikz v0.4.3.
% Copyright (c) 2008--2013, Nico Schlömer <nico.schloemer@gmail.com>
% All rights reserved.
% 
\tikzsetnextfilename{plots/flashingAlphaNln_pdf}
\begin{tikzpicture}

\begin{axis}[%
width=\mytikzpicwidth,
height=\mytikzpicheight,
scale only axis,
xmin=0.0,
xmax=5.0,
xlabel={Time $[\text{s}]$},
ymin=0.5,
ymax=1.0,
ylabel={$\alpha_g$ [-] @ 2 [in] from Inlet},
legend style={at={(0.03,0.03)},anchor=south west,draw=black,fill=white,legend cell align=left}
]
\addplot [
color=black,
solid,
mark=o,
mark options={solid}
]
table[row sep=crcr]{
0.0 1.0\\
0.0118019115179777 1\\
0.0226854234933853 1\\
0.0340500771999359 1\\
0.0491764321923256 1\\
0.0619497485458851 1\\
0.0774054601788521 1\\
0.0961068719625473 1\\
0.106882445514202 1\\
0.118735581636429 1\\
0.13177402317524 1\\
0.146116316318512 1\\
0.161892831325531 1\\
0.179247006773949 1\\
0.198336601257324 1\\
0.21933513879776 1\\
0.242433547973633 1\\
0.267841786146164 1\\
0.29579085111618 1\\
0.32653483748436 1\\
0.360353201627731 1\\
0.397553414106369 1\\
0.438473641872406 1\\
0.483485877513886 1\\
0.532999336719513 1\\
0.587464153766632 1\\
0.647375464439392 1\\
0.71327793598175 1\\
0.7857705950737 1\\
0.865512549877167 1\\
0.953228712081909 1\\
1.04971647262573 0.995161294937134\\
1.1558530330658 0.979693293571472\\
1.27260315418243 0.957357287406921\\
1.32053518295288 0.947412252426147\\
1.35773289203644 0.93841552734375\\
1.38946354389191 0.929924786090851\\
1.42200791835785 0.920374691486359\\
1.45245742797852 0.910801649093628\\
1.48083031177521 0.901349127292633\\
1.50743794441223 0.892022550106049\\
1.53255617618561 0.882808029651642\\
1.55641222000122 0.873687744140625\\
1.57917487621307 0.864649832248688\\
1.60098683834076 0.855681836605072\\
1.62195324897766 0.846776962280273\\
1.64217150211334 0.837925493717194\\
1.66171324253082 0.829122722148895\\
1.68064773082733 0.820361614227295\\
1.69902181625366 0.811640918254852\\
1.71688628196716 0.802955090999603\\
1.73428571224213 0.794298350811005\\
1.75125479698181 0.785774171352386\\
1.76759457588196 0.777539014816284\\
1.78327512741089 0.769592463970184\\
1.79837548732758 0.761874079704285\\
1.81300187110901 0.754316449165344\\
1.82727634906769 0.746847033500671\\
1.84123587608337 0.739443123340607\\
1.85493552684784 0.732072591781616\\
1.86838006973267 0.724733471870422\\
1.88161540031433 0.71740061044693\\
1.89463996887207 0.710077345371246\\
1.90748333930969 0.702748000621796\\
1.92014336585999 0.695416688919067\\
1.93264579772949 0.688069760799408\\
1.94498836994171 0.683255314826965\\
1.95688509941101 0.681701004505157\\
1.96837878227234 0.682874798774719\\
1.97949528694153 0.685770511627197\\
1.99014973640442 0.689332842826843\\
2.00136232376099 0.693269908428192\\
2.01271271705627 0.697051227092743\\
2.02417707443237 0.700365483760834\\
2.0357928276062 0.703082680702209\\
2.04750537872314 0.705152451992035\\
2.05927753448486 0.706590950489044\\
2.07107305526733 0.707451343536377\\
2.08286380767822 0.707899689674377\\
2.09455490112305 0.708136200904846\\
2.10611391067505 0.708497524261475\\
2.11754536628723 0.709739506244659\\
2.12883257865906 0.713333308696747\\
2.13992094993591 0.720506727695465\\
2.15075325965881 0.729838073253632\\
2.16134643554688 0.738739728927612\\
2.1717312335968 0.746206641197205\\
2.18185663223267 0.752267837524414\\
2.20230388641357 0.762538492679596\\
2.21264386177063 0.767454922199249\\
2.22270774841309 0.772260308265686\\
2.23349833488464 0.777447164058685\\
2.24430298805237 0.782548725605011\\
2.25509786605835 0.787428200244904\\
2.26583743095398 0.791951775550842\\
2.27648019790649 0.7960165143013\\
2.29678177833557 0.802417457103729\\
2.30735301971436 0.804985463619232\\
2.32744646072388 0.808435916900635\\
2.34562826156616 0.810013592243195\\
2.36075878143311 0.810389518737793\\
2.3733606338501 0.810078859329224\\
2.38860893249512 0.809010863304138\\
2.40705943107605 0.806161463260651\\
2.41769051551819 0.803072869777679\\
2.42919039726257 0.798441052436829\\
2.4407947063446 0.792471528053284\\
2.45252275466919 0.78523051738739\\
2.46437191963196 0.776840925216675\\
2.47634053230286 0.767454028129578\\
2.48842525482178 0.757237017154694\\
2.50060844421387 0.746374726295471\\
2.51288247108459 0.73504239320755\\
2.52523136138916 0.723420560359955\\
2.53763508796692 0.711685955524445\\
2.55007195472717 0.699997961521149\\
2.56252241134644 0.688497424125671\\
2.57496857643127 0.677303493022919\\
2.58739495277405 0.666511058807373\\
2.59979104995728 0.656353831291199\\
2.61202311515808 0.647194385528564\\
2.62403297424316 0.639329731464386\\
2.63583588600159 0.636150002479553\\
2.64725995063782 0.638070404529572\\
2.65829944610596 0.643946290016174\\
2.66903948783875 0.652319014072418\\
2.68084859848022 0.662393510341644\\
2.69513750076294 0.675918996334076\\
2.71242713928223 0.693264961242676\\
2.7327516078949 0.713853061199188\\
2.74304962158203 0.723939299583435\\
2.75329804420471 0.733607172966003\\
2.76352214813232 0.742822408676147\\
2.77671360969543 0.753786623477936\\
2.79198384284973 0.765558302402496\\
2.81012392044067 0.778243958950043\\
2.82984495162964 0.7897989153862\\
2.84034872055054 0.795110523700714\\
2.85096287727356 0.800727307796478\\
2.86159515380859 0.807240605354309\\
2.87224507331848 0.813998699188232\\
2.88282418251038 0.819493353366852\\
2.89327502250671 0.82299679517746\\
2.90423226356506 0.824577033519745\\
2.91532301902771 0.824596226215363\\
2.92652130126953 0.823707103729248\\
2.93784737586975 0.822374880313873\\
2.94925856590271 0.820825397968292\\
2.96071529388428 0.819130122661591\\
2.97219562530518 0.817319929599762\\
2.98368406295776 0.815415620803833\\
2.99517321586609 0.813416659832001\\
3.00665736198425 0.811287820339203\\
3.018141746521 0.808829605579376\\
3.02963495254517 0.805378913879395\\
3.04114675521851 0.80073493719101\\
3.05272746086121 0.794805824756622\\
3.06440591812134 0.787599384784698\\
3.07619190216064 0.779191017150879\\
3.08806824684143 0.769720613956451\\
3.10001063346863 0.759366571903229\\
3.11200356483459 0.748313188552856\\
3.12402725219727 0.736752152442932\\
3.13606834411621 0.724875271320343\\
3.14812350273132 0.712887644767761\\
3.16019630432129 0.701007485389709\\
3.17221331596375 0.689547657966614\\
3.18425130844116 0.678698837757111\\
3.1963107585907 0.668757975101471\\
3.20831441879272 0.660067141056061\\
3.22024464607239 0.653569519519806\\
3.2320544719696 0.65085232257843\\
3.24364876747131 0.651964008808136\\
3.25497913360596 0.656622290611267\\
3.26603627204895 0.664417147636414\\
3.27683329582214 0.675070106983185\\
3.28738880157471 0.688274800777435\\
3.30090045928955 0.707143545150757\\
3.31724953651428 0.733064949512482\\
3.3364520072937 0.763977468013763\\
3.35590267181396 0.788724601268768\\
3.37547159194946 0.80456668138504\\
3.39097309112549 0.812018096446991\\
3.40369939804077 0.815524220466614\\
3.41909837722778 0.816869139671326\\
3.43773102760315 0.815859615802765\\
3.44846701622009 0.815338432788849\\
3.45963358879089 0.815738797187805\\
3.47084331512451 0.816300868988037\\
3.4820818901062 0.815788686275482\\
3.49336910247803 0.813811302185059\\
3.50474190711975 0.810532748699188\\
3.51621460914612 0.806376278400421\\
3.52777361869812 0.801736891269684\\
3.53940916061401 0.796741425991058\\
3.55113887786865 0.791134297847748\\
3.5629894733429 0.784699440002441\\
3.57491517066956 0.777324438095093\\
3.58686423301697 0.769021987915039\\
3.59883189201355 0.759876608848572\\
3.61085295677185 0.750011324882507\\
3.62294244766235 0.739583373069763\\
3.63507795333862 0.728764295578003\\
3.64724469184875 0.717696487903595\\
3.65939402580261 0.70654833316803\\
3.67149782180786 0.695481717586517\\
3.68352150917053 0.684714615345001\\
3.69545078277588 0.674504101276398\\
3.70728397369385 0.665115237236023\\
3.71901512145996 0.656813383102417\\
3.73064255714417 0.650577366352081\\
3.74213266372681 0.648110270500183\\
3.75340151786804 0.649445056915283\\
3.7644145488739 0.654268085956573\\
3.77517652511597 0.662119626998901\\
3.78571152687073 0.672529995441437\\
3.79605078697205 0.685197651386261\\
3.80622386932373 0.699702262878418\\
3.82096457481384 0.721896111965179\\
3.83880090713501 0.750025570392609\\
3.85830807685852 0.778035581111908\\
3.87783885002136 0.797684550285339\\
3.89773917198181 0.809440314769745\\
3.90787315368652 0.812820613384247\\
3.91816449165344 0.814779281616211\\
3.93745827674866 0.815654277801514\\
3.94821214675903 0.815431535243988\\
3.95906329154968 0.815935730934143\\
3.96999645233154 0.817408263683319\\
3.98096656799316 0.818963408470154\\
3.991947889328 0.819588601589203\\
4.00295209884644 0.818919479846954\\
4.01402473449707 0.817019820213318\\
4.02517938613892 0.81412672996521\\
4.03638935089111 0.810531377792358\\
4.04771900177002 0.806409060955048\\
4.05928993225098 0.801859438419342\\
4.07111120223999 0.796508491039276\\
4.08298397064209 0.790336966514587\\
4.09486865997314 0.783325910568237\\
4.10676383972168 0.775501728057861\\
4.11869525909424 0.766916155815125\\
4.13065099716187 0.757630288600922\\
4.14261865615845 0.747690081596375\\
4.15452194213867 0.737242519855499\\
4.16634511947632 0.726497709751129\\
4.17807626724243 0.715754806995392\\
4.18974590301514 0.705283641815186\\
4.20136594772339 0.695355176925659\\
4.21294355392456 0.686231434345245\\
4.22449111938477 0.678154289722443\\
4.23601484298706 0.671893894672394\\
4.24750518798828 0.668983280658722\\
4.25887632369995 0.669286847114563\\
4.27007293701172 0.672407865524292\\
4.28108787536621 0.677898943424225\\
4.29192972183228 0.685348749160767\\
4.30261182785034 0.694470524787903\\
4.31315040588379 0.70512866973877\\
4.32356023788452 0.716991424560547\\
4.33452129364014 0.729404628276825\\
4.34778356552124 0.745184481143951\\
4.36383104324341 0.764494061470032\\
4.38311052322388 0.784277379512787\\
4.3932032585144 0.79193127155304\\
4.4033842086792 0.797783195972443\\
4.41344928741455 0.801898837089539\\
4.42365694046021 0.804547786712646\\
4.43415594100952 0.805887162685394\\
4.44478559494019 0.806167900562286\\
4.4555401802063 0.805964827537537\\
4.46635627746582 0.806215047836304\\
4.47721815109253 0.807421147823334\\
4.48809051513672 0.808788061141968\\
4.4989447593689 0.80925053358078\\
4.50979852676392 0.808584213256836\\
4.52070713043213 0.806863725185394\\
4.53168821334839 0.80427610874176\\
4.54263353347778 0.801081836223602\\
4.55374670028687 0.797418534755707\\
4.56528568267822 0.793280065059662\\
4.57695484161377 0.78841096162796\\
4.58858823776245 0.782783329486847\\
4.6002082824707 0.776357889175415\\
4.61189317703247 0.769100785255432\\
4.62363004684448 0.761059939861298\\
4.63552665710449 0.75224769115448\\
4.64763736724854 0.742677927017212\\
4.65980005264282 0.732520461082459\\
4.67189693450928 0.721945226192474\\
4.68388748168945 0.711209774017334\\
4.69575357437134 0.700695872306824\\
4.70748949050903 0.690852165222168\\
4.71910238265991 0.682121574878693\\
4.73061180114746 0.674898386001587\\
4.74203300476074 0.671482026576996\\
4.75329446792603 0.672371327877045\\
4.7643346786499 0.676974236965179\\
4.77515077590942 0.684498250484467\\
4.78576755523682 0.694226086139679\\
4.79621171951294 0.705526888370514\\
4.8065128326416 0.717834830284119\\
4.81669855117798 0.730810105800629\\
4.82679224014282 0.744160354137421\\
4.8414249420166 0.76282000541687\\
4.859130859375 0.782685697078705\\
4.87904977798462 0.797951638698578\\
4.88912153244019 0.802902400493622\\
4.89928483963013 0.806248068809509\\
4.90958023071289 0.808123350143433\\
4.92001533508301 0.808634281158447\\
4.93059682846069 0.807890236377716\\
4.94941663742065 0.804432272911072\\
4.9602427482605 0.801936388015747\\
4.97117471694946 0.800463736057281\\
4.98216915130615 0.800458490848541\\
5 0.801366329193115\\
};
\addlegendentry{$\dtmax{} = \expneg{1.0}{0}{[s]}$};

\addplot [
color=black,
solid,
mark=+,
mark options={solid}
]
table[row sep=crcr]{
0 1\\
0.0118019115179777 1\\
0.0226854234933853 1\\
0.0340500771999359 1\\
0.0491764321923256 1\\
0.0619497485458851 1\\
0.0774054601788521 1\\
0.0961068719625473 1\\
0.106882445514202 1\\
0.118735581636429 1\\
0.13177402317524 1\\
0.146116316318512 1\\
0.161892831325531 1\\
0.179247006773949 1\\
0.198336601257324 1\\
0.21933513879776 1\\
0.242433547973633 1\\
0.267841786146164 1\\
0.29579085111618 1\\
0.32653483748436 1\\
0.360353201627731 1\\
0.397553414106369 1\\
0.438473641872406 1\\
0.483485877513886 1\\
0.532999336719513 1\\
0.587464153766632 1\\
0.647375464439392 1\\
0.71327793598175 1\\
0.7857705950737 1\\
0.865512549877167 1\\
0.953228712081909 1\\
1.04971647262573 0.995161294937134\\
1.14971649646759 0.981202840805054\\
1.24971640110016 0.964120328426361\\
1.30053281784058 0.953730702400208\\
1.33893167972565 0.944734871387482\\
1.37577772140503 0.935036301612854\\
1.41023123264313 0.925137042999268\\
1.44182658195496 0.9154012799263\\
1.47101235389709 0.905854403972626\\
1.49824774265289 0.896464765071869\\
1.52388393878937 0.887201070785522\\
1.54817187786102 0.878042936325073\\
1.57130980491638 0.868972480297089\\
1.59344398975372 0.859978795051575\\
1.6147004365921 0.851049840450287\\
1.63517260551453 0.842179179191589\\
1.65494728088379 0.833357930183411\\
1.67408645153046 0.824583053588867\\
1.69265270233154 0.815847635269165\\
1.710693359375 0.807148039340973\\
1.72825372219086 0.798480331897736\\
1.74536681175232 0.789871454238892\\
1.76200985908508 0.781474113464355\\
1.77794885635376 0.773400783538818\\
1.79327201843262 0.765585124492645\\
1.80806839466095 0.757964074611664\\
1.82245671749115 0.750465869903564\\
1.83652806282043 0.743035733699799\\
1.85030794143677 0.735658407211304\\
1.86384534835815 0.728305518627167\\
1.87714087963104 0.720978081226349\\
1.89023864269257 0.713651835918427\\
1.90313458442688 0.706331849098206\\
1.9158570766449 0.699002921581268\\
1.92840397357941 0.691669166088104\\
1.9407993555069 0.685332894325256\\
1.95291471481323 0.682097434997559\\
1.96465003490448 0.681982517242432\\
1.97588443756104 0.684126496315002\\
1.9866806268692 0.687433362007141\\
1.99778032302856 0.69133734703064\\
2.00900340080261 0.695275604724884\\
2.02036333084106 0.698879957199097\\
2.03189587593079 0.701945126056671\\
2.04355406761169 0.704373180866241\\
2.0552933216095 0.706152737140656\\
2.06707572937012 0.707323253154755\\
2.0788676738739 0.707997798919678\\
2.09061312675476 0.708405375480652\\
2.10223054885864 0.708902955055237\\
2.11370897293091 0.710277557373047\\
2.12503623962402 0.71404242515564\\
2.136150598526 0.721379935741425\\
2.14699602127075 0.730758845806122\\
2.1575915813446 0.739610314369202\\
2.16798114776611 0.746988117694855\\
2.17810916900635 0.752950251102448\\
2.19861006736755 0.763028740882874\\
2.2089672088623 0.767849028110504\\
2.21904635429382 0.772561609745026\\
2.22985172271729 0.777646720409393\\
2.24066185951233 0.782642126083374\\
2.25146412849426 0.787420272827148\\
2.2622447013855 0.791858613491058\\
2.27260613441467 0.795736610889435\\
2.29255986213684 0.801916182041168\\
2.30319738388062 0.804464101791382\\
2.32355046272278 0.807902634143829\\
2.3420250415802 0.809480369091034\\
2.35715985298157 0.809899926185608\\
2.36990928649902 0.809662997722626\\
2.38533616065979 0.808698415756226\\
2.40400266647339 0.805973529815674\\
2.41475820541382 0.802925944328308\\
2.42625689506531 0.798380136489868\\
2.43785905838013 0.792508482933044\\
2.44958400726318 0.785378694534302\\
2.46142792701721 0.777113795280457\\
2.47339081764221 0.767861247062683\\
2.4854679107666 0.757785081863403\\
2.49764442443848 0.747063219547272\\
2.50991177558899 0.73586642742157\\
2.52225399017334 0.724367737770081\\
2.53465104103088 0.712740063667297\\
2.54708194732666 0.701139450073242\\
2.55952668190002 0.689704298973084\\
2.57196760177612 0.678551852703094\\
2.58438992500305 0.667777955532074\\
2.59678220748901 0.657625734806061\\
2.60899806022644 0.648454368114471\\
2.6209888458252 0.640539586544037\\
2.63277387619019 0.637021899223328\\
2.64419460296631 0.638606131076813\\
2.65522933006287 0.644217014312744\\
2.66596651077271 0.65240341424942\\
2.67762064933777 0.662158787250519\\
2.69172215461731 0.675208687782288\\
2.70878481864929 0.691880881786346\\
2.72900438308716 0.711828052997589\\
2.73932576179504 0.721691310405731\\
2.74959468841553 0.731154441833496\\
2.75983572006226 0.740185737609863\\
2.77291202545166 0.750841915607452\\
2.78830885887146 0.762519896030426\\
2.80667853355408 0.775223076343536\\
2.8262140750885 0.78701114654541\\
2.84571003913879 0.797741830348969\\
2.85642600059509 0.804443359375\\
2.86720728874207 0.811687648296356\\
2.87802314758301 0.817861318588257\\
2.88886737823486 0.821977138519287\\
2.89976835250854 0.824001967906952\\
2.91077423095703 0.824383080005646\\
2.92191886901855 0.823754847049713\\
2.9332013130188 0.822613120079041\\
2.94430899620056 0.821269452571869\\
2.9557466506958 0.81975793838501\\
2.9672155380249 0.818131387233734\\
2.97869729995728 0.816402673721313\\
2.99018120765686 0.814574897289276\\
3.00165987014771 0.812633156776428\\
3.01313447952271 0.81043666601181\\
3.02461242675781 0.807852923870087\\
3.03609657287598 0.804229438304901\\
3.04761719703674 0.799383640289307\\
3.05921459197998 0.793252348899841\\
3.07091021537781 0.785865306854248\\
3.08270359039307 0.777320206165314\\
3.09456205368042 0.767776310443878\\
3.10646939277649 0.757400929927826\\
3.11841011047363 0.746369183063507\\
3.13038539886475 0.734845459461212\\
3.14239931106567 0.723006010055542\\
3.15434217453003 0.711172997951508\\
3.16546750068665 0.700289070606232\\
3.17758107185364 0.688842594623566\\
3.189617395401 0.678138613700867\\
3.20145630836487 0.668542802333832\\
3.21335864067078 0.660118103027344\\
3.22526288032532 0.654100954532623\\
3.23705554008484 0.651877880096436\\
3.24862623214722 0.653448581695557\\
3.25993156433105 0.658495485782623\\
3.27096676826477 0.666596531867981\\
3.28174424171448 0.677468240261078\\
3.29228401184082 0.69079601764679\\
3.30546855926514 0.709105372428894\\
3.32142210006714 0.733945071697235\\
3.340411901474 0.763932406902313\\
3.35988259315491 0.788448929786682\\
3.37951111793518 0.804207503795624\\
3.3895115852356 0.809291779994965\\
3.40205717086792 0.813735544681549\\
3.41723728179932 0.816038548946381\\
3.4356050491333 0.8155397772789\\
3.44618844985962 0.814639031887054\\
3.45729732513428 0.814429879188538\\
3.46846532821655 0.815291881561279\\
3.47966051101685 0.816169381141663\\
3.4908709526062 0.815887451171875\\
3.50212430953979 0.814078867435455\\
3.51346206665039 0.810959815979004\\
3.52487444877625 0.8069988489151\\
3.53636074066162 0.80257248878479\\
3.54794073104858 0.797780692577362\\
3.55965638160706 0.79229211807251\\
3.57149004936218 0.785905182361603\\
3.58335900306702 0.77855396270752\\
3.59523272514343 0.770267307758331\\
3.60715103149414 0.761110126972198\\
3.61914253234863 0.751213014125824\\
3.63121247291565 0.740739762783051\\
3.6433367729187 0.729866683483124\\
3.6555004119873 0.718740224838257\\
3.66765594482422 0.707531988620758\\
3.67977452278137 0.696396589279175\\
3.69182777404785 0.685528159141541\\
3.70379018783569 0.675175368785858\\
3.71564841270447 0.665605187416077\\
3.7274022102356 0.657072961330414\\
3.73905062675476 0.65020364522934\\
3.75057649612427 0.647026777267456\\
3.7618887424469 0.64766263961792\\
3.772944688797 0.651867926120758\\
3.78374242782593 0.659224390983582\\
3.79430341720581 0.669286608695984\\
3.80466079711914 0.681766331195831\\
3.8148398399353 0.696270406246185\\
3.82878112792969 0.717297255992889\\
3.84565019607544 0.744315564632416\\
3.86506748199463 0.773511409759521\\
3.8844747543335 0.794669091701508\\
3.90423941612244 0.807747960090637\\
3.91430234909058 0.811704456806183\\
3.92453002929688 0.814187109470367\\
3.94485449790955 0.815711617469788\\
3.95557594299316 0.81569117307663\\
3.96640586853027 0.816253483295441\\
3.97732615470886 0.817738950252533\\
3.98829221725464 0.819298684597015\\
3.99927663803101 0.8199343085289\\
4.0102915763855 0.819298028945923\\
4.0213623046875 0.817460536956787\\
4.03252840042114 0.814645290374756\\
4.04375171661377 0.811136901378632\\
4.05504465103149 0.807148456573486\\
4.06652069091797 0.802746951580048\\
4.07826662063599 0.797543168067932\\
4.09013366699219 0.791480481624603\\
4.10201740264893 0.78456175327301\\
4.11389303207397 0.776817560195923\\
4.12580823898315 0.768294155597687\\
4.13775444030762 0.759063065052032\\
4.14973115921021 0.749175131320953\\
4.16165590286255 0.73876017332077\\
4.17351150512695 0.727992713451386\\
4.18527746200562 0.717156291007996\\
4.1969780921936 0.706516265869141\\
4.20862865447998 0.696336090564728\\
4.22023868560791 0.686873137950897\\
4.23181676864624 0.678363740444183\\
4.24337244033813 0.67101776599884\\
4.25489950180054 0.666756629943848\\
4.26636028289795 0.665884077548981\\
4.27764844894409 0.668061435222626\\
4.28874731063843 0.672789573669434\\
4.2996621131897 0.679592192173004\\
4.31040906906128 0.688120186328888\\
4.32100534439087 0.698228776454926\\
4.33146572113037 0.70968621969223\\
4.34404468536377 0.723746836185455\\
4.3592643737793 0.741794586181641\\
4.37768077850342 0.763797521591187\\
4.38775014877319 0.774831235408783\\
4.39779615402222 0.784202814102173\\
4.40788316726685 0.791758418083191\\
4.41804885864258 0.797546446323395\\
4.42816925048828 0.801643908023834\\
4.4384503364563 0.804276525974274\\
4.44892454147339 0.805598735809326\\
4.45952606201172 0.805891096591949\\
4.47025203704834 0.805743455886841\\
4.48108243942261 0.806086242198944\\
4.49192428588867 0.807396173477173\\
4.50274896621704 0.808818697929382\\
4.51354837417603 0.809278309345245\\
4.52435827255249 0.808516025543213\\
4.53524160385132 0.806708753108978\\
4.54612350463867 0.804100334644318\\
4.55704927444458 0.800905585289001\\
4.56821393966675 0.797218203544617\\
4.57975006103516 0.793026804924011\\
4.59155225753784 0.787966787815094\\
4.60341024398804 0.782008111476898\\
4.61525440216064 0.775165975093842\\
4.62709808349609 0.767480492591858\\
4.63897943496704 0.759010791778564\\
4.65090799331665 0.749877393245697\\
4.66293478012085 0.740144550800323\\
4.67511367797852 0.729819655418396\\
4.68726539611816 0.71910697221756\\
4.69932460784912 0.70824122428894\\
4.71125364303589 0.697589337825775\\
4.72304391860962 0.687602460384369\\
4.73469972610474 0.678761661052704\\
4.74622631072998 0.671527147293091\\
4.75763416290283 0.668424010276794\\
4.76885795593262 0.669960260391235\\
4.77983808517456 0.675454914569855\\
4.79057884216309 0.683999061584473\\
4.80111217498779 0.694799602031708\\
4.81147193908691 0.707173705101013\\
4.82168626785278 0.720496892929077\\
4.83178567886353 0.734402894973755\\
4.84179449081421 0.748568773269653\\
4.85605573654175 0.767548382282257\\
4.87331199645996 0.786561846733093\\
4.89321565628052 0.800960421562195\\
4.90325212478638 0.805472612380981\\
4.91343593597412 0.808429837226868\\
4.92376661300659 0.809934735298157\\
4.93423652648926 0.810089707374573\\
4.94437646865845 0.809108793735504\\
4.96500968933105 0.804499328136444\\
4.97588157653809 0.801755487918854\\
4.9868688583374 0.800155282020569\\
5 0.800595223903656\\
};
\addlegendentry{$\dtmax{} = \expneg{1.0}{1}{[s]}$};

\addplot [
color=black,
dash pattern=on 1pt off 3pt on 3pt off 3pt
]
table[row sep=crcr]{
0 1\\
0.0118019115179777 1\\
0.0226854234933853 1\\
0.0340500771999359 1\\
0.0491764321923256 1\\
0.0619497485458851 1\\
0.0774054601788521 1\\
0.0961068719625473 1\\
0.106106869876385 1\\
0.116106867790222 1\\
0.12610687315464 1\\
0.136106878519058 1\\
0.146106868982315 1\\
0.156106874346733 1\\
0.16610686480999 1\\
0.176106870174408 1\\
0.186106875538826 1\\
0.196106866002083 1\\
0.206106871366501 1\\
0.216106876730919 1\\
0.226106867194176 1\\
0.236106872558594 1\\
0.246106877923012 1\\
0.25610688328743 1\\
0.266106873750687 1\\
0.276106864213943 1\\
0.286106884479523 1\\
0.29610687494278 1\\
0.306106865406036 1\\
0.316106885671616 1\\
0.326106876134872 1\\
0.336106866598129 1\\
0.346106857061386 1\\
0.356106877326965 1\\
0.366106867790222 1\\
0.376106858253479 1\\
0.386106878519058 1\\
0.396106868982315 1\\
0.406106859445572 1\\
0.416106879711151 1\\
0.426106870174408 1\\
0.436106860637665 1\\
0.446106880903244 1\\
0.456106871366501 1\\
0.466106861829758 1\\
0.476106882095337 1\\
0.486106872558594 1\\
0.496106863021851 1\\
0.506106853485107 1\\
0.516106843948364 1\\
0.526106894016266 1\\
0.536106884479523 1\\
0.54610687494278 1\\
0.556106865406036 1\\
0.566106855869293 1\\
0.57610684633255 1\\
0.586106896400452 1\\
0.596106886863709 1\\
0.606106877326965 1\\
0.616106867790222 1\\
0.626106858253479 1\\
0.636106848716736 1\\
0.646106898784637 1\\
0.656106889247894 1\\
0.666106879711151 1\\
0.676106870174408 1\\
0.686106860637665 1\\
0.696106851100922 1\\
0.706106901168823 1\\
0.71610689163208 1\\
0.726106882095337 1\\
0.736106872558594 1\\
0.746106863021851 1\\
0.756106853485107 1\\
0.766106843948364 1\\
0.776106894016266 1\\
0.786106884479523 1\\
0.79610687494278 1\\
0.806106865406036 1\\
0.816106855869293 1\\
0.82610684633255 1\\
0.836106896400452 1\\
0.846106886863709 1\\
0.856106877326965 1\\
0.866106867790222 1\\
0.876106858253479 1\\
0.886106848716736 1\\
0.896106898784637 1\\
0.906106889247894 1\\
0.916106879711151 1\\
0.926106870174408 1\\
0.936106860637665 1\\
0.946106851100922 1\\
0.956106901168823 1\\
0.96610689163208 1\\
0.976106882095337 1\\
0.986106872558594 1\\
0.996106863021851 1\\
1.00610685348511 0.999956965446472\\
1.01610684394836 0.999844312667847\\
1.02610683441162 0.999650895595551\\
1.03610682487488 0.999364256858826\\
1.04610681533813 0.999102711677551\\
1.05610692501068 0.998735845088959\\
1.06610691547394 0.998294591903687\\
1.07610690593719 0.997780919075012\\
1.08610689640045 0.997193872928619\\
1.09610688686371 0.996532380580902\\
1.10610687732697 0.99579530954361\\
1.11610686779022 0.994981467723846\\
1.12610685825348 0.994089782238007\\
1.13610684871674 0.993119180202484\\
1.14610683917999 0.992068588733673\\
1.15610682964325 0.990937054157257\\
1.16610682010651 0.989723563194275\\
1.17610692977905 0.98842716217041\\
1.18610692024231 0.987046897411346\\
1.19610691070557 0.985581874847412\\
1.20610690116882 0.984031200408936\\
1.21610689163208 0.982393980026245\\
1.22610688209534 0.980669438838959\\
1.23610687255859 0.97885674238205\\
1.24610686302185 0.976970553398132\\
1.25610685348511 0.975025713443756\\
1.26610684394836 0.973024845123291\\
1.27610683441162 0.970966815948486\\
1.28610682487488 0.96884959936142\\
1.29610681533813 0.966671526432037\\
1.30610692501068 0.964431345462799\\
1.31610691547394 0.962128043174744\\
1.32610690593719 0.959761023521423\\
1.33610689640045 0.957329988479614\\
1.34610688686371 0.954834640026093\\
1.35610687732697 0.952274918556213\\
1.36610686779022 0.949650704860687\\
1.37610685825348 0.946962058544159\\
1.38610684871674 0.944208979606628\\
1.39610683917999 0.94139152765274\\
1.40610682964325 0.938509702682495\\
1.41610682010651 0.935563623905182\\
1.42610692977905 0.932553350925446\\
1.43610692024231 0.929478883743286\\
1.44610691070557 0.926340341567993\\
1.45610690116882 0.923137664794922\\
1.46610689163208 0.919871032238007\\
1.47610688209534 0.916540324687958\\
1.48610687255859 0.91314572095871\\
1.49610686302185 0.909687161445618\\
1.50610685348511 0.906164646148682\\
1.51610684394836 0.902578234672546\\
1.52610683441162 0.898927927017212\\
1.53610682487488 0.895213663578033\\
1.54610681533813 0.891435563564301\\
1.55610692501068 0.887593448162079\\
1.56610691547394 0.883687496185303\\
1.57610690593719 0.879717528820038\\
1.58610689640045 0.875683605670929\\
1.59610688686371 0.871585667133331\\
1.60610687732697 0.86742377281189\\
1.61610686779022 0.863197803497314\\
1.62610685825348 0.85890781879425\\
1.63610684871674 0.854553639888763\\
1.64610683917999 0.850135445594788\\
1.65610682964325 0.845652937889099\\
1.66610682010651 0.841106355190277\\
1.67610692977905 0.836495339870453\\
1.68610692024231 0.83182018995285\\
1.69610691070557 0.827080547809601\\
1.70610690116882 0.822276651859283\\
1.71610689163208 0.817408204078674\\
1.72610688209534 0.812475323677063\\
1.73610687255859 0.807477831840515\\
1.74610686302185 0.80241584777832\\
1.75610685348511 0.79728901386261\\
1.76610684394836 0.79212361574173\\
1.77610683441162 0.786948323249817\\
1.78610682487488 0.781767070293427\\
1.79610681533813 0.776574015617371\\
1.80610692501068 0.771362125873566\\
1.81610691547394 0.766123354434967\\
1.82610690593719 0.760851204395294\\
1.83610689640045 0.755539238452911\\
1.84610688686371 0.750183165073395\\
1.85610687732697 0.744778275489807\\
1.86610686779022 0.739321291446686\\
1.87610685825348 0.733808696269989\\
1.88610684871674 0.728238642215729\\
1.89610683917999 0.722608327865601\\
1.90610682964325 0.716916382312775\\
1.91610682010651 0.711160659790039\\
1.92610692977905 0.705340147018433\\
1.93610692024231 0.699453175067902\\
1.94610691070557 0.693498730659485\\
1.95610690116882 0.688878417015076\\
1.96610689163208 0.686013281345367\\
1.97610688209534 0.685046792030334\\
1.98610687255859 0.685810148715973\\
1.99610686302185 0.687853157520294\\
2.00610685348511 0.690696775913239\\
2.01610684394836 0.693881452083588\\
2.02610683441162 0.697031676769257\\
2.03610682487488 0.699911773204803\\
2.04610681533813 0.702396750450134\\
2.05610680580139 0.704437375068665\\
2.06610679626465 0.706028699874878\\
2.07610678672791 0.707190573215485\\
2.08610677719116 0.70796012878418\\
2.09610676765442 0.70842307806015\\
2.10610675811768 0.708731114864349\\
2.11610698699951 0.709120392799377\\
2.12610697746277 0.710074186325073\\
2.13610696792603 0.712531864643097\\
2.14610695838928 0.717687129974365\\
2.15610694885254 0.725770473480225\\
2.1661069393158 0.735181629657745\\
2.17610692977905 0.743984639644623\\
2.18610692024231 0.751362264156342\\
2.20580863952637 0.762390494346619\\
2.21580862998962 0.76704865694046\\
2.22580862045288 0.771529853343964\\
2.23580861091614 0.775954365730286\\
2.24580860137939 0.780333817005157\\
2.25580859184265 0.784617364406586\\
2.26580858230591 0.788728713989258\\
2.27580857276917 0.792590379714966\\
2.28580856323242 0.79613471031189\\
2.30537819862366 0.801970720291138\\
2.31537818908691 0.8043332695961\\
2.32537817955017 0.806260168552399\\
2.33537817001343 0.807749569416046\\
2.35460209846497 0.809431731700897\\
2.36516523361206 0.809803783893585\\
2.3773078918457 0.809759616851807\\
2.39200067520142 0.809092998504639\\
2.40977883338928 0.807324230670929\\
2.41977882385254 0.805356740951538\\
2.4297788143158 0.802433729171753\\
2.43977880477905 0.798523426055908\\
2.44977879524231 0.793653249740601\\
2.45977878570557 0.78788959980011\\
2.46977877616882 0.781322777271271\\
2.47977900505066 0.774055361747742\\
2.48977899551392 0.766194820404053\\
2.49977898597717 0.757847905158997\\
2.50977897644043 0.749117851257324\\
2.51977896690369 0.740102171897888\\
2.52977895736694 0.730891168117523\\
2.5397789478302 0.721573054790497\\
2.54977893829346 0.712226033210754\\
2.55977892875671 0.702919542789459\\
2.56977891921997 0.693714678287506\\
2.57977890968323 0.684664607048035\\
2.58977890014648 0.675815641880035\\
2.59977889060974 0.6672243475914\\
2.609778881073 0.659029841423035\\
2.61977887153625 0.651364207267761\\
2.62977886199951 0.644376993179321\\
2.63977885246277 0.638965427875519\\
2.64977884292603 0.637212872505188\\
2.65977883338928 0.639246344566345\\
2.66977882385254 0.644522905349731\\
2.6797788143158 0.652118504047394\\
2.69397473335266 0.664411604404449\\
2.71014857292175 0.679757535457611\\
2.72946786880493 0.698830008506775\\
2.73946785926819 0.708608984947205\\
2.74946784973145 0.718158185482025\\
2.7594678401947 0.727396726608276\\
2.76946783065796 0.736265420913696\\
2.78308033943176 0.747479975223541\\
2.79642605781555 0.757693529129028\\
2.81257438659668 0.769148290157318\\
2.83187890052795 0.781566798686981\\
2.85093808174133 0.792942047119141\\
2.86093807220459 0.799354195594788\\
2.87093806266785 0.80668169260025\\
2.8809380531311 0.814186632633209\\
2.89093804359436 0.820585906505585\\
2.9009382724762 0.825098633766174\\
2.91093826293945 0.827679574489594\\
2.92093825340271 0.828702688217163\\
2.93093824386597 0.828658044338226\\
2.94093823432922 0.827964246273041\\
2.95093822479248 0.826874732971191\\
2.96093821525574 0.825526833534241\\
2.97093820571899 0.823988378047943\\
2.98093819618225 0.822294950485229\\
2.99093818664551 0.820473611354828\\
3.00093817710876 0.818543136119843\\
3.01093816757202 0.816509068012238\\
3.02093815803528 0.814280390739441\\
3.03093814849854 0.811753571033478\\
3.04093813896179 0.808503210544586\\
3.05093812942505 0.804398715496063\\
3.06093811988831 0.799385905265808\\
3.07093811035156 0.793475210666656\\
3.08093810081482 0.786719083786011\\
3.09093809127808 0.779196500778198\\
3.10093808174133 0.77099996805191\\
3.11093807220459 0.762227535247803\\
3.12093806266785 0.752978444099426\\
3.1309380531311 0.743351519107819\\
3.14093804359436 0.733451008796692\\
3.1509382724762 0.72339141368866\\
3.16093826293945 0.71329653263092\\
3.17093825340271 0.70329761505127\\
3.18093824386597 0.69353461265564\\
3.19093823432922 0.684153318405151\\
3.20093822479248 0.675312578678131\\
3.21093821525574 0.667177855968475\\
3.22093820571899 0.659914374351501\\
3.23093819618225 0.654005706310272\\
3.24093818664551 0.650746464729309\\
3.25093817710876 0.650305449962616\\
3.26093816757202 0.652706801891327\\
3.27093815803528 0.65788596868515\\
3.28093814849854 0.66572093963623\\
3.29093813896179 0.676148355007172\\
3.30093812942505 0.68905383348465\\
3.31571698188782 0.710605263710022\\
3.33297657966614 0.738640487194061\\
3.35262060165405 0.769679307937622\\
3.37212371826172 0.792487025260925\\
3.39176940917969 0.806650340557098\\
3.40641617774963 0.812799870967865\\
3.4179515838623 0.815607488155365\\
3.43190956115723 0.81671804189682\\
3.44879865646362 0.81596714258194\\
3.46852993965149 0.81590348482132\\
3.47852993011475 0.816891551017761\\
3.488529920578 0.81759387254715\\
3.49852991104126 0.817313194274902\\
3.50852990150452 0.81586217880249\\
3.51852989196777 0.813408255577087\\
3.52852988243103 0.810251891613007\\
3.53852987289429 0.806672990322113\\
3.54852986335754 0.802799463272095\\
3.5585298538208 0.798711895942688\\
3.56852984428406 0.794169247150421\\
3.57852983474731 0.789050340652466\\
3.58852982521057 0.783281743526459\\
3.59852981567383 0.776849329471588\\
3.60852980613709 0.769788086414337\\
3.61852979660034 0.762165725231171\\
3.62853002548218 0.75407886505127\\
3.63853001594543 0.745622575283051\\
3.64853000640869 0.736882925033569\\
3.65852999687195 0.727932155132294\\
3.66852998733521 0.718838036060333\\
3.67852997779846 0.709664642810822\\
3.68852996826172 0.70048463344574\\
3.69852995872498 0.691396176815033\\
3.70852994918823 0.682524919509888\\
3.71852993965149 0.674022018909454\\
3.72852993011475 0.666045904159546\\
3.738529920578 0.658757925033569\\
3.74852991104126 0.652327477931976\\
3.75852990150452 0.648453831672668\\
3.76852989196777 0.64752334356308\\
3.77852988243103 0.649580836296082\\
3.78852987289429 0.654496192932129\\
3.79852986335754 0.662030577659607\\
3.8085298538208 0.67190158367157\\
3.81852984428406 0.683942794799805\\
3.82852983474731 0.697884976863861\\
3.84355807304382 0.720107853412628\\
3.85999274253845 0.745204627513886\\
3.87921619415283 0.772725164890289\\
3.89872860908508 0.793408215045929\\
3.91853308677673 0.80622935295105\\
3.92853307723999 0.8100705742836\\
3.93853306770325 0.812432169914246\\
3.94853329658508 0.813512802124023\\
3.96640586853027 0.813192844390869\\
3.97640585899353 0.812332332134247\\
3.98640584945679 0.81198662519455\\
3.99640583992004 0.812609910964966\\
4.0064058303833 0.813860952854156\\
4.01640605926514 0.814874351024628\\
4.02640581130981 0.815016448497772\\
4.03640604019165 0.814122557640076\\
4.04640579223633 0.812283396720886\\
4.05640602111816 0.80969625711441\\
4.06640577316284 0.80657833814621\\
4.07640600204468 0.803087413311005\\
4.08640575408936 0.799334645271301\\
4.09640598297119 0.795072793960571\\
4.10640573501587 0.790241003036499\\
4.11640596389771 0.784802556037903\\
4.12640619277954 0.778752207756042\\
4.13640594482422 0.772121012210846\\
4.14640617370605 0.764955699443817\\
4.15640592575073 0.757294893264771\\
4.16640615463257 0.749167501926422\\
4.17640590667725 0.740612804889679\\
4.18640613555908 0.731719255447388\\
4.19640588760376 0.722633957862854\\
4.2064061164856 0.713544487953186\\
4.21640586853027 0.704649925231934\\
4.22640609741211 0.69614839553833\\
4.23640584945679 0.688233196735382\\
4.24640607833862 0.681093871593475\\
4.2564058303833 0.674999237060547\\
4.26640605926514 0.671409845352173\\
4.27640581130981 0.670398354530334\\
4.28640604019165 0.671854138374329\\
4.29640579223633 0.675595462322235\\
4.30640602111816 0.681405663490295\\
4.31640577316284 0.689053297042847\\
4.32640600204468 0.698309004306793\\
4.33640575408936 0.709011077880859\\
4.34640598297119 0.72091281414032\\
4.36180782318115 0.739965200424194\\
4.38015365600586 0.763446688652039\\
4.39015340805054 0.775131523609161\\
4.40015363693237 0.784979164600372\\
4.41015338897705 0.792850732803345\\
4.42015361785889 0.798846364021301\\
4.43015336990356 0.803141295909882\\
4.4401535987854 0.805925250053406\\
4.45015335083008 0.807398319244385\\
4.46015357971191 0.807812929153442\\
4.47015333175659 0.807574093341827\\
4.48015356063843 0.807344913482666\\
4.49015331268311 0.807849526405334\\
4.50015354156494 0.809225559234619\\
4.51015329360962 0.810811698436737\\
4.52015352249146 0.811743021011353\\
4.53015327453613 0.811646521091461\\
4.54015350341797 0.81048971414566\\
4.55015325546265 0.808405816555023\\
4.56015348434448 0.805594444274902\\
4.57015323638916 0.80226081609726\\
4.580153465271 0.798520267009735\\
4.59015321731567 0.794547736644745\\
4.60015344619751 0.790102958679199\\
4.61015367507935 0.785140573978424\\
4.62015342712402 0.779634475708008\\
4.63015365600586 0.773586273193359\\
4.64015340805054 0.767021834850311\\
4.65015363693237 0.759997487068176\\
4.66015338897705 0.752563536167145\\
4.67015361785889 0.744746804237366\\
4.68015336990356 0.736557006835938\\
4.6901535987854 0.728027641773224\\
4.70015335083008 0.719255805015564\\
4.71015357971191 0.710406720638275\\
4.72015333175659 0.701692521572113\\
4.73015356063843 0.693352699279785\\
4.74015331268311 0.685628116130829\\
4.75015354156494 0.678747177124023\\
4.76015329360962 0.673375725746155\\
4.77015352249146 0.670919716358185\\
4.78015327453613 0.671354174613953\\
4.79015350341797 0.67444121837616\\
4.80015325546265 0.679882228374481\\
4.81015348434448 0.687357306480408\\
4.82015323638916 0.696516096591949\\
4.830153465271 0.7070352435112\\
4.84015321731567 0.718628644943237\\
4.85015344619751 0.73105251789093\\
4.86591100692749 0.750853538513184\\
4.88213491439819 0.770575642585754\\
4.90140724182129 0.78909558057785\\
4.92139911651611 0.800952315330505\\
4.93139886856079 0.804397940635681\\
4.94139909744263 0.80641770362854\\
4.9513988494873 0.807182729244232\\
4.96139907836914 0.806907296180725\\
4.97139883041382 0.805934011936188\\
4.98139905929565 0.804864048957825\\
5 0.805012345314026\\
};
\addlegendentry{$\dtmax{} = \expneg{1.0}{2}{[s]}$};

\addplot [
color=black,
dotted
]
table[row sep=crcr]{
0 1\\
0.0109122171998024 1\\
0.020912216976285 1\\
0.0309122167527676 1\\
0.0409122183918953 1\\
0.0509122163057327 1\\
0.0609122179448605 1\\
0.0709122195839882 1\\
0.0809122174978256 1\\
0.0909122154116631 1\\
0.100912220776081 1\\
0.110912218689919 1\\
0.120912216603756 1\\
0.130912214517593 1\\
0.140912219882011 1\\
0.150912210345268 1\\
0.160912215709686 1\\
0.170912221074104 1\\
0.180912211537361 1\\
0.190912216901779 1\\
0.200912222266197 1\\
0.210912212729454 1\\
0.220912218093872 1\\
0.23091222345829 1\\
0.240912213921547 1\\
0.250912219285965 1\\
0.260912209749222 1\\
0.270912230014801 1\\
0.280912220478058 1\\
0.290912210941315 1\\
0.300912231206894 1\\
0.310912221670151 1\\
0.320912212133408 1\\
0.330912202596664 1\\
0.340912222862244 1\\
0.3509122133255 1\\
0.360912203788757 1\\
0.370912224054337 1\\
0.380912214517593 1\\
0.39091220498085 1\\
0.400912225246429 1\\
0.410912215709686 1\\
0.420912206172943 1\\
0.430912226438522 1\\
0.440912216901779 1\\
0.450912207365036 1\\
0.460912227630615 1\\
0.470912218093872 1\\
0.480912208557129 1\\
0.490912228822708 1\\
0.500912189483643 1\\
0.510912239551544 1\\
0.520912230014801 1\\
0.530912220478058 1\\
0.540912210941315 1\\
0.550912201404572 1\\
0.560912191867828 1\\
0.57091224193573 1\\
0.580912232398987 1\\
0.590912222862244 1\\
0.6009122133255 1\\
0.610912203788757 1\\
0.620912194252014 1\\
0.630912244319916 1\\
0.640912234783173 1\\
0.650912225246429 1\\
0.660912215709686 1\\
0.670912206172943 1\\
0.6809121966362 1\\
0.690912246704102 1\\
0.700912237167358 1\\
0.710912227630615 1\\
0.720912218093872 1\\
0.730912208557129 1\\
0.740912199020386 1\\
0.750912189483643 1\\
0.760912239551544 1\\
0.770912230014801 1\\
0.780912220478058 1\\
0.790912210941315 1\\
0.800912201404572 1\\
0.810912191867828 1\\
0.82091224193573 1\\
0.830912232398987 1\\
0.840912222862244 1\\
0.8509122133255 1\\
0.860912203788757 1\\
0.870912194252014 1\\
0.880912244319916 1\\
0.890912234783173 1\\
0.900912225246429 1\\
0.910912215709686 1\\
0.920912206172943 1\\
0.9309121966362 1\\
0.940912246704102 1\\
0.950912237167358 1\\
0.960912227630615 1\\
0.970912218093872 1\\
0.980912208557129 1\\
0.990912199020386 1\\
1.00091218948364 0.999999403953552\\
1.0109121799469 0.999951004981995\\
1.02091217041016 0.999801218509674\\
1.03091216087341 0.999619722366333\\
1.04091227054596 0.99938303232193\\
1.05091226100922 0.999078094959259\\
1.06091225147247 0.998703598976135\\
1.07091224193573 0.998258233070374\\
1.08091223239899 0.997740685939789\\
1.09091222286224 0.997149765491486\\
1.1009122133255 0.996484279632568\\
1.11091220378876 0.995742976665497\\
1.12091219425201 0.99492484331131\\
1.13091218471527 0.994028747081757\\
1.14091217517853 0.993053555488586\\
1.15091216564178 0.991998314857483\\
1.16091227531433 0.990862011909485\\
1.17091226577759 0.989643633365631\\
1.18091225624084 0.988342225551605\\
1.1909122467041 0.986956894397736\\
1.20091223716736 0.985486745834351\\
1.21091222763062 0.983930826187134\\
1.22091221809387 0.982288360595703\\
1.23091220855713 0.980558454990387\\
1.24091219902039 0.978742241859436\\
1.25091218948364 0.976857841014862\\
1.2609121799469 0.974917590618134\\
1.27091217041016 0.972922563552856\\
1.28091216087341 0.970870435237885\\
1.29091227054596 0.968758881092072\\
1.30091226100922 0.966585874557495\\
1.31091225147247 0.964350044727325\\
1.32091224193573 0.962050497531891\\
1.33091223239899 0.959686756134033\\
1.34091222286224 0.957258403301239\\
1.3509122133255 0.954765439033508\\
1.36091220378876 0.952207684516907\\
1.37091219425201 0.949585139751434\\
1.38091218471527 0.946897983551025\\
1.39091217517853 0.94414609670639\\
1.40091216564178 0.941329717636108\\
1.41091227531433 0.938448846340179\\
1.42091226577759 0.935503602027893\\
1.43091225624084 0.932493984699249\\
1.4409122467041 0.929420113563538\\
1.45091223716736 0.926282048225403\\
1.46091222763062 0.923079788684845\\
1.47091221809387 0.919813454151154\\
1.48091220855713 0.916483104228973\\
1.49091219902039 0.913088619709015\\
1.50091218948364 0.909630179405212\\
1.5109121799469 0.906107723712921\\
1.52091217041016 0.902521312236786\\
1.53091216087341 0.898870885372162\\
1.54091227054596 0.895156562328339\\
1.55091226100922 0.891378223896027\\
1.56091225147247 0.887535870075226\\
1.57091224193573 0.883629560470581\\
1.58091223239899 0.879659295082092\\
1.59091222286224 0.87562495470047\\
1.6009122133255 0.871526598930359\\
1.61091220378876 0.867364168167114\\
1.62091219425201 0.863137662410736\\
1.63091218471527 0.85884702205658\\
1.64091217517853 0.854492247104645\\
1.65091216564178 0.850073218345642\\
1.66091227531433 0.845590054988861\\
1.67091226577759 0.841042578220367\\
1.68091225624084 0.836430788040161\\
1.6909122467041 0.831754684448242\\
1.70091223716736 0.827014148235321\\
1.71091222763062 0.822209239006042\\
1.72091221809387 0.817339777946472\\
1.73091220855713 0.812405824661255\\
1.74091219902039 0.807407259941101\\
1.75091218948364 0.802344024181366\\
1.7609121799469 0.797218143939972\\
1.77091217041016 0.792059898376465\\
1.78091216087341 0.786895573139191\\
1.79091227054596 0.781727075576782\\
1.80091226100922 0.776547908782959\\
1.81091225147247 0.771349489688873\\
1.82091224193573 0.766123652458191\\
1.83091223239899 0.76086300611496\\
1.84091222286224 0.755561470985413\\
1.8509122133255 0.750213861465454\\
1.86091220378876 0.74481600522995\\
1.87091219425201 0.739364266395569\\
1.88091218471527 0.733855724334717\\
1.89091217517853 0.728287994861603\\
1.90091216564178 0.72265899181366\\
1.91091227531433 0.716966867446899\\
1.92091226577759 0.711210131645203\\
1.93091225624084 0.705387473106384\\
1.9409122467041 0.699497580528259\\
1.95091223716736 0.693539321422577\\
1.96091222763062 0.688261330127716\\
1.97091221809387 0.684676289558411\\
1.98091220855713 0.68313455581665\\
1.99091219902039 0.683633327484131\\
2.00091218948364 0.685791969299316\\
2.0109121799469 0.689072847366333\\
2.02091217041016 0.692882001399994\\
2.03091216087341 0.696716547012329\\
2.04091215133667 0.700252950191498\\
2.05091214179993 0.703319549560547\\
2.06091213226318 0.705849945545197\\
2.07091212272644 0.707840502262115\\
2.0809121131897 0.709323585033417\\
2.09091210365295 0.710365235805511\\
2.10091233253479 0.71111661195755\\
2.11091232299805 0.711830615997314\\
2.1209123134613 0.712979018688202\\
2.13091230392456 0.715432405471802\\
2.14091229438782 0.720313191413879\\
2.15091228485107 0.728042781352997\\
2.16091227531433 0.737618863582611\\
2.17091226577759 0.747379541397095\\
2.18091225624084 0.756090223789215\\
2.1909122467041 0.763356268405914\\
2.20091223716736 0.769346177577972\\
2.21091222763062 0.774413049221039\\
2.22091221809387 0.778878509998322\\
2.23091220855713 0.782964944839478\\
2.24091219902039 0.786796092987061\\
2.25091218948364 0.790422022342682\\
2.2609121799469 0.793846428394318\\
2.27091217041016 0.797047972679138\\
2.28091216087341 0.79999440908432\\
2.29091215133667 0.80265086889267\\
2.30091214179993 0.804982662200928\\
2.31091213226318 0.806957900524139\\
2.32091212272644 0.808556258678436\\
2.3309121131897 0.809782445430756\\
2.34091210365295 0.81066220998764\\
2.35091233253479 0.811224102973938\\
2.36091232299805 0.81148886680603\\
2.3709123134613 0.811468183994293\\
2.38091230392456 0.811168253421783\\
2.39091229438782 0.810576319694519\\
2.40091228485107 0.809367299079895\\
2.41091227531433 0.807311832904816\\
2.42091226577759 0.804319024085999\\
2.43091225624084 0.80035924911499\\
2.4409122467041 0.795454502105713\\
2.45091223716736 0.789664506912231\\
2.46091222763062 0.783074140548706\\
2.47091221809387 0.775783538818359\\
2.48091220855713 0.76789927482605\\
2.49091219902039 0.759528577327728\\
2.50091218948364 0.750774919986725\\
2.5109121799469 0.741735994815826\\
2.52091217041016 0.732505083084106\\
2.53091216087341 0.723168313503265\\
2.54091215133667 0.713801383972168\\
2.55091214179993 0.704470634460449\\
2.56091213226318 0.695233166217804\\
2.57091212272644 0.686137795448303\\
2.5809121131897 0.677226960659027\\
2.59091210365295 0.668567776679993\\
2.60091233253479 0.660272777080536\\
2.61091232299805 0.65245521068573\\
2.6209123134613 0.645246148109436\\
2.63091230392456 0.639041662216187\\
2.64091229438782 0.636090457439423\\
2.65091228485107 0.637251973152161\\
2.66091227531433 0.642246007919312\\
2.67091226577759 0.650151193141937\\
2.68091225624084 0.659752190113068\\
2.6909122467041 0.669918417930603\\
2.70091223716736 0.680207252502441\\
2.71091222763062 0.69042581319809\\
2.72091221809387 0.70045417547226\\
2.73091220855713 0.710199534893036\\
2.74091219902039 0.719590365886688\\
2.75091218948364 0.72857403755188\\
2.7609121799469 0.737115502357483\\
2.77091217041016 0.745195627212524\\
2.78091216087341 0.752812802791595\\
2.79091215133667 0.760003983974457\\
2.80091214179993 0.76685905456543\\
2.81091213226318 0.773537576198578\\
2.82091212272644 0.780348539352417\\
2.8309121131897 0.787751853466034\\
2.84091210365295 0.796122550964355\\
2.85091233253479 0.805265367031097\\
2.86091232299805 0.814260721206665\\
2.8709123134613 0.822083175182343\\
2.88091230392456 0.828176319599152\\
2.89091229438782 0.832477152347565\\
2.90091228485107 0.835188329219818\\
2.91091227531433 0.836589634418488\\
2.92091226577759 0.836945414543152\\
2.93091225624084 0.836475133895874\\
2.9409122467041 0.835353910923004\\
2.95091223716736 0.833726823329926\\
2.96091222763062 0.831713676452637\\
2.97091221809387 0.829408824443817\\
2.98091220855713 0.82688307762146\\
2.99091219902039 0.824180543422699\\
3.00091218948364 0.821102619171143\\
3.0109121799469 0.817370533943176\\
3.02091217041016 0.812869310379028\\
3.03091216087341 0.807524681091309\\
3.04091215133667 0.801318645477295\\
3.05091214179993 0.794282615184784\\
3.06091213226318 0.786482155323029\\
3.07091212272644 0.77800452709198\\
3.0809121131897 0.768946826457977\\
3.09091210365295 0.759409487247467\\
3.10091233253479 0.749497413635254\\
3.11091232299805 0.739310681819916\\
3.1209123134613 0.728944659233093\\
3.13091230392456 0.71849799156189\\
3.14091229438782 0.708077251911163\\
3.15091228485107 0.697797119617462\\
3.16091227531433 0.687777698040009\\
3.17091226577759 0.678144693374634\\
3.18091225624084 0.669030487537384\\
3.1909122467041 0.66057550907135\\
3.20091223716736 0.652926743030548\\
3.21091222763062 0.646330118179321\\
3.22091221809387 0.642163872718811\\
3.23091220855713 0.640998423099518\\
3.24091219902039 0.64290452003479\\
3.25091218948364 0.647839605808258\\
3.2609121799469 0.655662417411804\\
3.27091217041016 0.666186630725861\\
3.28091216087341 0.679280638694763\\
3.29091215133667 0.694531917572021\\
3.30091214179993 0.711003422737122\\
3.31091213226318 0.727971911430359\\
3.32091212272644 0.744828283786774\\
3.3309121131897 0.761016726493835\\
3.34091210365295 0.775664687156677\\
3.35091233253479 0.788127362728119\\
3.36091232299805 0.798281013965607\\
3.3709123134613 0.806279122829437\\
3.38091230392456 0.812374949455261\\
3.39091229438782 0.816839754581451\\
3.40091228485107 0.820004522800446\\
3.41091227531433 0.822325766086578\\
3.42091226577759 0.824423730373383\\
3.43091225624084 0.826865673065186\\
3.4409122467041 0.829745173454285\\
3.45091223716736 0.832590341567993\\
3.46091222763062 0.834736764431\\
3.47091221809387 0.835735559463501\\
3.48091220855713 0.835468173027039\\
3.49091219902039 0.83406013250351\\
3.50091218948364 0.831749856472015\\
3.5109121799469 0.828769266605377\\
3.52091217041016 0.825304627418518\\
3.53091216087341 0.82144021987915\\
3.54091215133667 0.816978633403778\\
3.55091214179993 0.811868607997894\\
3.56091213226318 0.806085169315338\\
3.57091212272644 0.799628078937531\\
3.5809121131897 0.792529404163361\\
3.59091210365295 0.784848511219025\\
3.60091233253479 0.776669323444366\\
3.61091232299805 0.768085777759552\\
3.6209123134613 0.759184896945953\\
3.63091230392456 0.75004118680954\\
3.64091229438782 0.7407106757164\\
3.65091228485107 0.73123562335968\\
3.66091227531433 0.72165858745575\\
3.67091226577759 0.712039709091187\\
3.68091225624084 0.702459394931793\\
3.6909122467041 0.693011462688446\\
3.70091223716736 0.683795988559723\\
3.71091222763062 0.674915850162506\\
3.72091221809387 0.666474461555481\\
3.73091220855713 0.658576607704163\\
3.74091219902039 0.651327311992645\\
3.75091218948364 0.64500480890274\\
3.7609121799469 0.64108270406723\\
3.77091217041016 0.639950692653656\\
3.78091216087341 0.641586542129517\\
3.79091215133667 0.645866692066193\\
3.80091214179993 0.652600467205048\\
3.81091213226318 0.661646246910095\\
3.82091212272644 0.673020362854004\\
3.8309121131897 0.686289608478546\\
3.84091210365295 0.700617015361786\\
3.85091233253479 0.715462625026703\\
3.86091232299805 0.7303666472435\\
3.8709123134613 0.744964122772217\\
3.88091230392456 0.758974373340607\\
3.89091229438782 0.77200973033905\\
3.90091228485107 0.783465564250946\\
3.91091227531433 0.793035209178925\\
3.92091226577759 0.800721347332001\\
3.93091225624084 0.806713283061981\\
3.9409122467041 0.81134021282196\\
3.95091223716736 0.815109252929688\\
3.96091222763062 0.818694710731506\\
3.97091221809387 0.82262521982193\\
3.98091220855713 0.826843559741974\\
3.99091219902039 0.830741703510284\\
4.00091218948364 0.833652138710022\\
4.01091241836548 0.835217416286469\\
4.02091217041016 0.835407972335815\\
4.03091239929199 0.834388434886932\\
4.04091215133667 0.832411885261536\\
4.05091238021851 0.829723954200745\\
4.06091213226318 0.826514422893524\\
4.07091236114502 0.822921276092529\\
4.0809121131897 0.818938374519348\\
4.09091234207153 0.814403533935547\\
4.10091209411621 0.809287130832672\\
4.11091232299805 0.803566098213196\\
4.12091207504272 0.797240078449249\\
4.13091230392456 0.790334463119507\\
4.14091205596924 0.782903075218201\\
4.15091228485107 0.775016605854034\\
4.16091203689575 0.766728520393372\\
4.17091226577759 0.75806999206543\\
4.18091201782227 0.749068975448608\\
4.1909122467041 0.739779233932495\\
4.20091199874878 0.730294346809387\\
4.21091222763062 0.720740973949432\\
4.22091197967529 0.711266160011292\\
4.23091220855713 0.702022254467011\\
4.24091243743896 0.69315642118454\\
4.25091218948364 0.684808015823364\\
4.26091241836548 0.677113354206085\\
4.27091217041016 0.670197010040283\\
4.28091239929199 0.664705753326416\\
4.29091215133667 0.661830246448517\\
4.30091238021851 0.661508798599243\\
4.31091213226318 0.663547217845917\\
4.32091236114502 0.66770988702774\\
4.3309121131897 0.673743903636932\\
4.34091234207153 0.681405484676361\\
4.35091209411621 0.690530240535736\\
4.36091232299805 0.701035737991333\\
4.37091207504272 0.712521851062775\\
4.38091230392456 0.724452614784241\\
4.39091205596924 0.736521184444427\\
4.40091228485107 0.748486042022705\\
4.41091203689575 0.760156214237213\\
4.42091226577759 0.771245837211609\\
4.43091201782227 0.781173348426819\\
4.4409122467041 0.78956013917923\\
4.45091199874878 0.796321749687195\\
4.46091222763062 0.801571071147919\\
4.47091197967529 0.805581986904144\\
4.48091220855713 0.808829784393311\\
4.49091243743896 0.811987340450287\\
4.50091218948364 0.815616488456726\\
4.51091241836548 0.819686412811279\\
4.52091217041016 0.823569774627686\\
4.53091239929199 0.82653284072876\\
4.54091215133667 0.828150391578674\\
4.55091238021851 0.828352749347687\\
4.56091213226318 0.827287495136261\\
4.57091236114502 0.825206398963928\\
4.5809121131897 0.822381734848022\\
4.59091234207153 0.819017887115479\\
4.60091209411621 0.815209329128265\\
4.61091232299805 0.810791432857513\\
4.62091207504272 0.805740714073181\\
4.63091230392456 0.800055325031281\\
4.64091205596924 0.793749213218689\\
4.65091228485107 0.786859154701233\\
4.66091203689575 0.77944141626358\\
4.67091226577759 0.771568834781647\\
4.68091201782227 0.763330519199371\\
4.6909122467041 0.754799783229828\\
4.70091199874878 0.746013581752777\\
4.71091222763062 0.736984014511108\\
4.72091197967529 0.727736711502075\\
4.73091220855713 0.718344151973724\\
4.74091243743896 0.708929300308228\\
4.75091218948364 0.699649810791016\\
4.76091241836548 0.690683662891388\\
4.77091217041016 0.682224154472351\\
4.78091239929199 0.674480795860291\\
4.79091215133667 0.667661845684052\\
4.80091238021851 0.662885665893555\\
4.81091213226318 0.661472976207733\\
4.82091236114502 0.663296341896057\\
4.8309121131897 0.668055832386017\\
4.84091234207153 0.675382316112518\\
4.85091209411621 0.684880673885345\\
4.86091232299805 0.69611793756485\\
4.87091207504272 0.708604276180267\\
4.88091230392456 0.72198748588562\\
4.89091205596924 0.735850989818573\\
4.90091228485107 0.749596655368805\\
4.91091203689575 0.762883424758911\\
4.92091226577759 0.77516371011734\\
4.93091201782227 0.785801708698273\\
4.9409122467041 0.794553577899933\\
4.95091199874878 0.80144214630127\\
4.96091222763062 0.806610763072968\\
4.97091197967529 0.810247600078583\\
4.98091220855713 0.812573254108429\\
4.99091243743896 0.813886106014252\\
5 0.81459504365921\\
};
\addlegendentry{$\dtmax{} = \expneg{1.0}{3}{[s]}$};

\addplot [
color=black,
dashed
]
table[row sep=crcr]{
0 1\\
0.0100556658580899 1\\
0.0200556665658951 1\\
0.0300556663423777 1\\
0.0400556661188602 1\\
0.050055667757988 1\\
0.0600556656718254 1\\
0.0700556635856628 1\\
0.0800556689500809 1\\
0.0900556668639183 1\\
0.100055664777756 1\\
0.110055662691593 1\\
0.120055668056011 1\\
0.130055665969849 1\\
0.140055671334267 1\\
0.150055661797524 1\\
0.160055667161942 1\\
0.17005567252636 1\\
0.180055662989616 1\\
0.190055668354034 1\\
0.200055658817291 1\\
0.210055664181709 1\\
0.220055669546127 1\\
0.230055660009384 1\\
0.240055665373802 1\\
0.25005567073822 1\\
0.260055661201477 1\\
0.270055651664734 1\\
0.280055671930313 1\\
0.29005566239357 1\\
0.300055652856827 1\\
0.310055673122406 1\\
0.320055663585663 1\\
0.33005565404892 1\\
0.340055674314499 1\\
0.350055664777756 1\\
0.360055655241013 1\\
0.370055675506592 1\\
0.380055665969849 1\\
0.390055656433105 1\\
0.400055676698685 1\\
0.410055667161942 1\\
0.420055657625198 1\\
0.430055677890778 1\\
0.440055668354034 1\\
0.450055658817291 1\\
0.46005567908287 1\\
0.470055669546127 1\\
0.480055660009384 1\\
0.490055680274963 1\\
0.50005567073822 1\\
0.510055661201477 1\\
0.520055651664734 1\\
0.530055642127991 1\\
0.540055692195892 1\\
0.550055682659149 1\\
0.560055673122406 1\\
0.570055663585663 1\\
0.58005565404892 1\\
0.590055644512177 1\\
0.600055694580078 1\\
0.610055685043335 1\\
0.620055675506592 1\\
0.630055665969849 1\\
0.640055656433105 1\\
0.650055646896362 1\\
0.660055637359619 1\\
0.670055687427521 1\\
0.680055677890778 1\\
0.690055668354034 1\\
0.700055658817291 1\\
0.710055649280548 1\\
0.720055639743805 1\\
0.730055689811707 1\\
0.740055680274963 1\\
0.75005567073822 1\\
0.760055661201477 1\\
0.770055651664734 1\\
0.780055642127991 1\\
0.790055692195892 1\\
0.800055682659149 1\\
0.810055673122406 1\\
0.820055663585663 1\\
0.83005565404892 1\\
0.840055644512177 1\\
0.850055694580078 1\\
0.860055685043335 1\\
0.870055675506592 1\\
0.880055665969849 1\\
0.890055656433105 1\\
0.900055646896362 1\\
0.910055637359619 1\\
0.920055687427521 1\\
0.930055677890778 1\\
0.940055668354034 1\\
0.950055658817291 1\\
0.960055649280548 1\\
0.970055639743805 1\\
0.980055689811707 1\\
0.990055680274963 1\\
1.00005567073822 1\\
1.01005566120148 0.999947428703308\\
1.02005565166473 0.999781966209412\\
1.03005564212799 0.999620318412781\\
1.04005563259125 0.999392151832581\\
1.0500556230545 0.999095976352692\\
1.06005561351776 0.998730361461639\\
1.07005572319031 0.998293995857239\\
1.08005571365356 0.997785687446594\\
1.09005570411682 0.997204065322876\\
1.10005569458008 0.996548056602478\\
1.11005568504334 0.99581640958786\\
1.12005567550659 0.995007991790771\\
1.13005566596985 0.994121789932251\\
1.14005565643311 0.993156671524048\\
1.15005564689636 0.992111563682556\\
1.16005563735962 0.990985572338104\\
1.17005562782288 0.989777565002441\\
1.18005561828613 0.988486707210541\\
1.19005560874939 0.987112045288086\\
1.20005571842194 0.985652625560761\\
1.21005570888519 0.984107613563538\\
1.22005569934845 0.982476115226746\\
1.23005568981171 0.980757296085358\\
1.24005568027496 0.978951454162598\\
1.25005567073822 0.977075278759003\\
1.26005566120148 0.975142896175385\\
1.27005565166473 0.973155975341797\\
1.28005564212799 0.971112370491028\\
1.29005563259125 0.969009518623352\\
1.3000556230545 0.966845393180847\\
1.31005561351776 0.964618444442749\\
1.32005572319031 0.962327837944031\\
1.33005571365356 0.959973037242889\\
1.34005570411682 0.957553625106812\\
1.35005569458008 0.955069482326508\\
1.36005568504334 0.952520549297333\\
1.37005567550659 0.949906826019287\\
1.38005566596985 0.94722843170166\\
1.39005565643311 0.944485306739807\\
1.40005564689636 0.941677629947662\\
1.41005563735962 0.938805460929871\\
1.42005562782288 0.935868859291077\\
1.43005561828613 0.932867884635925\\
1.44005560874939 0.929802656173706\\
1.45005571842194 0.926673173904419\\
1.46005570888519 0.923479557037354\\
1.47005569934845 0.92022180557251\\
1.48005568981171 0.916899979114532\\
1.49005568027496 0.913514077663422\\
1.50005567073822 0.910064160823822\\
1.51005566120148 0.906550228595734\\
1.52005565166473 0.902972340583801\\
1.53005564212799 0.899330496788025\\
1.54005563259125 0.89562463760376\\
1.5500556230545 0.891854822635651\\
1.56005561351776 0.888020992279053\\
1.57005572319031 0.884123206138611\\
1.58005571365356 0.88016140460968\\
1.59005570411682 0.876135587692261\\
1.60005569458008 0.872045695781708\\
1.61005568504334 0.867891788482666\\
1.62005567550659 0.863673746585846\\
1.63005566596985 0.859391629695892\\
1.64005565643311 0.855045318603516\\
1.65005564689636 0.850634872913361\\
1.66005563735962 0.846160113811493\\
1.67005562782288 0.841621160507202\\
1.68005561828613 0.837017893791199\\
1.69005560874939 0.832350254058838\\
1.70005571842194 0.82761824131012\\
1.71005570888519 0.822821795940399\\
1.72005569934845 0.817960917949677\\
1.73005568981171 0.813035428524017\\
1.74005568027496 0.808045387268066\\
1.75005567073822 0.802990674972534\\
1.76005566120148 0.797872245311737\\
1.77005565166473 0.792716443538666\\
1.78005564212799 0.787553429603577\\
1.79005563259125 0.782386958599091\\
1.8000556230545 0.77721107006073\\
1.81005561351776 0.772017121315002\\
1.82005572319031 0.766796708106995\\
1.83005571365356 0.761542320251465\\
1.84005570411682 0.756247639656067\\
1.85005569458008 0.750907421112061\\
1.86005568504334 0.745517194271088\\
1.87005567550659 0.740073382854462\\
1.88005566596985 0.734573066234589\\
1.89005565643311 0.729013621807098\\
1.90005564689636 0.723393023014069\\
1.91005563735962 0.717709481716156\\
1.92005562782288 0.711961388587952\\
1.93005561828613 0.706147372722626\\
1.94005560874939 0.700266301631927\\
1.95005571842194 0.694316923618317\\
1.96005570888519 0.688726127147675\\
1.97005569934845 0.684719562530518\\
1.98005568981171 0.682762682437897\\
1.99005568027496 0.682944536209106\\
2.00005555152893 0.684946656227112\\
2.01005578041077 0.68822193145752\\
2.02005577087402 0.692149758338928\\
2.03005576133728 0.696172177791595\\
2.04005575180054 0.699921190738678\\
2.05005574226379 0.70319652557373\\
2.06005573272705 0.705916941165924\\
2.07005572319031 0.708073019981384\\
2.08005571365356 0.709694802761078\\
2.09005570411682 0.710845947265625\\
2.10005569458008 0.711670517921448\\
2.11005568504334 0.71240359544754\\
2.12005567550659 0.713474452495575\\
2.13005566596985 0.715694069862366\\
2.14005565643311 0.72019225358963\\
2.15005564689636 0.727587044239044\\
2.16005563735962 0.737067759037018\\
2.17005562782288 0.747016549110413\\
2.18005561828613 0.75606644153595\\
2.19005560874939 0.763664543628693\\
2.20005559921265 0.769892036914825\\
2.2100555896759 0.775082290172577\\
2.22005558013916 0.779570817947388\\
2.23005557060242 0.783605515956879\\
2.24005556106567 0.787335813045502\\
2.25005555152893 0.790832877159119\\
2.26005578041077 0.794116020202637\\
2.27005577087402 0.797174692153931\\
2.28005576133728 0.799983859062195\\
2.29005575180054 0.802512407302856\\
2.30005574226379 0.804727077484131\\
2.31005573272705 0.806596517562866\\
2.32005572319031 0.808101892471313\\
2.33005571365356 0.809250116348267\\
2.34005570411682 0.810067415237427\\
2.35005569458008 0.810580432415009\\
2.36005568504334 0.810807466506958\\
2.37005567550659 0.810757994651794\\
2.38005566596985 0.810436189174652\\
2.39005565643311 0.809808015823364\\
2.40005564689636 0.808527410030365\\
2.41005563735962 0.806402921676636\\
2.42005562782288 0.803346753120422\\
2.43005561828613 0.799330174922943\\
2.44005560874939 0.794373869895935\\
2.45005559921265 0.788535118103027\\
2.4600555896759 0.781897485256195\\
2.47005558013916 0.774561285972595\\
2.48005557060242 0.766634404659271\\
2.49005556106567 0.758225440979004\\
2.50005555152893 0.749439239501953\\
2.51005578041077 0.740374565124512\\
2.52005577087402 0.731126070022583\\
2.53005576133728 0.721779882907867\\
2.54005575180054 0.712411522865295\\
2.55005574226379 0.703087031841278\\
2.56005573272705 0.693863213062286\\
2.57005572319031 0.684788703918457\\
2.58005571365356 0.675905704498291\\
2.59005570411682 0.667280972003937\\
2.60005569458008 0.659027338027954\\
2.61005568504334 0.651259481906891\\
2.62005567550659 0.64411187171936\\
2.63005566596985 0.638089895248413\\
2.64005565643311 0.635475873947144\\
2.65005564689636 0.636992394924164\\
2.66005563735962 0.642294049263\\
2.67005562782288 0.650389790534973\\
2.68005561828613 0.66011655330658\\
2.69005560874939 0.670444130897522\\
2.70005559921265 0.680894017219543\\
2.7100555896759 0.69127082824707\\
2.72005558013916 0.701449930667877\\
2.73005557060242 0.711334228515625\\
2.74005556106567 0.720849096775055\\
2.75005555152893 0.729940533638\\
2.76005578041077 0.738573670387268\\
2.77005577087402 0.746732950210571\\
2.78005576133728 0.75442761182785\\
2.79005575180054 0.761723875999451\\
2.80005574226379 0.768760919570923\\
2.81005573272705 0.775801479816437\\
2.82005572319031 0.783307373523712\\
2.83005571365356 0.791747987270355\\
2.84005570411682 0.801149427890778\\
2.85005569458008 0.810765027999878\\
2.86005568504334 0.81949907541275\\
2.87005567550659 0.826596915721893\\
2.88005566596985 0.831840455532074\\
2.89005565643311 0.83536297082901\\
2.90005564689636 0.837432265281677\\
2.91005563735962 0.838325142860413\\
2.92005562782288 0.838279187679291\\
2.93005561828613 0.837484121322632\\
2.94005560874939 0.836091339588165\\
2.95005559921265 0.834230363368988\\
2.9600555896759 0.832008361816406\\
2.97005558013916 0.829510569572449\\
2.98005557060242 0.826798796653748\\
2.99005556106567 0.823849558830261\\
3.00005555152893 0.820338606834412\\
3.01005578041077 0.816118240356445\\
3.02005577087402 0.811094701290131\\
3.03005576133728 0.805221617221832\\
3.04005575180054 0.798506081104279\\
3.05005574226379 0.790995538234711\\
3.06005573272705 0.782766699790955\\
3.07005572319031 0.773912668228149\\
3.08005571365356 0.764532744884491\\
3.09005570411682 0.754732191562653\\
3.10005569458008 0.74461418390274\\
3.11005568504334 0.734275102615356\\
3.12005567550659 0.723810791969299\\
3.13005566596985 0.713323354721069\\
3.14005565643311 0.702923178672791\\
3.15005564689636 0.692727208137512\\
3.16005563735962 0.6828573346138\\
3.17005562782288 0.673441410064697\\
3.18005561828613 0.664614915847778\\
3.19005560874939 0.656521081924438\\
3.20005559921265 0.6493119597435\\
3.2100555896759 0.643689274787903\\
3.22005558013916 0.640969336032867\\
3.23005557060242 0.641327738761902\\
3.24005556106567 0.64477926492691\\
3.25005555152893 0.651229858398438\\
3.26005578041077 0.660486876964569\\
3.27005577087402 0.672317624092102\\
3.28005576133728 0.686516880989075\\
3.29005575180054 0.702502965927124\\
3.30005574226379 0.71934050321579\\
3.31005573272705 0.736350178718567\\
3.32005572319031 0.752970933914185\\
3.33005571365356 0.768543124198914\\
3.34005570411682 0.78220933675766\\
3.35005569458008 0.793600916862488\\
3.36005568504334 0.802746832370758\\
3.37005567550659 0.809874594211578\\
3.38005566596985 0.815254032611847\\
3.39005565643311 0.819222211837769\\
3.40005564689636 0.822238445281982\\
3.41005563735962 0.824924468994141\\
3.42005562782288 0.827865660190582\\
3.43005561828613 0.831193685531616\\
3.44005560874939 0.834470629692078\\
3.45005559921265 0.837045133113861\\
3.4600555896759 0.838456928730011\\
3.47005558013916 0.838563680648804\\
3.48005557060242 0.837459683418274\\
3.49005556106567 0.835370063781738\\
3.50005555152893 0.832540452480316\\
3.51005578041077 0.829169869422913\\
3.52005577087402 0.825400471687317\\
3.53005576133728 0.821102261543274\\
3.54005575180054 0.816166758537292\\
3.55005574226379 0.810570001602173\\
3.56005573272705 0.804301142692566\\
3.57005572319031 0.797378361225128\\
3.58005571365356 0.789849638938904\\
3.59005570411682 0.781789541244507\\
3.60005569458008 0.773292362689972\\
3.61005568504334 0.764449298381805\\
3.62005567550659 0.755341768264771\\
3.63005566596985 0.746031045913696\\
3.64005565643311 0.73656153678894\\
3.65005564689636 0.726971328258514\\
3.66005563735962 0.717311680316925\\
3.67005562782288 0.707655310630798\\
3.68005561828613 0.698091566562653\\
3.69005560874939 0.688718318939209\\
3.70005559921265 0.679637014865875\\
3.7100555896759 0.670949876308441\\
3.72005558013916 0.662759780883789\\
3.73005557060242 0.655170261859894\\
3.74005556106567 0.648285984992981\\
3.75005555152893 0.643000721931458\\
3.76005578041077 0.640484094619751\\
3.77005577087402 0.640774786472321\\
3.78005576133728 0.64378821849823\\
3.79005575180054 0.649352967739105\\
3.80005574226379 0.65724766254425\\
3.81005573272705 0.667321383953094\\
3.82005572319031 0.679539859294891\\
3.83005571365356 0.693313419818878\\
3.84005570411682 0.707874894142151\\
3.85005569458008 0.722723782062531\\
3.86005568504334 0.737446546554565\\
3.87005567550659 0.751723110675812\\
3.88005566596985 0.765283107757568\\
3.89005565643311 0.777613043785095\\
3.90005564689636 0.788216233253479\\
3.91005563735962 0.796943128108978\\
3.92005562782288 0.803911566734314\\
3.93005561828613 0.809428155422211\\
3.94005560874939 0.813998341560364\\
3.95005559921265 0.81830245256424\\
3.9600555896759 0.822883486747742\\
3.97005558013916 0.827703356742859\\
3.98005557060242 0.832173526287079\\
3.99005556106567 0.835632145404816\\
4.00005578994751 0.837709724903107\\
4.01005554199219 0.838359415531158\\
4.02005577087402 0.837725639343262\\
4.0300555229187 0.836036622524261\\
4.04005575180054 0.833545684814453\\
4.05005550384521 0.830461859703064\\
4.06005573272705 0.826937556266785\\
4.07005548477173 0.823050916194916\\
4.08005571365356 0.818639755249023\\
4.09005546569824 0.813654541969299\\
4.10005569458008 0.808072805404663\\
4.11005544662476 0.801887035369873\\
4.12005567550659 0.795113205909729\\
4.13005590438843 0.787792921066284\\
4.14005565643311 0.779995858669281\\
4.15005588531494 0.771788001060486\\
4.16005563735962 0.763211667537689\\
4.17005586624146 0.754295170307159\\
4.18005561828613 0.745076537132263\\
4.19005584716797 0.735627353191376\\
4.20005559921265 0.726055562496185\\
4.21005582809448 0.716494679450989\\
4.22005558013916 0.707091271877289\\
4.230055809021 0.697990417480469\\
4.24005556106567 0.689327418804169\\
4.25005578994751 0.681229948997498\\
4.26005554199219 0.67382025718689\\
4.27005577087402 0.667220413684845\\
4.2800555229187 0.662580728530884\\
4.29005575180054 0.660546362400055\\
4.30005550384521 0.660974681377411\\
4.31005573272705 0.663656651973724\\
4.32005548477173 0.668352723121643\\
4.33005571365356 0.674814999103546\\
4.34005546569824 0.682811677455902\\
4.35005569458008 0.692189633846283\\
4.36005544662476 0.702848672866821\\
4.37005567550659 0.714362978935242\\
4.38005590438843 0.726243197917938\\
4.39005565643311 0.738207519054413\\
4.40005588531494 0.750032901763916\\
4.41005563735962 0.761541306972504\\
4.42005586624146 0.77244108915329\\
4.43005561828613 0.782171189785004\\
4.44005584716797 0.790392339229584\\
4.45005559921265 0.797060191631317\\
4.46005582809448 0.802361071109772\\
4.47005558013916 0.806710541248322\\
4.480055809021 0.810754358768463\\
4.49005556106567 0.815118491649628\\
4.50005578994751 0.81991046667099\\
4.51005554199219 0.824587881565094\\
4.52005577087402 0.828408241271973\\
4.5300555229187 0.830882668495178\\
4.54005575180054 0.831881284713745\\
4.55005550384521 0.831510782241821\\
4.56005573272705 0.829980671405792\\
4.57005548477173 0.827542364597321\\
4.58005571365356 0.824441850185394\\
4.59005546569824 0.820853888988495\\
4.60005569458008 0.816763043403625\\
4.61005544662476 0.812026143074036\\
4.62005567550659 0.806649029254913\\
4.63005590438843 0.800637543201447\\
4.64005565643311 0.794015645980835\\
4.65005588531494 0.786828994750977\\
4.66005563735962 0.779140710830688\\
4.67005586624146 0.771032869815826\\
4.68005561828613 0.762596428394318\\
4.69005584716797 0.753898978233337\\
4.70005559921265 0.744966864585876\\
4.71005582809448 0.735809862613678\\
4.72005558013916 0.726461172103882\\
4.730055809021 0.71700257062912\\
4.74005556106567 0.707561314105988\\
4.75005578994751 0.698293566703796\\
4.76005554199219 0.689372897148132\\
4.77005577087402 0.68098646402359\\
4.7800555229187 0.673335611820221\\
4.79005575180054 0.666619122028351\\
4.80005550384521 0.662141263484955\\
4.81005573272705 0.661036968231201\\
4.82005548477173 0.663147747516632\\
4.83005571365356 0.668159365653992\\
4.84005546569824 0.675692081451416\\
4.85005569458008 0.685342729091644\\
4.86005544662476 0.696662604808807\\
4.87005567550659 0.709140121936798\\
4.88005590438843 0.722396552562714\\
4.89005565643311 0.736092329025269\\
4.90005588531494 0.749655246734619\\
4.91005563735962 0.762749135494232\\
4.92005586624146 0.77489846944809\\
4.93005561828613 0.78548389673233\\
4.94005584716797 0.794238090515137\\
4.95005559921265 0.801163911819458\\
4.96005582809448 0.806397080421448\\
4.97005558013916 0.810135841369629\\
4.980055809021 0.812640786170959\\
4.99005556106567 0.814299583435059\\
5 0.815687239170074\\
};
\addlegendentry{$\dtmax{} = \expneg{1.0}{4}{[s]}$};

\addplot [
color=black,
solid
]
table[row sep=crcr]{
0 1\\
0.0100024435669184 1\\
0.020002443343401 1\\
0.0300024431198835 1\\
0.0400024428963661 1\\
0.0500024445354939 1\\
0.0600024424493313 1\\
0.070002444088459 1\\
0.0800024420022964 1\\
0.0900024399161339 1\\
0.100002445280552 1\\
0.110002443194389 1\\
0.120002441108227 1\\
0.130002439022064 1\\
0.140002444386482 1\\
0.1500024497509 1\\
0.160002440214157 1\\
0.170002445578575 1\\
0.180002436041832 1\\
0.19000244140625 1\\
0.200002446770668 1\\
0.210002437233925 1\\
0.220002442598343 1\\
0.230002447962761 1\\
0.240002438426018 1\\
0.250002443790436 1\\
0.260002434253693 1\\
0.270002454519272 1\\
0.280002444982529 1\\
0.290002435445786 1\\
0.300002455711365 1\\
0.310002446174622 1\\
0.320002436637878 1\\
0.330002456903458 1\\
0.340002447366714 1\\
0.350002437829971 1\\
0.360002458095551 1\\
0.370002448558807 1\\
0.380002439022064 1\\
0.390002429485321 1\\
0.4000024497509 1\\
0.410002440214157 1\\
0.420002430677414 1\\
0.430002450942993 1\\
0.44000244140625 1\\
0.450002431869507 1\\
0.460002452135086 1\\
0.470002442598343 1\\
0.4800024330616 1\\
0.490002453327179 1\\
0.500002443790436 1\\
0.510002434253693 1\\
0.520002424716949 1\\
0.530002415180206 1\\
0.540002465248108 1\\
0.550002455711365 1\\
0.560002446174622 1\\
0.570002436637878 1\\
0.580002427101135 1\\
0.590002417564392 1\\
0.600002467632294 1\\
0.610002458095551 1\\
0.620002448558807 1\\
0.630002439022064 1\\
0.640002429485321 1\\
0.650002419948578 1\\
0.660002470016479 1\\
0.670002460479736 1\\
0.680002450942993 1\\
0.69000244140625 1\\
0.700002431869507 1\\
0.710002422332764 1\\
0.720002472400665 1\\
0.730002462863922 1\\
0.740002453327179 1\\
0.750002443790436 1\\
0.760002434253693 1\\
0.770002424716949 1\\
0.780002415180206 1\\
0.790002465248108 1\\
0.800002455711365 1\\
0.810002446174622 1\\
0.820002436637878 1\\
0.830002427101135 1\\
0.840002417564392 1\\
0.850002467632294 1\\
0.860002458095551 1\\
0.870002448558807 1\\
0.880002439022064 1\\
0.890002429485321 1\\
0.900002419948578 1\\
0.910002470016479 1\\
0.920002460479736 1\\
0.930002450942993 1\\
0.94000244140625 1\\
0.950002431869507 1\\
0.960002422332764 1\\
0.970002472400665 1\\
0.980002462863922 1\\
0.990002453327179 1\\
1.00000238418579 1\\
1.01000249385834 0.999940812587738\\
1.02000248432159 0.99977570772171\\
1.03000247478485 0.999614715576172\\
1.04000246524811 0.999387145042419\\
1.05000245571136 0.999091565608978\\
1.06000244617462 0.998726546764374\\
1.07000243663788 0.99829089641571\\
1.08000242710114 0.997783184051514\\
1.09000241756439 0.997202277183533\\
1.10000240802765 0.996546864509583\\
1.11000239849091 0.995815932750702\\
1.12000238895416 0.995008230209351\\
1.13000249862671 0.994122684001923\\
1.14000248908997 0.993158280849457\\
1.15000247955322 0.992113947868347\\
1.16000247001648 0.990988612174988\\
1.17000246047974 0.989781439304352\\
1.18000245094299 0.988491296768188\\
1.19000244140625 0.987117409706116\\
1.20000243186951 0.985658764839172\\
1.21000242233276 0.984114527702332\\
1.22000241279602 0.982483804225922\\
1.23000240325928 0.980765759944916\\
1.24000239372253 0.978960752487183\\
1.25000238418579 0.97708523273468\\
1.26000249385834 0.975153505802155\\
1.27000248432159 0.973167300224304\\
1.28000247478485 0.971124410629272\\
1.29000246524811 0.969022274017334\\
1.30000245571136 0.966858863830566\\
1.31000244617462 0.964632630348206\\
1.32000243663788 0.962342798709869\\
1.33000242710114 0.959988653659821\\
1.34000241756439 0.957569897174835\\
1.35000240802765 0.955086469650269\\
1.36000239849091 0.952538251876831\\
1.37000238895416 0.949925184249878\\
1.38000249862671 0.947247445583344\\
1.39000248908997 0.944505035877228\\
1.40000247955322 0.941698014736176\\
1.41000247001648 0.938826501369476\\
1.42000246047974 0.935890555381775\\
1.43000245094299 0.932890236377716\\
1.44000244140625 0.929825663566589\\
1.45000243186951 0.926696836948395\\
1.46000242233276 0.923503875732422\\
1.47000241279602 0.920246779918671\\
1.48000240325928 0.916925609111786\\
1.49000239372253 0.913540363311768\\
1.50000238418579 0.910091102123261\\
1.51000249385834 0.906577885150909\\
1.52000248432159 0.903000593185425\\
1.53000247478485 0.899359405040741\\
1.54000246524811 0.895654201507568\\
1.55000245571136 0.891884982585907\\
1.56000244617462 0.888051867485046\\
1.57000243663788 0.884154677391052\\
1.58000242710114 0.880193531513214\\
1.59000241756439 0.876168370246887\\
1.60000240802765 0.872079133987427\\
1.61000239849091 0.867925882339478\\
1.62000238895416 0.86370849609375\\
1.63000249862671 0.859426975250244\\
1.64000248908997 0.85508131980896\\
1.65000247955322 0.850671470165253\\
1.66000247001648 0.846197426319122\\
1.67000246047974 0.841659069061279\\
1.68000245094299 0.837056457996368\\
1.69000244140625 0.8323894739151\\
1.70000243186951 0.827658116817474\\
1.71000242233276 0.822862327098846\\
1.72000241279602 0.818002045154572\\
1.73000240325928 0.813077211380005\\
1.74000239372253 0.808087766170502\\
1.75000238418579 0.803033709526062\\
1.76000249385834 0.797915875911713\\
1.77000248432159 0.792760372161865\\
1.78000247478485 0.787597477436066\\
1.79000246524811 0.782431244850159\\
1.80000245571136 0.777255713939667\\
1.81000244617462 0.772062182426453\\
1.82000243663788 0.766842186450958\\
1.83000242710114 0.761588335037231\\
1.84000241756439 0.756294250488281\\
1.85000240802765 0.750954568386078\\
1.86000239849091 0.745564937591553\\
1.87000238895416 0.740121781826019\\
1.88000249862671 0.734622061252594\\
1.89000248908997 0.729063272476196\\
1.90000247955322 0.723443329334259\\
1.91000247001648 0.717760384082794\\
1.92000246047974 0.712012946605682\\
1.93000245094299 0.706199586391449\\
1.94000244140625 0.700319111347198\\
1.95000243186951 0.69437038898468\\
1.96000242233276 0.688745439052582\\
1.97000241279602 0.684689223766327\\
1.98000240325928 0.682687282562256\\
1.99000239372253 0.682839155197144\\
2.00000238418579 0.684832692146301\\
2.01000237464905 0.6881183385849\\
2.0200023651123 0.692070066928864\\
2.03000235557556 0.696122884750366\\
2.04000234603882 0.699903011322021\\
2.05000233650208 0.703207314014435\\
2.06000232696533 0.705953180789948\\
2.07000255584717 0.708130538463593\\
2.08000254631042 0.709769666194916\\
2.09000253677368 0.710934162139893\\
2.10000252723694 0.711767792701721\\
2.1100025177002 0.71250307559967\\
2.12000250816345 0.713563203811646\\
2.13000249862671 0.715750575065613\\
2.14000248908997 0.720194697380066\\
2.15000247955322 0.727541267871857\\
2.16000247001648 0.737007617950439\\
2.17000246047974 0.746980786323547\\
2.18000245094299 0.75606632232666\\
2.19000244140625 0.76367050409317\\
2.20000243186951 0.769839704036713\\
2.21000242233276 0.774884164333344\\
2.22000241279602 0.779128074645996\\
2.23000240325928 0.782818675041199\\
2.24000239372253 0.786113321781158\\
2.25000238418579 0.789097189903259\\
2.26000237464905 0.791806697845459\\
2.2700023651123 0.794249296188354\\
2.28000235557556 0.796417117118835\\
2.29000234603882 0.798294723033905\\
2.30000233650208 0.799862682819366\\
2.31000232696533 0.801101744174957\\
2.32000255584717 0.802003145217896\\
2.33000254631042 0.802580416202545\\
2.34000253677368 0.802861630916595\\
2.35000252723694 0.802873730659485\\
2.3600025177002 0.802634477615356\\
2.37000250816345 0.802153050899506\\
2.38000249862671 0.801429092884064\\
2.39000248908997 0.800221979618073\\
2.40000247955322 0.798298180103302\\
2.41000247001648 0.795549929141998\\
2.42000246047974 0.791922390460968\\
2.43000245094299 0.787411272525787\\
2.44000244140625 0.78205281496048\\
2.45000243186951 0.775913238525391\\
2.46000242233276 0.769079387187958\\
2.47000241279602 0.761649787425995\\
2.48000240325928 0.75372725725174\\
2.49000239372253 0.745413899421692\\
2.50000238418579 0.736807584762573\\
2.51000237464905 0.727999925613403\\
2.5200023651123 0.719075739383698\\
2.53000235557556 0.71011346578598\\
2.54000234603882 0.701183497905731\\
2.55000233650208 0.692348182201386\\
2.56000232696533 0.683664500713348\\
2.57000255584717 0.675186276435852\\
2.58000254631042 0.666962385177612\\
2.59000253677368 0.659052014350891\\
2.60000252723694 0.651571691036224\\
2.6100025177002 0.644659519195557\\
2.62000250816345 0.639001190662384\\
2.63000249862671 0.636265516281128\\
2.64000248908997 0.636875569820404\\
2.65000247955322 0.640672266483307\\
2.66000247001648 0.646990716457367\\
2.67000246047974 0.654961466789246\\
2.68000245094299 0.663949131965637\\
2.69000244140625 0.67361444234848\\
2.70000243186951 0.683757841587067\\
2.71000242233276 0.694252073764801\\
2.72000241279602 0.704960525035858\\
2.73000240325928 0.715604186058044\\
2.74000239372253 0.726011276245117\\
2.75000238418579 0.736059725284576\\
2.76000237464905 0.74566262960434\\
2.7700023651123 0.754777073860168\\
2.78000235557556 0.763424098491669\\
2.79000234603882 0.771661460399628\\
2.80000233650208 0.779595911502838\\
2.81000232696533 0.787380576133728\\
2.82000255584717 0.795373737812042\\
2.83000254631042 0.803945481777191\\
2.84000253677368 0.812890529632568\\
2.85000252723694 0.821369171142578\\
2.8600025177002 0.828521132469177\\
2.87000250816345 0.833931267261505\\
2.88000249862671 0.837590038776398\\
2.89000248908997 0.839699506759644\\
2.90000247955322 0.840515732765198\\
2.91000247001648 0.840286910533905\\
2.92000246047974 0.839248299598694\\
2.93000245094299 0.837615489959717\\
2.94000244140625 0.835552573204041\\
2.95000243186951 0.833161592483521\\
2.96000242233276 0.83050400018692\\
2.97000241279602 0.827616453170776\\
2.98000240325928 0.82451593875885\\
2.99000239372253 0.820978939533234\\
3.00000238418579 0.816808044910431\\
3.01000237464905 0.811919093132019\\
3.0200023651123 0.806261420249939\\
3.03000235557556 0.799826622009277\\
3.04000234603882 0.792640388011932\\
3.05000233650208 0.78475558757782\\
3.06000232696533 0.776243686676025\\
3.07000255584717 0.76718533039093\\
3.08000254631042 0.757668673992157\\
3.09000253677368 0.747788429260254\\
3.10000252723694 0.737645208835602\\
3.1100025177002 0.727348446846008\\
3.12000250816345 0.717015266418457\\
3.13000249862671 0.706768274307251\\
3.14000248908997 0.696734964847565\\
3.15000247955322 0.687047123908997\\
3.16000247001648 0.677841663360596\\
3.17000246047974 0.669269144535065\\
3.18000245094299 0.661491990089417\\
3.19000244140625 0.65468418598175\\
3.20000243186951 0.650017201900482\\
3.21000242233276 0.648434698581696\\
3.22000241279602 0.650020599365234\\
3.23000240325928 0.654721796512604\\
3.24000239372253 0.662368059158325\\
3.25000238418579 0.672687172889709\\
3.26000237464905 0.68532407283783\\
3.2700023651123 0.699792206287384\\
3.28000235557556 0.715433180332184\\
3.29000234603882 0.731622636318207\\
3.30000233650208 0.74785315990448\\
3.31000232696533 0.763584196567535\\
3.32000255584717 0.777789115905762\\
3.33000254631042 0.789840221405029\\
3.34000253677368 0.799637496471405\\
3.35000252723694 0.80733323097229\\
3.3600025177002 0.813177108764648\\
3.37000250816345 0.817418158054352\\
3.38000249862671 0.820353150367737\\
3.39000248908997 0.822400808334351\\
3.40000247955322 0.824148237705231\\
3.41000247001648 0.826191127300262\\
3.42000246047974 0.828715562820435\\
3.43000245094299 0.831327855587006\\
3.44000244140625 0.833372116088867\\
3.45000243186951 0.834353864192963\\
3.46000242233276 0.834100604057312\\
3.47000241279602 0.832691431045532\\
3.48000240325928 0.830315828323364\\
3.49000239372253 0.827174186706543\\
3.50000238418579 0.823444366455078\\
3.51000237464905 0.819157183170319\\
3.5200023651123 0.814174652099609\\
3.53000235557556 0.80850887298584\\
3.54000234603882 0.802171409130096\\
3.55000233650208 0.795189440250397\\
3.56000232696533 0.787611842155457\\
3.57000255584717 0.779510974884033\\
3.58000254631042 0.770982265472412\\
3.59000253677368 0.762116491794586\\
3.60000252723694 0.75298810005188\\
3.6100025177002 0.743649303913116\\
3.62000250816345 0.734138906002045\\
3.63000249862671 0.72450578212738\\
3.64000248908997 0.714825749397278\\
3.65000247955322 0.705201268196106\\
3.66000247001648 0.695751786231995\\
3.67000246047974 0.686601459980011\\
3.68000245094299 0.677873849868774\\
3.69000244140625 0.669694840908051\\
3.70000243186951 0.662191450595856\\
3.71000242233276 0.655487895011902\\
3.72000241279602 0.650723576545715\\
3.73000240325928 0.648885309696198\\
3.74000239372253 0.649973750114441\\
3.75000238418579 0.653858482837677\\
3.76000237464905 0.660316050052643\\
3.7700023651123 0.669060170650482\\
3.78000235557556 0.679768145084381\\
3.79000234603882 0.692050218582153\\
3.80000233650208 0.705434203147888\\
3.81000232696533 0.71948653459549\\
3.82000255584717 0.733878612518311\\
3.83000254631042 0.748284876346588\\
3.84000253677368 0.762206435203552\\
3.85000252723694 0.775071322917938\\
3.8600025177002 0.786237180233002\\
3.87000250816345 0.795463681221008\\
3.88000249862671 0.802794516086578\\
3.89000248908997 0.80842137336731\\
3.90000247955322 0.812635660171509\\
3.91000247001648 0.815868079662323\\
3.92000246047974 0.818732440471649\\
3.93000245094299 0.821850001811981\\
3.94000244140625 0.825400650501251\\
3.95000243186951 0.828947901725769\\
3.96000242233276 0.831803321838379\\
3.97000241279602 0.833474278450012\\
3.98000240325928 0.833809614181519\\
3.99000239372253 0.832912147045136\\
4.00000238418579 0.830981612205505\\
4.01000261306763 0.828209102153778\\
4.0200023651123 0.824769854545593\\
4.03000259399414 0.820816099643707\\
4.04000234603882 0.816367745399475\\
4.05000257492065 0.811372339725494\\
4.06000232696533 0.805842399597168\\
4.07000255584717 0.799781084060669\\
4.08000230789185 0.793201327323914\\
4.09000253677368 0.786133170127869\\
4.10000228881836 0.778631329536438\\
4.1100025177002 0.770748496055603\\
4.12000226974487 0.762511432170868\\
4.13000249862671 0.753932535648346\\
4.14000225067139 0.745044589042664\\
4.15000247955322 0.735920429229736\\
4.1600022315979 0.726675033569336\\
4.17000246047974 0.717452049255371\\
4.18000221252441 0.708409070968628\\
4.19000244140625 0.699700772762299\\
4.20000267028809 0.691461563110352\\
4.21000242233276 0.683844923973084\\
4.2200026512146 0.676974713802338\\
4.23000240325928 0.671293795108795\\
4.24000263214111 0.668199121952057\\
4.25000238418579 0.667755961418152\\
4.26000261306763 0.66975611448288\\
4.2700023651123 0.67393434047699\\
4.28000259399414 0.680002391338348\\
4.29000234603882 0.687669932842255\\
4.30000257492065 0.696648836135864\\
4.31000232696533 0.706620633602142\\
4.32000255584717 0.717259347438812\\
4.33000230789185 0.728299975395203\\
4.34000253677368 0.739564657211304\\
4.35000228881836 0.750948131084442\\
4.3600025177002 0.762172162532806\\
4.37000226974487 0.772863268852234\\
4.38000249862671 0.782439708709717\\
4.39000225067139 0.79052060842514\\
4.40000247955322 0.797021567821503\\
4.4100022315979 0.802044987678528\\
4.42000246047974 0.805847465991974\\
4.43000221252441 0.808882057666779\\
4.44000244140625 0.811804234981537\\
4.45000267028809 0.815179049968719\\
4.46000242233276 0.819006741046906\\
4.4700026512146 0.822681307792664\\
4.48000240325928 0.825474560260773\\
4.49000263214111 0.826951384544373\\
4.50000238418579 0.827024698257446\\
4.51000261306763 0.825829267501831\\
4.5200023651123 0.823601603507996\\
4.53000259399414 0.820559978485107\\
4.54000234603882 0.816870450973511\\
4.55000257492065 0.812584280967712\\
4.56000232696533 0.8076251745224\\
4.57000255584717 0.802040159702301\\
4.58000230789185 0.79586386680603\\
4.59000253677368 0.789134502410889\\
4.60000228881836 0.78190416097641\\
4.6100025177002 0.774235308170319\\
4.62000226974487 0.766205191612244\\
4.63000249862671 0.757900238037109\\
4.64000225067139 0.749383747577667\\
4.65000247955322 0.740675866603851\\
4.6600022315979 0.731779456138611\\
4.67000246047974 0.722719192504883\\
4.68000221252441 0.71356862783432\\
4.69000244140625 0.704450368881226\\
4.70000267028809 0.6955206990242\\
4.71000242233276 0.686956346035004\\
4.7200026512146 0.678950309753418\\
4.73000240325928 0.671709358692169\\
4.74000263214111 0.665537655353546\\
4.75000238418579 0.66224879026413\\
4.76000261306763 0.662468492984772\\
4.7700023651123 0.66599428653717\\
4.78000259399414 0.672457456588745\\
4.79000234603882 0.681407451629639\\
4.80000257492065 0.692347288131714\\
4.81000232696533 0.704669594764709\\
4.82000255584717 0.717771351337433\\
4.83000230789185 0.731148719787598\\
4.84000253677368 0.744415760040283\\
4.85000228881836 0.757317245006561\\
4.8600025177002 0.769666492938995\\
4.87000226974487 0.780902028083801\\
4.88000249862671 0.790477156639099\\
4.89000225067139 0.798252642154694\\
4.90000247955322 0.804286897182465\\
4.9100022315979 0.808729112148285\\
4.92000246047974 0.811772644519806\\
4.93000221252441 0.813669919967651\\
4.94000244140625 0.81481146812439\\
4.95000267028809 0.815780937671661\\
4.96000242233276 0.817187070846558\\
4.9700026512146 0.819230139255524\\
4.98000240325928 0.821508407592773\\
4.99000263214111 0.823348879814148\\
5 0.824254989624023\\
};
\addlegendentry{$\dtmax{} = \expneg{1.0}{5}{[s]}$};

\end{axis}
\end{tikzpicture}%
\caption{Nonlinear solver flashing solutions.}
\label{fig:flashingAlphaNln}
\end{figure}

The two solvers, when applied to the same problem that contains highly nonlinear physics, produce two different timestep-size invariant solutions.
These two solutions achieve qualitative timestep-size invariance at different timestep sizes.
The parameter of interest in the solution produced by the nonlinear solver with a 1 [s] \dtmax{} is qualitatively close to that produced with a 0.1 [s] \dtmax{}, \fig{fig:flashingDtInsensitiveNln}.
The same level of qualitative timestep-size invariance is achieved in the linear solver solutions between a \dtmax{} of \expneg{1.0}{3} [s] and \expneg{1.0}{4} [s], \fig{fig:flashingDtInsensitiveLin}.

\begin{figure}[h!tb]
\centering
% This file was created by matlab2tikz v0.4.3.
% Copyright (c) 2008--2013, Nico Schlömer <nico.schloemer@gmail.com>
% All rights reserved.
% 
\tikzsetnextfilename{plots/flashingDtInsensitiveNln_pdf}
\begin{tikzpicture}

\begin{axis}[%
width=\mytikzpicwidth,
height=\mytikzpicheight,
scale only axis,
xmin=0.0,
xmax=5.0,
xlabel={Time $[\text{s}]$},
ymin=0.5,
ymax=1.0,
ylabel={$\alpha_g$ [-] @ 2 [in] from Inlet},
legend style={at={(0.03,0.03)},anchor=south west,draw=black,fill=white,legend cell align=left}
]
\addplot [
color=black,
dotted
]
table[row sep=crcr]{
0.0 1.0\\
0.0118019115179777 1\\
0.0226854234933853 1\\
0.0340500771999359 1\\
0.0491764321923256 1\\
0.0619497485458851 1\\
0.0774054601788521 1\\
0.0961068719625473 1\\
0.106882445514202 1\\
0.118735581636429 1\\
0.13177402317524 1\\
0.146116316318512 1\\
0.161892831325531 1\\
0.179247006773949 1\\
0.198336601257324 1\\
0.21933513879776 1\\
0.242433547973633 1\\
0.267841786146164 1\\
0.29579085111618 1\\
0.32653483748436 1\\
0.360353201627731 1\\
0.397553414106369 1\\
0.438473641872406 1\\
0.483485877513886 1\\
0.532999336719513 1\\
0.587464153766632 1\\
0.647375464439392 1\\
0.71327793598175 1\\
0.7857705950737 1\\
0.865512549877167 1\\
0.953228712081909 1\\
1.04971647262573 0.995161294937134\\
1.1558530330658 0.979693293571472\\
1.27260315418243 0.957357287406921\\
1.32053518295288 0.947412252426147\\
1.35773289203644 0.93841552734375\\
1.38946354389191 0.929924786090851\\
1.42200791835785 0.920374691486359\\
1.45245742797852 0.910801649093628\\
1.48083031177521 0.901349127292633\\
1.50743794441223 0.892022550106049\\
1.53255617618561 0.882808029651642\\
1.55641222000122 0.873687744140625\\
1.57917487621307 0.864649832248688\\
1.60098683834076 0.855681836605072\\
1.62195324897766 0.846776962280273\\
1.64217150211334 0.837925493717194\\
1.66171324253082 0.829122722148895\\
1.68064773082733 0.820361614227295\\
1.69902181625366 0.811640918254852\\
1.71688628196716 0.802955090999603\\
1.73428571224213 0.794298350811005\\
1.75125479698181 0.785774171352386\\
1.76759457588196 0.777539014816284\\
1.78327512741089 0.769592463970184\\
1.79837548732758 0.761874079704285\\
1.81300187110901 0.754316449165344\\
1.82727634906769 0.746847033500671\\
1.84123587608337 0.739443123340607\\
1.85493552684784 0.732072591781616\\
1.86838006973267 0.724733471870422\\
1.88161540031433 0.71740061044693\\
1.89463996887207 0.710077345371246\\
1.90748333930969 0.702748000621796\\
1.92014336585999 0.695416688919067\\
1.93264579772949 0.688069760799408\\
1.94498836994171 0.683255314826965\\
1.95688509941101 0.681701004505157\\
1.96837878227234 0.682874798774719\\
1.97949528694153 0.685770511627197\\
1.99014973640442 0.689332842826843\\
2.00136232376099 0.693269908428192\\
2.01271271705627 0.697051227092743\\
2.02417707443237 0.700365483760834\\
2.0357928276062 0.703082680702209\\
2.04750537872314 0.705152451992035\\
2.05927753448486 0.706590950489044\\
2.07107305526733 0.707451343536377\\
2.08286380767822 0.707899689674377\\
2.09455490112305 0.708136200904846\\
2.10611391067505 0.708497524261475\\
2.11754536628723 0.709739506244659\\
2.12883257865906 0.713333308696747\\
2.13992094993591 0.720506727695465\\
2.15075325965881 0.729838073253632\\
2.16134643554688 0.738739728927612\\
2.1717312335968 0.746206641197205\\
2.18185663223267 0.752267837524414\\
2.20230388641357 0.762538492679596\\
2.21264386177063 0.767454922199249\\
2.22270774841309 0.772260308265686\\
2.23349833488464 0.777447164058685\\
2.24430298805237 0.782548725605011\\
2.25509786605835 0.787428200244904\\
2.26583743095398 0.791951775550842\\
2.27648019790649 0.7960165143013\\
2.29678177833557 0.802417457103729\\
2.30735301971436 0.804985463619232\\
2.32744646072388 0.808435916900635\\
2.34562826156616 0.810013592243195\\
2.36075878143311 0.810389518737793\\
2.3733606338501 0.810078859329224\\
2.38860893249512 0.809010863304138\\
2.40705943107605 0.806161463260651\\
2.41769051551819 0.803072869777679\\
2.42919039726257 0.798441052436829\\
2.4407947063446 0.792471528053284\\
2.45252275466919 0.78523051738739\\
2.46437191963196 0.776840925216675\\
2.47634053230286 0.767454028129578\\
2.48842525482178 0.757237017154694\\
2.50060844421387 0.746374726295471\\
2.51288247108459 0.73504239320755\\
2.52523136138916 0.723420560359955\\
2.53763508796692 0.711685955524445\\
2.55007195472717 0.699997961521149\\
2.56252241134644 0.688497424125671\\
2.57496857643127 0.677303493022919\\
2.58739495277405 0.666511058807373\\
2.59979104995728 0.656353831291199\\
2.61202311515808 0.647194385528564\\
2.62403297424316 0.639329731464386\\
2.63583588600159 0.636150002479553\\
2.64725995063782 0.638070404529572\\
2.65829944610596 0.643946290016174\\
2.66903948783875 0.652319014072418\\
2.68084859848022 0.662393510341644\\
2.69513750076294 0.675918996334076\\
2.71242713928223 0.693264961242676\\
2.7327516078949 0.713853061199188\\
2.74304962158203 0.723939299583435\\
2.75329804420471 0.733607172966003\\
2.76352214813232 0.742822408676147\\
2.77671360969543 0.753786623477936\\
2.79198384284973 0.765558302402496\\
2.81012392044067 0.778243958950043\\
2.82984495162964 0.7897989153862\\
2.84034872055054 0.795110523700714\\
2.85096287727356 0.800727307796478\\
2.86159515380859 0.807240605354309\\
2.87224507331848 0.813998699188232\\
2.88282418251038 0.819493353366852\\
2.89327502250671 0.82299679517746\\
2.90423226356506 0.824577033519745\\
2.91532301902771 0.824596226215363\\
2.92652130126953 0.823707103729248\\
2.93784737586975 0.822374880313873\\
2.94925856590271 0.820825397968292\\
2.96071529388428 0.819130122661591\\
2.97219562530518 0.817319929599762\\
2.98368406295776 0.815415620803833\\
2.99517321586609 0.813416659832001\\
3.00665736198425 0.811287820339203\\
3.018141746521 0.808829605579376\\
3.02963495254517 0.805378913879395\\
3.04114675521851 0.80073493719101\\
3.05272746086121 0.794805824756622\\
3.06440591812134 0.787599384784698\\
3.07619190216064 0.779191017150879\\
3.08806824684143 0.769720613956451\\
3.10001063346863 0.759366571903229\\
3.11200356483459 0.748313188552856\\
3.12402725219727 0.736752152442932\\
3.13606834411621 0.724875271320343\\
3.14812350273132 0.712887644767761\\
3.16019630432129 0.701007485389709\\
3.17221331596375 0.689547657966614\\
3.18425130844116 0.678698837757111\\
3.1963107585907 0.668757975101471\\
3.20831441879272 0.660067141056061\\
3.22024464607239 0.653569519519806\\
3.2320544719696 0.65085232257843\\
3.24364876747131 0.651964008808136\\
3.25497913360596 0.656622290611267\\
3.26603627204895 0.664417147636414\\
3.27683329582214 0.675070106983185\\
3.28738880157471 0.688274800777435\\
3.30090045928955 0.707143545150757\\
3.31724953651428 0.733064949512482\\
3.3364520072937 0.763977468013763\\
3.35590267181396 0.788724601268768\\
3.37547159194946 0.80456668138504\\
3.39097309112549 0.812018096446991\\
3.40369939804077 0.815524220466614\\
3.41909837722778 0.816869139671326\\
3.43773102760315 0.815859615802765\\
3.44846701622009 0.815338432788849\\
3.45963358879089 0.815738797187805\\
3.47084331512451 0.816300868988037\\
3.4820818901062 0.815788686275482\\
3.49336910247803 0.813811302185059\\
3.50474190711975 0.810532748699188\\
3.51621460914612 0.806376278400421\\
3.52777361869812 0.801736891269684\\
3.53940916061401 0.796741425991058\\
3.55113887786865 0.791134297847748\\
3.5629894733429 0.784699440002441\\
3.57491517066956 0.777324438095093\\
3.58686423301697 0.769021987915039\\
3.59883189201355 0.759876608848572\\
3.61085295677185 0.750011324882507\\
3.62294244766235 0.739583373069763\\
3.63507795333862 0.728764295578003\\
3.64724469184875 0.717696487903595\\
3.65939402580261 0.70654833316803\\
3.67149782180786 0.695481717586517\\
3.68352150917053 0.684714615345001\\
3.69545078277588 0.674504101276398\\
3.70728397369385 0.665115237236023\\
3.71901512145996 0.656813383102417\\
3.73064255714417 0.650577366352081\\
3.74213266372681 0.648110270500183\\
3.75340151786804 0.649445056915283\\
3.7644145488739 0.654268085956573\\
3.77517652511597 0.662119626998901\\
3.78571152687073 0.672529995441437\\
3.79605078697205 0.685197651386261\\
3.80622386932373 0.699702262878418\\
3.82096457481384 0.721896111965179\\
3.83880090713501 0.750025570392609\\
3.85830807685852 0.778035581111908\\
3.87783885002136 0.797684550285339\\
3.89773917198181 0.809440314769745\\
3.90787315368652 0.812820613384247\\
3.91816449165344 0.814779281616211\\
3.93745827674866 0.815654277801514\\
3.94821214675903 0.815431535243988\\
3.95906329154968 0.815935730934143\\
3.96999645233154 0.817408263683319\\
3.98096656799316 0.818963408470154\\
3.991947889328 0.819588601589203\\
4.00295209884644 0.818919479846954\\
4.01402473449707 0.817019820213318\\
4.02517938613892 0.81412672996521\\
4.03638935089111 0.810531377792358\\
4.04771900177002 0.806409060955048\\
4.05928993225098 0.801859438419342\\
4.07111120223999 0.796508491039276\\
4.08298397064209 0.790336966514587\\
4.09486865997314 0.783325910568237\\
4.10676383972168 0.775501728057861\\
4.11869525909424 0.766916155815125\\
4.13065099716187 0.757630288600922\\
4.14261865615845 0.747690081596375\\
4.15452194213867 0.737242519855499\\
4.16634511947632 0.726497709751129\\
4.17807626724243 0.715754806995392\\
4.18974590301514 0.705283641815186\\
4.20136594772339 0.695355176925659\\
4.21294355392456 0.686231434345245\\
4.22449111938477 0.678154289722443\\
4.23601484298706 0.671893894672394\\
4.24750518798828 0.668983280658722\\
4.25887632369995 0.669286847114563\\
4.27007293701172 0.672407865524292\\
4.28108787536621 0.677898943424225\\
4.29192972183228 0.685348749160767\\
4.30261182785034 0.694470524787903\\
4.31315040588379 0.70512866973877\\
4.32356023788452 0.716991424560547\\
4.33452129364014 0.729404628276825\\
4.34778356552124 0.745184481143951\\
4.36383104324341 0.764494061470032\\
4.38311052322388 0.784277379512787\\
4.3932032585144 0.79193127155304\\
4.4033842086792 0.797783195972443\\
4.41344928741455 0.801898837089539\\
4.42365694046021 0.804547786712646\\
4.43415594100952 0.805887162685394\\
4.44478559494019 0.806167900562286\\
4.4555401802063 0.805964827537537\\
4.46635627746582 0.806215047836304\\
4.47721815109253 0.807421147823334\\
4.48809051513672 0.808788061141968\\
4.4989447593689 0.80925053358078\\
4.50979852676392 0.808584213256836\\
4.52070713043213 0.806863725185394\\
4.53168821334839 0.80427610874176\\
4.54263353347778 0.801081836223602\\
4.55374670028687 0.797418534755707\\
4.56528568267822 0.793280065059662\\
4.57695484161377 0.78841096162796\\
4.58858823776245 0.782783329486847\\
4.6002082824707 0.776357889175415\\
4.61189317703247 0.769100785255432\\
4.62363004684448 0.761059939861298\\
4.63552665710449 0.75224769115448\\
4.64763736724854 0.742677927017212\\
4.65980005264282 0.732520461082459\\
4.67189693450928 0.721945226192474\\
4.68388748168945 0.711209774017334\\
4.69575357437134 0.700695872306824\\
4.70748949050903 0.690852165222168\\
4.71910238265991 0.682121574878693\\
4.73061180114746 0.674898386001587\\
4.74203300476074 0.671482026576996\\
4.75329446792603 0.672371327877045\\
4.7643346786499 0.676974236965179\\
4.77515077590942 0.684498250484467\\
4.78576755523682 0.694226086139679\\
4.79621171951294 0.705526888370514\\
4.8065128326416 0.717834830284119\\
4.81669855117798 0.730810105800629\\
4.82679224014282 0.744160354137421\\
4.8414249420166 0.76282000541687\\
4.859130859375 0.782685697078705\\
4.87904977798462 0.797951638698578\\
4.88912153244019 0.802902400493622\\
4.89928483963013 0.806248068809509\\
4.90958023071289 0.808123350143433\\
4.92001533508301 0.808634281158447\\
4.93059682846069 0.807890236377716\\
4.94941663742065 0.804432272911072\\
4.9602427482605 0.801936388015747\\
4.97117471694946 0.800463736057281\\
4.98216915130615 0.800458490848541\\
5 0.801366329193115\\
};
\addlegendentry{$\dtmax{} = \expneg{1.0}{0}{[s]}$};

\addplot [
color=black,
dashed
]
table[row sep=crcr]{
0 1\\
0.0118019115179777 1\\
0.0226854234933853 1\\
0.0340500771999359 1\\
0.0491764321923256 1\\
0.0619497485458851 1\\
0.0774054601788521 1\\
0.0961068719625473 1\\
0.106882445514202 1\\
0.118735581636429 1\\
0.13177402317524 1\\
0.146116316318512 1\\
0.161892831325531 1\\
0.179247006773949 1\\
0.198336601257324 1\\
0.21933513879776 1\\
0.242433547973633 1\\
0.267841786146164 1\\
0.29579085111618 1\\
0.32653483748436 1\\
0.360353201627731 1\\
0.397553414106369 1\\
0.438473641872406 1\\
0.483485877513886 1\\
0.532999336719513 1\\
0.587464153766632 1\\
0.647375464439392 1\\
0.71327793598175 1\\
0.7857705950737 1\\
0.865512549877167 1\\
0.953228712081909 1\\
1.04971647262573 0.995161294937134\\
1.14971649646759 0.981202840805054\\
1.24971640110016 0.964120328426361\\
1.30053281784058 0.953730702400208\\
1.33893167972565 0.944734871387482\\
1.37577772140503 0.935036301612854\\
1.41023123264313 0.925137042999268\\
1.44182658195496 0.9154012799263\\
1.47101235389709 0.905854403972626\\
1.49824774265289 0.896464765071869\\
1.52388393878937 0.887201070785522\\
1.54817187786102 0.878042936325073\\
1.57130980491638 0.868972480297089\\
1.59344398975372 0.859978795051575\\
1.6147004365921 0.851049840450287\\
1.63517260551453 0.842179179191589\\
1.65494728088379 0.833357930183411\\
1.67408645153046 0.824583053588867\\
1.69265270233154 0.815847635269165\\
1.710693359375 0.807148039340973\\
1.72825372219086 0.798480331897736\\
1.74536681175232 0.789871454238892\\
1.76200985908508 0.781474113464355\\
1.77794885635376 0.773400783538818\\
1.79327201843262 0.765585124492645\\
1.80806839466095 0.757964074611664\\
1.82245671749115 0.750465869903564\\
1.83652806282043 0.743035733699799\\
1.85030794143677 0.735658407211304\\
1.86384534835815 0.728305518627167\\
1.87714087963104 0.720978081226349\\
1.89023864269257 0.713651835918427\\
1.90313458442688 0.706331849098206\\
1.9158570766449 0.699002921581268\\
1.92840397357941 0.691669166088104\\
1.9407993555069 0.685332894325256\\
1.95291471481323 0.682097434997559\\
1.96465003490448 0.681982517242432\\
1.97588443756104 0.684126496315002\\
1.9866806268692 0.687433362007141\\
1.99778032302856 0.69133734703064\\
2.00900340080261 0.695275604724884\\
2.02036333084106 0.698879957199097\\
2.03189587593079 0.701945126056671\\
2.04355406761169 0.704373180866241\\
2.0552933216095 0.706152737140656\\
2.06707572937012 0.707323253154755\\
2.0788676738739 0.707997798919678\\
2.09061312675476 0.708405375480652\\
2.10223054885864 0.708902955055237\\
2.11370897293091 0.710277557373047\\
2.12503623962402 0.71404242515564\\
2.136150598526 0.721379935741425\\
2.14699602127075 0.730758845806122\\
2.1575915813446 0.739610314369202\\
2.16798114776611 0.746988117694855\\
2.17810916900635 0.752950251102448\\
2.19861006736755 0.763028740882874\\
2.2089672088623 0.767849028110504\\
2.21904635429382 0.772561609745026\\
2.22985172271729 0.777646720409393\\
2.24066185951233 0.782642126083374\\
2.25146412849426 0.787420272827148\\
2.2622447013855 0.791858613491058\\
2.27260613441467 0.795736610889435\\
2.29255986213684 0.801916182041168\\
2.30319738388062 0.804464101791382\\
2.32355046272278 0.807902634143829\\
2.3420250415802 0.809480369091034\\
2.35715985298157 0.809899926185608\\
2.36990928649902 0.809662997722626\\
2.38533616065979 0.808698415756226\\
2.40400266647339 0.805973529815674\\
2.41475820541382 0.802925944328308\\
2.42625689506531 0.798380136489868\\
2.43785905838013 0.792508482933044\\
2.44958400726318 0.785378694534302\\
2.46142792701721 0.777113795280457\\
2.47339081764221 0.767861247062683\\
2.4854679107666 0.757785081863403\\
2.49764442443848 0.747063219547272\\
2.50991177558899 0.73586642742157\\
2.52225399017334 0.724367737770081\\
2.53465104103088 0.712740063667297\\
2.54708194732666 0.701139450073242\\
2.55952668190002 0.689704298973084\\
2.57196760177612 0.678551852703094\\
2.58438992500305 0.667777955532074\\
2.59678220748901 0.657625734806061\\
2.60899806022644 0.648454368114471\\
2.6209888458252 0.640539586544037\\
2.63277387619019 0.637021899223328\\
2.64419460296631 0.638606131076813\\
2.65522933006287 0.644217014312744\\
2.66596651077271 0.65240341424942\\
2.67762064933777 0.662158787250519\\
2.69172215461731 0.675208687782288\\
2.70878481864929 0.691880881786346\\
2.72900438308716 0.711828052997589\\
2.73932576179504 0.721691310405731\\
2.74959468841553 0.731154441833496\\
2.75983572006226 0.740185737609863\\
2.77291202545166 0.750841915607452\\
2.78830885887146 0.762519896030426\\
2.80667853355408 0.775223076343536\\
2.8262140750885 0.78701114654541\\
2.84571003913879 0.797741830348969\\
2.85642600059509 0.804443359375\\
2.86720728874207 0.811687648296356\\
2.87802314758301 0.817861318588257\\
2.88886737823486 0.821977138519287\\
2.89976835250854 0.824001967906952\\
2.91077423095703 0.824383080005646\\
2.92191886901855 0.823754847049713\\
2.9332013130188 0.822613120079041\\
2.94430899620056 0.821269452571869\\
2.9557466506958 0.81975793838501\\
2.9672155380249 0.818131387233734\\
2.97869729995728 0.816402673721313\\
2.99018120765686 0.814574897289276\\
3.00165987014771 0.812633156776428\\
3.01313447952271 0.81043666601181\\
3.02461242675781 0.807852923870087\\
3.03609657287598 0.804229438304901\\
3.04761719703674 0.799383640289307\\
3.05921459197998 0.793252348899841\\
3.07091021537781 0.785865306854248\\
3.08270359039307 0.777320206165314\\
3.09456205368042 0.767776310443878\\
3.10646939277649 0.757400929927826\\
3.11841011047363 0.746369183063507\\
3.13038539886475 0.734845459461212\\
3.14239931106567 0.723006010055542\\
3.15434217453003 0.711172997951508\\
3.16546750068665 0.700289070606232\\
3.17758107185364 0.688842594623566\\
3.189617395401 0.678138613700867\\
3.20145630836487 0.668542802333832\\
3.21335864067078 0.660118103027344\\
3.22526288032532 0.654100954532623\\
3.23705554008484 0.651877880096436\\
3.24862623214722 0.653448581695557\\
3.25993156433105 0.658495485782623\\
3.27096676826477 0.666596531867981\\
3.28174424171448 0.677468240261078\\
3.29228401184082 0.69079601764679\\
3.30546855926514 0.709105372428894\\
3.32142210006714 0.733945071697235\\
3.340411901474 0.763932406902313\\
3.35988259315491 0.788448929786682\\
3.37951111793518 0.804207503795624\\
3.3895115852356 0.809291779994965\\
3.40205717086792 0.813735544681549\\
3.41723728179932 0.816038548946381\\
3.4356050491333 0.8155397772789\\
3.44618844985962 0.814639031887054\\
3.45729732513428 0.814429879188538\\
3.46846532821655 0.815291881561279\\
3.47966051101685 0.816169381141663\\
3.4908709526062 0.815887451171875\\
3.50212430953979 0.814078867435455\\
3.51346206665039 0.810959815979004\\
3.52487444877625 0.8069988489151\\
3.53636074066162 0.80257248878479\\
3.54794073104858 0.797780692577362\\
3.55965638160706 0.79229211807251\\
3.57149004936218 0.785905182361603\\
3.58335900306702 0.77855396270752\\
3.59523272514343 0.770267307758331\\
3.60715103149414 0.761110126972198\\
3.61914253234863 0.751213014125824\\
3.63121247291565 0.740739762783051\\
3.6433367729187 0.729866683483124\\
3.6555004119873 0.718740224838257\\
3.66765594482422 0.707531988620758\\
3.67977452278137 0.696396589279175\\
3.69182777404785 0.685528159141541\\
3.70379018783569 0.675175368785858\\
3.71564841270447 0.665605187416077\\
3.7274022102356 0.657072961330414\\
3.73905062675476 0.65020364522934\\
3.75057649612427 0.647026777267456\\
3.7618887424469 0.64766263961792\\
3.772944688797 0.651867926120758\\
3.78374242782593 0.659224390983582\\
3.79430341720581 0.669286608695984\\
3.80466079711914 0.681766331195831\\
3.8148398399353 0.696270406246185\\
3.82878112792969 0.717297255992889\\
3.84565019607544 0.744315564632416\\
3.86506748199463 0.773511409759521\\
3.8844747543335 0.794669091701508\\
3.90423941612244 0.807747960090637\\
3.91430234909058 0.811704456806183\\
3.92453002929688 0.814187109470367\\
3.94485449790955 0.815711617469788\\
3.95557594299316 0.81569117307663\\
3.96640586853027 0.816253483295441\\
3.97732615470886 0.817738950252533\\
3.98829221725464 0.819298684597015\\
3.99927663803101 0.8199343085289\\
4.0102915763855 0.819298028945923\\
4.0213623046875 0.817460536956787\\
4.03252840042114 0.814645290374756\\
4.04375171661377 0.811136901378632\\
4.05504465103149 0.807148456573486\\
4.06652069091797 0.802746951580048\\
4.07826662063599 0.797543168067932\\
4.09013366699219 0.791480481624603\\
4.10201740264893 0.78456175327301\\
4.11389303207397 0.776817560195923\\
4.12580823898315 0.768294155597687\\
4.13775444030762 0.759063065052032\\
4.14973115921021 0.749175131320953\\
4.16165590286255 0.73876017332077\\
4.17351150512695 0.727992713451386\\
4.18527746200562 0.717156291007996\\
4.1969780921936 0.706516265869141\\
4.20862865447998 0.696336090564728\\
4.22023868560791 0.686873137950897\\
4.23181676864624 0.678363740444183\\
4.24337244033813 0.67101776599884\\
4.25489950180054 0.666756629943848\\
4.26636028289795 0.665884077548981\\
4.27764844894409 0.668061435222626\\
4.28874731063843 0.672789573669434\\
4.2996621131897 0.679592192173004\\
4.31040906906128 0.688120186328888\\
4.32100534439087 0.698228776454926\\
4.33146572113037 0.70968621969223\\
4.34404468536377 0.723746836185455\\
4.3592643737793 0.741794586181641\\
4.37768077850342 0.763797521591187\\
4.38775014877319 0.774831235408783\\
4.39779615402222 0.784202814102173\\
4.40788316726685 0.791758418083191\\
4.41804885864258 0.797546446323395\\
4.42816925048828 0.801643908023834\\
4.4384503364563 0.804276525974274\\
4.44892454147339 0.805598735809326\\
4.45952606201172 0.805891096591949\\
4.47025203704834 0.805743455886841\\
4.48108243942261 0.806086242198944\\
4.49192428588867 0.807396173477173\\
4.50274896621704 0.808818697929382\\
4.51354837417603 0.809278309345245\\
4.52435827255249 0.808516025543213\\
4.53524160385132 0.806708753108978\\
4.54612350463867 0.804100334644318\\
4.55704927444458 0.800905585289001\\
4.56821393966675 0.797218203544617\\
4.57975006103516 0.793026804924011\\
4.59155225753784 0.787966787815094\\
4.60341024398804 0.782008111476898\\
4.61525440216064 0.775165975093842\\
4.62709808349609 0.767480492591858\\
4.63897943496704 0.759010791778564\\
4.65090799331665 0.749877393245697\\
4.66293478012085 0.740144550800323\\
4.67511367797852 0.729819655418396\\
4.68726539611816 0.71910697221756\\
4.69932460784912 0.70824122428894\\
4.71125364303589 0.697589337825775\\
4.72304391860962 0.687602460384369\\
4.73469972610474 0.678761661052704\\
4.74622631072998 0.671527147293091\\
4.75763416290283 0.668424010276794\\
4.76885795593262 0.669960260391235\\
4.77983808517456 0.675454914569855\\
4.79057884216309 0.683999061584473\\
4.80111217498779 0.694799602031708\\
4.81147193908691 0.707173705101013\\
4.82168626785278 0.720496892929077\\
4.83178567886353 0.734402894973755\\
4.84179449081421 0.748568773269653\\
4.85605573654175 0.767548382282257\\
4.87331199645996 0.786561846733093\\
4.89321565628052 0.800960421562195\\
4.90325212478638 0.805472612380981\\
4.91343593597412 0.808429837226868\\
4.92376661300659 0.809934735298157\\
4.93423652648926 0.810089707374573\\
4.94437646865845 0.809108793735504\\
4.96500968933105 0.804499328136444\\
4.97588157653809 0.801755487918854\\
4.9868688583374 0.800155282020569\\
5 0.800595223903656\\
};
\addlegendentry{$\dtmax{} = \expneg{1.0}{1}{[s]}$};

\addplot [
color=black,
dash pattern=on 1pt off 3pt on 3pt off 3pt
]
table[row sep=crcr]{
0 1\\
0.0118019115179777 1\\
0.0226854234933853 1\\
0.0340500771999359 1\\
0.0491764321923256 1\\
0.0619497485458851 1\\
0.0774054601788521 1\\
0.0961068719625473 1\\
0.106106869876385 1\\
0.116106867790222 1\\
0.12610687315464 1\\
0.136106878519058 1\\
0.146106868982315 1\\
0.156106874346733 1\\
0.16610686480999 1\\
0.176106870174408 1\\
0.186106875538826 1\\
0.196106866002083 1\\
0.206106871366501 1\\
0.216106876730919 1\\
0.226106867194176 1\\
0.236106872558594 1\\
0.246106877923012 1\\
0.25610688328743 1\\
0.266106873750687 1\\
0.276106864213943 1\\
0.286106884479523 1\\
0.29610687494278 1\\
0.306106865406036 1\\
0.316106885671616 1\\
0.326106876134872 1\\
0.336106866598129 1\\
0.346106857061386 1\\
0.356106877326965 1\\
0.366106867790222 1\\
0.376106858253479 1\\
0.386106878519058 1\\
0.396106868982315 1\\
0.406106859445572 1\\
0.416106879711151 1\\
0.426106870174408 1\\
0.436106860637665 1\\
0.446106880903244 1\\
0.456106871366501 1\\
0.466106861829758 1\\
0.476106882095337 1\\
0.486106872558594 1\\
0.496106863021851 1\\
0.506106853485107 1\\
0.516106843948364 1\\
0.526106894016266 1\\
0.536106884479523 1\\
0.54610687494278 1\\
0.556106865406036 1\\
0.566106855869293 1\\
0.57610684633255 1\\
0.586106896400452 1\\
0.596106886863709 1\\
0.606106877326965 1\\
0.616106867790222 1\\
0.626106858253479 1\\
0.636106848716736 1\\
0.646106898784637 1\\
0.656106889247894 1\\
0.666106879711151 1\\
0.676106870174408 1\\
0.686106860637665 1\\
0.696106851100922 1\\
0.706106901168823 1\\
0.71610689163208 1\\
0.726106882095337 1\\
0.736106872558594 1\\
0.746106863021851 1\\
0.756106853485107 1\\
0.766106843948364 1\\
0.776106894016266 1\\
0.786106884479523 1\\
0.79610687494278 1\\
0.806106865406036 1\\
0.816106855869293 1\\
0.82610684633255 1\\
0.836106896400452 1\\
0.846106886863709 1\\
0.856106877326965 1\\
0.866106867790222 1\\
0.876106858253479 1\\
0.886106848716736 1\\
0.896106898784637 1\\
0.906106889247894 1\\
0.916106879711151 1\\
0.926106870174408 1\\
0.936106860637665 1\\
0.946106851100922 1\\
0.956106901168823 1\\
0.96610689163208 1\\
0.976106882095337 1\\
0.986106872558594 1\\
0.996106863021851 1\\
1.00610685348511 0.999956965446472\\
1.01610684394836 0.999844312667847\\
1.02610683441162 0.999650895595551\\
1.03610682487488 0.999364256858826\\
1.04610681533813 0.999102711677551\\
1.05610692501068 0.998735845088959\\
1.06610691547394 0.998294591903687\\
1.07610690593719 0.997780919075012\\
1.08610689640045 0.997193872928619\\
1.09610688686371 0.996532380580902\\
1.10610687732697 0.99579530954361\\
1.11610686779022 0.994981467723846\\
1.12610685825348 0.994089782238007\\
1.13610684871674 0.993119180202484\\
1.14610683917999 0.992068588733673\\
1.15610682964325 0.990937054157257\\
1.16610682010651 0.989723563194275\\
1.17610692977905 0.98842716217041\\
1.18610692024231 0.987046897411346\\
1.19610691070557 0.985581874847412\\
1.20610690116882 0.984031200408936\\
1.21610689163208 0.982393980026245\\
1.22610688209534 0.980669438838959\\
1.23610687255859 0.97885674238205\\
1.24610686302185 0.976970553398132\\
1.25610685348511 0.975025713443756\\
1.26610684394836 0.973024845123291\\
1.27610683441162 0.970966815948486\\
1.28610682487488 0.96884959936142\\
1.29610681533813 0.966671526432037\\
1.30610692501068 0.964431345462799\\
1.31610691547394 0.962128043174744\\
1.32610690593719 0.959761023521423\\
1.33610689640045 0.957329988479614\\
1.34610688686371 0.954834640026093\\
1.35610687732697 0.952274918556213\\
1.36610686779022 0.949650704860687\\
1.37610685825348 0.946962058544159\\
1.38610684871674 0.944208979606628\\
1.39610683917999 0.94139152765274\\
1.40610682964325 0.938509702682495\\
1.41610682010651 0.935563623905182\\
1.42610692977905 0.932553350925446\\
1.43610692024231 0.929478883743286\\
1.44610691070557 0.926340341567993\\
1.45610690116882 0.923137664794922\\
1.46610689163208 0.919871032238007\\
1.47610688209534 0.916540324687958\\
1.48610687255859 0.91314572095871\\
1.49610686302185 0.909687161445618\\
1.50610685348511 0.906164646148682\\
1.51610684394836 0.902578234672546\\
1.52610683441162 0.898927927017212\\
1.53610682487488 0.895213663578033\\
1.54610681533813 0.891435563564301\\
1.55610692501068 0.887593448162079\\
1.56610691547394 0.883687496185303\\
1.57610690593719 0.879717528820038\\
1.58610689640045 0.875683605670929\\
1.59610688686371 0.871585667133331\\
1.60610687732697 0.86742377281189\\
1.61610686779022 0.863197803497314\\
1.62610685825348 0.85890781879425\\
1.63610684871674 0.854553639888763\\
1.64610683917999 0.850135445594788\\
1.65610682964325 0.845652937889099\\
1.66610682010651 0.841106355190277\\
1.67610692977905 0.836495339870453\\
1.68610692024231 0.83182018995285\\
1.69610691070557 0.827080547809601\\
1.70610690116882 0.822276651859283\\
1.71610689163208 0.817408204078674\\
1.72610688209534 0.812475323677063\\
1.73610687255859 0.807477831840515\\
1.74610686302185 0.80241584777832\\
1.75610685348511 0.79728901386261\\
1.76610684394836 0.79212361574173\\
1.77610683441162 0.786948323249817\\
1.78610682487488 0.781767070293427\\
1.79610681533813 0.776574015617371\\
1.80610692501068 0.771362125873566\\
1.81610691547394 0.766123354434967\\
1.82610690593719 0.760851204395294\\
1.83610689640045 0.755539238452911\\
1.84610688686371 0.750183165073395\\
1.85610687732697 0.744778275489807\\
1.86610686779022 0.739321291446686\\
1.87610685825348 0.733808696269989\\
1.88610684871674 0.728238642215729\\
1.89610683917999 0.722608327865601\\
1.90610682964325 0.716916382312775\\
1.91610682010651 0.711160659790039\\
1.92610692977905 0.705340147018433\\
1.93610692024231 0.699453175067902\\
1.94610691070557 0.693498730659485\\
1.95610690116882 0.688878417015076\\
1.96610689163208 0.686013281345367\\
1.97610688209534 0.685046792030334\\
1.98610687255859 0.685810148715973\\
1.99610686302185 0.687853157520294\\
2.00610685348511 0.690696775913239\\
2.01610684394836 0.693881452083588\\
2.02610683441162 0.697031676769257\\
2.03610682487488 0.699911773204803\\
2.04610681533813 0.702396750450134\\
2.05610680580139 0.704437375068665\\
2.06610679626465 0.706028699874878\\
2.07610678672791 0.707190573215485\\
2.08610677719116 0.70796012878418\\
2.09610676765442 0.70842307806015\\
2.10610675811768 0.708731114864349\\
2.11610698699951 0.709120392799377\\
2.12610697746277 0.710074186325073\\
2.13610696792603 0.712531864643097\\
2.14610695838928 0.717687129974365\\
2.15610694885254 0.725770473480225\\
2.1661069393158 0.735181629657745\\
2.17610692977905 0.743984639644623\\
2.18610692024231 0.751362264156342\\
2.20580863952637 0.762390494346619\\
2.21580862998962 0.76704865694046\\
2.22580862045288 0.771529853343964\\
2.23580861091614 0.775954365730286\\
2.24580860137939 0.780333817005157\\
2.25580859184265 0.784617364406586\\
2.26580858230591 0.788728713989258\\
2.27580857276917 0.792590379714966\\
2.28580856323242 0.79613471031189\\
2.30537819862366 0.801970720291138\\
2.31537818908691 0.8043332695961\\
2.32537817955017 0.806260168552399\\
2.33537817001343 0.807749569416046\\
2.35460209846497 0.809431731700897\\
2.36516523361206 0.809803783893585\\
2.3773078918457 0.809759616851807\\
2.39200067520142 0.809092998504639\\
2.40977883338928 0.807324230670929\\
2.41977882385254 0.805356740951538\\
2.4297788143158 0.802433729171753\\
2.43977880477905 0.798523426055908\\
2.44977879524231 0.793653249740601\\
2.45977878570557 0.78788959980011\\
2.46977877616882 0.781322777271271\\
2.47977900505066 0.774055361747742\\
2.48977899551392 0.766194820404053\\
2.49977898597717 0.757847905158997\\
2.50977897644043 0.749117851257324\\
2.51977896690369 0.740102171897888\\
2.52977895736694 0.730891168117523\\
2.5397789478302 0.721573054790497\\
2.54977893829346 0.712226033210754\\
2.55977892875671 0.702919542789459\\
2.56977891921997 0.693714678287506\\
2.57977890968323 0.684664607048035\\
2.58977890014648 0.675815641880035\\
2.59977889060974 0.6672243475914\\
2.609778881073 0.659029841423035\\
2.61977887153625 0.651364207267761\\
2.62977886199951 0.644376993179321\\
2.63977885246277 0.638965427875519\\
2.64977884292603 0.637212872505188\\
2.65977883338928 0.639246344566345\\
2.66977882385254 0.644522905349731\\
2.6797788143158 0.652118504047394\\
2.69397473335266 0.664411604404449\\
2.71014857292175 0.679757535457611\\
2.72946786880493 0.698830008506775\\
2.73946785926819 0.708608984947205\\
2.74946784973145 0.718158185482025\\
2.7594678401947 0.727396726608276\\
2.76946783065796 0.736265420913696\\
2.78308033943176 0.747479975223541\\
2.79642605781555 0.757693529129028\\
2.81257438659668 0.769148290157318\\
2.83187890052795 0.781566798686981\\
2.85093808174133 0.792942047119141\\
2.86093807220459 0.799354195594788\\
2.87093806266785 0.80668169260025\\
2.8809380531311 0.814186632633209\\
2.89093804359436 0.820585906505585\\
2.9009382724762 0.825098633766174\\
2.91093826293945 0.827679574489594\\
2.92093825340271 0.828702688217163\\
2.93093824386597 0.828658044338226\\
2.94093823432922 0.827964246273041\\
2.95093822479248 0.826874732971191\\
2.96093821525574 0.825526833534241\\
2.97093820571899 0.823988378047943\\
2.98093819618225 0.822294950485229\\
2.99093818664551 0.820473611354828\\
3.00093817710876 0.818543136119843\\
3.01093816757202 0.816509068012238\\
3.02093815803528 0.814280390739441\\
3.03093814849854 0.811753571033478\\
3.04093813896179 0.808503210544586\\
3.05093812942505 0.804398715496063\\
3.06093811988831 0.799385905265808\\
3.07093811035156 0.793475210666656\\
3.08093810081482 0.786719083786011\\
3.09093809127808 0.779196500778198\\
3.10093808174133 0.77099996805191\\
3.11093807220459 0.762227535247803\\
3.12093806266785 0.752978444099426\\
3.1309380531311 0.743351519107819\\
3.14093804359436 0.733451008796692\\
3.1509382724762 0.72339141368866\\
3.16093826293945 0.71329653263092\\
3.17093825340271 0.70329761505127\\
3.18093824386597 0.69353461265564\\
3.19093823432922 0.684153318405151\\
3.20093822479248 0.675312578678131\\
3.21093821525574 0.667177855968475\\
3.22093820571899 0.659914374351501\\
3.23093819618225 0.654005706310272\\
3.24093818664551 0.650746464729309\\
3.25093817710876 0.650305449962616\\
3.26093816757202 0.652706801891327\\
3.27093815803528 0.65788596868515\\
3.28093814849854 0.66572093963623\\
3.29093813896179 0.676148355007172\\
3.30093812942505 0.68905383348465\\
3.31571698188782 0.710605263710022\\
3.33297657966614 0.738640487194061\\
3.35262060165405 0.769679307937622\\
3.37212371826172 0.792487025260925\\
3.39176940917969 0.806650340557098\\
3.40641617774963 0.812799870967865\\
3.4179515838623 0.815607488155365\\
3.43190956115723 0.81671804189682\\
3.44879865646362 0.81596714258194\\
3.46852993965149 0.81590348482132\\
3.47852993011475 0.816891551017761\\
3.488529920578 0.81759387254715\\
3.49852991104126 0.817313194274902\\
3.50852990150452 0.81586217880249\\
3.51852989196777 0.813408255577087\\
3.52852988243103 0.810251891613007\\
3.53852987289429 0.806672990322113\\
3.54852986335754 0.802799463272095\\
3.5585298538208 0.798711895942688\\
3.56852984428406 0.794169247150421\\
3.57852983474731 0.789050340652466\\
3.58852982521057 0.783281743526459\\
3.59852981567383 0.776849329471588\\
3.60852980613709 0.769788086414337\\
3.61852979660034 0.762165725231171\\
3.62853002548218 0.75407886505127\\
3.63853001594543 0.745622575283051\\
3.64853000640869 0.736882925033569\\
3.65852999687195 0.727932155132294\\
3.66852998733521 0.718838036060333\\
3.67852997779846 0.709664642810822\\
3.68852996826172 0.70048463344574\\
3.69852995872498 0.691396176815033\\
3.70852994918823 0.682524919509888\\
3.71852993965149 0.674022018909454\\
3.72852993011475 0.666045904159546\\
3.738529920578 0.658757925033569\\
3.74852991104126 0.652327477931976\\
3.75852990150452 0.648453831672668\\
3.76852989196777 0.64752334356308\\
3.77852988243103 0.649580836296082\\
3.78852987289429 0.654496192932129\\
3.79852986335754 0.662030577659607\\
3.8085298538208 0.67190158367157\\
3.81852984428406 0.683942794799805\\
3.82852983474731 0.697884976863861\\
3.84355807304382 0.720107853412628\\
3.85999274253845 0.745204627513886\\
3.87921619415283 0.772725164890289\\
3.89872860908508 0.793408215045929\\
3.91853308677673 0.80622935295105\\
3.92853307723999 0.8100705742836\\
3.93853306770325 0.812432169914246\\
3.94853329658508 0.813512802124023\\
3.96640586853027 0.813192844390869\\
3.97640585899353 0.812332332134247\\
3.98640584945679 0.81198662519455\\
3.99640583992004 0.812609910964966\\
4.0064058303833 0.813860952854156\\
4.01640605926514 0.814874351024628\\
4.02640581130981 0.815016448497772\\
4.03640604019165 0.814122557640076\\
4.04640579223633 0.812283396720886\\
4.05640602111816 0.80969625711441\\
4.06640577316284 0.80657833814621\\
4.07640600204468 0.803087413311005\\
4.08640575408936 0.799334645271301\\
4.09640598297119 0.795072793960571\\
4.10640573501587 0.790241003036499\\
4.11640596389771 0.784802556037903\\
4.12640619277954 0.778752207756042\\
4.13640594482422 0.772121012210846\\
4.14640617370605 0.764955699443817\\
4.15640592575073 0.757294893264771\\
4.16640615463257 0.749167501926422\\
4.17640590667725 0.740612804889679\\
4.18640613555908 0.731719255447388\\
4.19640588760376 0.722633957862854\\
4.2064061164856 0.713544487953186\\
4.21640586853027 0.704649925231934\\
4.22640609741211 0.69614839553833\\
4.23640584945679 0.688233196735382\\
4.24640607833862 0.681093871593475\\
4.2564058303833 0.674999237060547\\
4.26640605926514 0.671409845352173\\
4.27640581130981 0.670398354530334\\
4.28640604019165 0.671854138374329\\
4.29640579223633 0.675595462322235\\
4.30640602111816 0.681405663490295\\
4.31640577316284 0.689053297042847\\
4.32640600204468 0.698309004306793\\
4.33640575408936 0.709011077880859\\
4.34640598297119 0.72091281414032\\
4.36180782318115 0.739965200424194\\
4.38015365600586 0.763446688652039\\
4.39015340805054 0.775131523609161\\
4.40015363693237 0.784979164600372\\
4.41015338897705 0.792850732803345\\
4.42015361785889 0.798846364021301\\
4.43015336990356 0.803141295909882\\
4.4401535987854 0.805925250053406\\
4.45015335083008 0.807398319244385\\
4.46015357971191 0.807812929153442\\
4.47015333175659 0.807574093341827\\
4.48015356063843 0.807344913482666\\
4.49015331268311 0.807849526405334\\
4.50015354156494 0.809225559234619\\
4.51015329360962 0.810811698436737\\
4.52015352249146 0.811743021011353\\
4.53015327453613 0.811646521091461\\
4.54015350341797 0.81048971414566\\
4.55015325546265 0.808405816555023\\
4.56015348434448 0.805594444274902\\
4.57015323638916 0.80226081609726\\
4.580153465271 0.798520267009735\\
4.59015321731567 0.794547736644745\\
4.60015344619751 0.790102958679199\\
4.61015367507935 0.785140573978424\\
4.62015342712402 0.779634475708008\\
4.63015365600586 0.773586273193359\\
4.64015340805054 0.767021834850311\\
4.65015363693237 0.759997487068176\\
4.66015338897705 0.752563536167145\\
4.67015361785889 0.744746804237366\\
4.68015336990356 0.736557006835938\\
4.6901535987854 0.728027641773224\\
4.70015335083008 0.719255805015564\\
4.71015357971191 0.710406720638275\\
4.72015333175659 0.701692521572113\\
4.73015356063843 0.693352699279785\\
4.74015331268311 0.685628116130829\\
4.75015354156494 0.678747177124023\\
4.76015329360962 0.673375725746155\\
4.77015352249146 0.670919716358185\\
4.78015327453613 0.671354174613953\\
4.79015350341797 0.67444121837616\\
4.80015325546265 0.679882228374481\\
4.81015348434448 0.687357306480408\\
4.82015323638916 0.696516096591949\\
4.830153465271 0.7070352435112\\
4.84015321731567 0.718628644943237\\
4.85015344619751 0.73105251789093\\
4.86591100692749 0.750853538513184\\
4.88213491439819 0.770575642585754\\
4.90140724182129 0.78909558057785\\
4.92139911651611 0.800952315330505\\
4.93139886856079 0.804397940635681\\
4.94139909744263 0.80641770362854\\
4.9513988494873 0.807182729244232\\
4.96139907836914 0.806907296180725\\
4.97139883041382 0.805934011936188\\
4.98139905929565 0.804864048957825\\
5 0.805012345314026\\
};
\addlegendentry{$\dtmax{} = \expneg{1.0}{2}{[s]}$};

\addplot [
color=black,
dashed
]
table[row sep=crcr]{
0 1\\
0.0109122171998024 1\\
0.020912216976285 1\\
0.0309122167527676 1\\
0.0409122183918953 1\\
0.0509122163057327 1\\
0.0609122179448605 1\\
0.0709122195839882 1\\
0.0809122174978256 1\\
0.0909122154116631 1\\
0.100912220776081 1\\
0.110912218689919 1\\
0.120912216603756 1\\
0.130912214517593 1\\
0.140912219882011 1\\
0.150912210345268 1\\
0.160912215709686 1\\
0.170912221074104 1\\
0.180912211537361 1\\
0.190912216901779 1\\
0.200912222266197 1\\
0.210912212729454 1\\
0.220912218093872 1\\
0.23091222345829 1\\
0.240912213921547 1\\
0.250912219285965 1\\
0.260912209749222 1\\
0.270912230014801 1\\
0.280912220478058 1\\
0.290912210941315 1\\
0.300912231206894 1\\
0.310912221670151 1\\
0.320912212133408 1\\
0.330912202596664 1\\
0.340912222862244 1\\
0.3509122133255 1\\
0.360912203788757 1\\
0.370912224054337 1\\
0.380912214517593 1\\
0.39091220498085 1\\
0.400912225246429 1\\
0.410912215709686 1\\
0.420912206172943 1\\
0.430912226438522 1\\
0.440912216901779 1\\
0.450912207365036 1\\
0.460912227630615 1\\
0.470912218093872 1\\
0.480912208557129 1\\
0.490912228822708 1\\
0.500912189483643 1\\
0.510912239551544 1\\
0.520912230014801 1\\
0.530912220478058 1\\
0.540912210941315 1\\
0.550912201404572 1\\
0.560912191867828 1\\
0.57091224193573 1\\
0.580912232398987 1\\
0.590912222862244 1\\
0.6009122133255 1\\
0.610912203788757 1\\
0.620912194252014 1\\
0.630912244319916 1\\
0.640912234783173 1\\
0.650912225246429 1\\
0.660912215709686 1\\
0.670912206172943 1\\
0.6809121966362 1\\
0.690912246704102 1\\
0.700912237167358 1\\
0.710912227630615 1\\
0.720912218093872 1\\
0.730912208557129 1\\
0.740912199020386 1\\
0.750912189483643 1\\
0.760912239551544 1\\
0.770912230014801 1\\
0.780912220478058 1\\
0.790912210941315 1\\
0.800912201404572 1\\
0.810912191867828 1\\
0.82091224193573 1\\
0.830912232398987 1\\
0.840912222862244 1\\
0.8509122133255 1\\
0.860912203788757 1\\
0.870912194252014 1\\
0.880912244319916 1\\
0.890912234783173 1\\
0.900912225246429 1\\
0.910912215709686 1\\
0.920912206172943 1\\
0.9309121966362 1\\
0.940912246704102 1\\
0.950912237167358 1\\
0.960912227630615 1\\
0.970912218093872 1\\
0.980912208557129 1\\
0.990912199020386 1\\
1.00091218948364 0.999999403953552\\
1.0109121799469 0.999951004981995\\
1.02091217041016 0.999801218509674\\
1.03091216087341 0.999619722366333\\
1.04091227054596 0.99938303232193\\
1.05091226100922 0.999078094959259\\
1.06091225147247 0.998703598976135\\
1.07091224193573 0.998258233070374\\
1.08091223239899 0.997740685939789\\
1.09091222286224 0.997149765491486\\
1.1009122133255 0.996484279632568\\
1.11091220378876 0.995742976665497\\
1.12091219425201 0.99492484331131\\
1.13091218471527 0.994028747081757\\
1.14091217517853 0.993053555488586\\
1.15091216564178 0.991998314857483\\
1.16091227531433 0.990862011909485\\
1.17091226577759 0.989643633365631\\
1.18091225624084 0.988342225551605\\
1.1909122467041 0.986956894397736\\
1.20091223716736 0.985486745834351\\
1.21091222763062 0.983930826187134\\
1.22091221809387 0.982288360595703\\
1.23091220855713 0.980558454990387\\
1.24091219902039 0.978742241859436\\
1.25091218948364 0.976857841014862\\
1.2609121799469 0.974917590618134\\
1.27091217041016 0.972922563552856\\
1.28091216087341 0.970870435237885\\
1.29091227054596 0.968758881092072\\
1.30091226100922 0.966585874557495\\
1.31091225147247 0.964350044727325\\
1.32091224193573 0.962050497531891\\
1.33091223239899 0.959686756134033\\
1.34091222286224 0.957258403301239\\
1.3509122133255 0.954765439033508\\
1.36091220378876 0.952207684516907\\
1.37091219425201 0.949585139751434\\
1.38091218471527 0.946897983551025\\
1.39091217517853 0.94414609670639\\
1.40091216564178 0.941329717636108\\
1.41091227531433 0.938448846340179\\
1.42091226577759 0.935503602027893\\
1.43091225624084 0.932493984699249\\
1.4409122467041 0.929420113563538\\
1.45091223716736 0.926282048225403\\
1.46091222763062 0.923079788684845\\
1.47091221809387 0.919813454151154\\
1.48091220855713 0.916483104228973\\
1.49091219902039 0.913088619709015\\
1.50091218948364 0.909630179405212\\
1.5109121799469 0.906107723712921\\
1.52091217041016 0.902521312236786\\
1.53091216087341 0.898870885372162\\
1.54091227054596 0.895156562328339\\
1.55091226100922 0.891378223896027\\
1.56091225147247 0.887535870075226\\
1.57091224193573 0.883629560470581\\
1.58091223239899 0.879659295082092\\
1.59091222286224 0.87562495470047\\
1.6009122133255 0.871526598930359\\
1.61091220378876 0.867364168167114\\
1.62091219425201 0.863137662410736\\
1.63091218471527 0.85884702205658\\
1.64091217517853 0.854492247104645\\
1.65091216564178 0.850073218345642\\
1.66091227531433 0.845590054988861\\
1.67091226577759 0.841042578220367\\
1.68091225624084 0.836430788040161\\
1.6909122467041 0.831754684448242\\
1.70091223716736 0.827014148235321\\
1.71091222763062 0.822209239006042\\
1.72091221809387 0.817339777946472\\
1.73091220855713 0.812405824661255\\
1.74091219902039 0.807407259941101\\
1.75091218948364 0.802344024181366\\
1.7609121799469 0.797218143939972\\
1.77091217041016 0.792059898376465\\
1.78091216087341 0.786895573139191\\
1.79091227054596 0.781727075576782\\
1.80091226100922 0.776547908782959\\
1.81091225147247 0.771349489688873\\
1.82091224193573 0.766123652458191\\
1.83091223239899 0.76086300611496\\
1.84091222286224 0.755561470985413\\
1.8509122133255 0.750213861465454\\
1.86091220378876 0.74481600522995\\
1.87091219425201 0.739364266395569\\
1.88091218471527 0.733855724334717\\
1.89091217517853 0.728287994861603\\
1.90091216564178 0.72265899181366\\
1.91091227531433 0.716966867446899\\
1.92091226577759 0.711210131645203\\
1.93091225624084 0.705387473106384\\
1.9409122467041 0.699497580528259\\
1.95091223716736 0.693539321422577\\
1.96091222763062 0.688261330127716\\
1.97091221809387 0.684676289558411\\
1.98091220855713 0.68313455581665\\
1.99091219902039 0.683633327484131\\
2.00091218948364 0.685791969299316\\
2.0109121799469 0.689072847366333\\
2.02091217041016 0.692882001399994\\
2.03091216087341 0.696716547012329\\
2.04091215133667 0.700252950191498\\
2.05091214179993 0.703319549560547\\
2.06091213226318 0.705849945545197\\
2.07091212272644 0.707840502262115\\
2.0809121131897 0.709323585033417\\
2.09091210365295 0.710365235805511\\
2.10091233253479 0.71111661195755\\
2.11091232299805 0.711830615997314\\
2.1209123134613 0.712979018688202\\
2.13091230392456 0.715432405471802\\
2.14091229438782 0.720313191413879\\
2.15091228485107 0.728042781352997\\
2.16091227531433 0.737618863582611\\
2.17091226577759 0.747379541397095\\
2.18091225624084 0.756090223789215\\
2.1909122467041 0.763356268405914\\
2.20091223716736 0.769346177577972\\
2.21091222763062 0.774413049221039\\
2.22091221809387 0.778878509998322\\
2.23091220855713 0.782964944839478\\
2.24091219902039 0.786796092987061\\
2.25091218948364 0.790422022342682\\
2.2609121799469 0.793846428394318\\
2.27091217041016 0.797047972679138\\
2.28091216087341 0.79999440908432\\
2.29091215133667 0.80265086889267\\
2.30091214179993 0.804982662200928\\
2.31091213226318 0.806957900524139\\
2.32091212272644 0.808556258678436\\
2.3309121131897 0.809782445430756\\
2.34091210365295 0.81066220998764\\
2.35091233253479 0.811224102973938\\
2.36091232299805 0.81148886680603\\
2.3709123134613 0.811468183994293\\
2.38091230392456 0.811168253421783\\
2.39091229438782 0.810576319694519\\
2.40091228485107 0.809367299079895\\
2.41091227531433 0.807311832904816\\
2.42091226577759 0.804319024085999\\
2.43091225624084 0.80035924911499\\
2.4409122467041 0.795454502105713\\
2.45091223716736 0.789664506912231\\
2.46091222763062 0.783074140548706\\
2.47091221809387 0.775783538818359\\
2.48091220855713 0.76789927482605\\
2.49091219902039 0.759528577327728\\
2.50091218948364 0.750774919986725\\
2.5109121799469 0.741735994815826\\
2.52091217041016 0.732505083084106\\
2.53091216087341 0.723168313503265\\
2.54091215133667 0.713801383972168\\
2.55091214179993 0.704470634460449\\
2.56091213226318 0.695233166217804\\
2.57091212272644 0.686137795448303\\
2.5809121131897 0.677226960659027\\
2.59091210365295 0.668567776679993\\
2.60091233253479 0.660272777080536\\
2.61091232299805 0.65245521068573\\
2.6209123134613 0.645246148109436\\
2.63091230392456 0.639041662216187\\
2.64091229438782 0.636090457439423\\
2.65091228485107 0.637251973152161\\
2.66091227531433 0.642246007919312\\
2.67091226577759 0.650151193141937\\
2.68091225624084 0.659752190113068\\
2.6909122467041 0.669918417930603\\
2.70091223716736 0.680207252502441\\
2.71091222763062 0.69042581319809\\
2.72091221809387 0.70045417547226\\
2.73091220855713 0.710199534893036\\
2.74091219902039 0.719590365886688\\
2.75091218948364 0.72857403755188\\
2.7609121799469 0.737115502357483\\
2.77091217041016 0.745195627212524\\
2.78091216087341 0.752812802791595\\
2.79091215133667 0.760003983974457\\
2.80091214179993 0.76685905456543\\
2.81091213226318 0.773537576198578\\
2.82091212272644 0.780348539352417\\
2.8309121131897 0.787751853466034\\
2.84091210365295 0.796122550964355\\
2.85091233253479 0.805265367031097\\
2.86091232299805 0.814260721206665\\
2.8709123134613 0.822083175182343\\
2.88091230392456 0.828176319599152\\
2.89091229438782 0.832477152347565\\
2.90091228485107 0.835188329219818\\
2.91091227531433 0.836589634418488\\
2.92091226577759 0.836945414543152\\
2.93091225624084 0.836475133895874\\
2.9409122467041 0.835353910923004\\
2.95091223716736 0.833726823329926\\
2.96091222763062 0.831713676452637\\
2.97091221809387 0.829408824443817\\
2.98091220855713 0.82688307762146\\
2.99091219902039 0.824180543422699\\
3.00091218948364 0.821102619171143\\
3.0109121799469 0.817370533943176\\
3.02091217041016 0.812869310379028\\
3.03091216087341 0.807524681091309\\
3.04091215133667 0.801318645477295\\
3.05091214179993 0.794282615184784\\
3.06091213226318 0.786482155323029\\
3.07091212272644 0.77800452709198\\
3.0809121131897 0.768946826457977\\
3.09091210365295 0.759409487247467\\
3.10091233253479 0.749497413635254\\
3.11091232299805 0.739310681819916\\
3.1209123134613 0.728944659233093\\
3.13091230392456 0.71849799156189\\
3.14091229438782 0.708077251911163\\
3.15091228485107 0.697797119617462\\
3.16091227531433 0.687777698040009\\
3.17091226577759 0.678144693374634\\
3.18091225624084 0.669030487537384\\
3.1909122467041 0.66057550907135\\
3.20091223716736 0.652926743030548\\
3.21091222763062 0.646330118179321\\
3.22091221809387 0.642163872718811\\
3.23091220855713 0.640998423099518\\
3.24091219902039 0.64290452003479\\
3.25091218948364 0.647839605808258\\
3.2609121799469 0.655662417411804\\
3.27091217041016 0.666186630725861\\
3.28091216087341 0.679280638694763\\
3.29091215133667 0.694531917572021\\
3.30091214179993 0.711003422737122\\
3.31091213226318 0.727971911430359\\
3.32091212272644 0.744828283786774\\
3.3309121131897 0.761016726493835\\
3.34091210365295 0.775664687156677\\
3.35091233253479 0.788127362728119\\
3.36091232299805 0.798281013965607\\
3.3709123134613 0.806279122829437\\
3.38091230392456 0.812374949455261\\
3.39091229438782 0.816839754581451\\
3.40091228485107 0.820004522800446\\
3.41091227531433 0.822325766086578\\
3.42091226577759 0.824423730373383\\
3.43091225624084 0.826865673065186\\
3.4409122467041 0.829745173454285\\
3.45091223716736 0.832590341567993\\
3.46091222763062 0.834736764431\\
3.47091221809387 0.835735559463501\\
3.48091220855713 0.835468173027039\\
3.49091219902039 0.83406013250351\\
3.50091218948364 0.831749856472015\\
3.5109121799469 0.828769266605377\\
3.52091217041016 0.825304627418518\\
3.53091216087341 0.82144021987915\\
3.54091215133667 0.816978633403778\\
3.55091214179993 0.811868607997894\\
3.56091213226318 0.806085169315338\\
3.57091212272644 0.799628078937531\\
3.5809121131897 0.792529404163361\\
3.59091210365295 0.784848511219025\\
3.60091233253479 0.776669323444366\\
3.61091232299805 0.768085777759552\\
3.6209123134613 0.759184896945953\\
3.63091230392456 0.75004118680954\\
3.64091229438782 0.7407106757164\\
3.65091228485107 0.73123562335968\\
3.66091227531433 0.72165858745575\\
3.67091226577759 0.712039709091187\\
3.68091225624084 0.702459394931793\\
3.6909122467041 0.693011462688446\\
3.70091223716736 0.683795988559723\\
3.71091222763062 0.674915850162506\\
3.72091221809387 0.666474461555481\\
3.73091220855713 0.658576607704163\\
3.74091219902039 0.651327311992645\\
3.75091218948364 0.64500480890274\\
3.7609121799469 0.64108270406723\\
3.77091217041016 0.639950692653656\\
3.78091216087341 0.641586542129517\\
3.79091215133667 0.645866692066193\\
3.80091214179993 0.652600467205048\\
3.81091213226318 0.661646246910095\\
3.82091212272644 0.673020362854004\\
3.8309121131897 0.686289608478546\\
3.84091210365295 0.700617015361786\\
3.85091233253479 0.715462625026703\\
3.86091232299805 0.7303666472435\\
3.8709123134613 0.744964122772217\\
3.88091230392456 0.758974373340607\\
3.89091229438782 0.77200973033905\\
3.90091228485107 0.783465564250946\\
3.91091227531433 0.793035209178925\\
3.92091226577759 0.800721347332001\\
3.93091225624084 0.806713283061981\\
3.9409122467041 0.81134021282196\\
3.95091223716736 0.815109252929688\\
3.96091222763062 0.818694710731506\\
3.97091221809387 0.82262521982193\\
3.98091220855713 0.826843559741974\\
3.99091219902039 0.830741703510284\\
4.00091218948364 0.833652138710022\\
4.01091241836548 0.835217416286469\\
4.02091217041016 0.835407972335815\\
4.03091239929199 0.834388434886932\\
4.04091215133667 0.832411885261536\\
4.05091238021851 0.829723954200745\\
4.06091213226318 0.826514422893524\\
4.07091236114502 0.822921276092529\\
4.0809121131897 0.818938374519348\\
4.09091234207153 0.814403533935547\\
4.10091209411621 0.809287130832672\\
4.11091232299805 0.803566098213196\\
4.12091207504272 0.797240078449249\\
4.13091230392456 0.790334463119507\\
4.14091205596924 0.782903075218201\\
4.15091228485107 0.775016605854034\\
4.16091203689575 0.766728520393372\\
4.17091226577759 0.75806999206543\\
4.18091201782227 0.749068975448608\\
4.1909122467041 0.739779233932495\\
4.20091199874878 0.730294346809387\\
4.21091222763062 0.720740973949432\\
4.22091197967529 0.711266160011292\\
4.23091220855713 0.702022254467011\\
4.24091243743896 0.69315642118454\\
4.25091218948364 0.684808015823364\\
4.26091241836548 0.677113354206085\\
4.27091217041016 0.670197010040283\\
4.28091239929199 0.664705753326416\\
4.29091215133667 0.661830246448517\\
4.30091238021851 0.661508798599243\\
4.31091213226318 0.663547217845917\\
4.32091236114502 0.66770988702774\\
4.3309121131897 0.673743903636932\\
4.34091234207153 0.681405484676361\\
4.35091209411621 0.690530240535736\\
4.36091232299805 0.701035737991333\\
4.37091207504272 0.712521851062775\\
4.38091230392456 0.724452614784241\\
4.39091205596924 0.736521184444427\\
4.40091228485107 0.748486042022705\\
4.41091203689575 0.760156214237213\\
4.42091226577759 0.771245837211609\\
4.43091201782227 0.781173348426819\\
4.4409122467041 0.78956013917923\\
4.45091199874878 0.796321749687195\\
4.46091222763062 0.801571071147919\\
4.47091197967529 0.805581986904144\\
4.48091220855713 0.808829784393311\\
4.49091243743896 0.811987340450287\\
4.50091218948364 0.815616488456726\\
4.51091241836548 0.819686412811279\\
4.52091217041016 0.823569774627686\\
4.53091239929199 0.82653284072876\\
4.54091215133667 0.828150391578674\\
4.55091238021851 0.828352749347687\\
4.56091213226318 0.827287495136261\\
4.57091236114502 0.825206398963928\\
4.5809121131897 0.822381734848022\\
4.59091234207153 0.819017887115479\\
4.60091209411621 0.815209329128265\\
4.61091232299805 0.810791432857513\\
4.62091207504272 0.805740714073181\\
4.63091230392456 0.800055325031281\\
4.64091205596924 0.793749213218689\\
4.65091228485107 0.786859154701233\\
4.66091203689575 0.77944141626358\\
4.67091226577759 0.771568834781647\\
4.68091201782227 0.763330519199371\\
4.6909122467041 0.754799783229828\\
4.70091199874878 0.746013581752777\\
4.71091222763062 0.736984014511108\\
4.72091197967529 0.727736711502075\\
4.73091220855713 0.718344151973724\\
4.74091243743896 0.708929300308228\\
4.75091218948364 0.699649810791016\\
4.76091241836548 0.690683662891388\\
4.77091217041016 0.682224154472351\\
4.78091239929199 0.674480795860291\\
4.79091215133667 0.667661845684052\\
4.80091238021851 0.662885665893555\\
4.81091213226318 0.661472976207733\\
4.82091236114502 0.663296341896057\\
4.8309121131897 0.668055832386017\\
4.84091234207153 0.675382316112518\\
4.85091209411621 0.684880673885345\\
4.86091232299805 0.69611793756485\\
4.87091207504272 0.708604276180267\\
4.88091230392456 0.72198748588562\\
4.89091205596924 0.735850989818573\\
4.90091228485107 0.749596655368805\\
4.91091203689575 0.762883424758911\\
4.92091226577759 0.77516371011734\\
4.93091201782227 0.785801708698273\\
4.9409122467041 0.794553577899933\\
4.95091199874878 0.80144214630127\\
4.96091222763062 0.806610763072968\\
4.97091197967529 0.810247600078583\\
4.98091220855713 0.812573254108429\\
4.99091243743896 0.813886106014252\\
5 0.81459504365921\\
};
\addlegendentry{$\dtmax{} = \expneg{1.0}{3}{[s]}$};

\addplot [
color=black,
solid
]
table[row sep=crcr]{
0 1\\
0.0100556658580899 1\\
0.0200556665658951 1\\
0.0300556663423777 1\\
0.0400556661188602 1\\
0.050055667757988 1\\
0.0600556656718254 1\\
0.0700556635856628 1\\
0.0800556689500809 1\\
0.0900556668639183 1\\
0.100055664777756 1\\
0.110055662691593 1\\
0.120055668056011 1\\
0.130055665969849 1\\
0.140055671334267 1\\
0.150055661797524 1\\
0.160055667161942 1\\
0.17005567252636 1\\
0.180055662989616 1\\
0.190055668354034 1\\
0.200055658817291 1\\
0.210055664181709 1\\
0.220055669546127 1\\
0.230055660009384 1\\
0.240055665373802 1\\
0.25005567073822 1\\
0.260055661201477 1\\
0.270055651664734 1\\
0.280055671930313 1\\
0.29005566239357 1\\
0.300055652856827 1\\
0.310055673122406 1\\
0.320055663585663 1\\
0.33005565404892 1\\
0.340055674314499 1\\
0.350055664777756 1\\
0.360055655241013 1\\
0.370055675506592 1\\
0.380055665969849 1\\
0.390055656433105 1\\
0.400055676698685 1\\
0.410055667161942 1\\
0.420055657625198 1\\
0.430055677890778 1\\
0.440055668354034 1\\
0.450055658817291 1\\
0.46005567908287 1\\
0.470055669546127 1\\
0.480055660009384 1\\
0.490055680274963 1\\
0.50005567073822 1\\
0.510055661201477 1\\
0.520055651664734 1\\
0.530055642127991 1\\
0.540055692195892 1\\
0.550055682659149 1\\
0.560055673122406 1\\
0.570055663585663 1\\
0.58005565404892 1\\
0.590055644512177 1\\
0.600055694580078 1\\
0.610055685043335 1\\
0.620055675506592 1\\
0.630055665969849 1\\
0.640055656433105 1\\
0.650055646896362 1\\
0.660055637359619 1\\
0.670055687427521 1\\
0.680055677890778 1\\
0.690055668354034 1\\
0.700055658817291 1\\
0.710055649280548 1\\
0.720055639743805 1\\
0.730055689811707 1\\
0.740055680274963 1\\
0.75005567073822 1\\
0.760055661201477 1\\
0.770055651664734 1\\
0.780055642127991 1\\
0.790055692195892 1\\
0.800055682659149 1\\
0.810055673122406 1\\
0.820055663585663 1\\
0.83005565404892 1\\
0.840055644512177 1\\
0.850055694580078 1\\
0.860055685043335 1\\
0.870055675506592 1\\
0.880055665969849 1\\
0.890055656433105 1\\
0.900055646896362 1\\
0.910055637359619 1\\
0.920055687427521 1\\
0.930055677890778 1\\
0.940055668354034 1\\
0.950055658817291 1\\
0.960055649280548 1\\
0.970055639743805 1\\
0.980055689811707 1\\
0.990055680274963 1\\
1.00005567073822 1\\
1.01005566120148 0.999947428703308\\
1.02005565166473 0.999781966209412\\
1.03005564212799 0.999620318412781\\
1.04005563259125 0.999392151832581\\
1.0500556230545 0.999095976352692\\
1.06005561351776 0.998730361461639\\
1.07005572319031 0.998293995857239\\
1.08005571365356 0.997785687446594\\
1.09005570411682 0.997204065322876\\
1.10005569458008 0.996548056602478\\
1.11005568504334 0.99581640958786\\
1.12005567550659 0.995007991790771\\
1.13005566596985 0.994121789932251\\
1.14005565643311 0.993156671524048\\
1.15005564689636 0.992111563682556\\
1.16005563735962 0.990985572338104\\
1.17005562782288 0.989777565002441\\
1.18005561828613 0.988486707210541\\
1.19005560874939 0.987112045288086\\
1.20005571842194 0.985652625560761\\
1.21005570888519 0.984107613563538\\
1.22005569934845 0.982476115226746\\
1.23005568981171 0.980757296085358\\
1.24005568027496 0.978951454162598\\
1.25005567073822 0.977075278759003\\
1.26005566120148 0.975142896175385\\
1.27005565166473 0.973155975341797\\
1.28005564212799 0.971112370491028\\
1.29005563259125 0.969009518623352\\
1.3000556230545 0.966845393180847\\
1.31005561351776 0.964618444442749\\
1.32005572319031 0.962327837944031\\
1.33005571365356 0.959973037242889\\
1.34005570411682 0.957553625106812\\
1.35005569458008 0.955069482326508\\
1.36005568504334 0.952520549297333\\
1.37005567550659 0.949906826019287\\
1.38005566596985 0.94722843170166\\
1.39005565643311 0.944485306739807\\
1.40005564689636 0.941677629947662\\
1.41005563735962 0.938805460929871\\
1.42005562782288 0.935868859291077\\
1.43005561828613 0.932867884635925\\
1.44005560874939 0.929802656173706\\
1.45005571842194 0.926673173904419\\
1.46005570888519 0.923479557037354\\
1.47005569934845 0.92022180557251\\
1.48005568981171 0.916899979114532\\
1.49005568027496 0.913514077663422\\
1.50005567073822 0.910064160823822\\
1.51005566120148 0.906550228595734\\
1.52005565166473 0.902972340583801\\
1.53005564212799 0.899330496788025\\
1.54005563259125 0.89562463760376\\
1.5500556230545 0.891854822635651\\
1.56005561351776 0.888020992279053\\
1.57005572319031 0.884123206138611\\
1.58005571365356 0.88016140460968\\
1.59005570411682 0.876135587692261\\
1.60005569458008 0.872045695781708\\
1.61005568504334 0.867891788482666\\
1.62005567550659 0.863673746585846\\
1.63005566596985 0.859391629695892\\
1.64005565643311 0.855045318603516\\
1.65005564689636 0.850634872913361\\
1.66005563735962 0.846160113811493\\
1.67005562782288 0.841621160507202\\
1.68005561828613 0.837017893791199\\
1.69005560874939 0.832350254058838\\
1.70005571842194 0.82761824131012\\
1.71005570888519 0.822821795940399\\
1.72005569934845 0.817960917949677\\
1.73005568981171 0.813035428524017\\
1.74005568027496 0.808045387268066\\
1.75005567073822 0.802990674972534\\
1.76005566120148 0.797872245311737\\
1.77005565166473 0.792716443538666\\
1.78005564212799 0.787553429603577\\
1.79005563259125 0.782386958599091\\
1.8000556230545 0.77721107006073\\
1.81005561351776 0.772017121315002\\
1.82005572319031 0.766796708106995\\
1.83005571365356 0.761542320251465\\
1.84005570411682 0.756247639656067\\
1.85005569458008 0.750907421112061\\
1.86005568504334 0.745517194271088\\
1.87005567550659 0.740073382854462\\
1.88005566596985 0.734573066234589\\
1.89005565643311 0.729013621807098\\
1.90005564689636 0.723393023014069\\
1.91005563735962 0.717709481716156\\
1.92005562782288 0.711961388587952\\
1.93005561828613 0.706147372722626\\
1.94005560874939 0.700266301631927\\
1.95005571842194 0.694316923618317\\
1.96005570888519 0.688726127147675\\
1.97005569934845 0.684719562530518\\
1.98005568981171 0.682762682437897\\
1.99005568027496 0.682944536209106\\
2.00005555152893 0.684946656227112\\
2.01005578041077 0.68822193145752\\
2.02005577087402 0.692149758338928\\
2.03005576133728 0.696172177791595\\
2.04005575180054 0.699921190738678\\
2.05005574226379 0.70319652557373\\
2.06005573272705 0.705916941165924\\
2.07005572319031 0.708073019981384\\
2.08005571365356 0.709694802761078\\
2.09005570411682 0.710845947265625\\
2.10005569458008 0.711670517921448\\
2.11005568504334 0.71240359544754\\
2.12005567550659 0.713474452495575\\
2.13005566596985 0.715694069862366\\
2.14005565643311 0.72019225358963\\
2.15005564689636 0.727587044239044\\
2.16005563735962 0.737067759037018\\
2.17005562782288 0.747016549110413\\
2.18005561828613 0.75606644153595\\
2.19005560874939 0.763664543628693\\
2.20005559921265 0.769892036914825\\
2.2100555896759 0.775082290172577\\
2.22005558013916 0.779570817947388\\
2.23005557060242 0.783605515956879\\
2.24005556106567 0.787335813045502\\
2.25005555152893 0.790832877159119\\
2.26005578041077 0.794116020202637\\
2.27005577087402 0.797174692153931\\
2.28005576133728 0.799983859062195\\
2.29005575180054 0.802512407302856\\
2.30005574226379 0.804727077484131\\
2.31005573272705 0.806596517562866\\
2.32005572319031 0.808101892471313\\
2.33005571365356 0.809250116348267\\
2.34005570411682 0.810067415237427\\
2.35005569458008 0.810580432415009\\
2.36005568504334 0.810807466506958\\
2.37005567550659 0.810757994651794\\
2.38005566596985 0.810436189174652\\
2.39005565643311 0.809808015823364\\
2.40005564689636 0.808527410030365\\
2.41005563735962 0.806402921676636\\
2.42005562782288 0.803346753120422\\
2.43005561828613 0.799330174922943\\
2.44005560874939 0.794373869895935\\
2.45005559921265 0.788535118103027\\
2.4600555896759 0.781897485256195\\
2.47005558013916 0.774561285972595\\
2.48005557060242 0.766634404659271\\
2.49005556106567 0.758225440979004\\
2.50005555152893 0.749439239501953\\
2.51005578041077 0.740374565124512\\
2.52005577087402 0.731126070022583\\
2.53005576133728 0.721779882907867\\
2.54005575180054 0.712411522865295\\
2.55005574226379 0.703087031841278\\
2.56005573272705 0.693863213062286\\
2.57005572319031 0.684788703918457\\
2.58005571365356 0.675905704498291\\
2.59005570411682 0.667280972003937\\
2.60005569458008 0.659027338027954\\
2.61005568504334 0.651259481906891\\
2.62005567550659 0.64411187171936\\
2.63005566596985 0.638089895248413\\
2.64005565643311 0.635475873947144\\
2.65005564689636 0.636992394924164\\
2.66005563735962 0.642294049263\\
2.67005562782288 0.650389790534973\\
2.68005561828613 0.66011655330658\\
2.69005560874939 0.670444130897522\\
2.70005559921265 0.680894017219543\\
2.7100555896759 0.69127082824707\\
2.72005558013916 0.701449930667877\\
2.73005557060242 0.711334228515625\\
2.74005556106567 0.720849096775055\\
2.75005555152893 0.729940533638\\
2.76005578041077 0.738573670387268\\
2.77005577087402 0.746732950210571\\
2.78005576133728 0.75442761182785\\
2.79005575180054 0.761723875999451\\
2.80005574226379 0.768760919570923\\
2.81005573272705 0.775801479816437\\
2.82005572319031 0.783307373523712\\
2.83005571365356 0.791747987270355\\
2.84005570411682 0.801149427890778\\
2.85005569458008 0.810765027999878\\
2.86005568504334 0.81949907541275\\
2.87005567550659 0.826596915721893\\
2.88005566596985 0.831840455532074\\
2.89005565643311 0.83536297082901\\
2.90005564689636 0.837432265281677\\
2.91005563735962 0.838325142860413\\
2.92005562782288 0.838279187679291\\
2.93005561828613 0.837484121322632\\
2.94005560874939 0.836091339588165\\
2.95005559921265 0.834230363368988\\
2.9600555896759 0.832008361816406\\
2.97005558013916 0.829510569572449\\
2.98005557060242 0.826798796653748\\
2.99005556106567 0.823849558830261\\
3.00005555152893 0.820338606834412\\
3.01005578041077 0.816118240356445\\
3.02005577087402 0.811094701290131\\
3.03005576133728 0.805221617221832\\
3.04005575180054 0.798506081104279\\
3.05005574226379 0.790995538234711\\
3.06005573272705 0.782766699790955\\
3.07005572319031 0.773912668228149\\
3.08005571365356 0.764532744884491\\
3.09005570411682 0.754732191562653\\
3.10005569458008 0.74461418390274\\
3.11005568504334 0.734275102615356\\
3.12005567550659 0.723810791969299\\
3.13005566596985 0.713323354721069\\
3.14005565643311 0.702923178672791\\
3.15005564689636 0.692727208137512\\
3.16005563735962 0.6828573346138\\
3.17005562782288 0.673441410064697\\
3.18005561828613 0.664614915847778\\
3.19005560874939 0.656521081924438\\
3.20005559921265 0.6493119597435\\
3.2100555896759 0.643689274787903\\
3.22005558013916 0.640969336032867\\
3.23005557060242 0.641327738761902\\
3.24005556106567 0.64477926492691\\
3.25005555152893 0.651229858398438\\
3.26005578041077 0.660486876964569\\
3.27005577087402 0.672317624092102\\
3.28005576133728 0.686516880989075\\
3.29005575180054 0.702502965927124\\
3.30005574226379 0.71934050321579\\
3.31005573272705 0.736350178718567\\
3.32005572319031 0.752970933914185\\
3.33005571365356 0.768543124198914\\
3.34005570411682 0.78220933675766\\
3.35005569458008 0.793600916862488\\
3.36005568504334 0.802746832370758\\
3.37005567550659 0.809874594211578\\
3.38005566596985 0.815254032611847\\
3.39005565643311 0.819222211837769\\
3.40005564689636 0.822238445281982\\
3.41005563735962 0.824924468994141\\
3.42005562782288 0.827865660190582\\
3.43005561828613 0.831193685531616\\
3.44005560874939 0.834470629692078\\
3.45005559921265 0.837045133113861\\
3.4600555896759 0.838456928730011\\
3.47005558013916 0.838563680648804\\
3.48005557060242 0.837459683418274\\
3.49005556106567 0.835370063781738\\
3.50005555152893 0.832540452480316\\
3.51005578041077 0.829169869422913\\
3.52005577087402 0.825400471687317\\
3.53005576133728 0.821102261543274\\
3.54005575180054 0.816166758537292\\
3.55005574226379 0.810570001602173\\
3.56005573272705 0.804301142692566\\
3.57005572319031 0.797378361225128\\
3.58005571365356 0.789849638938904\\
3.59005570411682 0.781789541244507\\
3.60005569458008 0.773292362689972\\
3.61005568504334 0.764449298381805\\
3.62005567550659 0.755341768264771\\
3.63005566596985 0.746031045913696\\
3.64005565643311 0.73656153678894\\
3.65005564689636 0.726971328258514\\
3.66005563735962 0.717311680316925\\
3.67005562782288 0.707655310630798\\
3.68005561828613 0.698091566562653\\
3.69005560874939 0.688718318939209\\
3.70005559921265 0.679637014865875\\
3.7100555896759 0.670949876308441\\
3.72005558013916 0.662759780883789\\
3.73005557060242 0.655170261859894\\
3.74005556106567 0.648285984992981\\
3.75005555152893 0.643000721931458\\
3.76005578041077 0.640484094619751\\
3.77005577087402 0.640774786472321\\
3.78005576133728 0.64378821849823\\
3.79005575180054 0.649352967739105\\
3.80005574226379 0.65724766254425\\
3.81005573272705 0.667321383953094\\
3.82005572319031 0.679539859294891\\
3.83005571365356 0.693313419818878\\
3.84005570411682 0.707874894142151\\
3.85005569458008 0.722723782062531\\
3.86005568504334 0.737446546554565\\
3.87005567550659 0.751723110675812\\
3.88005566596985 0.765283107757568\\
3.89005565643311 0.777613043785095\\
3.90005564689636 0.788216233253479\\
3.91005563735962 0.796943128108978\\
3.92005562782288 0.803911566734314\\
3.93005561828613 0.809428155422211\\
3.94005560874939 0.813998341560364\\
3.95005559921265 0.81830245256424\\
3.9600555896759 0.822883486747742\\
3.97005558013916 0.827703356742859\\
3.98005557060242 0.832173526287079\\
3.99005556106567 0.835632145404816\\
4.00005578994751 0.837709724903107\\
4.01005554199219 0.838359415531158\\
4.02005577087402 0.837725639343262\\
4.0300555229187 0.836036622524261\\
4.04005575180054 0.833545684814453\\
4.05005550384521 0.830461859703064\\
4.06005573272705 0.826937556266785\\
4.07005548477173 0.823050916194916\\
4.08005571365356 0.818639755249023\\
4.09005546569824 0.813654541969299\\
4.10005569458008 0.808072805404663\\
4.11005544662476 0.801887035369873\\
4.12005567550659 0.795113205909729\\
4.13005590438843 0.787792921066284\\
4.14005565643311 0.779995858669281\\
4.15005588531494 0.771788001060486\\
4.16005563735962 0.763211667537689\\
4.17005586624146 0.754295170307159\\
4.18005561828613 0.745076537132263\\
4.19005584716797 0.735627353191376\\
4.20005559921265 0.726055562496185\\
4.21005582809448 0.716494679450989\\
4.22005558013916 0.707091271877289\\
4.230055809021 0.697990417480469\\
4.24005556106567 0.689327418804169\\
4.25005578994751 0.681229948997498\\
4.26005554199219 0.67382025718689\\
4.27005577087402 0.667220413684845\\
4.2800555229187 0.662580728530884\\
4.29005575180054 0.660546362400055\\
4.30005550384521 0.660974681377411\\
4.31005573272705 0.663656651973724\\
4.32005548477173 0.668352723121643\\
4.33005571365356 0.674814999103546\\
4.34005546569824 0.682811677455902\\
4.35005569458008 0.692189633846283\\
4.36005544662476 0.702848672866821\\
4.37005567550659 0.714362978935242\\
4.38005590438843 0.726243197917938\\
4.39005565643311 0.738207519054413\\
4.40005588531494 0.750032901763916\\
4.41005563735962 0.761541306972504\\
4.42005586624146 0.77244108915329\\
4.43005561828613 0.782171189785004\\
4.44005584716797 0.790392339229584\\
4.45005559921265 0.797060191631317\\
4.46005582809448 0.802361071109772\\
4.47005558013916 0.806710541248322\\
4.480055809021 0.810754358768463\\
4.49005556106567 0.815118491649628\\
4.50005578994751 0.81991046667099\\
4.51005554199219 0.824587881565094\\
4.52005577087402 0.828408241271973\\
4.5300555229187 0.830882668495178\\
4.54005575180054 0.831881284713745\\
4.55005550384521 0.831510782241821\\
4.56005573272705 0.829980671405792\\
4.57005548477173 0.827542364597321\\
4.58005571365356 0.824441850185394\\
4.59005546569824 0.820853888988495\\
4.60005569458008 0.816763043403625\\
4.61005544662476 0.812026143074036\\
4.62005567550659 0.806649029254913\\
4.63005590438843 0.800637543201447\\
4.64005565643311 0.794015645980835\\
4.65005588531494 0.786828994750977\\
4.66005563735962 0.779140710830688\\
4.67005586624146 0.771032869815826\\
4.68005561828613 0.762596428394318\\
4.69005584716797 0.753898978233337\\
4.70005559921265 0.744966864585876\\
4.71005582809448 0.735809862613678\\
4.72005558013916 0.726461172103882\\
4.730055809021 0.71700257062912\\
4.74005556106567 0.707561314105988\\
4.75005578994751 0.698293566703796\\
4.76005554199219 0.689372897148132\\
4.77005577087402 0.68098646402359\\
4.7800555229187 0.673335611820221\\
4.79005575180054 0.666619122028351\\
4.80005550384521 0.662141263484955\\
4.81005573272705 0.661036968231201\\
4.82005548477173 0.663147747516632\\
4.83005571365356 0.668159365653992\\
4.84005546569824 0.675692081451416\\
4.85005569458008 0.685342729091644\\
4.86005544662476 0.696662604808807\\
4.87005567550659 0.709140121936798\\
4.88005590438843 0.722396552562714\\
4.89005565643311 0.736092329025269\\
4.90005588531494 0.749655246734619\\
4.91005563735962 0.762749135494232\\
4.92005586624146 0.77489846944809\\
4.93005561828613 0.78548389673233\\
4.94005584716797 0.794238090515137\\
4.95005559921265 0.801163911819458\\
4.96005582809448 0.806397080421448\\
4.97005558013916 0.810135841369629\\
4.980055809021 0.812640786170959\\
4.99005556106567 0.814299583435059\\
5 0.815687239170074\\
};
\addlegendentry{$\dtmax{} = \expneg{1.0}{4}{[s]}$};

\end{axis}
\end{tikzpicture}%
\caption{Nonlinear solver timestep-size insensitive flashing solution.}
\label{fig:flashingDtInsensitiveNln}
\end{figure}

\begin{figure}[h!tb]
\centering
% This file was created by matlab2tikz v0.4.3.
% Copyright (c) 2008--2013, Nico Schlömer <nico.schloemer@gmail.com>
% All rights reserved.
% 
\tikzsetnextfilename{plots/flashingDtInsensitiveLin_eps}
\begin{tikzpicture}

\begin{axis}[%
width=0.8\textwidth,
%height=0.630967741935484\textwidth,
height=0.491294629700995\textwidth,
scale only axis,
xmin=0.0,
xmax=5.0,
xlabel={Time $[\text{s}]$},
ymin=0.5,
ymax=1.0,
ylabel={$\alpha_g$ [-] @ 2 [in] from Inlet},
legend style={at={(0.03,0.03)},anchor=south west,draw=black,fill=white,legend cell align=left}
]
\addplot [
color=black,
dotted
]
table[row sep=crcr]{
0.0 1.0\\
0.0118019115179777 1\\
0.0226854234933853 1\\
0.0340500771999359 1\\
0.0491764321923256 1\\
0.0619497485458851 1\\
0.0774054601788521 1\\
0.0961068719625473 1\\
0.106882445514202 1\\
0.118735581636429 1\\
0.13177402317524 1\\
0.146116316318512 1\\
0.161892831325531 1\\
0.179247006773949 1\\
0.198336601257324 1\\
0.21933513879776 1\\
0.242433547973633 1\\
0.267841786146164 1\\
0.29579085111618 1\\
0.32653483748436 1\\
0.360353201627731 1\\
0.397553414106369 1\\
0.438473641872406 1\\
0.483485877513886 1\\
0.532999336719513 1\\
0.587464153766632 1\\
0.647375464439392 1\\
0.71327793598175 1\\
0.7857705950737 1\\
0.865512549877167 1\\
0.953228712081909 1\\
1.04971647262573 0.997021615505219\\
1.1558530330658 0.985307455062866\\
1.2407523393631 0.970276594161987\\
1.29757130146027 0.957477152347565\\
1.34479975700378 0.945545017719269\\
1.38458728790283 0.934612452983856\\
1.41912090778351 0.924424767494202\\
1.45013403892517 0.914681613445282\\
1.47872579097748 0.90517920255661\\
1.5054817199707 0.895826578140259\\
1.53073275089264 0.88658881187439\\
1.55469310283661 0.877453088760376\\
1.57754015922546 0.868404984474182\\
1.59942138195038 0.859430253505707\\
1.62045061588287 0.850519120693207\\
1.64072275161743 0.841663300991058\\
1.6603125333786 0.832857310771942\\
1.67928779125214 0.824094474315643\\
1.69770324230194 0.815370202064514\\
1.71560847759247 0.806679904460907\\
1.73303949832916 0.79802268743515\\
1.75003552436829 0.789430022239685\\
1.76655149459839 0.781055510044098\\
1.78238117694855 0.772998511791229\\
1.79760921001434 0.765193462371826\\
1.81230998039246 0.757585883140564\\
1.82662653923035 0.750090539455414\\
1.84061861038208 0.742669582366943\\
1.85433006286621 0.735296964645386\\
1.86779499053955 0.72795307636261\\
1.88103151321411 0.720627963542938\\
1.89406371116638 0.71331000328064\\
1.90690815448761 0.70599102973938\\
1.91956806182861 0.698671102523804\\
1.93206679821014 0.691338658332825\\
1.94440972805023 0.683992326259613\\
1.95660436153412 0.676629960536957\\
1.96866297721863 0.671738088130951\\
1.98027718067169 0.672812759876251\\
1.99125838279724 0.677815139293671\\
2.00170540809631 0.684370279312134\\
2.01211428642273 0.691331386566162\\
2.02296447753906 0.698101699352264\\
2.03420996665955 0.704113781452179\\
2.04571580886841 0.709079086780548\\
2.05740904808044 0.712941884994507\\
2.06921482086182 0.71580046415329\\
2.08101534843445 0.717850506305695\\
2.09267902374268 0.719229459762573\\
2.10420393943787 0.720068991184235\\
2.1155960559845 0.720495879650116\\
2.12691164016724 0.720689058303833\\
2.13817572593689 0.721016407012939\\
2.14939737319946 0.722309648990631\\
2.16053771972656 0.725886106491089\\
2.17153453826904 0.732412099838257\\
2.18234181404114 0.740104675292969\\
2.19297122955322 0.747277319431305\\
2.2034854888916 0.753501296043396\\
2.21399807929993 0.758969664573669\\
2.22421646118164 0.76383113861084\\
2.24485516548157 0.773090600967407\\
2.2556414604187 0.777811586856842\\
2.26643514633179 0.78242814540863\\
2.27722930908203 0.786858320236206\\
2.28802180290222 0.791011035442352\\
2.2988224029541 0.794807195663452\\
2.30910420417786 0.798032104969025\\
2.32745814323425 0.80274897813797\\
2.34450769424438 0.805849254131317\\
2.36390542984009 0.807824075222015\\
2.37500143051147 0.808235883712769\\
2.38614940643311 0.808172166347504\\
2.39734673500061 0.807666897773743\\
2.40859293937683 0.806699872016907\\
2.41979336738586 0.805263638496399\\
2.43107962608337 0.802674949169159\\
2.44180870056152 0.79903507232666\\
2.4530987739563 0.793921113014221\\
2.46468305587769 0.787385225296021\\
2.47643661499023 0.779563665390015\\
2.48832750320435 0.77061265707016\\
2.50033926963806 0.760712027549744\\
2.5124671459198 0.750045478343964\\
2.5246946811676 0.738811135292053\\
2.53700971603394 0.727198243141174\\
2.54939413070679 0.71539968252182\\
2.56182551383972 0.703596353530884\\
2.57428097724915 0.691951513290405\\
2.58673787117004 0.680607497692108\\
2.59917783737183 0.669684410095215\\
2.61158609390259 0.659270703792572\\
2.62395215034485 0.649651288986206\\
2.6361358165741 0.641215026378632\\
2.64810156822205 0.635994970798492\\
2.6597592830658 0.635662436485291\\
2.67102980613709 0.639588057994843\\
2.68197822570801 0.64653742313385\\
2.69267916679382 0.655223429203033\\
2.70332479476929 0.665210664272308\\
2.71397757530212 0.676258027553558\\
2.72448706626892 0.687942206859589\\
2.74232935905457 0.708736419677734\\
2.75233578681946 0.723106145858765\\
2.76248455047607 0.746678531169891\\
2.77299618721008 0.767698109149933\\
2.78397274017334 0.786229491233826\\
2.79858231544495 0.805485188961029\\
2.81091928482056 0.817172288894653\\
2.82584691047668 0.826655030250549\\
2.84390950202942 0.832634270191193\\
2.85421967506409 0.83385705947876\\
2.86443328857422 0.83380115032196\\
2.87539649009705 0.832415580749512\\
2.88654828071594 0.829780161380768\\
2.89785838127136 0.825988829135895\\
2.90905976295471 0.821340024471283\\
2.92766308784485 0.812942206859589\\
2.9380214214325 0.808586359024048\\
2.94941568374634 0.80469822883606\\
2.96126461029053 0.801513075828552\\
2.97307300567627 0.798118710517883\\
2.98484778404236 0.793721735477448\\
2.99659848213196 0.787129402160645\\
3.00838160514832 0.778874039649963\\
3.02029156684875 0.769327104091644\\
3.03234505653381 0.758802771568298\\
3.04447245597839 0.747617602348328\\
3.05664157867432 0.736036121845245\\
3.06882643699646 0.724325954914093\\
3.08100342750549 0.712768971920013\\
3.09315085411072 0.701684951782227\\
3.10524773597717 0.691458225250244\\
3.11726975440979 0.682536721229553\\
3.12919759750366 0.675742566585541\\
3.14099335670471 0.672509431838989\\
3.15258574485779 0.673115789890289\\
3.16393399238586 0.677472770214081\\
3.1750283241272 0.685294687747955\\
3.18587756156921 0.696173489093781\\
3.19650077819824 0.709694087505341\\
3.20691919326782 0.725244760513306\\
3.21715044975281 0.742033123970032\\
3.22718858718872 0.758418679237366\\
3.2467474937439 0.784654200077057\\
3.26547813415527 0.801358640193939\\
3.28411507606506 0.811337649822235\\
3.29831433296204 0.815494418144226\\
3.3087158203125 0.816964566707611\\
3.32302951812744 0.816825926303864\\
3.33405566215515 0.815179109573364\\
3.3473973274231 0.811917245388031\\
3.36354088783264 0.807795643806458\\
3.38307428359985 0.80543839931488\\
3.3943293094635 0.804838478565216\\
3.40593409538269 0.803381621837616\\
3.41763758659363 0.800470113754272\\
3.42948269844055 0.796141386032104\\
3.44148325920105 0.790801346302032\\
3.45353007316589 0.784392714500427\\
3.4655749797821 0.777148246765137\\
3.48354387283325 0.765116512775421\\
3.4938976764679 0.757575452327728\\
3.50528645515442 0.748863220214844\\
3.51756191253662 0.739044904708862\\
3.52981090545654 0.728857576847076\\
3.54198026657104 0.718411505222321\\
3.55405426025391 0.707884550094604\\
3.56603574752808 0.697501063346863\\
3.57792735099792 0.687490999698639\\
3.58974385261536 0.678059160709381\\
3.60148811340332 0.669394671916962\\
3.61318278312683 0.661635458469391\\
3.62483644485474 0.656085550785065\\
3.63638854026794 0.653754055500031\\
3.64775395393372 0.654484272003174\\
3.65890121459961 0.657902538776398\\
3.66983270645142 0.663596034049988\\
3.68056321144104 0.671233534812927\\
3.69111466407776 0.680723905563354\\
3.70150089263916 0.691940426826477\\
3.71173763275146 0.704434335231781\\
3.72185039520264 0.717744290828705\\
3.73186469078064 0.731415927410126\\
3.75170636177063 0.758384227752686\\
3.7714056968689 0.78208589553833\\
3.7911114692688 0.798542380332947\\
3.8088846206665 0.807387351989746\\
3.82693791389465 0.81149297952652\\
3.84111881256104 0.812401294708252\\
3.85744023323059 0.812185168266296\\
3.87718939781189 0.813103318214417\\
3.88805747032166 0.814119160175323\\
3.89892983436584 0.814466416835785\\
3.90982556343079 0.813674628734589\\
3.92078495025635 0.811778545379639\\
3.9318573474884 0.808977901935577\\
3.94304871559143 0.805517554283142\\
3.95445036888123 0.801541984081268\\
3.96601796150208 0.797248244285584\\
3.97746133804321 0.792374908924103\\
3.98889112472534 0.786799848079681\\
4.00039577484131 0.780433714389801\\
4.01201629638672 0.773246467113495\\
4.02377510070801 0.765269994735718\\
4.03563594818115 0.756564557552338\\
4.04752445220947 0.747200310230255\\
4.05936098098755 0.737273216247559\\
4.0711030960083 0.726977527141571\\
4.08272695541382 0.716625571250916\\
4.09423494338989 0.706573486328125\\
4.10562992095947 0.697181880474091\\
4.11694383621216 0.688742518424988\\
4.12819576263428 0.681517958641052\\
4.13939952850342 0.677124381065369\\
4.1505651473999 0.676256716251373\\
4.16158246994019 0.678564786911011\\
4.17241191864014 0.683457136154175\\
4.18306303024292 0.690370500087738\\
4.1935658454895 0.698813796043396\\
4.2039475440979 0.708419322967529\\
4.21423053741455 0.718966424465179\\
4.22443199157715 0.730270206928253\\
4.2345666885376 0.742040574550629\\
4.24464702606201 0.754002392292023\\
4.25469064712524 0.765895545482636\\
4.26471757888794 0.776910960674286\\
4.27471828460693 0.78636509180069\\
4.2847638130188 0.794048726558685\\
4.29489898681641 0.799984812736511\\
4.30505514144897 0.804239630699158\\
4.31983995437622 0.807999908924103\\
4.33454179763794 0.808928072452545\\
4.34839296340942 0.807726681232452\\
4.36166000366211 0.805512011051178\\
4.37771320343018 0.802959859371185\\
4.39713716506958 0.802385091781616\\
4.40819406509399 0.802595138549805\\
4.41921901702881 0.801958739757538\\
4.43028879165649 0.800117373466492\\
4.44142246246338 0.797202825546265\\
4.45270013809204 0.793436765670776\\
4.46414184570313 0.789081275463104\\
4.47538042068481 0.784011363983154\\
4.48650979995728 0.778233647346497\\
4.49770545959473 0.771687626838684\\
4.50898551940918 0.764402031898499\\
4.52039957046509 0.756402134895325\\
4.5319881439209 0.747746407985687\\
4.54377889633179 0.73848021030426\\
4.55586624145508 0.728542983531952\\
4.56815242767334 0.718027353286743\\
4.58033037185669 0.707345902919769\\
4.59236669540405 0.696863949298859\\
4.60425090789795 0.687036275863647\\
4.6159782409668 0.678349554538727\\
4.62757444381714 0.671239972114563\\
4.63906097412109 0.668458938598633\\
4.65037202835083 0.669922590255737\\
4.66144800186157 0.675077080726624\\
4.67229557037354 0.683269858360291\\
4.68293762207031 0.693869650363922\\
4.69340085983276 0.706282436847687\\
4.70371246337891 0.719945728778839\\
4.71390056610107 0.734441220760345\\
4.72399282455444 0.749322056770325\\
4.73401403427124 0.7640261054039\\
4.75387620925903 0.788362562656403\\
4.77353286743164 0.804273962974548\\
4.79333353042603 0.813261687755585\\
4.80498886108398 0.816133856773376\\
4.81894111633301 0.817074656486511\\
4.83303546905518 0.815881550312042\\
4.84702253341675 0.813483476638794\\
4.86394691467285 0.810876548290253\\
4.8844256401062 0.810437440872192\\
4.89526796340942 0.810307621955872\\
4.9058313369751 0.809220314025879\\
4.91633558273315 0.807012498378754\\
4.92694330215454 0.803822040557861\\
4.93775272369385 0.799894034862518\\
4.94876766204834 0.795530676841736\\
4.96000909805298 0.790432095527649\\
4.97146701812744 0.784575462341309\\
4.98318338394165 0.777906775474548\\
5 0.767241597175598\\
};
\addlegendentry{$\dtmax{} = \expneg{1.0}{0}{[s]}$};

\addplot [
color=black,
dashed
]
table[row sep=crcr]{
0 1\\
0.0118019115179777 1\\
0.0226854234933853 1\\
0.0340500771999359 1\\
0.0491764321923256 1\\
0.0619497485458851 1\\
0.0774054601788521 1\\
0.0961068719625473 1\\
0.106882445514202 1\\
0.118735581636429 1\\
0.13177402317524 1\\
0.146116316318512 1\\
0.161892831325531 1\\
0.179247006773949 1\\
0.198336601257324 1\\
0.21933513879776 1\\
0.242433547973633 1\\
0.267841786146164 1\\
0.29579085111618 1\\
0.32653483748436 1\\
0.360353201627731 1\\
0.397553414106369 1\\
0.438473641872406 1\\
0.483485877513886 1\\
0.532999336719513 1\\
0.587464153766632 1\\
0.647375464439392 1\\
0.71327793598175 1\\
0.7857705950737 1\\
0.865512549877167 1\\
0.953228712081909 1\\
1.04971647262573 0.997021615505219\\
1.14971649646759 0.986351609230042\\
1.2388744354248 0.970792293548584\\
1.29616057872772 0.957939326763153\\
1.34285283088684 0.946207404136658\\
1.38287508487701 0.935246229171753\\
1.41761469841003 0.925026714801788\\
1.44881391525269 0.915248930454254\\
1.47750616073608 0.905734837055206\\
1.50434100627899 0.896374046802521\\
1.52965128421783 0.887132167816162\\
1.55366480350494 0.87799197435379\\
1.57655787467957 0.868940055370331\\
1.59847927093506 0.859962105751038\\
1.61954414844513 0.851048231124878\\
1.63984787464142 0.842190027236938\\
1.65946626663208 0.833381831645966\\
1.67846739292145 0.824617147445679\\
1.69690656661987 0.815891146659851\\
1.7148334980011 0.807199358940125\\
1.73228430747986 0.798540771007538\\
1.74929881095886 0.789937436580658\\
1.76585221290588 0.781542837619781\\
1.78171455860138 0.773469805717468\\
1.79697132110596 0.765652060508728\\
1.81169188022614 0.758037328720093\\
1.82602417469025 0.750537514686584\\
1.84002900123596 0.743113934993744\\
1.85375189781189 0.735739588737488\\
1.86722791194916 0.728394269943237\\
1.88047194480896 0.721070051193237\\
1.89351260662079 0.713752150535584\\
1.90636217594147 0.706435024738312\\
1.91903162002563 0.699114441871643\\
1.93153655529022 0.691782832145691\\
1.94388437271118 0.684438705444336\\
1.95608687400818 0.67707633972168\\
1.96814596652985 0.671899914741516\\
1.9798002243042 0.672712087631226\\
1.99079525470734 0.677557289600372\\
2.00123715400696 0.68403571844101\\
2.01163697242737 0.690981984138489\\
2.02247047424316 0.697771728038788\\
2.03369832038879 0.703823328018188\\
2.04519176483154 0.708838582038879\\
2.05687618255615 0.712753057479858\\
2.06867718696594 0.715651094913483\\
2.08048534393311 0.717734813690186\\
2.09215831756592 0.719141185283661\\
2.10369086265564 0.720003366470337\\
2.11508893966675 0.720449388027191\\
2.12640833854675 0.720659196376801\\
2.13767504692078 0.721001625061035\\
2.1488983631134 0.722309768199921\\
2.1600399017334 0.725902020931244\\
2.17103719711304 0.732443273067474\\
2.18184471130371 0.740142703056335\\
2.19247317314148 0.747315406799316\\
2.2029869556427 0.753536224365234\\
2.21349930763245 0.758999824523926\\
2.223717212677 0.763855814933777\\
2.24435567855835 0.773103356361389\\
2.25514197349548 0.777817904949188\\
2.26593589782715 0.782427966594696\\
2.27673006057739 0.786851823329926\\
2.28752279281616 0.79099839925766\\
2.2983238697052 0.794788479804993\\
2.30863809585571 0.798012614250183\\
2.32703828811646 0.802706897258759\\
2.34415745735168 0.805782437324524\\
2.36360955238342 0.807716012001038\\
2.37471151351929 0.808102190494537\\
2.38586473464966 0.808014631271362\\
2.39706635475159 0.807488203048706\\
2.40831589698792 0.806493878364563\\
2.41951632499695 0.805034399032593\\
2.43080687522888 0.802421391010284\\
2.44148182868958 0.798784017562866\\
2.45279097557068 0.793648898601532\\
2.4643726348877 0.787106215953827\\
2.47612810134888 0.779280304908752\\
2.48802018165588 0.770330369472504\\
2.50003337860107 0.760434985160828\\
2.51216316223145 0.749777495861053\\
2.52439212799072 0.738555312156677\\
2.53670859336853 0.726956903934479\\
2.54909420013428 0.715174615383148\\
2.56152653694153 0.703388333320618\\
2.57398223876953 0.6917604804039\\
2.58643913269043 0.680432438850403\\
2.59887886047363 0.669523179531097\\
2.61128640174866 0.65911990404129\\
2.6236515045166 0.6495201587677\\
2.63582587242126 0.641115844249725\\
2.64778065681458 0.636001825332642\\
2.659423828125 0.635791957378387\\
2.67067933082581 0.639823436737061\\
2.68161129951477 0.646842300891876\\
2.69230103492737 0.65556663274765\\
2.70294141769409 0.66556191444397\\
2.71359300613403 0.676590740680695\\
2.72410321235657 0.688234090805054\\
2.74203324317932 0.709035813808441\\
2.7522988319397 0.724397957324982\\
2.76380586624146 0.752012193202972\\
2.77405524253845 0.772793173789978\\
2.78475761413574 0.790979564189911\\
2.7990026473999 0.809657752513886\\
2.81103157997131 0.820952892303467\\
2.82558655738831 0.830124080181122\\
2.84319829940796 0.835985720157623\\
2.85334610939026 0.837234139442444\\
2.86414122581482 0.837133586406708\\
2.87511372566223 0.835717022418976\\
2.88625264167786 0.833053946495056\\
2.89754724502563 0.829225063323975\\
2.90883326530457 0.824456036090851\\
2.92768001556396 0.815642714500427\\
2.93819761276245 0.810966014862061\\
2.94976687431335 0.806747138500214\\
2.96163415908813 0.803638696670532\\
2.97345376014709 0.800946414470673\\
2.98522281646729 0.797514140605927\\
2.9969425201416 0.791235864162445\\
3.00866484642029 0.78293764591217\\
3.02053761482239 0.77314168214798\\
3.03258919715881 0.7622349858284\\
3.04473614692688 0.750587403774261\\
3.0569326877594 0.738493502140045\\
3.0691511631012 0.726226091384888\\
3.08136558532715 0.714067995548248\\
3.09355187416077 0.702342867851257\\
3.10568857192993 0.691434383392334\\
3.11775040626526 0.681787312030792\\
3.12971878051758 0.673895657062531\\
3.14157199859619 0.669552743434906\\
3.15323138237 0.669406414031982\\
3.16463661193848 0.6733757853508\\
3.17577147483826 0.681038796901703\\
3.18664932250977 0.692003190517426\\
3.19729042053223 0.705828011035919\\
3.20771455764771 0.721758902072906\\
3.21794104576111 0.738930284976959\\
3.2279806137085 0.756408095359802\\
3.24761056900024 0.785510063171387\\
3.26646447181702 0.804471671581268\\
3.2851665019989 0.816074788570404\\
3.30015993118286 0.821442127227783\\
3.31109523773193 0.823484838008881\\
3.32280397415161 0.824106812477112\\
3.33651351928711 0.822816729545593\\
3.34809064865112 0.820433855056763\\
3.36209893226624 0.81698751449585\\
3.37904906272888 0.81411737203598\\
3.39955854415894 0.812461078166962\\
3.41116738319397 0.810398995876312\\
3.42286372184753 0.806844294071198\\
3.43470287322998 0.801888048648834\\
3.4465663433075 0.79605621099472\\
3.45787000656128 0.789692580699921\\
3.47379469871521 0.779717266559601\\
3.4930636882782 0.766135513782501\\
3.50416612625122 0.757685422897339\\
3.51637887954712 0.747951924800873\\
3.52864861488342 0.737764000892639\\
3.54084873199463 0.727272868156433\\
3.55296158790588 0.716613948345184\\
3.56498837471008 0.705977141857147\\
3.57693076133728 0.695570886135101\\
3.58879470825195 0.68558794260025\\
3.60058760643005 0.676199913024902\\
3.6123309135437 0.66754686832428\\
3.62403297424316 0.659753262996674\\
3.63569378852844 0.654087781906128\\
3.64725041389465 0.65157949924469\\
3.65861344337463 0.65206652879715\\
3.66975116729736 0.655189096927643\\
3.68066668510437 0.660557746887207\\
3.69137978553772 0.667939364910126\\
3.70191335678101 0.677314460277557\\
3.71228361129761 0.688492953777313\\
3.72250723838806 0.701058208942413\\
3.7325963973999 0.714500308036804\\
3.75248122215271 0.742160975933075\\
3.77218985557556 0.768642783164978\\
3.79181623458862 0.789814829826355\\
3.81167793273926 0.803487062454224\\
3.82728266716003 0.809676468372345\\
3.84524822235107 0.812438070774078\\
3.85556268692017 0.812902748584747\\
3.86804342269897 0.812742531299591\\
3.88314509391785 0.812954246997833\\
3.90141797065735 0.814873397350311\\
3.91194653511047 0.815695583820343\\
3.92283082008362 0.815373361110687\\
3.93378806114197 0.813795745372772\\
3.94480562210083 0.811182737350464\\
3.95595955848694 0.807796180248261\\
3.96731543540955 0.803848028182983\\
3.97884368896484 0.799512267112732\\
3.99042797088623 0.79450535774231\\
4.0019850730896 0.788779079914093\\
4.0135235786438 0.782292366027832\\
4.02511358261108 0.775009036064148\\
4.03680324554443 0.766947686672211\\
4.04859256744385 0.758141338825226\\
4.06042098999023 0.748647809028625\\
4.07218551635742 0.73859578371048\\
4.08385276794434 0.728193044662476\\
4.09540271759033 0.717761337757111\\
4.10684585571289 0.707655429840088\\
4.11820030212402 0.69822359085083\\
4.12947559356689 0.689776718616486\\
4.14068651199341 0.682577967643738\\
4.15185976028442 0.678209066390991\\
4.16300344467163 0.67734032869339\\
4.17401742935181 0.679639637470245\\
4.18484449386597 0.684535086154938\\
4.19550132751465 0.691486358642578\\
4.20601320266724 0.700023174285889\\
4.21640634536743 0.709792375564575\\
4.22670125961304 0.720569550991058\\
4.23691654205322 0.732152104377747\\
4.24706697463989 0.744234085083008\\
4.25716781616211 0.75652402639389\\
4.2672381401062 0.768650829792023\\
4.27728843688965 0.779620468616486\\
4.28730154037476 0.788829803466797\\
4.29739427566528 0.796225607395172\\
4.30756187438965 0.801840007305145\\
4.31767320632935 0.805773377418518\\
4.33207941055298 0.809107422828674\\
4.34596586227417 0.809810817241669\\
4.35955810546875 0.808497190475464\\
4.372718334198 0.806077480316162\\
4.38864183425903 0.803068578243256\\
4.40790939331055 0.801928520202637\\
4.41901111602783 0.802043914794922\\
4.43009757995605 0.801405787467957\\
4.4411883354187 0.79950225353241\\
4.45242166519165 0.79637348651886\\
4.46370792388916 0.792328178882599\\
4.47520685195923 0.787645101547241\\
4.48653984069824 0.782306373119354\\
4.49777412414551 0.776326954364777\\
4.5090594291687 0.769645631313324\\
4.52043962478638 0.762257635593414\\
4.53196239471436 0.754171311855316\\
4.54368162155151 0.745421946048737\\
4.55567598342896 0.73598438501358\\
4.56790256500244 0.725897014141083\\
4.58011674880981 0.715412199497223\\
4.59221315383911 0.704851686954498\\
4.60417032241821 0.694629490375519\\
4.61597204208374 0.68523496389389\\
4.62763738632202 0.67713737487793\\
4.63918876647949 0.671582758426666\\
4.65063285827637 0.670404672622681\\
4.66186189651489 0.673338711261749\\
4.67285776138306 0.67968761920929\\
4.68363618850708 0.688751339912415\\
4.69422245025635 0.699863612651825\\
4.7046422958374 0.712404668331146\\
4.71492433547974 0.725851655006409\\
4.72509813308716 0.73985880613327\\
4.73518991470337 0.754067301750183\\
4.74522352218628 0.767756819725037\\
4.7651481628418 0.789979934692383\\
4.78499126434326 0.80428010225296\\
4.80508518218994 0.811938107013702\\
4.81628894805908 0.814046561717987\\
4.8298454284668 0.814343512058258\\
4.84408235549927 0.812585413455963\\
4.85892581939697 0.809483766555786\\
4.87688589096069 0.806458413600922\\
4.88723421096802 0.805952787399292\\
4.89843463897705 0.805981874465942\\
4.90928411483765 0.805604934692383\\
4.91984224319458 0.804205298423767\\
4.93034839630127 0.801742315292358\\
4.94095993041992 0.798401355743408\\
4.95174980163574 0.794463753700256\\
4.96273040771484 0.789646208286285\\
4.97390985488892 0.783975720405579\\
4.9853401184082 0.777438700199127\\
5 0.768127679824829\\
};
\addlegendentry{$\dtmax{} = \expneg{1.0}{1}{[s]}$};

\addplot [
color=black,
dash pattern=on 1pt off 3pt on 3pt off 3pt
]
table[row sep=crcr]{
0 1\\
0.0118019115179777 1\\
0.0226854234933853 1\\
0.0340500771999359 1\\
0.0491764321923256 1\\
0.0619497485458851 1\\
0.0774054601788521 1\\
0.0961068719625473 1\\
0.106106869876385 1\\
0.116106867790222 1\\
0.12610687315464 1\\
0.136106878519058 1\\
0.146106868982315 1\\
0.156106874346733 1\\
0.16610686480999 1\\
0.176106870174408 1\\
0.186106875538826 1\\
0.196106866002083 1\\
0.206106871366501 1\\
0.216106876730919 1\\
0.226106867194176 1\\
0.236106872558594 1\\
0.246106877923012 1\\
0.25610688328743 1\\
0.266106873750687 1\\
0.276106864213943 1\\
0.286106884479523 1\\
0.29610687494278 1\\
0.306106865406036 1\\
0.316106885671616 1\\
0.326106876134872 1\\
0.336106866598129 1\\
0.346106857061386 1\\
0.356106877326965 1\\
0.366106867790222 1\\
0.376106858253479 1\\
0.386106878519058 1\\
0.396106868982315 1\\
0.406106859445572 1\\
0.416106879711151 1\\
0.426106870174408 1\\
0.436106860637665 1\\
0.446106880903244 1\\
0.456106871366501 1\\
0.466106861829758 1\\
0.476106882095337 1\\
0.486106872558594 1\\
0.496106863021851 1\\
0.506106853485107 1\\
0.516106843948364 1\\
0.526106894016266 1\\
0.536106884479523 1\\
0.54610687494278 1\\
0.556106865406036 1\\
0.566106855869293 1\\
0.57610684633255 1\\
0.586106896400452 1\\
0.596106886863709 1\\
0.606106877326965 1\\
0.616106867790222 1\\
0.626106858253479 1\\
0.636106848716736 1\\
0.646106898784637 1\\
0.656106889247894 1\\
0.666106879711151 1\\
0.676106870174408 1\\
0.686106860637665 1\\
0.696106851100922 1\\
0.706106901168823 1\\
0.71610689163208 1\\
0.726106882095337 1\\
0.736106872558594 1\\
0.746106863021851 1\\
0.756106853485107 1\\
0.766106843948364 1\\
0.776106894016266 1\\
0.786106884479523 1\\
0.79610687494278 1\\
0.806106865406036 1\\
0.816106855869293 1\\
0.82610684633255 1\\
0.836106896400452 1\\
0.846106886863709 1\\
0.856106877326965 1\\
0.866106867790222 1\\
0.876106858253479 1\\
0.886106848716736 1\\
0.896106898784637 1\\
0.906106889247894 1\\
0.916106879711151 1\\
0.926106870174408 1\\
0.936106860637665 1\\
0.946106851100922 1\\
0.956106901168823 1\\
0.96610689163208 1\\
0.976106882095337 1\\
0.986106872558594 1\\
0.996106863021851 1\\
1.00610685348511 0.999962091445923\\
1.01610684394836 0.999858677387238\\
1.02610683441162 0.999682128429413\\
1.03610682487488 0.999420166015625\\
1.04610681533813 0.999130368232727\\
1.05610692501068 0.998766839504242\\
1.06610691547394 0.998329222202301\\
1.07610690593719 0.997817575931549\\
1.08610689640045 0.997231781482697\\
1.09610688686371 0.996571123600006\\
1.10610687732697 0.995834708213806\\
1.11610686779022 0.995021402835846\\
1.12610685825348 0.994130253791809\\
1.13610684871674 0.99316018819809\\
1.14610683917999 0.992110073566437\\
1.15610682964325 0.990979015827179\\
1.16610682010651 0.989766001701355\\
1.17610692977905 0.988470077514648\\
1.18610692024231 0.987090289592743\\
1.19610691070557 0.985625684261322\\
1.20610690116882 0.984075427055359\\
1.21610689163208 0.982438683509827\\
1.22610688209534 0.980714559555054\\
1.23610687255859 0.978902280330658\\
1.24610686302185 0.977015495300293\\
1.25610685348511 0.975069344043732\\
1.26610684394836 0.973066926002502\\
1.27610683441162 0.971007227897644\\
1.28610682487488 0.968888461589813\\
1.29610681533813 0.966709017753601\\
1.30610692501068 0.964467525482178\\
1.31610691547394 0.962163150310516\\
1.32610690593719 0.959795236587524\\
1.33610689640045 0.957363426685333\\
1.34610688686371 0.954867362976074\\
1.35610687732697 0.952307045459747\\
1.36610686779022 0.949682354927063\\
1.37610685825348 0.946993231773376\\
1.38610684871674 0.944239735603333\\
1.39610683917999 0.941421926021576\\
1.40610682964325 0.938539803028107\\
1.41610682010651 0.935593485832214\\
1.42610692977905 0.932582914829254\\
1.43610692024231 0.92950826883316\\
1.44610691070557 0.926369488239288\\
1.45610690116882 0.923166692256927\\
1.46610689163208 0.919899880886078\\
1.47610688209534 0.91656905412674\\
1.48610687255859 0.913174331188202\\
1.49610686302185 0.90971565246582\\
1.50610685348511 0.906193017959595\\
1.51610684394836 0.902606546878815\\
1.52610683441162 0.898956120014191\\
1.53610682487488 0.895241796970367\\
1.54610681533813 0.89146363735199\\
1.55610692501068 0.887621462345123\\
1.56610691547394 0.883715450763702\\
1.57610690593719 0.879745423793793\\
1.58610689640045 0.875711441040039\\
1.59610688686371 0.871613562107086\\
1.60610687732697 0.867451608181\\
1.61610686779022 0.863225638866425\\
1.62610685825348 0.858935594558716\\
1.63610684871674 0.854581415653229\\
1.64610683917999 0.850163161754608\\
1.65610682964325 0.845680713653564\\
1.66610682010651 0.841134011745453\\
1.67610692977905 0.836523115634918\\
1.68610692024231 0.831847906112671\\
1.69610691070557 0.827108323574066\\
1.70610690116882 0.822304368019104\\
1.71610689163208 0.81743597984314\\
1.72610688209534 0.812503039836884\\
1.73610687255859 0.80750560760498\\
1.74610686302185 0.802443623542786\\
1.75610685348511 0.797316908836365\\
1.76610684394836 0.792151093482971\\
1.77610683441162 0.78697544336319\\
1.78610682487488 0.781793296337128\\
1.79610681533813 0.776599764823914\\
1.80610692501068 0.771387100219727\\
1.81610691547394 0.766147792339325\\
1.82610690593719 0.76087474822998\\
1.83610689640045 0.755562543869019\\
1.84610688686371 0.750205993652344\\
1.85610687732697 0.744800865650177\\
1.86610686779022 0.739343285560608\\
1.87610685825348 0.733830749988556\\
1.88610684871674 0.728260457515717\\
1.89610683917999 0.722630143165588\\
1.90610682964325 0.716938018798828\\
1.91610682010651 0.711182296276093\\
1.92610692977905 0.705361664295197\\
1.93610692024231 0.699474811553955\\
1.94610691070557 0.693520188331604\\
1.95610690116882 0.687497079372406\\
1.96610689163208 0.681404232978821\\
1.97610688209534 0.675240397453308\\
1.98610687255859 0.673442900180817\\
1.99610686302185 0.675594449043274\\
2.00610685348511 0.680603623390198\\
2.01610684394836 0.68700110912323\\
2.02610683441162 0.693596124649048\\
2.03610682487488 0.699695527553558\\
2.04610681533813 0.704988956451416\\
2.05610680580139 0.709391117095947\\
2.06610679626465 0.712919175624847\\
2.07610678672791 0.715633451938629\\
2.08610677719116 0.71770191192627\\
2.09610676765442 0.719236552715302\\
2.10610675811768 0.720321118831635\\
2.11610698699951 0.721030235290527\\
2.12610697746277 0.721451163291931\\
2.13610696792603 0.721728563308716\\
2.14610695838928 0.722166061401367\\
2.15610694885254 0.723395168781281\\
2.1661069393158 0.726407766342163\\
2.17610692977905 0.731940031051636\\
2.18610692024231 0.739692091941834\\
2.19610691070557 0.747829377651215\\
2.20610690116882 0.755074679851532\\
2.21610689163208 0.761131048202515\\
2.23586487770081 0.770546674728394\\
2.24586486816406 0.774604558944702\\
2.25586485862732 0.778461694717407\\
2.26586484909058 0.782198548316956\\
2.27586483955383 0.785822153091431\\
2.28586506843567 0.78929740190506\\
2.29586505889893 0.792571008205414\\
2.30586504936218 0.795587480068207\\
2.31586503982544 0.798298001289368\\
2.33441281318665 0.802402794361115\\
2.35220170021057 0.805108547210693\\
2.37192106246948 0.806630849838257\\
2.38192105293274 0.806852400302887\\
2.391921043396 0.806747615337372\\
2.40192103385925 0.806344926357269\\
2.41192102432251 0.805608093738556\\
2.42192101478577 0.804523885250092\\
2.43192100524902 0.802556216716766\\
2.44192099571228 0.799595654010773\\
2.45192098617554 0.795610964298248\\
2.46192097663879 0.790633320808411\\
2.47192096710205 0.784735143184662\\
2.48192095756531 0.778014123439789\\
2.49192094802856 0.770580589771271\\
2.50192093849182 0.762549877166748\\
2.51192092895508 0.754036486148834\\
2.52192115783691 0.745150566101074\\
2.53192114830017 0.735996007919312\\
2.54192113876343 0.726669132709503\\
2.55192112922668 0.717261433601379\\
2.56192111968994 0.707854926586151\\
2.5719211101532 0.698522567749023\\
2.58192110061646 0.689328610897064\\
2.59192109107971 0.680329322814941\\
2.60192108154297 0.671574413776398\\
2.61192107200623 0.66310328245163\\
2.62192106246948 0.655068755149841\\
2.63192105293274 0.647618174552917\\
2.641921043396 0.640919923782349\\
2.65192103385925 0.636759400367737\\
2.66192102432251 0.636161267757416\\
2.67192101478577 0.639037311077118\\
2.68192100524902 0.644774317741394\\
2.69192099571228 0.652492821216583\\
2.70192098617554 0.66138631105423\\
2.71192097663879 0.671156525611877\\
2.72192096710205 0.681590616703033\\
2.73192095756531 0.692513167858124\\
2.74995803833008 0.712703168392181\\
2.76597833633423 0.733335316181183\\
2.77814722061157 0.748456358909607\\
2.79020619392395 0.761836171150208\\
2.80279803276062 0.774248242378235\\
2.81343126296997 0.78347510099411\\
2.82629728317261 0.792748868465424\\
2.8418653011322 0.800914645195007\\
2.86070275306702 0.807012021541595\\
2.87070274353027 0.809516072273254\\
2.88070273399353 0.812399804592133\\
2.89070272445679 0.815916955471039\\
2.90070271492004 0.819491028785706\\
2.9107027053833 0.82219135761261\\
2.92070269584656 0.823555707931519\\
2.93070268630981 0.823628425598145\\
2.94070267677307 0.822681427001953\\
2.95070266723633 0.821015298366547\\
2.96070265769959 0.818899154663086\\
2.97070264816284 0.816536068916321\\
2.9807026386261 0.814059555530548\\
2.99070262908936 0.811542391777039\\
3.00070261955261 0.808987438678741\\
3.01070261001587 0.806229531764984\\
3.02070260047913 0.802899062633514\\
3.03070259094238 0.798822820186615\\
3.04070281982422 0.793902933597565\\
3.05070281028748 0.788117468357086\\
3.06070280075073 0.781497478485107\\
3.07070279121399 0.774112224578857\\
3.08070278167725 0.766055643558502\\
3.0907027721405 0.757433950901031\\
3.10070276260376 0.748358130455017\\
3.11070275306702 0.738940536975861\\
3.12070274353027 0.729297876358032\\
3.13070273399353 0.719562590122223\\
3.14070272445679 0.70988392829895\\
3.15070271492004 0.70042610168457\\
3.1607027053833 0.691370069980621\\
3.17070269584656 0.68291175365448\\
3.18070268630981 0.675264537334442\\
3.19070267677307 0.668651342391968\\
3.20070266723633 0.664222478866577\\
3.21070265769959 0.662512719631195\\
3.22070264816284 0.663645923137665\\
3.2307026386261 0.667584538459778\\
3.24070262908936 0.674204111099243\\
3.25070261955261 0.683318674564362\\
3.26070261001587 0.694737732410431\\
3.27070260047913 0.708245992660522\\
3.28070259094238 0.723328649997711\\
3.29070281982422 0.739359438419342\\
3.30070281028748 0.755724430084229\\
3.32038044929504 0.783443987369537\\
3.3397479057312 0.801862239837646\\
3.35937190055847 0.812804460525513\\
3.37386155128479 0.817092716693878\\
3.38459610939026 0.818577587604523\\
3.39596796035767 0.818568587303162\\
3.40832281112671 0.816993534564972\\
3.42327213287354 0.813476026058197\\
3.44136095046997 0.809085369110107\\
3.45136094093323 0.807813167572021\\
3.46136093139648 0.807467699050903\\
3.47136092185974 0.807378172874451\\
3.481360912323 0.806684732437134\\
3.49136090278625 0.804951250553131\\
3.50136089324951 0.802177488803864\\
3.51136088371277 0.798493981361389\\
3.52136087417603 0.794176459312439\\
3.53136110305786 0.789161384105682\\
3.54136109352112 0.783513605594635\\
3.55136108398438 0.777254402637482\\
3.56136107444763 0.770410597324371\\
3.57136106491089 0.763030648231506\\
3.58136105537415 0.75519335269928\\
3.5913610458374 0.746980249881744\\
3.60136103630066 0.738459587097168\\
3.61136102676392 0.729691565036774\\
3.62136101722717 0.72072958946228\\
3.63136100769043 0.711642980575562\\
3.64136099815369 0.702532887458801\\
3.65136098861694 0.693526089191437\\
3.6613609790802 0.684766411781311\\
3.67136096954346 0.676409840583801\\
3.68136096000671 0.668611943721771\\
3.69136095046997 0.661519169807434\\
3.70136094093323 0.6552774310112\\
3.71136093139648 0.651764571666718\\
3.72136092185974 0.650922834873199\\
3.731360912323 0.652737319469452\\
3.74136090278625 0.657097995281219\\
3.75136089324951 0.663823068141937\\
3.76136088371277 0.672700762748718\\
3.77136087417603 0.683616518974304\\
3.78136110305786 0.696394681930542\\
3.79136109352112 0.710539758205414\\
3.81122899055481 0.740612089633942\\
3.83080196380615 0.769472599029541\\
3.85026788711548 0.791813433170319\\
3.86996960639954 0.806104421615601\\
3.88323998451233 0.812059164047241\\
3.89929723739624 0.815580487251282\\
3.9114396572113 0.816496014595032\\
3.92577981948853 0.816308498382568\\
3.94313144683838 0.816388368606567\\
3.96312928199768 0.817972362041473\\
3.97312927246094 0.818166613578796\\
3.98312926292419 0.817338883876801\\
3.99312925338745 0.815512001514435\\
4.00312948226929 0.812868058681488\\
4.01312923431396 0.809622764587402\\
4.0231294631958 0.805924236774445\\
4.03312921524048 0.801922619342804\\
4.04312944412231 0.797400236129761\\
4.05312919616699 0.792328238487244\\
4.06312942504883 0.786689460277557\\
4.07312917709351 0.780489146709442\\
4.08312940597534 0.773759484291077\\
4.09312915802002 0.766552627086639\\
4.10312938690186 0.758903980255127\\
4.11312913894653 0.75083327293396\\
4.12312936782837 0.742350339889526\\
4.13312911987305 0.733511328697205\\
4.14312934875488 0.724441051483154\\
4.15312910079956 0.715320289134979\\
4.1631293296814 0.706358075141907\\
4.17312908172607 0.697763442993164\\
4.18312931060791 0.689739644527435\\
4.19312953948975 0.682476758956909\\
4.20312929153442 0.676151871681213\\
4.21312952041626 0.672262012958527\\
4.22312927246094 0.671193778514862\\
4.23312950134277 0.67277979850769\\
4.24312925338745 0.676703035831451\\
4.25312948226929 0.682623744010925\\
4.26312923431396 0.690193653106689\\
4.2731294631958 0.699078261852264\\
4.28312921524048 0.709094703197479\\
4.29312944412231 0.720081329345703\\
4.30312919616699 0.731767892837524\\
4.31312942504883 0.743849396705627\\
4.32312917709351 0.756034851074219\\
4.33312940597534 0.768044412136078\\
4.34312915802002 0.779016733169556\\
4.35312938690186 0.788335084915161\\
4.36312913894653 0.795807898044586\\
4.37312936782837 0.801497578620911\\
4.38924884796143 0.807466745376587\\
4.40431070327759 0.809827208518982\\
4.41981267929077 0.80956894159317\\
4.43557548522949 0.807857990264893\\
4.4546480178833 0.806498110294342\\
4.46464824676514 0.806619703769684\\
4.47464799880981 0.806630849838257\\
4.48464822769165 0.805952847003937\\
4.49464845657349 0.804404735565186\\
4.50464820861816 0.802066564559937\\
4.5146484375 0.799104452133179\\
4.52464818954468 0.795631051063538\\
4.53464841842651 0.791801810264587\\
4.54464817047119 0.787368178367615\\
4.55464839935303 0.782315313816071\\
4.56464815139771 0.776647329330444\\
4.57464838027954 0.770392775535584\\
4.58464813232422 0.763600170612335\\
4.59464836120605 0.756330668926239\\
4.60464811325073 0.748667299747467\\
4.61464834213257 0.740674316883087\\
4.62464809417725 0.732382357120514\\
4.63464832305908 0.723809599876404\\
4.64464807510376 0.715000927448273\\
4.6546483039856 0.706070005893707\\
4.66464805603027 0.697206377983093\\
4.67464828491211 0.688659369945526\\
4.68464803695679 0.680716395378113\\
4.69464826583862 0.673687279224396\\
4.7046480178833 0.668332636356354\\
4.71464824676514 0.666353344917297\\
4.72464799880981 0.667799592018127\\
4.73464822769165 0.672410368919373\\
4.74464845657349 0.67981094121933\\
4.75464820861816 0.689598619937897\\
4.7646484375 0.701322555541992\\
4.77464818954468 0.714470207691193\\
4.78464841842651 0.728623747825623\\
4.79464817047119 0.743395566940308\\
4.81454849243164 0.772128939628601\\
4.83430004119873 0.793891966342926\\
4.85395812988281 0.807590425014496\\
4.8739070892334 0.814804553985596\\
4.8856954574585 0.816753149032593\\
4.89849042892456 0.816828906536102\\
4.91286373138428 0.814811944961548\\
4.92903614044189 0.810916543006897\\
4.94835424423218 0.807147085666656\\
4.95835447311401 0.806505858898163\\
4.96835422515869 0.806391656398773\\
4.97835445404053 0.805948376655579\\
4.98835420608521 0.804552257061005\\
5 0.801996648311615\\
};
\addlegendentry{$\dtmax{} = \expneg{1.0}{2}{[s]}$};

\addplot [
color=black,
dashed
]
table[row sep=crcr]{
0 1\\
0.0109122171998024 1\\
0.020912216976285 1\\
0.0309122167527676 1\\
0.0409122183918953 1\\
0.0509122163057327 1\\
0.0609122179448605 1\\
0.0709122195839882 1\\
0.0809122174978256 1\\
0.0909122154116631 1\\
0.100912220776081 1\\
0.110912218689919 1\\
0.120912216603756 1\\
0.130912214517593 1\\
0.140912219882011 1\\
0.150912210345268 1\\
0.160912215709686 1\\
0.170912221074104 1\\
0.180912211537361 1\\
0.190912216901779 1\\
0.200912222266197 1\\
0.210912212729454 1\\
0.220912218093872 1\\
0.23091222345829 1\\
0.240912213921547 1\\
0.250912219285965 1\\
0.260912209749222 1\\
0.270912230014801 1\\
0.280912220478058 1\\
0.290912210941315 1\\
0.300912231206894 1\\
0.310912221670151 1\\
0.320912212133408 1\\
0.330912202596664 1\\
0.340912222862244 1\\
0.3509122133255 1\\
0.360912203788757 1\\
0.370912224054337 1\\
0.380912214517593 1\\
0.39091220498085 1\\
0.400912225246429 1\\
0.410912215709686 1\\
0.420912206172943 1\\
0.430912226438522 1\\
0.440912216901779 1\\
0.450912207365036 1\\
0.460912227630615 1\\
0.470912218093872 1\\
0.480912208557129 1\\
0.490912228822708 1\\
0.500912189483643 1\\
0.510912239551544 1\\
0.520912230014801 1\\
0.530912220478058 1\\
0.540912210941315 1\\
0.550912201404572 1\\
0.560912191867828 1\\
0.57091224193573 1\\
0.580912232398987 1\\
0.590912222862244 1\\
0.6009122133255 1\\
0.610912203788757 1\\
0.620912194252014 1\\
0.630912244319916 1\\
0.640912234783173 1\\
0.650912225246429 1\\
0.660912215709686 1\\
0.670912206172943 1\\
0.6809121966362 1\\
0.690912246704102 1\\
0.700912237167358 1\\
0.710912227630615 1\\
0.720912218093872 1\\
0.730912208557129 1\\
0.740912199020386 1\\
0.750912189483643 1\\
0.760912239551544 1\\
0.770912230014801 1\\
0.780912220478058 1\\
0.790912210941315 1\\
0.800912201404572 1\\
0.810912191867828 1\\
0.82091224193573 1\\
0.830912232398987 1\\
0.840912222862244 1\\
0.8509122133255 1\\
0.860912203788757 1\\
0.870912194252014 1\\
0.880912244319916 1\\
0.890912234783173 1\\
0.900912225246429 1\\
0.910912215709686 1\\
0.920912206172943 1\\
0.9309121966362 1\\
0.940912246704102 1\\
0.950912237167358 1\\
0.960912227630615 1\\
0.970912218093872 1\\
0.980912208557129 1\\
0.990912199020386 1\\
1.00091218948364 0.999999403953552\\
1.0109121799469 0.999952852725983\\
1.02091217041016 0.999807059764862\\
1.03091216087341 0.999626159667969\\
1.04091227054596 0.99938952922821\\
1.05091226100922 0.999084651470184\\
1.06091225147247 0.998710215091705\\
1.07091224193573 0.998264908790588\\
1.08091223239899 0.997747421264648\\
1.09091222286224 0.997156500816345\\
1.1009122133255 0.996491074562073\\
1.11091220378876 0.995749831199646\\
1.12091219425201 0.994931697845459\\
1.13091218471527 0.994035661220551\\
1.14091217517853 0.993060529232025\\
1.15091216564178 0.992005348205566\\
1.16091227531433 0.990869045257568\\
1.17091226577759 0.989650726318359\\
1.18091225624084 0.988349378108978\\
1.1909122467041 0.986964106559753\\
1.20091223716736 0.985493957996368\\
1.21091222763062 0.983938097953796\\
1.22091221809387 0.982295632362366\\
1.23091220855713 0.980565786361694\\
1.24091219902039 0.978749573230743\\
1.25091218948364 0.97686505317688\\
1.2609121799469 0.974924743175507\\
1.27091217041016 0.972929537296295\\
1.28091216087341 0.970877349376678\\
1.29091227054596 0.968765735626221\\
1.30091226100922 0.966592669487\\
1.31091225147247 0.964356780052185\\
1.32091224193573 0.96205723285675\\
1.33091223239899 0.959693431854248\\
1.34091222286224 0.957265079021454\\
1.3509122133255 0.954772055149078\\
1.36091220378876 0.952214300632477\\
1.37091219425201 0.949591755867004\\
1.38091218471527 0.946904540061951\\
1.39091217517853 0.94415271282196\\
1.40091216564178 0.941336274147034\\
1.41091227531433 0.938455402851105\\
1.42091226577759 0.935510158538818\\
1.43091225624084 0.932500541210175\\
1.4409122467041 0.929426670074463\\
1.45091223716736 0.926288545131683\\
1.46091222763062 0.92308634519577\\
1.47091221809387 0.919820010662079\\
1.48091220855713 0.916489601135254\\
1.49091219902039 0.913095116615295\\
1.50091218948364 0.909636676311493\\
1.5109121799469 0.906114220619202\\
1.52091217041016 0.902527809143066\\
1.53091216087341 0.898877441883087\\
1.54091227054596 0.895163059234619\\
1.55091226100922 0.891384720802307\\
1.56091225147247 0.887542426586151\\
1.57091224193573 0.883636116981506\\
1.58091223239899 0.879665791988373\\
1.59091222286224 0.87563145160675\\
1.6009122133255 0.871533095836639\\
1.61091220378876 0.86737072467804\\
1.62091219425201 0.863144159317017\\
1.63091218471527 0.858853578567505\\
1.64091217517853 0.854498744010925\\
1.65091216564178 0.850079834461212\\
1.66091227531433 0.845596611499786\\
1.67091226577759 0.841049134731293\\
1.68091225624084 0.836437404155731\\
1.6909122467041 0.831761300563812\\
1.70091223716736 0.827020823955536\\
1.71091222763062 0.822215855121613\\
1.72091221809387 0.817346453666687\\
1.73091220855713 0.81241250038147\\
1.74091219902039 0.807413935661316\\
1.75091218948364 0.802350699901581\\
1.7609121799469 0.797224819660187\\
1.77091217041016 0.792066514492035\\
1.78091216087341 0.786902129650116\\
1.79091227054596 0.781733572483063\\
1.80091226100922 0.77655440568924\\
1.81091225147247 0.771355986595154\\
1.82091224193573 0.766130089759827\\
1.83091223239899 0.760869443416595\\
1.84091222286224 0.755567908287048\\
1.8509122133255 0.75022029876709\\
1.86091220378876 0.744822382926941\\
1.87091219425201 0.73937064409256\\
1.88091218471527 0.733862102031708\\
1.89091217517853 0.728294372558594\\
1.90091216564178 0.722665250301361\\
1.91091227531433 0.716973125934601\\
1.92091226577759 0.711216390132904\\
1.93091225624084 0.705393612384796\\
1.9409122467041 0.699503660202026\\
1.95091223716736 0.693545401096344\\
1.96091222763062 0.687517642974854\\
1.97091221809387 0.681419491767883\\
1.98091220855713 0.675339877605438\\
1.99091219902039 0.671939432621002\\
2.00091218948364 0.672845304012299\\
2.0109121799469 0.67756325006485\\
2.02091217041016 0.684561610221863\\
2.03091216087341 0.692242205142975\\
2.04091215133667 0.69954389333725\\
2.05091214179993 0.70595246553421\\
2.06091213226318 0.711305320262909\\
2.07091212272644 0.715611636638641\\
2.0809121131897 0.718981444835663\\
2.09091210365295 0.721584022045136\\
2.10091233253479 0.723548352718353\\
2.11091232299805 0.72497832775116\\
2.1209123134613 0.72597324848175\\
2.13091230392456 0.726661384105682\\
2.14091229438782 0.72727245092392\\
2.15091228485107 0.728283703327179\\
2.16091227531433 0.730589926242828\\
2.17091226577759 0.735309839248657\\
2.18091225624084 0.742783904075623\\
2.1909122467041 0.751851081848145\\
2.20091223716736 0.76084440946579\\
2.21091222763062 0.768752753734589\\
2.22091221809387 0.775392532348633\\
2.23091220855713 0.780988156795502\\
2.24091219902039 0.785843372344971\\
2.25091218948364 0.79020357131958\\
2.2609121799469 0.794229090213776\\
2.27091217041016 0.79800546169281\\
2.28091216087341 0.801564812660217\\
2.29091215133667 0.804906666278839\\
2.30091214179993 0.808014094829559\\
2.31091213226318 0.810865163803101\\
2.32091212272644 0.813438892364502\\
2.3309121131897 0.815719187259674\\
2.34091210365295 0.817695617675781\\
2.35091233253479 0.819363832473755\\
2.36091232299805 0.82072502374649\\
2.3709123134613 0.821785032749176\\
2.38091230392456 0.822552978992462\\
2.39091229438782 0.823040843009949\\
2.40091228485107 0.823262214660645\\
2.41091227531433 0.823231875896454\\
2.42091226577759 0.822965145111084\\
2.43091225624084 0.822477757930756\\
2.4409122467041 0.821795642375946\\
2.45091223716736 0.820953845977783\\
2.46091222763062 0.819938063621521\\
2.47091221809387 0.81873881816864\\
2.48091220855713 0.817265868186951\\
2.49091219902039 0.815247058868408\\
2.50091218948364 0.812597692012787\\
2.5109121799469 0.80927437543869\\
2.52091217041016 0.805270433425903\\
2.53091216087341 0.800607085227966\\
2.54091215133667 0.795323967933655\\
2.55091214179993 0.789470911026001\\
2.56091213226318 0.783102810382843\\
2.57091212272644 0.776276230812073\\
2.5809121131897 0.769047498703003\\
2.59091210365295 0.761471211910248\\
2.60091233253479 0.753600120544434\\
2.61091232299805 0.745484530925751\\
2.6209123134613 0.737172305583954\\
2.63091230392456 0.72871059179306\\
2.64091229438782 0.720146477222443\\
2.65091228485107 0.711527168750763\\
2.66091227531433 0.702904999256134\\
2.67091226577759 0.694366931915283\\
2.68091225624084 0.686018884181976\\
2.6909122467041 0.677932322025299\\
2.70091223716736 0.670169293880463\\
2.71091222763062 0.662797391414642\\
2.72091221809387 0.655899703502655\\
2.73091220855713 0.64977103471756\\
2.74091219902039 0.646510541439056\\
2.75091218948364 0.647044837474823\\
2.7609121799469 0.651242554187775\\
2.77091217041016 0.658095061779022\\
2.78091216087341 0.66633677482605\\
2.79091215133667 0.674999475479126\\
2.80091214179993 0.683576941490173\\
2.81091213226318 0.691911041736603\\
2.82091212272644 0.699935555458069\\
2.8309121131897 0.707617700099945\\
2.84091210365295 0.714940965175629\\
2.85091233253479 0.721898794174194\\
2.86091232299805 0.728491544723511\\
2.8709123134613 0.734728813171387\\
2.88091230392456 0.740634500980377\\
2.89091229438782 0.746261894702911\\
2.90091228485107 0.751728773117065\\
2.91091227531433 0.757295548915863\\
2.92091226577759 0.763482868671417\\
2.93091225624084 0.771054744720459\\
2.9409122467041 0.780495941638947\\
2.95091223716736 0.791216492652893\\
2.96091222763062 0.801689028739929\\
2.97091221809387 0.81060254573822\\
2.98091220855713 0.817466020584106\\
2.99091219902039 0.822388470172882\\
3.00091218948364 0.825709521770477\\
3.0109121799469 0.827783286571503\\
3.02091217041016 0.828899741172791\\
3.03091216087341 0.829273998737335\\
3.04091215133667 0.829058825969696\\
3.05091214179993 0.828367590904236\\
3.06091213226318 0.827286183834076\\
3.07091212272644 0.825883269309998\\
3.0809121131897 0.824209868907928\\
3.09091210365295 0.822309374809265\\
3.10091233253479 0.820216357707977\\
3.11091232299805 0.817870676517487\\
3.1209123134613 0.814916551113129\\
3.13091230392456 0.811188697814941\\
3.14091229438782 0.806582510471344\\
3.15091228485107 0.801053822040558\\
3.16091227531433 0.794615924358368\\
3.17091226577759 0.78732568025589\\
3.18091225624084 0.779271304607391\\
3.1909122467041 0.77055823802948\\
3.20091223716736 0.761298298835754\\
3.21091222763062 0.751603901386261\\
3.22091221809387 0.741589844226837\\
3.23091220855713 0.731367170810699\\
3.24091219902039 0.721043705940247\\
3.25091218948364 0.710732221603394\\
3.2609121799469 0.700554370880127\\
3.27091217041016 0.690639972686768\\
3.28091216087341 0.681127607822418\\
3.29091215133667 0.672172605991364\\
3.30091214179993 0.663939714431763\\
3.31091213226318 0.656597554683685\\
3.32091212272644 0.65067994594574\\
3.3309121131897 0.647422075271606\\
3.34091210365295 0.647075235843658\\
3.35091233253479 0.649716019630432\\
3.36091232299805 0.655319929122925\\
3.3709123134613 0.663762748241425\\
3.38091230392456 0.674860596656799\\
3.39091229438782 0.688429832458496\\
3.40091228485107 0.703957259654999\\
3.41091227531433 0.720547795295715\\
3.42091226577759 0.737484812736511\\
3.43091225624084 0.754151344299316\\
3.4409122467041 0.769686877727509\\
3.45091223716736 0.78318327665329\\
3.46091222763062 0.794339239597321\\
3.47091221809387 0.803230047225952\\
3.48091220855713 0.810102581977844\\
3.49091219902039 0.815205156803131\\
3.50091218948364 0.81881046295166\\
3.5109121799469 0.821267664432526\\
3.52091217041016 0.823071300983429\\
3.53091216087341 0.824860811233521\\
3.54091215133667 0.827122986316681\\
3.55091214179993 0.829797685146332\\
3.56091213226318 0.832309544086456\\
3.57091212272644 0.834004402160645\\
3.5809121131897 0.834514856338501\\
3.59091210365295 0.83380651473999\\
3.60091233253479 0.83205384016037\\
3.61091232299805 0.829488635063171\\
3.6209123134613 0.826323986053467\\
3.63091230392456 0.822728455066681\\
3.64091229438782 0.818664371967316\\
3.65091228485107 0.8139768242836\\
3.66091227531433 0.808642268180847\\
3.67091226577759 0.802645087242126\\
3.68091225624084 0.795998632907867\\
3.6909122467041 0.788745880126953\\
3.70091223716736 0.780953586101532\\
3.71091222763062 0.772709548473358\\
3.72091221809387 0.764100432395935\\
3.73091220855713 0.755201518535614\\
3.74091219902039 0.74607390165329\\
3.75091218948364 0.736763179302216\\
3.7609121799469 0.72730964422226\\
3.77091217041016 0.717767834663391\\
3.78091216087341 0.708213984966278\\
3.79091215133667 0.698740541934967\\
3.80091214179993 0.689447343349457\\
3.81091213226318 0.680436968803406\\
3.82091212272644 0.671812653541565\\
3.8309121131897 0.663679361343384\\
3.84091210365295 0.656140565872192\\
3.85091233253479 0.649300396442413\\
3.86091232299805 0.644113719463348\\
3.8709123134613 0.641621291637421\\
3.88091230392456 0.641854405403137\\
3.89091229438782 0.644734978675842\\
3.90091228485107 0.650105714797974\\
3.91091227531433 0.657790899276733\\
3.92091226577759 0.667745530605316\\
3.93091225624084 0.679926037788391\\
3.9409122467041 0.693589627742767\\
3.95091223716736 0.708071172237396\\
3.96091222763062 0.722869753837585\\
3.97091221809387 0.737572908401489\\
3.98091220855713 0.751859486103058\\
3.99091219902039 0.765444397926331\\
4.00091218948364 0.777760922908783\\
4.01091241836548 0.78830361366272\\
4.02091217041016 0.796930611133575\\
4.03091239929199 0.803745746612549\\
4.04091215133667 0.80900377035141\\
4.05091238021851 0.813113152980804\\
4.06091213226318 0.816679358482361\\
4.07091236114502 0.820374727249146\\
4.0809121131897 0.824493527412415\\
4.09091234207153 0.828671276569366\\
4.10091209411621 0.832186818122864\\
4.11091232299805 0.834482789039612\\
4.12091207504272 0.835369169712067\\
4.13091230392456 0.834928333759308\\
4.14091205596924 0.83338063955307\\
4.15091228485107 0.830990672111511\\
4.16091203689575 0.827979862689972\\
4.17091226577759 0.824512362480164\\
4.18091201782227 0.820676684379578\\
4.1909122467041 0.816316545009613\\
4.20091199874878 0.811381101608276\\
4.21091222763062 0.805849194526672\\
4.22091197967529 0.799715459346771\\
4.23091220855713 0.792998731136322\\
4.24091243743896 0.785741984844208\\
4.25091218948364 0.778014183044434\\
4.26091241836548 0.769879996776581\\
4.27091217041016 0.761378347873688\\
4.28091239929199 0.752533137798309\\
4.29091215133667 0.743380427360535\\
4.30091238021851 0.733992457389832\\
4.31091213226318 0.724480271339417\\
4.32091236114502 0.714981496334076\\
4.3309121131897 0.70564740896225\\
4.34091234207153 0.696629405021667\\
4.35091209411621 0.68807178735733\\
4.36091232299805 0.680115759372711\\
4.37091207504272 0.672895967960358\\
4.38091230392456 0.666680812835693\\
4.39091205596924 0.662864446640015\\
4.40091228485107 0.66170346736908\\
4.41091203689575 0.663016438484192\\
4.42091226577759 0.666563928127289\\
4.43091201782227 0.672081470489502\\
4.4409122467041 0.679303050041199\\
4.45091199874878 0.688007295131683\\
4.46091222763062 0.698078036308289\\
4.47091197967529 0.709319770336151\\
4.48091220855713 0.72116231918335\\
4.49091243743896 0.733215630054474\\
4.50091218948364 0.745215177536011\\
4.51091241836548 0.756957590579987\\
4.52091217041016 0.768231749534607\\
4.53091239929199 0.77852338552475\\
4.54091215133667 0.787356317043304\\
4.55091238021851 0.794566750526428\\
4.56091213226318 0.800205826759338\\
4.57091236114502 0.804476082324982\\
4.5809121131897 0.80775111913681\\
4.59091234207153 0.810625851154327\\
4.60091209411621 0.81378048658371\\
4.61091232299805 0.81751012802124\\
4.62091207504272 0.821420311927795\\
4.63091230392456 0.824738562107086\\
4.64091205596924 0.826858282089233\\
4.65091228485107 0.82756108045578\\
4.66091203689575 0.826919853687286\\
4.67091226577759 0.825155735015869\\
4.68091201782227 0.822548687458038\\
4.6909122467041 0.819331645965576\\
4.70091199874878 0.815654814243317\\
4.71091222763062 0.811395406723022\\
4.72091197967529 0.806496143341064\\
4.73091220855713 0.800960004329681\\
4.74091243743896 0.79479992389679\\
4.75091218948364 0.788051009178162\\
4.76091241836548 0.780767917633057\\
4.77091217041016 0.773018836975098\\
4.78091239929199 0.764889299869537\\
4.79091215133667 0.756460309028625\\
4.80091238021851 0.747780621051788\\
4.81091213226318 0.73886501789093\\
4.82091236114502 0.729726612567902\\
4.8309121131897 0.720416605472565\\
4.84091234207153 0.711039304733276\\
4.85091209411621 0.701741516590118\\
4.86091232299805 0.692694783210754\\
4.87091207504272 0.684087634086609\\
4.88091230392456 0.676126956939697\\
4.89091205596924 0.669027149677277\\
4.90091228485107 0.663430571556091\\
4.91091203689575 0.661058127880096\\
4.92091226577759 0.661990106105804\\
4.93091201782227 0.665974736213684\\
4.9409122467041 0.672668695449829\\
4.95091199874878 0.681691586971283\\
4.96091222763062 0.692631900310516\\
4.97091197967529 0.704998373985291\\
4.98091220855713 0.718379437923431\\
4.99091243743896 0.732422590255737\\
5 0.745228171348572\\
};
\addlegendentry{$\dtmax{} = \expneg{1.0}{3}{[s]}$};

\addplot [
color=black,
solid
]
table[row sep=crcr]{
0 1\\
0.0100556658580899 1\\
0.0200556665658951 1\\
0.0300556663423777 1\\
0.0400556661188602 1\\
0.050055667757988 1\\
0.0600556656718254 1\\
0.0700556635856628 1\\
0.0800556689500809 1\\
0.0900556668639183 1\\
0.100055664777756 1\\
0.110055662691593 1\\
0.120055668056011 1\\
0.130055665969849 1\\
0.140055671334267 1\\
0.150055661797524 1\\
0.160055667161942 1\\
0.17005567252636 1\\
0.180055662989616 1\\
0.190055668354034 1\\
0.200055658817291 1\\
0.210055664181709 1\\
0.220055669546127 1\\
0.230055660009384 1\\
0.240055665373802 1\\
0.25005567073822 1\\
0.260055661201477 1\\
0.270055651664734 1\\
0.280055671930313 1\\
0.29005566239357 1\\
0.300055652856827 1\\
0.310055673122406 1\\
0.320055663585663 1\\
0.33005565404892 1\\
0.340055674314499 1\\
0.350055664777756 1\\
0.360055655241013 1\\
0.370055675506592 1\\
0.380055665969849 1\\
0.390055656433105 1\\
0.400055676698685 1\\
0.410055667161942 1\\
0.420055657625198 1\\
0.430055677890778 1\\
0.440055668354034 1\\
0.450055658817291 1\\
0.46005567908287 1\\
0.470055669546127 1\\
0.480055660009384 1\\
0.490055680274963 1\\
0.50005567073822 1\\
0.510055661201477 1\\
0.520055651664734 1\\
0.530055642127991 1\\
0.540055692195892 1\\
0.550055682659149 1\\
0.560055673122406 1\\
0.570055663585663 1\\
0.58005565404892 1\\
0.590055644512177 1\\
0.600055694580078 1\\
0.610055685043335 1\\
0.620055675506592 1\\
0.630055665969849 1\\
0.640055656433105 1\\
0.650055646896362 1\\
0.660055637359619 1\\
0.670055687427521 1\\
0.680055677890778 1\\
0.690055668354034 1\\
0.700055658817291 1\\
0.710055649280548 1\\
0.720055639743805 1\\
0.730055689811707 1\\
0.740055680274963 1\\
0.75005567073822 1\\
0.760055661201477 1\\
0.770055651664734 1\\
0.780055642127991 1\\
0.790055692195892 1\\
0.800055682659149 1\\
0.810055673122406 1\\
0.820055663585663 1\\
0.83005565404892 1\\
0.840055644512177 1\\
0.850055694580078 1\\
0.860055685043335 1\\
0.870055675506592 1\\
0.880055665969849 1\\
0.890055656433105 1\\
0.900055646896362 1\\
0.910055637359619 1\\
0.920055687427521 1\\
0.930055677890778 1\\
0.940055668354034 1\\
0.950055658817291 1\\
0.960055649280548 1\\
0.970055639743805 1\\
0.980055689811707 1\\
0.990055680274963 1\\
1.00005567073822 1\\
1.01005566120148 0.999947726726532\\
1.02005565166473 0.999781370162964\\
1.03005564212799 0.999619662761688\\
1.04005563259125 0.999391496181488\\
1.0500556230545 0.999095261096954\\
1.06005561351776 0.998729705810547\\
1.07005572319031 0.998293340206146\\
1.08005571365356 0.997784972190857\\
1.09005570411682 0.997203409671783\\
1.10005569458008 0.996547341346741\\
1.11005568504334 0.995815694332123\\
1.12005567550659 0.995007336139679\\
1.13005566596985 0.994121074676514\\
1.14005565643311 0.993155956268311\\
1.15005564689636 0.992110908031464\\
1.16005563735962 0.990984857082367\\
1.17005562782288 0.989776909351349\\
1.18005561828613 0.988485991954803\\
1.19005560874939 0.987111330032349\\
1.20005571842194 0.985651910305023\\
1.21005570888519 0.9841068983078\\
1.22005569934845 0.982475340366364\\
1.23005568981171 0.980756521224976\\
1.24005568027496 0.97895073890686\\
1.25005567073822 0.977074503898621\\
1.26005566120148 0.975142180919647\\
1.27005565166473 0.97315526008606\\
1.28005564212799 0.971111655235291\\
1.29005563259125 0.969008803367615\\
1.3000556230545 0.96684467792511\\
1.31005561351776 0.964617788791656\\
1.32005572319031 0.962327182292938\\
1.33005571365356 0.959972321987152\\
1.34005570411682 0.957552969455719\\
1.35005569458008 0.955068826675415\\
1.36005568504334 0.95251989364624\\
1.37005567550659 0.949906170368195\\
1.38005566596985 0.947227776050568\\
1.39005565643311 0.944484651088715\\
1.40005564689636 0.94167697429657\\
1.41005563735962 0.938804805278778\\
1.42005562782288 0.935868203639984\\
1.43005561828613 0.932867228984833\\
1.44005560874939 0.929802000522614\\
1.45005571842194 0.926672518253326\\
1.46005570888519 0.923478901386261\\
1.47005569934845 0.920221149921417\\
1.48005568981171 0.91689932346344\\
1.49005568027496 0.913513422012329\\
1.50005567073822 0.910063564777374\\
1.51005566120148 0.906549632549286\\
1.52005565166473 0.902971744537354\\
1.53005564212799 0.899329900741577\\
1.54005563259125 0.895624041557312\\
1.5500556230545 0.891854226589203\\
1.56005561351776 0.88802045583725\\
1.57005572319031 0.884122669696808\\
1.58005571365356 0.880160868167877\\
1.59005570411682 0.876135051250458\\
1.60005569458008 0.872045159339905\\
1.61005568504334 0.867891252040863\\
1.62005567550659 0.863673269748688\\
1.63005566596985 0.859391152858734\\
1.64005565643311 0.855044841766357\\
1.65005564689636 0.850634396076202\\
1.66005563735962 0.846159696578979\\
1.67005562782288 0.841620743274689\\
1.68005561828613 0.837017476558685\\
1.69005560874939 0.832349836826324\\
1.70005571842194 0.827617883682251\\
1.71005570888519 0.822821438312531\\
1.72005569934845 0.817960500717163\\
1.73005568981171 0.813035070896149\\
1.74005568027496 0.808045029640198\\
1.75005567073822 0.802990317344666\\
1.76005566120148 0.797871887683868\\
1.77005565166473 0.792716145515442\\
1.78005564212799 0.787553131580353\\
1.79005563259125 0.782386720180511\\
1.8000556230545 0.777210831642151\\
1.81005561351776 0.772016942501068\\
1.82005572319031 0.76679652929306\\
1.83005571365356 0.761542141437531\\
1.84005570411682 0.756247460842133\\
1.85005569458008 0.750907182693481\\
1.86005568504334 0.745516955852509\\
1.87005567550659 0.740073204040527\\
1.88005566596985 0.73457282781601\\
1.89005565643311 0.729013383388519\\
1.90005564689636 0.72339278459549\\
1.91005563735962 0.717709183692932\\
1.92005562782288 0.711961030960083\\
1.93005561828613 0.706147015094757\\
1.94005560874939 0.700265824794769\\
1.95005571842194 0.694316387176514\\
1.96005570888519 0.68829756975174\\
1.97005569934845 0.682208359241486\\
1.98005568981171 0.676063299179077\\
1.99005568027496 0.67209005355835\\
2.00005555152893 0.672386348247528\\
2.01005578041077 0.676676213741302\\
2.02005577087402 0.68353408575058\\
2.03005576133728 0.691292345523834\\
2.04005575180054 0.698781788349152\\
2.05005574226379 0.705409705638886\\
2.06005573272705 0.710973799228668\\
2.07005572319031 0.715468525886536\\
2.08005571365356 0.718993604183197\\
2.09005570411682 0.721721053123474\\
2.10005569458008 0.723787128925323\\
2.11005568504334 0.725301206111908\\
2.12005567550659 0.726367950439453\\
2.13005566596985 0.727122843265533\\
2.14005565643311 0.727808713912964\\
2.15005564689636 0.728926301002502\\
2.16005563735962 0.731399357318878\\
2.17005562782288 0.73634797334671\\
2.18005561828613 0.744077146053314\\
2.19005560874939 0.753436326980591\\
2.20005559921265 0.762768507003784\\
2.2100555896759 0.771009683609009\\
2.22005558013916 0.7779141664505\\
2.23005557060242 0.783676147460938\\
2.24005556106567 0.788599252700806\\
2.25005555152893 0.792941212654114\\
2.26005578041077 0.796880662441254\\
2.27005577087402 0.800520837306976\\
2.28005576133728 0.80391001701355\\
2.29005575180054 0.807060956954956\\
2.30005574226379 0.809967458248138\\
2.31005573272705 0.812615513801575\\
2.32005572319031 0.814990222454071\\
2.33005571365356 0.817079603672028\\
2.34005570411682 0.818875849246979\\
2.35005569458008 0.820376336574554\\
2.36005568504334 0.821582853794098\\
2.37005567550659 0.822501242160797\\
2.38005566596985 0.823140442371368\\
2.39005565643311 0.823511838912964\\
2.40005564689636 0.82362824678421\\
2.41005563735962 0.823503613471985\\
2.42005562782288 0.823152422904968\\
2.43005561828613 0.822600305080414\\
2.44005560874939 0.821887075901031\\
2.45005559921265 0.820997178554535\\
2.4600555896759 0.819917798042297\\
2.47005558013916 0.818645119667053\\
2.48005557060242 0.817046284675598\\
2.49005556106567 0.814895510673523\\
2.50005555152893 0.812122464179993\\
2.51005578041077 0.80868798494339\\
2.52005577087402 0.804587244987488\\
2.53005576133728 0.799841463565826\\
2.54005575180054 0.794488906860352\\
2.55005574226379 0.788577437400818\\
2.56005573272705 0.782159924507141\\
2.57005572319031 0.775290787220001\\
2.58005571365356 0.768024504184723\\
2.59005570411682 0.760414183139801\\
2.60005569458008 0.752511143684387\\
2.61005568504334 0.744364559650421\\
2.62005567550659 0.736021995544434\\
2.63005566596985 0.727531373500824\\
2.64005565643311 0.71893972158432\\
2.65005564689636 0.710294961929321\\
2.66005563735962 0.701651096343994\\
2.67005562782288 0.693099558353424\\
2.68005561828613 0.684744715690613\\
2.69005560874939 0.676657378673553\\
2.70005559921265 0.668899893760681\\
2.7100555896759 0.661541342735291\\
2.72005558013916 0.654667913913727\\
2.73005557060242 0.648640632629395\\
2.74005556106567 0.645641207695007\\
2.75005555152893 0.646485567092896\\
2.76005578041077 0.650976717472076\\
2.77005577087402 0.658193469047546\\
2.78005576133728 0.66678649187088\\
2.79005575180054 0.675744950771332\\
2.80005574226379 0.684580028057098\\
2.81005573272705 0.693134725093842\\
2.82005572319031 0.701343536376953\\
2.83005571365356 0.709175288677216\\
2.84005570411682 0.716615855693817\\
2.85005569458008 0.72366189956665\\
2.86005568504334 0.7303187251091\\
2.87005567550659 0.736603736877441\\
2.88005566596985 0.742558658123016\\
2.89005565643311 0.748276948928833\\
2.90005564689636 0.753968060016632\\
2.91005563735962 0.760070383548737\\
2.92005562782288 0.767307102680206\\
2.93005561828613 0.776335954666138\\
2.94005560874939 0.786964654922485\\
2.95005559921265 0.797959506511688\\
2.9600555896759 0.807880282402039\\
2.97005558013916 0.81595903635025\\
2.98005557060242 0.822075843811035\\
2.99005556106567 0.826438426971436\\
3.00005555152893 0.829345643520355\\
3.01005578041077 0.831075966358185\\
3.02005577087402 0.831853926181793\\
3.03005576133728 0.831849098205566\\
3.04005575180054 0.831187069416046\\
3.05005574226379 0.829966425895691\\
3.06005573272705 0.828271448612213\\
3.07005572319031 0.826188981533051\\
3.08005571365356 0.823799312114716\\
3.09005570411682 0.82111781835556\\
3.10005569458008 0.817826747894287\\
3.11005568504334 0.813765704631805\\
3.12005567550659 0.808844029903412\\
3.13005566596985 0.803015470504761\\
3.14005565643311 0.796288788318634\\
3.15005564689636 0.788721442222595\\
3.16005563735962 0.780405402183533\\
3.17005562782288 0.771453619003296\\
3.18005561828613 0.761985778808594\\
3.19005560874939 0.752123653888702\\
3.20005559921265 0.741985499858856\\
3.2100555896759 0.731679022312164\\
3.22005558013916 0.721302568912506\\
3.23005557060242 0.710951209068298\\
3.24005556106567 0.700726389884949\\
3.25005555152893 0.690740168094635\\
3.26005578041077 0.681114971637726\\
3.27005577087402 0.671987593173981\\
3.28005576133728 0.66350930929184\\
3.29005575180054 0.655835926532745\\
3.30005574226379 0.649209558963776\\
3.31005573272705 0.644904553890228\\
3.32005572319031 0.643498599529266\\
3.33005571365356 0.64510852098465\\
3.34005570411682 0.649744868278503\\
3.35005569458008 0.657317638397217\\
3.36005568504334 0.667640447616577\\
3.37005567550659 0.680481612682343\\
3.38005566596985 0.695534467697144\\
3.39005565643311 0.712074100971222\\
3.40005564689636 0.729236483573914\\
3.41005563735962 0.746363162994385\\
3.42005562782288 0.762783706188202\\
3.43005561828613 0.777498185634613\\
3.44005560874939 0.789928197860718\\
3.45005559921265 0.800009071826935\\
3.4600555896759 0.807941973209381\\
3.47005558013916 0.813996732234955\\
3.48005557060242 0.818458497524261\\
3.49005556106567 0.821704208850861\\
3.50005555152893 0.824261784553528\\
3.51005578041077 0.826777279376984\\
3.52005577087402 0.829691767692566\\
3.53005576133728 0.832885086536407\\
3.54005575180054 0.835778415203094\\
3.55005574226379 0.837765455245972\\
3.56005573272705 0.838518559932709\\
3.57005572319031 0.83800995349884\\
3.58005571365356 0.836407482624054\\
3.59005570411682 0.833948254585266\\
3.60005569458008 0.830848813056946\\
3.61005568504334 0.827280640602112\\
3.62005567550659 0.823312878608704\\
3.63005566596985 0.818763494491577\\
3.64005565643311 0.813595592975616\\
3.65005564689636 0.807788193225861\\
3.66005563735962 0.801341354846954\\
3.67005562782288 0.794283509254456\\
3.68005561828613 0.786669313907623\\
3.69005560874939 0.778579294681549\\
3.70005559921265 0.770098984241486\\
3.7100555896759 0.761303424835205\\
3.72005558013916 0.752253770828247\\
3.73005557060242 0.742995738983154\\
3.74005556106567 0.733568131923676\\
3.75005555152893 0.724023342132568\\
3.76005578041077 0.714435636997223\\
3.77005577087402 0.704895675182343\\
3.78005576133728 0.695501089096069\\
3.79005575180054 0.686350643634796\\
3.80005574226379 0.677541851997375\\
3.81005573272705 0.66917222738266\\
3.82005572319031 0.661339938640594\\
3.83005571365356 0.654139757156372\\
3.84005570411682 0.647838056087494\\
3.85005569458008 0.643849015235901\\
3.86005568504334 0.642520546913147\\
3.87005567550659 0.643815875053406\\
3.88005566596985 0.647614002227783\\
3.89005565643311 0.653734922409058\\
3.90005564689636 0.662006735801697\\
3.91005563735962 0.672401368618011\\
3.92005562782288 0.684789538383484\\
3.93005561828613 0.698384404182434\\
3.94005560874939 0.712630689144135\\
3.95005559921265 0.727082908153534\\
3.9600555896759 0.741376936435699\\
3.97005558013916 0.755227625370026\\
3.98005557060242 0.768339514732361\\
3.99005556106567 0.780121266841888\\
4.00005578994751 0.790145397186279\\
4.01005554199219 0.798339188098907\\
4.02005577087402 0.804884254932404\\
4.0300555229187 0.810170590877533\\
4.04005575180054 0.814807116985321\\
4.05005550384521 0.819474101066589\\
4.06005573272705 0.8244708776474\\
4.07005548477173 0.82944130897522\\
4.08005571365356 0.833689272403717\\
4.09005546569824 0.836669743061066\\
4.10005569458008 0.838184356689453\\
4.11005544662476 0.838298380374908\\
4.12005567550659 0.837203443050385\\
4.13005590438843 0.835140347480774\\
4.14005565643311 0.832354247570038\\
4.15005588531494 0.829033613204956\\
4.16005563735962 0.825311899185181\\
4.17005586624146 0.821146011352539\\
4.18005561828613 0.816400051116943\\
4.19005584716797 0.811061263084412\\
4.20005559921265 0.80511885881424\\
4.21005582809448 0.798581898212433\\
4.22005558013916 0.79148256778717\\
4.230055809021 0.783878564834595\\
4.24005556106567 0.775846481323242\\
4.25005578994751 0.767446100711823\\
4.26005554199219 0.758713245391846\\
4.27005577087402 0.749672114849091\\
4.2800555229187 0.740365564823151\\
4.29005575180054 0.730874121189117\\
4.30005550384521 0.721313059329987\\
4.31005573272705 0.711820244789124\\
4.32005548477173 0.702544331550598\\
4.33005571365356 0.693632245063782\\
4.34005546569824 0.685223162174225\\
4.35005569458008 0.677453458309174\\
4.36005544662476 0.670450210571289\\
4.37005567550659 0.664719581604004\\
4.38005590438843 0.661634206771851\\
4.39005565643311 0.661201000213623\\
4.40005588531494 0.663207650184631\\
4.41005563735962 0.667396366596222\\
4.42005586624146 0.673490226268768\\
4.43005561828613 0.681215643882751\\
4.44005584716797 0.690338432788849\\
4.45005559921265 0.700692176818848\\
4.46005582809448 0.712074756622314\\
4.47005558013916 0.723930776119232\\
4.480055809021 0.73590487241745\\
4.49005556106567 0.747757852077484\\
4.50005578994751 0.759305477142334\\
4.51005554199219 0.770326733589172\\
4.52005577087402 0.780300498008728\\
4.5300555229187 0.788813829421997\\
4.54005575180054 0.795752704143524\\
4.55005550384521 0.801227271556854\\
4.56005573272705 0.80554586648941\\
4.57005548477173 0.809248089790344\\
4.58005571365356 0.813024342060089\\
4.59005546569824 0.81730192899704\\
4.60005569458008 0.821843981742859\\
4.61005544662476 0.825940132141113\\
4.62005567550659 0.828926265239716\\
4.63005590438843 0.830485224723816\\
4.64005565643311 0.83061558008194\\
4.65005588531494 0.829485952854156\\
4.66005563735962 0.827337563037872\\
4.67005586624146 0.824433743953705\\
4.68005561828613 0.820978105068207\\
4.69005584716797 0.817043900489807\\
4.70005559921265 0.812465727329254\\
4.71005582809448 0.807241320610046\\
4.72005558013916 0.801378130912781\\
4.730055809021 0.794898331165314\\
4.74005556106567 0.787845134735107\\
4.75005578994751 0.780279695987701\\
4.76005554199219 0.772278010845184\\
4.77005577087402 0.763931691646576\\
4.7800555229187 0.755318582057953\\
4.79005575180054 0.746475577354431\\
4.80005550384521 0.737411618232727\\
4.81005573272705 0.728146433830261\\
4.82005548477173 0.718743741512299\\
4.83005571365356 0.709317624568939\\
4.84005546569824 0.700017869472504\\
4.85005569458008 0.691015005111694\\
4.86005544662476 0.682494878768921\\
4.87005567550659 0.674660682678223\\
4.88005590438843 0.667719841003418\\
4.89005565643311 0.662560164928436\\
4.90005588531494 0.66076785326004\\
4.91005563735962 0.662286579608917\\
4.92005586624146 0.666826069355011\\
4.93005561828613 0.67401248216629\\
4.94005584716797 0.683441042900085\\
4.95005559921265 0.694668292999268\\
4.96005582809448 0.707163274288177\\
4.97005558013916 0.720495522022247\\
4.980055809021 0.734345257282257\\
4.99005556106567 0.748146116733551\\
5 0.761430323123932\\
};
\addlegendentry{$\dtmax{} = \expneg{1.0}{4}{[s]}$};

\end{axis}
\end{tikzpicture}%
\caption{Linear solver timestep-size insensitive flashing solution.}
\label{fig:flashingDtInsensitiveLin}
\end{figure}

The timestep-size insensitive solution produced by the nonlinear solver occurs at a maximum timestep size three orders of magnitude greater than that achieved by the linear solver in \cobra{}.
An examination of the nonlinear residual over the course of the transient provides insight into why this behavior is observed.
For the linear runs, the scaled residuals were evaluated after the single Newton step, $\tilde{\vec{F}}(\vec{x}^{1})$.
For the nonlinear runs, the scaled nonlinear residuals were evaluated at the end of the Newton-loop.
The scaled residual after the single Newton step of the linear solver for flashing problem indicates that, even for small timestep sizes, the solution obtained still does not satisfy the discrete nonlinear equations, \fig{fig:flashingResidualLin}.
This is in contrast to the nonlinear flashing problem, which shows a lower residual over the course of the simulation, \fig{fig:flashingResidualNln}.

\begin{figure}[h!tb]
\centering
% This file was created by matlab2tikz v0.4.3.
% Copyright (c) 2008--2013, Nico Schlömer <nico.schloemer@gmail.com>
% All rights reserved.
% 
\tikzsetnextfilename{plots/flashingResidualLin_pdf}
\begin{tikzpicture}

\begin{axis}[%
width=\mytikzpicwidth,
height=\mytikzpicheight,
scale only axis,
xmin=0.0,
xmax=5.0,
xlabel={Time $[\text{s}]$},
ymin=0.0,
ymax=1.0,
ylabel={$\frac{||\tilde{\vec{F}}^{1}||_2}{\sqrt{N}} [\text{-}]$},
legend style={draw=black,fill=white,legend cell align=left}
]
\addplot [
color=black,
solid
]
table[row sep=crcr]{
0.0 0.0\\
0.0118019115179777 0.0587126165628433\\
0.0226854234933853 0.0285983271896839\\
0.0340500771999359 0.00159192155115306\\
0.0491764321923256 0.000239071261603385\\
0.0619497485458851 9.50025860220194e-05\\
0.0774054601788521 4.91790633532219e-05\\
0.0961068719625473 1.92077841347782e-05\\
0.106882445514202 1.07606856545317e-05\\
0.118735581636429 5.95184746998711e-06\\
0.13177402317524 3.7097906897543e-06\\
0.146116316318512 2.94333449346595e-06\\
0.161892831325531 2.91213154923753e-06\\
0.179247006773949 3.19456057695788e-06\\
0.198336601257324 3.58624720320222e-06\\
0.21933513879776 4.01584566134261e-06\\
0.242433547973633 4.46334524895065e-06\\
0.267841786146164 4.92200570079149e-06\\
0.29579085111618 5.3815997489437e-06\\
0.32653483748436 5.82741313337465e-06\\
0.360353201627731 6.24657468506484e-06\\
0.397553414106369 6.62043157717562e-06\\
0.438473641872406 6.93314814270707e-06\\
0.483485877513886 7.1661884248897e-06\\
0.532999336719513 7.30282272343175e-06\\
0.587464153766632 7.32965463612345e-06\\
0.647375464439392 7.23708808436641e-06\\
0.71327793598175 7.02161696608528e-06\\
0.7857705950737 6.68499023959157e-06\\
0.865512549877167 6.23925961917848e-06\\
0.953228712081909 5.69963003727025e-06\\
1.04971647262573 0.33204597234726\\
1.1558530330658 0.336790084838867\\
1.2407523393631 0.328962415456772\\
1.29757130146027 0.333233326673508\\
1.34479975700378 0.33340972661972\\
1.38458728790283 0.329768389463425\\
1.41912090778351 0.333215266466141\\
1.45013403892517 0.333174020051956\\
1.47872579097748 0.332872986793518\\
1.5054817199707 0.330253571271896\\
1.53073275089264 0.318578839302063\\
1.55469310283661 0.332582265138626\\
1.57754015922546 0.33292755484581\\
1.59942138195038 0.32984733581543\\
1.62045061588287 0.332773387432098\\
1.64072275161743 0.332683533430099\\
1.6603125333786 0.332362473011017\\
1.67928779125214 0.332771897315979\\
1.69770324230194 0.332803398370743\\
1.71560847759247 0.332294225692749\\
1.73303949832916 0.332072615623474\\
1.75003552436829 0.331977009773254\\
1.76655149459839 0.329615294933319\\
1.78238117694855 0.332006931304932\\
1.79760921001434 0.331158697605133\\
1.81230998039246 0.331648468971252\\
1.82662653923035 0.33092138171196\\
1.84061861038208 0.331399768590927\\
1.85433006286621 0.331704914569855\\
1.86779499053955 0.331235855817795\\
1.88103151321411 0.331660598516464\\
1.89406371116638 0.330760926008224\\
1.90690815448761 0.331171303987503\\
1.91956806182861 0.331273853778839\\
1.93206679821014 0.329556882381439\\
1.94440972805023 0.331549435853958\\
1.95660436153412 0.33220100402832\\
1.96866297721863 0.332119524478912\\
1.98027718067169 0.34158843755722\\
1.99125838279724 0.375829488039017\\
2.00170540809631 0.380835026502609\\
2.01211428642273 0.378497421741486\\
2.02296447753906 0.360979914665222\\
2.03420996665955 0.372260570526123\\
2.04571580886841 0.393531978130341\\
2.05740904808044 0.397901386022568\\
2.06921482086182 0.408225029706955\\
2.08101534843445 0.427144706249237\\
2.09267902374268 0.423482120037079\\
2.10420393943787 0.430286318063736\\
2.1155960559845 0.439723670482636\\
2.12691164016724 0.456177055835724\\
2.13817572593689 0.465044111013412\\
2.14939737319946 0.466731756925583\\
2.16053771972656 0.4650918841362\\
2.17153453826904 0.449060082435608\\
2.18234181404114 0.441680520772934\\
2.19297122955322 0.429318308830261\\
2.2034854888916 0.426954001188278\\
2.21399807929993 0.414101839065552\\
2.22421646118164 0.413749009370804\\
2.24485516548157 0.393529415130615\\
2.2556414604187 0.394234269857407\\
2.26643514633179 0.377154558897018\\
2.27722930908203 0.365379899740219\\
2.28802180290222 0.343568444252014\\
2.2988224029541 0.331944763660431\\
2.30910420417786 0.310833752155304\\
2.32745814323425 0.276948511600494\\
2.34450769424438 0.238746866583824\\
2.36390542984009 0.188516572117805\\
2.37500143051147 0.155285388231277\\
2.38614940643311 0.105763450264931\\
2.39734673500061 0.0547187589108944\\
2.40859293937683 0.0340193621814251\\
2.41979336738586 0.0286634601652622\\
2.43107962608337 0.0256046261638403\\
2.44180870056152 0.0283177811652422\\
2.4530987739563 0.029298972338438\\
2.46468305587769 0.0179398227483034\\
2.47643661499023 0.0148950954899192\\
2.48832750320435 0.0136609487235546\\
2.50033926963806 0.012873187661171\\
2.5124671459198 0.0122316880151629\\
2.5246946811676 0.0115456944331527\\
2.53700971603394 0.0108857564628124\\
2.54939413070679 0.0102071752771735\\
2.56182551383972 0.00959595292806625\\
2.57428097724915 0.00907755084335804\\
2.58673787117004 0.00881083495914936\\
2.59917783737183 0.00895689241588116\\
2.61158609390259 0.0095886243507266\\
2.62395215034485 0.0128246489912272\\
2.6361358165741 0.0230051130056381\\
2.64810156822205 0.10405508428812\\
2.6597592830658 0.0660716518759727\\
2.67102980613709 0.0374695770442486\\
2.68197822570801 0.0245122108608484\\
2.69267916679382 0.129981815814972\\
2.70332479476929 0.136767461895943\\
2.71397757530212 0.137567281723022\\
2.72448706626892 0.136601120233536\\
2.74232935905457 0.149624675512314\\
2.75233578681946 0.199981942772865\\
2.76248455047607 0.192638456821442\\
2.77299618721008 0.192955374717712\\
2.78397274017334 0.201442912220955\\
2.79858231544495 0.222252145409584\\
2.81091928482056 0.218545794487\\
2.82584691047668 0.232722252607346\\
2.84390950202942 0.238271445035934\\
2.85421967506409 0.273861825466156\\
2.86443328857422 0.276464879512787\\
2.87539649009705 0.276581138372421\\
2.88654828071594 0.276655524969101\\
2.89785838127136 0.276325285434723\\
2.90905976295471 0.284836232662201\\
2.92766308784485 0.306283831596375\\
2.9380214214325 0.301489144563675\\
2.94941568374634 0.291349053382874\\
2.96126461029053 0.278435468673706\\
2.97307300567627 0.256808876991272\\
2.98484778404236 0.227518945932388\\
2.99659848213196 0.202596515417099\\
3.00838160514832 0.177799984812737\\
3.02029156684875 0.155081480741501\\
3.03234505653381 0.149996012449265\\
3.04447245597839 0.116311855614185\\
3.05664157867432 0.0882383957505226\\
3.06882643699646 0.072237990796566\\
3.08100342750549 0.0709943175315857\\
3.09315085411072 0.070290133357048\\
3.10524773597717 0.0786574706435204\\
3.11726975440979 0.0694149434566498\\
3.12919759750366 0.0802714079618454\\
3.14099335670471 0.0846292823553085\\
3.15258574485779 0.0731671378016472\\
3.16393399238586 0.0722993239760399\\
3.1750283241272 0.0649187043309212\\
3.18587756156921 0.0635097250342369\\
3.19650077819824 0.061320599168539\\
3.20691919326782 0.052296593785286\\
3.21715044975281 0.144148319959641\\
3.22718858718872 0.142802685499191\\
3.2467474937439 0.175584524869919\\
3.26547813415527 0.146210059523582\\
3.28411507606506 0.190335065126419\\
3.29831433296204 0.193450957536697\\
3.3087158203125 0.194435074925423\\
3.32302951812744 0.201969102025032\\
3.33405566215515 0.225963354110718\\
3.3473973274231 0.223974734544754\\
3.36354088783264 0.22722715139389\\
3.38307428359985 0.157342448830605\\
3.3943293094635 0.123202979564667\\
3.40593409538269 0.09215147793293\\
3.41763758659363 0.0698718801140785\\
3.42948269844055 0.0598904937505722\\
3.44148325920105 0.0610218644142151\\
3.45353007316589 0.0589085854589939\\
3.4655749797821 0.0575833171606064\\
3.48354387283325 0.0413004830479622\\
3.4938976764679 0.0383882187306881\\
3.50528645515442 0.064517542719841\\
3.51756191253662 0.0769790410995483\\
3.52981090545654 0.0396254248917103\\
3.54198026657104 0.0351126715540886\\
3.55405426025391 0.0330242402851582\\
3.56603574752808 0.0350552871823311\\
3.57792735099792 0.0394757129251957\\
3.58974385261536 0.0690236911177635\\
3.60148811340332 0.0728900954127312\\
3.61318278312683 0.0380488224327564\\
3.62483644485474 0.0365817695856094\\
3.63638854026794 0.0358196794986725\\
3.64775395393372 0.0329988226294518\\
3.65890121459961 0.0298173706978559\\
3.66983270645142 0.0282572060823441\\
3.68056321144104 0.0269394367933273\\
3.69111466407776 0.0239591002464294\\
3.70150089263916 0.0322645008563995\\
3.71173763275146 0.137666195631027\\
3.72185039520264 0.1380345672369\\
3.73186469078064 0.138317823410034\\
3.75170636177063 0.137706309556961\\
3.7714056968689 0.143020510673523\\
3.7911114692688 0.141689538955688\\
3.8088846206665 0.139084339141846\\
3.82693791389465 0.151857599616051\\
3.84111881256104 0.148120939731598\\
3.85744023323059 0.131235837936401\\
3.87718939781189 0.0746459737420082\\
3.88805747032166 0.0389378033578396\\
3.89892983436584 0.0291490629315376\\
3.90982556343079 0.0290522556751966\\
3.92078495025635 0.0286090020090342\\
3.9318573474884 0.0280207097530365\\
3.94304871559143 0.0864942669868469\\
3.95445036888123 0.097194641828537\\
3.96601796150208 0.0498479194939137\\
3.97746133804321 0.0405964814126492\\
3.98889112472534 0.0300959385931492\\
4.00039577484131 0.0245345272123814\\
4.01201629638672 0.0223103631287813\\
4.02377510070801 0.0756979733705521\\
4.03563594818115 0.0477104969322681\\
4.04752445220947 0.029522854834795\\
4.05936098098755 0.0216570664197207\\
4.0711030960083 0.0186283849179745\\
4.08272695541382 0.0215883832424879\\
4.09423494338989 0.0357664190232754\\
4.10562992095947 0.0989215970039368\\
4.11694383621216 0.0482349656522274\\
4.12819576263428 0.0381343513727188\\
4.13939952850342 0.0431578867137432\\
4.1505651473999 0.0500016584992409\\
4.16158246994019 0.0589028410613537\\
4.17241191864014 0.0371414832770824\\
4.18306303024292 0.0333185307681561\\
4.1935658454895 0.0321636460721493\\
4.2039475440979 0.0322539247572422\\
4.21423053741455 0.0303367022424936\\
4.22443199157715 0.0292453337460756\\
4.2345666885376 0.139310091733933\\
4.24464702606201 0.140263512730598\\
4.25469064712524 0.141089901328087\\
4.26471757888794 0.140937030315399\\
4.27471828460693 0.148881167173386\\
4.2847638130188 0.147169575095177\\
4.29489898681641 0.144873455166817\\
4.30505514144897 0.144310653209686\\
4.31983995437622 0.156272888183594\\
4.33454179763794 0.192166447639465\\
4.34839296340942 0.192323550581932\\
4.36166000366211 0.16527134180069\\
4.37771320343018 0.14575856924057\\
4.39713716506958 0.0642791166901588\\
4.40819406509399 0.0359888225793839\\
4.41921901702881 0.0399795360863209\\
4.43028879165649 0.0407806932926178\\
4.44142246246338 0.0355032943189144\\
4.45270013809204 0.0331317223608494\\
4.46414184570313 0.101089060306549\\
4.47538042068481 0.064799576997757\\
4.48650979995728 0.0492305383086205\\
4.49770545959473 0.0391870774328709\\
4.50898551940918 0.0370247662067413\\
4.52039957046509 0.0369821973145008\\
4.5319881439209 0.0650779902935028\\
4.54377889633179 0.0739063695073128\\
4.55586624145508 0.0412180013954639\\
4.56815242767334 0.0297446176409721\\
4.58033037185669 0.0226887091994286\\
4.59236669540405 0.0197397097945213\\
4.60425090789795 0.0214046780019999\\
4.6159782409668 0.0270643047988415\\
4.62757444381714 0.0332201756536961\\
4.63906097412109 0.0436664894223213\\
4.65037202835083 0.0427762717008591\\
4.66144800186157 0.0444549322128296\\
4.67229557037354 0.0561453849077225\\
4.68293762207031 0.0544929020106792\\
4.69340085983276 0.0380258746445179\\
4.70371246337891 0.032597117125988\\
4.71390056610107 0.0299225114285946\\
4.72399282455444 0.126350983977318\\
4.73401403427124 0.142800822854042\\
4.75387620925903 0.15225875377655\\
4.77353286743164 0.143942400813103\\
4.79333353042603 0.142059028148651\\
4.80498886108398 0.154430270195007\\
4.81894111633301 0.192262828350067\\
4.83303546905518 0.191309779882431\\
4.84702253341675 0.16914901137352\\
4.86394691467285 0.147393345832825\\
4.8844256401062 0.128200963139534\\
4.89526796340942 0.107559353113174\\
4.9058313369751 0.0552440024912357\\
4.91633558273315 0.0470645129680634\\
4.92694330215454 0.0387377552688122\\
4.93775272369385 0.034064631909132\\
4.94876766204834 0.0301800705492496\\
4.96000909805298 0.0271451026201248\\
4.97146701812744 0.027506347745657\\
4.98318338394165 0.0287582315504551\\
5 0.0796136632561684\\
};
\addlegendentry{$\dtmax{} = \expneg{1.0}{0}{[s]}$};

\addplot [
color=black,
dashed
]
table[row sep=crcr]{
0 0\\
0.0100024435669184 0.000128392173792236\\
0.020002443343401 6.02472937316634e-05\\
0.0300024431198835 5.46693227079231e-05\\
0.0400024428963661 5.83433138672262e-05\\
0.0500024445354939 4.89636149723083e-05\\
0.0600024424493313 3.00926421914482e-05\\
0.070002444088459 1.0196364200965e-05\\
0.0800024420022964 1.04512491816422e-05\\
0.0900024399161339 2.04495208890876e-05\\
0.100002445280552 2.6816551326192e-05\\
0.110002443194389 3.19695718644653e-05\\
0.120002441108227 3.67247812391724e-05\\
0.130002439022064 3.98295633203816e-05\\
0.140002444386482 4.00869066652376e-05\\
0.1500024497509 3.73531402146909e-05\\
0.160002440214157 3.21916704706382e-05\\
0.170002445578575 2.5294642910012e-05\\
0.180002436041832 1.72060754266568e-05\\
0.19000244140625 8.43446559883887e-06\\
0.200002446770668 2.06480785891472e-06\\
0.210002437233925 9.81223001872422e-06\\
0.220002442598343 1.74728429556126e-05\\
0.230002447962761 2.36686973948963e-05\\
0.240002438426018 2.80097283393843e-05\\
0.250002443790436 3.02968910546042e-05\\
0.260002434253693 3.04747518384829e-05\\
0.270002454519272 2.86283484456362e-05\\
0.280002444982529 2.49637705564965e-05\\
0.290002435445786 1.97952595044626e-05\\
0.300002455711365 1.35327709358535e-05\\
0.310002446174622 6.64151184537332e-06\\
0.320002436637878 4.20813591972546e-07\\
0.330002456903458 7.10116000846028e-06\\
0.340002447366714 1.30462221932248e-05\\
0.350002437829971 1.78694281203207e-05\\
0.360002458095551 2.12970717257122e-05\\
0.370002448558807 2.31621288548922e-05\\
0.380002439022064 2.34041999647161e-05\\
0.390002429485321 2.20812125917291e-05\\
0.4000024497509 1.93518917512847e-05\\
0.410002440214157 1.54553326865425e-05\\
0.420002430677414 1.07006471807836e-05\\
0.430002450942993 5.44092608834035e-06\\
0.44000244140625 5.9699630128307e-08\\
0.450002431869507 5.11480675413623e-06\\
0.460002452135086 9.71306872088462e-06\\
0.470002442598343 1.34666324811406e-05\\
0.4800024330616 1.61631123773986e-05\\
0.490002453327179 1.76676367118489e-05\\
0.500002443790436 1.79295184352668e-05\\
0.510002434253693 1.6988611605484e-05\\
0.520002424716949 1.49638317452627e-05\\
0.530002415180206 1.20335844258079e-05\\
0.540002465248108 8.43199904920766e-06\\
0.550002455711365 4.42821192336851e-06\\
0.560002446174622 3.08071435028978e-07\\
0.570002436637878 3.65794176104828e-06\\
0.580002427101135 7.2036887104332e-06\\
0.590002417564392 1.01155746961012e-05\\
0.600002467632294 1.22283126984257e-05\\
0.610002458095551 1.3434225365927e-05\\
0.620002448558807 1.36926028062589e-05\\
0.630002439022064 1.3028075954935e-05\\
0.640002429485321 1.15321317935013e-05\\
0.650002419948578 9.33666797209298e-06\\
0.660002470016479 6.61748208585777e-06\\
0.670002460479736 3.57943213202816e-06\\
0.680002450942993 4.393021413307e-07\\
0.69000244140625 2.59338662544906e-06\\
0.700002431869507 5.3173480409896e-06\\
0.710002422332764 7.56712415750371e-06\\
0.720002472400665 9.21410628507147e-06\\
0.730002462863922 1.01745445135748e-05\\
0.740002453327179 1.04143391581601e-05\\
0.750002443790436 9.95083428279031e-06\\
0.760002434253693 8.84977634996176e-06\\
0.770002424716949 7.21070227882592e-06\\
0.780002415180206 5.16681438966771e-06\\
0.790002465248108 2.87148304778384e-06\\
0.800002455711365 4.90382376483467e-07\\
0.810002446174622 1.81957250333653e-06\\
0.820002436637878 3.90185869036941e-06\\
0.830002427101135 5.63107687412412e-06\\
0.840002417564392 6.90794468027889e-06\\
0.850002467632294 7.66765060689067e-06\\
0.860002458095551 7.88106444815639e-06\\
0.870002448558807 7.56096596887801e-06\\
0.880002439022064 6.75444152875571e-06\\
0.890002429485321 5.53820791537873e-06\\
0.900002419948578 4.00935914512957e-06\\
0.910002470016479 2.28467183660541e-06\\
0.920002460479736 4.88516320729104e-07\\
0.930002450942993 1.26045927117957e-06\\
0.94000244140625 2.84218367596623e-06\\
0.950002431869507 4.16269813285908e-06\\
0.960002422332764 5.14574230692233e-06\\
0.970002472400665 5.74041450818186e-06\\
0.980002462863922 5.92505284657818e-06\\
0.990002453327179 5.70703377889004e-06\\
1.00000238418579 0.000504793773870915\\
1.01000249385834 0.0412166155874729\\
1.02000248432159 0.00418564584106207\\
1.03000247478485 0.00489615881815553\\
1.04000246524811 0.00612836331129074\\
1.05000245571136 0.0131445769220591\\
1.06000244617462 0.0163558293133974\\
1.07000243663788 0.021405877545476\\
1.08000242710114 0.0286258719861507\\
1.09000241756439 0.0405961126089096\\
1.10000240802765 0.0577634796500206\\
1.11000239849091 0.0803521499037743\\
1.12000238895416 0.106092169880867\\
1.13000249862671 0.116244062781334\\
1.14000248908997 0.116275742650032\\
1.15000247955322 0.140848711133003\\
1.16000247001648 0.142537966370583\\
1.17000246047974 0.156588688492775\\
1.18000245094299 0.151083812117577\\
1.19000244140625 0.123580247163773\\
1.20000243186951 0.127819240093231\\
1.21000242233276 0.143384143710136\\
1.22000241279602 0.148436427116394\\
1.23000240325928 0.150626748800278\\
1.24000239372253 0.131008461117744\\
1.25000238418579 0.122889831662178\\
1.26000249385834 0.122686453163624\\
1.27000248432159 0.124654427170753\\
1.28000247478485 0.128655880689621\\
1.29000246524811 0.133452758193016\\
1.30000245571136 0.13911771774292\\
1.31000244617462 0.146012142300606\\
1.32000243663788 0.153844803571701\\
1.33000242710114 0.162643492221832\\
1.34000241756439 0.171056002378464\\
1.35000240802765 0.17987909913063\\
1.36000239849091 0.189277023077011\\
1.37000238895416 0.198039025068283\\
1.38000249862671 0.205282911658287\\
1.39000248908997 0.210593536496162\\
1.40000247955322 0.214290663599968\\
1.41000247001648 0.206510618329048\\
1.42000246047974 0.210677623748779\\
1.43000245094299 0.21570710837841\\
1.44000244140625 0.219304606318474\\
1.45000243186951 0.218568876385689\\
1.46000242233276 0.218530580401421\\
1.47000241279602 0.218722015619278\\
1.48000240325928 0.21878308057785\\
1.49000239372253 0.218736261129379\\
1.50000238418579 0.219331905245781\\
1.51000249385834 0.220657497644424\\
1.52000248432159 0.223699152469635\\
1.53000247478485 0.224303171038628\\
1.54000246524811 0.226756244897842\\
1.55000245571136 0.229307532310486\\
1.56000244617462 0.232032865285873\\
1.57000243663788 0.23065473139286\\
1.58000242710114 0.23198276758194\\
1.59000241756439 0.238483041524887\\
1.60000240802765 0.225073099136353\\
1.61000239849091 0.227017492055893\\
1.62000238895416 0.230554729700089\\
1.63000249862671 0.23638916015625\\
1.64000248908997 0.242252722382545\\
1.65000247955322 0.247121632099152\\
1.66000247001648 0.251515448093414\\
1.67000246047974 0.254184305667877\\
1.68000245094299 0.256021857261658\\
1.69000244140625 0.257623672485352\\
1.70000243186951 0.259016275405884\\
1.71000242233276 0.258914828300476\\
1.72000241279602 0.258710324764252\\
1.73000240325928 0.258977085351944\\
1.74000239372253 0.25866511464119\\
1.75000238418579 0.257850050926209\\
1.76000249385834 0.251418471336365\\
1.77000248432159 0.246716052293777\\
1.78000247478485 0.244912192225456\\
1.79000246524811 0.249000325798988\\
1.80000245571136 0.244198903441429\\
1.81000244617462 0.246361032128334\\
1.82000243663788 0.245321303606033\\
1.83000242710114 0.244051158428192\\
1.84000241756439 0.242277935147285\\
1.85000240802765 0.240205466747284\\
1.86000239849091 0.237715125083923\\
1.87000238895416 0.228389084339142\\
1.88000249862671 0.226893395185471\\
1.89000248908997 0.224135026335716\\
1.90000247955322 0.217808991670609\\
1.91000247001648 0.224073603749275\\
1.92000246047974 0.207143068313599\\
1.93000245094299 0.202157884836197\\
1.94000244140625 0.196426808834076\\
1.95000243186951 0.207292780280113\\
1.96000242233276 0.184458553791046\\
1.97000241279602 0.187390074133873\\
1.98000240325928 0.173893660306931\\
1.99000239372253 0.249106481671333\\
2.00000238418579 0.24591064453125\\
2.01000237464905 0.195691615343094\\
2.0200023651123 0.188503473997116\\
2.03000235557556 0.191040128469467\\
2.04000234603882 0.197851091623306\\
2.05000233650208 0.235224559903145\\
2.06000232696533 0.297277867794037\\
2.07000255584717 0.314071774482727\\
2.08000254631042 0.346367686986923\\
2.09000253677368 0.35330918431282\\
2.10000252723694 0.379978716373444\\
2.1100025177002 0.387315064668655\\
2.12000250816345 0.407187402248383\\
2.13000249862671 0.410179704427719\\
2.14000248908997 0.402796477079391\\
2.15000247955322 0.394715070724487\\
2.16000247001648 0.387815475463867\\
2.17000246047974 0.379257142543793\\
2.18000245094299 0.350320041179657\\
2.19000244140625 0.339370489120483\\
2.20000243186951 0.305555284023285\\
2.21000242233276 0.281397998332977\\
2.22000241279602 0.260028690099716\\
2.23000240325928 0.225479438900948\\
2.24000239372253 0.182948544621468\\
2.25000238418579 0.127957925200462\\
2.26000237464905 0.06406319886446\\
2.2700023651123 0.0227234698832035\\
2.28000235557556 0.00663416180759668\\
2.29000234603882 0.00183809641748667\\
2.30000233650208 0.000507588672917336\\
2.31000232696533 0.000144250050652772\\
2.32000255584717 3.9507845940534e-05\\
2.33000254631042 1.6485681044287e-05\\
2.34000253677368 1.04409164123354e-05\\
2.35000252723694 7.79215224611107e-06\\
2.3600025177002 7.03815885572112e-06\\
2.37000250816345 6.97477980793337e-06\\
2.38000249862671 7.0654632509104e-06\\
2.39000248908997 7.18666706234217e-06\\
2.40000247955322 5.97626376475091e-06\\
2.41000247001648 6.40526786810369e-06\\
2.42000246047974 6.96122060617199e-06\\
2.43000245094299 7.22517279427848e-06\\
2.44000244140625 7.16322529115132e-06\\
2.45000243186951 7.07874050931423e-06\\
2.46000242233276 6.09923927186173e-06\\
2.47000241279602 6.85128134136903e-06\\
2.48000240325928 7.71332906879252e-06\\
2.49000239372253 8.47826049721334e-06\\
2.50000238418579 9.04999069462065e-06\\
2.51000237464905 9.23127663554624e-06\\
2.5200023651123 8.58314888319001e-06\\
2.53000235557556 8.05094987299526e-06\\
2.54000234603882 7.60122338760993e-06\\
2.55000233650208 7.2187726800621e-06\\
2.56000232696533 6.88722184349899e-06\\
2.57000255584717 6.59840088701458e-06\\
2.58000254631042 6.36441973256296e-06\\
2.59000253677368 6.17141631664708e-06\\
2.60000252723694 5.76362936044461e-06\\
2.6100025177002 5.17048874826287e-06\\
2.62000250816345 4.72997407996445e-06\\
2.63000249862671 4.60334103991045e-06\\
2.64000248908997 4.56572979601333e-06\\
2.65000247955322 4.60404498880962e-06\\
2.66000247001648 5.89556748309406e-06\\
2.67000246047974 6.92907451593783e-06\\
2.68000245094299 8.60667842061957e-06\\
2.69000244140625 1.22296150948387e-05\\
2.70000243186951 2.16445841942914e-05\\
2.71000242233276 4.15786744270008e-05\\
2.72000241279602 6.97597570251673e-05\\
2.73000240325928 4.17177652707323e-05\\
2.74000239372253 2.65652379312087e-05\\
2.75000238418579 1.67500456882408e-05\\
2.76000237464905 1.06720926851267e-05\\
2.7700023651123 6.94996469974285e-06\\
2.78000235557556 6.16361148786382e-06\\
2.79000234603882 5.69083204027265e-06\\
2.80000233650208 2.00796748686116e-05\\
2.81000232696533 0.000976551498752087\\
2.82000255584717 0.102232240140438\\
2.83000254631042 0.105300292372704\\
2.84000253677368 0.110922373831272\\
2.85000252723694 0.110724955797195\\
2.8600025177002 0.105321064591408\\
2.87000250816345 0.0667495205998421\\
2.88000249862671 0.0821823477745056\\
2.89000248908997 0.0556310452520847\\
2.90000247955322 0.0183839146047831\\
2.91000247001648 0.00361464847810566\\
2.92000246047974 0.000645863881800324\\
2.93000245094299 0.000117697549285367\\
2.94000244140625 2.86333706753794e-05\\
2.95000243186951 2.42925852944609e-05\\
2.96000242233276 2.27949076361256e-05\\
2.97000241279602 2.16574935620883e-05\\
2.98000240325928 1.9772300220211e-05\\
2.99000239372253 1.70197745319456e-05\\
3.00000238418579 1.61721181939356e-05\\
3.01000237464905 1.5894898751867e-05\\
3.0200023651123 1.60567160492064e-05\\
3.03000235557556 1.67180496646324e-05\\
3.04000234603882 1.83376905624755e-05\\
3.05000233650208 1.96841247088742e-05\\
3.06000232696533 2.10186808544677e-05\\
3.07000255584717 1.4559432202077e-05\\
3.08000254631042 1.46555184983299e-05\\
3.09000253677368 1.56037549459143e-05\\
3.10000252723694 1.78385143954074e-05\\
3.1100025177002 2.30459536396666e-05\\
3.12000250816345 3.77248179574963e-05\\
3.13000249862671 3.96136383642443e-05\\
3.14000248908997 3.94748203689232e-05\\
3.15000247955322 2.33860537264263e-05\\
3.16000247001648 1.72988875419833e-05\\
3.17000246047974 1.41918253575568e-05\\
3.18000245094299 1.24537109513767e-05\\
3.19000244140625 1.16000210255152e-05\\
3.20000243186951 1.14178610601812e-05\\
3.21000242233276 1.1841781088151e-05\\
3.22000241279602 1.29047157315654e-05\\
3.23000240325928 1.48349081428023e-05\\
3.24000239372253 4.6354212827282e-05\\
3.25000238418579 2.88778246613219e-05\\
3.26000237464905 2.51957917498657e-05\\
3.2700023651123 2.25601324927993e-05\\
3.28000235557556 3.27766065311152e-05\\
3.29000234603882 3.03253891615896e-05\\
3.30000233650208 2.84756151813781e-05\\
3.31000232696533 2.47402258537477e-05\\
3.32000255584717 2.20436049858108e-05\\
3.33000254631042 1.98419329535682e-05\\
3.34000253677368 1.80937240656931e-05\\
3.35000252723694 1.61452117026784e-05\\
3.3600025177002 1.53906403284054e-05\\
3.37000250816345 1.84812106454046e-05\\
3.38000249862671 0.00017531419871375\\
3.39000248908997 0.131272837519646\\
3.40000247955322 0.0836066603660584\\
3.41000247001648 0.16694350540638\\
3.42000246047974 0.171210274100304\\
3.43000245094299 0.166044250130653\\
3.44000244140625 0.150296717882156\\
3.45000243186951 0.141604289412498\\
3.46000242233276 0.139817431569099\\
3.47000241279602 0.128538951277733\\
3.48000240325928 0.122270666062832\\
3.49000239372253 0.120977640151978\\
3.50000238418579 0.119255870580673\\
3.51000237464905 0.115941725671291\\
3.5200023651123 0.108752563595772\\
3.53000235557556 0.0562011525034904\\
3.54000234603882 0.108287520706654\\
3.55000233650208 0.117412649095058\\
3.56000232696533 0.121947310864925\\
3.57000255584717 0.124648109078407\\
3.58000254631042 0.126644194126129\\
3.59000253677368 0.127780005335808\\
3.60000252723694 0.12841884791851\\
3.6100025177002 0.126084640622139\\
3.62000250816345 0.153363645076752\\
3.63000249862671 0.184595391154289\\
3.64000248908997 0.187946021556854\\
3.65000247955322 0.188705459237099\\
3.66000247001648 0.188925340771675\\
3.67000246047974 0.188832506537437\\
3.68000245094299 0.188391283154488\\
3.69000244140625 0.18756902217865\\
3.70000243186951 0.185738399624825\\
3.71000242233276 0.179652661085129\\
3.72000241279602 0.158646076917648\\
3.73000240325928 0.145686417818069\\
3.74000239372253 0.136566311120987\\
3.75000238418579 0.142257526516914\\
3.76000237464905 0.141229718923569\\
3.7700023651123 0.174052372574806\\
3.78000235557556 0.18833664059639\\
3.79000234603882 0.188349187374115\\
3.80000233650208 0.177689701318741\\
3.81000232696533 0.165078401565552\\
3.82000255584717 0.181270018219948\\
3.83000254631042 0.182342201471329\\
3.84000253677368 0.181427031755447\\
3.85000252723694 0.178346335887909\\
3.8600025177002 0.17018261551857\\
3.87000250816345 0.144700407981873\\
3.88000249862671 0.155822232365608\\
3.89000248908997 0.168658331036568\\
3.90000247955322 0.172348454594612\\
3.91000247001648 0.17258332669735\\
3.92000246047974 0.196822956204414\\
3.93000245094299 0.205538421869278\\
3.94000244140625 0.213791325688362\\
3.95000243186951 0.217949792742729\\
3.96000242233276 0.201646536588669\\
3.97000241279602 0.183939039707184\\
3.98000240325928 0.176545396447182\\
3.99000239372253 0.193025484681129\\
4.00000238418579 0.190923497080803\\
4.01000261306763 0.187059476971626\\
4.0200023651123 0.168521329760551\\
4.03000259399414 0.140506759285927\\
4.04000234603882 0.132763847708702\\
4.05000257492065 0.121481865644455\\
4.06000232696533 0.120824173092842\\
4.07000255584717 0.124072931706905\\
4.08000230789185 0.084032729268074\\
4.09000253677368 0.0343126095831394\\
4.10000228881836 0.00854924600571394\\
4.1100025177002 0.0018682120135054\\
4.12000226974487 0.000412222900195047\\
4.13000249862671 9.19122321647592e-05\\
4.14000225067139 2.71545650321059e-05\\
4.15000247955322 2.02910705411341e-05\\
4.1600022315979 2.025787216553e-05\\
4.17000246047974 2.15551499422872e-05\\
4.18000221252441 1.92973730008816e-05\\
4.19000244140625 4.52931599284057e-05\\
4.20000267028809 3.95247589040082e-05\\
4.21000242233276 2.90679399768123e-05\\
4.2200026512146 2.16889111470664e-05\\
4.23000240325928 1.77464789885562e-05\\
4.24000263214111 1.60928902914748e-05\\
4.25000238418579 1.66579757205909e-05\\
4.26000261306763 1.91958170034923e-05\\
4.2700023651123 2.32814036280615e-05\\
4.28000259399414 2.95789031952154e-05\\
4.29000234603882 3.29070244333707e-05\\
4.30000257492065 0.000316718593239784\\
4.31000232696533 0.000209983714739792\\
4.32000255584717 0.000190893217222765\\
4.33000230789185 0.000110101995232981\\
4.34000253677368 6.52258822810836e-05\\
4.35000228881836 4.90059210278559e-05\\
4.3600025177002 4.59606635558885e-05\\
4.37000226974487 3.34612414008006e-05\\
4.38000249862671 2.56543007708387e-05\\
4.39000225067139 2.29686229431536e-05\\
4.40000247955322 2.15815489355009e-05\\
4.4100022315979 2.05366013688035e-05\\
4.42000246047974 1.96865548787173e-05\\
4.43000221252441 2.92873810394667e-05\\
4.44000244140625 0.075147844851017\\
4.45000267028809 0.100618571043015\\
4.46000242233276 0.107247039675713\\
4.4700026512146 0.111743435263634\\
4.48000240325928 0.10029074549675\\
4.49000263214111 0.0829147472977638\\
4.50000238418579 0.0520169548690319\\
4.51000261306763 0.0261607803404331\\
4.5200023651123 0.00643136678263545\\
4.53000259399414 0.00105710211209953\\
4.54000234603882 0.000224101197090931\\
4.55000257492065 4.72318824904505e-05\\
4.56000232696533 2.65788876276929e-05\\
4.57000255584717 2.49676559178624e-05\\
4.58000230789185 2.54588576353854e-05\\
4.59000253677368 2.47733041760512e-05\\
4.60000228881836 2.39409910136601e-05\\
4.6100025177002 2.52512181759812e-05\\
4.62000226974487 2.53685357165523e-05\\
4.63000249862671 0.000148716702824458\\
4.64000225067139 4.77782341476996e-05\\
4.65000247955322 3.48722496710252e-05\\
4.6600022315979 3.1570340070175e-05\\
4.67000246047974 0.000184312913916074\\
4.68000221252441 2.15299678529846e-05\\
4.69000244140625 2.21122954826569e-05\\
4.70000267028809 2.00250124180457e-05\\
4.71000242233276 1.86692595889326e-05\\
4.7200026512146 5.00495225423947e-05\\
4.73000240325928 4.40793992311228e-05\\
4.74000263214111 3.08978524117265e-05\\
4.75000238418579 2.36902251344873e-05\\
4.76000261306763 2.08788878808264e-05\\
4.7700023651123 2.1829773686477e-05\\
4.78000259399414 1.9924253138015e-05\\
4.79000234603882 2.05562446353724e-05\\
4.80000257492065 2.30625482799951e-05\\
4.81000232696533 2.72891975328093e-05\\
4.82000255584717 3.50873524439521e-05\\
4.83000230789185 4.09120148106012e-05\\
4.84000253677368 5.9633075579768e-05\\
4.85000228881836 5.40017281309702e-05\\
4.8600025177002 3.50336631527171e-05\\
4.87000226974487 2.6201496439171e-05\\
4.88000249862671 2.27468699449673e-05\\
4.89000225067139 2.14257634070236e-05\\
4.90000247955322 2.15234558709199e-05\\
4.9100022315979 2.22137259697774e-05\\
4.92000246047974 2.27720847760793e-05\\
4.93000221252441 2.3456657800125e-05\\
4.94000244140625 2.41579436988104e-05\\
4.95000267028809 0.00011628740321612\\
4.96000242233276 0.101251661777496\\
4.9700026512146 0.11322782933712\\
4.98000240325928 0.117401570081711\\
4.99000263214111 0.113021850585938\\
5 0.0932144373655319\\
};
\addlegendentry{$\dtmax{} = \expneg{1.0}{5}{[s]}$};

\end{axis}
\end{tikzpicture}%
\caption{Residual of the flashing solution for the linear solver.}
\label{fig:flashingResidualLin}
\end{figure}


\begin{figure}[h!tb]
\centering
% This file was created by matlab2tikz v0.4.3.
% Copyright (c) 2008--2013, Nico Schlömer <nico.schloemer@gmail.com>
% All rights reserved.
% 
\tikzsetnextfilename{plots/flashingResidualNln_pdf}
\begin{tikzpicture}

\begin{axis}[%
width=\mytikzpicwidth,
height=\mytikzpicheight,
scale only axis,
xmin=0.0,
xmax=5.0,
xlabel={Time $[\text{s}]$},
ymin=0.0,
ymax=1.0,
ylabel={$\frac{||\tilde{\vec{F}}^{k})||_2}{\sqrt{N}} [\text{-}]$},
legend style={draw=black,fill=white,legend cell align=left}
]
\addplot [
color=black,
solid
]
table[row sep=crcr]{
0.0 0.0\\
0.0118019115179777 0.00423532025888562\\
0.0226854234933853 1.39742217797334e-08\\
0.0340500771999359 7.6850176355947e-07\\
0.0491764321923256 7.84264457820427e-08\\
0.0619497485458851 5.36052837674106e-08\\
0.0774054601788521 2.90473209929587e-08\\
0.0961068719625473 1.53359653864982e-08\\
0.106882445514202 1.49856163034201e-08\\
0.118735581636429 1.48468313199146e-08\\
0.13177402317524 1.48538710220691e-08\\
0.146116316318512 1.48443453085179e-08\\
0.161892831325531 1.48489140983088e-08\\
0.179247006773949 1.48308654246421e-08\\
0.198336601257324 1.48408183520132e-08\\
0.21933513879776 1.48286058987424e-08\\
0.242433547973633 1.48437724334372e-08\\
0.267841786146164 1.48414427414423e-08\\
0.29579085111618 1.48372825137244e-08\\
0.32653483748436 1.4841080364647e-08\\
0.360353201627731 1.48416816614372e-08\\
0.397553414106369 1.48293679558265e-08\\
0.438473641872406 1.4847274520946e-08\\
0.483485877513886 1.48424907919775e-08\\
0.532999336719513 1.48320893345044e-08\\
0.587464153766632 1.48377896636021e-08\\
0.647375464439392 1.48567425029e-08\\
0.71327793598175 1.48529482046911e-08\\
0.7857705950737 1.48374876829394e-08\\
0.865512549877167 1.48965790813804e-08\\
0.953228712081909 1.4854768082273e-08\\
1.04971647262573 0.120755851268768\\
1.1558530330658 0.000327158428262919\\
1.27260315418243 0.000134138375869952\\
1.32053518295288 0.00012300125672482\\
1.35773289203644 0.000139527575811371\\
1.38946354389191 0.000261212495388463\\
1.42200791835785 0.000297853664960712\\
1.45245742797852 3.77871401724406e-05\\
1.48083031177521 0.000279012456303462\\
1.50743794441223 0.000105044440715574\\
1.53255617618561 0.000328434660332277\\
1.55641222000122 8.50439246278256e-05\\
1.57917487621307 0.000329231144860387\\
1.60098683834076 8.64163012010977e-05\\
1.62195324897766 0.000347074295859784\\
1.64217150211334 0.000101464283943642\\
1.66171324253082 0.000361766142304987\\
1.68064773082733 9.8845484899357e-05\\
1.69902181625366 0.000415356480516493\\
1.71688628196716 0.000108255379018374\\
1.73428571224213 0.000412338529713452\\
1.75125479698181 6.00154598942027e-05\\
1.76759457588196 0.000587772461585701\\
1.78327512741089 3.68122455256525e-05\\
1.79837548732758 0.000279853527899832\\
1.81300187110901 6.46500484435819e-05\\
1.82727634906769 0.000485818192828447\\
1.84123587608337 2.61603854596615e-05\\
1.85493552684784 0.000289555260678753\\
1.86838006973267 1.66789959621383e-05\\
1.88161540031433 0.000609477981925011\\
1.89463996887207 1.83859992830548e-05\\
1.90748333930969 0.000337323697749525\\
1.92014336585999 2.62540652329335e-05\\
1.93264579772949 0.000343883235473186\\
1.94498836994171 0.00179508491419256\\
1.95688509941101 0.000903258100152016\\
1.96837878227234 0.00195984868332744\\
1.97949528694153 0.000883848872035742\\
1.99014973640442 0.000873287906870246\\
2.00136232376099 0.000329253787640482\\
2.01271271705627 0.00262205651961267\\
2.02417707443237 0.0904020592570305\\
2.0357928276062 0.106950797140598\\
2.04750537872314 0.0917682498693466\\
2.05927753448486 0.0505534783005714\\
2.07107305526733 0.0442990176379681\\
2.08286380767822 0.043390266597271\\
2.09455490112305 0.0590679869055748\\
2.10611391067505 0.0737124606966972\\
2.11754536628723 0.0863998308777809\\
2.12883257865906 0.0974802747368813\\
2.13992094993591 0.107626833021641\\
2.15075325965881 0.071136400103569\\
2.16134643554688 0.083956703543663\\
2.1717312335968 0.0951121151447296\\
2.18185663223267 0.10719671100378\\
2.20230388641357 0.080010823905468\\
2.21264386177063 0.0421233661472797\\
2.22270774841309 0.0279478952288628\\
2.23349833488464 0.0109151797369123\\
2.24430298805237 0.00138873257674277\\
2.25509786605835 0.000178839472937398\\
2.26583743095398 2.88400569843361e-05\\
2.27648019790649 9.64054743235465e-06\\
2.29678177833557 1.10712426248938e-06\\
2.30735301971436 7.285808578672e-07\\
2.32744646072388 7.11490883986698e-07\\
2.34562826156616 1.71950929939158e-07\\
2.36075878143311 2.32587311188581e-07\\
2.3733606338501 2.53604554245612e-07\\
2.38860893249512 6.93772861382058e-08\\
2.40705943107605 1.06278719158581e-07\\
2.41769051551819 1.62635544143086e-07\\
2.42919039726257 1.90234558772318e-07\\
2.4407947063446 1.80849198727628e-07\\
2.45252275466919 1.80212424538695e-07\\
2.46437191963196 1.72840245227235e-07\\
2.47634053230286 1.61462921255406e-07\\
2.48842525482178 1.47561777907868e-07\\
2.50060844421387 1.3236547147244e-07\\
2.51288247108459 1.1634308094699e-07\\
2.52523136138916 1.00689092619177e-07\\
2.53763508796692 8.67045031327507e-08\\
2.55007195472717 7.60192477855526e-08\\
2.56252241134644 6.81910705679911e-08\\
2.57496857643127 6.36047658986172e-08\\
2.58739495277405 6.23595610704797e-08\\
2.59979104995728 9.05314010424263e-08\\
2.61202311515808 1.12424388021282e-07\\
2.62403297424316 1.36007798801074e-07\\
2.63583588600159 1.5190701674328e-07\\
2.64725995063782 1.0065950561966e-07\\
2.65829944610596 3.07255135112428e-07\\
2.66903948783875 2.65496311158131e-07\\
2.68084859848022 0.000102336205600295\\
2.69513750076294 0.000153550747199915\\
2.71242713928223 0.000409223808674142\\
2.7327516078949 0.00116157857701182\\
2.74304962158203 0.00192262220662087\\
2.75329804420471 0.00305987219326198\\
2.76352214813232 0.00533653702586889\\
2.77671360969543 0.00600009271875024\\
2.79198384284973 0.00762994959950447\\
2.81012392044067 0.000882942578755319\\
2.82984495162964 0.000183849319000728\\
2.84034872055054 2.99882722174516e-05\\
2.85096287727356 4.15180193158449e-06\\
2.86159515380859 1.18717844088678e-06\\
2.87224507331848 6.74997693295154e-07\\
2.88282418251038 6.01630404162279e-07\\
2.89327502250671 5.22432003435824e-07\\
2.90423226356506 3.59961944695897e-07\\
2.91532301902771 3.46262424955057e-07\\
2.92652130126953 6.45970885670977e-07\\
2.93784737586975 2.61889397279447e-07\\
2.94925856590271 1.19340185733563e-07\\
2.96071529388428 9.96542297571068e-08\\
2.97219562530518 1.33797328771834e-07\\
2.98368406295776 1.92083845718116e-07\\
2.99517321586609 4.18241853594736e-07\\
3.00665736198425 2.69086825710474e-07\\
3.018141746521 3.12129344592904e-07\\
3.02963495254517 2.87170024648731e-07\\
3.04114675521851 2.46596869146742e-07\\
3.05272746086121 2.26874149689138e-07\\
3.06440591812134 2.19612104501721e-07\\
3.07619190216064 1.89460607202818e-07\\
3.08806824684143 1.84035357619905e-07\\
3.10001063346863 1.54731395696217e-07\\
3.11200356483459 1.51225961531054e-07\\
3.12402725219727 1.34885368652249e-07\\
3.13606834411621 1.15370106357204e-07\\
3.14812350273132 1.16493552582142e-07\\
3.16019630432129 9.15383253641266e-08\\
3.17221331596375 8.13702243362968e-08\\
3.18425130844116 7.93048542391261e-08\\
3.1963107585907 9.28240524444846e-08\\
3.20831441879272 9.73455840380666e-08\\
3.22024464607239 1.08857562963749e-07\\
3.2320544719696 1.54444990130287e-07\\
3.24364876747131 1.63800478958365e-07\\
3.25497913360596 4.36782897850208e-07\\
3.26603627204895 6.527927780553e-07\\
3.27683329582214 2.05772252570569e-07\\
3.28738880157471 4.00606410266846e-08\\
3.30090045928955 9.51585680013523e-05\\
3.31724953651428 0.000139090669108555\\
3.3364520072937 0.000385930616175756\\
3.35590267181396 0.00103991490323097\\
3.37547159194946 0.00271596340462565\\
3.39097309112549 0.00319617846980691\\
3.40369939804077 0.000375775067368522\\
3.41909837722778 9.22563995118253e-05\\
3.43773102760315 2.59466596617131e-05\\
3.44846701622009 4.657259069063e-06\\
3.45963358879089 6.56577185509377e-07\\
3.47084331512451 6.30290344361129e-07\\
3.4820818901062 6.97029577167996e-07\\
3.49336910247803 6.3591130583518e-07\\
3.50474190711975 3.52506333456404e-07\\
3.51621460914612 5.83688176902797e-07\\
3.52777361869812 3.56360516207133e-07\\
3.53940916061401 1.22609463915069e-07\\
3.55113887786865 1.27757019185992e-07\\
3.5629894733429 1.10117362339679e-07\\
3.57491517066956 1.94242630868757e-07\\
3.58686423301697 3.1798256827642e-07\\
3.59883189201355 4.15099975725752e-07\\
3.61085295677185 4.24533510567926e-07\\
3.62294244766235 3.55846253796699e-07\\
3.63507795333862 2.39646851696307e-07\\
3.64724469184875 1.44834075399558e-07\\
3.65939402580261 1.54106260197295e-07\\
3.67149782180786 1.47833219443783e-07\\
3.68352150917053 1.25750261759094e-07\\
3.69545078277588 1.05400289385216e-07\\
3.70728397369385 1.03503595028087e-07\\
3.71901512145996 9.1644217548037e-08\\
3.73064255714417 1.19783948093755e-07\\
3.74213266372681 1.99509329945613e-07\\
3.75340151786804 2.08503479370847e-07\\
3.7644145488739 2.80759280713028e-07\\
3.77517652511597 4.36538812209619e-07\\
3.78571152687073 4.47773857104039e-07\\
3.79605078697205 2.95041672870866e-07\\
3.80622386932373 5.45732596890502e-08\\
3.82096457481384 9.301371028414e-05\\
3.83880090713501 0.000141277894726954\\
3.85830807685852 0.000399702257709578\\
3.87783885002136 0.00100108853075653\\
3.89773917198181 0.00193407642655075\\
3.90787315368652 0.00122464133892208\\
3.91816449165344 0.000187894096598029\\
3.93745827674866 7.11203665559879e-06\\
3.94821214675903 1.14122576633235e-06\\
3.95906329154968 6.15514238688775e-07\\
3.96999645233154 8.44541716560343e-07\\
3.98096656799316 8.2123051470262e-07\\
3.991947889328 7.63016828386753e-07\\
4.00295209884644 7.85911481671064e-07\\
4.01402473449707 6.54191808280302e-07\\
4.02517938613892 2.82261396478134e-07\\
4.03638935089111 1.87671403750755e-07\\
4.04771900177002 1.41080263915683e-07\\
4.05928993225098 1.29252541114511e-07\\
4.07111120223999 1.16106697589657e-07\\
4.08298397064209 1.01230043014766e-07\\
4.09486865997314 8.81382717921042e-08\\
4.10676383972168 9.16731792699466e-08\\
4.11869525909424 2.68866671149226e-07\\
4.13065099716187 9.68847970739262e-08\\
4.14261865615845 9.95408910853257e-08\\
4.15452194213867 8.79647714668863e-08\\
4.16634511947632 7.83432554385399e-08\\
4.17807626724243 7.08684808614635e-08\\
4.18974590301514 6.43019433255176e-08\\
4.20136594772339 7.21090032129723e-08\\
4.21294355392456 1.26304428249568e-07\\
4.22449111938477 2.10342534501251e-07\\
4.23601484298706 2.6990085189027e-07\\
4.24750518798828 2.93443406462757e-07\\
4.25887632369995 3.26424128616054e-07\\
4.27007293701172 5.60930686788197e-07\\
4.28108787536621 7.98999622020347e-07\\
4.29192972183228 4.5005961624156e-08\\
4.30261182785034 4.73166863912411e-07\\
4.31315040588379 4.50184444389379e-07\\
4.32356023788452 4.87740805965586e-07\\
4.33452129364014 8.5919164121151e-05\\
4.34778356552124 0.000228160628466867\\
4.36383104324341 0.000519908440764993\\
4.38311052322388 0.00061847158940509\\
4.3932032585144 0.0009708491852507\\
4.4033842086792 0.00123000494204462\\
4.41344928741455 0.00100403709802777\\
4.42365694046021 0.00049413088709116\\
4.43415594100952 5.21238544024527e-05\\
4.44478559494019 9.26482971408404e-06\\
4.4555401802063 1.27164514651668e-06\\
4.46635627746582 6.39896825305186e-07\\
4.47721815109253 8.33762726415443e-07\\
4.48809051513672 7.267276487255e-07\\
4.4989447593689 2.73876850087618e-07\\
4.50979852676392 5.59896989216213e-07\\
4.52070713043213 5.12236169925018e-07\\
4.53168821334839 2.26270202574597e-07\\
4.54263353347778 1.64162500482234e-07\\
4.55374670028687 1.07460671472381e-07\\
4.56528568267822 2.22616904466122e-07\\
4.57695484161377 1.33110447109175e-07\\
4.58858823776245 9.92286359746686e-08\\
4.6002082824707 1.3169434964766e-07\\
4.61189317703247 6.35187502595613e-09\\
4.62363004684448 1.35162693482016e-07\\
4.63552665710449 1.10618351811809e-07\\
4.64763736724854 8.98544811889224e-08\\
4.65980005264282 8.9114188028816e-08\\
4.67189693450928 8.13502367691399e-08\\
4.68388748168945 7.70996209098485e-08\\
4.69575357437134 7.74943700321273e-08\\
4.70748949050903 7.25950286550869e-08\\
4.71910238265991 9.70205888961573e-08\\
4.73061180114746 1.90604069416622e-07\\
4.74203300476074 3.20196306802245e-07\\
4.75329446792603 4.8184665502049e-07\\
4.7643346786499 5.16721740950743e-07\\
4.77515077590942 7.701908089075e-07\\
4.78576755523682 4.42407568357339e-08\\
4.79621171951294 6.02527805426689e-08\\
4.8065128326416 9.17380418741232e-07\\
4.81669855117798 6.80408675179933e-07\\
4.82679224014282 8.208033932533e-07\\
4.8414249420166 6.71745874569751e-05\\
4.859130859375 0.000209648394957185\\
4.87904977798462 0.000570966978557408\\
4.88912153244019 0.0009465865441598\\
4.89928483963013 0.00150460365694016\\
4.90958023071289 0.00203741062432528\\
4.92001533508301 0.00142546917777509\\
4.93059682846069 0.000365513697033748\\
4.94941663742065 7.95927826402476e-06\\
4.9602427482605 1.31672209136013e-06\\
4.97117471694946 6.90754973220464e-07\\
4.98216915130615 9.29024224660679e-07\\
5 3.41656090085962e-07\\
};
\addlegendentry{$\dtmax{} = \expneg{1.0}{0}{[s]}$};

\addplot [
color=black,
dashed
]
table[row sep=crcr]{
0 0\\
0.0100024435669184 1.51565622275029e-08\\
0.020002443343401 1.57098263287025e-08\\
0.0300024431198835 1.58614277268043e-08\\
0.0400024428963661 1.59312989467253e-08\\
0.0500024445354939 1.59712154612635e-08\\
0.0600024424493313 1.60116329084303e-08\\
0.070002444088459 1.6018427473341e-08\\
0.0800024420022964 1.59707145286347e-08\\
0.0900024399161339 1.59282915745962e-08\\
0.100002445280552 1.58701851660226e-08\\
0.110002443194389 1.58436357367009e-08\\
0.120002441108227 1.58099080493912e-08\\
0.130002439022064 1.58019748397464e-08\\
0.140002444386482 1.58102775316138e-08\\
0.1500024497509 1.58154573881575e-08\\
0.160002440214157 1.58368909097817e-08\\
0.170002445578575 1.58681796591509e-08\\
0.180002436041832 1.59048489933866e-08\\
0.19000244140625 1.59543684929986e-08\\
0.200002446770668 1.60019766326513e-08\\
0.210002437233925 1.60043800434551e-08\\
0.220002442598343 1.60062434417796e-08\\
0.230002447962761 1.59940487520771e-08\\
0.240002438426018 1.60022075590405e-08\\
0.250002443790436 1.59986210945817e-08\\
0.260002434253693 1.59930095833261e-08\\
0.270002454519272 1.59981432545919e-08\\
0.280002444982529 1.59939084198868e-08\\
0.290002435445786 1.6001383329467e-08\\
0.300002455711365 1.60002961990813e-08\\
0.310002446174622 1.60017599171169e-08\\
0.320002436637878 4.25368767764667e-07\\
0.330002456903458 1.59599977678226e-08\\
0.340002447366714 1.59418931389155e-08\\
0.350002437829971 1.59043267444758e-08\\
0.360002458095551 1.58883999290538e-08\\
0.370002448558807 1.58776565228891e-08\\
0.380002439022064 1.58722830434499e-08\\
0.390002429485321 1.58857975662841e-08\\
0.4000024497509 1.5893396820843e-08\\
0.410002440214157 1.59245132635988e-08\\
0.420002430677414 1.59358002349563e-08\\
0.430002450942993 1.59655080267385e-08\\
0.44000244140625 5.44633493859692e-08\\
0.450002431869507 1.60065365406581e-08\\
0.460002452135086 1.59967825652529e-08\\
0.470002442598343 1.59919402165087e-08\\
0.4800024330616 1.6000464952981e-08\\
0.490002453327179 1.59943720490219e-08\\
0.500002443790436 1.60005058091883e-08\\
0.510002434253693 1.59943454036693e-08\\
0.520002424716949 1.59951305533923e-08\\
0.530002415180206 1.59993067683217e-08\\
0.540002465248108 1.59926241138919e-08\\
0.550002455711365 1.59985926728723e-08\\
0.560002446174622 2.99205709097805e-07\\
0.570002436637878 1.59792694631733e-08\\
0.580002427101135 1.5961454380431e-08\\
0.590002417564392 1.59425184165229e-08\\
0.600002467632294 1.59311728253897e-08\\
0.610002458095551 1.59249289310992e-08\\
0.620002448558807 1.59220689965878e-08\\
0.630002439022064 1.59295137081017e-08\\
0.640002429485321 1.59338764404993e-08\\
0.650002419948578 1.59456785553402e-08\\
0.660002470016479 1.59626445395133e-08\\
0.670002460479736 1.59736117666398e-08\\
0.680002450942993 4.30711622811941e-07\\
0.69000244140625 1.60026765172461e-08\\
0.700002431869507 1.59961963674959e-08\\
0.710002422332764 1.59919562037203e-08\\
0.720002472400665 1.59953312817152e-08\\
0.730002462863922 1.59993049919649e-08\\
0.740002453327179 1.59915121145104e-08\\
0.750002443790436 1.60019606454398e-08\\
0.760002434253693 1.59942228350474e-08\\
0.770002424716949 1.60037085805698e-08\\
0.780002415180206 1.59957682654976e-08\\
0.790002465248108 1.5992545954191e-08\\
0.800002455711365 4.81415725062107e-07\\
0.810002446174622 1.59852522330084e-08\\
0.820002436637878 1.59747717276559e-08\\
0.830002427101135 1.59657087550613e-08\\
0.840002417564392 1.59562514312483e-08\\
0.850002467632294 1.59620086037648e-08\\
0.860002458095551 1.59531392540657e-08\\
0.870002448558807 1.59630868523664e-08\\
0.880002439022064 1.59671813548812e-08\\
0.890002429485321 1.59667230548166e-08\\
0.900002419948578 1.59734057092464e-08\\
0.910002470016479 1.59827031609439e-08\\
0.920002460479736 4.79609695958061e-07\\
0.930002450942993 1.60047335384661e-08\\
0.94000244140625 1.59944910649301e-08\\
0.950002431869507 1.59999942184186e-08\\
0.960002422332764 1.60074158372936e-08\\
0.970002472400665 1.59905422236761e-08\\
0.980002462863922 1.59915511943609e-08\\
0.990002453327179 1.59945425792785e-08\\
1.00000238418579 0.00050815922440961\\
1.01000249385834 0.0410696566104889\\
1.02000248432159 0.00340246222913265\\
1.03000247478485 0.00489192456007004\\
1.04000246524811 0.00612180121243\\
1.05000245571136 0.00704742316156626\\
1.06000244617462 0.007607935462147\\
1.07000243663788 0.00774358352646232\\
1.08000242710114 0.00739566748961806\\
1.09000241756439 0.00650678901001811\\
1.10000240802765 0.00502084894105792\\
1.11000239849091 0.00269550830125809\\
1.12000238895416 0.000194031643331982\\
1.13000249862671 0.000215394888073206\\
1.14000248908997 0.000101640893262811\\
1.15000247955322 0.000124544589198194\\
1.16000247001648 0.000132040309836157\\
1.17000246047974 4.77079674965353e-06\\
1.18000245094299 4.38867346019833e-06\\
1.19000244140625 1.62485241617105e-06\\
1.20000243186951 9.3362530151353e-07\\
1.21000242233276 5.7475926951156e-07\\
1.22000241279602 6.16891099980421e-07\\
1.23000240325928 6.2723700011702e-07\\
1.24000239372253 9.73693431660649e-07\\
1.25000238418579 8.89684145022329e-07\\
1.26000249385834 0.000115404342068359\\
1.27000248432159 0.000121052500617225\\
1.28000247478485 0.000128965373733081\\
1.29000246524811 0.000138778501423076\\
1.30000245571136 0.00015062969760038\\
1.31000244617462 0.000164592958753929\\
1.32000243663788 0.000180957722477615\\
1.33000242710114 0.000200085123651661\\
1.34000241756439 0.000222275237319991\\
1.35000240802765 0.000123862104373984\\
1.36000239849091 0.000137950439238921\\
1.37000238895416 0.000152969849295914\\
1.38000249862671 0.00016811492969282\\
1.39000248908997 0.000180734277819283\\
1.40000247955322 0.0001915590983117\\
1.41000247001648 0.000376973301172256\\
1.42000246047974 0.00019437876471784\\
1.43000245094299 0.000212670609471388\\
1.44000244140625 0.000217790438910015\\
1.45000243186951 0.000222911694436334\\
1.46000242233276 0.000228499775403179\\
1.47000241279602 0.000234875857131556\\
1.48000240325928 0.000242312205955386\\
1.49000239372253 0.000250987475737929\\
1.50000238418579 0.000261018663877621\\
1.51000249385834 0.000272479868726805\\
1.52000248432159 0.000285382353467867\\
1.53000247478485 0.000299674429697916\\
1.54000246524811 0.000315214070724323\\
1.55000245571136 0.000166821788297966\\
1.56000244617462 0.000353725627064705\\
1.57000243663788 9.40254612942226e-05\\
1.58000242710114 0.000199551563127898\\
1.59000241756439 0.000211599399335682\\
1.60000240802765 0.000223942581214942\\
1.61000239849091 0.000462400174001232\\
1.62000238895416 0.000475215725600719\\
1.63000249862671 0.000487676617922261\\
1.64000248908997 0.000250217592110857\\
1.65000247955322 0.000511890801135451\\
1.66000247001648 0.000262136338278651\\
1.67000246047974 0.000267976807663217\\
1.68000245094299 0.000273795594694093\\
1.69000244140625 0.000279646919807419\\
1.70000243186951 0.000285577116301283\\
1.71000242233276 0.000291612057480961\\
1.72000241279602 0.000297751015750691\\
1.73000240325928 0.000303955923300236\\
1.74000239372253 0.000310170813463628\\
1.75000238418579 0.000316315941745415\\
1.76000249385834 0.000286395370494574\\
1.77000248432159 0.000260352098848671\\
1.78000247478485 0.000265632959781215\\
1.79000246524811 0.000279887404758483\\
1.80000245571136 0.000296684709610417\\
1.81000244617462 0.000313594995532185\\
1.82000243663788 0.000329527334542945\\
1.83000242710114 0.000343799445545301\\
1.84000241756439 0.00035581158590503\\
1.85000240802765 0.000365004991181195\\
1.86000239849091 0.000370881811250001\\
1.87000238895416 0.000373154121916741\\
1.88000249862671 0.000371553935110569\\
1.89000248908997 0.000366066466085613\\
1.90000247955322 0.000356807373464108\\
1.91000247001648 0.000343940686434507\\
1.92000246047974 0.000327863090205938\\
1.93000245094299 0.000617111625615507\\
1.94000244140625 0.000567893031984568\\
1.95000243186951 0.000123761492432095\\
1.96000242233276 0.00238826056011021\\
1.97000241279602 0.00132463057525456\\
1.98000240325928 0.000328758091200143\\
1.99000239372253 0.000438790593761951\\
2.00000238418579 0.000237832762650214\\
2.01000237464905 0.000321186613291502\\
2.0200023651123 0.0115328757092357\\
2.03000235557556 0.0147291496396065\\
2.04000234603882 0.0145687544718385\\
2.05000233650208 0.0144734755158424\\
2.06000232696533 0.00369767704978585\\
2.07000255584717 0.00663779908791184\\
2.08000254631042 0.00680770538747311\\
2.09000253677368 0.00634780572727323\\
2.10000252723694 0.00626655342057347\\
2.1100025177002 0.00581314880400896\\
2.12000250816345 0.00572197558358312\\
2.13000249862671 0.00582711771130562\\
2.14000248908997 0.0133001869544387\\
2.15000247955322 0.0121876485645771\\
2.16000247001648 0.0119377858936787\\
2.17000246047974 0.0100041944533587\\
2.18000245094299 0.00765474606305361\\
2.19000244140625 0.00990696530789137\\
2.20000243186951 0.00313609861768782\\
2.21000242233276 0.000556535262148827\\
2.22000241279602 0.000197924731764942\\
2.23000240325928 5.28423661307897e-05\\
2.24000239372253 1.22738310892601e-05\\
2.25000238418579 3.10001041725627e-06\\
2.26000237464905 4.90989179979806e-07\\
2.2700023651123 5.54073977809821e-08\\
2.28000235557556 5.18347391675889e-08\\
2.29000234603882 4.96793575166521e-07\\
2.30000233650208 4.92396274864859e-08\\
2.31000232696533 1.21383330053959e-08\\
2.32000255584717 1.04265698297468e-08\\
2.33000254631042 6.79179912310701e-09\\
2.34000253677368 2.92968937998239e-09\\
2.35000252723694 1.26774302167121e-09\\
2.3600025177002 7.63413054993833e-10\\
2.37000250816345 6.43422981561059e-10\\
2.38000249862671 6.56667442644476e-10\\
2.39000248908997 8.24097401430635e-10\\
2.40000247955322 8.45049141773302e-10\\
2.41000247001648 9.3127983102903e-10\\
2.42000246047974 9.56266954510454e-10\\
2.43000245094299 9.36787203364986e-10\\
2.44000244140625 8.92996232515486e-10\\
2.45000243186951 8.38798419611209e-10\\
2.46000242233276 7.84404985409992e-10\\
2.47000241279602 7.3653055876477e-10\\
2.48000240325928 6.81189937790094e-10\\
2.49000239372253 6.5862565401531e-10\\
2.50000238418579 6.33723240550665e-10\\
2.51000237464905 6.12312811565374e-10\\
2.5200023651123 5.88992632444274e-10\\
2.53000235557556 5.72379421637237e-10\\
2.54000234603882 5.6071652876355e-10\\
2.55000233650208 5.50650691710786e-10\\
2.56000232696533 5.41171663037687e-10\\
2.57000255584717 5.32646593498498e-10\\
2.58000254631042 5.56498347403789e-10\\
2.59000253677368 1.00419461634971e-09\\
2.60000252723694 1.07052811060271e-09\\
2.6100025177002 9.80586056797961e-10\\
2.62000250816345 1.05627862012625e-09\\
2.63000249862671 2.16060969115972e-09\\
2.64000248908997 2.99763613931248e-09\\
2.65000247955322 3.21194537633573e-09\\
2.66000247001648 2.86043699837535e-09\\
2.67000246047974 1.83579507151421e-09\\
2.68000245094299 1.46402934220191e-09\\
2.69000244140625 1.33428879056652e-09\\
2.70000243186951 4.28470237068268e-09\\
2.71000242233276 1.93465396591819e-07\\
2.72000241279602 0.00028876747819595\\
2.73000240325928 0.000437576411059126\\
2.74000239372253 0.000747591140680015\\
2.75000238418579 0.00108329101931304\\
2.76000237464905 0.00127332017291337\\
2.7700023651123 0.00125952332746238\\
2.78000235557556 0.000162581898621283\\
2.79000234603882 4.5971100917086e-05\\
2.80000233650208 4.13863926951308e-06\\
2.81000232696533 1.03685295016476e-06\\
2.82000255584717 7.9066251146287e-07\\
2.83000254631042 1.79610168515865e-07\\
2.84000253677368 1.92753446981442e-07\\
2.85000252723694 1.18886006816865e-07\\
2.8600025177002 2.59348791331604e-08\\
2.87000250816345 1.04285504676227e-08\\
2.88000249862671 8.13271583410824e-07\\
2.89000248908997 1.85811160235971e-07\\
2.90000247955322 2.2252875098161e-09\\
2.91000247001648 8.15037093371274e-10\\
2.92000246047974 7.8634487810092e-10\\
2.93000245094299 7.64587282375828e-10\\
2.94000244140625 7.53088646998634e-10\\
2.95000243186951 7.65989549567081e-10\\
2.96000242233276 8.02076738359858e-10\\
2.97000241279602 8.56015702765944e-10\\
2.98000240325928 9.31206500798254e-10\\
2.99000239372253 1.07767583745755e-09\\
3.00000238418579 1.05261943605939e-09\\
3.01000237464905 9.27369403491696e-10\\
3.0200023651123 8.47869829900816e-10\\
3.03000235557556 7.54440010464208e-10\\
3.04000234603882 6.93189339262545e-10\\
3.05000233650208 6.57171095319598e-10\\
3.06000232696533 6.39029218429954e-10\\
3.07000255584717 6.26207585785465e-10\\
3.08000254631042 6.1459481948134e-10\\
3.09000253677368 6.08061212492572e-10\\
3.10000252723694 5.69968738872717e-10\\
3.1100025177002 5.12928977070004e-10\\
3.12000250816345 4.33442864933298e-10\\
3.13000249862671 4.02870570237468e-10\\
3.14000248908997 2.15494533328808e-09\\
3.15000247955322 2.74004308131737e-09\\
3.16000247001648 5.54920054351982e-10\\
3.17000246047974 6.98609614602219e-10\\
3.18000245094299 9.41444255886381e-10\\
3.19000244140625 1.77297920789243e-09\\
3.20000243186951 1.96706961830273e-09\\
3.21000242233276 2.36217334581568e-09\\
3.22000241279602 2.96843771785404e-09\\
3.23000240325928 3.68008024054234e-09\\
3.24000239372253 4.37998659563732e-09\\
3.25000238418579 5.00334040864914e-09\\
3.26000237464905 5.42536682246464e-09\\
3.2700023651123 5.44463185647714e-09\\
3.28000235557556 5.06021180513017e-09\\
3.29000234603882 5.56564927478576e-09\\
3.30000233650208 1.38481951239555e-07\\
3.31000232696533 9.51956753851846e-05\\
3.32000255584717 0.000340977712767199\\
3.33000254631042 0.000648080080281943\\
3.34000253677368 0.00102985114790499\\
3.35000252723694 0.000872277596499771\\
3.3600025177002 0.000349052279489115\\
3.37000250816345 6.5081927459687e-05\\
3.38000249862671 7.08671723259613e-06\\
3.39000248908997 2.65180028691248e-06\\
3.40000247955322 5.84901101774449e-07\\
3.41000247001648 7.89429236647266e-07\\
3.42000246047974 1.37273987732556e-08\\
3.43000245094299 0.00177636044099927\\
3.44000244140625 0.00771449180319905\\
3.45000243186951 0.00791492406278849\\
3.46000242233276 0.0105178644880652\\
3.47000241279602 0.00732903461903334\\
3.48000240325928 0.00919430516660213\\
3.49000239372253 0.00401872722432017\\
3.50000238418579 0.00047216410166584\\
3.51000237464905 0.00314942444674671\\
3.5200023651123 0.00489901099354029\\
3.53000235557556 0.00816644355654716\\
3.54000234603882 0.0105635989457369\\
3.55000233650208 0.0123247113078833\\
3.56000232696533 0.0136939380317926\\
3.57000255584717 0.00739720836281776\\
3.58000254631042 0.00786276534199715\\
3.59000253677368 0.00827065296471119\\
3.60000252723694 0.00866758357733488\\
3.6100025177002 0.00903809536248446\\
3.62000250816345 0.00934482831507921\\
3.63000249862671 0.00952955149114132\\
3.64000248908997 0.00951993931084871\\
3.65000247955322 0.00920657347887754\\
3.66000247001648 0.00850498396903276\\
3.67000246047974 0.0147686209529638\\
3.68000245094299 0.011791112832725\\
3.69000244140625 0.00839055981487036\\
3.70000243186951 0.0100014042109251\\
3.71000242233276 0.00828787870705128\\
3.72000241279602 0.00922734010964632\\
3.73000240325928 0.013150074519217\\
3.74000239372253 0.0119997886940837\\
3.75000238418579 0.00927824061363935\\
3.76000237464905 0.0132236806675792\\
3.7700023651123 0.0085790166631341\\
3.78000235557556 0.00939048826694489\\
3.79000234603882 0.0164122898131609\\
3.80000233650208 0.0114266416057944\\
3.81000232696533 0.011317933909595\\
3.82000255584717 0.0122111337259412\\
3.83000254631042 0.00391522003337741\\
3.84000253677368 0.00198435038328171\\
3.85000252723694 0.00246640644036233\\
3.8600025177002 0.00826789624989033\\
3.87000250816345 0.0112413689494133\\
3.88000249862671 0.0121408943086863\\
3.89000248908997 0.00785490032285452\\
3.90000247955322 0.012080947868526\\
3.91000247001648 0.0141061497852206\\
3.92000246047974 0.00403726706281304\\
3.93000245094299 0.000778461690060794\\
3.94000244140625 0.000279245432466269\\
3.95000243186951 8.12224461697042e-05\\
3.96000242233276 2.39328510360792e-05\\
3.97000241279602 6.8143053795211e-06\\
3.98000240325928 8.81652113093878e-07\\
3.99000239372253 7.3668232403179e-08\\
4.00000238418579 5.04048962568504e-08\\
4.01000261306763 3.32778569145376e-07\\
4.0200023651123 1.48042040848395e-08\\
4.03000259399414 1.71081726563216e-09\\
4.04000234603882 1.4874323994718e-09\\
4.05000257492065 1.26073040895847e-09\\
4.06000232696533 1.17106013775015e-09\\
4.07000255584717 1.09207476395312e-09\\
4.08000230789185 1.0391845162161e-09\\
4.09000253677368 9.93183535413777e-10\\
4.10000228881836 9.73422675798474e-10\\
4.1100025177002 9.48495393338078e-10\\
4.12000226974487 9.95718729690509e-10\\
4.13000249862671 1.03349417912568e-09\\
4.14000225067139 1.03772535009483e-09\\
4.15000247955322 1.01641772776162e-09\\
4.1600022315979 9.69711200227152e-10\\
4.17000246047974 9.48259470945345e-10\\
4.18000221252441 1.22241716749016e-09\\
4.19000244140625 6.42415670881746e-07\\
4.20000267028809 8.59413660236896e-08\\
4.21000242233276 4.8840686162066e-08\\
4.2200026512146 1.49587933151452e-08\\
4.23000240325928 5.31473220988232e-09\\
4.24000263214111 2.94461943717295e-09\\
4.25000238418579 2.97643043545293e-09\\
4.26000261306763 3.83231624212499e-09\\
4.2700023651123 4.62803129153144e-09\\
4.28000259399414 4.69443905970479e-09\\
4.29000234603882 4.83792783612103e-09\\
4.30000257492065 4.8796211515878e-09\\
4.31000232696533 4.78125095071391e-09\\
4.32000255584717 4.48192816193682e-09\\
4.33000230789185 3.98313249050375e-09\\
4.34000253677368 2.55354848377465e-08\\
4.35000228881836 2.20780348172411e-05\\
4.3600025177002 0.000326247303746641\\
4.37000226974487 0.000313760829158127\\
4.38000249862671 0.000494900916237384\\
4.39000225067139 0.000307004898786545\\
4.40000247955322 0.00011409907892812\\
4.4100022315979 3.63277467840817e-05\\
4.42000246047974 7.66870743973413e-06\\
4.43000221252441 1.05536025785113e-06\\
4.44000244140625 8.63243315052387e-07\\
4.45000267028809 1.95155578808226e-07\\
4.46000242233276 3.70335015986711e-07\\
4.4700026512146 9.05821817553942e-09\\
4.48000240325928 1.17076703887165e-08\\
4.49000263214111 8.47623216060356e-09\\
4.50000238418579 7.42416172982985e-09\\
4.51000261306763 6.60292087673042e-09\\
4.5200023651123 5.34613153746477e-09\\
4.53000259399414 3.28165339347208e-09\\
4.54000234603882 3.7580618617028e-09\\
4.55000257492065 1.96646166017445e-09\\
4.56000232696533 5.23481880065901e-09\\
4.57000255584717 2.77987677321789e-09\\
4.58000230789185 2.6684225939988e-09\\
4.59000253677368 9.42664470926502e-08\\
4.60000228881836 3.03288089753551e-07\\
4.6100025177002 1.50988803682139e-08\\
4.62000226974487 1.98300997844569e-09\\
4.63000249862671 2.40131514672726e-09\\
4.64000225067139 2.23262341947361e-09\\
4.65000247955322 2.12899231577524e-09\\
4.6600022315979 2.04761607669468e-09\\
4.67000246047974 2.06846739736477e-09\\
4.68000221252441 2.2076276362526e-09\\
4.69000244140625 2.33722685649695e-09\\
4.70000267028809 2.53918064352376e-09\\
4.71000242233276 2.96784907760639e-09\\
4.7200026512146 3.73024988675752e-09\\
4.73000240325928 4.73591166283427e-09\\
4.74000263214111 7.31180049839963e-09\\
4.75000238418579 9.37319644123136e-09\\
4.76000261306763 1.59549085054778e-08\\
4.7700023651123 1.53271368930064e-08\\
4.78000259399414 6.78502543038917e-09\\
4.79000234603882 8.57898285744341e-09\\
4.80000257492065 9.05091734892949e-09\\
4.81000232696533 7.72064900900205e-09\\
4.82000255584717 6.70965771831789e-09\\
4.83000230789185 6.06915273593245e-09\\
4.84000253677368 5.22651033563193e-09\\
4.85000228881836 4.91755169917951e-09\\
4.8600025177002 2.3834807905132e-08\\
4.87000226974487 0.000153458138811402\\
4.88000249862671 0.000402011501137167\\
4.89000225067139 0.000713031855411828\\
4.90000247955322 0.000680526136420667\\
4.9100022315979 0.00029282525065355\\
4.92000246047974 8.97381833055988e-05\\
4.93000221252441 1.38201648951508e-05\\
4.94000244140625 2.35020320360491e-06\\
4.95000267028809 5.17105490871472e-07\\
4.96000242233276 7.13850511147029e-07\\
4.9700026512146 1.21156595866978e-08\\
4.98000240325928 2.29836309983966e-08\\
4.99000263214111 1.2989989528478e-08\\
5 5.8536313574109e-09\\
};
\addlegendentry{$\dtmax{} = \expneg{1.0}{5}{[s]}$};

\end{axis}
\end{tikzpicture}%
\caption{Residual of the flashing solution for the nonlinear solver.}
\label{fig:flashingResidualNln}
\end{figure}

The reduction in the residual produced by the linear solver, \fig{fig:flashingResidualLin}, shows that the reduction of the maximum allowable timestep size now serves two purposes.
The first is that as \dtmax{} is reduced the nonlinear physics are being better resolved; however, even for small \dtmax{}, the residuals are still large compared to those of the nonlinear solution, \fig{fig:flashingResidualNln}.
The second purpose in reducing \dtmax{} is to decrease the error due to the discrete approximation of the temporal integral of the governing equations.
While the reduction of the maximum timestep size in the linear solver serves the purpose of reducing the residual, the nonlinear solver performs this task naturally.
Therefore, in the nonlinear solver the reduction of the maximum timestep size primarily serves to reduce the error from the approximate discrete temporal integral.

One way to quantify the difference between the solutions is by measuring the residual convergence metrics outlined in \sect{sect:temporal_convergence}.
These metrics provide a measure of how poorly the discrete nonlinear equations are being solved at every timestep in the transient.
The resolution of the residual allows for a solution that is less sensitive to timestep size selection than for a solution that does not resolve the residual.
The convergence metrics will be used to determine if a qualitative temporal-convergence determination can lead to the acceptance of a solution that is not nonlinearly converged.

For each of the test cases, the two different metrics were compared at the different \dtmax{}.
These convergence metrics were evaluated during post-processing.
To examine efficacy of the the temporal convergence criteria, both the average and moment-based temporal convergence criteria were evaluated for each of the 24 simulations.
The values of these metrics will be compared between different solvers for the same test problem. 
The two different nonlinear convergence metrics will now be examined for the flashing problem.
\tab{tab:flashingMetric} shows both the average metric, $\tilde{R}$, and the moment-based metric, $\tilde{R}_{\text{M}}$.
\tab{tab:singleMetric} shows the two metrics for the single-phase problem.

\begin{table}[h!tb]
\centering
\singlespace
\pgfplotstabletypeset[sci zerofill,sci E, col sep=comma,
	columns/0/.style={ column name=\dtmax{}, precision=1},
	columns/1/.style={ column name=Linear, precision=3},
	columns/2/.style={ column name=Nonlinear, precision=3},
	columns/3/.style={ column name=Linear, precision=3},
	columns/4/.style={ column name=Nonlinear, precision=3},
	every head row/.style={
		before row={
			\toprule
			&\multicolumn{2}{c}{$\tilde{R}$} & \multicolumn{2}{c}{$\tilde{R}_{M}$}\\
		},
		after row=\midrule
	},
	every last row/.style={
after row=\bottomrule}]{tables/flashingMetricData.tex}
\caption{Nonlinear convergence metrics for flashing problem.}
\label{tab:flashingMetric}
\end{table}

\begin{table}[h!tb]
\centering
\singlespace
\pgfplotstabletypeset[sci zerofill,sci E, col sep=comma,
	columns/0/.style={ column name=\dtmax{}, precision=1},
	columns/1/.style={ column name=Linear, precision=3},
	columns/2/.style={ column name=Nonlinear, precision=3},
	columns/3/.style={ column name=Linear, precision=3},
	columns/4/.style={ column name=Nonlinear, precision=3},
	every head row/.style={
		before row={
			\toprule
			&\multicolumn{2}{c}{$\tilde{R}$} & \multicolumn{2}{c}{$\tilde{R}_{M}$}\\
		},
		after row=\midrule
	},
	every last row/.style={
after row=\bottomrule}]{tables/singleMetricData.tex}
\caption{Nonlinear convergence metrics for the single-phase problem.}
\label{tab:singleMetric}
\end{table}

%============================================================================
% Problem that shows that the domain decomposition algorithm is implemented correctly.
%============================================================================
\section{Complex Geometry Problem}
\label{sect:complexProblem}
The complex geometry problem tests whether the domain decomposition algorithm was implemented properly.
To properly test the domain decomposition algorithm many different permutations of the composition of the nonlinear domain are required.
The presence of both axial momentum flow paths and transverse momentum flow paths is necessary.
If the nonlinear domain of the decomposed problem is subjected to just a single Newton step, then the resulting dual domain solution should match within numerical round off that of a single domain with a single Newton step regardless of the composition of the nonlinear domain.
A model of the General Electric nine-rod subchannel experimental facility \cite{Lahey1970} suits the requirements of this test.

\subsection{Model}
\label{sect:complexModel}

The General Electric (GE) experiments were conducted in an experimental facility with a nine-rod bundle representative of a BWR channel.
The exact geometry of the experimental facilities can be found in the GE technical report \cite{Lahey1970}.
The basic layout of the portion of the facility that is of interest to this work is as follows: there is an inlet plenum that is opens into a nine-rod bundle depicted in \fig{fig:channel_layout}; above the rod-bundle there is an outlet plenum.

\begin{figure}[h!tb]
\centering
\begin{center}
\begin{tikzpicture}

%Rounded corners
\draw (3,2) arc (0:90:1);
\draw (-2,3) arc (90:180:1);
\draw (-3,-2) arc (180:270:1);
\draw (2,-3) arc (270:360:1);

%Grid of circles
\draw (0,0) circle (0.75);
\draw (0,2) circle (0.75);
\draw (0,-2) circle (0.75);
\draw (2,0) circle (0.75);
\draw (2,2) circle (0.75);
\draw (2,-2) circle (0.75);
\draw (-2,0) circle (0.75);
\draw (-2,2) circle (0.75);
\draw (-2,-2) circle (0.75);

%Box
\draw (-3,-2) -- (-3,2);
\draw (3,-2) -- (3,2);
\draw (2,-3) -- (-2,-3);
\draw (-2,3) -- (2,3);

%Misc lines
\draw [dashed] (-2,3) -- (-2,2.75);
\draw [dashed] (0,3) -- (0,2.75);
\draw [dashed] (-2,1.25) -- (-2,0.75);
\draw [dashed] (0,1.25) -- (0,0.75);
\draw [dashed] (-3,2) -- (-2.75,2);
\draw [dashed] (-3,0) -- (-2.75,0);
\draw [dashed] (-1.25,2) -- (-0.75,2);
\draw [dashed] (-1.25,0) -- (-0.75,0);

\draw [dashed] (0,0) -- (-0.5,0.5);
\draw [dashed] (2,-2) -- (1.5,-1.5);

%Misc arrows and labels
\draw [<-] (-3,4) -- (-1,4);
\draw [->] (1,4) -- (3,4);
\draw (-3,4.25) -- (-3,3.25);
\draw (3,4.25) -- (3,3.25);
\draw (0,4) node {2.316 SQ.};

\draw [<-] (0,-3.5) -- (0.5,-3.5);
\draw [->] (1.5,-3.5) -- (2,-3.5);
\draw (2,-3.7) -- (2,-3);
\draw (0,-3.7) -- (0,-2.5);
\draw (1,-3.5) node {0.738};
\draw (1,-4) node {(TYP.)};

\draw [<->] (2,-3.5) -- (3,-3.5);
\draw (3,-3.7) -- (3,-3);
\draw (2.5,-3.8) node {0.420};
\draw (2.5,-4.3) node {(TYP.)};

\draw [->] (-3.4,2.25) -- (-2.55,2.65);
\draw (-5,2.25) node {Corner Subchannel};

\draw [->] (-3.4,0.75) -- (-1,1.25);
\draw (-5,0.75) node {Center Subchannel};

\draw [->] (0.1,3.5) -- (-0.25,3.5) -- (-1,2.5);
\draw (1.5,3.5) node {Side Subchannel};

\draw [->] (-3.4,-3.5) -- (-2.75,-2.75);
\draw (-5,-3.5) node {0.40 Radius (TYP.)};

\draw [->] (3.3,-1) -- (2.6,-1.6);
\draw (5,-1) node {0.570 Diameter Rods};

\draw (-3,3) node {C1};
\draw (-1.5,3.5) node {W1};
\draw (1,-2.75) node {CC1};

\draw (3,-2.75) node {C2};
\draw (1,-2.4) node {W2};
\draw (-2,0.5) node {CC2};

\draw (0,0.5) node {1};
\draw (2,-1.5) node {2};

\end{tikzpicture}
\end{center}
\caption{Position of pressure taps for setting isokinetic conditions. Note splitter positions for the various subchannels.}
\label{fig:position_of_pressure taps}
\caption{Cross-section of rod bundle geometry adapted from GE experiment \cite{Lahey1970}.}
\label{fig:channel_layout}
\end{figure}

The \cobra{} model is broken into three sections.
The first section is the inlet plenum, while the third section is the outlet plenum.
The inlet plenum is 12 [in] tall with a cross-sectional area of 2.93 $[ \text{in}^2]$.
The outlet plenum is identically dimensioned.
The second section represents the rod-bundle.
This section is 121.5 [in] tall.
The rod-bundle is approximated by a four-by-four grid of sixteen channels.
There are the four corner subchannels, the eight side subchannels, and the four center subchannels in the model.
The dimensions for the various channels are shown in \fig{fig:channel_layout}.

The primary purpose of this problem is to test the ability of the domain decomposition algorithm to accurately recover the linear solution when the nonlinear domain is restricted to a single Newton step.
For that reason the details of the boundary and initial conditions are omitted here.

There are eighteen channels in this model: the 16 subchannels in the rod bundle, the inlet plenum, and the outlet plenum.
That means that there are $2^{18}$ possible ways to decompose this problem with the domain decomposition algorithm.
The time required to run all possible permutations is prohibitive, so only a subset are simulated here.
To test if the domain decomposition algorithm was implemented correctly, 256 different permutations were selected at random and run.

\subsection{Results}
\label{sect:complexResults}

The intent of this problem is to determine if the domain decomposition algorithm is properly implemented.
First, recognize that the use of the domain decomposition algorithm with only a single Newton step permitted in the nonlinear domain results a single Newton step being taken over the entire domain.
This provided the basis for the following comparison.
Regardless of how the $2^{18}$ channels of the problem are allocated between a linear and nonlinear domain, the solutions produced by the domain decomposition algorithm should be within round-off error to that produced by the single domain linear solver.
The caveat is that the different linear algebra involved with the domain decomposition will not produce the same floating-point solution as that produced by the linear solver.
This floating-point variance should not impact the solution above the level of numerical round-off error.
However, the traditional \cobra{} graphics file is created with single precision variables.
These single precision variables were incapable of capturing the error present in the simulation since, to single precision, the results were always the same between the domain decomposed and the non-decomposed solutions.
A special version of \cobra{} was compiled that produced the variables of interest in double precision.

There were 256 different simulations run: one where the entire domain is linear, one where the entire domain is nonlinear solver, and 254 where a random number of random channels are nonlinear.
This collection is a representative subset of the $2^{18}$ possible decompositions.
For each of these simulations, the pressure at two locations were recorded: the pressures near the top and near the bottom of the upper left corner channel of \fig{fig:channel_layout} were chosen.
The differences between these pressures and those produced by the linear solver were squared and integrated over the course of the transient.
The bar graph in \fig{fig:complexBar} shows the distribution of these error integrals at both the top and bottom of the channel.

\begin{figure}[h!tb]
\centering
% This file was created by matlab2tikz v0.4.4 running on MATLAB 8.1.
% Copyright (c) 2008--2013, Nico Schlömer <nico.schloemer@gmail.com>
% All rights reserved.
% 
% The latest updates can be retrieved from
%   http://www.mathworks.com/matlabcentral/fileexchange/22022-matlab2tikz
% where you can also make suggestions and rate matlab2tikz.
% 
\tikzsetnextfilename{plots/complexBar_eps}
%
% defining custom colors
\definecolor{mycolor1}{rgb}{0.313725501298904,0.313725501298904,0.313725501298904}%
\definecolor{mycolor2}{rgb}{0.941176474094391,0.941176474094391,0.941176474094391}%
%
\begin{tikzpicture}

\begin{axis}[%
width=0.8\textwidth,
%height=0.630967741935484\textwidth,
height=0.491294629700995\textwidth,
area legend,
scale only axis,
xmin=7e-12,
xmax=7e-10,
xlabel={Error [psia]},
ymin=0,
ymax=0.35,
ylabel={Frequency [-]},
axis x line*=bottom,
axis y line*=left,
legend style={at={(0.03,0.97)},anchor=north west,draw=black,fill=white,legend cell align=left}
]
\addplot[ybar,bar width=0.01087013866087131\textwidth,bar shift=-0.00366883666304457\textwidth,draw=black,fill=mycolor1] plot coordinates{(4.87585793962353e-10,0.0393700787401575)
(5.07560571350041e-10,0.137795275590551)
(5.27535348737729e-10,0.28740157480315)
(5.47510126125417e-10,0.283464566929134)
(5.67484903513105e-10,0.12992125984252)
(5.87459680900793e-10,0.062992125984252)
(6.07434458288481e-10,0.0393700787401575)
(6.27409235676168e-10,0.0078740157480315)
(6.47384013063856e-10,0.0078740157480315)
(6.67358790451544e-10,0.00393700787401575)};

\addlegendentry{bottom [in]};

\addplot [
color=black,
solid,
forget plot
]
table[row sep=crcr]{
7e-11 0\\
7e-10 0\\
};
\addplot[ybar,bar width=0.00265379546791768\textwidth,bar shift=0.00103362216744855\textwidth,draw=black,fill=mycolor2] plot coordinates{(7.59143858886091e-11,0.015748031496063)
(8.15418843558291e-11,0.106299212598425)
(8.71693828230491e-11,0.224409448818898)
(9.27968812902691e-11,0.244094488188976)
(9.84243797574891e-11,0.236220472440945)
(1.04051878224709e-10,0.0905511811023622)
(1.09679376691929e-10,0.031496062992126)
(1.15306875159149e-10,0.0354330708661417)
(1.20934373626369e-10,0.0078740157480315)
(1.26561872093589e-10,0.0078740157480315)};

\addlegendentry{top [in]};

\end{axis}
\end{tikzpicture}%

\caption{Histogram of pressure error at two locations.}
\label{fig:complexBar}
\end{figure}

These results indicate that the domain decomposition algorithm was implemented as intended.

%============================================================================
% Problem that shows that when it is possible to isolate the nonlinearities, the computational cost can be lower.
%============================================================================
\section{Valve-Type Problem}
\label{sect:valveProblem}

This problem is designed to demonstrate the ability of the domain decomposition algorithm to resolve isolated nonlinearities at a lower computational cost than the fully nonlinear solver. 
First, the model will be discussed and initial and boundary conditions provided.
Second, a timestep sensitivity study of the solution for both the linear and the nonlinear solver will be conducted.
Next, the portion of the domain with the nonlinearities will be isolated via the domain decomposition algorithm and another timestep sensitivity study will be conducted.
Finally, the solution and run time data from the three methods will be compared.

\subsection{Model}
\label{subsect:valveModel}

The physical system being modeled is that of four pipes.
Each pipe is modeled as having ten continuity volumes, each with a height, $\dx{}$, of 4 [in] and a cross-sectional area, $ A_{c} $, of 16 $[\text{in}^{2}]$.
Two pipes have a fixed obstruction, manifesting as a loss coefficients, along their lengths.
One pipe has a complete obstruction, preventing flow.
The last pipe has a boundary condition that imitates the behavior of a valve by using a  time dependent loss coefficient at one point.
Initially fully open, the valve slowly closes.
While closing, the valve stops twice at cross-sectional areas that are equivalent to the two loss coefficients of the two partially obstructed pipes.
After fully closing, the valve returns to the first stopping point.

The pipes are initially full of subcooled liquid at 800 $[ \text{psia}] $.
The bottom and top of the pipes are open to reservoirs.
The bottom reservoir is full of subcooled liquid with a time variant pressure.
The pressure starts at at 802 $[ \text{psia} ] $ and increases to 818 $[ \text{psia}]$ over the first second, staying there for the duration of the problem.
The top reservoir is full of subcooled liquid at 800 $[\text{psia} ] $.
This pressure does not change as a function of time.
The $\Delta P$ that exists between these two reservoirs is what drives the flow in this problem.

The model used in \cobra{} is a single section.
Within that section there are four channels.
These channels are hydrodynamically isolated from each other.
They correspond to the four pipes.
The two partially-obstructed pipes have user-defined loss coefficients placed 15 [in] up the pipe.
The two coefficients are 40 $[-]$ and 160 $[-]$.
The fully-obstructed channel has a no-flow boundary condition at 15 [in].
The fourth channel has a time dependent area ratio multiplier, $A(t)_r$, at 15 [in].
This multiplier acts to reduce the area of the momentum flow path, $A_m$, used in the determination of form losses and wall drag.
However, these four channels have had their wall drag terms artificially reduced.
Reducing the wall drag allows the impact of the loss coefficients upon the solution to be isolated.
Since there is no wall drag in this problem, the area ratio can be related to the effective loss coefficient, $K(t)$, by \eqref{eqn:valveLossCoefficient}.

\begin{equation}
\label{eqn:valveLossCoefficient}
K(t) = \frac{K_{o}}{{A(t)_r}^2}
\end{equation}

The base loss coefficient, $K_o$, for channel four is 10 [-].
Using \eqref{eqn:valveLossCoefficient}, the equivalent area ratios for the loss coefficients of the two partially obstructed pipes, 40 [-] and 160 [-], are 0.5 and 0.25 respectively.
The time dependent area ratio multiplier for channel four is given by \eqref{eqn:valveTransLoss}.

\begin{equation}
\label{eqn:valveTransLoss}
A(t)_{r} = \left\{
\begin{array}{cclrcll}
 1.0 & [-] & , & & t & \leq 3 & [\text{s}] \\
 1.0 - 0.5 \left( t - 3\right) & [-] & , & 3\; [\text{s}] < & t & \leq 4 & [\text{s}] \\
 0.5 & [-] & , & 4\; [\text{s}] < & t & \leq 7 & [\text{s}] \\
 0.5 - 0.25 \left( t - 7\right) & [-] & , &  7\; [\text{s}] < & t & \leq 8 & [\text{s}] \\
 0.25 & [-] & , & 8\; [\text{s}] < & t & \leq 11 & [\text{s}] \\
 0.25 - 0.25 \left( t - 11\right) & [-] & , & 11\; [\text{s}] < & t & \leq 12 & [\text{s}] \\
 0 & [-] & , & 12\; [\text{s}] < & t & \leq 15 & [\text{s}] \\
 0.5 \left( t - 15\right) & [-] & , & & t & > 16 & [\text{s}]
\end{array}\right.
\end{equation}

The simulation is run to an end time of 19 seconds.
The functional form of the transient area boundary condition was chosen so that the solution of the transient area pipe would mimic each of the three other pipes for portions of the transient.

\subsection{Results}
\label{subsect:valveResults}

The parameter of interest is the flow of liquid at the 15 [in] above the inlet.
This location corresponds to the location of either the loss coefficients, the no-flow boundary condition, or the variable momentum-area boundary condition in each of the respective pipes.
For this study, $\dt{}_{0}$, $r_{f}$, and $n_{t}$ used in \eqref{eqn:timeStepAlgo} were 1.0, 2.0, and 10, respectively.
Upon examining the solution, it was determined that timestep size was \dtcrnt{} limited for a portion of the transient until \dtmax{} = \expneg{1.56}{2}{[s]} was reached.
The timestep history of the simulations from \dtmax{} = 1.0 through \dtmax{} = \expneg{6.25}{2}{[s]} were identical for both the linear and the nonlinear solution.
As such, the first four simulation results for each solver were dropped from the data sets since they represented duplicate results.
The solution obtained with \dtmax{} = \expneg{6.25}{2}{[s]} with the linear solver is shown in \fig{fig:valveLin6pt25em02}.

\begin{figure}[h!tb]
\centering
% This file was created by matlab2tikz v0.4.3.
% Copyright (c) 2008--2013, Nico Schlömer <nico.schloemer@gmail.com>
% All rights reserved.
% 
\tikzsetnextfilename{plots/valveLin6pt2500em02_pdf}
\begin{tikzpicture}

\begin{axis}[%
width=\mytikzpicwidth,
height=\mytikzpicheight,
scale only axis,
xmin=0,
xmax=19,
xlabel={Time $[\text{s}]$},
ymin=-5,
ymax=125,
ylabel={Flow Rate $[ \frac{\lbm{}}{\text{s}} ]$},
legend style={draw=black,fill=white,legend cell align=left}
]
\addplot [
color=black,
solid
]
table[row sep=crcr]{
0 0\\
9.99999974737875e-06 7.3426820179634e-09\\
2.49999993684469e-05 6.82794052409008e-07\\
4.75000015285332e-05 1.61540210683597e-05\\
8.12500002211891e-05 0.000229488563491032\\
0.000131875000079162 0.00192578241694719\\
0.000207812496228144 0.00898927729576826\\
0.000321718747727573 0.023475144058466\\
0.000492578139528632 0.0383497066795826\\
0.00074886716902256 0.0502717047929764\\
0.00113330082967877 0.0704957619309425\\
0.00170995120424777 0.1085309907794\\
0.00257492670789361 0.167750924825668\\
0.00387239013798535 0.25895568728447\\
0.00529959984123707 0.363645792007446\\
0.00686953077092767 0.484240502119064\\
0.0085964547470212 0.623413980007172\\
0.0104960706084967 0.784365355968475\\
0.0125856483355165 0.970891773700714\\
0.014884184114635 1.18749308586121\\
0.0174125730991364 1.43950164318085\\
0.020193800330162 1.73323345184326\\
0.0232531521469355 2.07616090774536\\
0.0266184378415346 2.4771089553833\\
0.0303202513605356 2.94646978378296\\
0.0343922488391399 3.49643588066101\\
0.0388714447617531 4.14123106002808\\
0.0437985584139824 4.89732694625854\\
0.0492183864116669 5.78360366821289\\
0.0551801957190037 6.8213996887207\\
0.0617381855845451 8.03437232971191\\
0.0689519718289375 9.44805526733398\\
0.0768871381878853 11.0889587402344\\
0.0856158286333084 12.9830656051636\\
0.095217376947403 15.1535520553589\\
0.105779089033604 17.6176853179932\\
0.117396965622902 20.3830242156982\\
0.13017663359642 23.4434223175049\\
0.144234269857407 26.7757205963135\\
0.159697666764259 30.3384552001953\\
0.176707401871681 34.0738868713379\\
0.195418119430542 37.9140815734863\\
0.215999901294708 41.790412902832\\
0.238639861345291 45.644416809082\\
0.2635438144207 49.436939239502\\
0.290938168764114 53.1529197692871\\
0.321071952581406 56.8006935119629\\
0.354219108819962 60.406681060791\\
0.390680998563766 64.007698059082\\
0.427072286605835 67.3160705566406\\
0.461725652217865 70.2629699707031\\
0.494970262050629 72.9423904418945\\
0.527021527290344 75.4129028320313\\
0.55803918838501 77.7141189575195\\
0.588148653507233 79.8743591308594\\
0.617450594902039 81.914680480957\\
0.646027386188507 83.8512649536133\\
0.67394757270813 85.6969223022461\\
0.701269030570984 87.4619827270508\\
0.728040993213654 89.1549606323242\\
0.754306197166443 90.7829818725586\\
0.780101537704468 92.3520812988281\\
0.805459678173065 93.8674163818359\\
0.830409228801727 95.3334503173828\\
0.854975879192352 96.7540817260742\\
0.8791823387146 98.1327209472656\\
0.903049230575562 99.4723892211914\\
0.926595151424408 100.775764465332\\
0.949836909770966 102.04524230957\\
0.972789883613586 103.282974243164\\
0.995468020439148 104.49089050293\\
1.01788425445557 105.216697692871\\
1.0400505065918 105.585403442383\\
1.06216442584991 105.77458190918\\
1.08427441120148 105.871871948242\\
1.10638225078583 105.921897888184\\
1.12848901748657 105.947624206543\\
1.15059518814087 105.96085357666\\
1.1727010011673 105.967651367188\\
1.19480681419373 105.971153259277\\
1.21691238880157 105.972953796387\\
1.23901796340942 105.973876953125\\
1.26112222671509 105.974349975586\\
1.28322517871857 105.974594116211\\
1.30532777309418 105.974716186523\\
1.32742953300476 105.974784851074\\
1.34953081607819 105.974815368652\\
1.37163174152374 105.974830627441\\
1.39373195171356 105.974838256836\\
1.41583180427551 105.97484588623\\
1.43793165683746 105.97484588623\\
1.46003234386444 105.974853515625\\
1.48213422298431 105.974853515625\\
1.50423765182495 105.974853515625\\
1.52634274959564 105.974853515625\\
1.5484482049942 105.974853515625\\
1.57055377960205 105.974853515625\\
1.5926593542099 105.974853515625\\
1.61476469039917 105.974853515625\\
1.63686990737915 105.974853515625\\
1.65897512435913 105.974853515625\\
1.68108010292053 105.974853515625\\
1.70318508148193 105.974853515625\\
1.72528982162476 105.974853515625\\
1.74739444255829 105.974853515625\\
1.76949906349182 105.974853515625\\
1.79160356521606 105.974853515625\\
1.81370830535889 105.974853515625\\
1.83581340312958 105.974853515625\\
1.85791862010956 105.974853515625\\
1.88002419471741 105.974853515625\\
1.90212976932526 105.974853515625\\
1.92423534393311 105.974853515625\\
1.94634091854095 105.974853515625\\
1.9684464931488 105.974853515625\\
1.99055194854736 105.974853515625\\
2.01265740394592 105.974853515625\\
2.03476285934448 105.974853515625\\
2.05686831474304 105.974853515625\\
2.0789737701416 105.974853515625\\
2.10107922554016 105.974853515625\\
2.12318468093872 105.974853515625\\
2.14529013633728 105.974853515625\\
2.16739559173584 105.974853515625\\
2.1895010471344 105.974853515625\\
2.21160674095154 105.974853515625\\
2.2337121963501 105.974853515625\\
2.25581765174866 105.974853515625\\
2.2779233455658 105.974853515625\\
2.30002880096436 105.974853515625\\
2.32213449478149 105.974853515625\\
2.34423995018005 105.974853515625\\
2.36634540557861 105.974853515625\\
2.38845109939575 105.974853515625\\
2.41055655479431 105.974853515625\\
2.43266201019287 105.974853515625\\
2.45476770401001 105.974853515625\\
2.47687315940857 105.974853515625\\
2.49897861480713 105.974853515625\\
2.52108430862427 105.974853515625\\
2.54318976402283 105.974853515625\\
2.56529545783997 105.974853515625\\
2.58740091323853 105.974853515625\\
2.60950636863709 105.974853515625\\
2.63161206245422 105.974853515625\\
2.65371751785278 105.974853515625\\
2.67582321166992 105.974853515625\\
2.69792866706848 105.974853515625\\
2.72003412246704 105.974853515625\\
2.74213981628418 105.974853515625\\
2.76424527168274 105.974853515625\\
2.78635096549988 105.974853515625\\
2.80845642089844 105.974853515625\\
2.830561876297 105.974853515625\\
2.85266757011414 105.974853515625\\
2.8747730255127 105.974853515625\\
2.89687871932983 105.974853515625\\
2.91898417472839 105.974853515625\\
2.94108963012695 105.974853515625\\
2.96319532394409 105.974853515625\\
2.98530077934265 105.974853515625\\
3.00740647315979 105.783866882324\\
3.02951192855835 105.111946105957\\
3.05169749259949 104.186538696289\\
3.07412838935852 103.119384765625\\
3.09681701660156 101.971786499023\\
3.11977505683899 100.775596618652\\
3.14301300048828 99.5470809936523\\
3.16654181480408 98.2943115234375\\
3.19037246704102 97.0210647583008\\
3.21451687812805 95.7289123535156\\
3.23898792266846 94.4182662963867\\
3.26379895210266 93.0889282226563\\
3.28896427154541 91.7403869628906\\
3.31449961662292 90.371940612793\\
3.34042167663574 88.9827423095703\\
3.36674857139587 87.5718612670898\\
3.39349937438965 86.1382827758789\\
3.42069554328918 84.6808700561523\\
3.4483597278595 83.1984176635742\\
3.47651672363281 81.6895980834961\\
3.50519371032715 80.1529693603516\\
3.53442049026489 78.5869369506836\\
3.56422996520996 76.9897613525391\\
3.59465765953064 75.3595352172852\\
3.6257438659668 73.6941299438477\\
3.65753245353699 71.9911880493164\\
3.69007301330566 70.2481002807617\\
3.72342085838318 68.4618988037109\\
3.75763893127441 66.629280090332\\
3.79279804229736 64.746467590332\\
3.82897973060608 62.8091583251953\\
3.86627721786499 60.8124084472656\\
3.90479946136475 58.7504920959473\\
3.94467377662659 56.61669921875\\
3.9860508441925 54.4031181335449\\
4.02911138534546 53.2985801696777\\
4.07284307479858 53.053783416748\\
4.11683511734009 53.001636505127\\
4.16100072860718 52.9906539916992\\
4.20520448684692 52.9883499145508\\
4.24941444396973 52.987865447998\\
4.29362487792969 52.9877662658691\\
4.33783578872681 52.9877433776855\\
4.38204622268677 52.9877395629883\\
4.42625665664673 52.9877395629883\\
4.47046709060669 52.9877395629883\\
4.51467752456665 52.9877395629883\\
4.55888795852661 52.9877395629883\\
4.60309839248657 52.9877395629883\\
4.64730882644653 52.9877395629883\\
4.69151973724365 52.9877395629883\\
4.73573064804077 52.9877395629883\\
4.77994155883789 52.9877395629883\\
4.82415246963501 52.9877395629883\\
4.86836338043213 52.9877395629883\\
4.91257381439209 52.9877395629883\\
4.95678472518921 52.9877395629883\\
5.00099563598633 52.9877395629883\\
5.04520654678345 52.9877395629883\\
5.08941698074341 52.9877395629883\\
5.13362789154053 52.9877395629883\\
5.17783832550049 52.9877395629883\\
5.22204923629761 52.9877395629883\\
5.26626014709473 52.9877395629883\\
5.31047105789185 52.9877395629883\\
5.35468196868896 52.9877395629883\\
5.39889240264893 52.9877395629883\\
5.44310331344604 52.9877395629883\\
5.48731422424316 52.9877395629883\\
5.53152513504028 52.9877395629883\\
5.5757360458374 52.9877395629883\\
5.61994695663452 52.9877395629883\\
5.66415786743164 52.9877395629883\\
5.7083683013916 52.9877395629883\\
5.75257921218872 52.9877395629883\\
5.79679012298584 52.9877395629883\\
5.84100103378296 52.9877395629883\\
5.88521194458008 52.9877395629883\\
5.9294228553772 52.9877395629883\\
5.97363376617432 52.9877395629883\\
6.01784420013428 52.9877395629883\\
6.0620551109314 52.9877395629883\\
6.10626602172852 52.9877395629883\\
6.15047693252563 52.9877395629883\\
6.19468784332275 52.9877395629883\\
6.23889875411987 52.9877395629883\\
6.28310966491699 52.9877395629883\\
6.32732009887695 52.9877395629883\\
6.37153100967407 52.9877395629883\\
6.41574192047119 52.9877395629883\\
6.45995283126831 52.9877395629883\\
6.50416374206543 52.9877395629883\\
6.54837465286255 52.9877395629883\\
6.59258508682251 52.9877395629883\\
6.63679599761963 52.9877395629883\\
6.68100690841675 52.9877395629883\\
6.72521781921387 52.9877395629883\\
6.76942873001099 52.9877395629883\\
6.81363964080811 52.9877395629883\\
6.85785055160522 52.9877395629883\\
6.90206098556519 52.9877395629883\\
6.9462718963623 52.9877395629883\\
6.99048280715942 52.9877395629883\\
7.03469371795654 52.2620544433594\\
7.07890462875366 51.1824531555176\\
7.12311553955078 50.0258483886719\\
7.1673264503479 48.8525352478027\\
7.21153688430786 47.6756553649902\\
7.25574779510498 46.4980316162109\\
7.2999587059021 45.3202667236328\\
7.34416961669922 44.1424942016602\\
7.38838052749634 42.9647331237793\\
7.43259143829346 41.7869987487793\\
7.47680234909058 40.6092872619629\\
7.52101278305054 39.4316062927246\\
7.56522369384766 38.2539520263672\\
7.60943460464478 37.0763359069824\\
7.65364551544189 35.898754119873\\
7.69785642623901 34.7212142944336\\
7.74206733703613 33.5437202453613\\
7.78627824783325 32.3662757873535\\
7.83048868179321 31.1888885498047\\
7.87469959259033 30.0115623474121\\
7.91891050338745 28.8343067169189\\
7.96312141418457 27.6571311950684\\
8.00733184814453 26.6465225219727\\
8.05154323577881 26.5120487213135\\
8.09575366973877 26.4960327148438\\
8.13996505737305 26.4941577911377\\
8.18417549133301 26.4939384460449\\
8.22838687896729 26.4939117431641\\
8.27259731292725 26.4939098358154\\
8.31680774688721 26.4939098358154\\
8.36101913452148 26.4939079284668\\
8.40522956848145 26.4939079284668\\
8.44944095611572 26.4939079284668\\
8.49365139007568 26.4939079284668\\
8.53786277770996 26.4939079284668\\
8.58207321166992 26.4939079284668\\
8.62628364562988 26.4939079284668\\
8.67049503326416 26.4939079284668\\
8.71470546722412 26.4939079284668\\
8.7589168548584 26.4939079284668\\
8.80312728881836 26.4939079284668\\
8.84733867645264 26.4939079284668\\
8.8915491104126 26.4939079284668\\
8.93575954437256 26.4939079284668\\
8.97997093200684 26.4939079284668\\
9.0241813659668 26.4939079284668\\
9.06839275360107 26.4939079284668\\
9.11260318756104 26.4939079284668\\
9.156813621521 26.4939079284668\\
9.20102500915527 26.4939079284668\\
9.24523544311523 26.4939079284668\\
9.28944683074951 26.4939079284668\\
9.33365726470947 26.4939079284668\\
9.37786865234375 26.4939079284668\\
9.42207908630371 26.4939079284668\\
9.46628952026367 26.4939079284668\\
9.51050090789795 26.4939079284668\\
9.55471134185791 26.4939079284668\\
9.59892272949219 26.4939079284668\\
9.64313316345215 26.4939079284668\\
9.68734455108643 26.4939079284668\\
9.73155498504639 26.4939079284668\\
9.77576541900635 26.4939079284668\\
9.81997680664063 26.4939079284668\\
9.86418724060059 26.4939079284668\\
9.90839862823486 26.4939079284668\\
9.95260906219482 26.4939079284668\\
9.9968204498291 26.4939079284668\\
10.0410308837891 26.4939079284668\\
10.085241317749 26.4939079284668\\
10.1294527053833 26.4939079284668\\
10.1736631393433 26.4939079284668\\
10.2178745269775 26.4939079284668\\
10.2620849609375 26.4939079284668\\
10.3062963485718 26.4939079284668\\
10.3505067825317 26.4939079284668\\
10.3947172164917 26.4939079284668\\
10.438928604126 26.4939079284668\\
10.4831390380859 26.4939079284668\\
10.5273504257202 26.4939079284668\\
10.5715608596802 26.4939079284668\\
10.6157722473145 26.4939079284668\\
10.6599826812744 26.4939079284668\\
10.7041931152344 26.4939079284668\\
10.7484045028687 26.4939079284668\\
10.7926149368286 26.4939079284668\\
10.8368263244629 26.4939079284668\\
10.8810367584229 26.4939079284668\\
10.9252481460571 26.4939079284668\\
10.9694585800171 26.4939079284668\\
11.0136690139771 26.1752624511719\\
11.0578804016113 25.1147441864014\\
11.1020908355713 23.9547080993652\\
11.1463022232056 22.7804050445557\\
11.1905126571655 21.6041946411133\\
11.2347240447998 20.4278793334961\\
11.2789344787598 19.2517433166504\\
11.3231449127197 18.0758666992188\\
11.367356300354 16.9003086090088\\
11.411566734314 15.7251396179199\\
11.4557781219482 14.5504512786865\\
11.4778890609741 13.9317502975464\\
11.5 13.3384485244751\\
11.5243225097656 12.6927003860474\\
11.5510768890381 11.9840898513794\\
11.5805063247681 11.2058210372925\\
11.6128797531128 10.3513040542603\\
11.6484899520874 9.41370677947998\\
11.6876602172852 8.38610649108887\\
11.730749130249 7.26203107833862\\
11.774959564209 6.10678195953369\\
11.8191709518433 4.95429944992065\\
11.8633813858032 3.81441807746887\\
11.9075918197632 2.70163488388062\\
11.9518032073975 1.65573978424072\\
11.9960136413574 0.831265449523926\\
12.0402250289917 0\\
12.0844354629517 0\\
12.1286468505859 0\\
12.1728572845459 0\\
12.2170677185059 0\\
12.2612791061401 0\\
12.3054895401001 0\\
12.3497009277344 0\\
12.3939113616943 0\\
12.4381227493286 0\\
12.4823331832886 0\\
12.5265436172485 0\\
12.5707550048828 0\\
12.6149654388428 0\\
12.6591768264771 0\\
12.703387260437 0\\
12.7475986480713 0\\
12.7918090820313 0\\
12.8360195159912 0\\
12.8802309036255 0\\
12.9244413375854 0\\
12.9686527252197 0\\
13.0128631591797 0\\
13.057074546814 0\\
13.1012849807739 0\\
13.1454954147339 0\\
13.1897068023682 0\\
13.2339172363281 0\\
13.2781286239624 0\\
13.3223390579224 0\\
13.3665504455566 0\\
13.4107608795166 0\\
13.4549713134766 0\\
13.4991827011108 0\\
13.5433931350708 0\\
13.5876045227051 0\\
13.631814956665 0\\
13.6760263442993 0\\
13.7202367782593 0\\
13.7644472122192 0\\
13.8086585998535 0\\
13.8528690338135 0\\
13.8970804214478 0\\
13.9412908554077 0\\
13.985502243042 0\\
14.029712677002 0\\
14.0739231109619 0\\
14.1181344985962 0\\
14.1623449325562 0\\
14.2065563201904 0\\
14.2507667541504 0\\
14.2949781417847 0\\
14.3391885757446 0\\
14.3833990097046 0\\
14.4276103973389 0\\
14.4718208312988 0\\
14.5160322189331 0\\
14.5602426528931 0\\
14.6044540405273 0\\
14.6486644744873 0\\
14.6928749084473 0\\
14.7370862960815 0\\
14.7812967300415 0\\
14.8255081176758 0\\
14.8697185516357 0\\
14.9139289855957 0\\
14.95814037323 0\\
15.0023508071899 0.117176152765751\\
15.0465621948242 20.6927852630615\\
15.0907726287842 10.9598197937012\\
15.1349840164185 7.8855185508728\\
15.1791944503784 9.56354999542236\\
15.2234048843384 11.9345808029175\\
15.2676162719727 14.2009210586548\\
15.3118267059326 16.4927539825439\\
15.3560380935669 18.7888126373291\\
15.4002485275269 21.0889701843262\\
15.4444599151611 23.392017364502\\
15.4886703491211 25.6972122192383\\
15.5328807830811 28.0040435791016\\
15.5770921707153 30.3121509552002\\
15.6213026046753 32.621265411377\\
15.6655139923096 34.9311981201172\\
15.7097244262695 37.2417945861816\\
15.7539358139038 39.5529441833496\\
15.7981462478638 41.8645553588867\\
15.8423566818237 44.1765518188477\\
15.886568069458 46.488883972168\\
15.930778503418 48.8014984130859\\
15.9749898910522 51.1143608093262\\
16.0192012786865 52.6112480163574\\
16.0630474090576 52.9090232849121\\
16.1063537597656 52.9710159301758\\
16.1503849029541 52.9842262268066\\
16.1945571899414 52.987003326416\\
16.2387599945068 52.9875831604004\\
16.2829685211182 52.9877052307129\\
16.3271789550781 52.9877319335938\\
16.3713893890381 52.987735748291\\
16.4156017303467 52.9877395629883\\
16.4598121643066 52.9877395629883\\
16.5040225982666 52.9877395629883\\
16.5482330322266 52.9877395629883\\
16.5924434661865 52.9877395629883\\
16.6366558074951 52.9877395629883\\
16.6808662414551 52.9877395629883\\
16.725076675415 52.9877395629883\\
16.769287109375 52.9877395629883\\
16.813497543335 52.9877395629883\\
16.8577079772949 52.9877395629883\\
16.9019184112549 52.9877395629883\\
16.9461288452148 52.9877395629883\\
16.9903411865234 52.9877395629883\\
17.0345516204834 52.9877395629883\\
17.0787620544434 52.9877395629883\\
17.1229724884033 52.9877395629883\\
17.1671829223633 52.9877395629883\\
17.2113952636719 52.9877395629883\\
17.2556056976318 52.9877395629883\\
17.2998161315918 52.9877395629883\\
17.3440265655518 52.9877395629883\\
17.3882369995117 52.9877395629883\\
17.4324493408203 52.9877395629883\\
17.4766597747803 52.9877395629883\\
17.5208702087402 52.9877395629883\\
17.5650806427002 52.9877395629883\\
17.6092929840088 52.9877395629883\\
17.6535034179688 52.9877395629883\\
17.6977138519287 52.9877395629883\\
17.7419242858887 52.9877395629883\\
17.7861347198486 52.9877395629883\\
17.8303470611572 52.9877395629883\\
17.8745574951172 52.9877395629883\\
17.9187679290771 52.9877395629883\\
17.9629783630371 52.9877395629883\\
18.0071887969971 52.9877395629883\\
18.0514011383057 52.9877395629883\\
18.0956115722656 52.9877395629883\\
18.1398220062256 52.9877395629883\\
18.1840324401855 52.9877395629883\\
18.2282447814941 52.9877395629883\\
18.2724552154541 52.9877395629883\\
18.3166656494141 52.9877395629883\\
18.360876083374 52.9877395629883\\
18.405086517334 52.9877395629883\\
18.4492988586426 52.9877395629883\\
18.4935092926025 52.9877395629883\\
18.5377197265625 52.9877395629883\\
18.5819301605225 52.9877395629883\\
18.6261405944824 52.9877395629883\\
18.670352935791 52.9877395629883\\
18.714563369751 52.9877395629883\\
18.7587738037109 52.9877395629883\\
18.8029842376709 52.9877395629883\\
18.8471965789795 52.9877395629883\\
18.8914070129395 52.9877395629883\\
18.9356174468994 52.9877395629883\\
18.9678077697754 52.9877395629883\\
19 52.9877395629883\\
};
\addlegendentry{$A_{r}(t)$};

\addplot [
color=black,
dash pattern=on 1pt off 3pt on 3pt off 3pt
]
table[row sep=crcr]{
0 0\\
9.99999974737875e-06 7.34263050361506e-09\\
2.49999993684469e-05 6.82787117511907e-07\\
4.75000015285332e-05 1.61537918756949e-05\\
8.12500002211891e-05 0.000229484256124124\\
0.000131875000079162 0.00192573899403214\\
0.000207812496228144 0.00898905657231808\\
0.000321718747727573 0.0234745796769857\\
0.000492578139528632 0.0383488200604916\\
0.00074886716902256 0.0502703413367271\\
0.00113330082967877 0.0704929903149605\\
0.00170995120424777 0.108524538576603\\
0.00257492670789361 0.167731299996376\\
0.00387239013798535 0.258889704942703\\
0.00529959984123707 0.363474696874619\\
0.00686953077092767 0.483860403299332\\
0.0085964547470212 0.622648656368256\\
0.0104960706084967 0.782925307750702\\
0.0125856483355165 0.968311131000519\\
0.014884184114635 1.18303418159485\\
0.0174125730991364 1.43201053142548\\
0.020193800330162 1.72092187404633\\
0.0232531521469355 2.05627965927124\\
0.0266184378415346 2.44546175003052\\
0.0303202513605356 2.8966953754425\\
0.0343922488391399 3.41895937919617\\
0.0388714447617531 4.0217547416687\\
0.0437985584139824 4.71469640731812\\
0.0492183864116669 5.50687074661255\\
0.0551801957190037 6.40590953826904\\
0.0617381855845451 7.41679286956787\\
0.0689519718289375 8.54045581817627\\
0.0768871381878853 9.77241897583008\\
0.0856158286333084 11.1017980575562\\
0.095217376947403 12.5111665725708\\
0.105779089033604 13.9776849746704\\
0.117396965622902 15.4756441116333\\
0.13017663359642 16.9800872802734\\
0.144234269857407 18.4706878662109\\
0.159697666764259 19.9348068237305\\
0.176707401871681 21.3688697814941\\
0.195418119430542 22.7777843475342\\
0.215999901294708 24.1727924346924\\
0.238639861345291 25.5685958862305\\
0.2635438144207 26.9806213378906\\
0.290938168764114 28.4230575561523\\
0.321071952581406 29.907886505127\\
0.354219108819962 31.4447612762451\\
0.390680998563766 33.0414009094238\\
0.427072286605835 34.5504035949707\\
0.461725652217865 35.9212799072266\\
0.494970262050629 37.185417175293\\
0.527021527290344 38.3625907897949\\
0.55803918838501 39.4668655395508\\
0.588148653507233 40.5088539123535\\
0.617450594902039 41.4968566894531\\
0.646027386188507 42.4374961853027\\
0.67394757270813 43.3361740112305\\
0.701269030570984 44.1973571777344\\
0.728040993213654 45.024787902832\\
0.754306197166443 45.8216438293457\\
0.780101537704468 46.5906410217285\\
0.805459678173065 47.3341293334961\\
0.830409228801727 48.0541496276855\\
0.854975879192352 48.752498626709\\
0.8791823387146 49.4307518005371\\
0.903049230575562 50.0903167724609\\
0.926595151424408 50.7324409484863\\
0.949836909770966 51.3582496643066\\
0.972789883613586 51.9687538146973\\
0.995468020439148 52.5648574829102\\
1.01788425445557 52.8429985046387\\
1.0400505065918 52.9377365112305\\
1.06216442584991 52.9704132080078\\
1.08427441120148 52.9817276000977\\
1.10638225078583 52.9856491088867\\
1.12848901748657 52.9870109558105\\
1.15059518814087 52.9874839782715\\
1.1727010011673 52.9876518249512\\
1.19480681419373 52.9877090454102\\
1.21691238880157 52.9877319335938\\
1.23901796340942 52.9877395629883\\
1.26112222671509 52.9877395629883\\
1.28322517871857 52.9877433776855\\
1.30532777309418 52.9877433776855\\
1.32742953300476 52.9877395629883\\
1.34953081607819 52.9877395629883\\
1.37163174152374 52.9877395629883\\
1.39373195171356 52.9877395629883\\
1.41583180427551 52.9877395629883\\
1.43793165683746 52.9877395629883\\
1.46003234386444 52.9877395629883\\
1.48213422298431 52.9877395629883\\
1.50423765182495 52.9877395629883\\
1.52634274959564 52.9877395629883\\
1.5484482049942 52.9877395629883\\
1.57055377960205 52.9877395629883\\
1.5926593542099 52.9877395629883\\
1.61476469039917 52.9877395629883\\
1.63686990737915 52.9877395629883\\
1.65897512435913 52.9877395629883\\
1.68108010292053 52.9877395629883\\
1.70318508148193 52.9877395629883\\
1.72528982162476 52.9877395629883\\
1.74739444255829 52.9877395629883\\
1.76949906349182 52.9877395629883\\
1.79160356521606 52.9877395629883\\
1.81370830535889 52.9877395629883\\
1.83581340312958 52.9877395629883\\
1.85791862010956 52.9877395629883\\
1.88002419471741 52.9877395629883\\
1.90212976932526 52.9877395629883\\
1.92423534393311 52.9877395629883\\
1.94634091854095 52.9877395629883\\
1.9684464931488 52.9877395629883\\
1.99055194854736 52.9877395629883\\
2.01265740394592 52.9877395629883\\
2.03476285934448 52.9877395629883\\
2.05686831474304 52.9877395629883\\
2.0789737701416 52.9877395629883\\
2.10107922554016 52.9877395629883\\
2.12318468093872 52.9877395629883\\
2.14529013633728 52.9877395629883\\
2.16739559173584 52.9877395629883\\
2.1895010471344 52.9877395629883\\
2.21160674095154 52.9877395629883\\
2.2337121963501 52.9877395629883\\
2.25581765174866 52.9877395629883\\
2.2779233455658 52.9877395629883\\
2.30002880096436 52.9877395629883\\
2.32213449478149 52.9877395629883\\
2.34423995018005 52.9877395629883\\
2.36634540557861 52.9877395629883\\
2.38845109939575 52.9877395629883\\
2.41055655479431 52.9877395629883\\
2.43266201019287 52.9877395629883\\
2.45476770401001 52.9877395629883\\
2.47687315940857 52.9877395629883\\
2.49897861480713 52.9877395629883\\
2.52108430862427 52.9877395629883\\
2.54318976402283 52.9877395629883\\
2.56529545783997 52.9877395629883\\
2.58740091323853 52.9877395629883\\
2.60950636863709 52.9877395629883\\
2.63161206245422 52.9877395629883\\
2.65371751785278 52.9877395629883\\
2.67582321166992 52.9877395629883\\
2.69792866706848 52.9877395629883\\
2.72003412246704 52.9877395629883\\
2.74213981628418 52.9877395629883\\
2.76424527168274 52.9877395629883\\
2.78635096549988 52.9877395629883\\
2.80845642089844 52.9877395629883\\
2.830561876297 52.9877395629883\\
2.85266757011414 52.9877395629883\\
2.8747730255127 52.9877395629883\\
2.89687871932983 52.9877395629883\\
2.91898417472839 52.9877395629883\\
2.94108963012695 52.9877395629883\\
2.96319532394409 52.9877395629883\\
2.98530077934265 52.9877395629883\\
3.00740647315979 52.9877395629883\\
3.02951192855835 52.9877395629883\\
3.05169749259949 52.9877395629883\\
3.07412838935852 52.9877395629883\\
3.09681701660156 52.9877395629883\\
3.11977505683899 52.9877395629883\\
3.14301300048828 52.9877395629883\\
3.16654181480408 52.9877395629883\\
3.19037246704102 52.9877395629883\\
3.21451687812805 52.9877395629883\\
3.23898792266846 52.9877395629883\\
3.26379895210266 52.9877395629883\\
3.28896427154541 52.9877395629883\\
3.31449961662292 52.9877395629883\\
3.34042167663574 52.9877395629883\\
3.36674857139587 52.9877395629883\\
3.39349937438965 52.9877395629883\\
3.42069554328918 52.9877395629883\\
3.4483597278595 52.9877395629883\\
3.47651672363281 52.9877395629883\\
3.50519371032715 52.9877395629883\\
3.53442049026489 52.9877395629883\\
3.56422996520996 52.9877395629883\\
3.59465765953064 52.9877395629883\\
3.6257438659668 52.9877395629883\\
3.65753245353699 52.9877395629883\\
3.69007301330566 52.9877395629883\\
3.72342085838318 52.9877395629883\\
3.75763893127441 52.9877395629883\\
3.79279804229736 52.9877395629883\\
3.82897973060608 52.9877395629883\\
3.86627721786499 52.9877395629883\\
3.90479946136475 52.9877395629883\\
3.94467377662659 52.9877395629883\\
3.9860508441925 52.9877395629883\\
4.02911138534546 52.9877395629883\\
4.07284307479858 52.9877395629883\\
4.11683511734009 52.9877395629883\\
4.16100072860718 52.9877395629883\\
4.20520448684692 52.9877395629883\\
4.24941444396973 52.9877395629883\\
4.29362487792969 52.9877395629883\\
4.33783578872681 52.9877395629883\\
4.38204622268677 52.9877395629883\\
4.42625665664673 52.9877395629883\\
4.47046709060669 52.9877395629883\\
4.51467752456665 52.9877395629883\\
4.55888795852661 52.9877395629883\\
4.60309839248657 52.9877395629883\\
4.64730882644653 52.9877395629883\\
4.69151973724365 52.9877395629883\\
4.73573064804077 52.9877395629883\\
4.77994155883789 52.9877395629883\\
4.82415246963501 52.9877395629883\\
4.86836338043213 52.9877395629883\\
4.91257381439209 52.9877395629883\\
4.95678472518921 52.9877395629883\\
5.00099563598633 52.9877395629883\\
5.04520654678345 52.9877395629883\\
5.08941698074341 52.9877395629883\\
5.13362789154053 52.9877395629883\\
5.17783832550049 52.9877395629883\\
5.22204923629761 52.9877395629883\\
5.26626014709473 52.9877395629883\\
5.31047105789185 52.9877395629883\\
5.35468196868896 52.9877395629883\\
5.39889240264893 52.9877395629883\\
5.44310331344604 52.9877395629883\\
5.48731422424316 52.9877395629883\\
5.53152513504028 52.9877395629883\\
5.5757360458374 52.9877395629883\\
5.61994695663452 52.9877395629883\\
5.66415786743164 52.9877395629883\\
5.7083683013916 52.9877395629883\\
5.75257921218872 52.9877395629883\\
5.79679012298584 52.9877395629883\\
5.84100103378296 52.9877395629883\\
5.88521194458008 52.9877395629883\\
5.9294228553772 52.9877395629883\\
5.97363376617432 52.9877395629883\\
6.01784420013428 52.9877395629883\\
6.0620551109314 52.9877395629883\\
6.10626602172852 52.9877395629883\\
6.15047693252563 52.9877395629883\\
6.19468784332275 52.9877395629883\\
6.23889875411987 52.9877395629883\\
6.28310966491699 52.9877395629883\\
6.32732009887695 52.9877395629883\\
6.37153100967407 52.9877395629883\\
6.41574192047119 52.9877395629883\\
6.45995283126831 52.9877395629883\\
6.50416374206543 52.9877395629883\\
6.54837465286255 52.9877395629883\\
6.59258508682251 52.9877395629883\\
6.63679599761963 52.9877395629883\\
6.68100690841675 52.9877395629883\\
6.72521781921387 52.9877395629883\\
6.76942873001099 52.9877395629883\\
6.81363964080811 52.9877395629883\\
6.85785055160522 52.9877395629883\\
6.90206098556519 52.9877395629883\\
6.9462718963623 52.9877395629883\\
6.99048280715942 52.9877395629883\\
7.03469371795654 52.9877395629883\\
7.07890462875366 52.9877395629883\\
7.12311553955078 52.9877395629883\\
7.1673264503479 52.9877395629883\\
7.21153688430786 52.9877395629883\\
7.25574779510498 52.9877395629883\\
7.2999587059021 52.9877395629883\\
7.34416961669922 52.9877395629883\\
7.38838052749634 52.9877395629883\\
7.43259143829346 52.9877395629883\\
7.47680234909058 52.9877395629883\\
7.52101278305054 52.9877395629883\\
7.56522369384766 52.9877395629883\\
7.60943460464478 52.9877395629883\\
7.65364551544189 52.9877395629883\\
7.69785642623901 52.9877395629883\\
7.74206733703613 52.9877395629883\\
7.78627824783325 52.9877395629883\\
7.83048868179321 52.9877395629883\\
7.87469959259033 52.9877395629883\\
7.91891050338745 52.9877395629883\\
7.96312141418457 52.9877395629883\\
8.00733184814453 52.9877395629883\\
8.05154323577881 52.9877395629883\\
8.09575366973877 52.9877395629883\\
8.13996505737305 52.9877395629883\\
8.18417549133301 52.9877395629883\\
8.22838687896729 52.9877395629883\\
8.27259731292725 52.9877395629883\\
8.31680774688721 52.9877395629883\\
8.36101913452148 52.9877395629883\\
8.40522956848145 52.9877395629883\\
8.44944095611572 52.9877395629883\\
8.49365139007568 52.9877395629883\\
8.53786277770996 52.9877395629883\\
8.58207321166992 52.9877395629883\\
8.62628364562988 52.9877395629883\\
8.67049503326416 52.9877395629883\\
8.71470546722412 52.9877395629883\\
8.7589168548584 52.9877395629883\\
8.80312728881836 52.9877395629883\\
8.84733867645264 52.9877395629883\\
8.8915491104126 52.9877395629883\\
8.93575954437256 52.9877395629883\\
8.97997093200684 52.9877395629883\\
9.0241813659668 52.9877395629883\\
9.06839275360107 52.9877395629883\\
9.11260318756104 52.9877395629883\\
9.156813621521 52.9877395629883\\
9.20102500915527 52.9877395629883\\
9.24523544311523 52.9877395629883\\
9.28944683074951 52.9877395629883\\
9.33365726470947 52.9877395629883\\
9.37786865234375 52.9877395629883\\
9.42207908630371 52.9877395629883\\
9.46628952026367 52.9877395629883\\
9.51050090789795 52.9877395629883\\
9.55471134185791 52.9877395629883\\
9.59892272949219 52.9877395629883\\
9.64313316345215 52.9877395629883\\
9.68734455108643 52.9877395629883\\
9.73155498504639 52.9877395629883\\
9.77576541900635 52.9877395629883\\
9.81997680664063 52.9877395629883\\
9.86418724060059 52.9877395629883\\
9.90839862823486 52.9877395629883\\
9.95260906219482 52.9877395629883\\
9.9968204498291 52.9877395629883\\
10.0410308837891 52.9877395629883\\
10.085241317749 52.9877395629883\\
10.1294527053833 52.9877395629883\\
10.1736631393433 52.9877395629883\\
10.2178745269775 52.9877395629883\\
10.2620849609375 52.9877395629883\\
10.3062963485718 52.9877395629883\\
10.3505067825317 52.9877395629883\\
10.3947172164917 52.9877395629883\\
10.438928604126 52.9877395629883\\
10.4831390380859 52.9877395629883\\
10.5273504257202 52.9877395629883\\
10.5715608596802 52.9877395629883\\
10.6157722473145 52.9877395629883\\
10.6599826812744 52.9877395629883\\
10.7041931152344 52.9877395629883\\
10.7484045028687 52.9877395629883\\
10.7926149368286 52.9877395629883\\
10.8368263244629 52.9877395629883\\
10.8810367584229 52.9877395629883\\
10.9252481460571 52.9877395629883\\
10.9694585800171 52.9877395629883\\
11.0136690139771 52.9877395629883\\
11.0578804016113 52.9877395629883\\
11.1020908355713 52.9877395629883\\
11.1463022232056 52.9877395629883\\
11.1905126571655 52.9877395629883\\
11.2347240447998 52.9877395629883\\
11.2789344787598 52.9877395629883\\
11.3231449127197 52.9877395629883\\
11.367356300354 52.9877395629883\\
11.411566734314 52.9877395629883\\
11.4557781219482 52.9877395629883\\
11.4778890609741 52.9877395629883\\
11.5 52.9877395629883\\
11.5243225097656 52.9877395629883\\
11.5510768890381 52.9877395629883\\
11.5805063247681 52.9877395629883\\
11.6128797531128 52.9877395629883\\
11.6484899520874 52.9877395629883\\
11.6876602172852 52.9877395629883\\
11.730749130249 52.9877395629883\\
11.774959564209 52.9877395629883\\
11.8191709518433 52.9877395629883\\
11.8633813858032 52.9877395629883\\
11.9075918197632 52.9877395629883\\
11.9518032073975 52.9877395629883\\
11.9960136413574 52.9877395629883\\
12.0402250289917 52.9877395629883\\
12.0844354629517 52.9877395629883\\
12.1286468505859 52.9877395629883\\
12.1728572845459 52.9877395629883\\
12.2170677185059 52.9877395629883\\
12.2612791061401 52.9877395629883\\
12.3054895401001 52.9877395629883\\
12.3497009277344 52.9877395629883\\
12.3939113616943 52.9877395629883\\
12.4381227493286 52.9877395629883\\
12.4823331832886 52.9877395629883\\
12.5265436172485 52.9877395629883\\
12.5707550048828 52.9877395629883\\
12.6149654388428 52.9877395629883\\
12.6591768264771 52.9877395629883\\
12.703387260437 52.9877395629883\\
12.7475986480713 52.9877395629883\\
12.7918090820313 52.9877395629883\\
12.8360195159912 52.9877395629883\\
12.8802309036255 52.9877395629883\\
12.9244413375854 52.9877395629883\\
12.9686527252197 52.9877395629883\\
13.0128631591797 52.9877395629883\\
13.057074546814 52.9877395629883\\
13.1012849807739 52.9877395629883\\
13.1454954147339 52.9877395629883\\
13.1897068023682 52.9877395629883\\
13.2339172363281 52.9877395629883\\
13.2781286239624 52.9877395629883\\
13.3223390579224 52.9877395629883\\
13.3665504455566 52.9877395629883\\
13.4107608795166 52.9877395629883\\
13.4549713134766 52.9877395629883\\
13.4991827011108 52.9877395629883\\
13.5433931350708 52.9877395629883\\
13.5876045227051 52.9877395629883\\
13.631814956665 52.9877395629883\\
13.6760263442993 52.9877395629883\\
13.7202367782593 52.9877395629883\\
13.7644472122192 52.9877395629883\\
13.8086585998535 52.9877395629883\\
13.8528690338135 52.9877395629883\\
13.8970804214478 52.9877395629883\\
13.9412908554077 52.9877395629883\\
13.985502243042 52.9877395629883\\
14.029712677002 52.9877395629883\\
14.0739231109619 52.9877395629883\\
14.1181344985962 52.9877395629883\\
14.1623449325562 52.9877395629883\\
14.2065563201904 52.9877395629883\\
14.2507667541504 52.9877395629883\\
14.2949781417847 52.9877395629883\\
14.3391885757446 52.9877395629883\\
14.3833990097046 52.9877395629883\\
14.4276103973389 52.9877395629883\\
14.4718208312988 52.9877395629883\\
14.5160322189331 52.9877395629883\\
14.5602426528931 52.9877395629883\\
14.6044540405273 52.9877395629883\\
14.6486644744873 52.9877395629883\\
14.6928749084473 52.9877395629883\\
14.7370862960815 52.9877395629883\\
14.7812967300415 52.9877395629883\\
14.8255081176758 52.9877395629883\\
14.8697185516357 52.9877395629883\\
14.9139289855957 52.9877395629883\\
14.95814037323 52.9877395629883\\
15.0023508071899 52.9877395629883\\
15.0465621948242 52.9877395629883\\
15.0907726287842 52.9877395629883\\
15.1349840164185 52.9877395629883\\
15.1791944503784 52.9877395629883\\
15.2234048843384 52.9877395629883\\
15.2676162719727 52.9877395629883\\
15.3118267059326 52.9877395629883\\
15.3560380935669 52.9877395629883\\
15.4002485275269 52.9877395629883\\
15.4444599151611 52.9877395629883\\
15.4886703491211 52.9877395629883\\
15.5328807830811 52.9877395629883\\
15.5770921707153 52.9877395629883\\
15.6213026046753 52.9877395629883\\
15.6655139923096 52.9877395629883\\
15.7097244262695 52.9877395629883\\
15.7539358139038 52.9877395629883\\
15.7981462478638 52.9877395629883\\
15.8423566818237 52.9877395629883\\
15.886568069458 52.9877395629883\\
15.930778503418 52.9877395629883\\
15.9749898910522 52.9877395629883\\
16.0192012786865 52.9877395629883\\
16.0630474090576 52.9877395629883\\
16.1063537597656 52.9877395629883\\
16.1503849029541 52.9877395629883\\
16.1945571899414 52.9877395629883\\
16.2387599945068 52.9877395629883\\
16.2829685211182 52.9877395629883\\
16.3271789550781 52.9877395629883\\
16.3713893890381 52.9877395629883\\
16.4156017303467 52.9877395629883\\
16.4598121643066 52.9877395629883\\
16.5040225982666 52.9877395629883\\
16.5482330322266 52.9877395629883\\
16.5924434661865 52.9877395629883\\
16.6366558074951 52.9877395629883\\
16.6808662414551 52.9877395629883\\
16.725076675415 52.9877395629883\\
16.769287109375 52.9877395629883\\
16.813497543335 52.9877395629883\\
16.8577079772949 52.9877395629883\\
16.9019184112549 52.9877395629883\\
16.9461288452148 52.9877395629883\\
16.9903411865234 52.9877395629883\\
17.0345516204834 52.9877395629883\\
17.0787620544434 52.9877395629883\\
17.1229724884033 52.9877395629883\\
17.1671829223633 52.9877395629883\\
17.2113952636719 52.9877395629883\\
17.2556056976318 52.9877395629883\\
17.2998161315918 52.9877395629883\\
17.3440265655518 52.9877395629883\\
17.3882369995117 52.9877395629883\\
17.4324493408203 52.9877395629883\\
17.4766597747803 52.9877395629883\\
17.5208702087402 52.9877395629883\\
17.5650806427002 52.9877395629883\\
17.6092929840088 52.9877395629883\\
17.6535034179688 52.9877395629883\\
17.6977138519287 52.9877395629883\\
17.7419242858887 52.9877395629883\\
17.7861347198486 52.9877395629883\\
17.8303470611572 52.9877395629883\\
17.8745574951172 52.9877395629883\\
17.9187679290771 52.9877395629883\\
17.9629783630371 52.9877395629883\\
18.0071887969971 52.9877395629883\\
18.0514011383057 52.9877395629883\\
18.0956115722656 52.9877395629883\\
18.1398220062256 52.9877395629883\\
18.1840324401855 52.9877395629883\\
18.2282447814941 52.9877395629883\\
18.2724552154541 52.9877395629883\\
18.3166656494141 52.9877395629883\\
18.360876083374 52.9877395629883\\
18.405086517334 52.9877395629883\\
18.4492988586426 52.9877395629883\\
18.4935092926025 52.9877395629883\\
18.5377197265625 52.9877395629883\\
18.5819301605225 52.9877395629883\\
18.6261405944824 52.9877395629883\\
18.670352935791 52.9877395629883\\
18.714563369751 52.9877395629883\\
18.7587738037109 52.9877395629883\\
18.8029842376709 52.9877395629883\\
18.8471965789795 52.9877395629883\\
18.8914070129395 52.9877395629883\\
18.9356174468994 52.9877395629883\\
18.9678077697754 52.9877395629883\\
19 52.9877395629883\\
};
\addlegendentry{$A_{r} = 0.50$};

\addplot [
color=black,
dashed
]
table[row sep=crcr]{
0 0\\
9.99999974737875e-06 7.34242400213247e-09\\
2.49999993684469e-05 6.82759377923503e-07\\
4.75000015285332e-05 1.61528696480673e-05\\
8.12500002211891e-05 0.000229467026656494\\
0.000131875000079162 0.00192556483671069\\
0.000207812496228144 0.00898817554116249\\
0.000321718747727573 0.0234723221510649\\
0.000492578139528632 0.0383452847599983\\
0.00074886716902256 0.0502648875117302\\
0.00113330082967877 0.070481926202774\\
0.00170995120424777 0.108498722314835\\
0.00257492670789361 0.167652875185013\\
0.00387239013798535 0.258626192808151\\
0.00529959984123707 0.362792581319809\\
0.00686953077092767 0.482348293066025\\
0.0085964547470212 0.619613349437714\\
0.0104960706084967 0.777237176895142\\
0.0125856483355165 0.958173990249634\\
0.014884184114635 1.16564691066742\\
0.0174125730991364 1.40307796001434\\
0.020193800330162 1.67395949363709\\
0.0232531521469355 1.98164916038513\\
0.0266184378415346 2.32907176017761\\
0.0303202513605356 2.71832084655762\\
0.0343922488391399 3.15018105506897\\
0.0388714447617531 3.62361979484558\\
0.0437985584139824 4.13536167144775\\
0.0492183864116669 4.67968845367432\\
0.0551801957190037 5.24862861633301\\
0.0617381855845451 5.83265113830566\\
0.0689519718289375 6.42182493209839\\
0.0768871381878853 7.00725269317627\\
0.0856158286333084 7.58241939544678\\
0.095217376947403 8.14408874511719\\
0.105779089033604 8.6925163269043\\
0.117396965622902 9.23096466064453\\
0.13017663359642 9.76474666595459\\
0.144234269857407 10.3001012802124\\
0.159697666764259 10.8432216644287\\
0.176707401871681 11.3996133804321\\
0.195418119430542 11.9738235473633\\
0.215999901294708 12.56946849823\\
0.238639861345291 13.1894273757935\\
0.2635438144207 13.8360843658447\\
0.290938168764114 14.5115346908569\\
0.321071952581406 15.2177352905273\\
0.354219108819962 15.9565982818604\\
0.390680998563766 16.730037689209\\
0.427072286605835 17.4643230438232\\
0.461725652217865 18.1338748931885\\
0.494970262050629 18.7532711029053\\
0.527021527290344 19.3314189910889\\
0.55803918838501 19.8747310638428\\
0.588148653507233 20.3881320953369\\
0.617450594902039 20.8755073547363\\
0.646027386188507 21.3399829864502\\
0.67394757270813 21.7841186523438\\
0.701269030570984 22.2100391387939\\
0.728040993213654 22.6195373535156\\
0.754306197166443 23.0141296386719\\
0.780101537704468 23.3951263427734\\
0.805459678173065 23.7636528015137\\
0.830409228801727 24.1206970214844\\
0.854975879192352 24.4671268463135\\
0.8791823387146 24.8037052154541\\
0.903049230575562 25.1311111450195\\
0.926595151424408 25.4499549865723\\
0.949836909770966 25.760778427124\\
0.972789883613586 26.0640754699707\\
0.995468020439148 26.3602886199951\\
1.01788425445557 26.4663467407227\\
1.0400505065918 26.4881057739258\\
1.06216442584991 26.4926433563232\\
1.08427441120148 26.4935989379883\\
1.10638225078583 26.4938049316406\\
1.12848901748657 26.4938545227051\\
1.15059518814087 26.4938697814941\\
1.1727010011673 26.4938774108887\\
1.19480681419373 26.4938831329346\\
1.21691238880157 26.4938888549805\\
1.23901796340942 26.4938926696777\\
1.26112222671509 26.493896484375\\
1.28322517871857 26.4938983917236\\
1.30532777309418 26.4939002990723\\
1.32742953300476 26.4939041137695\\
1.34953081607819 26.4939060211182\\
1.37163174152374 26.4939060211182\\
1.39373195171356 26.4939079284668\\
1.41583180427551 26.4939079284668\\
1.43793165683746 26.4939098358154\\
1.46003234386444 26.4939098358154\\
1.48213422298431 26.4939098358154\\
1.50423765182495 26.4939098358154\\
1.52634274959564 26.4939098358154\\
1.5484482049942 26.4939098358154\\
1.57055377960205 26.4939098358154\\
1.5926593542099 26.4939098358154\\
1.61476469039917 26.4939098358154\\
1.63686990737915 26.4939098358154\\
1.65897512435913 26.4939098358154\\
1.68108010292053 26.4939098358154\\
1.70318508148193 26.4939098358154\\
1.72528982162476 26.4939098358154\\
1.74739444255829 26.4939098358154\\
1.76949906349182 26.4939098358154\\
1.79160356521606 26.4939098358154\\
1.81370830535889 26.4939098358154\\
1.83581340312958 26.4939098358154\\
1.85791862010956 26.4939098358154\\
1.88002419471741 26.4939098358154\\
1.90212976932526 26.4939098358154\\
1.92423534393311 26.4939098358154\\
1.94634091854095 26.4939098358154\\
1.9684464931488 26.4939098358154\\
1.99055194854736 26.4939098358154\\
2.01265740394592 26.4939098358154\\
2.03476285934448 26.4939098358154\\
2.05686831474304 26.4939079284668\\
2.0789737701416 26.4939079284668\\
2.10107922554016 26.4939079284668\\
2.12318468093872 26.4939079284668\\
2.14529013633728 26.4939079284668\\
2.16739559173584 26.4939079284668\\
2.1895010471344 26.4939079284668\\
2.21160674095154 26.4939079284668\\
2.2337121963501 26.4939079284668\\
2.25581765174866 26.4939079284668\\
2.2779233455658 26.4939079284668\\
2.30002880096436 26.4939079284668\\
2.32213449478149 26.4939079284668\\
2.34423995018005 26.4939079284668\\
2.36634540557861 26.4939079284668\\
2.38845109939575 26.4939079284668\\
2.41055655479431 26.4939079284668\\
2.43266201019287 26.4939079284668\\
2.45476770401001 26.4939079284668\\
2.47687315940857 26.4939079284668\\
2.49897861480713 26.4939079284668\\
2.52108430862427 26.4939079284668\\
2.54318976402283 26.4939079284668\\
2.56529545783997 26.4939079284668\\
2.58740091323853 26.4939079284668\\
2.60950636863709 26.4939079284668\\
2.63161206245422 26.4939079284668\\
2.65371751785278 26.4939079284668\\
2.67582321166992 26.4939079284668\\
2.69792866706848 26.4939079284668\\
2.72003412246704 26.4939079284668\\
2.74213981628418 26.4939079284668\\
2.76424527168274 26.4939079284668\\
2.78635096549988 26.4939079284668\\
2.80845642089844 26.4939079284668\\
2.830561876297 26.4939079284668\\
2.85266757011414 26.4939079284668\\
2.8747730255127 26.4939079284668\\
2.89687871932983 26.4939079284668\\
2.91898417472839 26.4939079284668\\
2.94108963012695 26.4939079284668\\
2.96319532394409 26.4939079284668\\
2.98530077934265 26.4939079284668\\
3.00740647315979 26.4939079284668\\
3.02951192855835 26.4939079284668\\
3.05169749259949 26.4939079284668\\
3.07412838935852 26.4939079284668\\
3.09681701660156 26.4939079284668\\
3.11977505683899 26.4939079284668\\
3.14301300048828 26.4939079284668\\
3.16654181480408 26.4939079284668\\
3.19037246704102 26.4939079284668\\
3.21451687812805 26.4939079284668\\
3.23898792266846 26.4939079284668\\
3.26379895210266 26.4939079284668\\
3.28896427154541 26.4939079284668\\
3.31449961662292 26.4939079284668\\
3.34042167663574 26.4939079284668\\
3.36674857139587 26.4939079284668\\
3.39349937438965 26.4939079284668\\
3.42069554328918 26.4939079284668\\
3.4483597278595 26.4939079284668\\
3.47651672363281 26.4939079284668\\
3.50519371032715 26.4939079284668\\
3.53442049026489 26.4939079284668\\
3.56422996520996 26.4939079284668\\
3.59465765953064 26.4939079284668\\
3.6257438659668 26.4939079284668\\
3.65753245353699 26.4939079284668\\
3.69007301330566 26.4939079284668\\
3.72342085838318 26.4939079284668\\
3.75763893127441 26.4939079284668\\
3.79279804229736 26.4939079284668\\
3.82897973060608 26.4939079284668\\
3.86627721786499 26.4939079284668\\
3.90479946136475 26.4939079284668\\
3.94467377662659 26.4939079284668\\
3.9860508441925 26.4939079284668\\
4.02911138534546 26.4939079284668\\
4.07284307479858 26.4939079284668\\
4.11683511734009 26.4939079284668\\
4.16100072860718 26.4939079284668\\
4.20520448684692 26.4939079284668\\
4.24941444396973 26.4939079284668\\
4.29362487792969 26.4939079284668\\
4.33783578872681 26.4939079284668\\
4.38204622268677 26.4939079284668\\
4.42625665664673 26.4939079284668\\
4.47046709060669 26.4939079284668\\
4.51467752456665 26.4939079284668\\
4.55888795852661 26.4939079284668\\
4.60309839248657 26.4939079284668\\
4.64730882644653 26.4939079284668\\
4.69151973724365 26.4939079284668\\
4.73573064804077 26.4939079284668\\
4.77994155883789 26.4939079284668\\
4.82415246963501 26.4939079284668\\
4.86836338043213 26.4939079284668\\
4.91257381439209 26.4939079284668\\
4.95678472518921 26.4939079284668\\
5.00099563598633 26.4939079284668\\
5.04520654678345 26.4939079284668\\
5.08941698074341 26.4939079284668\\
5.13362789154053 26.4939079284668\\
5.17783832550049 26.4939079284668\\
5.22204923629761 26.4939079284668\\
5.26626014709473 26.4939079284668\\
5.31047105789185 26.4939079284668\\
5.35468196868896 26.4939079284668\\
5.39889240264893 26.4939079284668\\
5.44310331344604 26.4939079284668\\
5.48731422424316 26.4939079284668\\
5.53152513504028 26.4939079284668\\
5.5757360458374 26.4939079284668\\
5.61994695663452 26.4939079284668\\
5.66415786743164 26.4939079284668\\
5.7083683013916 26.4939079284668\\
5.75257921218872 26.4939079284668\\
5.79679012298584 26.4939079284668\\
5.84100103378296 26.4939079284668\\
5.88521194458008 26.4939079284668\\
5.9294228553772 26.4939079284668\\
5.97363376617432 26.4939079284668\\
6.01784420013428 26.4939079284668\\
6.0620551109314 26.4939079284668\\
6.10626602172852 26.4939079284668\\
6.15047693252563 26.4939079284668\\
6.19468784332275 26.4939079284668\\
6.23889875411987 26.4939079284668\\
6.28310966491699 26.4939079284668\\
6.32732009887695 26.4939079284668\\
6.37153100967407 26.4939079284668\\
6.41574192047119 26.4939079284668\\
6.45995283126831 26.4939079284668\\
6.50416374206543 26.4939079284668\\
6.54837465286255 26.4939079284668\\
6.59258508682251 26.4939079284668\\
6.63679599761963 26.4939079284668\\
6.68100690841675 26.4939079284668\\
6.72521781921387 26.4939079284668\\
6.76942873001099 26.4939079284668\\
6.81363964080811 26.4939079284668\\
6.85785055160522 26.4939079284668\\
6.90206098556519 26.4939079284668\\
6.9462718963623 26.4939079284668\\
6.99048280715942 26.4939079284668\\
7.03469371795654 26.4939079284668\\
7.07890462875366 26.4939079284668\\
7.12311553955078 26.4939079284668\\
7.1673264503479 26.4939079284668\\
7.21153688430786 26.4939079284668\\
7.25574779510498 26.4939079284668\\
7.2999587059021 26.4939079284668\\
7.34416961669922 26.4939079284668\\
7.38838052749634 26.4939079284668\\
7.43259143829346 26.4939079284668\\
7.47680234909058 26.4939079284668\\
7.52101278305054 26.4939079284668\\
7.56522369384766 26.4939079284668\\
7.60943460464478 26.4939079284668\\
7.65364551544189 26.4939079284668\\
7.69785642623901 26.4939079284668\\
7.74206733703613 26.4939079284668\\
7.78627824783325 26.4939079284668\\
7.83048868179321 26.4939079284668\\
7.87469959259033 26.4939079284668\\
7.91891050338745 26.4939079284668\\
7.96312141418457 26.4939079284668\\
8.00733184814453 26.4939079284668\\
8.05154323577881 26.4939079284668\\
8.09575366973877 26.4939079284668\\
8.13996505737305 26.4939079284668\\
8.18417549133301 26.4939079284668\\
8.22838687896729 26.4939079284668\\
8.27259731292725 26.4939079284668\\
8.31680774688721 26.4939079284668\\
8.36101913452148 26.4939079284668\\
8.40522956848145 26.4939079284668\\
8.44944095611572 26.4939079284668\\
8.49365139007568 26.4939079284668\\
8.53786277770996 26.4939079284668\\
8.58207321166992 26.4939079284668\\
8.62628364562988 26.4939079284668\\
8.67049503326416 26.4939079284668\\
8.71470546722412 26.4939079284668\\
8.7589168548584 26.4939079284668\\
8.80312728881836 26.4939079284668\\
8.84733867645264 26.4939079284668\\
8.8915491104126 26.4939079284668\\
8.93575954437256 26.4939079284668\\
8.97997093200684 26.4939079284668\\
9.0241813659668 26.4939079284668\\
9.06839275360107 26.4939079284668\\
9.11260318756104 26.4939079284668\\
9.156813621521 26.4939079284668\\
9.20102500915527 26.4939079284668\\
9.24523544311523 26.4939079284668\\
9.28944683074951 26.4939079284668\\
9.33365726470947 26.4939079284668\\
9.37786865234375 26.4939079284668\\
9.42207908630371 26.4939079284668\\
9.46628952026367 26.4939079284668\\
9.51050090789795 26.4939079284668\\
9.55471134185791 26.4939079284668\\
9.59892272949219 26.4939079284668\\
9.64313316345215 26.4939079284668\\
9.68734455108643 26.4939079284668\\
9.73155498504639 26.4939079284668\\
9.77576541900635 26.4939079284668\\
9.81997680664063 26.4939079284668\\
9.86418724060059 26.4939079284668\\
9.90839862823486 26.4939079284668\\
9.95260906219482 26.4939079284668\\
9.9968204498291 26.4939079284668\\
10.0410308837891 26.4939079284668\\
10.085241317749 26.4939079284668\\
10.1294527053833 26.4939079284668\\
10.1736631393433 26.4939079284668\\
10.2178745269775 26.4939079284668\\
10.2620849609375 26.4939079284668\\
10.3062963485718 26.4939079284668\\
10.3505067825317 26.4939079284668\\
10.3947172164917 26.4939079284668\\
10.438928604126 26.4939079284668\\
10.4831390380859 26.4939079284668\\
10.5273504257202 26.4939079284668\\
10.5715608596802 26.4939079284668\\
10.6157722473145 26.4939079284668\\
10.6599826812744 26.4939079284668\\
10.7041931152344 26.4939079284668\\
10.7484045028687 26.4939079284668\\
10.7926149368286 26.4939079284668\\
10.8368263244629 26.4939079284668\\
10.8810367584229 26.4939079284668\\
10.9252481460571 26.4939079284668\\
10.9694585800171 26.4939079284668\\
11.0136690139771 26.4939079284668\\
11.0578804016113 26.4939079284668\\
11.1020908355713 26.4939079284668\\
11.1463022232056 26.4939079284668\\
11.1905126571655 26.4939079284668\\
11.2347240447998 26.4939079284668\\
11.2789344787598 26.4939079284668\\
11.3231449127197 26.4939079284668\\
11.367356300354 26.4939079284668\\
11.411566734314 26.4939079284668\\
11.4557781219482 26.4939079284668\\
11.4778890609741 26.4939079284668\\
11.5 26.4939079284668\\
11.5243225097656 26.4939079284668\\
11.5510768890381 26.4939079284668\\
11.5805063247681 26.4939079284668\\
11.6128797531128 26.4939079284668\\
11.6484899520874 26.4939079284668\\
11.6876602172852 26.4939079284668\\
11.730749130249 26.4939079284668\\
11.774959564209 26.4939079284668\\
11.8191709518433 26.4939079284668\\
11.8633813858032 26.4939079284668\\
11.9075918197632 26.4939079284668\\
11.9518032073975 26.4939079284668\\
11.9960136413574 26.4939079284668\\
12.0402250289917 26.4939079284668\\
12.0844354629517 26.4939079284668\\
12.1286468505859 26.4939079284668\\
12.1728572845459 26.4939079284668\\
12.2170677185059 26.4939079284668\\
12.2612791061401 26.4939079284668\\
12.3054895401001 26.4939079284668\\
12.3497009277344 26.4939079284668\\
12.3939113616943 26.4939079284668\\
12.4381227493286 26.4939079284668\\
12.4823331832886 26.4939079284668\\
12.5265436172485 26.4939079284668\\
12.5707550048828 26.4939079284668\\
12.6149654388428 26.4939079284668\\
12.6591768264771 26.4939079284668\\
12.703387260437 26.4939079284668\\
12.7475986480713 26.4939079284668\\
12.7918090820313 26.4939079284668\\
12.8360195159912 26.4939079284668\\
12.8802309036255 26.4939079284668\\
12.9244413375854 26.4939079284668\\
12.9686527252197 26.4939079284668\\
13.0128631591797 26.4939079284668\\
13.057074546814 26.4939079284668\\
13.1012849807739 26.4939079284668\\
13.1454954147339 26.4939079284668\\
13.1897068023682 26.4939079284668\\
13.2339172363281 26.4939079284668\\
13.2781286239624 26.4939079284668\\
13.3223390579224 26.4939079284668\\
13.3665504455566 26.4939079284668\\
13.4107608795166 26.4939079284668\\
13.4549713134766 26.4939079284668\\
13.4991827011108 26.4939079284668\\
13.5433931350708 26.4939079284668\\
13.5876045227051 26.4939079284668\\
13.631814956665 26.4939079284668\\
13.6760263442993 26.4939079284668\\
13.7202367782593 26.4939079284668\\
13.7644472122192 26.4939079284668\\
13.8086585998535 26.4939079284668\\
13.8528690338135 26.4939079284668\\
13.8970804214478 26.4939079284668\\
13.9412908554077 26.4939079284668\\
13.985502243042 26.4939079284668\\
14.029712677002 26.4939079284668\\
14.0739231109619 26.4939079284668\\
14.1181344985962 26.4939079284668\\
14.1623449325562 26.4939079284668\\
14.2065563201904 26.4939079284668\\
14.2507667541504 26.4939079284668\\
14.2949781417847 26.4939079284668\\
14.3391885757446 26.4939079284668\\
14.3833990097046 26.4939079284668\\
14.4276103973389 26.4939079284668\\
14.4718208312988 26.4939079284668\\
14.5160322189331 26.4939079284668\\
14.5602426528931 26.4939079284668\\
14.6044540405273 26.4939079284668\\
14.6486644744873 26.4939079284668\\
14.6928749084473 26.4939079284668\\
14.7370862960815 26.4939079284668\\
14.7812967300415 26.4939079284668\\
14.8255081176758 26.4939079284668\\
14.8697185516357 26.4939079284668\\
14.9139289855957 26.4939079284668\\
14.95814037323 26.4939079284668\\
15.0023508071899 26.4939079284668\\
15.0465621948242 26.4939079284668\\
15.0907726287842 26.4939079284668\\
15.1349840164185 26.4939079284668\\
15.1791944503784 26.4939079284668\\
15.2234048843384 26.4939079284668\\
15.2676162719727 26.4939079284668\\
15.3118267059326 26.4939079284668\\
15.3560380935669 26.4939079284668\\
15.4002485275269 26.4939079284668\\
15.4444599151611 26.4939079284668\\
15.4886703491211 26.4939079284668\\
15.5328807830811 26.4939079284668\\
15.5770921707153 26.4939079284668\\
15.6213026046753 26.4939079284668\\
15.6655139923096 26.4939079284668\\
15.7097244262695 26.4939079284668\\
15.7539358139038 26.4939079284668\\
15.7981462478638 26.4939079284668\\
15.8423566818237 26.4939079284668\\
15.886568069458 26.4939079284668\\
15.930778503418 26.4939079284668\\
15.9749898910522 26.4939079284668\\
16.0192012786865 26.4939079284668\\
16.0630474090576 26.4939079284668\\
16.1063537597656 26.4939079284668\\
16.1503849029541 26.4939079284668\\
16.1945571899414 26.4939079284668\\
16.2387599945068 26.4939079284668\\
16.2829685211182 26.4939079284668\\
16.3271789550781 26.4939079284668\\
16.3713893890381 26.4939079284668\\
16.4156017303467 26.4939079284668\\
16.4598121643066 26.4939079284668\\
16.5040225982666 26.4939079284668\\
16.5482330322266 26.4939079284668\\
16.5924434661865 26.4939079284668\\
16.6366558074951 26.4939079284668\\
16.6808662414551 26.4939079284668\\
16.725076675415 26.4939079284668\\
16.769287109375 26.4939079284668\\
16.813497543335 26.4939079284668\\
16.8577079772949 26.4939079284668\\
16.9019184112549 26.4939079284668\\
16.9461288452148 26.4939079284668\\
16.9903411865234 26.4939079284668\\
17.0345516204834 26.4939079284668\\
17.0787620544434 26.4939079284668\\
17.1229724884033 26.4939079284668\\
17.1671829223633 26.4939079284668\\
17.2113952636719 26.4939079284668\\
17.2556056976318 26.4939079284668\\
17.2998161315918 26.4939079284668\\
17.3440265655518 26.4939079284668\\
17.3882369995117 26.4939079284668\\
17.4324493408203 26.4939079284668\\
17.4766597747803 26.4939079284668\\
17.5208702087402 26.4939079284668\\
17.5650806427002 26.4939079284668\\
17.6092929840088 26.4939079284668\\
17.6535034179688 26.4939079284668\\
17.6977138519287 26.4939079284668\\
17.7419242858887 26.4939079284668\\
17.7861347198486 26.4939079284668\\
17.8303470611572 26.4939079284668\\
17.8745574951172 26.4939079284668\\
17.9187679290771 26.4939079284668\\
17.9629783630371 26.4939079284668\\
18.0071887969971 26.4939079284668\\
18.0514011383057 26.4939079284668\\
18.0956115722656 26.4939079284668\\
18.1398220062256 26.4939079284668\\
18.1840324401855 26.4939079284668\\
18.2282447814941 26.4939079284668\\
18.2724552154541 26.4939079284668\\
18.3166656494141 26.4939079284668\\
18.360876083374 26.4939079284668\\
18.405086517334 26.4939079284668\\
18.4492988586426 26.4939079284668\\
18.4935092926025 26.4939079284668\\
18.5377197265625 26.4939079284668\\
18.5819301605225 26.4939079284668\\
18.6261405944824 26.4939079284668\\
18.670352935791 26.4939079284668\\
18.714563369751 26.4939079284668\\
18.7587738037109 26.4939079284668\\
18.8029842376709 26.4939079284668\\
18.8471965789795 26.4939079284668\\
18.8914070129395 26.4939079284668\\
18.9356174468994 26.4939079284668\\
18.9678077697754 26.4939079284668\\
19 26.4939079284668\\
};
\addlegendentry{$A_{r} = 0.25$};

\addplot [
color=black,
dotted
]
table[row sep=crcr]{
0 0\\
9.99999974737875e-06 0\\
2.49999993684469e-05 0\\
4.75000015285332e-05 0\\
8.12500002211891e-05 0\\
0.000131875000079162 0\\
0.000207812496228144 0\\
0.000321718747727573 0\\
0.000492578139528632 0\\
0.00074886716902256 0\\
0.00113330082967877 0\\
0.00170995120424777 0\\
0.00257492670789361 0\\
0.00387239013798535 0\\
0.00529959984123707 0\\
0.00686953077092767 0\\
0.0085964547470212 0\\
0.0104960706084967 0\\
0.0125856483355165 0\\
0.014884184114635 0\\
0.0174125730991364 0\\
0.020193800330162 0\\
0.0232531521469355 0\\
0.0266184378415346 0\\
0.0303202513605356 0\\
0.0343922488391399 0\\
0.0388714447617531 0\\
0.0437985584139824 0\\
0.0492183864116669 0\\
0.0551801957190037 0\\
0.0617381855845451 0\\
0.0689519718289375 0\\
0.0768871381878853 0\\
0.0856158286333084 0\\
0.095217376947403 0\\
0.105779089033604 0\\
0.117396965622902 0\\
0.13017663359642 0\\
0.144234269857407 0\\
0.159697666764259 0\\
0.176707401871681 0\\
0.195418119430542 0\\
0.215999901294708 0\\
0.238639861345291 0\\
0.2635438144207 0\\
0.290938168764114 0\\
0.321071952581406 0\\
0.354219108819962 0\\
0.390680998563766 0\\
0.427072286605835 0\\
0.461725652217865 0\\
0.494970262050629 0\\
0.527021527290344 0\\
0.55803918838501 0\\
0.588148653507233 0\\
0.617450594902039 0\\
0.646027386188507 0\\
0.67394757270813 0\\
0.701269030570984 0\\
0.728040993213654 0\\
0.754306197166443 0\\
0.780101537704468 0\\
0.805459678173065 0\\
0.830409228801727 0\\
0.854975879192352 0\\
0.8791823387146 0\\
0.903049230575562 0\\
0.926595151424408 0\\
0.949836909770966 0\\
0.972789883613586 0\\
0.995468020439148 0\\
1.01788425445557 0\\
1.0400505065918 0\\
1.06216442584991 0\\
1.08427441120148 0\\
1.10638225078583 0\\
1.12848901748657 0\\
1.15059518814087 0\\
1.1727010011673 0\\
1.19480681419373 0\\
1.21691238880157 0\\
1.23901796340942 0\\
1.26112222671509 0\\
1.28322517871857 0\\
1.30532777309418 0\\
1.32742953300476 0\\
1.34953081607819 0\\
1.37163174152374 0\\
1.39373195171356 0\\
1.41583180427551 0\\
1.43793165683746 0\\
1.46003234386444 0\\
1.48213422298431 0\\
1.50423765182495 0\\
1.52634274959564 0\\
1.5484482049942 0\\
1.57055377960205 0\\
1.5926593542099 0\\
1.61476469039917 0\\
1.63686990737915 0\\
1.65897512435913 0\\
1.68108010292053 0\\
1.70318508148193 0\\
1.72528982162476 0\\
1.74739444255829 0\\
1.76949906349182 0\\
1.79160356521606 0\\
1.81370830535889 0\\
1.83581340312958 0\\
1.85791862010956 0\\
1.88002419471741 0\\
1.90212976932526 0\\
1.92423534393311 0\\
1.94634091854095 0\\
1.9684464931488 0\\
1.99055194854736 0\\
2.01265740394592 0\\
2.03476285934448 0\\
2.05686831474304 0\\
2.0789737701416 0\\
2.10107922554016 0\\
2.12318468093872 0\\
2.14529013633728 0\\
2.16739559173584 0\\
2.1895010471344 0\\
2.21160674095154 0\\
2.2337121963501 0\\
2.25581765174866 0\\
2.2779233455658 0\\
2.30002880096436 0\\
2.32213449478149 0\\
2.34423995018005 0\\
2.36634540557861 0\\
2.38845109939575 0\\
2.41055655479431 0\\
2.43266201019287 0\\
2.45476770401001 0\\
2.47687315940857 0\\
2.49897861480713 0\\
2.52108430862427 0\\
2.54318976402283 0\\
2.56529545783997 0\\
2.58740091323853 0\\
2.60950636863709 0\\
2.63161206245422 0\\
2.65371751785278 0\\
2.67582321166992 0\\
2.69792866706848 0\\
2.72003412246704 0\\
2.74213981628418 0\\
2.76424527168274 0\\
2.78635096549988 0\\
2.80845642089844 0\\
2.830561876297 0\\
2.85266757011414 0\\
2.8747730255127 0\\
2.89687871932983 0\\
2.91898417472839 0\\
2.94108963012695 0\\
2.96319532394409 0\\
2.98530077934265 0\\
3.00740647315979 0\\
3.02951192855835 0\\
3.05169749259949 0\\
3.07412838935852 0\\
3.09681701660156 0\\
3.11977505683899 0\\
3.14301300048828 0\\
3.16654181480408 0\\
3.19037246704102 0\\
3.21451687812805 0\\
3.23898792266846 0\\
3.26379895210266 0\\
3.28896427154541 0\\
3.31449961662292 0\\
3.34042167663574 0\\
3.36674857139587 0\\
3.39349937438965 0\\
3.42069554328918 0\\
3.4483597278595 0\\
3.47651672363281 0\\
3.50519371032715 0\\
3.53442049026489 0\\
3.56422996520996 0\\
3.59465765953064 0\\
3.6257438659668 0\\
3.65753245353699 0\\
3.69007301330566 0\\
3.72342085838318 0\\
3.75763893127441 0\\
3.79279804229736 0\\
3.82897973060608 0\\
3.86627721786499 0\\
3.90479946136475 0\\
3.94467377662659 0\\
3.9860508441925 0\\
4.02911138534546 0\\
4.07284307479858 0\\
4.11683511734009 0\\
4.16100072860718 0\\
4.20520448684692 0\\
4.24941444396973 0\\
4.29362487792969 0\\
4.33783578872681 0\\
4.38204622268677 0\\
4.42625665664673 0\\
4.47046709060669 0\\
4.51467752456665 0\\
4.55888795852661 0\\
4.60309839248657 0\\
4.64730882644653 0\\
4.69151973724365 0\\
4.73573064804077 0\\
4.77994155883789 0\\
4.82415246963501 0\\
4.86836338043213 0\\
4.91257381439209 0\\
4.95678472518921 0\\
5.00099563598633 0\\
5.04520654678345 0\\
5.08941698074341 0\\
5.13362789154053 0\\
5.17783832550049 0\\
5.22204923629761 0\\
5.26626014709473 0\\
5.31047105789185 0\\
5.35468196868896 0\\
5.39889240264893 0\\
5.44310331344604 0\\
5.48731422424316 0\\
5.53152513504028 0\\
5.5757360458374 0\\
5.61994695663452 0\\
5.66415786743164 0\\
5.7083683013916 0\\
5.75257921218872 0\\
5.79679012298584 0\\
5.84100103378296 0\\
5.88521194458008 0\\
5.9294228553772 0\\
5.97363376617432 0\\
6.01784420013428 0\\
6.0620551109314 0\\
6.10626602172852 0\\
6.15047693252563 0\\
6.19468784332275 0\\
6.23889875411987 0\\
6.28310966491699 0\\
6.32732009887695 0\\
6.37153100967407 0\\
6.41574192047119 0\\
6.45995283126831 0\\
6.50416374206543 0\\
6.54837465286255 0\\
6.59258508682251 0\\
6.63679599761963 0\\
6.68100690841675 0\\
6.72521781921387 0\\
6.76942873001099 0\\
6.81363964080811 0\\
6.85785055160522 0\\
6.90206098556519 0\\
6.9462718963623 0\\
6.99048280715942 0\\
7.03469371795654 0\\
7.07890462875366 0\\
7.12311553955078 0\\
7.1673264503479 0\\
7.21153688430786 0\\
7.25574779510498 0\\
7.2999587059021 0\\
7.34416961669922 0\\
7.38838052749634 0\\
7.43259143829346 0\\
7.47680234909058 0\\
7.52101278305054 0\\
7.56522369384766 0\\
7.60943460464478 0\\
7.65364551544189 0\\
7.69785642623901 0\\
7.74206733703613 0\\
7.78627824783325 0\\
7.83048868179321 0\\
7.87469959259033 0\\
7.91891050338745 0\\
7.96312141418457 0\\
8.00733184814453 0\\
8.05154323577881 0\\
8.09575366973877 0\\
8.13996505737305 0\\
8.18417549133301 0\\
8.22838687896729 0\\
8.27259731292725 0\\
8.31680774688721 0\\
8.36101913452148 0\\
8.40522956848145 0\\
8.44944095611572 0\\
8.49365139007568 0\\
8.53786277770996 0\\
8.58207321166992 0\\
8.62628364562988 0\\
8.67049503326416 0\\
8.71470546722412 0\\
8.7589168548584 0\\
8.80312728881836 0\\
8.84733867645264 0\\
8.8915491104126 0\\
8.93575954437256 0\\
8.97997093200684 0\\
9.0241813659668 0\\
9.06839275360107 0\\
9.11260318756104 0\\
9.156813621521 0\\
9.20102500915527 0\\
9.24523544311523 0\\
9.28944683074951 0\\
9.33365726470947 0\\
9.37786865234375 0\\
9.42207908630371 0\\
9.46628952026367 0\\
9.51050090789795 0\\
9.55471134185791 0\\
9.59892272949219 0\\
9.64313316345215 0\\
9.68734455108643 0\\
9.73155498504639 0\\
9.77576541900635 0\\
9.81997680664063 0\\
9.86418724060059 0\\
9.90839862823486 0\\
9.95260906219482 0\\
9.9968204498291 0\\
10.0410308837891 0\\
10.085241317749 0\\
10.1294527053833 0\\
10.1736631393433 0\\
10.2178745269775 0\\
10.2620849609375 0\\
10.3062963485718 0\\
10.3505067825317 0\\
10.3947172164917 0\\
10.438928604126 0\\
10.4831390380859 0\\
10.5273504257202 0\\
10.5715608596802 0\\
10.6157722473145 0\\
10.6599826812744 0\\
10.7041931152344 0\\
10.7484045028687 0\\
10.7926149368286 0\\
10.8368263244629 0\\
10.8810367584229 0\\
10.9252481460571 0\\
10.9694585800171 0\\
11.0136690139771 0\\
11.0578804016113 0\\
11.1020908355713 0\\
11.1463022232056 0\\
11.1905126571655 0\\
11.2347240447998 0\\
11.2789344787598 0\\
11.3231449127197 0\\
11.367356300354 0\\
11.411566734314 0\\
11.4557781219482 0\\
11.4778890609741 0\\
11.5 0\\
11.5243225097656 0\\
11.5510768890381 0\\
11.5805063247681 0\\
11.6128797531128 0\\
11.6484899520874 0\\
11.6876602172852 0\\
11.730749130249 0\\
11.774959564209 0\\
11.8191709518433 0\\
11.8633813858032 0\\
11.9075918197632 0\\
11.9518032073975 0\\
11.9960136413574 0\\
12.0402250289917 0\\
12.0844354629517 0\\
12.1286468505859 0\\
12.1728572845459 0\\
12.2170677185059 0\\
12.2612791061401 0\\
12.3054895401001 0\\
12.3497009277344 0\\
12.3939113616943 0\\
12.4381227493286 0\\
12.4823331832886 0\\
12.5265436172485 0\\
12.5707550048828 0\\
12.6149654388428 0\\
12.6591768264771 0\\
12.703387260437 0\\
12.7475986480713 0\\
12.7918090820313 0\\
12.8360195159912 0\\
12.8802309036255 0\\
12.9244413375854 0\\
12.9686527252197 0\\
13.0128631591797 0\\
13.057074546814 0\\
13.1012849807739 0\\
13.1454954147339 0\\
13.1897068023682 0\\
13.2339172363281 0\\
13.2781286239624 0\\
13.3223390579224 0\\
13.3665504455566 0\\
13.4107608795166 0\\
13.4549713134766 0\\
13.4991827011108 0\\
13.5433931350708 0\\
13.5876045227051 0\\
13.631814956665 0\\
13.6760263442993 0\\
13.7202367782593 0\\
13.7644472122192 0\\
13.8086585998535 0\\
13.8528690338135 0\\
13.8970804214478 0\\
13.9412908554077 0\\
13.985502243042 0\\
14.029712677002 0\\
14.0739231109619 0\\
14.1181344985962 0\\
14.1623449325562 0\\
14.2065563201904 0\\
14.2507667541504 0\\
14.2949781417847 0\\
14.3391885757446 0\\
14.3833990097046 0\\
14.4276103973389 0\\
14.4718208312988 0\\
14.5160322189331 0\\
14.5602426528931 0\\
14.6044540405273 0\\
14.6486644744873 0\\
14.6928749084473 0\\
14.7370862960815 0\\
14.7812967300415 0\\
14.8255081176758 0\\
14.8697185516357 0\\
14.9139289855957 0\\
14.95814037323 0\\
15.0023508071899 0\\
15.0465621948242 0\\
15.0907726287842 0\\
15.1349840164185 0\\
15.1791944503784 0\\
15.2234048843384 0\\
15.2676162719727 0\\
15.3118267059326 0\\
15.3560380935669 0\\
15.4002485275269 0\\
15.4444599151611 0\\
15.4886703491211 0\\
15.5328807830811 0\\
15.5770921707153 0\\
15.6213026046753 0\\
15.6655139923096 0\\
15.7097244262695 0\\
15.7539358139038 0\\
15.7981462478638 0\\
15.8423566818237 0\\
15.886568069458 0\\
15.930778503418 0\\
15.9749898910522 0\\
16.0192012786865 0\\
16.0630474090576 0\\
16.1063537597656 0\\
16.1503849029541 0\\
16.1945571899414 0\\
16.2387599945068 0\\
16.2829685211182 0\\
16.3271789550781 0\\
16.3713893890381 0\\
16.4156017303467 0\\
16.4598121643066 0\\
16.5040225982666 0\\
16.5482330322266 0\\
16.5924434661865 0\\
16.6366558074951 0\\
16.6808662414551 0\\
16.725076675415 0\\
16.769287109375 0\\
16.813497543335 0\\
16.8577079772949 0\\
16.9019184112549 0\\
16.9461288452148 0\\
16.9903411865234 0\\
17.0345516204834 0\\
17.0787620544434 0\\
17.1229724884033 0\\
17.1671829223633 0\\
17.2113952636719 0\\
17.2556056976318 0\\
17.2998161315918 0\\
17.3440265655518 0\\
17.3882369995117 0\\
17.4324493408203 0\\
17.4766597747803 0\\
17.5208702087402 0\\
17.5650806427002 0\\
17.6092929840088 0\\
17.6535034179688 0\\
17.6977138519287 0\\
17.7419242858887 0\\
17.7861347198486 0\\
17.8303470611572 0\\
17.8745574951172 0\\
17.9187679290771 0\\
17.9629783630371 0\\
18.0071887969971 0\\
18.0514011383057 0\\
18.0956115722656 0\\
18.1398220062256 0\\
18.1840324401855 0\\
18.2282447814941 0\\
18.2724552154541 0\\
18.3166656494141 0\\
18.360876083374 0\\
18.405086517334 0\\
18.4492988586426 0\\
18.4935092926025 0\\
18.5377197265625 0\\
18.5819301605225 0\\
18.6261405944824 0\\
18.670352935791 0\\
18.714563369751 0\\
18.7587738037109 0\\
18.8029842376709 0\\
18.8471965789795 0\\
18.8914070129395 0\\
18.9356174468994 0\\
18.9678077697754 0\\
19 0\\
};
\addlegendentry{$A_{r} = 0.00$};

\end{axis}
\end{tikzpicture}%
\caption{Valve problem with \dtmax{} = \expneg{6.25}{2}{[s]} with the linear solver.}
\label{fig:valveLin6pt25em02}
\end{figure}

This solution shows that the there is a non-physical spike in the mass flow rate when the closed valve starts to open at 15 [s].
This behavior is observed for the linear solver's solutions at successively smaller max timesteps through the smallest \dtmax{}, \expneg{1.95}{3}{[s]}.

\begin{figure}[h!tb]
\centering
% This file was created by matlab2tikz v0.4.3.
% Copyright (c) 2008--2013, Nico Schlömer <nico.schloemer@gmail.com>
% All rights reserved.
% 
\tikzsetnextfilename{plots/valveLinSols_eps}
\begin{tikzpicture}

\begin{axis}[%
width=0.8\textwidth,
height=0.508815326102268\textwidth,
scale only axis,
xmin=14.9,
xmax=15.2,
xlabel={Time $[\text{s}]$},
ymin=-1,
ymax=25,
ylabel={Flow Rate $[ \frac{\lbm{}}{\text{s}} ]$},
legend style={at={(0.03,0.97)},anchor=north west,draw=black,fill=white,legend cell align=left}
]
\addplot [
color=black,
solid
]
table[row sep=crcr]{
14.9139289855957 0\\
14.95814037323 0\\
15.0023508071899 0.117176152765751\\
15.0465621948242 20.6927852630615\\
15.0907726287842 10.9598197937012\\
15.1349840164185 7.8855185508728\\
15.1791944503784 9.56354999542236\\
};
\addlegendentry{$\Delta t_{\text{max}} = \expneg{6.2500}{2}{[\text{s}]}$};

\addplot [
color=black,
dash pattern=on 1pt off 3pt on 3pt off 3pt
]
table[row sep=crcr]{
14.9003047943115 0\\
14.9315547943115 0\\
14.9628047943115 0\\
14.9940547943115 0\\
15.0253047943115 11.3998517990112\\
15.0565547943115 6.12321281433105\\
15.0878047943115 4.86077785491943\\
15.1190547943115 6.43307399749756\\
15.1503047943115 8.03494358062744\\
15.1815547943115 9.64548206329346\\
};
\addlegendentry{$\Delta t_{\text{max}} = \expneg{3.1250}{2}{[\text{s}]}$};

\addplot [
color=black,
dashed
]
table[row sep=crcr]{
14.9059162139893 0\\
14.9215412139893 0\\
14.9371662139893 0\\
14.9527912139893 0\\
14.9684162139893 0\\
14.9840412139893 0\\
14.9996662139893 0\\
15.0152912139893 4.35183334350586\\
15.0309162139893 2.5003821849823\\
15.0465412139893 2.46754384040833\\
15.0621662139893 3.37599349021912\\
15.0777912139893 4.14937353134155\\
15.0934162139893 4.95952749252319\\
15.1090412139893 5.76840353012085\\
15.1246662139893 6.57980585098267\\
15.1402912139893 7.39263677597046\\
15.1559162139893 8.20651435852051\\
15.1715412139893 9.02116870880127\\
15.1871662139893 9.83641147613525\\
};
\addlegendentry{$\Delta t_{\text{max}} = \expneg{1.5625}{2}{[\text{s}]}$};

\addplot [
color=black,
dotted
]
table[row sep=crcr]{
14.905800819397 0\\
14.913613319397 0\\
14.921425819397 0\\
14.929238319397 0\\
14.937050819397 0\\
14.944863319397 0\\
14.952675819397 0\\
14.960488319397 0\\
14.968300819397 0\\
14.976113319397 0\\
14.983925819397 0\\
14.991738319397 0\\
14.999550819397 0\\
15.007363319397 1.08051860332489\\
15.015175819397 0.843475759029388\\
15.022988319397 1.2797132730484\\
15.030800819397 1.65828990936279\\
15.038613319397 2.06254577636719\\
15.046425819397 2.46494197845459\\
15.054238319397 2.86959600448608\\
15.062050819397 3.27524924278259\\
15.069863319397 3.68164348602295\\
15.077675819397 4.08856630325317\\
15.085488319397 4.49588108062744\\
15.093300819397 4.90349340438843\\
15.101113319397 5.3113374710083\\
15.108925819397 5.71936464309692\\
15.116738319397 6.12753915786743\\
15.124550819397 6.53583431243896\\
15.132363319397 6.9442286491394\\
15.140175819397 7.35270643234253\\
15.147988319397 7.76125431060791\\
15.155800819397 8.16986179351807\\
15.163613319397 8.57852077484131\\
15.171425819397 8.98722457885742\\
15.179238319397 9.39596652984619\\
15.187050819397 9.80474185943604\\
15.194863319397 10.213547706604\\
};
\addlegendentry{$\Delta t_{\text{max}} = \expneg{7.8125}{3}{[\text{s}]}$};

\addplot [
color=green,
solid
]
table[row sep=crcr]{
14.9033002853394 0\\
14.9072065353394 0\\
14.9111127853394 0\\
14.9150190353394 0\\
14.9189252853394 0\\
14.9228315353394 0\\
14.9267377853394 0\\
14.9306440353394 0\\
14.9345502853394 0\\
14.9384565353394 0\\
14.9423627853394 0\\
14.946268081665 0\\
14.950174331665 0\\
14.954080581665 0\\
14.957986831665 0\\
14.961893081665 0\\
14.965799331665 0\\
14.969705581665 0\\
14.973611831665 0\\
14.977518081665 0\\
14.981424331665 0\\
14.985330581665 0\\
14.989236831665 0\\
14.993143081665 0\\
14.997049331665 0\\
15.000955581665 0.0193258170038462\\
15.004861831665 0.484909266233444\\
15.008768081665 0.465420067310333\\
15.012674331665 0.704263806343079\\
15.016580581665 0.888620138168335\\
15.0204858779907 1.09210002422333\\
15.0243921279907 1.29347014427185\\
15.0282983779907 1.49598813056946\\
15.0322046279907 1.69893610477448\\
15.0361108779907 1.90221953392029\\
15.0400171279907 2.1057436466217\\
15.0439233779907 2.30944752693176\\
15.0478296279907 2.51328921318054\\
15.0517358779907 2.7172384262085\\
15.0556421279907 2.92127346992493\\
15.0595483779907 3.12537741661072\\
15.0634546279907 3.32953834533691\\
15.0673608779907 3.53374648094177\\
15.0712671279907 3.73799395561218\\
15.0751733779907 3.94227480888367\\
15.0790796279907 4.14658403396606\\
15.0829858779907 4.35091829299927\\
15.0868921279907 4.55527353286743\\
15.0907983779907 4.75964736938477\\
15.0947036743164 4.96403741836548\\
15.0986099243164 5.16844177246094\\
15.1025161743164 5.37285900115967\\
15.1064224243164 5.5772876739502\\
15.1103286743164 5.78172636032104\\
15.1142349243164 5.9861741065979\\
15.1181411743164 6.1906304359436\\
15.1220474243164 6.39509391784668\\
15.1259536743164 6.59956407546997\\
15.1298599243164 6.80404043197632\\
15.1337661743164 7.00852251052856\\
15.1376724243164 7.21300935745239\\
15.1415786743164 7.41750144958496\\
15.1454849243164 7.62199735641479\\
15.1493911743164 7.82649755477905\\
15.1532974243164 8.03100109100342\\
15.1572036743164 8.23550891876221\\
15.1611099243164 8.44001865386963\\
15.1650161743164 8.64453220367432\\
15.1689214706421 8.84904861450195\\
15.1728277206421 9.05356693267822\\
15.1767339706421 9.25808811187744\\
15.1806402206421 9.46261119842529\\
15.1845464706421 9.66713619232178\\
15.1884527206421 9.87166404724121\\
15.1923589706421 10.076192855835\\
15.1962652206421 10.2807235717773\\
};
\addlegendentry{$\Delta t_{\text{max}} = \expneg{3.9062}{3}{[\text{s}]}$};

\addplot [
color=green,
dotted
]
table[row sep=crcr]{
14.900221824646 0\\
14.902174949646 0\\
14.904128074646 0\\
14.906081199646 0\\
14.908034324646 0\\
14.909987449646 0\\
14.911940574646 0\\
14.913893699646 0\\
14.915846824646 0\\
14.9177989959717 0\\
14.9197521209717 0\\
14.9217052459717 0\\
14.9236583709717 0\\
14.9256114959717 0\\
14.9275646209717 0\\
14.9295177459717 0\\
14.9314708709717 0\\
14.9334239959717 0\\
14.9353771209717 0\\
14.9373302459717 0\\
14.9392833709717 0\\
14.9412364959717 0\\
14.9431896209717 0\\
14.9451427459717 0\\
14.9470958709717 0\\
14.9490489959717 0\\
14.9510021209717 0\\
14.9529552459717 0\\
14.9549083709717 0\\
14.9568614959717 0\\
14.9588146209717 0\\
14.9607677459717 0\\
14.9627208709717 0\\
14.9646739959717 0\\
14.9666271209717 0\\
14.9685802459717 0\\
14.9705333709717 0\\
14.9724864959717 0\\
14.9744396209717 0\\
14.9763927459717 0\\
14.9783458709717 0\\
14.9802989959717 0\\
14.9822521209717 0\\
14.9842052459717 0\\
14.9861583709717 0\\
14.9881114959717 0\\
14.9900646209717 0\\
14.9920167922974 0\\
14.9939699172974 0\\
14.9959230422974 0\\
14.9978761672974 0\\
14.9998292922974 0\\
15.0017824172974 0.0664489343762398\\
15.0037355422974 0.311820924282074\\
15.0056886672974 0.301760196685791\\
15.0076417922974 0.415858924388886\\
15.0095949172974 0.511866450309753\\
15.0115480422974 0.613315761089325\\
15.0135011672974 0.714363992214203\\
15.0154542922974 0.815780222415924\\
15.0174074172974 0.917371690273285\\
15.0193605422974 1.01909780502319\\
15.0213136672974 1.12092268466949\\
15.0232667922974 1.22282266616821\\
15.0252199172974 1.32478082180023\\
15.0271730422974 1.42678511142731\\
15.0291261672974 1.52882647514343\\
15.0310792922974 1.6308981180191\\
15.0330324172974 1.73299479484558\\
15.0349855422974 1.83511233329773\\
15.0369386672974 1.93724751472473\\
15.0388917922974 2.03939771652222\\
15.0408449172974 2.14156079292297\\
15.0427980422974 2.24373507499695\\
15.0447511672974 2.34591913223267\\
15.0467042922974 2.44811153411865\\
15.0486574172974 2.55031156539917\\
15.0506105422974 2.65251803398132\\
15.0525636672974 2.75473070144653\\
15.0545167922974 2.8569483757019\\
15.0564699172974 2.95917081832886\\
15.0584230422974 3.06139755249023\\
15.0603761672974 3.16362810134888\\
15.0623292922974 3.26586222648621\\
15.0642824172974 3.36809945106506\\
15.066234588623 3.47033953666687\\
15.068187713623 3.57258224487305\\
15.070140838623 3.67482733726501\\
15.072093963623 3.77707481384277\\
15.074047088623 3.87932419776917\\
15.076000213623 3.98157548904419\\
15.077953338623 4.08382844924927\\
15.079906463623 4.18608331680298\\
15.081859588623 4.28833961486816\\
15.083812713623 4.39059734344482\\
15.085765838623 4.4928560256958\\
15.087718963623 4.59511613845825\\
15.089672088623 4.69737720489502\\
15.091625213623 4.79963970184326\\
15.093578338623 4.90190315246582\\
15.095531463623 5.0041675567627\\
15.097484588623 5.10643243789673\\
15.099437713623 5.20869874954224\\
15.101390838623 5.31096506118774\\
15.103343963623 5.41323280334473\\
15.105297088623 5.51550102233887\\
15.107250213623 5.61776971817017\\
15.109203338623 5.72003936767578\\
15.111156463623 5.82230949401855\\
15.113109588623 5.92458009719849\\
15.115062713623 6.02685117721558\\
15.117015838623 6.12912273406982\\
15.118968963623 6.23139524459839\\
15.120922088623 6.33366775512695\\
15.122875213623 6.43594074249268\\
15.124828338623 6.53821420669556\\
15.126781463623 6.6404881477356\\
15.128734588623 6.74276208877563\\
15.130687713623 6.84503698348999\\
15.132640838623 6.94731187820435\\
15.134593963623 7.04958724975586\\
15.136547088623 7.15186262130737\\
15.138500213623 7.25413846969604\\
15.1404523849487 7.35641479492188\\
15.1424055099487 7.45869112014771\\
15.1443586349487 7.56096792221069\\
15.1463117599487 7.66324472427368\\
15.1482648849487 7.76552200317383\\
15.1502180099487 7.86779975891113\\
15.1521711349487 7.97007703781128\\
15.1541242599487 8.07235527038574\\
15.1560773849487 8.17463302612305\\
15.1580305099487 8.27691173553467\\
15.1599836349487 8.37919044494629\\
15.1619367599487 8.48146915435791\\
15.1638898849487 8.58374786376953\\
15.1658430099487 8.68602657318115\\
15.1677961349487 8.78830623626709\\
15.1697492599487 8.89058494567871\\
15.1717023849487 8.99286460876465\\
15.1736555099487 9.09514427185059\\
15.1756086349487 9.19742393493652\\
15.1775617599487 9.29970455169678\\
15.1795148849487 9.40198421478271\\
15.1814680099487 9.50426483154297\\
15.1834211349487 9.60654544830322\\
15.1853742599487 9.70882511138916\\
15.1873273849487 9.81110572814941\\
15.1892805099487 9.91338729858398\\
15.1912336349487 10.0156679153442\\
15.1931867599487 10.1179485321045\\
15.1951398849487 10.2202301025391\\
15.1970930099487 10.3225107192993\\
15.1990461349487 10.4247922897339\\
};
\addlegendentry{$\Delta t_{\text{max}} = \expneg{1.9531}{3}{[\text{s}]}$};

\end{axis}
\end{tikzpicture}%
\label{fig:valveLinSols}
\caption{Zoom of flow rate in linear solutions to the valve problem.}
\end{figure}

The presence of the non-physical behavior in the flow rate in the simulation with \dtmax{} = \expneg{1.56}{2}{[s]}, a \dtmax{} that produces a \dt{} history that is never \dtcrnt{} limited, necessitated that additional timestep refinements be examined for a smooth solution.
Subsequent timestep refinements produced results where the flow-rate spikes became progressively smaller; however, the time-step refinement study did not yield a solution that was free of this perturbation, as shown in \fig{fig:valveLinSols}.
The lack of a linear solution that is free from this perturbation is not an impediment to this study since the purpose is not to find a \dtmax{} that produces such a solution.
Instead, \tab{tab:valveLinTable} shows the run time statistics for the linear solver's solutions.

\begin{table}[h!tb]
\centering
\singlespace
\pgfplotstabletypeset[sci zerofill,sci E, col sep=comma,
	columns/0/.style={ column name= $[ \text{s} ]$, precision=1},
	columns/1/.style={ column name= $[ - ]$, int detect},
	columns/2/.style={ column name= $[ \text{s} ]$, precision=3},
	columns/3/.style={ column name= $[ \text{s} ]$, precision=3},
	every head row/.style={
		before row={
			\toprule
			\dtmax{} & $N_{t}$ & $T_{\text{CPU}}$& $\frac{T_{\text{CPU}}}{N_{t}}$ \\
		},
		after row=\midrule
	},
	every last row/.style={
after row=\bottomrule}]{tables/valveLinData.tex}

\caption{Run time data for the valve problem using the linear solver.}
\label{tab:valveLinTable}
\end{table}

In \tab{tab:valveLinTable} the first column shows the maximum timestep size for the simulation.
The second column shows the number of successful timesteps taken, $N_{t}$.
The third column is the CPU time required to complete simulation, $T_{\text{CPU}}$.
The final column is the the CPU time per timestep for a given simulation.
On average, the CPU time per timestep is \expneg{2.8}{3}{[s]} for the linear solution.

Next, the nonlinear solver was used to perform the same timestep sensitivity study.
The same set of \dtmax{} were run.
The solution at \dtmax{} = \expneg{6.25}{2}{[s]} is shown in \fig{fig:valveNln6pt25em02}

\begin{figure}[h!tb]
\centering
% This file was created by matlab2tikz v0.4.3.
% Copyright (c) 2008--2013, Nico Schlömer <nico.schloemer@gmail.com>
% All rights reserved.
% 
\tikzsetnextfilename{plots/valveNln6pt2500em02_eps}
\begin{tikzpicture}

\begin{axis}[%
width=0.8\textwidth,
%height=0.630967741935484\textwidth,
height=0.491294629700995\textwidth,
scale only axis,
xmin=0,
xmax=20,
xlabel={Time $[\text{s}]$},
ymin=-5,
ymax=125,
ylabel={Flow Rate $[ \frac{\lbm{}}{\text{s}} ]$},
legend style={draw=black,fill=white,legend cell align=left}
]
\addplot [
color=black,
solid
]
table[row sep=crcr]{
0 0\\
9.99999974737875e-06 -2.68723052698761e-08\\
2.49999993684469e-05 5.97379596456449e-07\\
4.75000015285332e-05 1.5992671251297e-05\\
8.12500002211891e-05 0.000229224911890924\\
0.000131875000079162 0.00192544842138886\\
0.000207812496228144 0.00898910593241453\\
0.000321718747727573 0.0234754215925932\\
0.000492578139528632 0.0383502058684826\\
0.00074886716902256 0.0502719767391682\\
0.00113330082967877 0.0704958811402321\\
0.00170995120424777 0.108530916273594\\
0.00257492670789361 0.167750313878059\\
0.00387239013798535 0.258952885866165\\
0.00529959984123707 0.363639861345291\\
0.00686953077092767 0.484229981899261\\
0.0085964547470212 0.623396694660187\\
0.0104960706084967 0.784338176250458\\
0.0125856483355165 0.970849931240082\\
0.014884184114635 1.18742954730988\\
0.0174125730991364 1.43940579891205\\
0.020193800330162 1.73308956623077\\
0.0232531521469355 2.07594513893127\\
0.0266184378415346 2.4767849445343\\
0.0303202513605356 2.94598364830017\\
0.0343922488391399 3.49570560455322\\
0.0388714447617531 4.14013338088989\\
0.0437985584139824 4.89567756652832\\
0.0492183864116669 5.78112745285034\\
0.0551801957190037 6.81768989562988\\
0.0617381855845451 8.02883529663086\\
0.0689519718289375 9.43983745574951\\
0.0768871381878853 11.0768632888794\\
0.0856158286333084 12.9654607772827\\
0.095217376947403 15.1283092498779\\
0.105779089033604 17.5821762084961\\
0.117396965622902 20.3342571258545\\
0.13017663359642 23.3783740997314\\
0.144234269857407 26.6919059753418\\
0.159697666764259 30.2346706390381\\
0.176707401871681 33.9509391784668\\
0.195418119430542 37.7751693725586\\
0.215999901294708 41.6409111022949\\
0.238639861345291 45.4909553527832\\
0.2635438144207 49.2860527038574\\
0.290938168764114 53.0097274780273\\
0.321071952581406 56.6680793762207\\
0.354219108819962 60.2851753234863\\
0.390680998563766 63.8959426879883\\
0.427134186029434 67.2247619628906\\
0.461836785078049 70.1926651000977\\
0.495115727186203 72.8892974853516\\
0.527189493179321 75.372917175293\\
0.558222532272339 77.6839752197266\\
0.588342726230621 79.8516006469727\\
0.617652356624603 81.8974990844727\\
0.646234631538391 83.8383636474609\\
0.674158751964569 85.6873397827148\\
0.701483011245728 87.4550094604492\\
0.728256940841675 89.1500778198242\\
0.754523456096649 90.77978515625\\
0.780319631099701 92.3502655029297\\
0.805678188800812 93.8667373657227\\
0.830627858638763 95.3337249755859\\
0.855194389820099 96.7551498413086\\
0.879400551319122 98.1344680786133\\
0.903267025947571 99.4747085571289\\
0.926812350749969 100.778564453125\\
0.950053453445435 102.048461914063\\
0.973005652427673 103.286552429199\\
0.995683073997498 104.494781494141\\
1.01809847354889 105.217460632324\\
1.0402637720108 105.585472106934\\
1.06237816810608 105.774543762207\\
1.08448839187622 105.871826171875\\
1.10659623146057 105.921867370605\\
1.12870299816132 105.947608947754\\
1.15080916881561 105.960845947266\\
1.17291498184204 105.967651367188\\
1.19502067565918 105.971153259277\\
1.21712636947632 105.972946166992\\
1.23923194408417 105.973876953125\\
1.26133751869202 105.974349975586\\
1.28344309329987 105.974594116211\\
1.30554866790771 105.974716186523\\
1.32765424251556 105.974784851074\\
1.34975969791412 105.974815368652\\
1.37186527252197 105.974830627441\\
1.39397084712982 105.974838256836\\
1.41607642173767 105.97484588623\\
1.43818199634552 105.97484588623\\
1.46028745174408 105.974853515625\\
1.48239302635193 105.974853515625\\
1.50449860095978 105.974853515625\\
1.52660417556763 105.974853515625\\
1.54870975017548 105.974853515625\\
1.57081520557404 105.974853515625\\
1.59292078018188 105.974853515625\\
1.61502635478973 105.974853515625\\
1.63713192939758 105.974853515625\\
1.65923750400543 105.974853515625\\
1.68134295940399 105.974853515625\\
1.70344853401184 105.974853515625\\
1.72555410861969 105.974853515625\\
1.74765968322754 105.974853515625\\
1.76976525783539 105.974853515625\\
1.79187071323395 105.974853515625\\
1.8139762878418 105.974853515625\\
1.83608186244965 105.974853515625\\
1.8581874370575 105.974853515625\\
1.88029301166534 105.974853515625\\
1.9023984670639 105.974853515625\\
1.92450404167175 105.974853515625\\
1.9466096162796 105.974853515625\\
1.96871519088745 105.974853515625\\
1.9908207654953 105.974853515625\\
2.01292634010315 105.974853515625\\
2.03503179550171 105.974853515625\\
2.05713748931885 105.974853515625\\
2.07924294471741 105.974853515625\\
2.10134840011597 105.974853515625\\
2.12345409393311 105.974853515625\\
2.14555954933167 105.974853515625\\
2.1676652431488 105.974853515625\\
2.18977069854736 105.974853515625\\
2.21187615394592 105.974853515625\\
2.23398184776306 105.974853515625\\
2.25608730316162 105.974853515625\\
2.27819299697876 105.974853515625\\
2.30029845237732 105.974853515625\\
2.32240390777588 105.974853515625\\
2.34450960159302 105.974853515625\\
2.36661505699158 105.974853515625\\
2.38872075080872 105.974853515625\\
2.41082620620728 105.974853515625\\
2.43293166160584 105.974853515625\\
2.45503735542297 105.974853515625\\
2.47714281082153 105.974853515625\\
2.49924850463867 105.974853515625\\
2.52135396003723 105.974853515625\\
2.54345941543579 105.974853515625\\
2.56556510925293 105.974853515625\\
2.58767056465149 105.974853515625\\
2.60977625846863 105.974853515625\\
2.63188171386719 105.974853515625\\
2.65398716926575 105.974853515625\\
2.67609286308289 105.974853515625\\
2.69819831848145 105.974853515625\\
2.72030401229858 105.974853515625\\
2.74240946769714 105.974853515625\\
2.7645149230957 105.974853515625\\
2.78662061691284 105.974853515625\\
2.8087260723114 105.974853515625\\
2.83083176612854 105.974853515625\\
2.8529372215271 105.974853515625\\
2.87504267692566 105.974853515625\\
2.8971483707428 105.974853515625\\
2.91925382614136 105.974853515625\\
2.9413595199585 105.974853515625\\
2.96346497535706 105.974853515625\\
2.98557043075562 105.974853515625\\
3.00767612457275 105.776901245117\\
3.02978157997131 105.100280761719\\
3.0519700050354 104.171333312988\\
3.07440447807312 103.101539611816\\
3.09709739685059 101.951995849609\\
3.12005996704102 100.754364013672\\
3.14330291748047 99.5247421264648\\
3.166836977005 98.2710571289063\\
3.19067335128784 96.9970169067383\\
3.21482396125793 95.7041015625\\
3.2393012046814 94.392707824707\\
3.26411890983582 93.0626068115234\\
3.28929138183594 91.7132720947266\\
3.31483387947083 90.3440017700195\\
3.34076333045959 88.9539413452148\\
3.36709761619568 87.5421447753906\\
3.39385652542114 86.1075820922852\\
3.42106103897095 84.6491241455078\\
3.44873404502869 83.1655502319336\\
3.47690057754517 81.6555328369141\\
3.50558733940125 80.1175994873047\\
3.53482460975647 78.550163269043\\
3.5646448135376 76.9514770507813\\
3.59508395195007 75.3195953369141\\
3.62618231773376 73.6523895263672\\
3.65798377990723 71.9474716186523\\
3.69053816795349 70.2022018432617\\
3.72390079498291 68.4135818481445\\
3.75813484191895 66.5782623291016\\
3.79331135749817 64.6924133300781\\
3.82951164245605 62.7516708374023\\
3.86682963371277 60.7510070800781\\
3.90537428855896 58.6845817565918\\
3.94527339935303 56.5455436706543\\
3.98667812347412 54.3257865905762\\
4.02976989746094 53.2729530334473\\
4.07350301742554 53.0479316711426\\
4.11750602722168 53.0003852844238\\
4.16167545318604 52.9903907775879\\
4.2058801651001 52.9882926940918\\
4.2500901222229 52.9878540039063\\
4.29430055618286 52.9877624511719\\
4.33851099014282 52.9877433776855\\
4.38272142410278 52.9877395629883\\
4.4269323348999 52.9877395629883\\
4.47114324569702 52.9877395629883\\
4.51535367965698 52.9877395629883\\
4.5595645904541 52.9877395629883\\
4.60377550125122 52.9877395629883\\
4.64798641204834 52.9877395629883\\
4.69219732284546 52.9877395629883\\
4.73640823364258 52.9877395629883\\
4.7806191444397 52.9877395629883\\
4.82482957839966 52.9877395629883\\
4.86904048919678 52.9877395629883\\
4.9132513999939 52.9877395629883\\
4.95746231079102 52.9877395629883\\
5.00167322158813 52.9877395629883\\
5.04588413238525 52.9877395629883\\
5.09009504318237 52.9877395629883\\
5.13430547714233 52.9877395629883\\
5.17851638793945 52.9877395629883\\
5.22272729873657 52.9877395629883\\
5.26693820953369 52.9877395629883\\
5.31114912033081 52.9877395629883\\
5.35536003112793 52.9877395629883\\
5.39957046508789 52.9877395629883\\
5.44378137588501 52.9877395629883\\
5.48799228668213 52.9877395629883\\
5.53220319747925 52.9877395629883\\
5.57641410827637 52.9877395629883\\
5.62062501907349 52.9877395629883\\
5.66483592987061 52.9877395629883\\
5.70904636383057 52.9877395629883\\
5.75325727462769 52.9877395629883\\
5.7974681854248 52.9877395629883\\
5.84167909622192 52.9877395629883\\
5.88589000701904 52.9877395629883\\
5.93010091781616 52.9877395629883\\
5.97431182861328 52.9877395629883\\
6.01852226257324 52.9877395629883\\
6.06273317337036 52.9877395629883\\
6.10694408416748 52.9877395629883\\
6.1511549949646 52.9877395629883\\
6.19536590576172 52.9877395629883\\
6.23957681655884 52.9877395629883\\
6.28378772735596 52.9877395629883\\
6.32799816131592 52.9877395629883\\
6.37220907211304 52.9877395629883\\
6.41641998291016 52.9877395629883\\
6.46063089370728 52.9877395629883\\
6.50484180450439 52.9877395629883\\
6.54905271530151 52.9877395629883\\
6.59326362609863 52.9877395629883\\
6.63747406005859 52.9877395629883\\
6.68168497085571 52.9877395629883\\
6.72589588165283 52.9877395629883\\
6.77010679244995 52.9877395629883\\
6.81431770324707 52.9877395629883\\
6.85852861404419 52.9877395629883\\
6.90273904800415 52.9877395629883\\
6.94694995880127 52.9877395629883\\
6.99116086959839 52.9877395629883\\
7.03537178039551 52.2437324523926\\
7.07958269119263 51.1552505493164\\
7.12379360198975 49.995246887207\\
7.16800451278687 48.8206787109375\\
7.21221494674683 47.643196105957\\
7.25642585754395 46.465145111084\\
7.30063676834106 45.286979675293\\
7.34484767913818 44.1087989807129\\
7.3890585899353 42.9306106567383\\
7.43326950073242 41.7524223327637\\
7.47748041152954 40.5742340087891\\
7.5216908454895 39.3960456848145\\
7.56590175628662 38.2178573608398\\
7.61011266708374 37.039665222168\\
7.65432357788086 35.8614768981934\\
7.69853448867798 34.6832847595215\\
7.7427453994751 33.5050964355469\\
7.78695631027222 32.326904296875\\
7.83116674423218 31.1487159729004\\
7.8753776550293 29.9705257415771\\
7.91958856582642 28.7923336029053\\
7.96379947662354 27.614143371582\\
8.0080099105835 26.624584197998\\
8.05222129821777 26.5094051361084\\
8.09643173217773 26.495719909668\\
8.14064311981201 26.4941215515137\\
8.18485355377197 26.4939346313477\\
8.22906494140625 26.4939117431641\\
8.27327537536621 26.4939098358154\\
8.31748580932617 26.4939098358154\\
8.36169719696045 26.4939079284668\\
8.40590763092041 26.4939079284668\\
8.45011901855469 26.4939079284668\\
8.49432945251465 26.4939079284668\\
8.53854084014893 26.4939079284668\\
8.58275127410889 26.4939079284668\\
8.62696170806885 26.4939079284668\\
8.67117309570313 26.4939079284668\\
8.71538352966309 26.4939079284668\\
8.75959491729736 26.4939079284668\\
8.80380535125732 26.4939079284668\\
8.8480167388916 26.4939079284668\\
8.89222717285156 26.4939079284668\\
8.93643760681152 26.4939079284668\\
8.9806489944458 26.4939079284668\\
9.02485942840576 26.4939079284668\\
9.06907081604004 26.4939079284668\\
9.11328125 26.4939079284668\\
9.15749168395996 26.4939079284668\\
9.20170307159424 26.4939079284668\\
9.2459135055542 26.4939079284668\\
9.29012489318848 26.4939079284668\\
9.33433532714844 26.4939079284668\\
9.37854671478271 26.4939079284668\\
9.42275714874268 26.4939079284668\\
9.46696758270264 26.4939079284668\\
9.51117897033691 26.4939079284668\\
9.55538940429688 26.4939079284668\\
9.59960079193115 26.4939079284668\\
9.64381122589111 26.4939079284668\\
9.68802261352539 26.4939079284668\\
9.73223304748535 26.4939079284668\\
9.77644348144531 26.4939079284668\\
9.82065486907959 26.4939079284668\\
9.86486530303955 26.4939079284668\\
9.90907669067383 26.4939079284668\\
9.95328712463379 26.4939079284668\\
9.99749851226807 26.4939079284668\\
10.041708946228 26.4939079284668\\
10.085919380188 26.4939079284668\\
10.1301307678223 26.4939079284668\\
10.1743412017822 26.4939079284668\\
10.2185525894165 26.4939079284668\\
10.2627630233765 26.4939079284668\\
10.3069744110107 26.4939079284668\\
10.3511848449707 26.4939079284668\\
10.3953952789307 26.4939079284668\\
10.4396066665649 26.4939079284668\\
10.4838171005249 26.4939079284668\\
10.5280284881592 26.4939079284668\\
10.5722389221191 26.4939079284668\\
10.6164503097534 26.4939079284668\\
10.6606607437134 26.4939079284668\\
10.7048711776733 26.4939079284668\\
10.7490825653076 26.4939079284668\\
10.7932929992676 26.4939079284668\\
10.8375043869019 26.4939079284668\\
10.8817148208618 26.4939079284668\\
10.9259262084961 26.4939079284668\\
10.9701366424561 26.4939079284668\\
11.014347076416 26.1576271057129\\
11.0585584640503 25.0768013000488\\
11.1027688980103 23.90940284729\\
11.1469802856445 22.7323703765869\\
11.1911907196045 21.5542984008789\\
11.2354021072388 20.3761215209961\\
11.2796125411987 19.1979331970215\\
11.3238229751587 18.0197467803955\\
11.368034362793 16.8415603637695\\
11.4122447967529 15.6633768081665\\
11.4561223983765 14.4940776824951\\
11.5 13.3247852325439\\
11.54421043396 12.14661693573\\
11.5884218215942 10.9684581756592\\
11.6326322555542 9.79013824462891\\
11.6768436431885 8.61193943023682\\
11.7210540771484 7.43374538421631\\
11.7652654647827 6.25555229187012\\
11.8094758987427 5.07735872268677\\
11.8536863327026 3.89916563034058\\
11.8978977203369 2.72097659111023\\
11.9421081542969 1.54277789592743\\
11.9863195419312 0.36458432674408\\
12.0305299758911 0\\
12.0747413635254 0\\
12.1189517974854 0\\
12.1631622314453 0\\
12.2073736190796 0\\
12.2515840530396 0\\
12.2957954406738 0\\
12.3400058746338 0\\
12.3842172622681 0\\
12.428427696228 0\\
12.472638130188 0\\
12.5168495178223 0\\
12.5610599517822 0\\
12.6052713394165 0\\
12.6494817733765 0\\
12.6936931610107 0\\
12.7379035949707 0\\
12.7821140289307 0\\
12.8263254165649 0\\
12.8705358505249 0\\
12.9147472381592 0\\
12.9589576721191 0\\
13.0031690597534 0\\
13.0473794937134 0\\
13.0915899276733 0\\
13.1358013153076 0\\
13.1800117492676 0\\
13.2242231369019 0\\
13.2684335708618 0\\
13.3126440048218 0\\
13.3568553924561 0\\
13.401065826416 0\\
13.4452772140503 0\\
13.4894876480103 0\\
13.5336990356445 0\\
13.5779094696045 0\\
13.6221199035645 0\\
13.6663312911987 0\\
13.7105417251587 0\\
13.754753112793 0\\
13.7989635467529 0\\
13.8431749343872 0\\
13.8873853683472 0\\
13.9315958023071 0\\
13.9758071899414 0\\
14.0200176239014 0\\
14.0642290115356 0\\
14.1084394454956 0\\
14.1526508331299 0\\
14.1968612670898 0\\
14.2410717010498 0\\
14.2852830886841 0\\
14.329493522644 0\\
14.3737049102783 0\\
14.4179153442383 0\\
14.4621267318726 0\\
14.5063371658325 0\\
14.5505475997925 0\\
14.5947589874268 0\\
14.6389694213867 0\\
14.683180809021 0\\
14.727391242981 0\\
14.7716026306152 0\\
14.8158130645752 0\\
14.8600234985352 0\\
14.9042348861694 0\\
14.9484453201294 0\\
14.9926567077637 0\\
15.0368671417236 1.93454301357269\\
15.0810785293579 4.24624395370483\\
15.1252889633179 6.56158781051636\\
15.1694993972778 8.87692451477051\\
15.2137107849121 11.1922979354858\\
15.2579212188721 13.5076742172241\\
15.3021326065063 15.8230504989624\\
15.3463430404663 18.1384258270264\\
15.3905544281006 20.4538021087646\\
15.4347648620605 22.7691783905029\\
15.4789752960205 25.0845527648926\\
15.5231866836548 27.3999271392822\\
15.5673971176147 29.7153015136719\\
15.611608505249 32.0306739807129\\
15.655818939209 34.3460464477539\\
15.7000303268433 36.6614189147949\\
15.7442407608032 38.9767913818359\\
15.7884511947632 41.2921600341797\\
15.8326625823975 43.6075286865234\\
15.8768730163574 45.9228973388672\\
15.9210844039917 48.2382850646973\\
15.9652948379517 50.5536499023438\\
16.0095062255859 52.4762725830078\\
16.0537166595459 52.8805809020996\\
16.0967807769775 52.9648361206055\\
16.1407527923584 52.9829216003418\\
16.1849117279053 52.9867286682129\\
16.2291107177734 52.9875259399414\\
16.2733192443848 52.9876937866211\\
16.3175296783447 52.9877281188965\\
16.3617401123047 52.987735748291\\
16.4059524536133 52.9877395629883\\
16.4501628875732 52.9877395629883\\
16.4943733215332 52.9877395629883\\
16.5385837554932 52.9877395629883\\
16.5827941894531 52.9877395629883\\
16.6270065307617 52.9877395629883\\
16.6712169647217 52.9877395629883\\
16.7154273986816 52.9877395629883\\
16.7596378326416 52.9877395629883\\
16.8038482666016 52.9877395629883\\
16.8480606079102 52.9877395629883\\
16.8922710418701 52.9877395629883\\
16.9364814758301 52.9877395629883\\
16.98069190979 52.9877395629883\\
17.0249042510986 52.9877395629883\\
17.0691146850586 52.9877395629883\\
17.1133251190186 52.9877395629883\\
17.1575355529785 52.9877395629883\\
17.2017459869385 52.9877395629883\\
17.2459583282471 52.9877395629883\\
17.290168762207 52.9877395629883\\
17.334379196167 52.9877395629883\\
17.378589630127 52.9877395629883\\
17.4228000640869 52.9877395629883\\
17.4670124053955 52.9877395629883\\
17.5112228393555 52.9877395629883\\
17.5554332733154 52.9877395629883\\
17.5996437072754 52.9877395629883\\
17.643856048584 52.9877395629883\\
17.6880664825439 52.9877395629883\\
17.7322769165039 52.9877395629883\\
17.7764873504639 52.9877395629883\\
17.8206977844238 52.9877395629883\\
17.8649101257324 52.9877395629883\\
17.9091205596924 52.9877395629883\\
17.9533309936523 52.9877395629883\\
17.9975414276123 52.9877395629883\\
18.0417518615723 52.9877395629883\\
18.0859642028809 52.9877395629883\\
18.1301746368408 52.9877395629883\\
18.1743850708008 52.9877395629883\\
18.2185955047607 52.9877395629883\\
18.2628059387207 52.9877395629883\\
18.3070182800293 52.9877395629883\\
18.3512287139893 52.9877395629883\\
18.3954391479492 52.9877395629883\\
18.4396495819092 52.9877395629883\\
18.4838619232178 52.9877395629883\\
18.5280723571777 52.9877395629883\\
18.5722827911377 52.9877395629883\\
18.6164932250977 52.9877395629883\\
18.6607036590576 52.9877395629883\\
18.7049160003662 52.9877395629883\\
18.7491264343262 52.9877395629883\\
18.7933368682861 52.9877395629883\\
18.8375473022461 52.9877395629883\\
18.8817577362061 52.9877395629883\\
18.9259700775146 52.9877395629883\\
18.962984085083 52.9877395629883\\
19 52.9877395629883\\
};
\addlegendentry{$A_{r}(t)$};

\addplot [
color=black,
dash pattern=on 1pt off 3pt on 3pt off 3pt
]
table[row sep=crcr]{
0 0\\
9.99999974737875e-06 -2.6872250202814e-08\\
2.49999993684469e-05 5.97376981659181e-07\\
4.75000015285332e-05 1.5992558473954e-05\\
8.12500002211891e-05 0.000229222685447894\\
0.000131875000079162 0.00192542537115514\\
0.000207812496228144 0.00898898299783468\\
0.000321718747727573 0.0234750732779503\\
0.000492578139528632 0.0383495762944222\\
0.00074886716902256 0.0502709187567234\\
0.00113330082967877 0.0704936757683754\\
0.00170995120424777 0.10852438211441\\
0.00257492670789361 0.167728826403618\\
0.00387239013798535 0.258878469467163\\
0.00529959984123707 0.36345100402832\\
0.00686953077092767 0.483818382024765\\
0.0085964547470212 0.62257993221283\\
0.0104960706084967 0.782817423343658\\
0.0125856483355165 0.968145906925201\\
0.014884184114635 1.18278455734253\\
0.0174125730991364 1.43163669109344\\
0.020193800330162 1.72036504745483\\
0.0232531521469355 2.05545377731323\\
0.0266184378415346 2.44424080848694\\
0.0303202513605356 2.89489793777466\\
0.0343922488391399 3.41632533073425\\
0.0388714447617531 4.01791954040527\\
0.0437985584139824 4.70916080474854\\
0.0492183864116669 5.49897241592407\\
0.0551801957190037 6.39480876922607\\
0.0617381855845451 7.40148830413818\\
0.0689519718289375 8.51985263824463\\
0.0768871381878853 9.74547004699707\\
0.0856158286333084 11.0677270889282\\
0.095217376947403 12.4697322845459\\
0.105779089033604 13.9294090270996\\
0.117396965622902 15.4219026565552\\
0.13017663359642 16.9229755401611\\
0.144234269857407 18.4126529693604\\
0.159697666764259 19.8781642913818\\
0.176707401871681 21.3153514862061\\
0.195418119430542 22.7282943725586\\
0.215999901294708 24.1273956298828\\
0.238639861345291 25.5267066955566\\
0.2635438144207 26.9412879943848\\
0.290938168764114 28.3852310180664\\
0.321071952581406 29.8706207275391\\
0.354219108819962 31.407299041748\\
0.390680998563766 33.0031890869141\\
0.427134186029434 34.5181922912598\\
0.461836785078049 35.8965759277344\\
0.495115727186203 37.1668472290039\\
0.527189493179321 38.3486099243164\\
0.558222532272339 39.4563026428223\\
0.588342726230621 40.5008735656738\\
0.617652356624603 41.4908561706543\\
0.646234631538391 42.4330406188965\\
0.674158751964569 43.3329467773438\\
0.701483011245728 44.1951179504395\\
0.728256940841675 45.0233535766602\\
0.754523456096649 45.8208694458008\\
0.780319631099701 46.5904159545898\\
0.805678188800812 47.3343620300293\\
0.830627858638763 48.0547714233398\\
0.855194389820099 48.7534446716309\\
0.879400551319122 49.4319801330566\\
0.903267025947571 50.0917816162109\\
0.926812350749969 50.7341117858887\\
0.950053453445435 51.3600959777832\\
0.973005652427673 51.9707489013672\\
0.995683073997498 52.5669860839844\\
1.01809847354889 52.8432464599609\\
1.0402637720108 52.9377632141113\\
1.06237816810608 52.9704170227051\\
1.08448839187622 52.9817276000977\\
1.10659623146057 52.9856491088867\\
1.12870299816132 52.9870109558105\\
1.15080916881561 52.9874839782715\\
1.17291498184204 52.9876518249512\\
1.19502067565918 52.9877090454102\\
1.21712636947632 52.9877319335938\\
1.23923194408417 52.9877395629883\\
1.26133751869202 52.9877395629883\\
1.28344309329987 52.9877433776855\\
1.30554866790771 52.9877433776855\\
1.32765424251556 52.9877395629883\\
1.34975969791412 52.9877395629883\\
1.37186527252197 52.9877395629883\\
1.39397084712982 52.9877395629883\\
1.41607642173767 52.9877395629883\\
1.43818199634552 52.9877395629883\\
1.46028745174408 52.9877395629883\\
1.48239302635193 52.9877395629883\\
1.50449860095978 52.9877395629883\\
1.52660417556763 52.9877395629883\\
1.54870975017548 52.9877395629883\\
1.57081520557404 52.9877395629883\\
1.59292078018188 52.9877395629883\\
1.61502635478973 52.9877395629883\\
1.63713192939758 52.9877395629883\\
1.65923750400543 52.9877395629883\\
1.68134295940399 52.9877395629883\\
1.70344853401184 52.9877395629883\\
1.72555410861969 52.9877395629883\\
1.74765968322754 52.9877395629883\\
1.76976525783539 52.9877395629883\\
1.79187071323395 52.9877395629883\\
1.8139762878418 52.9877395629883\\
1.83608186244965 52.9877395629883\\
1.8581874370575 52.9877395629883\\
1.88029301166534 52.9877395629883\\
1.9023984670639 52.9877395629883\\
1.92450404167175 52.9877395629883\\
1.9466096162796 52.9877395629883\\
1.96871519088745 52.9877395629883\\
1.9908207654953 52.9877395629883\\
2.01292634010315 52.9877395629883\\
2.03503179550171 52.9877395629883\\
2.05713748931885 52.9877395629883\\
2.07924294471741 52.9877395629883\\
2.10134840011597 52.9877395629883\\
2.12345409393311 52.9877395629883\\
2.14555954933167 52.9877395629883\\
2.1676652431488 52.9877395629883\\
2.18977069854736 52.9877395629883\\
2.21187615394592 52.9877395629883\\
2.23398184776306 52.9877395629883\\
2.25608730316162 52.9877395629883\\
2.27819299697876 52.9877395629883\\
2.30029845237732 52.9877395629883\\
2.32240390777588 52.9877395629883\\
2.34450960159302 52.9877395629883\\
2.36661505699158 52.9877395629883\\
2.38872075080872 52.9877395629883\\
2.41082620620728 52.9877395629883\\
2.43293166160584 52.9877395629883\\
2.45503735542297 52.9877395629883\\
2.47714281082153 52.9877395629883\\
2.49924850463867 52.9877395629883\\
2.52135396003723 52.9877395629883\\
2.54345941543579 52.9877395629883\\
2.56556510925293 52.9877395629883\\
2.58767056465149 52.9877395629883\\
2.60977625846863 52.9877395629883\\
2.63188171386719 52.9877395629883\\
2.65398716926575 52.9877395629883\\
2.67609286308289 52.9877395629883\\
2.69819831848145 52.9877395629883\\
2.72030401229858 52.9877395629883\\
2.74240946769714 52.9877395629883\\
2.7645149230957 52.9877395629883\\
2.78662061691284 52.9877395629883\\
2.8087260723114 52.9877395629883\\
2.83083176612854 52.9877395629883\\
2.8529372215271 52.9877395629883\\
2.87504267692566 52.9877395629883\\
2.8971483707428 52.9877395629883\\
2.91925382614136 52.9877395629883\\
2.9413595199585 52.9877395629883\\
2.96346497535706 52.9877395629883\\
2.98557043075562 52.9877395629883\\
3.00767612457275 52.9877395629883\\
3.02978157997131 52.9877395629883\\
3.0519700050354 52.9877395629883\\
3.07440447807312 52.9877395629883\\
3.09709739685059 52.9877395629883\\
3.12005996704102 52.9877395629883\\
3.14330291748047 52.9877395629883\\
3.166836977005 52.9877395629883\\
3.19067335128784 52.9877395629883\\
3.21482396125793 52.9877395629883\\
3.2393012046814 52.9877395629883\\
3.26411890983582 52.9877395629883\\
3.28929138183594 52.9877395629883\\
3.31483387947083 52.9877395629883\\
3.34076333045959 52.9877395629883\\
3.36709761619568 52.9877395629883\\
3.39385652542114 52.9877395629883\\
3.42106103897095 52.9877395629883\\
3.44873404502869 52.9877395629883\\
3.47690057754517 52.9877395629883\\
3.50558733940125 52.9877395629883\\
3.53482460975647 52.9877395629883\\
3.5646448135376 52.9877395629883\\
3.59508395195007 52.9877395629883\\
3.62618231773376 52.9877395629883\\
3.65798377990723 52.9877395629883\\
3.69053816795349 52.9877395629883\\
3.72390079498291 52.9877395629883\\
3.75813484191895 52.9877395629883\\
3.79331135749817 52.9877395629883\\
3.82951164245605 52.9877395629883\\
3.86682963371277 52.9877395629883\\
3.90537428855896 52.9877395629883\\
3.94527339935303 52.9877395629883\\
3.98667812347412 52.9877395629883\\
4.02976989746094 52.9877395629883\\
4.07350301742554 52.9877395629883\\
4.11750602722168 52.9877395629883\\
4.16167545318604 52.9877395629883\\
4.2058801651001 52.9877395629883\\
4.2500901222229 52.9877395629883\\
4.29430055618286 52.9877395629883\\
4.33851099014282 52.9877395629883\\
4.38272142410278 52.9877395629883\\
4.4269323348999 52.9877395629883\\
4.47114324569702 52.9877395629883\\
4.51535367965698 52.9877395629883\\
4.5595645904541 52.9877395629883\\
4.60377550125122 52.9877395629883\\
4.64798641204834 52.9877395629883\\
4.69219732284546 52.9877395629883\\
4.73640823364258 52.9877395629883\\
4.7806191444397 52.9877395629883\\
4.82482957839966 52.9877395629883\\
4.86904048919678 52.9877395629883\\
4.9132513999939 52.9877395629883\\
4.95746231079102 52.9877395629883\\
5.00167322158813 52.9877395629883\\
5.04588413238525 52.9877395629883\\
5.09009504318237 52.9877395629883\\
5.13430547714233 52.9877395629883\\
5.17851638793945 52.9877395629883\\
5.22272729873657 52.9877395629883\\
5.26693820953369 52.9877395629883\\
5.31114912033081 52.9877395629883\\
5.35536003112793 52.9877395629883\\
5.39957046508789 52.9877395629883\\
5.44378137588501 52.9877395629883\\
5.48799228668213 52.9877395629883\\
5.53220319747925 52.9877395629883\\
5.57641410827637 52.9877395629883\\
5.62062501907349 52.9877395629883\\
5.66483592987061 52.9877395629883\\
5.70904636383057 52.9877395629883\\
5.75325727462769 52.9877395629883\\
5.7974681854248 52.9877395629883\\
5.84167909622192 52.9877395629883\\
5.88589000701904 52.9877395629883\\
5.93010091781616 52.9877395629883\\
5.97431182861328 52.9877395629883\\
6.01852226257324 52.9877395629883\\
6.06273317337036 52.9877395629883\\
6.10694408416748 52.9877395629883\\
6.1511549949646 52.9877395629883\\
6.19536590576172 52.9877395629883\\
6.23957681655884 52.9877395629883\\
6.28378772735596 52.9877395629883\\
6.32799816131592 52.9877395629883\\
6.37220907211304 52.9877395629883\\
6.41641998291016 52.9877395629883\\
6.46063089370728 52.9877395629883\\
6.50484180450439 52.9877395629883\\
6.54905271530151 52.9877395629883\\
6.59326362609863 52.9877395629883\\
6.63747406005859 52.9877395629883\\
6.68168497085571 52.9877395629883\\
6.72589588165283 52.9877395629883\\
6.77010679244995 52.9877395629883\\
6.81431770324707 52.9877395629883\\
6.85852861404419 52.9877395629883\\
6.90273904800415 52.9877395629883\\
6.94694995880127 52.9877395629883\\
6.99116086959839 52.9877395629883\\
7.03537178039551 52.9877395629883\\
7.07958269119263 52.9877395629883\\
7.12379360198975 52.9877395629883\\
7.16800451278687 52.9877395629883\\
7.21221494674683 52.9877395629883\\
7.25642585754395 52.9877395629883\\
7.30063676834106 52.9877395629883\\
7.34484767913818 52.9877395629883\\
7.3890585899353 52.9877395629883\\
7.43326950073242 52.9877395629883\\
7.47748041152954 52.9877395629883\\
7.5216908454895 52.9877395629883\\
7.56590175628662 52.9877395629883\\
7.61011266708374 52.9877395629883\\
7.65432357788086 52.9877395629883\\
7.69853448867798 52.9877395629883\\
7.7427453994751 52.9877395629883\\
7.78695631027222 52.9877395629883\\
7.83116674423218 52.9877395629883\\
7.8753776550293 52.9877395629883\\
7.91958856582642 52.9877395629883\\
7.96379947662354 52.9877395629883\\
8.0080099105835 52.9877395629883\\
8.05222129821777 52.9877395629883\\
8.09643173217773 52.9877395629883\\
8.14064311981201 52.9877395629883\\
8.18485355377197 52.9877395629883\\
8.22906494140625 52.9877395629883\\
8.27327537536621 52.9877395629883\\
8.31748580932617 52.9877395629883\\
8.36169719696045 52.9877395629883\\
8.40590763092041 52.9877395629883\\
8.45011901855469 52.9877395629883\\
8.49432945251465 52.9877395629883\\
8.53854084014893 52.9877395629883\\
8.58275127410889 52.9877395629883\\
8.62696170806885 52.9877395629883\\
8.67117309570313 52.9877395629883\\
8.71538352966309 52.9877395629883\\
8.75959491729736 52.9877395629883\\
8.80380535125732 52.9877395629883\\
8.8480167388916 52.9877395629883\\
8.89222717285156 52.9877395629883\\
8.93643760681152 52.9877395629883\\
8.9806489944458 52.9877395629883\\
9.02485942840576 52.9877395629883\\
9.06907081604004 52.9877395629883\\
9.11328125 52.9877395629883\\
9.15749168395996 52.9877395629883\\
9.20170307159424 52.9877395629883\\
9.2459135055542 52.9877395629883\\
9.29012489318848 52.9877395629883\\
9.33433532714844 52.9877395629883\\
9.37854671478271 52.9877395629883\\
9.42275714874268 52.9877395629883\\
9.46696758270264 52.9877395629883\\
9.51117897033691 52.9877395629883\\
9.55538940429688 52.9877395629883\\
9.59960079193115 52.9877395629883\\
9.64381122589111 52.9877395629883\\
9.68802261352539 52.9877395629883\\
9.73223304748535 52.9877395629883\\
9.77644348144531 52.9877395629883\\
9.82065486907959 52.9877395629883\\
9.86486530303955 52.9877395629883\\
9.90907669067383 52.9877395629883\\
9.95328712463379 52.9877395629883\\
9.99749851226807 52.9877395629883\\
10.041708946228 52.9877395629883\\
10.085919380188 52.9877395629883\\
10.1301307678223 52.9877395629883\\
10.1743412017822 52.9877395629883\\
10.2185525894165 52.9877395629883\\
10.2627630233765 52.9877395629883\\
10.3069744110107 52.9877395629883\\
10.3511848449707 52.9877395629883\\
10.3953952789307 52.9877395629883\\
10.4396066665649 52.9877395629883\\
10.4838171005249 52.9877395629883\\
10.5280284881592 52.9877395629883\\
10.5722389221191 52.9877395629883\\
10.6164503097534 52.9877395629883\\
10.6606607437134 52.9877395629883\\
10.7048711776733 52.9877395629883\\
10.7490825653076 52.9877395629883\\
10.7932929992676 52.9877395629883\\
10.8375043869019 52.9877395629883\\
10.8817148208618 52.9877395629883\\
10.9259262084961 52.9877395629883\\
10.9701366424561 52.9877395629883\\
11.014347076416 52.9877395629883\\
11.0585584640503 52.9877395629883\\
11.1027688980103 52.9877395629883\\
11.1469802856445 52.9877395629883\\
11.1911907196045 52.9877395629883\\
11.2354021072388 52.9877395629883\\
11.2796125411987 52.9877395629883\\
11.3238229751587 52.9877395629883\\
11.368034362793 52.9877395629883\\
11.4122447967529 52.9877395629883\\
11.4561223983765 52.9877395629883\\
11.5 52.9877395629883\\
11.54421043396 52.9877395629883\\
11.5884218215942 52.9877395629883\\
11.6326322555542 52.9877395629883\\
11.6768436431885 52.9877395629883\\
11.7210540771484 52.9877395629883\\
11.7652654647827 52.9877395629883\\
11.8094758987427 52.9877395629883\\
11.8536863327026 52.9877395629883\\
11.8978977203369 52.9877395629883\\
11.9421081542969 52.9877395629883\\
11.9863195419312 52.9877395629883\\
12.0305299758911 52.9877395629883\\
12.0747413635254 52.9877395629883\\
12.1189517974854 52.9877395629883\\
12.1631622314453 52.9877395629883\\
12.2073736190796 52.9877395629883\\
12.2515840530396 52.9877395629883\\
12.2957954406738 52.9877395629883\\
12.3400058746338 52.9877395629883\\
12.3842172622681 52.9877395629883\\
12.428427696228 52.9877395629883\\
12.472638130188 52.9877395629883\\
12.5168495178223 52.9877395629883\\
12.5610599517822 52.9877395629883\\
12.6052713394165 52.9877395629883\\
12.6494817733765 52.9877395629883\\
12.6936931610107 52.9877395629883\\
12.7379035949707 52.9877395629883\\
12.7821140289307 52.9877395629883\\
12.8263254165649 52.9877395629883\\
12.8705358505249 52.9877395629883\\
12.9147472381592 52.9877395629883\\
12.9589576721191 52.9877395629883\\
13.0031690597534 52.9877395629883\\
13.0473794937134 52.9877395629883\\
13.0915899276733 52.9877395629883\\
13.1358013153076 52.9877395629883\\
13.1800117492676 52.9877395629883\\
13.2242231369019 52.9877395629883\\
13.2684335708618 52.9877395629883\\
13.3126440048218 52.9877395629883\\
13.3568553924561 52.9877395629883\\
13.401065826416 52.9877395629883\\
13.4452772140503 52.9877395629883\\
13.4894876480103 52.9877395629883\\
13.5336990356445 52.9877395629883\\
13.5779094696045 52.9877395629883\\
13.6221199035645 52.9877395629883\\
13.6663312911987 52.9877395629883\\
13.7105417251587 52.9877395629883\\
13.754753112793 52.9877395629883\\
13.7989635467529 52.9877395629883\\
13.8431749343872 52.9877395629883\\
13.8873853683472 52.9877395629883\\
13.9315958023071 52.9877395629883\\
13.9758071899414 52.9877395629883\\
14.0200176239014 52.9877395629883\\
14.0642290115356 52.9877395629883\\
14.1084394454956 52.9877395629883\\
14.1526508331299 52.9877395629883\\
14.1968612670898 52.9877395629883\\
14.2410717010498 52.9877395629883\\
14.2852830886841 52.9877395629883\\
14.329493522644 52.9877395629883\\
14.3737049102783 52.9877395629883\\
14.4179153442383 52.9877395629883\\
14.4621267318726 52.9877395629883\\
14.5063371658325 52.9877395629883\\
14.5505475997925 52.9877395629883\\
14.5947589874268 52.9877395629883\\
14.6389694213867 52.9877395629883\\
14.683180809021 52.9877395629883\\
14.727391242981 52.9877395629883\\
14.7716026306152 52.9877395629883\\
14.8158130645752 52.9877395629883\\
14.8600234985352 52.9877395629883\\
14.9042348861694 52.9877395629883\\
14.9484453201294 52.9877395629883\\
14.9926567077637 52.9877395629883\\
15.0368671417236 52.9877395629883\\
15.0810785293579 52.9877395629883\\
15.1252889633179 52.9877395629883\\
15.1694993972778 52.9877395629883\\
15.2137107849121 52.9877395629883\\
15.2579212188721 52.9877395629883\\
15.3021326065063 52.9877395629883\\
15.3463430404663 52.9877395629883\\
15.3905544281006 52.9877395629883\\
15.4347648620605 52.9877395629883\\
15.4789752960205 52.9877395629883\\
15.5231866836548 52.9877395629883\\
15.5673971176147 52.9877395629883\\
15.611608505249 52.9877395629883\\
15.655818939209 52.9877395629883\\
15.7000303268433 52.9877395629883\\
15.7442407608032 52.9877395629883\\
15.7884511947632 52.9877395629883\\
15.8326625823975 52.9877395629883\\
15.8768730163574 52.9877395629883\\
15.9210844039917 52.9877395629883\\
15.9652948379517 52.9877395629883\\
16.0095062255859 52.9877395629883\\
16.0537166595459 52.9877395629883\\
16.0967807769775 52.9877395629883\\
16.1407527923584 52.9877395629883\\
16.1849117279053 52.9877395629883\\
16.2291107177734 52.9877395629883\\
16.2733192443848 52.9877395629883\\
16.3175296783447 52.9877395629883\\
16.3617401123047 52.9877395629883\\
16.4059524536133 52.9877395629883\\
16.4501628875732 52.9877395629883\\
16.4943733215332 52.9877395629883\\
16.5385837554932 52.9877395629883\\
16.5827941894531 52.9877395629883\\
16.6270065307617 52.9877395629883\\
16.6712169647217 52.9877395629883\\
16.7154273986816 52.9877395629883\\
16.7596378326416 52.9877395629883\\
16.8038482666016 52.9877395629883\\
16.8480606079102 52.9877395629883\\
16.8922710418701 52.9877395629883\\
16.9364814758301 52.9877395629883\\
16.98069190979 52.9877395629883\\
17.0249042510986 52.9877395629883\\
17.0691146850586 52.9877395629883\\
17.1133251190186 52.9877395629883\\
17.1575355529785 52.9877395629883\\
17.2017459869385 52.9877395629883\\
17.2459583282471 52.9877395629883\\
17.290168762207 52.9877395629883\\
17.334379196167 52.9877395629883\\
17.378589630127 52.9877395629883\\
17.4228000640869 52.9877395629883\\
17.4670124053955 52.9877395629883\\
17.5112228393555 52.9877395629883\\
17.5554332733154 52.9877395629883\\
17.5996437072754 52.9877395629883\\
17.643856048584 52.9877395629883\\
17.6880664825439 52.9877395629883\\
17.7322769165039 52.9877395629883\\
17.7764873504639 52.9877395629883\\
17.8206977844238 52.9877395629883\\
17.8649101257324 52.9877395629883\\
17.9091205596924 52.9877395629883\\
17.9533309936523 52.9877395629883\\
17.9975414276123 52.9877395629883\\
18.0417518615723 52.9877395629883\\
18.0859642028809 52.9877395629883\\
18.1301746368408 52.9877395629883\\
18.1743850708008 52.9877395629883\\
18.2185955047607 52.9877395629883\\
18.2628059387207 52.9877395629883\\
18.3070182800293 52.9877395629883\\
18.3512287139893 52.9877395629883\\
18.3954391479492 52.9877395629883\\
18.4396495819092 52.9877395629883\\
18.4838619232178 52.9877395629883\\
18.5280723571777 52.9877395629883\\
18.5722827911377 52.9877395629883\\
18.6164932250977 52.9877395629883\\
18.6607036590576 52.9877395629883\\
18.7049160003662 52.9877395629883\\
18.7491264343262 52.9877395629883\\
18.7933368682861 52.9877395629883\\
18.8375473022461 52.9877395629883\\
18.8817577362061 52.9877395629883\\
18.9259700775146 52.9877395629883\\
18.962984085083 52.9877395629883\\
19 52.9877395629883\\
};
\addlegendentry{$A_{r} = 0.50$};

\addplot [
color=black,
dashed
]
table[row sep=crcr]{
0 0\\
9.99999974737875e-06 -2.6871862957023e-08\\
2.49999993684469e-05 5.97366636156949e-07\\
4.75000015285332e-05 1.59920964506455e-05\\
8.12500002211891e-05 0.000229213779675774\\
0.000131875000079162 0.00192533293738961\\
0.000207812496228144 0.00898849032819271\\
0.000321718747727573 0.0234736856073141\\
0.000492578139528632 0.0383470691740513\\
0.00074886716902256 0.0502666942775249\\
0.00113330082967877 0.0704848617315292\\
0.00170995120424777 0.108498260378838\\
0.00257492670789361 0.167642965912819\\
0.00387239013798535 0.258581548929214\\
0.00529959984123707 0.362698644399643\\
0.00686953077092767 0.482182741165161\\
0.0085964547470212 0.619344651699066\\
0.0104960706084967 0.776820123195648\\
0.0125856483355165 0.957544326782227\\
0.014884184114635 1.16471421718597\\
0.0174125730991364 1.40171658992767\\
0.020193800330162 1.67199862003326\\
0.0232531521469355 1.97886300086975\\
0.0266184378415346 2.32517242431641\\
0.0303202513605356 2.71296048164368\\
0.0343922488391399 3.14296746253967\\
0.0388714447617531 3.61415767669678\\
0.0437985584139824 4.12331914901733\\
0.0492183864116669 4.66488456726074\\
0.0551801957190037 5.23112821578979\\
0.0617381855845451 5.81282091140747\\
0.0689519718289375 6.40033102035522\\
0.0768871381878853 6.98496770858765\\
0.0856158286333084 7.5602650642395\\
0.095217376947403 8.12285900115967\\
0.105779089033604 8.67273902893066\\
0.117396965622902 9.21284675598145\\
0.13017663359642 9.74819564819336\\
0.144234269857407 10.284818649292\\
0.159697666764259 10.8288078308105\\
0.176707401871681 11.3856592178345\\
0.195418119430542 11.9599752426147\\
0.215999901294708 12.5554504394531\\
0.238639861345291 13.1750411987305\\
0.2635438144207 13.821192741394\\
0.290938168764114 14.4960412979126\\
0.321071952581406 15.2015686035156\\
0.354219108819962 15.9396982192993\\
0.390680998563766 16.7123508453369\\
0.427134186029434 17.4496593475342\\
0.461836785078049 18.1229228973389\\
0.495115727186203 18.7451457977295\\
0.527189493179321 19.3253307342529\\
0.558222532272339 19.8701438903809\\
0.588342726230621 20.3846778869629\\
0.617652356624603 20.8729286193848\\
0.646234631538391 21.3380908966064\\
0.674158751964569 21.7827758789063\\
0.701483011245728 22.2091426849365\\
0.728256940841675 22.6190032958984\\
0.754523456096649 23.0138988494873\\
0.780319631099701 23.3951454162598\\
0.805678188800812 23.7638816833496\\
0.830627858638763 24.1211051940918\\
0.855194389820099 24.4676837921143\\
0.879400551319122 24.8043899536133\\
0.903267025947571 25.1319065093994\\
0.926812350749969 25.4508438110352\\
0.950053453445435 25.7617492675781\\
0.973005652427673 26.0651168823242\\
0.995683073997498 26.3613910675049\\
1.01809847354889 26.4664058685303\\
1.0402637720108 26.488109588623\\
1.06237816810608 26.4926452636719\\
1.08448839187622 26.4935989379883\\
1.10659623146057 26.4938049316406\\
1.12870299816132 26.4938545227051\\
1.15080916881561 26.4938697814941\\
1.17291498184204 26.4938774108887\\
1.19502067565918 26.4938831329346\\
1.21712636947632 26.4938888549805\\
1.23923194408417 26.4938926696777\\
1.26133751869202 26.493896484375\\
1.28344309329987 26.4938983917236\\
1.30554866790771 26.4939022064209\\
1.32765424251556 26.4939041137695\\
1.34975969791412 26.4939060211182\\
1.37186527252197 26.4939060211182\\
1.39397084712982 26.4939079284668\\
1.41607642173767 26.4939079284668\\
1.43818199634552 26.4939098358154\\
1.46028745174408 26.4939098358154\\
1.48239302635193 26.4939098358154\\
1.50449860095978 26.4939098358154\\
1.52660417556763 26.4939098358154\\
1.54870975017548 26.4939098358154\\
1.57081520557404 26.4939098358154\\
1.59292078018188 26.4939098358154\\
1.61502635478973 26.4939098358154\\
1.63713192939758 26.4939098358154\\
1.65923750400543 26.4939098358154\\
1.68134295940399 26.4939098358154\\
1.70344853401184 26.4939098358154\\
1.72555410861969 26.4939098358154\\
1.74765968322754 26.4939098358154\\
1.76976525783539 26.4939098358154\\
1.79187071323395 26.4939098358154\\
1.8139762878418 26.4939098358154\\
1.83608186244965 26.4939098358154\\
1.8581874370575 26.4939098358154\\
1.88029301166534 26.4939098358154\\
1.9023984670639 26.4939098358154\\
1.92450404167175 26.4939098358154\\
1.9466096162796 26.4939098358154\\
1.96871519088745 26.4939098358154\\
1.9908207654953 26.4939098358154\\
2.01292634010315 26.4939098358154\\
2.03503179550171 26.4939098358154\\
2.05713748931885 26.4939079284668\\
2.07924294471741 26.4939079284668\\
2.10134840011597 26.4939079284668\\
2.12345409393311 26.4939079284668\\
2.14555954933167 26.4939079284668\\
2.1676652431488 26.4939079284668\\
2.18977069854736 26.4939079284668\\
2.21187615394592 26.4939079284668\\
2.23398184776306 26.4939079284668\\
2.25608730316162 26.4939079284668\\
2.27819299697876 26.4939079284668\\
2.30029845237732 26.4939079284668\\
2.32240390777588 26.4939079284668\\
2.34450960159302 26.4939079284668\\
2.36661505699158 26.4939079284668\\
2.38872075080872 26.4939079284668\\
2.41082620620728 26.4939079284668\\
2.43293166160584 26.4939079284668\\
2.45503735542297 26.4939079284668\\
2.47714281082153 26.4939079284668\\
2.49924850463867 26.4939079284668\\
2.52135396003723 26.4939079284668\\
2.54345941543579 26.4939079284668\\
2.56556510925293 26.4939079284668\\
2.58767056465149 26.4939079284668\\
2.60977625846863 26.4939079284668\\
2.63188171386719 26.4939079284668\\
2.65398716926575 26.4939079284668\\
2.67609286308289 26.4939079284668\\
2.69819831848145 26.4939079284668\\
2.72030401229858 26.4939079284668\\
2.74240946769714 26.4939079284668\\
2.7645149230957 26.4939079284668\\
2.78662061691284 26.4939079284668\\
2.8087260723114 26.4939079284668\\
2.83083176612854 26.4939079284668\\
2.8529372215271 26.4939079284668\\
2.87504267692566 26.4939079284668\\
2.8971483707428 26.4939079284668\\
2.91925382614136 26.4939079284668\\
2.9413595199585 26.4939079284668\\
2.96346497535706 26.4939079284668\\
2.98557043075562 26.4939079284668\\
3.00767612457275 26.4939079284668\\
3.02978157997131 26.4939079284668\\
3.0519700050354 26.4939079284668\\
3.07440447807312 26.4939079284668\\
3.09709739685059 26.4939079284668\\
3.12005996704102 26.4939079284668\\
3.14330291748047 26.4939079284668\\
3.166836977005 26.4939079284668\\
3.19067335128784 26.4939079284668\\
3.21482396125793 26.4939079284668\\
3.2393012046814 26.4939079284668\\
3.26411890983582 26.4939079284668\\
3.28929138183594 26.4939079284668\\
3.31483387947083 26.4939079284668\\
3.34076333045959 26.4939079284668\\
3.36709761619568 26.4939079284668\\
3.39385652542114 26.4939079284668\\
3.42106103897095 26.4939079284668\\
3.44873404502869 26.4939079284668\\
3.47690057754517 26.4939079284668\\
3.50558733940125 26.4939079284668\\
3.53482460975647 26.4939079284668\\
3.5646448135376 26.4939079284668\\
3.59508395195007 26.4939079284668\\
3.62618231773376 26.4939079284668\\
3.65798377990723 26.4939079284668\\
3.69053816795349 26.4939079284668\\
3.72390079498291 26.4939079284668\\
3.75813484191895 26.4939079284668\\
3.79331135749817 26.4939079284668\\
3.82951164245605 26.4939079284668\\
3.86682963371277 26.4939079284668\\
3.90537428855896 26.4939079284668\\
3.94527339935303 26.4939079284668\\
3.98667812347412 26.4939079284668\\
4.02976989746094 26.4939079284668\\
4.07350301742554 26.4939079284668\\
4.11750602722168 26.4939079284668\\
4.16167545318604 26.4939079284668\\
4.2058801651001 26.4939079284668\\
4.2500901222229 26.4939079284668\\
4.29430055618286 26.4939079284668\\
4.33851099014282 26.4939079284668\\
4.38272142410278 26.4939079284668\\
4.4269323348999 26.4939079284668\\
4.47114324569702 26.4939079284668\\
4.51535367965698 26.4939079284668\\
4.5595645904541 26.4939079284668\\
4.60377550125122 26.4939079284668\\
4.64798641204834 26.4939079284668\\
4.69219732284546 26.4939079284668\\
4.73640823364258 26.4939079284668\\
4.7806191444397 26.4939079284668\\
4.82482957839966 26.4939079284668\\
4.86904048919678 26.4939079284668\\
4.9132513999939 26.4939079284668\\
4.95746231079102 26.4939079284668\\
5.00167322158813 26.4939079284668\\
5.04588413238525 26.4939079284668\\
5.09009504318237 26.4939079284668\\
5.13430547714233 26.4939079284668\\
5.17851638793945 26.4939079284668\\
5.22272729873657 26.4939079284668\\
5.26693820953369 26.4939079284668\\
5.31114912033081 26.4939079284668\\
5.35536003112793 26.4939079284668\\
5.39957046508789 26.4939079284668\\
5.44378137588501 26.4939079284668\\
5.48799228668213 26.4939079284668\\
5.53220319747925 26.4939079284668\\
5.57641410827637 26.4939079284668\\
5.62062501907349 26.4939079284668\\
5.66483592987061 26.4939079284668\\
5.70904636383057 26.4939079284668\\
5.75325727462769 26.4939079284668\\
5.7974681854248 26.4939079284668\\
5.84167909622192 26.4939079284668\\
5.88589000701904 26.4939079284668\\
5.93010091781616 26.4939079284668\\
5.97431182861328 26.4939079284668\\
6.01852226257324 26.4939079284668\\
6.06273317337036 26.4939079284668\\
6.10694408416748 26.4939079284668\\
6.1511549949646 26.4939079284668\\
6.19536590576172 26.4939079284668\\
6.23957681655884 26.4939079284668\\
6.28378772735596 26.4939079284668\\
6.32799816131592 26.4939079284668\\
6.37220907211304 26.4939079284668\\
6.41641998291016 26.4939079284668\\
6.46063089370728 26.4939079284668\\
6.50484180450439 26.4939079284668\\
6.54905271530151 26.4939079284668\\
6.59326362609863 26.4939079284668\\
6.63747406005859 26.4939079284668\\
6.68168497085571 26.4939079284668\\
6.72589588165283 26.4939079284668\\
6.77010679244995 26.4939079284668\\
6.81431770324707 26.4939079284668\\
6.85852861404419 26.4939079284668\\
6.90273904800415 26.4939079284668\\
6.94694995880127 26.4939079284668\\
6.99116086959839 26.4939079284668\\
7.03537178039551 26.4939079284668\\
7.07958269119263 26.4939079284668\\
7.12379360198975 26.4939079284668\\
7.16800451278687 26.4939079284668\\
7.21221494674683 26.4939079284668\\
7.25642585754395 26.4939079284668\\
7.30063676834106 26.4939079284668\\
7.34484767913818 26.4939079284668\\
7.3890585899353 26.4939079284668\\
7.43326950073242 26.4939079284668\\
7.47748041152954 26.4939079284668\\
7.5216908454895 26.4939079284668\\
7.56590175628662 26.4939079284668\\
7.61011266708374 26.4939079284668\\
7.65432357788086 26.4939079284668\\
7.69853448867798 26.4939079284668\\
7.7427453994751 26.4939079284668\\
7.78695631027222 26.4939079284668\\
7.83116674423218 26.4939079284668\\
7.8753776550293 26.4939079284668\\
7.91958856582642 26.4939079284668\\
7.96379947662354 26.4939079284668\\
8.0080099105835 26.4939079284668\\
8.05222129821777 26.4939079284668\\
8.09643173217773 26.4939079284668\\
8.14064311981201 26.4939079284668\\
8.18485355377197 26.4939079284668\\
8.22906494140625 26.4939079284668\\
8.27327537536621 26.4939079284668\\
8.31748580932617 26.4939079284668\\
8.36169719696045 26.4939079284668\\
8.40590763092041 26.4939079284668\\
8.45011901855469 26.4939079284668\\
8.49432945251465 26.4939079284668\\
8.53854084014893 26.4939079284668\\
8.58275127410889 26.4939079284668\\
8.62696170806885 26.4939079284668\\
8.67117309570313 26.4939079284668\\
8.71538352966309 26.4939079284668\\
8.75959491729736 26.4939079284668\\
8.80380535125732 26.4939079284668\\
8.8480167388916 26.4939079284668\\
8.89222717285156 26.4939079284668\\
8.93643760681152 26.4939079284668\\
8.9806489944458 26.4939079284668\\
9.02485942840576 26.4939079284668\\
9.06907081604004 26.4939079284668\\
9.11328125 26.4939079284668\\
9.15749168395996 26.4939079284668\\
9.20170307159424 26.4939079284668\\
9.2459135055542 26.4939079284668\\
9.29012489318848 26.4939079284668\\
9.33433532714844 26.4939079284668\\
9.37854671478271 26.4939079284668\\
9.42275714874268 26.4939079284668\\
9.46696758270264 26.4939079284668\\
9.51117897033691 26.4939079284668\\
9.55538940429688 26.4939079284668\\
9.59960079193115 26.4939079284668\\
9.64381122589111 26.4939079284668\\
9.68802261352539 26.4939079284668\\
9.73223304748535 26.4939079284668\\
9.77644348144531 26.4939079284668\\
9.82065486907959 26.4939079284668\\
9.86486530303955 26.4939079284668\\
9.90907669067383 26.4939079284668\\
9.95328712463379 26.4939079284668\\
9.99749851226807 26.4939079284668\\
10.041708946228 26.4939079284668\\
10.085919380188 26.4939079284668\\
10.1301307678223 26.4939079284668\\
10.1743412017822 26.4939079284668\\
10.2185525894165 26.4939079284668\\
10.2627630233765 26.4939079284668\\
10.3069744110107 26.4939079284668\\
10.3511848449707 26.4939079284668\\
10.3953952789307 26.4939079284668\\
10.4396066665649 26.4939079284668\\
10.4838171005249 26.4939079284668\\
10.5280284881592 26.4939079284668\\
10.5722389221191 26.4939079284668\\
10.6164503097534 26.4939079284668\\
10.6606607437134 26.4939079284668\\
10.7048711776733 26.4939079284668\\
10.7490825653076 26.4939079284668\\
10.7932929992676 26.4939079284668\\
10.8375043869019 26.4939079284668\\
10.8817148208618 26.4939079284668\\
10.9259262084961 26.4939079284668\\
10.9701366424561 26.4939079284668\\
11.014347076416 26.4939079284668\\
11.0585584640503 26.4939079284668\\
11.1027688980103 26.4939079284668\\
11.1469802856445 26.4939079284668\\
11.1911907196045 26.4939079284668\\
11.2354021072388 26.4939079284668\\
11.2796125411987 26.4939079284668\\
11.3238229751587 26.4939079284668\\
11.368034362793 26.4939079284668\\
11.4122447967529 26.4939079284668\\
11.4561223983765 26.4939079284668\\
11.5 26.4939079284668\\
11.54421043396 26.4939079284668\\
11.5884218215942 26.4939079284668\\
11.6326322555542 26.4939079284668\\
11.6768436431885 26.4939079284668\\
11.7210540771484 26.4939079284668\\
11.7652654647827 26.4939079284668\\
11.8094758987427 26.4939079284668\\
11.8536863327026 26.4939079284668\\
11.8978977203369 26.4939079284668\\
11.9421081542969 26.4939079284668\\
11.9863195419312 26.4939079284668\\
12.0305299758911 26.4939079284668\\
12.0747413635254 26.4939079284668\\
12.1189517974854 26.4939079284668\\
12.1631622314453 26.4939079284668\\
12.2073736190796 26.4939079284668\\
12.2515840530396 26.4939079284668\\
12.2957954406738 26.4939079284668\\
12.3400058746338 26.4939079284668\\
12.3842172622681 26.4939079284668\\
12.428427696228 26.4939079284668\\
12.472638130188 26.4939079284668\\
12.5168495178223 26.4939079284668\\
12.5610599517822 26.4939079284668\\
12.6052713394165 26.4939079284668\\
12.6494817733765 26.4939079284668\\
12.6936931610107 26.4939079284668\\
12.7379035949707 26.4939079284668\\
12.7821140289307 26.4939079284668\\
12.8263254165649 26.4939079284668\\
12.8705358505249 26.4939079284668\\
12.9147472381592 26.4939079284668\\
12.9589576721191 26.4939079284668\\
13.0031690597534 26.4939079284668\\
13.0473794937134 26.4939079284668\\
13.0915899276733 26.4939079284668\\
13.1358013153076 26.4939079284668\\
13.1800117492676 26.4939079284668\\
13.2242231369019 26.4939079284668\\
13.2684335708618 26.4939079284668\\
13.3126440048218 26.4939079284668\\
13.3568553924561 26.4939079284668\\
13.401065826416 26.4939079284668\\
13.4452772140503 26.4939079284668\\
13.4894876480103 26.4939079284668\\
13.5336990356445 26.4939079284668\\
13.5779094696045 26.4939079284668\\
13.6221199035645 26.4939079284668\\
13.6663312911987 26.4939079284668\\
13.7105417251587 26.4939079284668\\
13.754753112793 26.4939079284668\\
13.7989635467529 26.4939079284668\\
13.8431749343872 26.4939079284668\\
13.8873853683472 26.4939079284668\\
13.9315958023071 26.4939079284668\\
13.9758071899414 26.4939079284668\\
14.0200176239014 26.4939079284668\\
14.0642290115356 26.4939079284668\\
14.1084394454956 26.4939079284668\\
14.1526508331299 26.4939079284668\\
14.1968612670898 26.4939079284668\\
14.2410717010498 26.4939079284668\\
14.2852830886841 26.4939079284668\\
14.329493522644 26.4939079284668\\
14.3737049102783 26.4939079284668\\
14.4179153442383 26.4939079284668\\
14.4621267318726 26.4939079284668\\
14.5063371658325 26.4939079284668\\
14.5505475997925 26.4939079284668\\
14.5947589874268 26.4939079284668\\
14.6389694213867 26.4939079284668\\
14.683180809021 26.4939079284668\\
14.727391242981 26.4939079284668\\
14.7716026306152 26.4939079284668\\
14.8158130645752 26.4939079284668\\
14.8600234985352 26.4939079284668\\
14.9042348861694 26.4939079284668\\
14.9484453201294 26.4939079284668\\
14.9926567077637 26.4939079284668\\
15.0368671417236 26.4939079284668\\
15.0810785293579 26.4939079284668\\
15.1252889633179 26.4939079284668\\
15.1694993972778 26.4939079284668\\
15.2137107849121 26.4939079284668\\
15.2579212188721 26.4939079284668\\
15.3021326065063 26.4939079284668\\
15.3463430404663 26.4939079284668\\
15.3905544281006 26.4939079284668\\
15.4347648620605 26.4939079284668\\
15.4789752960205 26.4939079284668\\
15.5231866836548 26.4939079284668\\
15.5673971176147 26.4939079284668\\
15.611608505249 26.4939079284668\\
15.655818939209 26.4939079284668\\
15.7000303268433 26.4939079284668\\
15.7442407608032 26.4939079284668\\
15.7884511947632 26.4939079284668\\
15.8326625823975 26.4939079284668\\
15.8768730163574 26.4939079284668\\
15.9210844039917 26.4939079284668\\
15.9652948379517 26.4939079284668\\
16.0095062255859 26.4939079284668\\
16.0537166595459 26.4939079284668\\
16.0967807769775 26.4939079284668\\
16.1407527923584 26.4939079284668\\
16.1849117279053 26.4939079284668\\
16.2291107177734 26.4939079284668\\
16.2733192443848 26.4939079284668\\
16.3175296783447 26.4939079284668\\
16.3617401123047 26.4939079284668\\
16.4059524536133 26.4939079284668\\
16.4501628875732 26.4939079284668\\
16.4943733215332 26.4939079284668\\
16.5385837554932 26.4939079284668\\
16.5827941894531 26.4939079284668\\
16.6270065307617 26.4939079284668\\
16.6712169647217 26.4939079284668\\
16.7154273986816 26.4939079284668\\
16.7596378326416 26.4939079284668\\
16.8038482666016 26.4939079284668\\
16.8480606079102 26.4939079284668\\
16.8922710418701 26.4939079284668\\
16.9364814758301 26.4939079284668\\
16.98069190979 26.4939079284668\\
17.0249042510986 26.4939079284668\\
17.0691146850586 26.4939079284668\\
17.1133251190186 26.4939079284668\\
17.1575355529785 26.4939079284668\\
17.2017459869385 26.4939079284668\\
17.2459583282471 26.4939079284668\\
17.290168762207 26.4939079284668\\
17.334379196167 26.4939079284668\\
17.378589630127 26.4939079284668\\
17.4228000640869 26.4939079284668\\
17.4670124053955 26.4939079284668\\
17.5112228393555 26.4939079284668\\
17.5554332733154 26.4939079284668\\
17.5996437072754 26.4939079284668\\
17.643856048584 26.4939079284668\\
17.6880664825439 26.4939079284668\\
17.7322769165039 26.4939079284668\\
17.7764873504639 26.4939079284668\\
17.8206977844238 26.4939079284668\\
17.8649101257324 26.4939079284668\\
17.9091205596924 26.4939079284668\\
17.9533309936523 26.4939079284668\\
17.9975414276123 26.4939079284668\\
18.0417518615723 26.4939079284668\\
18.0859642028809 26.4939079284668\\
18.1301746368408 26.4939079284668\\
18.1743850708008 26.4939079284668\\
18.2185955047607 26.4939079284668\\
18.2628059387207 26.4939079284668\\
18.3070182800293 26.4939079284668\\
18.3512287139893 26.4939079284668\\
18.3954391479492 26.4939079284668\\
18.4396495819092 26.4939079284668\\
18.4838619232178 26.4939079284668\\
18.5280723571777 26.4939079284668\\
18.5722827911377 26.4939079284668\\
18.6164932250977 26.4939079284668\\
18.6607036590576 26.4939079284668\\
18.7049160003662 26.4939079284668\\
18.7491264343262 26.4939079284668\\
18.7933368682861 26.4939079284668\\
18.8375473022461 26.4939079284668\\
18.8817577362061 26.4939079284668\\
18.9259700775146 26.4939079284668\\
18.962984085083 26.4939079284668\\
19 26.4939079284668\\
};
\addlegendentry{$A_{r} = 0.25$};

\addplot [
color=black,
dotted
]
table[row sep=crcr]{
0 0\\
9.99999974737875e-06 0\\
2.49999993684469e-05 0\\
4.75000015285332e-05 0\\
8.12500002211891e-05 0\\
0.000131875000079162 0\\
0.000207812496228144 0\\
0.000321718747727573 0\\
0.000492578139528632 0\\
0.00074886716902256 0\\
0.00113330082967877 0\\
0.00170995120424777 0\\
0.00257492670789361 0\\
0.00387239013798535 0\\
0.00529959984123707 0\\
0.00686953077092767 0\\
0.0085964547470212 0\\
0.0104960706084967 0\\
0.0125856483355165 0\\
0.014884184114635 0\\
0.0174125730991364 0\\
0.020193800330162 0\\
0.0232531521469355 0\\
0.0266184378415346 0\\
0.0303202513605356 0\\
0.0343922488391399 0\\
0.0388714447617531 0\\
0.0437985584139824 0\\
0.0492183864116669 0\\
0.0551801957190037 0\\
0.0617381855845451 0\\
0.0689519718289375 0\\
0.0768871381878853 0\\
0.0856158286333084 0\\
0.095217376947403 0\\
0.105779089033604 0\\
0.117396965622902 0\\
0.13017663359642 0\\
0.144234269857407 0\\
0.159697666764259 0\\
0.176707401871681 0\\
0.195418119430542 0\\
0.215999901294708 0\\
0.238639861345291 0\\
0.2635438144207 0\\
0.290938168764114 0\\
0.321071952581406 0\\
0.354219108819962 0\\
0.390680998563766 0\\
0.427134186029434 0\\
0.461836785078049 0\\
0.495115727186203 0\\
0.527189493179321 0\\
0.558222532272339 0\\
0.588342726230621 0\\
0.617652356624603 0\\
0.646234631538391 0\\
0.674158751964569 0\\
0.701483011245728 0\\
0.728256940841675 0\\
0.754523456096649 0\\
0.780319631099701 0\\
0.805678188800812 0\\
0.830627858638763 0\\
0.855194389820099 0\\
0.879400551319122 0\\
0.903267025947571 0\\
0.926812350749969 0\\
0.950053453445435 0\\
0.973005652427673 0\\
0.995683073997498 0\\
1.01809847354889 0\\
1.0402637720108 0\\
1.06237816810608 0\\
1.08448839187622 0\\
1.10659623146057 0\\
1.12870299816132 0\\
1.15080916881561 0\\
1.17291498184204 0\\
1.19502067565918 0\\
1.21712636947632 0\\
1.23923194408417 0\\
1.26133751869202 0\\
1.28344309329987 0\\
1.30554866790771 0\\
1.32765424251556 0\\
1.34975969791412 0\\
1.37186527252197 0\\
1.39397084712982 0\\
1.41607642173767 0\\
1.43818199634552 0\\
1.46028745174408 0\\
1.48239302635193 0\\
1.50449860095978 0\\
1.52660417556763 0\\
1.54870975017548 0\\
1.57081520557404 0\\
1.59292078018188 0\\
1.61502635478973 0\\
1.63713192939758 0\\
1.65923750400543 0\\
1.68134295940399 0\\
1.70344853401184 0\\
1.72555410861969 0\\
1.74765968322754 0\\
1.76976525783539 0\\
1.79187071323395 0\\
1.8139762878418 0\\
1.83608186244965 0\\
1.8581874370575 0\\
1.88029301166534 0\\
1.9023984670639 0\\
1.92450404167175 0\\
1.9466096162796 0\\
1.96871519088745 0\\
1.9908207654953 0\\
2.01292634010315 0\\
2.03503179550171 0\\
2.05713748931885 0\\
2.07924294471741 0\\
2.10134840011597 0\\
2.12345409393311 0\\
2.14555954933167 0\\
2.1676652431488 0\\
2.18977069854736 0\\
2.21187615394592 0\\
2.23398184776306 0\\
2.25608730316162 0\\
2.27819299697876 0\\
2.30029845237732 0\\
2.32240390777588 0\\
2.34450960159302 0\\
2.36661505699158 0\\
2.38872075080872 0\\
2.41082620620728 0\\
2.43293166160584 0\\
2.45503735542297 0\\
2.47714281082153 0\\
2.49924850463867 0\\
2.52135396003723 0\\
2.54345941543579 0\\
2.56556510925293 0\\
2.58767056465149 0\\
2.60977625846863 0\\
2.63188171386719 0\\
2.65398716926575 0\\
2.67609286308289 0\\
2.69819831848145 0\\
2.72030401229858 0\\
2.74240946769714 0\\
2.7645149230957 0\\
2.78662061691284 0\\
2.8087260723114 0\\
2.83083176612854 0\\
2.8529372215271 0\\
2.87504267692566 0\\
2.8971483707428 0\\
2.91925382614136 0\\
2.9413595199585 0\\
2.96346497535706 0\\
2.98557043075562 0\\
3.00767612457275 0\\
3.02978157997131 0\\
3.0519700050354 0\\
3.07440447807312 0\\
3.09709739685059 0\\
3.12005996704102 0\\
3.14330291748047 0\\
3.166836977005 0\\
3.19067335128784 0\\
3.21482396125793 0\\
3.2393012046814 0\\
3.26411890983582 0\\
3.28929138183594 0\\
3.31483387947083 0\\
3.34076333045959 0\\
3.36709761619568 0\\
3.39385652542114 0\\
3.42106103897095 0\\
3.44873404502869 0\\
3.47690057754517 0\\
3.50558733940125 0\\
3.53482460975647 0\\
3.5646448135376 0\\
3.59508395195007 0\\
3.62618231773376 0\\
3.65798377990723 0\\
3.69053816795349 0\\
3.72390079498291 0\\
3.75813484191895 0\\
3.79331135749817 0\\
3.82951164245605 0\\
3.86682963371277 0\\
3.90537428855896 0\\
3.94527339935303 0\\
3.98667812347412 0\\
4.02976989746094 0\\
4.07350301742554 0\\
4.11750602722168 0\\
4.16167545318604 0\\
4.2058801651001 0\\
4.2500901222229 0\\
4.29430055618286 0\\
4.33851099014282 0\\
4.38272142410278 0\\
4.4269323348999 0\\
4.47114324569702 0\\
4.51535367965698 0\\
4.5595645904541 0\\
4.60377550125122 0\\
4.64798641204834 0\\
4.69219732284546 0\\
4.73640823364258 0\\
4.7806191444397 0\\
4.82482957839966 0\\
4.86904048919678 0\\
4.9132513999939 0\\
4.95746231079102 0\\
5.00167322158813 0\\
5.04588413238525 0\\
5.09009504318237 0\\
5.13430547714233 0\\
5.17851638793945 0\\
5.22272729873657 0\\
5.26693820953369 0\\
5.31114912033081 0\\
5.35536003112793 0\\
5.39957046508789 0\\
5.44378137588501 0\\
5.48799228668213 0\\
5.53220319747925 0\\
5.57641410827637 0\\
5.62062501907349 0\\
5.66483592987061 0\\
5.70904636383057 0\\
5.75325727462769 0\\
5.7974681854248 0\\
5.84167909622192 0\\
5.88589000701904 0\\
5.93010091781616 0\\
5.97431182861328 0\\
6.01852226257324 0\\
6.06273317337036 0\\
6.10694408416748 0\\
6.1511549949646 0\\
6.19536590576172 0\\
6.23957681655884 0\\
6.28378772735596 0\\
6.32799816131592 0\\
6.37220907211304 0\\
6.41641998291016 0\\
6.46063089370728 0\\
6.50484180450439 0\\
6.54905271530151 0\\
6.59326362609863 0\\
6.63747406005859 0\\
6.68168497085571 0\\
6.72589588165283 0\\
6.77010679244995 0\\
6.81431770324707 0\\
6.85852861404419 0\\
6.90273904800415 0\\
6.94694995880127 0\\
6.99116086959839 0\\
7.03537178039551 0\\
7.07958269119263 0\\
7.12379360198975 0\\
7.16800451278687 0\\
7.21221494674683 0\\
7.25642585754395 0\\
7.30063676834106 0\\
7.34484767913818 0\\
7.3890585899353 0\\
7.43326950073242 0\\
7.47748041152954 0\\
7.5216908454895 0\\
7.56590175628662 0\\
7.61011266708374 0\\
7.65432357788086 0\\
7.69853448867798 0\\
7.7427453994751 0\\
7.78695631027222 0\\
7.83116674423218 0\\
7.8753776550293 0\\
7.91958856582642 0\\
7.96379947662354 0\\
8.0080099105835 0\\
8.05222129821777 0\\
8.09643173217773 0\\
8.14064311981201 0\\
8.18485355377197 0\\
8.22906494140625 0\\
8.27327537536621 0\\
8.31748580932617 0\\
8.36169719696045 0\\
8.40590763092041 0\\
8.45011901855469 0\\
8.49432945251465 0\\
8.53854084014893 0\\
8.58275127410889 0\\
8.62696170806885 0\\
8.67117309570313 0\\
8.71538352966309 0\\
8.75959491729736 0\\
8.80380535125732 0\\
8.8480167388916 0\\
8.89222717285156 0\\
8.93643760681152 0\\
8.9806489944458 0\\
9.02485942840576 0\\
9.06907081604004 0\\
9.11328125 0\\
9.15749168395996 0\\
9.20170307159424 0\\
9.2459135055542 0\\
9.29012489318848 0\\
9.33433532714844 0\\
9.37854671478271 0\\
9.42275714874268 0\\
9.46696758270264 0\\
9.51117897033691 0\\
9.55538940429688 0\\
9.59960079193115 0\\
9.64381122589111 0\\
9.68802261352539 0\\
9.73223304748535 0\\
9.77644348144531 0\\
9.82065486907959 0\\
9.86486530303955 0\\
9.90907669067383 0\\
9.95328712463379 0\\
9.99749851226807 0\\
10.041708946228 0\\
10.085919380188 0\\
10.1301307678223 0\\
10.1743412017822 0\\
10.2185525894165 0\\
10.2627630233765 0\\
10.3069744110107 0\\
10.3511848449707 0\\
10.3953952789307 0\\
10.4396066665649 0\\
10.4838171005249 0\\
10.5280284881592 0\\
10.5722389221191 0\\
10.6164503097534 0\\
10.6606607437134 0\\
10.7048711776733 0\\
10.7490825653076 0\\
10.7932929992676 0\\
10.8375043869019 0\\
10.8817148208618 0\\
10.9259262084961 0\\
10.9701366424561 0\\
11.014347076416 0\\
11.0585584640503 0\\
11.1027688980103 0\\
11.1469802856445 0\\
11.1911907196045 0\\
11.2354021072388 0\\
11.2796125411987 0\\
11.3238229751587 0\\
11.368034362793 0\\
11.4122447967529 0\\
11.4561223983765 0\\
11.5 0\\
11.54421043396 0\\
11.5884218215942 0\\
11.6326322555542 0\\
11.6768436431885 0\\
11.7210540771484 0\\
11.7652654647827 0\\
11.8094758987427 0\\
11.8536863327026 0\\
11.8978977203369 0\\
11.9421081542969 0\\
11.9863195419312 0\\
12.0305299758911 0\\
12.0747413635254 0\\
12.1189517974854 0\\
12.1631622314453 0\\
12.2073736190796 0\\
12.2515840530396 0\\
12.2957954406738 0\\
12.3400058746338 0\\
12.3842172622681 0\\
12.428427696228 0\\
12.472638130188 0\\
12.5168495178223 0\\
12.5610599517822 0\\
12.6052713394165 0\\
12.6494817733765 0\\
12.6936931610107 0\\
12.7379035949707 0\\
12.7821140289307 0\\
12.8263254165649 0\\
12.8705358505249 0\\
12.9147472381592 0\\
12.9589576721191 0\\
13.0031690597534 0\\
13.0473794937134 0\\
13.0915899276733 0\\
13.1358013153076 0\\
13.1800117492676 0\\
13.2242231369019 0\\
13.2684335708618 0\\
13.3126440048218 0\\
13.3568553924561 0\\
13.401065826416 0\\
13.4452772140503 0\\
13.4894876480103 0\\
13.5336990356445 0\\
13.5779094696045 0\\
13.6221199035645 0\\
13.6663312911987 0\\
13.7105417251587 0\\
13.754753112793 0\\
13.7989635467529 0\\
13.8431749343872 0\\
13.8873853683472 0\\
13.9315958023071 0\\
13.9758071899414 0\\
14.0200176239014 0\\
14.0642290115356 0\\
14.1084394454956 0\\
14.1526508331299 0\\
14.1968612670898 0\\
14.2410717010498 0\\
14.2852830886841 0\\
14.329493522644 0\\
14.3737049102783 0\\
14.4179153442383 0\\
14.4621267318726 0\\
14.5063371658325 0\\
14.5505475997925 0\\
14.5947589874268 0\\
14.6389694213867 0\\
14.683180809021 0\\
14.727391242981 0\\
14.7716026306152 0\\
14.8158130645752 0\\
14.8600234985352 0\\
14.9042348861694 0\\
14.9484453201294 0\\
14.9926567077637 0\\
15.0368671417236 0\\
15.0810785293579 0\\
15.1252889633179 0\\
15.1694993972778 0\\
15.2137107849121 0\\
15.2579212188721 0\\
15.3021326065063 0\\
15.3463430404663 0\\
15.3905544281006 0\\
15.4347648620605 0\\
15.4789752960205 0\\
15.5231866836548 0\\
15.5673971176147 0\\
15.611608505249 0\\
15.655818939209 0\\
15.7000303268433 0\\
15.7442407608032 0\\
15.7884511947632 0\\
15.8326625823975 0\\
15.8768730163574 0\\
15.9210844039917 0\\
15.9652948379517 0\\
16.0095062255859 0\\
16.0537166595459 0\\
16.0967807769775 0\\
16.1407527923584 0\\
16.1849117279053 0\\
16.2291107177734 0\\
16.2733192443848 0\\
16.3175296783447 0\\
16.3617401123047 0\\
16.4059524536133 0\\
16.4501628875732 0\\
16.4943733215332 0\\
16.5385837554932 0\\
16.5827941894531 0\\
16.6270065307617 0\\
16.6712169647217 0\\
16.7154273986816 0\\
16.7596378326416 0\\
16.8038482666016 0\\
16.8480606079102 0\\
16.8922710418701 0\\
16.9364814758301 0\\
16.98069190979 0\\
17.0249042510986 0\\
17.0691146850586 0\\
17.1133251190186 0\\
17.1575355529785 0\\
17.2017459869385 0\\
17.2459583282471 0\\
17.290168762207 0\\
17.334379196167 0\\
17.378589630127 0\\
17.4228000640869 0\\
17.4670124053955 0\\
17.5112228393555 0\\
17.5554332733154 0\\
17.5996437072754 0\\
17.643856048584 0\\
17.6880664825439 0\\
17.7322769165039 0\\
17.7764873504639 0\\
17.8206977844238 0\\
17.8649101257324 0\\
17.9091205596924 0\\
17.9533309936523 0\\
17.9975414276123 0\\
18.0417518615723 0\\
18.0859642028809 0\\
18.1301746368408 0\\
18.1743850708008 0\\
18.2185955047607 0\\
18.2628059387207 0\\
18.3070182800293 0\\
18.3512287139893 0\\
18.3954391479492 0\\
18.4396495819092 0\\
18.4838619232178 0\\
18.5280723571777 0\\
18.5722827911377 0\\
18.6164932250977 0\\
18.6607036590576 0\\
18.7049160003662 0\\
18.7491264343262 0\\
18.7933368682861 0\\
18.8375473022461 0\\
18.8817577362061 0\\
18.9259700775146 0\\
18.962984085083 0\\
19 0\\
};
\addlegendentry{$A_{r} = 0.00$};

\end{axis}
\end{tikzpicture}%
\caption{Valve problem with \dtmax{} = \expneg{6.25}{2}{[s]} with the nonlinear solver.}
\label{fig:valveNln6pt25em02}
\end{figure}

The solution obtained at \dtmax{} = \expneg{6.25}{2}{[s]} does not exhibit the nonphysical spike in the flow rate that the linear solution displays.
This solution is also obtained at the largest \dt{} run in the study, \expneg{1.0}{0}{[s]}.
Reducing the \dtmax{} produces solutions that are qualitatively identical to that shown above.
\fig{fig:valveNlnSols} shows the same zoomed view of at 15 [s], when the valve opens.

\begin{figure}[h!tb]
\centering
% This file was created by matlab2tikz v0.4.3.
% Copyright (c) 2008--2013, Nico Schlömer <nico.schloemer@gmail.com>
% All rights reserved.
% 
\tikzsetnextfilename{plots/valveNlnSols_eps}
\begin{tikzpicture}

\begin{axis}[%
width=0.8\textwidth,
height=0.630967741935484\textwidth,
scale only axis,
xmin=14.9,
xmax=15.2,
xlabel={Time $[\text{s}]$},
ymin=-1,
ymax=25,
ylabel={Flow Rate $[ \frac{\lbm{}}{\text{s}} ]$},
legend style={at={(0.03,0.97)},anchor=north west,draw=black,fill=white,legend cell align=left}
]
\addplot [
color=black,
solid
]
table[row sep=crcr]{
14.9042348861694 0\\
14.9484453201294 0\\
14.9926567077637 0\\
15.0368671417236 1.93454301357269\\
15.0810785293579 4.24624395370483\\
15.1252889633179 6.56158781051636\\
15.1694993972778 8.87692451477051\\
};
\addlegendentry{$\dtmax{} = \expneg{6.25}{2}{[s]}$};

\addplot [
color=black,
dash pattern=on 1pt off 3pt on 3pt off 3pt
]
table[row sep=crcr]{
14.9000101089478 0\\
14.9312601089478 0\\
14.9625101089478 0\\
14.9937601089478 0\\
15.0250101089478 1.31288325786591\\
15.0562601089478 2.94648385047913\\
15.0875101089478 4.58305835723877\\
15.1187601089478 6.21962881088257\\
15.1500101089478 7.8562273979187\\
15.1812601089478 9.49282932281494\\
};
\addlegendentry{$\dtmax{} = \expneg{3.125}{2}{[s]}$};

\addplot [
color=black,
dashed
]
table[row sep=crcr]{
14.9059162139893 0\\
14.9215412139893 0\\
14.9371662139893 0\\
14.9527912139893 0\\
14.9684162139893 0\\
14.9840412139893 0\\
14.9996662139893 0\\
15.0152912139893 0.801035642623901\\
15.0309162139893 1.61912643909454\\
15.0465412139893 2.43743634223938\\
15.0621662139893 3.25572896003723\\
15.0777912139893 4.07402896881104\\
15.0934162139893 4.89233016967773\\
15.1090412139893 5.71063137054443\\
15.1246662139893 6.52893209457397\\
15.1402912139893 7.34723329544067\\
15.1559162139893 8.16562271118164\\
15.1715412139893 8.98390865325928\\
15.1871662139893 9.80219173431396\\
};
\addlegendentry{$\dtmax{} = \expneg{1.5625}{2}{[s]}$};

\addplot [
color=black,
dotted
]
table[row sep=crcr]{
14.905800819397 0\\
14.913613319397 0\\
14.921425819397 0\\
14.929238319397 0\\
14.937050819397 0\\
14.944863319397 0\\
14.952675819397 0\\
14.960488319397 0\\
14.968300819397 0\\
14.976113319397 0\\
14.983925819397 0\\
14.991738319397 0\\
14.999550819397 0\\
15.007363319397 0.385915666818619\\
15.015175819397 0.794798016548157\\
15.022988319397 1.2039543390274\\
15.030800819397 1.61310076713562\\
15.038613319397 2.02225089073181\\
15.046425819397 2.43140125274658\\
15.054238319397 2.84055185317993\\
15.062050819397 3.24970245361328\\
15.069863319397 3.65885305404663\\
15.077675819397 4.06804847717285\\
15.085488319397 4.47719144821167\\
15.093300819397 4.88633298873901\\
15.101113319397 5.29547691345215\\
15.108925819397 5.70462226867676\\
15.116738319397 6.11376953125\\
15.124550819397 6.5229172706604\\
15.132363319397 6.93206596374512\\
15.140175819397 7.34121513366699\\
15.147988319397 7.75036430358887\\
15.155800819397 8.15951347351074\\
15.163613319397 8.56866359710693\\
15.171425819397 8.97781372070313\\
15.179238319397 9.38696384429932\\
15.187050819397 9.79611396789551\\
15.194863319397 10.2052640914917\\
};
\addlegendentry{$\dtmax{} = \expneg{7.8125}{3}{[s]}$};

\addplot [
color=green,
solid
]
table[row sep=crcr]{
14.9033002853394 0\\
14.9072065353394 0\\
14.9111127853394 0\\
14.9150190353394 0\\
14.9189252853394 0\\
14.9228315353394 0\\
14.9267377853394 0\\
14.9306440353394 0\\
14.9345502853394 0\\
14.9384565353394 0\\
14.9423627853394 0\\
14.946268081665 0\\
14.950174331665 0\\
14.954080581665 0\\
14.957986831665 0\\
14.961893081665 0\\
14.965799331665 0\\
14.969705581665 0\\
14.973611831665 0\\
14.977518081665 0\\
14.981424331665 0\\
14.985330581665 0\\
14.989236831665 0\\
14.993143081665 0\\
14.997049331665 0\\
15.000955581665 0.0385664850473404\\
15.004861831665 0.254443943500519\\
15.008768081665 0.459155857563019\\
15.012674331665 0.66375058889389\\
15.016580581665 0.868322134017944\\
15.0204858779907 1.07289445400238\\
15.0243921279907 1.27746713161469\\
15.0282983779907 1.48203980922699\\
15.0322046279907 1.68661248683929\\
15.0361108779907 1.8911851644516\\
15.0400171279907 2.09577798843384\\
15.0439233779907 2.30034732818604\\
15.0478296279907 2.50491619110107\\
15.0517358779907 2.70948576927185\\
15.0556421279907 2.91405630111694\\
15.0595483779907 3.11862730979919\\
15.0634546279907 3.3231987953186\\
15.0673608779907 3.52777051925659\\
15.0712671279907 3.73234248161316\\
15.0751733779907 3.93691468238831\\
15.0790796279907 4.14148712158203\\
15.0829858779907 4.34605932235718\\
15.0868921279907 4.55063152313232\\
15.0907983779907 4.75520420074463\\
15.0947036743164 4.95977640151978\\
15.0986099243164 5.16434907913208\\
15.1025161743164 5.36892175674438\\
15.1064224243164 5.57349395751953\\
15.1103286743164 5.77806663513184\\
15.1142349243164 5.98263931274414\\
15.1181411743164 6.18721199035645\\
15.1220474243164 6.39178466796875\\
15.1259536743164 6.5963568687439\\
15.1298599243164 6.8009295463562\\
15.1337661743164 7.00550222396851\\
15.1376724243164 7.21007490158081\\
15.1415786743164 7.41464757919312\\
15.1454849243164 7.61922025680542\\
15.1493911743164 7.82379293441772\\
15.1532974243164 8.02836513519287\\
15.1572036743164 8.23293781280518\\
15.1611099243164 8.43751049041748\\
15.1650161743164 8.64208316802979\\
15.1689214706421 8.84665584564209\\
15.1728277206421 9.05122852325439\\
15.1767339706421 9.2558012008667\\
15.1806402206421 9.460373878479\\
15.1845464706421 9.66494655609131\\
15.1884527206421 9.86951923370361\\
15.1923589706421 10.0740909576416\\
15.1962652206421 10.2786636352539\\
};
\addlegendentry{$\dtmax{} = \expneg{3.9062}{3}{[s]}$};

\addplot [
color=green,
dotted
]
table[row sep=crcr]{
14.900221824646 0\\
14.902174949646 0\\
14.904128074646 0\\
14.906081199646 0\\
14.908034324646 0\\
14.909987449646 0\\
14.911940574646 0\\
14.913893699646 0\\
14.915846824646 0\\
14.9177989959717 0\\
14.9197521209717 0\\
14.9217052459717 0\\
14.9236583709717 0\\
14.9256114959717 0\\
14.9275646209717 0\\
14.9295177459717 0\\
14.9314708709717 0\\
14.9334239959717 0\\
14.9353771209717 0\\
14.9373302459717 0\\
14.9392833709717 0\\
14.9412364959717 0\\
14.9431896209717 0\\
14.9451427459717 0\\
14.9470958709717 0\\
14.9490489959717 0\\
14.9510021209717 0\\
14.9529552459717 0\\
14.9549083709717 0\\
14.9568614959717 0\\
14.9588146209717 0\\
14.9607677459717 0\\
14.9627208709717 0\\
14.9646739959717 0\\
14.9666271209717 0\\
14.9685802459717 0\\
14.9705333709717 0\\
14.9724864959717 0\\
14.9744396209717 0\\
14.9763927459717 0\\
14.9783458709717 0\\
14.9802989959717 0\\
14.9822521209717 0\\
14.9842052459717 0\\
14.9861583709717 0\\
14.9881114959717 0\\
14.9900646209717 0\\
14.9920167922974 0\\
14.9939699172974 0\\
14.9959230422974 0\\
14.9978761672974 0\\
14.9998292922974 0\\
15.0017824172974 0.0934739857912064\\
15.0037355422974 0.195629969239235\\
15.0056886672974 0.297931283712387\\
15.0076417922974 0.400224566459656\\
15.0095949172974 0.502510845661163\\
15.0115480422974 0.604796946048737\\
15.0135011672974 0.707083284854889\\
15.0154542922974 0.809369623661041\\
15.0174074172974 0.911655962467194\\
15.0193605422974 1.01394236087799\\
15.0213136672974 1.11622869968414\\
15.0232667922974 1.2185150384903\\
15.0252199172974 1.32080137729645\\
15.0271730422974 1.4230877161026\\
15.0291261672974 1.52537405490875\\
15.0310792922974 1.6276603937149\\
15.0330324172974 1.72994840145111\\
15.0349855422974 1.83223474025726\\
15.0369386672974 1.93452084064484\\
15.0388917922974 2.0368070602417\\
15.0408449172974 2.13909316062927\\
15.0427980422974 2.24137926101685\\
15.0447511672974 2.343665599823\\
15.0467042922974 2.44595170021057\\
15.0486574172974 2.54823803901672\\
15.0506105422974 2.65052437782288\\
15.0525636672974 2.75281047821045\\
15.0545167922974 2.8550968170166\\
15.0564699172974 2.95738315582275\\
15.0584230422974 3.05966949462891\\
15.0603761672974 3.16195583343506\\
15.0623292922974 3.26424217224121\\
15.0642824172974 3.36652827262878\\
15.066234588623 3.46881461143494\\
15.068187713623 3.57110095024109\\
15.070140838623 3.67338728904724\\
15.072093963623 3.77567362785339\\
15.074047088623 3.87795996665955\\
15.076000213623 3.9802463054657\\
15.077953338623 4.08253288269043\\
15.079906463623 4.18481874465942\\
15.081859588623 4.28710508346558\\
15.083812713623 4.38939142227173\\
15.085765838623 4.49167776107788\\
15.087718963623 4.59396409988403\\
15.089672088623 4.69625043869019\\
15.091625213623 4.79853677749634\\
15.093578338623 4.90082311630249\\
15.095531463623 5.00310945510864\\
15.097484588623 5.10539579391479\\
15.099437713623 5.20768213272095\\
15.101390838623 5.3099684715271\\
15.103343963623 5.41225481033325\\
15.105297088623 5.5145411491394\\
15.107250213623 5.61682748794556\\
15.109203338623 5.71911382675171\\
15.111156463623 5.82140016555786\\
15.113109588623 5.92368650436401\\
15.115062713623 6.02597284317017\\
15.117015838623 6.12825918197632\\
15.118968963623 6.23054552078247\\
15.120922088623 6.33283185958862\\
15.122875213623 6.43511819839478\\
15.124828338623 6.53740453720093\\
15.126781463623 6.63969087600708\\
15.128734588623 6.74197721481323\\
15.130687713623 6.84426355361938\\
15.132640838623 6.94654989242554\\
15.134593963623 7.04883623123169\\
15.136547088623 7.15112257003784\\
15.138500213623 7.25340890884399\\
15.1404523849487 7.35569524765015\\
15.1424055099487 7.4579815864563\\
15.1443586349487 7.56026792526245\\
15.1463117599487 7.6625542640686\\
15.1482648849487 7.76484060287476\\
15.1502180099487 7.86712646484375\\
15.1521711349487 7.9694128036499\\
15.1541242599487 8.07169914245605\\
15.1560773849487 8.17398548126221\\
15.1580305099487 8.27627182006836\\
15.1599836349487 8.37855815887451\\
15.1619367599487 8.48084449768066\\
15.1638898849487 8.58313083648682\\
15.1658430099487 8.68541717529297\\
15.1677961349487 8.78770351409912\\
15.1697492599487 8.88998985290527\\
15.1717023849487 8.99227619171143\\
15.1736555099487 9.09456253051758\\
15.1756086349487 9.19684886932373\\
15.1775617599487 9.29913520812988\\
15.1795148849487 9.40142154693604\\
15.1814680099487 9.50370788574219\\
15.1834211349487 9.60599422454834\\
15.1853742599487 9.70828056335449\\
15.1873273849487 9.81056690216064\\
15.1892805099487 9.9128532409668\\
15.1912336349487 10.0151395797729\\
15.1931867599487 10.1174259185791\\
15.1951398849487 10.2197122573853\\
15.1970930099487 10.3219976425171\\
15.1990461349487 10.4242839813232\\
};
\addlegendentry{$\dtmax{} = \expneg{1.9531}{3}{[s]}$};

\end{axis}
\end{tikzpicture}%
\caption{Zoom of flow rate in nonlinear solutions to the valve problem.}
\label{fig:valveNlnSols}
\end{figure}

These results indicate that the linear solver's timestep refinement serves the purpose of trying to resolve an isolated linearization error in a single channel of the problem.
By using the nonlinear solver, that error is removed and subsequent timestep reductions are not necessary to obtain the physical solution.
However, \tab{tab:valveNlnTable} shows that the nonlinear solver takes more CPU time per timestep than the linear solver.

\begin{table}[h!tb]
\centering
\singlespace
\pgfplotstabletypeset[sci zerofill,sci E, col sep=comma,
	columns/0/.style={ column name= $[ \text{s} ]$, precision=1},
	columns/1/.style={ column name= $[ - ]$, int detect},
	columns/2/.style={ column name= $[ \text{s} ]$, precision=3},
	columns/3/.style={ column name= $[ \text{s} ]$, precision=3},
	every head row/.style={
		before row={
			\toprule
			\dtmax{} & $N_{t}$ & $T_{\text{CPU}}$& $\frac{T_{\text{CPU}}}{N_{t}}$ \\
		},
		after row=\midrule
	},
	every last row/.style={
after row=\bottomrule}]{tables/valveNlnData.tex}

\caption{Run time data for the valve problem using the nonlinear solver.}
\label{tab:valveNlnTable}
\end{table}

Given that the nonlinearities in this problem are clearly isolated in a single channel, using the domain decomposition algorithm provides an ideal test to determine if a nonlinearly converged solution can be obtained at a lower computational cost than the full nonlinear solver.
The use of the domain decomposition algorithm in this problem is restricted to the channel with the valve.
The other three channels are not included in the nonlinear domain.
The same set of timesteps is run to collect data.
\fig{fig:valveDom6pt25em02} shows the solution as obtained with \dtmax{} = \expneg{6.25}{2}{[s]}; the results in this figure are qualitatively indistinguishable from the results of the nonlinear solver.

\begin{figure}[h!tb]
\centering
% This file was created by matlab2tikz v0.4.3.
% Copyright (c) 2008--2013, Nico Schlömer <nico.schloemer@gmail.com>
% All rights reserved.
% 
\tikzsetnextfilename{plots/valveDom6pt2500em02_pdf}
\begin{tikzpicture}

\begin{axis}[%
width=\mytikzpicwidth,
height=\mytikzpicheight,
scale only axis,
xmin=0,
xmax=20,
xlabel={Time $[\text{s}]$},
ymin=-5,
ymax=125,
ylabel={Flow Rate $[ \frac{\lbm{}}{\text{s}} ]$},
legend style={draw=black,fill=white,legend cell align=left}
]
\addplot [
color=black,
solid
]
table[row sep=crcr]{
0 0\\
9.99999974737875e-06 -2.68723052698761e-08\\
2.49999993684469e-05 5.97379596456449e-07\\
4.75000015285332e-05 1.5992671251297e-05\\
8.12500002211891e-05 0.000229224911890924\\
0.000131875000079162 0.00192544842138886\\
0.000207812496228144 0.00898910593241453\\
0.000321718747727573 0.0234754215925932\\
0.000492578139528632 0.0383502058684826\\
0.00074886716902256 0.0502719767391682\\
0.00113330082967877 0.0704958811402321\\
0.00170995120424777 0.108530916273594\\
0.00257492670789361 0.167750313878059\\
0.00387239013798535 0.258952885866165\\
0.00529959984123707 0.363639861345291\\
0.00686953077092767 0.484229981899261\\
0.0085964547470212 0.623396694660187\\
0.0104960706084967 0.784338176250458\\
0.0125856483355165 0.970849931240082\\
0.014884184114635 1.18742954730988\\
0.0174125730991364 1.43940579891205\\
0.020193800330162 1.73308956623077\\
0.0232531521469355 2.07594513893127\\
0.0266184378415346 2.4767849445343\\
0.0303202513605356 2.94598364830017\\
0.0343922488391399 3.49570560455322\\
0.0388714447617531 4.14013338088989\\
0.0437985584139824 4.89567756652832\\
0.0492183864116669 5.78112745285034\\
0.0551801957190037 6.81768989562988\\
0.0617381855845451 8.02883529663086\\
0.0689519718289375 9.43983745574951\\
0.0768871381878853 11.0768632888794\\
0.0856158286333084 12.9654607772827\\
0.095217376947403 15.1283092498779\\
0.105779089033604 17.5821762084961\\
0.117396965622902 20.3342571258545\\
0.13017663359642 23.3783740997314\\
0.144234269857407 26.6919059753418\\
0.159697666764259 30.2346706390381\\
0.176707401871681 33.9509391784668\\
0.195418119430542 37.7751693725586\\
0.215999901294708 41.6409111022949\\
0.238639861345291 45.4909553527832\\
0.2635438144207 49.2860527038574\\
0.290938168764114 53.0097274780273\\
0.321071952581406 56.6680793762207\\
0.354219108819962 60.2851753234863\\
0.390680998563766 63.8959426879883\\
0.427134186029434 67.2247619628906\\
0.461836785078049 70.1926651000977\\
0.495115727186203 72.8892974853516\\
0.527189493179321 75.372917175293\\
0.558222532272339 77.6839752197266\\
0.588342726230621 79.8516006469727\\
0.617652356624603 81.8974990844727\\
0.646234631538391 83.8383636474609\\
0.674158751964569 85.6873397827148\\
0.701483011245728 87.4550094604492\\
0.728256940841675 89.1500778198242\\
0.754523456096649 90.77978515625\\
0.780319631099701 92.3502655029297\\
0.805678188800812 93.8667373657227\\
0.830627858638763 95.3337249755859\\
0.855194389820099 96.7551498413086\\
0.879400551319122 98.1344680786133\\
0.903267025947571 99.4747085571289\\
0.926812350749969 100.778564453125\\
0.950053453445435 102.048461914063\\
0.973005652427673 103.286552429199\\
0.995683073997498 104.494781494141\\
1.01809847354889 105.217460632324\\
1.0402637720108 105.585472106934\\
1.06237816810608 105.774543762207\\
1.08448839187622 105.871826171875\\
1.10659623146057 105.921867370605\\
1.12870299816132 105.947608947754\\
1.15080916881561 105.960845947266\\
1.17291498184204 105.967651367188\\
1.19502067565918 105.971153259277\\
1.21712636947632 105.972946166992\\
1.23923194408417 105.973876953125\\
1.26133751869202 105.974349975586\\
1.28344309329987 105.974594116211\\
1.30554866790771 105.974716186523\\
1.32765424251556 105.974784851074\\
1.34975969791412 105.974815368652\\
1.37186527252197 105.974830627441\\
1.39397084712982 105.974838256836\\
1.41607642173767 105.97484588623\\
1.43818199634552 105.97484588623\\
1.46028745174408 105.974853515625\\
1.48239302635193 105.974853515625\\
1.50449860095978 105.974853515625\\
1.52660417556763 105.974853515625\\
1.54870975017548 105.974853515625\\
1.57081520557404 105.974853515625\\
1.59292078018188 105.974853515625\\
1.61502635478973 105.974853515625\\
1.63713192939758 105.974853515625\\
1.65923750400543 105.974853515625\\
1.68134295940399 105.974853515625\\
1.70344853401184 105.974853515625\\
1.72555410861969 105.974853515625\\
1.74765968322754 105.974853515625\\
1.76976525783539 105.974853515625\\
1.79187071323395 105.974853515625\\
1.8139762878418 105.974853515625\\
1.83608186244965 105.974853515625\\
1.8581874370575 105.974853515625\\
1.88029301166534 105.974853515625\\
1.9023984670639 105.974853515625\\
1.92450404167175 105.974853515625\\
1.9466096162796 105.974853515625\\
1.96871519088745 105.974853515625\\
1.9908207654953 105.974853515625\\
2.01292634010315 105.974853515625\\
2.03503179550171 105.974853515625\\
2.05713748931885 105.974853515625\\
2.07924294471741 105.974853515625\\
2.10134840011597 105.974853515625\\
2.12345409393311 105.974853515625\\
2.14555954933167 105.974853515625\\
2.1676652431488 105.974853515625\\
2.18977069854736 105.974853515625\\
2.21187615394592 105.974853515625\\
2.23398184776306 105.974853515625\\
2.25608730316162 105.974853515625\\
2.27819299697876 105.974853515625\\
2.30029845237732 105.974853515625\\
2.32240390777588 105.974853515625\\
2.34450960159302 105.974853515625\\
2.36661505699158 105.974853515625\\
2.38872075080872 105.974853515625\\
2.41082620620728 105.974853515625\\
2.43293166160584 105.974853515625\\
2.45503735542297 105.974853515625\\
2.47714281082153 105.974853515625\\
2.49924850463867 105.974853515625\\
2.52135396003723 105.974853515625\\
2.54345941543579 105.974853515625\\
2.56556510925293 105.974853515625\\
2.58767056465149 105.974853515625\\
2.60977625846863 105.974853515625\\
2.63188171386719 105.974853515625\\
2.65398716926575 105.974853515625\\
2.67609286308289 105.974853515625\\
2.69819831848145 105.974853515625\\
2.72030401229858 105.974853515625\\
2.74240946769714 105.974853515625\\
2.7645149230957 105.974853515625\\
2.78662061691284 105.974853515625\\
2.8087260723114 105.974853515625\\
2.83083176612854 105.974853515625\\
2.8529372215271 105.974853515625\\
2.87504267692566 105.974853515625\\
2.8971483707428 105.974853515625\\
2.91925382614136 105.974853515625\\
2.9413595199585 105.974853515625\\
2.96346497535706 105.974853515625\\
2.98557043075562 105.974853515625\\
3.00767612457275 105.776901245117\\
3.02978157997131 105.100280761719\\
3.0519700050354 104.171333312988\\
3.07440447807312 103.101539611816\\
3.09709739685059 101.951995849609\\
3.12005996704102 100.754364013672\\
3.14330291748047 99.5247421264648\\
3.166836977005 98.2710571289063\\
3.19067335128784 96.9970092773438\\
3.21482396125793 95.7041015625\\
3.2393012046814 94.392707824707\\
3.26411890983582 93.0626068115234\\
3.28929138183594 91.7132720947266\\
3.31483387947083 90.3440017700195\\
3.34076333045959 88.9539413452148\\
3.36709761619568 87.5421447753906\\
3.39385652542114 86.1075820922852\\
3.42106103897095 84.6491241455078\\
3.44873404502869 83.1655502319336\\
3.47690057754517 81.6555328369141\\
3.50558733940125 80.1175994873047\\
3.53482460975647 78.550163269043\\
3.5646448135376 76.9514694213867\\
3.59508395195007 75.3195953369141\\
3.62618231773376 73.6523895263672\\
3.65798377990723 71.9474716186523\\
3.69053816795349 70.2022018432617\\
3.72390079498291 68.4135818481445\\
3.75813484191895 66.5782623291016\\
3.79331135749817 64.6924133300781\\
3.82951164245605 62.7516708374023\\
3.86682963371277 60.7510070800781\\
3.90537428855896 58.6845817565918\\
3.94527339935303 56.5455436706543\\
3.98667812347412 54.3257865905762\\
4.02976989746094 53.2729530334473\\
4.07350301742554 53.0479316711426\\
4.11750602722168 53.0003852844238\\
4.16167545318604 52.9903907775879\\
4.2058801651001 52.9882926940918\\
4.2500901222229 52.9878540039063\\
4.29430055618286 52.9877624511719\\
4.33851099014282 52.9877433776855\\
4.38272142410278 52.9877395629883\\
4.4269323348999 52.9877395629883\\
4.47114324569702 52.9877395629883\\
4.51535367965698 52.9877395629883\\
4.5595645904541 52.9877395629883\\
4.60377550125122 52.9877395629883\\
4.64798641204834 52.9877395629883\\
4.69219732284546 52.9877395629883\\
4.73640823364258 52.9877395629883\\
4.7806191444397 52.9877395629883\\
4.82482957839966 52.9877395629883\\
4.86904048919678 52.9877395629883\\
4.9132513999939 52.9877395629883\\
4.95746231079102 52.9877395629883\\
5.00167322158813 52.9877395629883\\
5.04588413238525 52.9877395629883\\
5.09009504318237 52.9877395629883\\
5.13430547714233 52.9877395629883\\
5.17851638793945 52.9877395629883\\
5.22272729873657 52.9877395629883\\
5.26693820953369 52.9877395629883\\
5.31114912033081 52.9877395629883\\
5.35536003112793 52.9877395629883\\
5.39957046508789 52.9877395629883\\
5.44378137588501 52.9877395629883\\
5.48799228668213 52.9877395629883\\
5.53220319747925 52.9877395629883\\
5.57641410827637 52.9877395629883\\
5.62062501907349 52.9877395629883\\
5.66483592987061 52.9877395629883\\
5.70904636383057 52.9877395629883\\
5.75325727462769 52.9877395629883\\
5.7974681854248 52.9877395629883\\
5.84167909622192 52.9877395629883\\
5.88589000701904 52.9877395629883\\
5.93010091781616 52.9877395629883\\
5.97431182861328 52.9877395629883\\
6.01852226257324 52.9877395629883\\
6.06273317337036 52.9877395629883\\
6.10694408416748 52.9877395629883\\
6.1511549949646 52.9877395629883\\
6.19536590576172 52.9877395629883\\
6.23957681655884 52.9877395629883\\
6.28378772735596 52.9877395629883\\
6.32799816131592 52.9877395629883\\
6.37220907211304 52.9877395629883\\
6.41641998291016 52.9877395629883\\
6.46063089370728 52.9877395629883\\
6.50484180450439 52.9877395629883\\
6.54905271530151 52.9877395629883\\
6.59326362609863 52.9877395629883\\
6.63747406005859 52.9877395629883\\
6.68168497085571 52.9877395629883\\
6.72589588165283 52.9877395629883\\
6.77010679244995 52.9877395629883\\
6.81431770324707 52.9877395629883\\
6.85852861404419 52.9877395629883\\
6.90273904800415 52.9877395629883\\
6.94694995880127 52.9877395629883\\
6.99116086959839 52.9877395629883\\
7.03537178039551 52.2437324523926\\
7.07958269119263 51.1552505493164\\
7.12379360198975 49.995246887207\\
7.16800451278687 48.8206787109375\\
7.21221494674683 47.643196105957\\
7.25642585754395 46.465145111084\\
7.30063676834106 45.286979675293\\
7.34484767913818 44.1087989807129\\
7.3890585899353 42.9306106567383\\
7.43326950073242 41.7524223327637\\
7.47748041152954 40.5742340087891\\
7.5216908454895 39.3960456848145\\
7.56590175628662 38.2178535461426\\
7.61011266708374 37.039665222168\\
7.65432357788086 35.8614768981934\\
7.69853448867798 34.6832847595215\\
7.7427453994751 33.5050964355469\\
7.78695631027222 32.326904296875\\
7.83116674423218 31.1487159729004\\
7.8753776550293 29.9705238342285\\
7.91958856582642 28.7923336029053\\
7.96379947662354 27.614143371582\\
8.0080099105835 26.624584197998\\
8.05222129821777 26.509183883667\\
8.09643173217773 26.4956932067871\\
8.14064311981201 26.4941177368164\\
8.18485355377197 26.4939346313477\\
8.22906494140625 26.4939117431641\\
8.27327537536621 26.4939098358154\\
8.31748580932617 26.4939098358154\\
8.36169719696045 26.4939079284668\\
8.40590763092041 26.4939079284668\\
8.45011901855469 26.4939079284668\\
8.49432945251465 26.4939079284668\\
8.53854084014893 26.4939079284668\\
8.58275127410889 26.4939079284668\\
8.62696170806885 26.4939079284668\\
8.67117309570313 26.4939079284668\\
8.71538352966309 26.4939079284668\\
8.75959491729736 26.4939079284668\\
8.80380535125732 26.4939079284668\\
8.8480167388916 26.4939079284668\\
8.89222717285156 26.4939079284668\\
8.93643760681152 26.4939079284668\\
8.9806489944458 26.4939079284668\\
9.02485942840576 26.4939079284668\\
9.06907081604004 26.4939079284668\\
9.11328125 26.4939079284668\\
9.15749263763428 26.4939079284668\\
9.20170307159424 26.4939079284668\\
9.2459135055542 26.4939079284668\\
9.29012489318848 26.4939079284668\\
9.33433532714844 26.4939079284668\\
9.37854671478271 26.4939079284668\\
9.42275714874268 26.4939079284668\\
9.46696758270264 26.4939079284668\\
9.51117897033691 26.4939079284668\\
9.55538940429688 26.4939079284668\\
9.59960079193115 26.4939079284668\\
9.64381122589111 26.4939079284668\\
9.68802261352539 26.4939079284668\\
9.73223304748535 26.4939079284668\\
9.77644348144531 26.4939079284668\\
9.82065486907959 26.4939079284668\\
9.86486530303955 26.4939079284668\\
9.90907669067383 26.4939079284668\\
9.95328712463379 26.4939079284668\\
9.99749851226807 26.4939079284668\\
10.041708946228 26.4939079284668\\
10.085919380188 26.4939079284668\\
10.1301307678223 26.4939079284668\\
10.1743412017822 26.4939079284668\\
10.2185525894165 26.4939079284668\\
10.2627630233765 26.4939079284668\\
10.3069744110107 26.4939079284668\\
10.3511848449707 26.4939079284668\\
10.3953952789307 26.4939079284668\\
10.4396066665649 26.4939079284668\\
10.4838171005249 26.4939079284668\\
10.5280284881592 26.4939079284668\\
10.5722389221191 26.4939079284668\\
10.6164503097534 26.4939079284668\\
10.6606607437134 26.4939079284668\\
10.7048711776733 26.4939079284668\\
10.7490825653076 26.4939079284668\\
10.7932929992676 26.4939079284668\\
10.8375043869019 26.4939079284668\\
10.8817148208618 26.4939079284668\\
10.9259262084961 26.4939079284668\\
10.9701366424561 26.4939079284668\\
11.014347076416 26.1576251983643\\
11.0585584640503 25.0768013000488\\
11.1027688980103 23.90940284729\\
11.1469802856445 22.7323703765869\\
11.1911907196045 21.5542984008789\\
11.2354021072388 20.3761196136475\\
11.2796125411987 19.1979331970215\\
11.3238229751587 18.0197467803955\\
11.368034362793 16.8415603637695\\
11.4122447967529 15.6633768081665\\
11.4561223983765 14.4940776824951\\
11.5 13.3247852325439\\
11.54421043396 12.1466178894043\\
11.5884218215942 10.9683303833008\\
11.6326322555542 9.79013252258301\\
11.6768436431885 8.61193943023682\\
11.7210540771484 7.43374538421631\\
11.7652654647827 6.25555229187012\\
11.8094758987427 5.07735872268677\\
11.8536863327026 3.89916563034058\\
11.8978977203369 2.72097659111023\\
11.9421081542969 1.54277789592743\\
11.9863195419312 0.36458432674408\\
12.0305299758911 0\\
12.0747413635254 0\\
12.1189517974854 0\\
12.1631622314453 0\\
12.2073736190796 0\\
12.2515840530396 0\\
12.2957954406738 0\\
12.3400058746338 0\\
12.3842172622681 0\\
12.428427696228 0\\
12.472638130188 0\\
12.5168495178223 0\\
12.5610599517822 0\\
12.6052713394165 0\\
12.6494817733765 0\\
12.6936931610107 0\\
12.7379035949707 0\\
12.7821140289307 0\\
12.8263254165649 0\\
12.8705358505249 0\\
12.9147472381592 0\\
12.9589576721191 0\\
13.0031690597534 0\\
13.0473794937134 0\\
13.0915899276733 0\\
13.1358013153076 0\\
13.1800117492676 0\\
13.2242231369019 0\\
13.2684335708618 0\\
13.3126440048218 0\\
13.3568553924561 0\\
13.401065826416 0\\
13.4452772140503 0\\
13.4894876480103 0\\
13.5336990356445 0\\
13.5779094696045 0\\
13.6221199035645 0\\
13.6663312911987 0\\
13.7105417251587 0\\
13.754753112793 0\\
13.7989635467529 0\\
13.8431749343872 0\\
13.8873853683472 0\\
13.9315958023071 0\\
13.9758071899414 0\\
14.0200176239014 0\\
14.0642290115356 0\\
14.1084394454956 0\\
14.1526508331299 0\\
14.1968612670898 0\\
14.2410717010498 0\\
14.2852830886841 0\\
14.329493522644 0\\
14.3737049102783 0\\
14.4179153442383 0\\
14.4621267318726 0\\
14.5063371658325 0\\
14.5505475997925 0\\
14.5947589874268 0\\
14.6389694213867 0\\
14.683180809021 0\\
14.727391242981 0\\
14.7716026306152 0\\
14.8158130645752 0\\
14.8600234985352 0\\
14.9042348861694 0\\
14.9484453201294 0\\
14.9926567077637 0\\
15.0368671417236 1.93454313278198\\
15.0810785293579 4.24624395370483\\
15.1252889633179 6.56158828735352\\
15.1694993972778 8.87692451477051\\
15.2137107849121 11.1922979354858\\
15.2579212188721 13.5076742172241\\
15.3021326065063 15.8230504989624\\
15.3463430404663 18.138427734375\\
15.3905544281006 20.4538021087646\\
15.4347648620605 22.7691783905029\\
15.4789752960205 25.0845527648926\\
15.5231866836548 27.3999271392822\\
15.5673971176147 29.7153015136719\\
15.611608505249 32.0306739807129\\
15.655818939209 34.3460464477539\\
15.7000303268433 36.6614189147949\\
15.7442407608032 38.9767913818359\\
15.7884511947632 41.2921600341797\\
15.8326625823975 43.6075286865234\\
15.8768730163574 45.9228973388672\\
15.9210844039917 48.2382850646973\\
15.9652948379517 50.5536499023438\\
16.0095062255859 52.4762725830078\\
16.0537166595459 52.8805809020996\\
16.0967807769775 52.9648361206055\\
16.1407527923584 52.9829216003418\\
16.1849117279053 52.9867286682129\\
16.2291107177734 52.9875259399414\\
16.2733192443848 52.9876937866211\\
16.3175296783447 52.9877281188965\\
16.3617401123047 52.987735748291\\
16.4059524536133 52.9877395629883\\
16.4501628875732 52.9877395629883\\
16.4943733215332 52.9877395629883\\
16.5385837554932 52.9877395629883\\
16.5827941894531 52.9877395629883\\
16.6270065307617 52.9877395629883\\
16.6712169647217 52.9877395629883\\
16.7154273986816 52.9877395629883\\
16.7596378326416 52.9877395629883\\
16.8038482666016 52.9877395629883\\
16.8480606079102 52.9877395629883\\
16.8922710418701 52.9877395629883\\
16.9364814758301 52.9877395629883\\
16.98069190979 52.9877395629883\\
17.0249042510986 52.9877395629883\\
17.0691146850586 52.9877395629883\\
17.1133251190186 52.9877395629883\\
17.1575355529785 52.9877395629883\\
17.2017459869385 52.9877395629883\\
17.2459583282471 52.9877395629883\\
17.290168762207 52.9877395629883\\
17.334379196167 52.9877395629883\\
17.378589630127 52.9877395629883\\
17.4228000640869 52.9877395629883\\
17.4670124053955 52.9877395629883\\
17.5112228393555 52.9877395629883\\
17.5554332733154 52.9877395629883\\
17.5996437072754 52.9877395629883\\
17.643856048584 52.9877395629883\\
17.6880664825439 52.9877395629883\\
17.7322769165039 52.9877395629883\\
17.7764873504639 52.9877395629883\\
17.8206977844238 52.9877395629883\\
17.8649101257324 52.9877395629883\\
17.9091205596924 52.9877395629883\\
17.9533309936523 52.9877395629883\\
17.9975414276123 52.9877395629883\\
18.0417518615723 52.9877395629883\\
18.0859642028809 52.9877395629883\\
18.1301746368408 52.9877395629883\\
18.1743850708008 52.9877395629883\\
18.2185955047607 52.9877395629883\\
18.2628078460693 52.9877395629883\\
18.3070182800293 52.9877395629883\\
18.3512287139893 52.9877395629883\\
18.3954391479492 52.9877395629883\\
18.4396495819092 52.9877395629883\\
18.4838619232178 52.9877395629883\\
18.5280723571777 52.9877395629883\\
18.5722827911377 52.9877395629883\\
18.6164932250977 52.9877395629883\\
18.6607036590576 52.9877395629883\\
18.7049160003662 52.9877395629883\\
18.7491264343262 52.9877395629883\\
18.7933368682861 52.9877395629883\\
18.8375473022461 52.9877395629883\\
18.8817577362061 52.9877395629883\\
18.9259700775146 52.9877395629883\\
18.962984085083 52.9877395629883\\
19 52.9877395629883\\
};
\addlegendentry{$A_{r}(t)$};

\addplot [
color=black,
dash pattern=on 1pt off 3pt on 3pt off 3pt
]
table[row sep=crcr]{
0 0\\
9.99999974737875e-06 7.34263050361506e-09\\
2.49999993684469e-05 6.82787117511907e-07\\
4.75000015285332e-05 1.61537918756949e-05\\
8.12500002211891e-05 0.000229484256124124\\
0.000131875000079162 0.00192573899403214\\
0.000207812496228144 0.00898905657231808\\
0.000321718747727573 0.0234745796769857\\
0.000492578139528632 0.0383488200604916\\
0.00074886716902256 0.0502703413367271\\
0.00113330082967877 0.0704929903149605\\
0.00170995120424777 0.108524538576603\\
0.00257492670789361 0.167731299996376\\
0.00387239013798535 0.258889704942703\\
0.00529959984123707 0.363474696874619\\
0.00686953077092767 0.483860403299332\\
0.0085964547470212 0.622648656368256\\
0.0104960706084967 0.782925307750702\\
0.0125856483355165 0.968311131000519\\
0.014884184114635 1.18303418159485\\
0.0174125730991364 1.43201053142548\\
0.020193800330162 1.72092187404633\\
0.0232531521469355 2.05627965927124\\
0.0266184378415346 2.44546175003052\\
0.0303202513605356 2.8966953754425\\
0.0343922488391399 3.41895937919617\\
0.0388714447617531 4.0217547416687\\
0.0437985584139824 4.71469640731812\\
0.0492183864116669 5.50687074661255\\
0.0551801957190037 6.40590953826904\\
0.0617381855845451 7.41679286956787\\
0.0689519718289375 8.54045581817627\\
0.0768871381878853 9.77241897583008\\
0.0856158286333084 11.1017980575562\\
0.095217376947403 12.5111665725708\\
0.105779089033604 13.9776849746704\\
0.117396965622902 15.4756441116333\\
0.13017663359642 16.9800872802734\\
0.144234269857407 18.4706878662109\\
0.159697666764259 19.9348068237305\\
0.176707401871681 21.3688697814941\\
0.195418119430542 22.7777843475342\\
0.215999901294708 24.1727924346924\\
0.238639861345291 25.5685958862305\\
0.2635438144207 26.9806213378906\\
0.290938168764114 28.4230575561523\\
0.321071952581406 29.907886505127\\
0.354219108819962 31.4447612762451\\
0.390680998563766 33.0414009094238\\
0.427134186029434 34.5529632568359\\
0.461836785078049 35.9256439208984\\
0.495115727186203 37.1908950805664\\
0.527189493179321 38.3686904907227\\
0.558222532272339 39.473316192627\\
0.588342726230621 40.515495300293\\
0.617652356624603 41.5035820007324\\
0.646234631538391 42.4442443847656\\
0.674158751964569 43.342903137207\\
0.701483011245728 44.2040367126465\\
0.728256940841675 45.0314025878906\\
0.754523456096649 45.8281784057617\\
0.780319631099701 46.5970878601074\\
0.805678188800812 47.3404846191406\\
0.830627858638763 48.0604095458984\\
0.855194389820099 48.7586631774902\\
0.879400551319122 49.436824798584\\
0.903267025947571 50.0962944030762\\
0.926812350749969 50.7383270263672\\
0.950053453445435 51.3640441894531\\
0.973005652427673 51.9744567871094\\
0.995683073997498 52.5704765319824\\
1.01809847354889 52.844913482666\\
1.0402637720108 52.9383926391602\\
1.06237816810608 52.9706420898438\\
1.08448839187622 52.9818077087402\\
1.10659623146057 52.9856758117676\\
1.12870299816132 52.9870223999023\\
1.15080916881561 52.9874877929688\\
1.17291498184204 52.9876518249512\\
1.19502067565918 52.9877090454102\\
1.21712636947632 52.9877319335938\\
1.23923194408417 52.9877395629883\\
1.26133751869202 52.9877395629883\\
1.28344309329987 52.9877433776855\\
1.30554866790771 52.9877433776855\\
1.32765424251556 52.9877395629883\\
1.34975969791412 52.9877395629883\\
1.37186527252197 52.9877395629883\\
1.39397084712982 52.9877395629883\\
1.41607642173767 52.9877395629883\\
1.43818199634552 52.9877395629883\\
1.46028745174408 52.9877395629883\\
1.48239302635193 52.9877395629883\\
1.50449860095978 52.9877395629883\\
1.52660417556763 52.9877395629883\\
1.54870975017548 52.9877395629883\\
1.57081520557404 52.9877395629883\\
1.59292078018188 52.9877395629883\\
1.61502635478973 52.9877395629883\\
1.63713192939758 52.9877395629883\\
1.65923750400543 52.9877395629883\\
1.68134295940399 52.9877395629883\\
1.70344853401184 52.9877395629883\\
1.72555410861969 52.9877395629883\\
1.74765968322754 52.9877395629883\\
1.76976525783539 52.9877395629883\\
1.79187071323395 52.9877395629883\\
1.8139762878418 52.9877395629883\\
1.83608186244965 52.9877395629883\\
1.8581874370575 52.9877395629883\\
1.88029301166534 52.9877395629883\\
1.9023984670639 52.9877395629883\\
1.92450404167175 52.9877395629883\\
1.9466096162796 52.9877395629883\\
1.96871519088745 52.9877395629883\\
1.9908207654953 52.9877395629883\\
2.01292634010315 52.9877395629883\\
2.03503179550171 52.9877395629883\\
2.05713748931885 52.9877395629883\\
2.07924294471741 52.9877395629883\\
2.10134840011597 52.9877395629883\\
2.12345409393311 52.9877395629883\\
2.14555954933167 52.9877395629883\\
2.1676652431488 52.9877395629883\\
2.18977069854736 52.9877395629883\\
2.21187615394592 52.9877395629883\\
2.23398184776306 52.9877395629883\\
2.25608730316162 52.9877395629883\\
2.27819299697876 52.9877395629883\\
2.30029845237732 52.9877395629883\\
2.32240390777588 52.9877395629883\\
2.34450960159302 52.9877395629883\\
2.36661505699158 52.9877395629883\\
2.38872075080872 52.9877395629883\\
2.41082620620728 52.9877395629883\\
2.43293166160584 52.9877395629883\\
2.45503735542297 52.9877395629883\\
2.47714281082153 52.9877395629883\\
2.49924850463867 52.9877395629883\\
2.52135396003723 52.9877395629883\\
2.54345941543579 52.9877395629883\\
2.56556510925293 52.9877395629883\\
2.58767056465149 52.9877395629883\\
2.60977625846863 52.9877395629883\\
2.63188171386719 52.9877395629883\\
2.65398716926575 52.9877395629883\\
2.67609286308289 52.9877395629883\\
2.69819831848145 52.9877395629883\\
2.72030401229858 52.9877395629883\\
2.74240946769714 52.9877395629883\\
2.7645149230957 52.9877395629883\\
2.78662061691284 52.9877395629883\\
2.8087260723114 52.9877395629883\\
2.83083176612854 52.9877395629883\\
2.8529372215271 52.9877395629883\\
2.87504267692566 52.9877395629883\\
2.8971483707428 52.9877395629883\\
2.91925382614136 52.9877395629883\\
2.9413595199585 52.9877395629883\\
2.96346497535706 52.9877395629883\\
2.98557043075562 52.9877395629883\\
3.00767612457275 52.9877395629883\\
3.02978157997131 52.9877395629883\\
3.0519700050354 52.9877395629883\\
3.07440447807312 52.9877395629883\\
3.09709739685059 52.9877395629883\\
3.12005996704102 52.9877395629883\\
3.14330291748047 52.9877395629883\\
3.166836977005 52.9877395629883\\
3.19067335128784 52.9877395629883\\
3.21482396125793 52.9877395629883\\
3.2393012046814 52.9877395629883\\
3.26411890983582 52.9877395629883\\
3.28929138183594 52.9877395629883\\
3.31483387947083 52.9877395629883\\
3.34076333045959 52.9877395629883\\
3.36709761619568 52.9877395629883\\
3.39385652542114 52.9877395629883\\
3.42106103897095 52.9877395629883\\
3.44873404502869 52.9877395629883\\
3.47690057754517 52.9877395629883\\
3.50558733940125 52.9877395629883\\
3.53482460975647 52.9877395629883\\
3.5646448135376 52.9877395629883\\
3.59508395195007 52.9877395629883\\
3.62618231773376 52.9877395629883\\
3.65798377990723 52.9877395629883\\
3.69053816795349 52.9877395629883\\
3.72390079498291 52.9877395629883\\
3.75813484191895 52.9877395629883\\
3.79331135749817 52.9877395629883\\
3.82951164245605 52.9877395629883\\
3.86682963371277 52.9877395629883\\
3.90537428855896 52.9877395629883\\
3.94527339935303 52.9877395629883\\
3.98667812347412 52.9877395629883\\
4.02976989746094 52.9877395629883\\
4.07350301742554 52.9877395629883\\
4.11750602722168 52.9877395629883\\
4.16167545318604 52.9877395629883\\
4.2058801651001 52.9877395629883\\
4.2500901222229 52.9877395629883\\
4.29430055618286 52.9877395629883\\
4.33851099014282 52.9877395629883\\
4.38272142410278 52.9877395629883\\
4.4269323348999 52.9877395629883\\
4.47114324569702 52.9877395629883\\
4.51535367965698 52.9877395629883\\
4.5595645904541 52.9877395629883\\
4.60377550125122 52.9877395629883\\
4.64798641204834 52.9877395629883\\
4.69219732284546 52.9877395629883\\
4.73640823364258 52.9877395629883\\
4.7806191444397 52.9877395629883\\
4.82482957839966 52.9877395629883\\
4.86904048919678 52.9877395629883\\
4.9132513999939 52.9877395629883\\
4.95746231079102 52.9877395629883\\
5.00167322158813 52.9877395629883\\
5.04588413238525 52.9877395629883\\
5.09009504318237 52.9877395629883\\
5.13430547714233 52.9877395629883\\
5.17851638793945 52.9877395629883\\
5.22272729873657 52.9877395629883\\
5.26693820953369 52.9877395629883\\
5.31114912033081 52.9877395629883\\
5.35536003112793 52.9877395629883\\
5.39957046508789 52.9877395629883\\
5.44378137588501 52.9877395629883\\
5.48799228668213 52.9877395629883\\
5.53220319747925 52.9877395629883\\
5.57641410827637 52.9877395629883\\
5.62062501907349 52.9877395629883\\
5.66483592987061 52.9877395629883\\
5.70904636383057 52.9877395629883\\
5.75325727462769 52.9877395629883\\
5.7974681854248 52.9877395629883\\
5.84167909622192 52.9877395629883\\
5.88589000701904 52.9877395629883\\
5.93010091781616 52.9877395629883\\
5.97431182861328 52.9877395629883\\
6.01852226257324 52.9877395629883\\
6.06273317337036 52.9877395629883\\
6.10694408416748 52.9877395629883\\
6.1511549949646 52.9877395629883\\
6.19536590576172 52.9877395629883\\
6.23957681655884 52.9877395629883\\
6.28378772735596 52.9877395629883\\
6.32799816131592 52.9877395629883\\
6.37220907211304 52.9877395629883\\
6.41641998291016 52.9877395629883\\
6.46063089370728 52.9877395629883\\
6.50484180450439 52.9877395629883\\
6.54905271530151 52.9877395629883\\
6.59326362609863 52.9877395629883\\
6.63747406005859 52.9877395629883\\
6.68168497085571 52.9877395629883\\
6.72589588165283 52.9877395629883\\
6.77010679244995 52.9877395629883\\
6.81431770324707 52.9877395629883\\
6.85852861404419 52.9877395629883\\
6.90273904800415 52.9877395629883\\
6.94694995880127 52.9877395629883\\
6.99116086959839 52.9877395629883\\
7.03537178039551 52.9877395629883\\
7.07958269119263 52.9877395629883\\
7.12379360198975 52.9877395629883\\
7.16800451278687 52.9877395629883\\
7.21221494674683 52.9877395629883\\
7.25642585754395 52.9877395629883\\
7.30063676834106 52.9877395629883\\
7.34484767913818 52.9877395629883\\
7.3890585899353 52.9877395629883\\
7.43326950073242 52.9877395629883\\
7.47748041152954 52.9877395629883\\
7.5216908454895 52.9877395629883\\
7.56590175628662 52.9877395629883\\
7.61011266708374 52.9877395629883\\
7.65432357788086 52.9877395629883\\
7.69853448867798 52.9877395629883\\
7.7427453994751 52.9877395629883\\
7.78695631027222 52.9877395629883\\
7.83116674423218 52.9877395629883\\
7.8753776550293 52.9877395629883\\
7.91958856582642 52.9877395629883\\
7.96379947662354 52.9877395629883\\
8.0080099105835 52.9877395629883\\
8.05222129821777 52.9877395629883\\
8.09643173217773 52.9877395629883\\
8.14064311981201 52.9877395629883\\
8.18485355377197 52.9877395629883\\
8.22906494140625 52.9877395629883\\
8.27327537536621 52.9877395629883\\
8.31748580932617 52.9877395629883\\
8.36169719696045 52.9877395629883\\
8.40590763092041 52.9877395629883\\
8.45011901855469 52.9877395629883\\
8.49432945251465 52.9877395629883\\
8.53854084014893 52.9877395629883\\
8.58275127410889 52.9877395629883\\
8.62696170806885 52.9877395629883\\
8.67117309570313 52.9877395629883\\
8.71538352966309 52.9877395629883\\
8.75959491729736 52.9877395629883\\
8.80380535125732 52.9877395629883\\
8.8480167388916 52.9877395629883\\
8.89222717285156 52.9877395629883\\
8.93643760681152 52.9877395629883\\
8.9806489944458 52.9877395629883\\
9.02485942840576 52.9877395629883\\
9.06907081604004 52.9877395629883\\
9.11328125 52.9877395629883\\
9.15749263763428 52.9877395629883\\
9.20170307159424 52.9877395629883\\
9.2459135055542 52.9877395629883\\
9.29012489318848 52.9877395629883\\
9.33433532714844 52.9877395629883\\
9.37854671478271 52.9877395629883\\
9.42275714874268 52.9877395629883\\
9.46696758270264 52.9877395629883\\
9.51117897033691 52.9877395629883\\
9.55538940429688 52.9877395629883\\
9.59960079193115 52.9877395629883\\
9.64381122589111 52.9877395629883\\
9.68802261352539 52.9877395629883\\
9.73223304748535 52.9877395629883\\
9.77644348144531 52.9877395629883\\
9.82065486907959 52.9877395629883\\
9.86486530303955 52.9877395629883\\
9.90907669067383 52.9877395629883\\
9.95328712463379 52.9877395629883\\
9.99749851226807 52.9877395629883\\
10.041708946228 52.9877395629883\\
10.085919380188 52.9877395629883\\
10.1301307678223 52.9877395629883\\
10.1743412017822 52.9877395629883\\
10.2185525894165 52.9877395629883\\
10.2627630233765 52.9877395629883\\
10.3069744110107 52.9877395629883\\
10.3511848449707 52.9877395629883\\
10.3953952789307 52.9877395629883\\
10.4396066665649 52.9877395629883\\
10.4838171005249 52.9877395629883\\
10.5280284881592 52.9877395629883\\
10.5722389221191 52.9877395629883\\
10.6164503097534 52.9877395629883\\
10.6606607437134 52.9877395629883\\
10.7048711776733 52.9877395629883\\
10.7490825653076 52.9877395629883\\
10.7932929992676 52.9877395629883\\
10.8375043869019 52.9877395629883\\
10.8817148208618 52.9877395629883\\
10.9259262084961 52.9877395629883\\
10.9701366424561 52.9877395629883\\
11.014347076416 52.9877395629883\\
11.0585584640503 52.9877395629883\\
11.1027688980103 52.9877395629883\\
11.1469802856445 52.9877395629883\\
11.1911907196045 52.9877395629883\\
11.2354021072388 52.9877395629883\\
11.2796125411987 52.9877395629883\\
11.3238229751587 52.9877395629883\\
11.368034362793 52.9877395629883\\
11.4122447967529 52.9877395629883\\
11.4561223983765 52.9877395629883\\
11.5 52.9877395629883\\
11.54421043396 52.9877395629883\\
11.5884218215942 52.9877395629883\\
11.6326322555542 52.9877395629883\\
11.6768436431885 52.9877395629883\\
11.7210540771484 52.9877395629883\\
11.7652654647827 52.9877395629883\\
11.8094758987427 52.9877395629883\\
11.8536863327026 52.9877395629883\\
11.8978977203369 52.9877395629883\\
11.9421081542969 52.9877395629883\\
11.9863195419312 52.9877395629883\\
12.0305299758911 52.9877395629883\\
12.0747413635254 52.9877395629883\\
12.1189517974854 52.9877395629883\\
12.1631622314453 52.9877395629883\\
12.2073736190796 52.9877395629883\\
12.2515840530396 52.9877395629883\\
12.2957954406738 52.9877395629883\\
12.3400058746338 52.9877395629883\\
12.3842172622681 52.9877395629883\\
12.428427696228 52.9877395629883\\
12.472638130188 52.9877395629883\\
12.5168495178223 52.9877395629883\\
12.5610599517822 52.9877395629883\\
12.6052713394165 52.9877395629883\\
12.6494817733765 52.9877395629883\\
12.6936931610107 52.9877395629883\\
12.7379035949707 52.9877395629883\\
12.7821140289307 52.9877395629883\\
12.8263254165649 52.9877395629883\\
12.8705358505249 52.9877395629883\\
12.9147472381592 52.9877395629883\\
12.9589576721191 52.9877395629883\\
13.0031690597534 52.9877395629883\\
13.0473794937134 52.9877395629883\\
13.0915899276733 52.9877395629883\\
13.1358013153076 52.9877395629883\\
13.1800117492676 52.9877395629883\\
13.2242231369019 52.9877395629883\\
13.2684335708618 52.9877395629883\\
13.3126440048218 52.9877395629883\\
13.3568553924561 52.9877395629883\\
13.401065826416 52.9877395629883\\
13.4452772140503 52.9877395629883\\
13.4894876480103 52.9877395629883\\
13.5336990356445 52.9877395629883\\
13.5779094696045 52.9877395629883\\
13.6221199035645 52.9877395629883\\
13.6663312911987 52.9877395629883\\
13.7105417251587 52.9877395629883\\
13.754753112793 52.9877395629883\\
13.7989635467529 52.9877395629883\\
13.8431749343872 52.9877395629883\\
13.8873853683472 52.9877395629883\\
13.9315958023071 52.9877395629883\\
13.9758071899414 52.9877395629883\\
14.0200176239014 52.9877395629883\\
14.0642290115356 52.9877395629883\\
14.1084394454956 52.9877395629883\\
14.1526508331299 52.9877395629883\\
14.1968612670898 52.9877395629883\\
14.2410717010498 52.9877395629883\\
14.2852830886841 52.9877395629883\\
14.329493522644 52.9877395629883\\
14.3737049102783 52.9877395629883\\
14.4179153442383 52.9877395629883\\
14.4621267318726 52.9877395629883\\
14.5063371658325 52.9877395629883\\
14.5505475997925 52.9877395629883\\
14.5947589874268 52.9877395629883\\
14.6389694213867 52.9877395629883\\
14.683180809021 52.9877395629883\\
14.727391242981 52.9877395629883\\
14.7716026306152 52.9877395629883\\
14.8158130645752 52.9877395629883\\
14.8600234985352 52.9877395629883\\
14.9042348861694 52.9877395629883\\
14.9484453201294 52.9877395629883\\
14.9926567077637 52.9877395629883\\
15.0368671417236 52.9877395629883\\
15.0810785293579 52.9877395629883\\
15.1252889633179 52.9877395629883\\
15.1694993972778 52.9877395629883\\
15.2137107849121 52.9877395629883\\
15.2579212188721 52.9877395629883\\
15.3021326065063 52.9877395629883\\
15.3463430404663 52.9877395629883\\
15.3905544281006 52.9877395629883\\
15.4347648620605 52.9877395629883\\
15.4789752960205 52.9877395629883\\
15.5231866836548 52.9877395629883\\
15.5673971176147 52.9877395629883\\
15.611608505249 52.9877395629883\\
15.655818939209 52.9877395629883\\
15.7000303268433 52.9877395629883\\
15.7442407608032 52.9877395629883\\
15.7884511947632 52.9877395629883\\
15.8326625823975 52.9877395629883\\
15.8768730163574 52.9877395629883\\
15.9210844039917 52.9877395629883\\
15.9652948379517 52.9877395629883\\
16.0095062255859 52.9877395629883\\
16.0537166595459 52.9877395629883\\
16.0967807769775 52.9877395629883\\
16.1407527923584 52.9877395629883\\
16.1849117279053 52.9877395629883\\
16.2291107177734 52.9877395629883\\
16.2733192443848 52.9877395629883\\
16.3175296783447 52.9877395629883\\
16.3617401123047 52.9877395629883\\
16.4059524536133 52.9877395629883\\
16.4501628875732 52.9877395629883\\
16.4943733215332 52.9877395629883\\
16.5385837554932 52.9877395629883\\
16.5827941894531 52.9877395629883\\
16.6270065307617 52.9877395629883\\
16.6712169647217 52.9877395629883\\
16.7154273986816 52.9877395629883\\
16.7596378326416 52.9877395629883\\
16.8038482666016 52.9877395629883\\
16.8480606079102 52.9877395629883\\
16.8922710418701 52.9877395629883\\
16.9364814758301 52.9877395629883\\
16.98069190979 52.9877395629883\\
17.0249042510986 52.9877395629883\\
17.0691146850586 52.9877395629883\\
17.1133251190186 52.9877395629883\\
17.1575355529785 52.9877395629883\\
17.2017459869385 52.9877395629883\\
17.2459583282471 52.9877395629883\\
17.290168762207 52.9877395629883\\
17.334379196167 52.9877395629883\\
17.378589630127 52.9877395629883\\
17.4228000640869 52.9877395629883\\
17.4670124053955 52.9877395629883\\
17.5112228393555 52.9877395629883\\
17.5554332733154 52.9877395629883\\
17.5996437072754 52.9877395629883\\
17.643856048584 52.9877395629883\\
17.6880664825439 52.9877395629883\\
17.7322769165039 52.9877395629883\\
17.7764873504639 52.9877395629883\\
17.8206977844238 52.9877395629883\\
17.8649101257324 52.9877395629883\\
17.9091205596924 52.9877395629883\\
17.9533309936523 52.9877395629883\\
17.9975414276123 52.9877395629883\\
18.0417518615723 52.9877395629883\\
18.0859642028809 52.9877395629883\\
18.1301746368408 52.9877395629883\\
18.1743850708008 52.9877395629883\\
18.2185955047607 52.9877395629883\\
18.2628078460693 52.9877395629883\\
18.3070182800293 52.9877395629883\\
18.3512287139893 52.9877395629883\\
18.3954391479492 52.9877395629883\\
18.4396495819092 52.9877395629883\\
18.4838619232178 52.9877395629883\\
18.5280723571777 52.9877395629883\\
18.5722827911377 52.9877395629883\\
18.6164932250977 52.9877395629883\\
18.6607036590576 52.9877395629883\\
18.7049160003662 52.9877395629883\\
18.7491264343262 52.9877395629883\\
18.7933368682861 52.9877395629883\\
18.8375473022461 52.9877395629883\\
18.8817577362061 52.9877395629883\\
18.9259700775146 52.9877395629883\\
18.962984085083 52.9877395629883\\
19 52.9877395629883\\
};
\addlegendentry{$A_{r} = 0.50$};

\addplot [
color=black,
dashed
]
table[row sep=crcr]{
0 0\\
9.99999974737875e-06 7.34242400213247e-09\\
2.49999993684469e-05 6.82759377923503e-07\\
4.75000015285332e-05 1.61528696480673e-05\\
8.12500002211891e-05 0.000229467026656494\\
0.000131875000079162 0.00192556483671069\\
0.000207812496228144 0.00898817554116249\\
0.000321718747727573 0.0234723221510649\\
0.000492578139528632 0.0383452847599983\\
0.00074886716902256 0.0502648875117302\\
0.00113330082967877 0.070481926202774\\
0.00170995120424777 0.108498722314835\\
0.00257492670789361 0.167652875185013\\
0.00387239013798535 0.258626192808151\\
0.00529959984123707 0.362792581319809\\
0.00686953077092767 0.482348293066025\\
0.0085964547470212 0.619613349437714\\
0.0104960706084967 0.777237176895142\\
0.0125856483355165 0.958173990249634\\
0.014884184114635 1.16564691066742\\
0.0174125730991364 1.40307796001434\\
0.020193800330162 1.67395949363709\\
0.0232531521469355 1.98164916038513\\
0.0266184378415346 2.32907176017761\\
0.0303202513605356 2.71832084655762\\
0.0343922488391399 3.15018105506897\\
0.0388714447617531 3.62361979484558\\
0.0437985584139824 4.13536167144775\\
0.0492183864116669 4.67968845367432\\
0.0551801957190037 5.24862861633301\\
0.0617381855845451 5.83265113830566\\
0.0689519718289375 6.42182493209839\\
0.0768871381878853 7.00725269317627\\
0.0856158286333084 7.58241939544678\\
0.095217376947403 8.14408874511719\\
0.105779089033604 8.6925163269043\\
0.117396965622902 9.23096466064453\\
0.13017663359642 9.76474666595459\\
0.144234269857407 10.3001012802124\\
0.159697666764259 10.8432216644287\\
0.176707401871681 11.3996133804321\\
0.195418119430542 11.9738235473633\\
0.215999901294708 12.56946849823\\
0.238639861345291 13.1894273757935\\
0.2635438144207 13.8360843658447\\
0.290938168764114 14.5115346908569\\
0.321071952581406 15.2177352905273\\
0.354219108819962 15.9565982818604\\
0.390680998563766 16.730037689209\\
0.427134186029434 17.4655838012695\\
0.461836785078049 18.1360149383545\\
0.495115727186203 18.7559604644775\\
0.527189493179321 19.3344173431396\\
0.558222532272339 19.8779067993164\\
0.588342726230621 20.3914070129395\\
0.617652356624603 20.8788280487061\\
0.646234631538391 21.3433170318604\\
0.674158751964569 21.7874450683594\\
0.701483011245728 22.2133445739746\\
0.728256940841675 22.6228103637695\\
0.754523456096649 23.0173664093018\\
0.780319631099701 23.3983211517334\\
0.805678188800812 23.7668037414551\\
0.830627858638763 24.1238040924072\\
0.855194389820099 24.4701862335205\\
0.879400551319122 24.8067188262939\\
0.903267025947571 25.1340789794922\\
0.926812350749969 25.4528770446777\\
0.950053453445435 25.7636566162109\\
0.973005652427673 26.0669097900391\\
0.995683073997498 26.3630809783936\\
1.01809847354889 26.4669189453125\\
1.0402637720108 26.488224029541\\
1.06237816810608 26.4926681518555\\
1.08448839187622 26.4936046600342\\
1.10659623146057 26.4938068389893\\
1.12870299816132 26.4938545227051\\
1.15080916881561 26.4938697814941\\
1.17291498184204 26.4938774108887\\
1.19502067565918 26.4938831329346\\
1.21712636947632 26.4938888549805\\
1.23923194408417 26.4938926696777\\
1.26133751869202 26.493896484375\\
1.28344309329987 26.4938983917236\\
1.30554866790771 26.4939022064209\\
1.32765424251556 26.4939041137695\\
1.34975969791412 26.4939060211182\\
1.37186527252197 26.4939060211182\\
1.39397084712982 26.4939079284668\\
1.41607642173767 26.4939079284668\\
1.43818199634552 26.4939098358154\\
1.46028745174408 26.4939098358154\\
1.48239302635193 26.4939098358154\\
1.50449860095978 26.4939098358154\\
1.52660417556763 26.4939098358154\\
1.54870975017548 26.4939098358154\\
1.57081520557404 26.4939098358154\\
1.59292078018188 26.4939098358154\\
1.61502635478973 26.4939098358154\\
1.63713192939758 26.4939098358154\\
1.65923750400543 26.4939098358154\\
1.68134295940399 26.4939098358154\\
1.70344853401184 26.4939098358154\\
1.72555410861969 26.4939098358154\\
1.74765968322754 26.4939098358154\\
1.76976525783539 26.4939098358154\\
1.79187071323395 26.4939098358154\\
1.8139762878418 26.4939098358154\\
1.83608186244965 26.4939098358154\\
1.8581874370575 26.4939098358154\\
1.88029301166534 26.4939098358154\\
1.9023984670639 26.4939098358154\\
1.92450404167175 26.4939098358154\\
1.9466096162796 26.4939098358154\\
1.96871519088745 26.4939098358154\\
1.9908207654953 26.4939098358154\\
2.01292634010315 26.4939098358154\\
2.03503179550171 26.4939098358154\\
2.05713748931885 26.4939079284668\\
2.07924294471741 26.4939079284668\\
2.10134840011597 26.4939079284668\\
2.12345409393311 26.4939079284668\\
2.14555954933167 26.4939079284668\\
2.1676652431488 26.4939079284668\\
2.18977069854736 26.4939079284668\\
2.21187615394592 26.4939079284668\\
2.23398184776306 26.4939079284668\\
2.25608730316162 26.4939079284668\\
2.27819299697876 26.4939079284668\\
2.30029845237732 26.4939079284668\\
2.32240390777588 26.4939079284668\\
2.34450960159302 26.4939079284668\\
2.36661505699158 26.4939079284668\\
2.38872075080872 26.4939079284668\\
2.41082620620728 26.4939079284668\\
2.43293166160584 26.4939079284668\\
2.45503735542297 26.4939079284668\\
2.47714281082153 26.4939079284668\\
2.49924850463867 26.4939079284668\\
2.52135396003723 26.4939079284668\\
2.54345941543579 26.4939079284668\\
2.56556510925293 26.4939079284668\\
2.58767056465149 26.4939079284668\\
2.60977625846863 26.4939079284668\\
2.63188171386719 26.4939079284668\\
2.65398716926575 26.4939079284668\\
2.67609286308289 26.4939079284668\\
2.69819831848145 26.4939079284668\\
2.72030401229858 26.4939079284668\\
2.74240946769714 26.4939079284668\\
2.7645149230957 26.4939079284668\\
2.78662061691284 26.4939079284668\\
2.8087260723114 26.4939079284668\\
2.83083176612854 26.4939079284668\\
2.8529372215271 26.4939079284668\\
2.87504267692566 26.4939079284668\\
2.8971483707428 26.4939079284668\\
2.91925382614136 26.4939079284668\\
2.9413595199585 26.4939079284668\\
2.96346497535706 26.4939079284668\\
2.98557043075562 26.4939079284668\\
3.00767612457275 26.4939079284668\\
3.02978157997131 26.4939079284668\\
3.0519700050354 26.4939079284668\\
3.07440447807312 26.4939079284668\\
3.09709739685059 26.4939079284668\\
3.12005996704102 26.4939079284668\\
3.14330291748047 26.4939079284668\\
3.166836977005 26.4939079284668\\
3.19067335128784 26.4939079284668\\
3.21482396125793 26.4939079284668\\
3.2393012046814 26.4939079284668\\
3.26411890983582 26.4939079284668\\
3.28929138183594 26.4939079284668\\
3.31483387947083 26.4939079284668\\
3.34076333045959 26.4939079284668\\
3.36709761619568 26.4939079284668\\
3.39385652542114 26.4939079284668\\
3.42106103897095 26.4939079284668\\
3.44873404502869 26.4939079284668\\
3.47690057754517 26.4939079284668\\
3.50558733940125 26.4939079284668\\
3.53482460975647 26.4939079284668\\
3.5646448135376 26.4939079284668\\
3.59508395195007 26.4939079284668\\
3.62618231773376 26.4939079284668\\
3.65798377990723 26.4939079284668\\
3.69053816795349 26.4939079284668\\
3.72390079498291 26.4939079284668\\
3.75813484191895 26.4939079284668\\
3.79331135749817 26.4939079284668\\
3.82951164245605 26.4939079284668\\
3.86682963371277 26.4939079284668\\
3.90537428855896 26.4939079284668\\
3.94527339935303 26.4939079284668\\
3.98667812347412 26.4939079284668\\
4.02976989746094 26.4939079284668\\
4.07350301742554 26.4939079284668\\
4.11750602722168 26.4939079284668\\
4.16167545318604 26.4939079284668\\
4.2058801651001 26.4939079284668\\
4.2500901222229 26.4939079284668\\
4.29430055618286 26.4939079284668\\
4.33851099014282 26.4939079284668\\
4.38272142410278 26.4939079284668\\
4.4269323348999 26.4939079284668\\
4.47114324569702 26.4939079284668\\
4.51535367965698 26.4939079284668\\
4.5595645904541 26.4939079284668\\
4.60377550125122 26.4939079284668\\
4.64798641204834 26.4939079284668\\
4.69219732284546 26.4939079284668\\
4.73640823364258 26.4939079284668\\
4.7806191444397 26.4939079284668\\
4.82482957839966 26.4939079284668\\
4.86904048919678 26.4939079284668\\
4.9132513999939 26.4939079284668\\
4.95746231079102 26.4939079284668\\
5.00167322158813 26.4939079284668\\
5.04588413238525 26.4939079284668\\
5.09009504318237 26.4939079284668\\
5.13430547714233 26.4939079284668\\
5.17851638793945 26.4939079284668\\
5.22272729873657 26.4939079284668\\
5.26693820953369 26.4939079284668\\
5.31114912033081 26.4939079284668\\
5.35536003112793 26.4939079284668\\
5.39957046508789 26.4939079284668\\
5.44378137588501 26.4939079284668\\
5.48799228668213 26.4939079284668\\
5.53220319747925 26.4939079284668\\
5.57641410827637 26.4939079284668\\
5.62062501907349 26.4939079284668\\
5.66483592987061 26.4939079284668\\
5.70904636383057 26.4939079284668\\
5.75325727462769 26.4939079284668\\
5.7974681854248 26.4939079284668\\
5.84167909622192 26.4939079284668\\
5.88589000701904 26.4939079284668\\
5.93010091781616 26.4939079284668\\
5.97431182861328 26.4939079284668\\
6.01852226257324 26.4939079284668\\
6.06273317337036 26.4939079284668\\
6.10694408416748 26.4939079284668\\
6.1511549949646 26.4939079284668\\
6.19536590576172 26.4939079284668\\
6.23957681655884 26.4939079284668\\
6.28378772735596 26.4939079284668\\
6.32799816131592 26.4939079284668\\
6.37220907211304 26.4939079284668\\
6.41641998291016 26.4939079284668\\
6.46063089370728 26.4939079284668\\
6.50484180450439 26.4939079284668\\
6.54905271530151 26.4939079284668\\
6.59326362609863 26.4939079284668\\
6.63747406005859 26.4939079284668\\
6.68168497085571 26.4939079284668\\
6.72589588165283 26.4939079284668\\
6.77010679244995 26.4939079284668\\
6.81431770324707 26.4939079284668\\
6.85852861404419 26.4939079284668\\
6.90273904800415 26.4939079284668\\
6.94694995880127 26.4939079284668\\
6.99116086959839 26.4939079284668\\
7.03537178039551 26.4939079284668\\
7.07958269119263 26.4939079284668\\
7.12379360198975 26.4939079284668\\
7.16800451278687 26.4939079284668\\
7.21221494674683 26.4939079284668\\
7.25642585754395 26.4939079284668\\
7.30063676834106 26.4939079284668\\
7.34484767913818 26.4939079284668\\
7.3890585899353 26.4939079284668\\
7.43326950073242 26.4939079284668\\
7.47748041152954 26.4939079284668\\
7.5216908454895 26.4939079284668\\
7.56590175628662 26.4939079284668\\
7.61011266708374 26.4939079284668\\
7.65432357788086 26.4939079284668\\
7.69853448867798 26.4939079284668\\
7.7427453994751 26.4939079284668\\
7.78695631027222 26.4939079284668\\
7.83116674423218 26.4939079284668\\
7.8753776550293 26.4939079284668\\
7.91958856582642 26.4939079284668\\
7.96379947662354 26.4939079284668\\
8.0080099105835 26.4939079284668\\
8.05222129821777 26.4939079284668\\
8.09643173217773 26.4939079284668\\
8.14064311981201 26.4939079284668\\
8.18485355377197 26.4939079284668\\
8.22906494140625 26.4939079284668\\
8.27327537536621 26.4939079284668\\
8.31748580932617 26.4939079284668\\
8.36169719696045 26.4939079284668\\
8.40590763092041 26.4939079284668\\
8.45011901855469 26.4939079284668\\
8.49432945251465 26.4939079284668\\
8.53854084014893 26.4939079284668\\
8.58275127410889 26.4939079284668\\
8.62696170806885 26.4939079284668\\
8.67117309570313 26.4939079284668\\
8.71538352966309 26.4939079284668\\
8.75959491729736 26.4939079284668\\
8.80380535125732 26.4939079284668\\
8.8480167388916 26.4939079284668\\
8.89222717285156 26.4939079284668\\
8.93643760681152 26.4939079284668\\
8.9806489944458 26.4939079284668\\
9.02485942840576 26.4939079284668\\
9.06907081604004 26.4939079284668\\
9.11328125 26.4939079284668\\
9.15749263763428 26.4939079284668\\
9.20170307159424 26.4939079284668\\
9.2459135055542 26.4939079284668\\
9.29012489318848 26.4939079284668\\
9.33433532714844 26.4939079284668\\
9.37854671478271 26.4939079284668\\
9.42275714874268 26.4939079284668\\
9.46696758270264 26.4939079284668\\
9.51117897033691 26.4939079284668\\
9.55538940429688 26.4939079284668\\
9.59960079193115 26.4939079284668\\
9.64381122589111 26.4939079284668\\
9.68802261352539 26.4939079284668\\
9.73223304748535 26.4939079284668\\
9.77644348144531 26.4939079284668\\
9.82065486907959 26.4939079284668\\
9.86486530303955 26.4939079284668\\
9.90907669067383 26.4939079284668\\
9.95328712463379 26.4939079284668\\
9.99749851226807 26.4939079284668\\
10.041708946228 26.4939079284668\\
10.085919380188 26.4939079284668\\
10.1301307678223 26.4939079284668\\
10.1743412017822 26.4939079284668\\
10.2185525894165 26.4939079284668\\
10.2627630233765 26.4939079284668\\
10.3069744110107 26.4939079284668\\
10.3511848449707 26.4939079284668\\
10.3953952789307 26.4939079284668\\
10.4396066665649 26.4939079284668\\
10.4838171005249 26.4939079284668\\
10.5280284881592 26.4939079284668\\
10.5722389221191 26.4939079284668\\
10.6164503097534 26.4939079284668\\
10.6606607437134 26.4939079284668\\
10.7048711776733 26.4939079284668\\
10.7490825653076 26.4939079284668\\
10.7932929992676 26.4939079284668\\
10.8375043869019 26.4939079284668\\
10.8817148208618 26.4939079284668\\
10.9259262084961 26.4939079284668\\
10.9701366424561 26.4939079284668\\
11.014347076416 26.4939079284668\\
11.0585584640503 26.4939079284668\\
11.1027688980103 26.4939079284668\\
11.1469802856445 26.4939079284668\\
11.1911907196045 26.4939079284668\\
11.2354021072388 26.4939079284668\\
11.2796125411987 26.4939079284668\\
11.3238229751587 26.4939079284668\\
11.368034362793 26.4939079284668\\
11.4122447967529 26.4939079284668\\
11.4561223983765 26.4939079284668\\
11.5 26.4939079284668\\
11.54421043396 26.4939079284668\\
11.5884218215942 26.4939079284668\\
11.6326322555542 26.4939079284668\\
11.6768436431885 26.4939079284668\\
11.7210540771484 26.4939079284668\\
11.7652654647827 26.4939079284668\\
11.8094758987427 26.4939079284668\\
11.8536863327026 26.4939079284668\\
11.8978977203369 26.4939079284668\\
11.9421081542969 26.4939079284668\\
11.9863195419312 26.4939079284668\\
12.0305299758911 26.4939079284668\\
12.0747413635254 26.4939079284668\\
12.1189517974854 26.4939079284668\\
12.1631622314453 26.4939079284668\\
12.2073736190796 26.4939079284668\\
12.2515840530396 26.4939079284668\\
12.2957954406738 26.4939079284668\\
12.3400058746338 26.4939079284668\\
12.3842172622681 26.4939079284668\\
12.428427696228 26.4939079284668\\
12.472638130188 26.4939079284668\\
12.5168495178223 26.4939079284668\\
12.5610599517822 26.4939079284668\\
12.6052713394165 26.4939079284668\\
12.6494817733765 26.4939079284668\\
12.6936931610107 26.4939079284668\\
12.7379035949707 26.4939079284668\\
12.7821140289307 26.4939079284668\\
12.8263254165649 26.4939079284668\\
12.8705358505249 26.4939079284668\\
12.9147472381592 26.4939079284668\\
12.9589576721191 26.4939079284668\\
13.0031690597534 26.4939079284668\\
13.0473794937134 26.4939079284668\\
13.0915899276733 26.4939079284668\\
13.1358013153076 26.4939079284668\\
13.1800117492676 26.4939079284668\\
13.2242231369019 26.4939079284668\\
13.2684335708618 26.4939079284668\\
13.3126440048218 26.4939079284668\\
13.3568553924561 26.4939079284668\\
13.401065826416 26.4939079284668\\
13.4452772140503 26.4939079284668\\
13.4894876480103 26.4939079284668\\
13.5336990356445 26.4939079284668\\
13.5779094696045 26.4939079284668\\
13.6221199035645 26.4939079284668\\
13.6663312911987 26.4939079284668\\
13.7105417251587 26.4939079284668\\
13.754753112793 26.4939079284668\\
13.7989635467529 26.4939079284668\\
13.8431749343872 26.4939079284668\\
13.8873853683472 26.4939079284668\\
13.9315958023071 26.4939079284668\\
13.9758071899414 26.4939079284668\\
14.0200176239014 26.4939079284668\\
14.0642290115356 26.4939079284668\\
14.1084394454956 26.4939079284668\\
14.1526508331299 26.4939079284668\\
14.1968612670898 26.4939079284668\\
14.2410717010498 26.4939079284668\\
14.2852830886841 26.4939079284668\\
14.329493522644 26.4939079284668\\
14.3737049102783 26.4939079284668\\
14.4179153442383 26.4939079284668\\
14.4621267318726 26.4939079284668\\
14.5063371658325 26.4939079284668\\
14.5505475997925 26.4939079284668\\
14.5947589874268 26.4939079284668\\
14.6389694213867 26.4939079284668\\
14.683180809021 26.4939079284668\\
14.727391242981 26.4939079284668\\
14.7716026306152 26.4939079284668\\
14.8158130645752 26.4939079284668\\
14.8600234985352 26.4939079284668\\
14.9042348861694 26.4939079284668\\
14.9484453201294 26.4939079284668\\
14.9926567077637 26.4939079284668\\
15.0368671417236 26.4939079284668\\
15.0810785293579 26.4939079284668\\
15.1252889633179 26.4939079284668\\
15.1694993972778 26.4939079284668\\
15.2137107849121 26.4939079284668\\
15.2579212188721 26.4939079284668\\
15.3021326065063 26.4939079284668\\
15.3463430404663 26.4939079284668\\
15.3905544281006 26.4939079284668\\
15.4347648620605 26.4939079284668\\
15.4789752960205 26.4939079284668\\
15.5231866836548 26.4939079284668\\
15.5673971176147 26.4939079284668\\
15.611608505249 26.4939079284668\\
15.655818939209 26.4939079284668\\
15.7000303268433 26.4939079284668\\
15.7442407608032 26.4939079284668\\
15.7884511947632 26.4939079284668\\
15.8326625823975 26.4939079284668\\
15.8768730163574 26.4939079284668\\
15.9210844039917 26.4939079284668\\
15.9652948379517 26.4939079284668\\
16.0095062255859 26.4939079284668\\
16.0537166595459 26.4939079284668\\
16.0967807769775 26.4939079284668\\
16.1407527923584 26.4939079284668\\
16.1849117279053 26.4939079284668\\
16.2291107177734 26.4939079284668\\
16.2733192443848 26.4939079284668\\
16.3175296783447 26.4939079284668\\
16.3617401123047 26.4939079284668\\
16.4059524536133 26.4939079284668\\
16.4501628875732 26.4939079284668\\
16.4943733215332 26.4939079284668\\
16.5385837554932 26.4939079284668\\
16.5827941894531 26.4939079284668\\
16.6270065307617 26.4939079284668\\
16.6712169647217 26.4939079284668\\
16.7154273986816 26.4939079284668\\
16.7596378326416 26.4939079284668\\
16.8038482666016 26.4939079284668\\
16.8480606079102 26.4939079284668\\
16.8922710418701 26.4939079284668\\
16.9364814758301 26.4939079284668\\
16.98069190979 26.4939079284668\\
17.0249042510986 26.4939079284668\\
17.0691146850586 26.4939079284668\\
17.1133251190186 26.4939079284668\\
17.1575355529785 26.4939079284668\\
17.2017459869385 26.4939079284668\\
17.2459583282471 26.4939079284668\\
17.290168762207 26.4939079284668\\
17.334379196167 26.4939079284668\\
17.378589630127 26.4939079284668\\
17.4228000640869 26.4939079284668\\
17.4670124053955 26.4939079284668\\
17.5112228393555 26.4939079284668\\
17.5554332733154 26.4939079284668\\
17.5996437072754 26.4939079284668\\
17.643856048584 26.4939079284668\\
17.6880664825439 26.4939079284668\\
17.7322769165039 26.4939079284668\\
17.7764873504639 26.4939079284668\\
17.8206977844238 26.4939079284668\\
17.8649101257324 26.4939079284668\\
17.9091205596924 26.4939079284668\\
17.9533309936523 26.4939079284668\\
17.9975414276123 26.4939079284668\\
18.0417518615723 26.4939079284668\\
18.0859642028809 26.4939079284668\\
18.1301746368408 26.4939079284668\\
18.1743850708008 26.4939079284668\\
18.2185955047607 26.4939079284668\\
18.2628078460693 26.4939079284668\\
18.3070182800293 26.4939079284668\\
18.3512287139893 26.4939079284668\\
18.3954391479492 26.4939079284668\\
18.4396495819092 26.4939079284668\\
18.4838619232178 26.4939079284668\\
18.5280723571777 26.4939079284668\\
18.5722827911377 26.4939079284668\\
18.6164932250977 26.4939079284668\\
18.6607036590576 26.4939079284668\\
18.7049160003662 26.4939079284668\\
18.7491264343262 26.4939079284668\\
18.7933368682861 26.4939079284668\\
18.8375473022461 26.4939079284668\\
18.8817577362061 26.4939079284668\\
18.9259700775146 26.4939079284668\\
18.962984085083 26.4939079284668\\
19 26.4939079284668\\
};
\addlegendentry{$A_{r} = 0.25$};

\addplot [
color=black,
dotted
]
table[row sep=crcr]{
0 0\\
9.99999974737875e-06 0\\
2.49999993684469e-05 0\\
4.75000015285332e-05 0\\
8.12500002211891e-05 0\\
0.000131875000079162 0\\
0.000207812496228144 0\\
0.000321718747727573 0\\
0.000492578139528632 0\\
0.00074886716902256 0\\
0.00113330082967877 0\\
0.00170995120424777 0\\
0.00257492670789361 0\\
0.00387239013798535 0\\
0.00529959984123707 0\\
0.00686953077092767 0\\
0.0085964547470212 0\\
0.0104960706084967 0\\
0.0125856483355165 0\\
0.014884184114635 0\\
0.0174125730991364 0\\
0.020193800330162 0\\
0.0232531521469355 0\\
0.0266184378415346 0\\
0.0303202513605356 0\\
0.0343922488391399 0\\
0.0388714447617531 0\\
0.0437985584139824 0\\
0.0492183864116669 0\\
0.0551801957190037 0\\
0.0617381855845451 0\\
0.0689519718289375 0\\
0.0768871381878853 0\\
0.0856158286333084 0\\
0.095217376947403 0\\
0.105779089033604 0\\
0.117396965622902 0\\
0.13017663359642 0\\
0.144234269857407 0\\
0.159697666764259 0\\
0.176707401871681 0\\
0.195418119430542 0\\
0.215999901294708 0\\
0.238639861345291 0\\
0.2635438144207 0\\
0.290938168764114 0\\
0.321071952581406 0\\
0.354219108819962 0\\
0.390680998563766 0\\
0.427134186029434 0\\
0.461836785078049 0\\
0.495115727186203 0\\
0.527189493179321 0\\
0.558222532272339 0\\
0.588342726230621 0\\
0.617652356624603 0\\
0.646234631538391 0\\
0.674158751964569 0\\
0.701483011245728 0\\
0.728256940841675 0\\
0.754523456096649 0\\
0.780319631099701 0\\
0.805678188800812 0\\
0.830627858638763 0\\
0.855194389820099 0\\
0.879400551319122 0\\
0.903267025947571 0\\
0.926812350749969 0\\
0.950053453445435 0\\
0.973005652427673 0\\
0.995683073997498 0\\
1.01809847354889 0\\
1.0402637720108 0\\
1.06237816810608 0\\
1.08448839187622 0\\
1.10659623146057 0\\
1.12870299816132 0\\
1.15080916881561 0\\
1.17291498184204 0\\
1.19502067565918 0\\
1.21712636947632 0\\
1.23923194408417 0\\
1.26133751869202 0\\
1.28344309329987 0\\
1.30554866790771 0\\
1.32765424251556 0\\
1.34975969791412 0\\
1.37186527252197 0\\
1.39397084712982 0\\
1.41607642173767 0\\
1.43818199634552 0\\
1.46028745174408 0\\
1.48239302635193 0\\
1.50449860095978 0\\
1.52660417556763 0\\
1.54870975017548 0\\
1.57081520557404 0\\
1.59292078018188 0\\
1.61502635478973 0\\
1.63713192939758 0\\
1.65923750400543 0\\
1.68134295940399 0\\
1.70344853401184 0\\
1.72555410861969 0\\
1.74765968322754 0\\
1.76976525783539 0\\
1.79187071323395 0\\
1.8139762878418 0\\
1.83608186244965 0\\
1.8581874370575 0\\
1.88029301166534 0\\
1.9023984670639 0\\
1.92450404167175 0\\
1.9466096162796 0\\
1.96871519088745 0\\
1.9908207654953 0\\
2.01292634010315 0\\
2.03503179550171 0\\
2.05713748931885 0\\
2.07924294471741 0\\
2.10134840011597 0\\
2.12345409393311 0\\
2.14555954933167 0\\
2.1676652431488 0\\
2.18977069854736 0\\
2.21187615394592 0\\
2.23398184776306 0\\
2.25608730316162 0\\
2.27819299697876 0\\
2.30029845237732 0\\
2.32240390777588 0\\
2.34450960159302 0\\
2.36661505699158 0\\
2.38872075080872 0\\
2.41082620620728 0\\
2.43293166160584 0\\
2.45503735542297 0\\
2.47714281082153 0\\
2.49924850463867 0\\
2.52135396003723 0\\
2.54345941543579 0\\
2.56556510925293 0\\
2.58767056465149 0\\
2.60977625846863 0\\
2.63188171386719 0\\
2.65398716926575 0\\
2.67609286308289 0\\
2.69819831848145 0\\
2.72030401229858 0\\
2.74240946769714 0\\
2.7645149230957 0\\
2.78662061691284 0\\
2.8087260723114 0\\
2.83083176612854 0\\
2.8529372215271 0\\
2.87504267692566 0\\
2.8971483707428 0\\
2.91925382614136 0\\
2.9413595199585 0\\
2.96346497535706 0\\
2.98557043075562 0\\
3.00767612457275 0\\
3.02978157997131 0\\
3.0519700050354 0\\
3.07440447807312 0\\
3.09709739685059 0\\
3.12005996704102 0\\
3.14330291748047 0\\
3.166836977005 0\\
3.19067335128784 0\\
3.21482396125793 0\\
3.2393012046814 0\\
3.26411890983582 0\\
3.28929138183594 0\\
3.31483387947083 0\\
3.34076333045959 0\\
3.36709761619568 0\\
3.39385652542114 0\\
3.42106103897095 0\\
3.44873404502869 0\\
3.47690057754517 0\\
3.50558733940125 0\\
3.53482460975647 0\\
3.5646448135376 0\\
3.59508395195007 0\\
3.62618231773376 0\\
3.65798377990723 0\\
3.69053816795349 0\\
3.72390079498291 0\\
3.75813484191895 0\\
3.79331135749817 0\\
3.82951164245605 0\\
3.86682963371277 0\\
3.90537428855896 0\\
3.94527339935303 0\\
3.98667812347412 0\\
4.02976989746094 0\\
4.07350301742554 0\\
4.11750602722168 0\\
4.16167545318604 0\\
4.2058801651001 0\\
4.2500901222229 0\\
4.29430055618286 0\\
4.33851099014282 0\\
4.38272142410278 0\\
4.4269323348999 0\\
4.47114324569702 0\\
4.51535367965698 0\\
4.5595645904541 0\\
4.60377550125122 0\\
4.64798641204834 0\\
4.69219732284546 0\\
4.73640823364258 0\\
4.7806191444397 0\\
4.82482957839966 0\\
4.86904048919678 0\\
4.9132513999939 0\\
4.95746231079102 0\\
5.00167322158813 0\\
5.04588413238525 0\\
5.09009504318237 0\\
5.13430547714233 0\\
5.17851638793945 0\\
5.22272729873657 0\\
5.26693820953369 0\\
5.31114912033081 0\\
5.35536003112793 0\\
5.39957046508789 0\\
5.44378137588501 0\\
5.48799228668213 0\\
5.53220319747925 0\\
5.57641410827637 0\\
5.62062501907349 0\\
5.66483592987061 0\\
5.70904636383057 0\\
5.75325727462769 0\\
5.7974681854248 0\\
5.84167909622192 0\\
5.88589000701904 0\\
5.93010091781616 0\\
5.97431182861328 0\\
6.01852226257324 0\\
6.06273317337036 0\\
6.10694408416748 0\\
6.1511549949646 0\\
6.19536590576172 0\\
6.23957681655884 0\\
6.28378772735596 0\\
6.32799816131592 0\\
6.37220907211304 0\\
6.41641998291016 0\\
6.46063089370728 0\\
6.50484180450439 0\\
6.54905271530151 0\\
6.59326362609863 0\\
6.63747406005859 0\\
6.68168497085571 0\\
6.72589588165283 0\\
6.77010679244995 0\\
6.81431770324707 0\\
6.85852861404419 0\\
6.90273904800415 0\\
6.94694995880127 0\\
6.99116086959839 0\\
7.03537178039551 0\\
7.07958269119263 0\\
7.12379360198975 0\\
7.16800451278687 0\\
7.21221494674683 0\\
7.25642585754395 0\\
7.30063676834106 0\\
7.34484767913818 0\\
7.3890585899353 0\\
7.43326950073242 0\\
7.47748041152954 0\\
7.5216908454895 0\\
7.56590175628662 0\\
7.61011266708374 0\\
7.65432357788086 0\\
7.69853448867798 0\\
7.7427453994751 0\\
7.78695631027222 0\\
7.83116674423218 0\\
7.8753776550293 0\\
7.91958856582642 0\\
7.96379947662354 0\\
8.0080099105835 0\\
8.05222129821777 0\\
8.09643173217773 0\\
8.14064311981201 0\\
8.18485355377197 0\\
8.22906494140625 0\\
8.27327537536621 0\\
8.31748580932617 0\\
8.36169719696045 0\\
8.40590763092041 0\\
8.45011901855469 0\\
8.49432945251465 0\\
8.53854084014893 0\\
8.58275127410889 0\\
8.62696170806885 0\\
8.67117309570313 0\\
8.71538352966309 0\\
8.75959491729736 0\\
8.80380535125732 0\\
8.8480167388916 0\\
8.89222717285156 0\\
8.93643760681152 0\\
8.9806489944458 0\\
9.02485942840576 0\\
9.06907081604004 0\\
9.11328125 0\\
9.15749263763428 0\\
9.20170307159424 0\\
9.2459135055542 0\\
9.29012489318848 0\\
9.33433532714844 0\\
9.37854671478271 0\\
9.42275714874268 0\\
9.46696758270264 0\\
9.51117897033691 0\\
9.55538940429688 0\\
9.59960079193115 0\\
9.64381122589111 0\\
9.68802261352539 0\\
9.73223304748535 0\\
9.77644348144531 0\\
9.82065486907959 0\\
9.86486530303955 0\\
9.90907669067383 0\\
9.95328712463379 0\\
9.99749851226807 0\\
10.041708946228 0\\
10.085919380188 0\\
10.1301307678223 0\\
10.1743412017822 0\\
10.2185525894165 0\\
10.2627630233765 0\\
10.3069744110107 0\\
10.3511848449707 0\\
10.3953952789307 0\\
10.4396066665649 0\\
10.4838171005249 0\\
10.5280284881592 0\\
10.5722389221191 0\\
10.6164503097534 0\\
10.6606607437134 0\\
10.7048711776733 0\\
10.7490825653076 0\\
10.7932929992676 0\\
10.8375043869019 0\\
10.8817148208618 0\\
10.9259262084961 0\\
10.9701366424561 0\\
11.014347076416 0\\
11.0585584640503 0\\
11.1027688980103 0\\
11.1469802856445 0\\
11.1911907196045 0\\
11.2354021072388 0\\
11.2796125411987 0\\
11.3238229751587 0\\
11.368034362793 0\\
11.4122447967529 0\\
11.4561223983765 0\\
11.5 0\\
11.54421043396 0\\
11.5884218215942 0\\
11.6326322555542 0\\
11.6768436431885 0\\
11.7210540771484 0\\
11.7652654647827 0\\
11.8094758987427 0\\
11.8536863327026 0\\
11.8978977203369 0\\
11.9421081542969 0\\
11.9863195419312 0\\
12.0305299758911 0\\
12.0747413635254 0\\
12.1189517974854 0\\
12.1631622314453 0\\
12.2073736190796 0\\
12.2515840530396 0\\
12.2957954406738 0\\
12.3400058746338 0\\
12.3842172622681 0\\
12.428427696228 0\\
12.472638130188 0\\
12.5168495178223 0\\
12.5610599517822 0\\
12.6052713394165 0\\
12.6494817733765 0\\
12.6936931610107 0\\
12.7379035949707 0\\
12.7821140289307 0\\
12.8263254165649 0\\
12.8705358505249 0\\
12.9147472381592 0\\
12.9589576721191 0\\
13.0031690597534 0\\
13.0473794937134 0\\
13.0915899276733 0\\
13.1358013153076 0\\
13.1800117492676 0\\
13.2242231369019 0\\
13.2684335708618 0\\
13.3126440048218 0\\
13.3568553924561 0\\
13.401065826416 0\\
13.4452772140503 0\\
13.4894876480103 0\\
13.5336990356445 0\\
13.5779094696045 0\\
13.6221199035645 0\\
13.6663312911987 0\\
13.7105417251587 0\\
13.754753112793 0\\
13.7989635467529 0\\
13.8431749343872 0\\
13.8873853683472 0\\
13.9315958023071 0\\
13.9758071899414 0\\
14.0200176239014 0\\
14.0642290115356 0\\
14.1084394454956 0\\
14.1526508331299 0\\
14.1968612670898 0\\
14.2410717010498 0\\
14.2852830886841 0\\
14.329493522644 0\\
14.3737049102783 0\\
14.4179153442383 0\\
14.4621267318726 0\\
14.5063371658325 0\\
14.5505475997925 0\\
14.5947589874268 0\\
14.6389694213867 0\\
14.683180809021 0\\
14.727391242981 0\\
14.7716026306152 0\\
14.8158130645752 0\\
14.8600234985352 0\\
14.9042348861694 0\\
14.9484453201294 0\\
14.9926567077637 0\\
15.0368671417236 0\\
15.0810785293579 0\\
15.1252889633179 0\\
15.1694993972778 0\\
15.2137107849121 0\\
15.2579212188721 0\\
15.3021326065063 0\\
15.3463430404663 0\\
15.3905544281006 0\\
15.4347648620605 0\\
15.4789752960205 0\\
15.5231866836548 0\\
15.5673971176147 0\\
15.611608505249 0\\
15.655818939209 0\\
15.7000303268433 0\\
15.7442407608032 0\\
15.7884511947632 0\\
15.8326625823975 0\\
15.8768730163574 0\\
15.9210844039917 0\\
15.9652948379517 0\\
16.0095062255859 0\\
16.0537166595459 0\\
16.0967807769775 0\\
16.1407527923584 0\\
16.1849117279053 0\\
16.2291107177734 0\\
16.2733192443848 0\\
16.3175296783447 0\\
16.3617401123047 0\\
16.4059524536133 0\\
16.4501628875732 0\\
16.4943733215332 0\\
16.5385837554932 0\\
16.5827941894531 0\\
16.6270065307617 0\\
16.6712169647217 0\\
16.7154273986816 0\\
16.7596378326416 0\\
16.8038482666016 0\\
16.8480606079102 0\\
16.8922710418701 0\\
16.9364814758301 0\\
16.98069190979 0\\
17.0249042510986 0\\
17.0691146850586 0\\
17.1133251190186 0\\
17.1575355529785 0\\
17.2017459869385 0\\
17.2459583282471 0\\
17.290168762207 0\\
17.334379196167 0\\
17.378589630127 0\\
17.4228000640869 0\\
17.4670124053955 0\\
17.5112228393555 0\\
17.5554332733154 0\\
17.5996437072754 0\\
17.643856048584 0\\
17.6880664825439 0\\
17.7322769165039 0\\
17.7764873504639 0\\
17.8206977844238 0\\
17.8649101257324 0\\
17.9091205596924 0\\
17.9533309936523 0\\
17.9975414276123 0\\
18.0417518615723 0\\
18.0859642028809 0\\
18.1301746368408 0\\
18.1743850708008 0\\
18.2185955047607 0\\
18.2628078460693 0\\
18.3070182800293 0\\
18.3512287139893 0\\
18.3954391479492 0\\
18.4396495819092 0\\
18.4838619232178 0\\
18.5280723571777 0\\
18.5722827911377 0\\
18.6164932250977 0\\
18.6607036590576 0\\
18.7049160003662 0\\
18.7491264343262 0\\
18.7933368682861 0\\
18.8375473022461 0\\
18.8817577362061 0\\
18.9259700775146 0\\
18.962984085083 0\\
19 0\\
};
\addlegendentry{$A_{r} = 0.00$};

\end{axis}
\end{tikzpicture}%
\caption{Valve problem with \dtmax{} = \expneg{6.25}{2}{[s]} and domain decomposition active.}
\label{fig:valveDom6pt25em02}
\end{figure}

The results for subsequently smaller timesteps are indistinguishable from those presented in \fig{fig:valveNlnSols}.
This result indicates that the nonlinear solution can be obtained with the domain decomposition algorithm.
Since the nonlinearities were designed to be isolated, this result is expected.
However, looking at the run time statistics, \tab{tab:valveDomTable}, provides additional information about the domain decomposition algorithm's performance.

\begin{table}[h!tb]
\centering
\singlespace
\pgfplotstabletypeset[sci zerofill,sci E, col sep=comma,
	columns/0/.style={ column name= $[ \text{s} ]$, precision=1},
	columns/1/.style={ column name= $[ - ]$, int detect},
	columns/2/.style={ column name= $[ \text{s} ]$, precision=3},
	columns/3/.style={ column name= $[ \text{s} ]$, precision=3},
	every head row/.style={
		before row={
			\toprule
			\dtmax{} & $N_{t}$ & $T_{\text{CPU}}$& $\frac{T_{\text{CPU}}}{N_{t}}$ \\
		},
		after row=\midrule
	},
	every last row/.style={
after row=\bottomrule}]{tables/valveDomData.tex}

\caption{Run time data for the valve problem using domain decomposition.}
\label{tab:valveDomTable}
\end{table}

The CPU time per time step of the domain decomposition is lower than that obtained by the nonlinear solver.
\fig{fig:valveRunTime} shows that the domain decomposition algorithm is capable of producing the same solution as the the nonlinear solver at less than half the cost per timestep than the nonlinear solver.

\begin{figure}[h!tb]
\centering
% This file was created by matlab2tikz v0.4.3.
% Copyright (c) 2008--2013, Nico Schlömer <nico.schloemer@gmail.com>
% All rights reserved.
% 
\tikzsetnextfilename{plots/valveRunTime_pdf}
\begin{tikzpicture}

\begin{axis}[%
width=\mytikzpicwidth,
height=\mytikzpicheight,
scale only axis,
xmin=0,
xmax=0.07,
xtick={0,0.01,...,0.07},
xlabel={$\dtmax{} [\text{s}]$},
ymin=0.002,
ymax=0.01,
ylabel={$ \frac{N_{\text{CPU}}}{N_{t}} [ \text{s} ]$},
legend style={at={(0.03,0.97)},anchor=north west,draw=black,fill=white,legend cell align=left}
]
\addplot [
color=black,
solid,
mark=*,
mark options={solid}
]
table[row sep=crcr]{
0.0625 0.00751539412885904\\
0.03125 0.00702985050156713\\
0.015625 0.0054405964910984\\
0.0078125 0.00503855757415295\\
0.0039062 0.00380773306824267\\
0.0019531 0.00335516757331789\\
};
\addlegendentry{Nonlinear};

\addplot [
color=black,
solid,
mark=square,
mark options={solid}
]
table[row sep=crcr]{
0.0625 0.00382859725505114\\
0.03125 0.00374729256145656\\
0.015625 0.00286465347744524\\
0.0078125 0.00276013999246061\\
0.0039062 0.00259938417002559\\
0.0019531 0.00242306245490909\\
};
\addlegendentry{Domain Decomposition};

\addplot [
color=black,
solid,
mark=diamond,
mark options={solid},
]
table[row sep=crcr]{
0.0625 0.00257529853843153\\
0.03125 0.00270016770809889\\
0.015625 0.00362813309766352\\
0.0078125 0.00295217777602375\\
0.0039062 0.0026698368601501\\
0.0019531 0.00252284295856953\\
};
\addlegendentry{Linear Solver};

\end{axis}
\end{tikzpicture}%
\caption{CPU time per timestep for the nonlinear and the domain decomposition runs.}
\label{fig:valveRunTime}
\end{figure}

As the \dtmax{} is reduced, the cost of the nonlinear solver decreases because there are fewer nonlinear iterates required per timestep.
This reduction in the number of Newton iterates per timestep is also evident in the domain decomposition run-time plot.
This problem was developed to show that for a problem with isolable nonlinearities, the application of the domain decomposition algorithm to resolve those nonlinearities will result in a smaller computational cost than the global nonlinear solver while producing results that are qualitatively as accurate.

%============================================================================
% Problem that shows that local nonlinearities can impact the global solution adversely.
%============================================================================

\section{Filling Pipe Problem}
\label{sect:vmp}
This problem was developed to illustrate that the use of the domain decomposition algorithm in the presence of localized nonlinearities produces a global solution that is in-line with the nonlinear solution.
The problem will be run with a large timestep to show how the solution from the linear and the nonlinear solvers differ.
The solution at smaller timesteps is shown to be in agreement with the analytic solution for both the linear and the nonlinear solver.
The solution produced by the domain decomposition algorithm is then shown as a function of timestep size.
Finally, the impact of resolving local nonlinearities on the global solution is discussed.

\subsection{Model}
\label{subsect:vmpModel}

A vertical pipe slowly filling with saturated water at atmospheric pressure is the basis for this simulation.
The physical pipe is 3 [ft] tall, having a 36 [in$^{2}$] cross-sectional area and a wetted perimeter of 24 [in].
The pipe is initially full of saturated steam at 14.7 [psia].
The top of the pipe is open to an infinite reservoir of saturated steam at 14.7 [psia].
Saturated water flows into the bottom of the pipe, displacing the steam.
\comment{It is a steam problem. No air.}
The boundary and initial conditions are contained in \tab{tab:vmpBCIC}.

\begin{table}[h!tb]
\centering
\singlespace
\pgfplotstabletypeset[fixed zerofill, col sep=comma,
	columns/problem/.style={column name=,string type, column type=l},
	columns/p/.style={ column name=Pressure,dec sep align, precision=1},
	columns/h/.style={ column name=Enthalpy,dec sep align, precision=1},
	columns/a_g/.style={ column name=$\alpha_g$, precision=1},
	columns/a_l/.style={ column name=$\alpha_l$, precision=1},
	columns/a_e/.style={ column name=$\alpha_e$, precision=1},
	every head row/.style={
		before row=\toprule,
		after row={& \multicolumn{2}{c}{$[\text{psia}]$} &\multicolumn{2}{c}{$[\frac{\text{BTU}}{\lbm{}}]$} &$[\text{-}]$ & $[\text{-}]$ & $[\text{-}]$\\ \midrule}},
	every last row/.style={
after row=\bottomrule}]{tables/vmpBCICData.tex}

\caption{Initial and boundary conditions for the fill problem.}
\label{tab:vmpBCIC}
\end{table}

The pressure at a given point in the pipe has a simple, closed-form, analytic solution under the assumptions listed below.
The first assumption is that the variations in the liquid's specific volume, $v_{l}$, within the range of pressures experienced during the course of the simulation can be neglected.
The liquid flow rate into the problem is given by \eqref{eqn:vmpMassFlowRate}.

\begin{equation}
\label{eqn:vmpMassFlowRate}
\dot{m}_{l}(t) = \left\{
\begin{array}{cclrcll}
 \dot{m}_{l,\text{max}}\cdot t & [\frac{ \lbm{} }{\text{s}}] & , & 0\; [\text{s}] < & t & \leq 1 & [\text{s}] \\
 \dot{m}_{l,\text{max}}        & [\frac{ \lbm{} }{\text{s}}] & , &                  & t & > 1    & [\text{s}]
\end{array}\right.
\end{equation}

The maximum flow rate, $\dot{m}_{l,\text{max}}$, is 0.406503 $ [ \frac{\lbm{}}{s} ]$.
The volumetric flow rate is given by \eqref{eqn:vmpVolumetricFlowRate}.

\begin{equation}
\label{eqn:vmpVolumetricFlowRate}
\dot{V}(t) = v_{l} \dot{m}_{l}(t)
\end{equation}

Since the pipe has a cross sectional area, $A_{c}$, the volumetric flow rate can be expressed as $\dot{V} = A_{c} \dot{h}(t)$, where $\dot{h}(t)$ is the rate of change in height as a function of time.
Integrating \eqref{eqn:vmpVolumetricFlowRate} produces the total volume of liquid in the pipe, $V(t)$, as a function of time, \eqref{eqn:vmpTotalLiquid}, which can represented as a function of height, $h(t)$.

\begin{equation}
\label{eqn:vmpTotalLiquid}
A_{c} h(t) = 
 \left\{
\begin{array}{cclrcll}
v_{l} \dot{m}_{l,\text{max}}\frac{t^{2}}{2} & [ \text{ft}^{3} ] & , & 0\; [\text{s}] < & t & \leq 1 & [\text{s}] \\
v_{l} \dot{m}_{l,\text{max}} \left( t - \onehalf  \right) & [ \text{ft}^{3} ] & , &                  & t & > 1    & [\text{s}]
\end{array}\right.
\end{equation}

Neglecting the hydrostatic head contribution from the vapor column, the pressure at any point and time in the problem, $P(t, h(t))$, is either 14.7 [psia], $P_o$, if the water level has yet to reach that point, or it is the hydrostatic pressure dictated by the height of the liquid column above that point, \eqref{eqn:vmpHydroStaticHead}.

\begin{equation}
\label{eqn:vmpHydroStaticHead}
P(t, h(t))= 
 \left\{
\begin{array}{cclrcll}
P_o & [ \text{psia} ] & , & 0\; [\text{s}] < & t & \leq t^{*}(h) & [\text{s}] \\
P_o + \frac{ h(t) - h }{ v_{l} } & [ \text{psia} ] & , &  & t & > t^{*}(h) & [\text{s}]
\end{array}\right.
\end{equation}

The function $t^{*}(h)$ in \eqref{eqn:vmpHydroStaticHead} is given by \eqref{eqn:vmpTimeStar} and represents the time it takes the water level to reach a given height.

\begin{equation}
\label{eqn:vmpTimeStar}
t^{*}(h)= 
 \left\{
\begin{array}{cclrcll}
\sqrt{\frac{2 A_c h}{v_{l} \dot{m}_{l,\text{max}}}} & [ \text{s} ] & , & 0\; [ \text{ft}] < & h & \leq \frac{v_l \dot{m}_{l,\text{max}}}{2 A_c} & [\text{ft}] \\
\frac{ A_c h }{v_{l} \dot{m}_{l,\text{max}}} + \onehalf & [ \text{s} ] & , &                & h & > \frac{v_l \dot{m}_{l,\text{max}}}{2 A_c} & [\text{ft}]
\end{array}\right.
\end{equation}

This analytic function for pressure evaluated at six different axial heights from 0.0 [s] to 75.0 [s] is shown in \fig{fig:vmpAnalyticSol}.

\begin{figure}[h!tb]
\centering
% This file was created by matlab2tikz v0.4.3.
% Copyright (c) 2008--2013, Nico Schlömer <nico.schloemer@gmail.com>
% All rights reserved.
% 
\tikzsetnextfilename{plots/vmpAnalytic_eps}
\begin{tikzpicture}

\begin{axis}[%
width=0.6\textwidth,
height=0.473225806451613\textwidth,
scale only axis,
xmin=0,
xmax=75,
xlabel={Time [s]},
ymin=14.6,
ymax=15.8,
ylabel={Pressure [psia]},
legend style={at={(0.03,0.97)},anchor=north west,draw=black,fill=white,legend cell align=left}
]
\addplot [
color=black,
solid
]
table[row sep=crcr]{
0 14.7\\
0.757575757575758 14.7\\
1.51515151515152 14.7\\
2.27272727272727 14.7\\
3.03030303030303 14.7\\
3.78787878787879 14.7\\
4.54545454545455 14.7\\
5.3030303030303 14.7\\
6.06060606060606 14.7\\
6.81818181818182 14.7020837110659\\
7.57575757575758 14.7106386353083\\
8.33333333333333 14.7191935595507\\
9.09090909090909 14.7277484837931\\
9.84848484848485 14.7363034080356\\
10.6060606060606 14.744858332278\\
11.3636363636364 14.7534132565204\\
12.1212121212121 14.7619681807628\\
12.8787878787879 14.7705231050053\\
13.6363636363636 14.7790780292477\\
14.3939393939394 14.7876329534901\\
15.1515151515152 14.7961878777325\\
15.9090909090909 14.804742801975\\
16.6666666666667 14.8132977262174\\
17.4242424242424 14.8218526504598\\
18.1818181818182 14.8304075747022\\
18.9393939393939 14.8389624989447\\
19.6969696969697 14.8475174231871\\
20.4545454545455 14.8560723474295\\
21.2121212121212 14.8646272716719\\
21.969696969697 14.8731821959144\\
22.7272727272727 14.8817371201568\\
23.4848484848485 14.8902920443992\\
24.2424242424242 14.8988469686416\\
25 14.9074018928841\\
25.7575757575758 14.9159568171265\\
26.5151515151515 14.9245117413689\\
27.2727272727273 14.9330666656113\\
28.030303030303 14.9416215898538\\
28.7878787878788 14.9501765140962\\
29.5454545454545 14.9587314383386\\
30.3030303030303 14.967286362581\\
31.0606060606061 14.9758412868234\\
31.8181818181818 14.9843962110659\\
32.5757575757576 14.9929511353083\\
33.3333333333333 15.0015060595507\\
34.0909090909091 15.0100609837931\\
34.8484848484849 15.0186159080356\\
35.6060606060606 15.027170832278\\
36.3636363636364 15.0357257565204\\
37.1212121212121 15.0442806807628\\
37.8787878787879 15.0528356050053\\
38.6363636363636 15.0613905292477\\
39.3939393939394 15.0699454534901\\
40.1515151515151 15.0785003777325\\
40.9090909090909 15.087055301975\\
41.6666666666667 15.0956102262174\\
42.4242424242424 15.1041651504598\\
43.1818181818182 15.1127200747022\\
43.9393939393939 15.1212749989447\\
44.6969696969697 15.1298299231871\\
45.4545454545455 15.1383848474295\\
46.2121212121212 15.1469397716719\\
46.969696969697 15.1554946959144\\
47.7272727272727 15.1640496201568\\
48.4848484848485 15.1726045443992\\
49.2424242424242 15.1811594686416\\
50 15.1897143928841\\
50.7575757575758 15.1982693171265\\
51.5151515151515 15.2068242413689\\
52.2727272727273 15.2153791656113\\
53.030303030303 15.2239340898538\\
53.7878787878788 15.2324890140962\\
54.5454545454545 15.2410439383386\\
55.3030303030303 15.249598862581\\
56.0606060606061 15.2581537868234\\
56.8181818181818 15.2667087110659\\
57.5757575757576 15.2752636353083\\
58.3333333333333 15.2838185595507\\
59.0909090909091 15.2923734837931\\
59.8484848484849 15.3009284080356\\
60.6060606060606 15.309483332278\\
61.3636363636364 15.3180382565204\\
62.1212121212121 15.3265931807628\\
62.8787878787879 15.3351481050053\\
63.6363636363636 15.3437030292477\\
64.3939393939394 15.3522579534901\\
65.1515151515152 15.3608128777325\\
65.9090909090909 15.369367801975\\
66.6666666666667 15.3779227262174\\
67.4242424242424 15.3864776504598\\
68.1818181818182 15.3950325747022\\
68.9393939393939 15.4035874989447\\
69.6969696969697 15.4121424231871\\
70.4545454545455 15.4206973474295\\
71.2121212121212 15.4292522716719\\
71.969696969697 15.4378071959144\\
72.7272727272727 15.4463621201568\\
73.4848484848485 15.4549170443992\\
74.2424242424242 15.4634719686416\\
75 15.4720268928841\\
};
\addlegendentry{2"};

\addplot [
color=black,
dash pattern=on 1pt off 3pt on 3pt off 3pt
]
table[row sep=crcr]{
0 14.7\\
0.757575757575758 14.7\\
1.51515151515152 14.7\\
2.27272727272727 14.7\\
3.03030303030303 14.7\\
3.78787878787879 14.7\\
4.54545454545455 14.7\\
5.3030303030303 14.7\\
6.06060606060606 14.7\\
6.81818181818182 14.7\\
7.57575757575758 14.7\\
8.33333333333333 14.7\\
9.09090909090909 14.7\\
9.84848484848485 14.7\\
10.6060606060606 14.7\\
11.3636363636364 14.7\\
12.1212121212121 14.7\\
12.8787878787879 14.7\\
13.6363636363636 14.7\\
14.3939393939394 14.7\\
15.1515151515152 14.7\\
15.9090909090909 14.7\\
16.6666666666667 14.7\\
17.4242424242424 14.7\\
18.1818181818182 14.7\\
18.9393939393939 14.7004337847128\\
19.6969696969697 14.7089887089552\\
20.4545454545455 14.7175436331976\\
21.2121212121212 14.72609855744\\
21.969696969697 14.7346534816825\\
22.7272727272727 14.7432084059249\\
23.4848484848485 14.7517633301673\\
24.2424242424242 14.7603182544097\\
25 14.7688731786522\\
25.7575757575758 14.7774281028946\\
26.5151515151515 14.785983027137\\
27.2727272727273 14.7945379513794\\
28.030303030303 14.8030928756219\\
28.7878787878788 14.8116477998643\\
29.5454545454545 14.8202027241067\\
30.3030303030303 14.8287576483491\\
31.0606060606061 14.8373125725916\\
31.8181818181818 14.845867496834\\
32.5757575757576 14.8544224210764\\
33.3333333333333 14.8629773453188\\
34.0909090909091 14.8715322695613\\
34.8484848484849 14.8800871938037\\
35.6060606060606 14.8886421180461\\
36.3636363636364 14.8971970422885\\
37.1212121212121 14.905751966531\\
37.8787878787879 14.9143068907734\\
38.6363636363636 14.9228618150158\\
39.3939393939394 14.9314167392582\\
40.1515151515151 14.9399716635007\\
40.9090909090909 14.9485265877431\\
41.6666666666667 14.9570815119855\\
42.4242424242424 14.9656364362279\\
43.1818181818182 14.9741913604704\\
43.9393939393939 14.9827462847128\\
44.6969696969697 14.9913012089552\\
45.4545454545455 14.9998561331976\\
46.2121212121212 15.0084110574401\\
46.969696969697 15.0169659816825\\
47.7272727272727 15.0255209059249\\
48.4848484848485 15.0340758301673\\
49.2424242424242 15.0426307544097\\
50 15.0511856786522\\
50.7575757575758 15.0597406028946\\
51.5151515151515 15.068295527137\\
52.2727272727273 15.0768504513794\\
53.030303030303 15.0854053756219\\
53.7878787878788 15.0939602998643\\
54.5454545454545 15.1025152241067\\
55.3030303030303 15.1110701483491\\
56.0606060606061 15.1196250725916\\
56.8181818181818 15.128179996834\\
57.5757575757576 15.1367349210764\\
58.3333333333333 15.1452898453188\\
59.0909090909091 15.1538447695613\\
59.8484848484849 15.1623996938037\\
60.6060606060606 15.1709546180461\\
61.3636363636364 15.1795095422885\\
62.1212121212121 15.188064466531\\
62.8787878787879 15.1966193907734\\
63.6363636363636 15.2051743150158\\
64.3939393939394 15.2137292392582\\
65.1515151515152 15.2222841635007\\
65.9090909090909 15.2308390877431\\
66.6666666666667 15.2393940119855\\
67.4242424242424 15.2479489362279\\
68.1818181818182 15.2565038604704\\
68.9393939393939 15.2650587847128\\
69.6969696969697 15.2736137089552\\
70.4545454545455 15.2821686331976\\
71.2121212121212 15.2907235574401\\
71.969696969697 15.2992784816825\\
72.7272727272727 15.3078334059249\\
73.4848484848485 15.3163883301673\\
74.2424242424242 15.3249432544097\\
75 15.3334981786522\\
};
\addlegendentry{6 "};

\addplot [
color=black,
dashed
]
table[row sep=crcr]{
0 14.7\\
0.757575757575758 14.7\\
1.51515151515152 14.7\\
2.27272727272727 14.7\\
3.03030303030303 14.7\\
3.78787878787879 14.7\\
4.54545454545455 14.7\\
5.3030303030303 14.7\\
6.06060606060606 14.7\\
6.81818181818182 14.7\\
7.57575757575758 14.7\\
8.33333333333333 14.7\\
9.09090909090909 14.7\\
9.84848484848485 14.7\\
10.6060606060606 14.7\\
11.3636363636364 14.7\\
12.1212121212121 14.7\\
12.8787878787879 14.7\\
13.6363636363636 14.7\\
14.3939393939394 14.7\\
15.1515151515152 14.7\\
15.9090909090909 14.7\\
16.6666666666667 14.7\\
17.4242424242424 14.7\\
18.1818181818182 14.7\\
18.9393939393939 14.7\\
19.6969696969697 14.7\\
20.4545454545455 14.7\\
21.2121212121212 14.7\\
21.969696969697 14.7\\
22.7272727272727 14.7\\
23.4848484848485 14.7\\
24.2424242424242 14.7\\
25 14.7\\
25.7575757575758 14.7\\
26.5151515151515 14.7\\
27.2727272727273 14.7\\
28.030303030303 14.7\\
28.7878787878788 14.7\\
29.5454545454545 14.7\\
30.3030303030303 14.7\\
31.0606060606061 14.7\\
31.8181818181818 14.7073387826021\\
32.5757575757576 14.7158937068445\\
33.3333333333333 14.724448631087\\
34.0909090909091 14.7330035553294\\
34.8484848484849 14.7415584795718\\
35.6060606060606 14.7501134038142\\
36.3636363636364 14.7586683280566\\
37.1212121212121 14.7672232522991\\
37.8787878787879 14.7757781765415\\
38.6363636363636 14.7843331007839\\
39.3939393939394 14.7928880250263\\
40.1515151515151 14.8014429492688\\
40.9090909090909 14.8099978735112\\
41.6666666666667 14.8185527977536\\
42.4242424242424 14.827107721996\\
43.1818181818182 14.8356626462385\\
43.9393939393939 14.8442175704809\\
44.6969696969697 14.8527724947233\\
45.4545454545455 14.8613274189657\\
46.2121212121212 14.8698823432082\\
46.969696969697 14.8784372674506\\
47.7272727272727 14.886992191693\\
48.4848484848485 14.8955471159354\\
49.2424242424242 14.9041020401779\\
50 14.9126569644203\\
50.7575757575758 14.9212118886627\\
51.5151515151515 14.9297668129051\\
52.2727272727273 14.9383217371476\\
53.030303030303 14.94687666139\\
53.7878787878788 14.9554315856324\\
54.5454545454545 14.9639865098748\\
55.3030303030303 14.9725414341173\\
56.0606060606061 14.9810963583597\\
56.8181818181818 14.9896512826021\\
57.5757575757576 14.9982062068445\\
58.3333333333333 15.006761131087\\
59.0909090909091 15.0153160553294\\
59.8484848484849 15.0238709795718\\
60.6060606060606 15.0324259038142\\
61.3636363636364 15.0409808280566\\
62.1212121212121 15.0495357522991\\
62.8787878787879 15.0580906765415\\
63.6363636363636 15.0666456007839\\
64.3939393939394 15.0752005250263\\
65.1515151515152 15.0837554492688\\
65.9090909090909 15.0923103735112\\
66.6666666666667 15.1008652977536\\
67.4242424242424 15.109420221996\\
68.1818181818182 15.1179751462385\\
68.9393939393939 15.1265300704809\\
69.6969696969697 15.1350849947233\\
70.4545454545455 15.1436399189657\\
71.2121212121212 15.1521948432082\\
71.969696969697 15.1607497674506\\
72.7272727272727 15.169304691693\\
73.4848484848485 15.1778596159354\\
74.2424242424242 15.1864145401779\\
75 15.1949694644203\\
};
\addlegendentry{10 "};

\addplot [
color=black,
dotted
]
table[row sep=crcr]{
0 14.7\\
0.757575757575758 14.7\\
1.51515151515152 14.7\\
2.27272727272727 14.7\\
3.03030303030303 14.7\\
3.78787878787879 14.7\\
4.54545454545455 14.7\\
5.3030303030303 14.7\\
6.06060606060606 14.7\\
6.81818181818182 14.7\\
7.57575757575758 14.7\\
8.33333333333333 14.7\\
9.09090909090909 14.7\\
9.84848484848485 14.7\\
10.6060606060606 14.7\\
11.3636363636364 14.7\\
12.1212121212121 14.7\\
12.8787878787879 14.7\\
13.6363636363636 14.7\\
14.3939393939394 14.7\\
15.1515151515152 14.7\\
15.9090909090909 14.7\\
16.6666666666667 14.7\\
17.4242424242424 14.7\\
18.1818181818182 14.7\\
18.9393939393939 14.7\\
19.6969696969697 14.7\\
20.4545454545455 14.7\\
21.2121212121212 14.7\\
21.969696969697 14.7\\
22.7272727272727 14.7\\
23.4848484848485 14.7\\
24.2424242424242 14.7\\
25 14.7\\
25.7575757575758 14.7\\
26.5151515151515 14.7\\
27.2727272727273 14.7\\
28.030303030303 14.7\\
28.7878787878788 14.7\\
29.5454545454545 14.7\\
30.3030303030303 14.7\\
31.0606060606061 14.7\\
31.8181818181818 14.7\\
32.5757575757576 14.7\\
33.3333333333333 14.7\\
34.0909090909091 14.7\\
34.8484848484849 14.7\\
35.6060606060606 14.7\\
36.3636363636364 14.7\\
37.1212121212121 14.7\\
37.8787878787879 14.7\\
38.6363636363636 14.7\\
39.3939393939394 14.7\\
40.1515151515151 14.7\\
40.9090909090909 14.7\\
41.6666666666667 14.7\\
42.4242424242424 14.7\\
43.1818181818182 14.7\\
43.9393939393939 14.705688856249\\
44.6969696969697 14.7142437804914\\
45.4545454545455 14.7227987047339\\
46.2121212121212 14.7313536289763\\
46.969696969697 14.7399085532187\\
47.7272727272727 14.7484634774611\\
48.4848484848485 14.7570184017035\\
49.2424242424242 14.765573325946\\
50 14.7741282501884\\
50.7575757575758 14.7826831744308\\
51.5151515151515 14.7912380986732\\
52.2727272727273 14.7997930229157\\
53.030303030303 14.8083479471581\\
53.7878787878788 14.8169028714005\\
54.5454545454545 14.8254577956429\\
55.3030303030303 14.8340127198854\\
56.0606060606061 14.8425676441278\\
56.8181818181818 14.8511225683702\\
57.5757575757576 14.8596774926126\\
58.3333333333333 14.8682324168551\\
59.0909090909091 14.8767873410975\\
59.8484848484849 14.8853422653399\\
60.6060606060606 14.8938971895823\\
61.3636363636364 14.9024521138248\\
62.1212121212121 14.9110070380672\\
62.8787878787879 14.9195619623096\\
63.6363636363636 14.928116886552\\
64.3939393939394 14.9366718107945\\
65.1515151515152 14.9452267350369\\
65.9090909090909 14.9537816592793\\
66.6666666666667 14.9623365835217\\
67.4242424242424 14.9708915077642\\
68.1818181818182 14.9794464320066\\
68.9393939393939 14.988001356249\\
69.6969696969697 14.9965562804914\\
70.4545454545455 15.0051112047339\\
71.2121212121212 15.0136661289763\\
71.969696969697 15.0222210532187\\
72.7272727272727 15.0307759774611\\
73.4848484848485 15.0393309017036\\
74.2424242424242 15.047885825946\\
75 15.0564407501884\\
};
\addlegendentry{14 "};

\addplot [
color=green,
solid
]
table[row sep=crcr]{
0 14.7\\
0.757575757575758 14.7\\
1.51515151515152 14.7\\
2.27272727272727 14.7\\
3.03030303030303 14.7\\
3.78787878787879 14.7\\
4.54545454545455 14.7\\
5.3030303030303 14.7\\
6.06060606060606 14.7\\
6.81818181818182 14.7\\
7.57575757575758 14.7\\
8.33333333333333 14.7\\
9.09090909090909 14.7\\
9.84848484848485 14.7\\
10.6060606060606 14.7\\
11.3636363636364 14.7\\
12.1212121212121 14.7\\
12.8787878787879 14.7\\
13.6363636363636 14.7\\
14.3939393939394 14.7\\
15.1515151515152 14.7\\
15.9090909090909 14.7\\
16.6666666666667 14.7\\
17.4242424242424 14.7\\
18.1818181818182 14.7\\
18.9393939393939 14.7\\
19.6969696969697 14.7\\
20.4545454545455 14.7\\
21.2121212121212 14.7\\
21.969696969697 14.7\\
22.7272727272727 14.7\\
23.4848484848485 14.7\\
24.2424242424242 14.7\\
25 14.7\\
25.7575757575758 14.7\\
26.5151515151515 14.7\\
27.2727272727273 14.7\\
28.030303030303 14.7\\
28.7878787878788 14.7\\
29.5454545454545 14.7\\
30.3030303030303 14.7\\
31.0606060606061 14.7\\
31.8181818181818 14.7\\
32.5757575757576 14.7\\
33.3333333333333 14.7\\
34.0909090909091 14.7\\
34.8484848484849 14.7\\
35.6060606060606 14.7\\
36.3636363636364 14.7\\
37.1212121212121 14.7\\
37.8787878787879 14.7\\
38.6363636363636 14.7\\
39.3939393939394 14.7\\
40.1515151515151 14.7\\
40.9090909090909 14.7\\
41.6666666666667 14.7\\
42.4242424242424 14.7\\
43.1818181818182 14.7\\
43.9393939393939 14.7\\
44.6969696969697 14.7\\
45.4545454545455 14.7\\
46.2121212121212 14.7\\
46.969696969697 14.7\\
47.7272727272727 14.7\\
48.4848484848485 14.7\\
49.2424242424242 14.7\\
50 14.7\\
50.7575757575758 14.7\\
51.5151515151515 14.7\\
52.2727272727273 14.7\\
53.030303030303 14.7\\
53.7878787878788 14.7\\
54.5454545454545 14.7\\
55.3030303030303 14.7\\
56.0606060606061 14.7040389298959\\
56.8181818181818 14.7125938541383\\
57.5757575757576 14.7211487783808\\
58.3333333333333 14.7297037026232\\
59.0909090909091 14.7382586268656\\
59.8484848484849 14.746813551108\\
60.6060606060606 14.7553684753505\\
61.3636363636364 14.7639233995929\\
62.1212121212121 14.7724783238353\\
62.8787878787879 14.7810332480777\\
63.6363636363636 14.7895881723201\\
64.3939393939394 14.7981430965626\\
65.1515151515152 14.806698020805\\
65.9090909090909 14.8152529450474\\
66.6666666666667 14.8238078692898\\
67.4242424242424 14.8323627935323\\
68.1818181818182 14.8409177177747\\
68.9393939393939 14.8494726420171\\
69.6969696969697 14.8580275662595\\
70.4545454545455 14.866582490502\\
71.2121212121212 14.8751374147444\\
71.969696969697 14.8836923389868\\
72.7272727272727 14.8922472632292\\
73.4848484848485 14.9008021874717\\
74.2424242424242 14.9093571117141\\
75 14.9179120359565\\
};
\addlegendentry{18 "};

\addplot [
color=green,
dash pattern=on 1pt off 3pt on 3pt off 3pt
]
table[row sep=crcr]{
0 14.7\\
0.757575757575758 14.7\\
1.51515151515152 14.7\\
2.27272727272727 14.7\\
3.03030303030303 14.7\\
3.78787878787879 14.7\\
4.54545454545455 14.7\\
5.3030303030303 14.7\\
6.06060606060606 14.7\\
6.81818181818182 14.7\\
7.57575757575758 14.7\\
8.33333333333333 14.7\\
9.09090909090909 14.7\\
9.84848484848485 14.7\\
10.6060606060606 14.7\\
11.3636363636364 14.7\\
12.1212121212121 14.7\\
12.8787878787879 14.7\\
13.6363636363636 14.7\\
14.3939393939394 14.7\\
15.1515151515152 14.7\\
15.9090909090909 14.7\\
16.6666666666667 14.7\\
17.4242424242424 14.7\\
18.1818181818182 14.7\\
18.9393939393939 14.7\\
19.6969696969697 14.7\\
20.4545454545455 14.7\\
21.2121212121212 14.7\\
21.969696969697 14.7\\
22.7272727272727 14.7\\
23.4848484848485 14.7\\
24.2424242424242 14.7\\
25 14.7\\
25.7575757575758 14.7\\
26.5151515151515 14.7\\
27.2727272727273 14.7\\
28.030303030303 14.7\\
28.7878787878788 14.7\\
29.5454545454545 14.7\\
30.3030303030303 14.7\\
31.0606060606061 14.7\\
31.8181818181818 14.7\\
32.5757575757576 14.7\\
33.3333333333333 14.7\\
34.0909090909091 14.7\\
34.8484848484849 14.7\\
35.6060606060606 14.7\\
36.3636363636364 14.7\\
37.1212121212121 14.7\\
37.8787878787879 14.7\\
38.6363636363636 14.7\\
39.3939393939394 14.7\\
40.1515151515151 14.7\\
40.9090909090909 14.7\\
41.6666666666667 14.7\\
42.4242424242424 14.7\\
43.1818181818182 14.7\\
43.9393939393939 14.7\\
44.6969696969697 14.7\\
45.4545454545455 14.7\\
46.2121212121212 14.7\\
46.969696969697 14.7\\
47.7272727272727 14.7\\
48.4848484848485 14.7\\
49.2424242424242 14.7\\
50 14.7\\
50.7575757575758 14.7\\
51.5151515151515 14.7\\
52.2727272727273 14.7\\
53.030303030303 14.7\\
53.7878787878788 14.7\\
54.5454545454545 14.7\\
55.3030303030303 14.7\\
56.0606060606061 14.7\\
56.8181818181818 14.7\\
57.5757575757576 14.7\\
58.3333333333333 14.7\\
59.0909090909091 14.7\\
59.8484848484849 14.7\\
60.6060606060606 14.7\\
61.3636363636364 14.7\\
62.1212121212121 14.7\\
62.8787878787879 14.7\\
63.6363636363636 14.7\\
64.3939393939394 14.7\\
65.1515151515152 14.7\\
65.9090909090909 14.7\\
66.6666666666667 14.7\\
67.4242424242424 14.7\\
68.1818181818182 14.7023890035428\\
68.9393939393939 14.7109439277852\\
69.6969696969697 14.7194988520277\\
70.4545454545455 14.7280537762701\\
71.2121212121212 14.7366087005125\\
71.969696969697 14.7451636247549\\
72.7272727272727 14.7537185489974\\
73.4848484848485 14.7622734732398\\
74.2424242424242 14.7708283974822\\
75 14.7793833217246\\
};
\addlegendentry{22 "};

\end{axis}
\end{tikzpicture}%
\caption{Analytic solution to the filling problem.}
\label{fig:vmpAnalyticSol}
\end{figure}

The \cobra{} model of this pipe is composed of two sections, each with a single channel.
While it is not necessary for so simple of a problem to have multiple sections, it was done to allow for the domain decomposition algorithm to run, since the algorithm uses channels to decompose the domain.
The channel in the bottom section is 1 [ft] tall and is broken into three continuity volumes, each 4 [in] high.
The top section's channel is 2 [ft] tall and is broken into six continuity volumes, each 4 [in] high.

First, the simulation was run with both the linear and the nonlinear solver.
The nonlinear convergence parameters were \kmax{} = 35, $f_{tol}$ = \expneg{1.0}{6}, $d_{tol}$ = \expneg{1.0}{8}. 
Each simulation was run for 75 $[ \text{s} ]$.
The largest permissible timestep size, \dtmax{}, was chosen as \expneg{1.0}{1}{[s]}.
The \dtmax{} was reduced several times over two orders of magnitude to determine timestep sensitivity.

For each simulation, a time-averaged error measure was calculated.
The error at a given point was determined by taking the square root of the the integral of the squared difference between the analytic pressure, $P_{a}$, and that calculated by \cobra{}, $P_{c}$, divided by the total simulation run time, $T$.
This error at the center of the lower six continuity volumes was summed to produce a measure of the error in the calculated result, \eqref{eqn:vpmPressureError}.
The bottom six continuity volumes were selected since the simulation was run only until they were filled.

\begin{equation}
\label{eqn:vpmPressureError}
e_{p} =  \frac{1}{\sqrt{T}} \sum_{i\,=\,1}^{6}\sqrt{\int_{0}^{T} \left(P_{i,c}(t) - P_{i,a}(t)\right)^{2} \mathrm{d} t}
\end{equation}

Next, the domain decomposition algorithm was activated.
There were two simulations in this run: one was with the bottom channel in the nonlinear domain and the other with the top channel in the nonlinear domain.
The \dtmax{} for these runs was chosen to be \expneg{1.0}{1}{[s]}.
The solution of these runs will be compared to the solutions produced by the single domain solvers.

\subsection{Results}
\label{subsect:vmpResults}

The simulation was first run with the linear solver.
The \dtmax{} was chosen to be \expneg{1.0}{1}{[s]}.
The pressure at the center of the bottom six continuity volumes in the problem are shown in \fig{fig:linFill1em1}.

\begin{figure}[h!tb]
\centering
% This file was created by matlab2tikz v0.4.3.
% Copyright (c) 2008--2013, Nico Schlömer <nico.schloemer@gmail.com>
% All rights reserved.
% 
\tikzsetnextfilename{plots/vmpLinear1em1_eps}
\begin{tikzpicture}

\begin{axis}[%
width=0.8\textwidth,
height=0.508815326102268\textwidth,
scale only axis,
xmin=0.0,
xmax=75,
xlabel={Time [s]},
ymin=14.5,
ymax=15.8,
ylabel={Pressure [psia]},
legend style={at={(0.03,0.97)},anchor=north west,draw=black,fill=white,legend cell align=left}
]
\addplot [
color=black,
solid
]
table[row sep=crcr]{
0 14.7007818222046\\
9.99999971718069e-10 14.7007818222046\\
2.49999998480632e-09 14.7007818222046\\
4.74999994892755e-09 14.7007818222046\\
8.1250002281763e-09 14.7007818222046\\
1.31874999809156e-08 14.7007818222046\\
2.0781250498203e-08 14.7007818222046\\
3.21718758300449e-08 14.7007818222046\\
4.92578138278077e-08 14.7007818222046\\
7.48867208244519e-08 14.7007818222046\\
1.13330081319418e-07 14.7007818222046\\
1.70995122061868e-07 14.7007818222046\\
2.57492672517401e-07 14.7007818222046\\
3.8723902662241e-07 14.7007818222046\\
5.81858500936505e-07 14.7007818222046\\
8.73787769251066e-07 14.7007818222046\\
1.31168167172291e-06 14.7007818222046\\
1.9685226106958e-06 14.7007827758789\\
2.95378367809462e-06 14.7007827758789\\
4.43167573394021e-06 14.7007827758789\\
6.64851359033491e-06 14.7007827758789\\
9.97376992017962e-06 14.7007837295532\\
1.49616553244414e-05 14.7007846832275\\
2.24434825213393e-05 14.7007856369019\\
3.36662233166862e-05 14.7007865905762\\
5.05003372381907e-05 14.7007884979248\\
7.57515081204474e-05 14.7007904052734\\
0.000113628258986864 14.7007932662964\\
0.000170443381648511 14.700795173645\\
0.000255666091106832 14.7007970809937\\
0.000383500126190484 14.7007970809937\\
0.000575251178815961 14.7007942199707\\
0.000862877757754177 14.7007904052734\\
0.00129431765526533 14.7007865905762\\
0.00194147753063589 14.7007837295532\\
0.00291221728548408 14.700779914856\\
0.00436832662671804 14.7007780075073\\
0.00597004732117057 14.700779914856\\
0.00773194013163447 14.7007827758789\\
0.009670021943748 14.7007837295532\\
0.0118019115179777 14.7007837295532\\
0.0141469910740852 14.7007827758789\\
0.0167265776544809 14.7007827758789\\
0.0195641238242388 14.7007827758789\\
0.0226854234933853 14.7007827758789\\
0.0261188540607691 14.7007827758789\\
0.0298956278711557 14.7007827758789\\
0.0340500771999359 14.7007827758789\\
0.0386199727654457 14.7007827758789\\
0.0436468608677387 14.7007827758789\\
0.0491764321923256 14.7007827758789\\
0.0552589632570744 14.7007827758789\\
0.0619497485458851 14.7007827758789\\
0.0693096145987511 14.7007827758789\\
0.0774054601788521 14.7007827758789\\
0.0863108932971954 14.700779914856\\
0.0961068719625473 14.7007827758789\\
0.106882445514202 14.7007827758789\\
0.118735581636429 14.7007827758789\\
0.13177402317524 14.7007827758789\\
0.146116316318512 14.7007827758789\\
0.161892831325531 14.7007827758789\\
0.179247006773949 14.7007827758789\\
0.198336601257324 14.7007827758789\\
0.21933513879776 14.7007827758789\\
0.242433547973633 14.7007827758789\\
0.267841786146164 14.7007827758789\\
0.29579085111618 14.7007827758789\\
0.32653483748436 14.7007827758789\\
0.360353201627731 14.7007827758789\\
0.397553414106369 14.7007837295532\\
0.438473641872406 14.7007837295532\\
0.483485877513886 14.7007837295532\\
0.532999336719513 14.7007837295532\\
0.587464153766632 14.7007837295532\\
0.647375464439392 14.7007837295532\\
0.71327793598175 14.7007837295532\\
0.7857705950737 14.7007837295532\\
0.865512549877167 14.7007837295532\\
0.953228712081909 14.7007837295532\\
1.04971647262573 14.7007837295532\\
1.14971649646759 14.7007827758789\\
1.24971640110016 14.7007827758789\\
1.34971642494202 14.7007827758789\\
1.44971644878387 14.7007827758789\\
1.54971647262573 14.7007827758789\\
1.64971649646759 14.7007827758789\\
1.74971640110016 14.7007827758789\\
1.84971642494202 14.7007827758789\\
1.94971644878387 14.7007827758789\\
2.04971647262573 14.7007827758789\\
2.1497163772583 14.7007827758789\\
2.24971652030945 14.7007827758789\\
2.34971642494202 14.7007827758789\\
2.44971656799316 14.7007827758789\\
2.54971647262573 14.7007827758789\\
2.6497163772583 14.7007827758789\\
2.74971652030945 14.7007827758789\\
2.84971642494202 14.7007827758789\\
2.94971656799316 14.7007827758789\\
3.04971647262573 14.7007827758789\\
3.1497163772583 14.7007827758789\\
3.24971652030945 14.7007827758789\\
3.34971642494202 14.7007827758789\\
3.44971656799316 14.7007827758789\\
3.54971647262573 14.7007827758789\\
3.6497163772583 14.7007827758789\\
3.74971652030945 14.7007827758789\\
3.84971642494202 14.7007827758789\\
3.94971656799316 14.7007827758789\\
4.04971647262573 14.7007827758789\\
4.1497163772583 14.7007827758789\\
4.24971628189087 14.7007827758789\\
4.3497166633606 14.7007827758789\\
4.44971656799316 14.7007827758789\\
4.54971647262573 14.7007827758789\\
4.6497163772583 14.7007827758789\\
4.74971628189087 14.7007827758789\\
4.8497166633606 14.7007827758789\\
4.94971656799316 14.7007827758789\\
5.04971647262573 14.7007837295532\\
5.1497163772583 14.7007837295532\\
5.24971628189087 14.7007837295532\\
5.3497166633606 14.7007837295532\\
5.44971656799316 14.7007837295532\\
5.54971647262573 14.7007837295532\\
5.6497163772583 14.7007837295532\\
5.74971628189087 14.7007884979248\\
5.8497166633606 14.7008085250854\\
5.94971656799316 14.70090675354\\
6.04971647262573 14.7014398574829\\
6.1497163772583 14.704080581665\\
6.24971628189087 14.7125511169434\\
6.3497166633606 14.725396156311\\
6.44971656799316 14.7330207824707\\
6.54971647262573 14.732325553894\\
6.6497163772583 14.733006477356\\
6.74971628189087 14.732626914978\\
6.8497166633606 14.7333736419678\\
6.94971656799316 14.7336568832397\\
7.04971647262573 14.734393119812\\
7.1497163772583 14.7347822189331\\
7.24971628189087 14.7354030609131\\
7.3497166633606 14.7358684539795\\
7.44971656799316 14.7364387512207\\
7.54971647262573 14.7369375228882\\
7.6497163772583 14.7374849319458\\
7.74971628189087 14.7379999160767\\
7.8497166633606 14.7385368347168\\
7.94971656799316 14.7390584945679\\
8.04971599578857 14.7395906448364\\
8.1497163772583 14.7401151657104\\
8.24971675872803 14.7406454086304\\
8.34971618652344 14.741171836853\\
8.44971656799316 14.7417011260986\\
8.54971599578857 14.7422285079956\\
8.6497163772583 14.7427568435669\\
8.74971675872803 14.7432851791382\\
8.84971618652344 14.743842124939\\
8.94971656799316 14.744704246521\\
9.04971599578857 14.7452869415283\\
9.1497163772583 14.7459402084351\\
9.24971675872803 14.7465677261353\\
9.34971618652344 14.7472019195557\\
9.44971656799316 14.7478313446045\\
9.54971599578857 14.748459815979\\
9.6497163772583 14.7490863800049\\
9.74971675872803 14.7497119903564\\
9.84971618652344 14.7503356933594\\
9.94971656799316 14.750958442688\\
10.0497159957886 14.751579284668\\
10.1497163772583 14.7521991729736\\
10.249716758728 14.752818107605\\
10.3497161865234 14.7534303665161\\
10.4497165679932 14.7537994384766\\
10.5497159957886 14.7542762756348\\
10.6497163772583 14.7548017501831\\
10.749716758728 14.7553043365479\\
10.8497161865234 14.7558059692383\\
10.9497165679932 14.7562980651855\\
11.0497159957886 14.7567825317383\\
11.1497163772583 14.7572574615479\\
11.249716758728 14.7577199935913\\
11.3497161865234 14.7581701278687\\
11.4497165679932 14.7586069107056\\
11.5497159957886 14.7590341567993\\
11.6497163772583 14.7594566345215\\
11.749716758728 14.7598266601563\\
11.8497161865234 14.7601566314697\\
11.9497165679932 14.7643003463745\\
12.0497159957886 14.7638158798218\\
12.1497163772583 14.7637815475464\\
12.249716758728 14.7665519714355\\
12.347146987915 14.7727012634277\\
12.4250946044922 14.7783174514771\\
12.4874591827393 14.7692184448242\\
12.5560598373413 14.7686729431152\\
12.6315202713013 14.7690563201904\\
12.714527130127 14.7695341110229\\
12.8058338165283 14.7866821289063\\
12.8497476577759 14.7367706298828\\
12.8980522155762 14.7679605484009\\
12.9143037796021 14.7766189575195\\
12.9277105331421 14.7797164916992\\
12.9424571990967 14.7722702026367\\
12.9586782455444 14.7723064422607\\
12.9765224456787 14.772406578064\\
12.9961500167847 14.7725305557251\\
13.0177412033081 14.77268409729\\
13.0414915084839 14.7728700637817\\
13.0676164627075 14.7730894088745\\
13.0963535308838 14.7733449935913\\
13.1279649734497 14.7736377716064\\
13.1627368927002 14.7739715576172\\
13.2009868621826 14.7743463516235\\
13.2430610656738 14.7747659683228\\
13.289342880249 14.7752323150635\\
13.3402528762817 14.7757482528687\\
13.3962545394897 14.7763185501099\\
13.4578561782837 14.7769460678101\\
13.5256175994873 14.7776365280151\\
13.600154876709 14.778395652771\\
13.6821460723877 14.7879543304443\\
13.7723369598389 14.7718334197998\\
13.871545791626 14.7812509536743\\
13.9715461730957 14.7859354019165\\
14.0715456008911 14.7874298095703\\
14.1215467453003 14.822286605835\\
14.1629066467285 14.8061380386353\\
14.2008075714111 14.8021659851074\\
14.202446937561 14.8055105209351\\
14.2042503356934 14.8240013122559\\
14.2062339782715 14.8408985137939\\
14.2084159851074 14.8495998382568\\
14.2108154296875 14.8495197296143\\
14.2134561538696 14.8429365158081\\
14.2163600921631 14.8314027786255\\
14.219554901123 14.8182067871094\\
14.223069190979 14.8070230484009\\
14.2269344329834 14.8084754943848\\
14.2311868667603 14.8092813491821\\
14.2358636856079 14.8092432022095\\
14.2410087585449 14.8087644577026\\
14.2466678619385 14.8074598312378\\
14.252893447876 14.8053932189941\\
14.2597417831421 14.8027753829956\\
14.2672748565674 14.7997331619263\\
14.2755603790283 14.7963581085205\\
14.2846755981445 14.7928276062012\\
14.2947015762329 14.7894258499146\\
14.3057298660278 14.7864694595337\\
14.3178615570068 14.7842407226563\\
14.3312063217163 14.781156539917\\
14.3458862304688 14.7773742675781\\
14.3620328903198 14.7738828659058\\
14.3797950744629 14.7717781066895\\
14.3993330001831 14.7722015380859\\
14.4208250045776 14.7758255004883\\
14.444465637207 14.7827854156494\\
14.4704713821411 14.7992000579834\\
14.4990768432617 14.8010311126709\\
14.5269069671631 14.8077249526978\\
14.5575199127197 14.8133459091187\\
14.591194152832 14.8202953338623\\
14.6282358169556 14.8221855163574\\
14.668981552124 14.8163824081421\\
14.7138023376465 14.8086853027344\\
14.7631044387817 14.8062963485718\\
14.8173370361328 14.8043746948242\\
14.8769931793213 14.8033285140991\\
14.9244384765625 14.8036556243896\\
14.971435546875 14.8023462295532\\
15.0163955688477 14.8014116287231\\
15.0584926605225 14.8017044067383\\
15.09974193573 14.8037843704224\\
15.1451168060303 14.8023767471313\\
15.1950283050537 14.7985095977783\\
15.2499313354492 14.798454284668\\
15.31032371521 14.7987003326416\\
15.3767566680908 14.7989416122437\\
15.4498319625854 14.7993688583374\\
15.5302152633667 14.7999668121338\\
15.6186370849609 14.8027420043945\\
15.7159004211426 14.7994613647461\\
15.8159008026123 14.8047113418579\\
15.9159002304077 14.8013105392456\\
16.0159015655518 14.8041286468506\\
16.0210952758789 14.8618545532227\\
16.0268077850342 14.7712955474854\\
16.0330924987793 14.8003377914429\\
16.0382099151611 14.8128252029419\\
16.0438404083252 14.8061847686768\\
16.0500316619873 14.8060998916626\\
16.0568428039551 14.8061246871948\\
16.0643348693848 14.8061370849609\\
16.0725765228271 14.8061580657959\\
16.0816440582275 14.806188583374\\
16.0916156768799 14.8062295913696\\
16.1025848388672 14.8062829971313\\
16.114652633667 14.8063507080078\\
16.1279258728027 14.8064346313477\\
16.1425266265869 14.8065366744995\\
16.1585865020752 14.806658744812\\
16.1762542724609 14.8068017959595\\
16.1956882476807 14.8069686889648\\
16.2170658111572 14.8071603775024\\
16.2405796051025 14.8073787689209\\
16.2664451599121 14.8076267242432\\
16.2948989868164 14.8079051971436\\
16.3261966705322 14.808217048645\\
16.3606243133545 14.8085632324219\\
16.398494720459 14.8089466094971\\
16.4401531219482 14.8093709945679\\
16.4859771728516 14.8098382949829\\
16.5363826751709 14.8103542327881\\
16.5918292999268 14.810920715332\\
16.6528205871582 14.8115453720093\\
16.7199096679688 14.8122320175171\\
16.7937088012695 14.8129873275757\\
16.8748874664307 14.8138189315796\\
16.9641857147217 14.8164825439453\\
17.0624122619629 14.813868522644\\
17.1624126434326 14.8169584274292\\
17.2624130249023 14.8240118026733\\
17.282413482666 14.823657989502\\
17.2906188964844 14.8352890014648\\
17.2996425628662 14.8353796005249\\
17.308614730835 14.8325834274292\\
17.3184852600098 14.8290567398071\\
17.3293399810791 14.8299474716187\\
17.3412818908691 14.8318319320679\\
17.3544178009033 14.8439884185791\\
17.3688678741455 14.8754425048828\\
17.3847618103027 14.900203704834\\
17.4022445678711 14.8964519500732\\
17.4214763641357 14.8773202896118\\
17.4426326751709 14.8638610839844\\
17.4659042358398 14.8489847183228\\
17.4915008544922 14.8338222503662\\
17.5196590423584 14.821475982666\\
17.5506324768066 14.8123626708984\\
17.5847034454346 14.807469367981\\
17.6221809387207 14.8068265914917\\
17.6634063720703 14.8099279403687\\
17.7087554931641 14.8187627792358\\
17.758638381958 14.8404922485352\\
17.8135089874268 14.8485536575317\\
17.8738670349121 14.842827796936\\
17.9402618408203 14.8422069549561\\
18.0132961273193 14.8383712768555\\
18.0585689544678 14.8328409194946\\
18.1083717346191 14.8343296051025\\
18.1509475708008 14.8309888839722\\
18.1891002655029 14.8310527801514\\
18.2310676574707 14.8325424194336\\
18.2772312164307 14.8346271514893\\
18.328010559082 14.8363342285156\\
18.3838691711426 14.8376054763794\\
18.4453125 14.8381900787354\\
18.5129013061523 14.8385066986084\\
18.5872478485107 14.838604927063\\
18.6690311431885 14.8386144638062\\
18.7589912414551 14.8385486602783\\
18.857946395874 14.8386554718018\\
18.9579467773438 14.8384847640991\\
19.0579471588135 14.8395795822144\\
19.1579456329346 14.8406105041504\\
19.2579460144043 14.841944694519\\
19.357946395874 14.842903137207\\
19.4579467773438 14.8440828323364\\
19.5579471588135 14.8451128005981\\
19.6579456329346 14.846266746521\\
19.7579460144043 14.8473405838013\\
19.857946395874 14.8484725952148\\
19.9579467773438 14.8495645523071\\
20.0579471588135 14.8506860733032\\
20.1579456329346 14.8517875671387\\
20.2579460144043 14.8529033660889\\
20.357946395874 14.8540096282959\\
20.4579467773438 14.8551225662231\\
20.5579471588135 14.8562459945679\\
20.6579456329346 14.8573503494263\\
20.7579460144043 14.8584671020508\\
20.857946395874 14.8595838546753\\
20.9579467773438 14.8607006072998\\
21.0579471588135 14.8618183135986\\
21.1579456329346 14.8630266189575\\
21.2579460144043 14.8643808364868\\
21.357946395874 14.8654346466064\\
21.4579467773438 14.8665714263916\\
21.5579471588135 14.8676843643188\\
21.6579456329346 14.8688049316406\\
21.7579460144043 14.8699235916138\\
21.857946395874 14.8710422515869\\
21.9579467773438 14.8721618652344\\
22.0579471588135 14.8732805252075\\
22.1579456329346 14.874400138855\\
22.2579460144043 14.8755197525024\\
22.357946395874 14.8766393661499\\
22.4579467773438 14.8777599334717\\
22.5579471588135 14.8788795471191\\
22.6579456329346 14.8799991607666\\
22.7579460144043 14.8809127807617\\
22.857946395874 14.8819284439087\\
22.9579467773438 14.8830671310425\\
23.0579471588135 14.8841619491577\\
23.1579456329346 14.8852710723877\\
23.2579460144043 14.8863725662231\\
23.357946395874 14.8874740600586\\
23.4579467773438 14.8885717391968\\
23.5579471588135 14.8896675109863\\
23.6579456329346 14.8907585144043\\
23.7579460144043 14.8918447494507\\
23.857946395874 14.8929300308228\\
23.9579467773438 14.8940086364746\\
24.0579471588135 14.8986959457397\\
24.1579456329346 14.8975744247437\\
24.2579460144043 14.897744178772\\
24.357946395874 14.8991279602051\\
24.4579467773438 14.900260925293\\
24.5579471588135 14.9015674591064\\
24.6579456329346 14.9091701507568\\
24.7421741485596 14.9124641418457\\
24.8095417022705 14.9085083007813\\
24.8836441040039 14.9052362442017\\
24.9651584625244 14.9058132171631\\
25.0548248291016 14.9230632781982\\
25.0960559844971 14.8717088699341\\
25.1414089202881 14.9016590118408\\
25.1613826751709 14.9160013198853\\
25.1780624389648 14.9176111221313\\
25.1964111328125 14.9090843200684\\
25.2165927886963 14.9091386795044\\
25.2387943267822 14.9092826843262\\
25.2632160186768 14.9094514846802\\
25.2900791168213 14.9096593856812\\
25.3196296691895 14.9099073410034\\
25.3521347045898 14.9101972579956\\
25.3878898620605 14.9105319976807\\
25.4272212982178 14.9109125137329\\
25.4704837799072 14.9113426208496\\
25.5180740356445 14.9118223190308\\
25.5704231262207 14.9123563766479\\
25.6280078887939 14.9129457473755\\
25.6913509368896 14.913595199585\\
25.7610282897949 14.9143095016479\\
25.8376731872559 14.9150953292847\\
25.9219818115234 14.9250659942627\\
26.014720916748 14.9083528518677\\
26.1147212982178 14.9180574417114\\
26.2147216796875 14.921932220459\\
26.3147220611572 14.927134513855\\
26.3647212982178 14.9778537750244\\
26.4022121429443 14.9465360641479\\
26.4049663543701 14.9284467697144\\
26.407995223999 14.935396194458\\
26.4113292694092 14.9388236999512\\
26.4149932861328 14.9347696304321\\
26.4190254211426 14.9316349029541\\
26.4234619140625 14.9242639541626\\
26.4283409118652 14.9146738052368\\
26.4337062835693 14.9070081710815\\
26.4396114349365 14.9011468887329\\
26.4461040496826 14.8947858810425\\
26.4532470703125 14.887113571167\\
26.4611053466797 14.8768863677979\\
26.469747543335 14.8629102706909\\
26.4792556762695 14.8461723327637\\
26.4897136688232 14.8287668228149\\
26.5012187957764 14.8169078826904\\
26.5138721466064 14.8114414215088\\
26.5277919769287 14.8120155334473\\
26.5431041717529 14.8180828094482\\
26.5599479675293 14.83216381073\\
26.5784759521484 14.8621435165405\\
26.59885597229 14.9094476699829\\
26.6212730407715 14.9460258483887\\
26.6459331512451 14.9301328659058\\
26.673059463501 14.9328136444092\\
26.7028980255127 14.9386129379272\\
26.7357215881348 14.9472522735596\\
26.7718257904053 14.9460687637329\\
26.8115406036377 14.9392356872559\\
26.8552284240723 14.935676574707\\
26.903284072876 14.9337797164917\\
26.9561443328857 14.9325113296509\\
27.0142917633057 14.9326648712158\\
27.0605869293213 14.9346504211426\\
27.1115131378174 14.933690071106\\
27.1556930541992 14.9336709976196\\
27.1975421905518 14.9352045059204\\
27.2435741424561 14.9364538192749\\
27.29421043396 14.9351358413696\\
27.3499088287354 14.932297706604\\
27.4111785888672 14.9325714111328\\
27.4785747528076 14.9331007003784\\
27.5527095794678 14.9335880279541\\
27.6342601776123 14.934289932251\\
27.7239646911621 14.9372901916504\\
27.8226375579834 14.9338979721069\\
27.9226379394531 14.9394845962524\\
28.0226383209229 14.9358282089233\\
28.1226387023926 14.9384756088257\\
28.1260833740234 15.0382537841797\\
28.1298713684082 14.8892765045166\\
28.1340389251709 14.9276351928711\\
28.1376724243164 15.0134801864624\\
28.1416683197021 14.8898229598999\\
28.1460647583008 14.9327945709229\\
28.1490650177002 14.9517526626587\\
28.1523666381836 14.9421472549438\\
28.1559963226318 14.9414167404175\\
28.1599903106689 14.9415578842163\\
28.1643829345703 14.9415702819824\\
28.1692161560059 14.9415721893311\\
28.1745319366455 14.9415788650513\\
28.1803779602051 14.9415903091431\\
28.1868114471436 14.9416065216064\\
28.1938858032227 14.94162940979\\
28.2016677856445 14.9416589736938\\
28.2102298736572 14.9416961669922\\
28.2196464538574 14.9417448043823\\
28.2300052642822 14.9418029785156\\
28.2413997650146 14.9418745040894\\
28.2539348602295 14.9419603347778\\
28.2677211761475 14.9420623779297\\
28.2828884124756 14.9421815872192\\
28.2995700836182 14.9423208236694\\
28.3179225921631 14.9424800872803\\
28.3381080627441 14.942663192749\\
28.3603134155273 14.94287109375\\
28.3847389221191 14.9431056976318\\
28.4116058349609 14.9433698654175\\
28.4411602020264 14.9436635971069\\
28.4736709594727 14.9439907073975\\
28.5094318389893 14.9443531036377\\
28.5487689971924 14.9447536468506\\
28.592041015625 14.9451961517334\\
28.6396389007568 14.9456825256348\\
28.6919975280762 14.9462194442749\\
28.7495918273926 14.9468097686768\\
28.8129444122314 14.9474592208862\\
28.8826332092285 14.9481735229492\\
28.9592895507813 14.9489593505859\\
29.0436134338379 14.9498233795166\\
29.1363677978516 14.9525756835938\\
29.2363681793213 14.9498739242554\\
29.336368560791 14.9530611038208\\
29.4363689422607 14.9611330032349\\
29.4563694000244 14.9623851776123\\
29.4642658233643 14.9859294891357\\
29.471960067749 14.9891338348389\\
29.4804267883301 14.9866094589233\\
29.4885120391846 14.9837093353271\\
29.497407913208 14.9834108352661\\
29.5071926116943 14.9845199584961\\
29.5179557800293 15.0109300613403\\
29.5297946929932 15.0415058135986\\
29.542818069458 15.0461511611938\\
29.5571441650391 15.0244150161743\\
29.5729026794434 15.0049772262573\\
29.5902366638184 14.9840297698975\\
29.6093044281006 14.9626836776733\\
29.630277633667 14.941764831543\\
29.6533489227295 14.9198904037476\\
29.6787281036377 14.8907403945923\\
29.6999015808105 14.876407623291\\
29.7231941223145 14.8805818557739\\
29.7488136291504 14.8946809768677\\
29.7769947052002 14.916898727417\\
29.8079967498779 14.9551162719727\\
29.8420963287354 14.9629096984863\\
29.8796062469482 14.964337348938\\
29.9208679199219 14.9686307907104\\
29.9662551879883 14.9501323699951\\
30.0161819458008 14.9829816818237\\
30.0711002349854 14.9656972885132\\
30.1315116882324 14.9662580490112\\
30.1804618835449 14.9625663757324\\
30.2286376953125 14.9588508605957\\
30.2746448516846 14.9575338363647\\
30.3156337738037 14.9569511413574\\
30.3508148193359 14.9623403549194\\
30.3895168304443 14.9652423858643\\
30.4320869445801 14.9663457870483\\
30.4789142608643 14.9659080505371\\
30.530424118042 14.9667615890503\\
30.587085723877 14.9676656723022\\
30.6491241455078 14.9685668945313\\
30.7173671722412 14.970085144043\\
30.7924346923828 14.9727687835693\\
30.8750076293945 14.9768199920654\\
30.9658393859863 14.9805240631104\\
31.0657539367676 14.9784851074219\\
31.1657543182373 14.9775447845459\\
31.265754699707 14.9767751693726\\
31.3657531738281 14.9764099121094\\
31.4657535552979 14.9764413833618\\
31.5657539367676 14.9777860641479\\
31.6657543182373 14.9787273406982\\
31.765754699707 14.9801406860352\\
31.8657531738281 14.9810800552368\\
31.9657535552979 14.9823312759399\\
32.0657539367676 14.9833765029907\\
32.1657524108887 14.9845590591431\\
32.265754699707 14.9856510162354\\
32.3657531738281 14.9868030548096\\
32.4657554626465 14.9879159927368\\
32.5657539367676 14.9890546798706\\
32.6657524108887 14.9901762008667\\
32.765754699707 14.9913234710693\\
32.8657531738281 14.9924306869507\\
32.9657554626465 14.9935617446899\\
33.0657539367676 14.9946889877319\\
33.1657524108887 14.9958181381226\\
33.265754699707 14.9969453811646\\
33.3657531738281 14.9980745315552\\
33.4657554626465 14.9992027282715\\
33.5657539367676 15.0003318786621\\
33.6657524108887 15.0017776489258\\
33.765754699707 15.0028581619263\\
33.8657531738281 15.0040016174316\\
33.9657554626465 15.0051259994507\\
34.0657539367676 15.0062570571899\\
34.1657524108887 15.0073852539063\\
34.265754699707 15.0085144042969\\
34.3657531738281 15.0096426010132\\
34.4657554626465 15.0107717514038\\
34.5657539367676 15.0118999481201\\
34.6657524108887 15.0130290985107\\
34.765754699707 15.0141572952271\\
34.8657531738281 15.0152864456177\\
34.9657554626465 15.016414642334\\
35.0657539367676 15.0175437927246\\
35.1657524108887 15.0185585021973\\
35.265754699707 15.0209302902222\\
35.3657531738281 15.021050453186\\
35.4657554626465 15.0217847824097\\
35.5657539367676 15.0230054855347\\
35.6657524108887 15.0240907669067\\
35.765754699707 15.0252275466919\\
35.8657531738281 15.0263442993164\\
35.9657554626465 15.0274677276611\\
36.0657539367676 15.0285882949829\\
36.1657524108887 15.029709815979\\
36.265754699707 15.0308656692505\\
36.3657531738281 15.0320091247559\\
36.4657554626465 15.0332450866699\\
36.5657539367676 15.0343761444092\\
36.6657524108887 15.0355224609375\\
36.765754699707 15.0367069244385\\
36.8657531738281 15.0379161834717\\
36.9657554626465 15.0391788482666\\
37.0657539367676 15.0455160140991\\
37.1539344787598 15.0494165420532\\
37.2244758605957 15.0481986999512\\
37.302074432373 15.0430421829224\\
37.3874282836914 15.0436573028564\\
37.4813194274902 15.0602931976318\\
37.5215835571289 15.0084915161133\\
37.5658721923828 15.0396490097046\\
37.585823059082 15.0534753799438\\
37.6024284362793 15.0553283691406\\
37.6206932067871 15.0469665527344\\
37.6407852172852 15.0470056533813\\
37.6628875732422 15.0471563339233\\
37.687198638916 15.0473251342773\\
37.713939666748 15.047532081604\\
37.7433547973633 15.047779083252\\
37.7757110595703 15.0480680465698\\
37.8113059997559 15.0484008789063\\
37.8504600524902 15.0487813949585\\
37.8935279846191 15.0492095947266\\
37.9409027099609 15.0496892929077\\
37.9930152893066 15.0502214431763\\
38.0503387451172 15.0508098602295\\
38.1133918762207 15.0514583587646\\
38.1827545166016 15.0521717071533\\
38.2590522766113 15.0529556274414\\
38.3429794311523 15.0630207061768\\
38.4352989196777 15.0461225509644\\
38.5352973937988 15.055908203125\\
38.6352996826172 15.0596828460693\\
38.7352981567383 15.0651807785034\\
38.7853012084961 15.1167736053467\\
38.822696685791 15.0908536911011\\
38.8252944946289 15.0733547210693\\
38.8281555175781 15.0774526596069\\
38.8313026428223 15.0795259475708\\
38.8347625732422 15.0770673751831\\
38.838565826416 15.0710439682007\\
38.8427543640137 15.0625610351563\\
38.8473587036133 15.0540237426758\\
38.852424621582 15.0471343994141\\
38.8579978942871 15.0409336090088\\
38.864128112793 15.0341482162476\\
38.8708724975586 15.0257711410522\\
38.878288269043 15.0134868621826\\
38.8864479064941 14.9969787597656\\
38.8954238891602 14.9751758575439\\
38.9052963256836 14.9578619003296\\
38.9161567687988 14.9400787353516\\
38.9281005859375 14.9417676925659\\
38.9412422180176 14.9435205459595\\
38.955696105957 14.9472627639771\\
38.9715957641602 14.955135345459\\
38.989086151123 14.9737100601196\\
39.0083236694336 15.0311689376831\\
39.029483795166 15.08482837677\\
39.0527648925781 15.0927820205688\\
39.078369140625 15.075701713562\\
39.1065368652344 15.0775184631348\\
39.137523651123 15.0812139511108\\
39.1716041564941 15.0810327529907\\
39.2090950012207 15.0780782699585\\
39.2477378845215 15.0748090744019\\
39.2902450561523 15.0725698471069\\
39.3370056152344 15.0710783004761\\
39.3884391784668 15.0702209472656\\
39.4450187683105 15.071102142334\\
39.4936180114746 15.0714464187622\\
39.5406875610352 15.071328163147\\
39.5853233337402 15.0714302062988\\
39.6272773742676 15.0737924575806\\
39.6734275817871 15.0737476348877\\
39.7241897583008 15.0724515914917\\
39.780029296875 15.0702171325684\\
39.8414497375488 15.0705289840698\\
39.9090156555176 15.0710573196411\\
39.9833374023438 15.0715703964233\\
40.0650939941406 15.0722875595093\\
40.1550216674805 15.0752973556519\\
40.2539443969727 15.0719032287598\\
40.353946685791 15.0775022506714\\
40.4539451599121 15.0738430023193\\
40.5539436340332 15.0763912200928\\
40.5569686889648 15.185115814209\\
40.5602951049805 15.0314807891846\\
40.5639533996582 15.0430164337158\\
40.5665168762207 15.1014156341553\\
40.5693359375 15.0995712280273\\
40.572437286377 15.0804147720337\\
40.5758476257324 15.078821182251\\
40.5796012878418 15.0795164108276\\
40.5837287902832 15.0796031951904\\
40.5882682800293 15.0795831680298\\
40.59326171875 15.0795783996582\\
40.5987548828125 15.0795812606812\\
40.6048011779785 15.0795869827271\\
40.6114463806152 15.0795993804932\\
40.6187591552734 15.0796175003052\\
40.6268005371094 15.0796432495117\\
40.6356506347656 15.0796785354614\\
40.6453819274902 15.0797243118286\\
40.6560897827148 15.0797815322876\\
40.6678657531738 15.079852104187\\
40.6808204650879 15.0799379348755\\
40.695068359375 15.0800409317017\\
40.7107429504395 15.0801630020142\\
40.7279853820801 15.0803050994873\\
40.7469482421875 15.0804691314697\\
40.7678108215332 15.0806579589844\\
40.7907600402832 15.0808734893799\\
40.8160057067871 15.0811157226563\\
40.8437728881836 15.0813884735107\\
40.8743171691895 15.0816926956177\\
40.9079170227051 15.08203125\\
40.9448776245117 15.0824069976807\\
40.9855308532715 15.0828218460083\\
41.0302505493164 15.0832786560059\\
41.0794448852539 15.0837831497192\\
41.1335563659668 15.0843372344971\\
41.1930809020996 15.0849475860596\\
41.2585563659668 15.0856199264526\\
41.3305778503418 15.0863580703735\\
41.4098052978516 15.0871706008911\\
41.4969520568848 15.0900230407715\\
41.5928153991699 15.0869903564453\\
41.692813873291 15.0902347564697\\
41.7928161621094 15.0984172821045\\
41.812816619873 15.1002035140991\\
41.8206024169922 15.1249942779541\\
41.8281707763672 15.1285877227783\\
41.8364944458008 15.1259145736694\\
41.8444328308105 15.1228017807007\\
41.8531684875488 15.1222820281982\\
41.8627777099609 15.122917175293\\
41.8733444213867 15.1460180282593\\
41.8849716186523 15.1806583404541\\
41.8977584838867 15.1841554641724\\
41.9118270874023 15.1632604598999\\
41.9272994995117 15.1431465148926\\
41.9443206787109 15.121753692627\\
41.9630432128906 15.100079536438\\
41.9836387634277 15.0787839889526\\
42.0062942504883 15.0537443161011\\
42.0312156677246 15.0245580673218\\
42.0477752685547 15.0123682022095\\
42.0659866333008 15.0136280059814\\
42.0860252380371 15.024938583374\\
42.1080627441406 15.0449419021606\\
42.1323051452637 15.0546989440918\\
42.1589736938477 15.0910930633545\\
42.1883087158203 15.0992889404297\\
42.2205772399902 15.1012601852417\\
42.2560729980469 15.0966300964355\\
42.2951164245605 15.1001777648926\\
42.3380661010742 15.1076736450195\\
42.3853073120117 15.1066675186157\\
42.4372749328613 15.1033849716187\\
42.494441986084 15.1053266525269\\
42.5416946411133 15.0951375961304\\
42.5936737060547 15.096531867981\\
42.6379699707031 15.0943431854248\\
42.6777877807617 15.0961751937866\\
42.711311340332 15.1033344268799\\
42.7481842041016 15.1048240661621\\
42.788745880127 15.1047687530518\\
42.833366394043 15.1033983230591\\
42.8824462890625 15.1040334701538\\
42.9364318847656 15.1046476364136\\
42.9958190917969 15.1051597595215\\
43.0611457824707 15.1059646606445\\
43.1330032348633 15.1072416305542\\
43.2120475769043 15.1095724105835\\
43.2989959716797 15.113748550415\\
43.3946380615234 15.1181840896606\\
43.4946365356445 15.1180944442749\\
43.5946388244629 15.1157741546631\\
43.694637298584 15.1150541305542\\
43.7946357727051 15.1145792007446\\
43.8946380615234 15.1145057678223\\
43.9946365356445 15.1158685684204\\
44.0946388244629 15.116792678833\\
44.194637298584 15.1182155609131\\
44.2946357727051 15.1191492080688\\
44.3946380615234 15.1204042434692\\
44.4946365356445 15.1214475631714\\
44.5946388244629 15.1226320266724\\
44.694637298584 15.1237230300903\\
44.7946357727051 15.1248760223389\\
44.8946380615234 15.1259880065918\\
44.9946365356445 15.1271266937256\\
45.0946388244629 15.1282634735107\\
45.194637298584 15.1293783187866\\
45.2946357727051 15.13050365448\\
45.3946380615234 15.1316337585449\\
45.4946365356445 15.1327610015869\\
45.5946388244629 15.1338901519775\\
45.694637298584 15.1350183486938\\
45.7946357727051 15.1361474990845\\
45.8946380615234 15.1372756958008\\
45.9946365356445 15.1384038925171\\
46.0946388244629 15.1398344039917\\
46.194637298584 15.1409349441528\\
46.2946357727051 15.1420736312866\\
46.3946380615234 15.14319896698\\
46.4946365356445 15.1443290710449\\
46.5946388244629 15.1454582214355\\
46.694637298584 15.1465864181519\\
46.7946357727051 15.1477155685425\\
46.8946380615234 15.1488437652588\\
46.9946365356445 15.1499729156494\\
47.0946388244629 15.15110206604\\
47.194637298584 15.1522302627563\\
47.2946357727051 15.153359413147\\
47.3946380615234 15.1544876098633\\
47.4946365356445 15.1556167602539\\
47.5946388244629 15.1566514968872\\
47.694637298584 15.1589956283569\\
47.7946357727051 15.1591262817383\\
47.8946380615234 15.159857749939\\
47.9946365356445 15.1610794067383\\
48.0946388244629 15.1621646881104\\
48.194637298584 15.1633014678955\\
48.2946357727051 15.16441822052\\
48.3946380615234 15.1655426025391\\
48.4946365356445 15.1666631698608\\
48.5946388244629 15.1677846908569\\
48.694637298584 15.1689376831055\\
48.7946357727051 15.1700830459595\\
48.8946380615234 15.1713132858276\\
48.9946365356445 15.1724462509155\\
49.0946388244629 15.1735925674438\\
49.194637298584 15.1747751235962\\
49.2946357727051 15.1759824752808\\
49.3946380615234 15.1772394180298\\
49.4946365356445 15.183144569397\\
49.5842514038086 15.1872692108154\\
49.6559371948242 15.1869163513184\\
49.7347946166992 15.1811637878418\\
49.8215370178223 15.1817903518677\\
49.9169502258301 15.1982421875\\
49.9568328857422 15.146143913269\\
50.0007057189941 15.1778249740601\\
50.0206336975098 15.1914901733398\\
50.0371971130371 15.1933851242065\\
50.0554161071777 15.185115814209\\
50.0754585266113 15.1851367950439\\
50.0975074768066 15.1852941513062\\
50.1217575073242 15.1854629516602\\
50.1484336853027 15.1856698989868\\
50.1777763366699 15.1859159469604\\
50.2100563049316 15.186203956604\\
50.2455635070801 15.1865367889404\\
50.2846183776855 15.186915397644\\
50.327579498291 15.1873426437378\\
50.3748397827148 15.1878204345703\\
50.4268226623535 15.1883516311646\\
50.4840087890625 15.1889390945435\\
50.5469093322754 15.18958568573\\
50.6161003112793 15.19029712677\\
50.692211151123 15.1910791397095\\
50.7759323120117 15.2011957168579\\
50.8680267333984 15.1841945648193\\
50.9680252075195 15.1940231323242\\
51.0680236816406 15.1977882385254\\
51.168025970459 15.2032928466797\\
51.2180252075195 15.2549610137939\\
51.2553977966309 15.2296571731567\\
51.2579727172852 15.2127141952515\\
51.2608032226563 15.2155904769897\\
51.263916015625 15.2179937362671\\
51.2673416137695 15.2154750823975\\
51.2711067199707 15.209321975708\\
51.2752532958984 15.2008104324341\\
51.2798080444336 15.1923246383667\\
51.2848243713379 15.1853818893433\\
51.290340423584 15.1792573928833\\
51.2964057922363 15.1724529266357\\
51.3030815124512 15.1639823913574\\
51.3104209899902 15.1515502929688\\
51.3184967041016 15.1346426010132\\
51.3273811340332 15.1127128601074\\
51.3371505737305 15.0955333709717\\
51.347900390625 15.0789575576782\\
51.3597221374512 15.0795888900757\\
51.372730255127 15.0811376571655\\
51.387035369873 15.0847425460815\\
51.4027709960938 15.0918245315552\\
51.4200820922852 15.1094789505005\\
51.4391212463379 15.1689157485962\\
51.4600677490234 15.2210750579834\\
51.4831085205078 15.2299270629883\\
51.5084533691406 15.2129240036011\\
51.5363311767578 15.2143421173096\\
51.5669975280762 15.2185926437378\\
51.6007308959961 15.2193460464478\\
51.6378364562988 15.2166843414307\\
51.6764678955078 15.2132549285889\\
51.7189598083496 15.2109603881836\\
51.7657051086426 15.2094097137451\\
51.8171195983887 15.2083053588867\\
51.8736801147461 15.2085933685303\\
51.9213600158691 15.2102651596069\\
51.9687728881836 15.2096061706543\\
52.0141105651855 15.2094478607178\\
52.056568145752 15.2113084793091\\
52.103271484375 15.2122058868408\\
52.1546440124512 15.2109136581421\\
52.2111549377441 15.208348274231\\
52.2733154296875 15.208643913269\\
52.3416938781738 15.2091865539551\\
52.4169082641602 15.2096929550171\\
52.4996490478516 15.2104158401489\\
52.5906600952148 15.2134113311768\\
52.6906585693359 15.2100734710693\\
52.790657043457 15.2156686782837\\
52.8906593322754 15.2120027542114\\
52.9906578063965 15.2145347595215\\
52.9936103820801 15.3199310302734\\
52.9968528747559 15.1782579421997\\
53.0004234313965 15.1784725189209\\
53.0029792785645 15.2348871231079\\
53.0057907104492 15.239634513855\\
53.0088844299316 15.2196378707886\\
53.0122871398926 15.2165288925171\\
53.0160293579102 15.2175464630127\\
53.0201454162598 15.2177867889404\\
53.0246734619141 15.2177562713623\\
53.0296516418457 15.2177438735962\\
53.0351333618164 15.2177457809448\\
53.0411605834961 15.2177515029907\\
53.0477867126465 15.2177629470825\\
53.0550804138184 15.2177810668945\\
53.063102722168 15.2178068161011\\
53.0719261169434 15.2178411483765\\
53.0816307067871 15.2178859710693\\
53.0923080444336 15.217942237854\\
53.1040496826172 15.2180118560791\\
53.1169700622559 15.2180976867676\\
53.1311798095703 15.2181997299194\\
53.1468124389648 15.2183208465576\\
53.164005279541 15.2184619903564\\
53.1829223632813 15.2186260223389\\
53.203727722168 15.2188138961792\\
53.2266120910645 15.2190275192261\\
53.2517852783203 15.2192697525024\\
53.2794799804688 15.2195415496826\\
53.3099403381348 15.2198448181152\\
53.343448638916 15.2201824188232\\
53.3803062438965 15.2205562591553\\
53.4208488464355 15.2209701538086\\
53.465446472168 15.2214260101318\\
53.5145034790039 15.2219285964966\\
53.5684700012207 15.2224817276001\\
53.6278305053711 15.223090171814\\
53.6931266784668 15.2237606048584\\
53.7649536132813 15.2244968414307\\
53.8439598083496 15.2253074645996\\
53.9308700561523 15.2281656265259\\
54.0264701843262 15.2251148223877\\
54.1264686584473 15.2283620834351\\
54.2264709472656 15.2365379333496\\
54.2464714050293 15.2386198043823\\
54.2542266845703 15.2631931304932\\
54.2617721557617 15.2669763565063\\
54.2700691223145 15.2642192840576\\
54.2779922485352 15.2610139846802\\
54.2867050170898 15.260425567627\\
54.2962913513184 15.26096534729\\
54.3068389892578 15.2839097976685\\
54.3184356689453 15.318567276001\\
54.3311958312988 15.322304725647\\
54.3452301025391 15.3014183044434\\
54.360668182373 15.281307220459\\
54.3776512145996 15.2599649429321\\
54.3963317871094 15.2383289337158\\
54.4168815612793 15.21706199646\\
54.4394836425781 15.1920490264893\\
54.4643478393555 15.1628932952881\\
54.4809799194336 15.1504936218262\\
54.4992752075195 15.1516256332397\\
54.519401550293 15.1630668640137\\
54.5415382385254 15.183177947998\\
54.5658912658691 15.193018913269\\
54.5926780700684 15.2295703887939\\
54.622142791748 15.2376203536987\\
54.6545562744141 15.2394208908081\\
54.6902084350586 15.2344551086426\\
54.7294273376465 15.2390375137329\\
54.7725677490234 15.2450160980225\\
54.8200225830078 15.2454128265381\\
54.8722229003906 15.2413148880005\\
54.9296417236328 15.2428646087646\\
54.9768371582031 15.2344007492065\\
55.028751373291 15.2336330413818\\
55.0732498168945 15.2323694229126\\
55.112865447998 15.2346811294556\\
55.1461601257324 15.241961479187\\
55.1827850341797 15.2431745529175\\
55.223072052002 15.2429599761963\\
55.2673873901367 15.2415571212769\\
55.3161315917969 15.2421951293945\\
55.3697547912598 15.2427835464478\\
55.4287376403809 15.243275642395\\
55.4936180114746 15.244047164917\\
55.5649871826172 15.2452487945557\\
55.643497467041 15.2474203109741\\
55.7298545837402 15.2514009475708\\
55.8248481750488 15.2560205459595\\
55.9248466491699 15.25670337677\\
56.024845123291 15.2540912628174\\
56.1248474121094 15.2533588409424\\
56.2248458862305 15.2528448104858\\
56.3248481750488 15.2526254653931\\
56.4248466491699 15.2539758682251\\
56.524845123291 15.2548961639404\\
56.6248474121094 15.2563190460205\\
56.7248458862305 15.2572526931763\\
56.8248481750488 15.2585077285767\\
56.9248466491699 15.2595510482788\\
57.024845123291 15.2607355117798\\
57.1248474121094 15.2618265151978\\
57.2248458862305 15.2629795074463\\
57.3248481750488 15.2640914916992\\
57.4248466491699 15.265230178833\\
57.524845123291 15.2663516998291\\
57.6248474121094 15.2674999237061\\
57.7248458862305 15.2686061859131\\
57.8248481750488 15.2697381973267\\
57.9248466491699 15.2708644866943\\
58.024845123291 15.271993637085\\
58.1248474121094 15.2731218338013\\
58.2248458862305 15.2742509841919\\
58.3248481750488 15.2753791809082\\
58.4248466491699 15.2765073776245\\
58.524845123291 15.2778968811035\\
58.6248474121094 15.2790489196777\\
58.7248458862305 15.2801733016968\\
58.8248481750488 15.2813034057617\\
58.9248466491699 15.2824325561523\\
59.024845123291 15.2835607528687\\
59.1248474121094 15.2846899032593\\
59.2248458862305 15.2858190536499\\
59.3248481750488 15.2869472503662\\
59.4248466491699 15.2880764007568\\
59.524845123291 15.2892045974731\\
59.6248474121094 15.2903337478638\\
59.7248458862305 15.2914628982544\\
59.8248481750488 15.2925910949707\\
59.9124221801758 15.2937297821045\\
60 15.29469871521\\
60.0963325500488 15.2968502044678\\
60.1963348388672 15.2969198226929\\
60.2963333129883 15.2976398468018\\
60.3963356018066 15.2988662719727\\
60.4963340759277 15.2999505996704\\
60.5963325500488 15.3010873794556\\
60.6963348388672 15.3022050857544\\
60.7963333129883 15.3033294677734\\
60.8963356018066 15.3044500350952\\
60.9963340759277 15.3055725097656\\
61.0963325500488 15.3067083358765\\
61.1963348388672 15.307861328125\\
61.2963333129883 15.3090677261353\\
61.3963356018066 15.3102102279663\\
61.4963340759277 15.3113584518433\\
61.5963325500488 15.3125314712524\\
61.6963348388672 15.3137254714966\\
61.7963333129883 15.314959526062\\
61.8963356018066 15.3182182312012\\
61.9940414428711 15.3241605758667\\
62.0722236633301 15.3283042907715\\
62.1347732543945 15.3189706802368\\
62.2035751342773 15.3367176055908\\
62.2478866577148 15.2942008972168\\
62.296630859375 15.3091068267822\\
62.3256530761719 15.3302955627441\\
62.3496742248535 15.4706392288208\\
62.376091003418 15.1986894607544\\
62.4051551818848 15.3181762695313\\
62.4213905334473 15.3258514404297\\
62.4340591430664 15.3262166976929\\
62.4479942321777 15.3228311538696\\
62.463321685791 15.3228530883789\\
62.4801826477051 15.3229732513428\\
62.4987297058105 15.3230962753296\\
62.5191307067871 15.3232421875\\
62.5415687561035 15.3234148025513\\
62.5662574768066 15.3236169815063\\
62.5934104919434 15.3238515853882\\
62.6232795715332 15.324122428894\\
62.656135559082 15.3244323730469\\
62.6922760009766 15.3247842788696\\
62.7320327758789 15.3251800537109\\
62.775764465332 15.3256235122681\\
62.8238677978516 15.3261165618896\\
62.8767852783203 15.32666015625\\
62.9349899291992 15.327260017395\\
62.9990196228027 15.327919960022\\
63.069450378418 15.3286447525024\\
63.1469230651855 15.3294410705566\\
63.2321434020996 15.3397207260132\\
63.3258857727051 15.3224306106567\\
63.4258842468262 15.3324546813965\\
63.5258865356445 15.3379039764404\\
63.6258850097656 15.3384323120117\\
63.6758842468262 15.3817491531372\\
63.7138481140137 15.3314323425293\\
63.7273941040039 15.2996273040771\\
63.7422943115234 15.3169603347778\\
63.7586822509766 15.3185386657715\\
63.7767105102539 15.2969207763672\\
63.7965431213379 15.3017091751099\\
63.8183555603027 15.3182992935181\\
63.8423538208008 15.3219213485718\\
63.8687477111816 15.318717956543\\
63.8977813720703 15.318018913269\\
63.9297218322754 15.3149900436401\\
63.9648513793945 15.3104887008667\\
64.0015869140625 15.3111572265625\\
64.022705078125 15.3167476654053\\
64.0459365844727 15.3334264755249\\
64.0714874267578 15.3389472961426\\
64.0996017456055 15.3463325500488\\
64.1305236816406 15.3547563552856\\
64.1645355224609 15.3535299301147\\
64.2019424438477 15.3490705490112\\
64.2431030273438 15.3479852676392\\
64.2883682250977 15.347713470459\\
64.3381729125977 15.347487449646\\
64.3929443359375 15.3475008010864\\
64.4477233886719 15.3484487533569\\
64.4993057250977 15.3487987518311\\
64.5481796264648 15.3489685058594\\
64.5940780639648 15.3492460250854\\
64.6370315551758 15.3512649536133\\
64.6842803955078 15.3517446517944\\
64.7362518310547 15.3502569198608\\
64.7934188842773 15.3483266830444\\
64.8563079833984 15.3486547470093\\
64.9254837036133 15.3492136001587\\
65.0015716552734 15.34974193573\\
65.0852737426758 15.3504848480225\\
65.1773452758789 15.3533477783203\\
65.27734375 15.350284576416\\
65.3773422241211 15.3556432723999\\
65.4773483276367 15.3522300720215\\
65.5773468017578 15.3544721603394\\
65.5794677734375 15.4745063781738\\
65.5817947387695 15.3445196151733\\
65.5843658447266 15.3259038925171\\
65.5871810913086 15.4183311462402\\
65.5902862548828 15.3203821182251\\
65.5937042236328 15.3265438079834\\
65.5974578857422 15.3662252426147\\
65.6015930175781 15.3682994842529\\
65.6061325073242 15.3584785461426\\
65.6111373901367 15.3575429916382\\
65.6166305541992 15.3578815460205\\
65.6226806640625 15.3579235076904\\
65.6293334960938 15.3579177856445\\
65.6366577148438 15.3579244613647\\
65.6447067260742 15.3579397201538\\
65.653564453125 15.3579635620117\\
65.6633071899414 15.3579988479614\\
65.6740188598633 15.3580465316772\\
65.6858062744141 15.3581075668335\\
65.6987762451172 15.3581857681274\\
65.7130355834961 15.3582801818848\\
65.7287292480469 15.3583946228027\\
65.7459869384766 15.35853099823\\
65.7649688720703 15.3586902618408\\
65.7858505249023 15.3588752746582\\
65.8088226318359 15.3590860366821\\
65.8340911865234 15.3593263626099\\
65.8618850708008 15.3595972061157\\
65.892463684082 15.3599004745483\\
65.9260940551758 15.3602380752563\\
65.9630889892578 15.3606128692627\\
66.0037841796875 15.3610277175903\\
66.0485458374023 15.3614854812622\\
66.097785949707 15.3619899749756\\
66.1519470214844 15.3625450134277\\
66.2115325927734 15.3631563186646\\
66.2770690917969 15.3638277053833\\
66.3491592407227 15.3645677566528\\
66.428466796875 15.3653812408447\\
66.5156936645508 15.3681468963623\\
66.6116485595703 15.3652858734131\\
66.7116470336914 15.3684387207031\\
66.811653137207 15.3778162002563\\
66.8316497802734 15.3898620605469\\
66.8391723632813 15.4199028015137\\
66.8464965820313 15.4246263504028\\
66.8545455932617 15.4207038879395\\
66.8622512817383 15.4169988632202\\
66.870735168457 15.4322299957275\\
66.8800659179688 15.4756212234497\\
66.8903274536133 15.4924879074097\\
66.9016189575195 15.4779796600342\\
66.9140319824219 15.4536113739014\\
66.9276962280273 15.4326419830322\\
66.9427185058594 15.4094114303589\\
66.9592437744141 15.384425163269\\
66.977424621582 15.3582563400269\\
66.997428894043 15.3190793991089\\
67.0194244384766 15.2925615310669\\
67.0415725708008 15.2754230499268\\
67.0659408569336 15.2683591842651\\
67.0927429199219 15.2687253952026\\
67.1222229003906 15.2986621856689\\
67.154655456543 15.345272064209\\
67.1903305053711 15.4217185974121\\
67.1995849609375 15.4155607223511\\
67.2097702026367 15.3985557556152\\
67.2209701538086 15.3900461196899\\
67.233283996582 15.3866586685181\\
67.2468414306641 15.385308265686\\
67.2617416381836 15.384801864624\\
67.2781448364258 15.3848314285278\\
67.2961807250977 15.385612487793\\
67.3160247802734 15.3863735198975\\
67.3378448486328 15.3816394805908\\
67.3618545532227 15.3846111297607\\
67.3882598876953 15.3821840286255\\
67.4173126220703 15.3834714889526\\
67.4492645263672 15.3825836181641\\
67.4844131469727 15.3801183700562\\
67.523078918457 15.3782930374146\\
67.5656127929688 15.3785448074341\\
67.6123962402344 15.3773174285889\\
67.6579818725586 15.3751306533813\\
67.6993789672852 15.3735256195068\\
67.7355346679688 15.3752956390381\\
67.7753067016602 15.3782510757446\\
67.8190612792969 15.3803510665894\\
67.8671875 15.381007194519\\
67.9201278686523 15.3830595016479\\
67.9783630371094 15.3832712173462\\
68.0421829223633 15.3856830596924\\
68.1070251464844 15.3884344100952\\
68.178352355957 15.3918466567993\\
68.2568130493164 15.3950309753418\\
68.3431243896484 15.3926219940186\\
68.4380569458008 15.3920574188232\\
68.5380630493164 15.3915920257568\\
68.6380615234375 15.3911762237549\\
68.7380599975586 15.3909158706665\\
68.8380584716797 15.3919887542725\\
68.9380569458008 15.3929557800293\\
69.0380630493164 15.3943405151367\\
69.1380615234375 15.3952960968018\\
69.2380599975586 15.3965358734131\\
69.3380584716797 15.3975887298584\\
69.4380569458008 15.3987655639648\\
69.5380630493164 15.3998613357544\\
69.6380615234375 15.40101146698\\
69.7380599975586 15.4021253585815\\
69.8380584716797 15.403263092041\\
69.9380569458008 15.4044008255005\\
70.0380630493164 15.4055147171021\\
70.1380615234375 15.4066410064697\\
70.2380599975586 15.4077701568604\\
70.3380584716797 15.4088983535767\\
70.4380569458008 15.4100275039673\\
70.5380630493164 15.4111557006836\\
70.6380615234375 15.4122838973999\\
70.7380599975586 15.4134120941162\\
70.8380584716797 15.4145412445068\\
70.9380569458008 15.4158201217651\\
71.0380630493164 15.4171123504639\\
71.1380615234375 15.4181985855103\\
71.2380599975586 15.419340133667\\
71.3380584716797 15.420464515686\\
71.4380569458008 15.421594619751\\
71.5380630493164 15.4227237701416\\
71.6380615234375 15.4238519668579\\
71.7380599975586 15.4249811172485\\
71.8380584716797 15.4261102676392\\
71.9380569458008 15.4272384643555\\
72.0380630493164 15.4283676147461\\
72.1380615234375 15.4294958114624\\
72.2380599975586 15.430624961853\\
72.3380584716797 15.4317531585693\\
72.4380569458008 15.4328804016113\\
72.5380630493164 15.4351453781128\\
72.6380615234375 15.435248374939\\
72.7380599975586 15.4360027313232\\
72.8380584716797 15.4372186660767\\
72.9380569458008 15.4383068084717\\
73.0380630493164 15.4394426345825\\
73.1380615234375 15.4405603408813\\
73.2380599975586 15.4416847229004\\
73.3380584716797 15.4428062438965\\
73.4380569458008 15.4439277648926\\
73.5380630493164 15.4450635910034\\
73.6380615234375 15.446216583252\\
73.7380599975586 15.4474229812622\\
73.8380584716797 15.4485664367676\\
73.9380569458008 15.4497146606445\\
74.0380630493164 15.4508867263794\\
74.1380615234375 15.4520797729492\\
74.2380599975586 15.453311920166\\
74.3380584716797 15.4563684463501\\
74.4363632202148 15.4624719619751\\
74.5150299072266 15.4665899276733\\
74.5779571533203 15.4576950073242\\
74.6471862792969 15.4580507278442\\
74.7233276367188 15.4585866928101\\
74.8070907592773 15.4592208862305\\
74.8992309570313 15.4758605957031\\
74.945930480957 15.4301147460938\\
74.9729614257813 15.4538860321045\\
74.9864807128906 15.469762802124\\
75 15.4706859588623\\
};
\addlegendentry{2"};

\addplot [
color=black,
dash pattern=on 1pt off 3pt on 3pt off 3pt
]
table[row sep=crcr]{
0 14.7006950378418\\
9.99999971718069e-10 14.7006950378418\\
2.49999998480632e-09 14.7006950378418\\
4.74999994892755e-09 14.7006950378418\\
8.1250002281763e-09 14.7006950378418\\
1.31874999809156e-08 14.7006950378418\\
2.0781250498203e-08 14.7006950378418\\
3.21718758300449e-08 14.7006950378418\\
4.92578138278077e-08 14.7006950378418\\
7.48867208244519e-08 14.7006950378418\\
1.13330081319418e-07 14.7006950378418\\
1.70995122061868e-07 14.7006950378418\\
2.57492672517401e-07 14.7006950378418\\
3.8723902662241e-07 14.7006950378418\\
5.81858500936505e-07 14.7006950378418\\
8.73787769251066e-07 14.7006950378418\\
1.31168167172291e-06 14.7006950378418\\
1.9685226106958e-06 14.7006950378418\\
2.95378367809462e-06 14.7006959915161\\
4.43167573394021e-06 14.7006959915161\\
6.64851359033491e-06 14.7006959915161\\
9.97376992017962e-06 14.7006959915161\\
1.49616553244414e-05 14.7006969451904\\
2.24434825213393e-05 14.7006978988647\\
3.36662233166862e-05 14.7006988525391\\
5.05003372381907e-05 14.7006998062134\\
7.57515081204474e-05 14.700701713562\\
0.000113628258986864 14.7007036209106\\
0.000170443381648511 14.7007055282593\\
0.000255666091106832 14.7007074356079\\
0.000383500126190484 14.7007083892822\\
0.000575251178815961 14.7007074356079\\
0.000862877757754177 14.7007036209106\\
0.00129431765526533 14.7007007598877\\
0.00194147753063589 14.7006969451904\\
0.00291221728548408 14.7006931304932\\
0.00436832662671804 14.7006912231445\\
0.00597004732117057 14.7006931304932\\
0.00773194013163447 14.7006950378418\\
0.009670021943748 14.7006959915161\\
0.0118019115179777 14.7006959915161\\
0.0141469910740852 14.7006959915161\\
0.0167265776544809 14.7006959915161\\
0.0195641238242388 14.7006959915161\\
0.0226854234933853 14.7006959915161\\
0.0261188540607691 14.7006959915161\\
0.0298956278711557 14.7006959915161\\
0.0340500771999359 14.7006959915161\\
0.0386199727654457 14.7006950378418\\
0.0436468608677387 14.7006950378418\\
0.0491764321923256 14.7006950378418\\
0.0552589632570744 14.7006950378418\\
0.0619497485458851 14.7006950378418\\
0.0693096145987511 14.7006959915161\\
0.0774054601788521 14.7006959915161\\
0.0863108932971954 14.7006931304932\\
0.0961068719625473 14.7006959915161\\
0.106882445514202 14.7006959915161\\
0.118735581636429 14.7006959915161\\
0.13177402317524 14.7006959915161\\
0.146116316318512 14.7006959915161\\
0.161892831325531 14.7006959915161\\
0.179247006773949 14.7006959915161\\
0.198336601257324 14.7006959915161\\
0.21933513879776 14.7006959915161\\
0.242433547973633 14.7006950378418\\
0.267841786146164 14.7006959915161\\
0.29579085111618 14.7006959915161\\
0.32653483748436 14.7006959915161\\
0.360353201627731 14.7006959915161\\
0.397553414106369 14.7006959915161\\
0.438473641872406 14.7006959915161\\
0.483485877513886 14.7006959915161\\
0.532999336719513 14.7006959915161\\
0.587464153766632 14.7006959915161\\
0.647375464439392 14.7006959915161\\
0.71327793598175 14.7006959915161\\
0.7857705950737 14.7006959915161\\
0.865512549877167 14.7006959915161\\
0.953228712081909 14.7006959915161\\
1.04971647262573 14.7006950378418\\
1.14971649646759 14.7006950378418\\
1.24971640110016 14.7006950378418\\
1.34971642494202 14.7006950378418\\
1.44971644878387 14.7006950378418\\
1.54971647262573 14.7006950378418\\
1.64971649646759 14.7006950378418\\
1.74971640110016 14.7006950378418\\
1.84971642494202 14.7006950378418\\
1.94971644878387 14.7006950378418\\
2.04971647262573 14.7006950378418\\
2.1497163772583 14.7006950378418\\
2.24971652030945 14.7006950378418\\
2.34971642494202 14.7006950378418\\
2.44971656799316 14.7006950378418\\
2.54971647262573 14.7006950378418\\
2.6497163772583 14.7006950378418\\
2.74971652030945 14.7006950378418\\
2.84971642494202 14.7006950378418\\
2.94971656799316 14.7006950378418\\
3.04971647262573 14.7006950378418\\
3.1497163772583 14.7006950378418\\
3.24971652030945 14.7006950378418\\
3.34971642494202 14.7006950378418\\
3.44971656799316 14.7006950378418\\
3.54971647262573 14.7006950378418\\
3.6497163772583 14.7006950378418\\
3.74971652030945 14.7006950378418\\
3.84971642494202 14.7006950378418\\
3.94971656799316 14.7006950378418\\
4.04971647262573 14.7006950378418\\
4.1497163772583 14.7006950378418\\
4.24971628189087 14.7006950378418\\
4.3497166633606 14.7006950378418\\
4.44971656799316 14.7006950378418\\
4.54971647262573 14.7006950378418\\
4.6497163772583 14.7006950378418\\
4.74971628189087 14.7006950378418\\
4.8497166633606 14.7006950378418\\
4.94971656799316 14.7006950378418\\
5.04971647262573 14.7006950378418\\
5.1497163772583 14.7006950378418\\
5.24971628189087 14.7006950378418\\
5.3497166633606 14.7006950378418\\
5.44971656799316 14.7006950378418\\
5.54971647262573 14.7006950378418\\
5.6497163772583 14.7006950378418\\
5.74971628189087 14.7006950378418\\
5.8497166633606 14.7006950378418\\
5.94971656799316 14.7006950378418\\
6.04971647262573 14.7006950378418\\
6.1497163772583 14.7006940841675\\
6.24971628189087 14.7006931304932\\
6.3497166633606 14.7006921768188\\
6.44971656799316 14.7006940841675\\
6.54971647262573 14.7006969451904\\
6.6497163772583 14.7006969451904\\
6.74971628189087 14.7006969451904\\
6.8497166633606 14.7006950378418\\
6.94971656799316 14.7006959915161\\
7.04971647262573 14.7006950378418\\
7.1497163772583 14.7006950378418\\
7.24971628189087 14.7006950378418\\
7.3497166633606 14.7006950378418\\
7.44971656799316 14.7006950378418\\
7.54971647262573 14.7006950378418\\
7.6497163772583 14.7006950378418\\
7.74971628189087 14.7006950378418\\
7.8497166633606 14.7006950378418\\
7.94971656799316 14.7006950378418\\
8.04971599578857 14.7006950378418\\
8.1497163772583 14.7006950378418\\
8.24971675872803 14.7006950378418\\
8.34971618652344 14.7006950378418\\
8.44971656799316 14.7006950378418\\
8.54971599578857 14.7006950378418\\
8.6497163772583 14.7006950378418\\
8.74971675872803 14.7006950378418\\
8.84971618652344 14.7006950378418\\
8.94971656799316 14.7006950378418\\
9.04971599578857 14.7006950378418\\
9.1497163772583 14.7006950378418\\
9.24971675872803 14.7006950378418\\
9.34971618652344 14.7006950378418\\
9.44971656799316 14.7006950378418\\
9.54971599578857 14.7006950378418\\
9.6497163772583 14.7006950378418\\
9.74971675872803 14.7006950378418\\
9.84971618652344 14.7006950378418\\
9.94971656799316 14.7006950378418\\
10.0497159957886 14.7006950378418\\
10.1497163772583 14.7006950378418\\
10.249716758728 14.7006950378418\\
10.3497161865234 14.7006950378418\\
10.4497165679932 14.7006950378418\\
10.5497159957886 14.7006950378418\\
10.6497163772583 14.7006950378418\\
10.749716758728 14.7006950378418\\
10.8497161865234 14.7006950378418\\
10.9497165679932 14.7006950378418\\
11.0497159957886 14.7006950378418\\
11.1497163772583 14.7006950378418\\
11.249716758728 14.7006950378418\\
11.3497161865234 14.7006950378418\\
11.4497165679932 14.7006950378418\\
11.5497159957886 14.7006950378418\\
11.6497163772583 14.7006959915161\\
11.749716758728 14.7006959915161\\
11.8497161865234 14.7006959915161\\
11.9497165679932 14.7006950378418\\
12.0497159957886 14.7006950378418\\
12.1497163772583 14.7006950378418\\
12.249716758728 14.7006950378418\\
12.347146987915 14.7006950378418\\
12.4250946044922 14.7006950378418\\
12.4874591827393 14.7006959915161\\
12.5560598373413 14.7006940841675\\
12.6315202713013 14.7006940841675\\
12.714527130127 14.7006950378418\\
12.8058338165283 14.7006959915161\\
12.8497476577759 14.7006959915161\\
12.8980522155762 14.7006950378418\\
12.9143037796021 14.7006959915161\\
12.9277105331421 14.7006978988647\\
12.9424571990967 14.7006978988647\\
12.9586782455444 14.7006969451904\\
12.9765224456787 14.7006969451904\\
12.9961500167847 14.7006959915161\\
13.0177412033081 14.7006959915161\\
13.0414915084839 14.7006959915161\\
13.0676164627075 14.7006959915161\\
13.0963535308838 14.7006959915161\\
13.1279649734497 14.7006959915161\\
13.1627368927002 14.7006959915161\\
13.2009868621826 14.7006959915161\\
13.2430610656738 14.7006959915161\\
13.289342880249 14.7006959915161\\
13.3402528762817 14.7006959915161\\
13.3962545394897 14.7006959915161\\
13.4578561782837 14.7006959915161\\
13.5256175994873 14.7006959915161\\
13.600154876709 14.7006959915161\\
13.6821460723877 14.7006959915161\\
13.7723369598389 14.7006969451904\\
13.871545791626 14.7006959915161\\
13.9715461730957 14.7007293701172\\
14.0715456008911 14.700870513916\\
14.1215467453003 14.7015914916992\\
14.1629066467285 14.7015705108643\\
14.2008075714111 14.7712888717651\\
14.202446937561 14.7973413467407\\
14.2042503356934 14.8541603088379\\
14.2062339782715 14.8689184188843\\
14.2084159851074 14.8619966506958\\
14.2108154296875 14.8516969680786\\
14.2134561538696 14.833625793457\\
14.2163600921631 14.8155269622803\\
14.219554901123 14.7990865707397\\
14.223069190979 14.7814884185791\\
14.2269344329834 14.7788047790527\\
14.2311868667603 14.7710580825806\\
14.2358636856079 14.764196395874\\
14.2410087585449 14.7578344345093\\
14.2466678619385 14.7517852783203\\
14.252893447876 14.7459573745728\\
14.2597417831421 14.7404403686523\\
14.2672748565674 14.7353935241699\\
14.2755603790283 14.7309741973877\\
14.2846755981445 14.727331161499\\
14.2947015762329 14.7246294021606\\
14.3057298660278 14.7229442596436\\
14.3178615570068 14.7221908569336\\
14.3312063217163 14.718150138855\\
14.3458862304688 14.7118816375732\\
14.3620328903198 14.7066335678101\\
14.3797950744629 14.7041082382202\\
14.3993330001831 14.7042331695557\\
14.4208250045776 14.7060146331787\\
14.444465637207 14.7087965011597\\
14.4704713821411 14.7034091949463\\
14.4990768432617 14.7011003494263\\
14.5269069671631 14.7006616592407\\
14.5575199127197 14.7006683349609\\
14.591194152832 14.7006826400757\\
14.6282358169556 14.7006950378418\\
14.668981552124 14.7007131576538\\
14.7138023376465 14.7007160186768\\
14.7631044387817 14.7007064819336\\
14.8173370361328 14.7007055282593\\
14.8769931793213 14.700704574585\\
14.9244384765625 14.7006988525391\\
14.971435546875 14.7006959915161\\
15.0163955688477 14.7006950378418\\
15.0584926605225 14.7006950378418\\
15.09974193573 14.7006940841675\\
15.1451168060303 14.7006940841675\\
15.1950283050537 14.7006959915161\\
15.2499313354492 14.7006950378418\\
15.31032371521 14.7006969451904\\
15.3767566680908 14.7006959915161\\
15.4498319625854 14.7006959915161\\
15.5302152633667 14.7006959915161\\
15.6186370849609 14.7006950378418\\
15.7159004211426 14.7006969451904\\
15.8159008026123 14.7006950378418\\
15.9159002304077 14.7006969451904\\
16.0159015655518 14.7006959915161\\
16.0210952758789 14.7006931304932\\
16.0268077850342 14.7007036209106\\
16.0330924987793 14.7006931304932\\
16.0382099151611 14.7006978988647\\
16.0438404083252 14.7006988525391\\
16.0500316619873 14.7006978988647\\
16.0568428039551 14.7006978988647\\
16.0643348693848 14.7006978988647\\
16.0725765228271 14.7006978988647\\
16.0816440582275 14.7006978988647\\
16.0916156768799 14.7006969451904\\
16.1025848388672 14.7006969451904\\
16.114652633667 14.7006969451904\\
16.1279258728027 14.7006969451904\\
16.1425266265869 14.7006969451904\\
16.1585865020752 14.7006969451904\\
16.1762542724609 14.7006969451904\\
16.1956882476807 14.7006969451904\\
16.2170658111572 14.7006969451904\\
16.2405796051025 14.7006969451904\\
16.2664451599121 14.7006969451904\\
16.2948989868164 14.7006969451904\\
16.3261966705322 14.7006969451904\\
16.3606243133545 14.7006969451904\\
16.398494720459 14.7006969451904\\
16.4401531219482 14.7006969451904\\
16.4859771728516 14.7006969451904\\
16.5363826751709 14.7006969451904\\
16.5918292999268 14.7006969451904\\
16.6528205871582 14.7006969451904\\
16.7199096679688 14.7006969451904\\
16.7937088012695 14.7006969451904\\
16.8748874664307 14.7006969451904\\
16.9641857147217 14.7006950378418\\
17.0624122619629 14.7006969451904\\
17.1624126434326 14.7006959915161\\
17.2624130249023 14.7007446289063\\
17.282413482666 14.7008991241455\\
17.2906188964844 14.7010221481323\\
17.2996425628662 14.7010688781738\\
17.308614730835 14.7010984420776\\
17.3184852600098 14.7011041641235\\
17.3293399810791 14.7011127471924\\
17.3412818908691 14.7024831771851\\
17.3544178009033 14.7275695800781\\
17.3688678741455 14.7853755950928\\
17.3847618103027 14.8174285888672\\
17.4022445678711 14.7983932495117\\
17.4214763641357 14.7616243362427\\
17.4426326751709 14.7631587982178\\
17.4659042358398 14.7555694580078\\
17.4915008544922 14.7472877502441\\
17.5196590423584 14.7405052185059\\
17.5506324768066 14.7360191345215\\
17.5847034454346 14.7344999313354\\
17.6221809387207 14.7355270385742\\
17.6634063720703 14.7383518218994\\
17.7087554931641 14.7449922561646\\
17.758638381958 14.7512359619141\\
17.8135089874268 14.7515563964844\\
17.8738670349121 14.7414798736572\\
17.9402618408203 14.739917755127\\
18.0132961273193 14.7363691329956\\
18.0585689544678 14.7272548675537\\
18.1083717346191 14.7288846969604\\
18.1509475708008 14.7240753173828\\
18.1891002655029 14.7205533981323\\
18.2310676574707 14.7198886871338\\
18.2772312164307 14.7213172912598\\
18.328010559082 14.72412109375\\
18.3838691711426 14.7261362075806\\
18.4453125 14.7280235290527\\
18.5129013061523 14.729790687561\\
18.5872478485107 14.7311811447144\\
18.6690311431885 14.7321882247925\\
18.7589912414551 14.7327442169189\\
18.857946395874 14.7330522537231\\
18.9579467773438 14.7327098846436\\
19.0579471588135 14.7331666946411\\
19.1579456329346 14.7335662841797\\
19.2579460144043 14.7342500686646\\
19.357946395874 14.7346525192261\\
19.4579467773438 14.7352237701416\\
19.5579471588135 14.7356719970703\\
19.6579456329346 14.736216545105\\
19.7579460144043 14.7366981506348\\
19.857946395874 14.7372255325317\\
19.9579467773438 14.7377214431763\\
20.0579471588135 14.7382392883301\\
20.1579456329346 14.7387428283691\\
20.2579460144043 14.7392578125\\
20.357946395874 14.7397651672363\\
20.4579467773438 14.7402782440186\\
20.5579471588135 14.7407932281494\\
20.6579456329346 14.7413101196289\\
20.7579460144043 14.7418270111084\\
20.857946395874 14.7423458099365\\
20.9579467773438 14.7428636550903\\
21.0579471588135 14.7433815002441\\
21.1579456329346 14.7439908981323\\
21.2579460144043 14.744779586792\\
21.357946395874 14.7453556060791\\
21.4579467773438 14.7459888458252\\
21.5579471588135 14.7466039657593\\
21.6579456329346 14.7472229003906\\
21.7579460144043 14.747838973999\\
21.857946395874 14.7484540939331\\
21.9579467773438 14.7490682601929\\
22.0579471588135 14.7496814727783\\
22.1579456329346 14.7502927780151\\
22.2579460144043 14.750904083252\\
22.357946395874 14.7515134811401\\
22.4579467773438 14.7521228790283\\
22.5579471588135 14.7527303695679\\
22.6579456329346 14.7533369064331\\
22.7579460144043 14.7537355422974\\
22.857946395874 14.7541751861572\\
22.9579467773438 14.7547063827515\\
23.0579471588135 14.7552003860474\\
23.1579456329346 14.7556991577148\\
23.2579460144043 14.7561874389648\\
23.357946395874 14.7566690444946\\
23.4579467773438 14.7571411132813\\
23.5579471588135 14.7576036453247\\
23.6579456329346 14.7580528259277\\
23.7579460144043 14.7584867477417\\
23.857946395874 14.7589073181152\\
23.9579467773438 14.7593097686768\\
24.0579471588135 14.7633056640625\\
24.1579456329346 14.7625570297241\\
24.2579460144043 14.7624015808105\\
24.357946395874 14.7632074356079\\
24.4579467773438 14.7638292312622\\
24.5579471588135 14.7646245956421\\
24.6579456329346 14.7717609405518\\
24.7421741485596 14.7758054733276\\
24.8095417022705 14.7720336914063\\
24.8836441040039 14.7685317993164\\
24.9651584625244 14.7689018249512\\
25.0548248291016 14.7859888076782\\
25.0960559844971 14.7341136932373\\
25.1414089202881 14.7641868591309\\
25.1613826751709 14.7782249450684\\
25.1780624389648 14.7797412872314\\
25.1964111328125 14.7712078094482\\
25.2165927886963 14.7712717056274\\
25.2387943267822 14.7714099884033\\
25.2632160186768 14.771577835083\\
25.2900791168213 14.7717838287354\\
25.3196296691895 14.7720289230347\\
25.3521347045898 14.7723159790039\\
25.3878898620605 14.772647857666\\
25.4272212982178 14.7730255126953\\
25.4704837799072 14.7734518051147\\
25.5180740356445 14.7739276885986\\
25.5704231262207 14.7744569778442\\
25.6280078887939 14.7750425338745\\
25.6913509368896 14.7756872177124\\
25.7610282897949 14.7763957977295\\
25.8376731872559 14.7771759033203\\
25.9219818115234 14.7871398925781\\
26.014720916748 14.7703218460083\\
26.1147212982178 14.7800798416138\\
26.2147216796875 14.783576965332\\
26.3147220611572 14.7887868881226\\
26.3647212982178 14.848048210144\\
26.4022121429443 14.8336534500122\\
26.4049663543701 14.8252029418945\\
26.407995223999 14.8308486938477\\
26.4113292694092 14.8344211578369\\
26.4149932861328 14.8308010101318\\
26.4190254211426 14.8281803131104\\
26.4234619140625 14.8212280273438\\
26.4283409118652 14.8120756149292\\
26.4337062835693 14.8048257827759\\
26.4396114349365 14.7993326187134\\
26.4461040496826 14.7933053970337\\
26.4532470703125 14.7859468460083\\
26.4611053466797 14.7760133743286\\
26.469747543335 14.7623176574707\\
26.4792556762695 14.7458114624023\\
26.4897136688232 14.7285528182983\\
26.5012187957764 14.716667175293\\
26.5138721466064 14.7106418609619\\
26.5277919769287 14.7095375061035\\
26.5431041717529 14.712498664856\\
26.5599479675293 14.7219457626343\\
26.5784759521484 14.7422571182251\\
26.59885597229 14.7919092178345\\
26.6212730407715 14.821683883667\\
26.6459331512451 14.8119859695435\\
26.673059463501 14.8118467330933\\
26.7028980255127 14.8162488937378\\
26.7357215881348 14.8232622146606\\
26.7718257904053 14.8208951950073\\
26.8115406036377 14.8129711151123\\
26.8552284240723 14.8080987930298\\
26.903284072876 14.8048315048218\\
26.9561443328857 14.8021869659424\\
27.0142917633057 14.8009271621704\\
27.0605869293213 14.8015432357788\\
27.1115131378174 14.799783706665\\
27.1556930541992 14.7989597320557\\
27.1975421905518 14.7998809814453\\
27.2435741424561 14.8006858825684\\
27.29421043396 14.7990341186523\\
27.3499088287354 14.7958717346191\\
27.4111785888672 14.7958831787109\\
27.4785747528076 14.7961912155151\\
27.5527095794678 14.7965116500854\\
27.6342601776123 14.7970752716064\\
27.7239646911621 14.7999620437622\\
27.8226375579834 14.796441078186\\
27.9226379394531 14.8019714355469\\
28.0226383209229 14.7982234954834\\
28.1226387023926 14.8008489608765\\
28.1260833740234 14.9006481170654\\
28.1298713684082 14.7492933273315\\
28.1340389251709 14.7903842926025\\
28.1376724243164 14.8755235671997\\
28.1416683197021 14.7502422332764\\
28.1460647583008 14.7953195571899\\
28.1490650177002 14.813307762146\\
28.1523666381836 14.8035535812378\\
28.1559963226318 14.803147315979\\
28.1599903106689 14.8032274246216\\
28.1643829345703 14.8032341003418\\
28.1692161560059 14.8032369613647\\
28.1745319366455 14.803243637085\\
28.1803779602051 14.8032541275024\\
28.1868114471436 14.8032703399658\\
28.1938858032227 14.8032932281494\\
28.2016677856445 14.8033227920532\\
28.2102298736572 14.8033609390259\\
28.2196464538574 14.8034076690674\\
28.2300052642822 14.803466796875\\
28.2413997650146 14.8035383224487\\
28.2539348602295 14.8036241531372\\
28.2677211761475 14.8037252426147\\
28.2828884124756 14.8038444519043\\
28.2995700836182 14.8039836883545\\
28.3179225921631 14.8041429519653\\
28.3381080627441 14.8043260574341\\
28.3603134155273 14.8045339584351\\
28.3847389221191 14.8047685623169\\
28.4116058349609 14.8050308227539\\
28.4411602020264 14.8053255081177\\
28.4736709594727 14.8056516647339\\
28.5094318389893 14.8060140609741\\
28.5487689971924 14.806414604187\\
28.592041015625 14.8068561553955\\
28.6396389007568 14.8073425292969\\
28.6919975280762 14.8078784942627\\
28.7495918273926 14.8084688186646\\
28.8129444122314 14.8091173171997\\
28.8826332092285 14.8098306655884\\
28.9592895507813 14.8106164932251\\
29.0436134338379 14.8114805221558\\
29.1363677978516 14.8142318725586\\
29.2363681793213 14.8115177154541\\
29.336368560791 14.8147134780884\\
29.4363689422607 14.8227472305298\\
29.4563694000244 14.825252532959\\
29.4642658233643 14.8501882553101\\
29.471960067749 14.8539094924927\\
29.4804267883301 14.8521890640259\\
29.4885120391846 14.8503618240356\\
29.497407913208 14.8512048721313\\
29.5071926116943 14.8537588119507\\
29.5179557800293 14.8819608688354\\
29.5297946929932 14.9143142700195\\
29.542818069458 14.9204559326172\\
29.5571441650391 14.9001007080078\\
29.5729026794434 14.8823947906494\\
29.5902366638184 14.8634014129639\\
29.6093044281006 14.844141960144\\
29.630277633667 14.8253393173218\\
29.6533489227295 14.8054790496826\\
29.6787281036377 14.7781705856323\\
29.6999015808105 14.765531539917\\
29.7231941223145 14.7706890106201\\
29.7488136291504 14.785083770752\\
29.7769947052002 14.8063268661499\\
29.8079967498779 14.8394737243652\\
29.8420963287354 14.8480978012085\\
29.8796062469482 14.845383644104\\
29.9208679199219 14.8454208374023\\
29.9662551879883 14.8260917663574\\
30.0161819458008 14.855565071106\\
30.0711002349854 14.8370428085327\\
30.1315116882324 14.8355121612549\\
30.1804618835449 14.830846786499\\
30.2286376953125 14.8257703781128\\
30.2746448516846 14.8235378265381\\
30.3156337738037 14.822135925293\\
30.3508148193359 14.8268299102783\\
30.3895168304443 14.8292980194092\\
30.4320869445801 14.8300714492798\\
30.4789142608643 14.8293828964233\\
30.530424118042 14.8300247192383\\
30.587085723877 14.8307418823242\\
30.6491241455078 14.8314914703369\\
30.7173671722412 14.8328800201416\\
30.7924346923828 14.8354530334473\\
30.8750076293945 14.8394088745117\\
30.9658393859863 14.8430318832397\\
31.0657539367676 14.8409242630005\\
31.1657543182373 14.8399267196655\\
31.265754699707 14.8391132354736\\
31.3657531738281 14.8387136459351\\
31.4657535552979 14.8387184143066\\
31.5657539367676 14.8400421142578\\
31.6657543182373 14.8409643173218\\
31.765754699707 14.8423614501953\\
31.8657531738281 14.8432855606079\\
31.9657535552979 14.844521522522\\
32.0657539367676 14.8455533981323\\
32.1657524108887 14.8467206954956\\
32.265754699707 14.8477993011475\\
32.3657531738281 14.8489370346069\\
32.4657554626465 14.8500356674194\\
32.5657539367676 14.8511610031128\\
32.6657524108887 14.8522691726685\\
32.765754699707 14.853404045105\\
32.8657531738281 14.8545045852661\\
32.9657554626465 14.8556299209595\\
33.0657539367676 14.8567495346069\\
33.1657524108887 14.8578729629517\\
33.265754699707 14.8589944839478\\
33.3657531738281 14.8601169586182\\
33.4657554626465 14.8612384796143\\
33.5657539367676 14.8623609542847\\
33.6657524108887 14.8638000488281\\
33.765754699707 14.8648748397827\\
33.8657531738281 14.8660125732422\\
33.9657554626465 14.867130279541\\
34.0657539367676 14.8682546615601\\
34.1657524108887 14.8693771362305\\
34.265754699707 14.8704996109009\\
34.3657531738281 14.8716220855713\\
34.4657554626465 14.872745513916\\
34.5657539367676 14.8738679885864\\
34.6657524108887 14.8749904632568\\
34.765754699707 14.8761138916016\\
34.8657531738281 14.877236366272\\
34.9657554626465 14.8783597946167\\
35.0657539367676 14.8794832229614\\
35.1657524108887 14.8804931640625\\
35.265754699707 14.8828592300415\\
35.3657531738281 14.8829736709595\\
35.4657554626465 14.8837022781372\\
35.5657539367676 14.8849182128906\\
35.6657524108887 14.8859987258911\\
35.765754699707 14.8871307373047\\
35.8657531738281 14.8882427215576\\
35.9657554626465 14.8893613815308\\
36.0657539367676 14.8904781341553\\
36.1657524108887 14.8915948867798\\
36.265754699707 14.8927459716797\\
36.3657531738281 14.8938865661621\\
36.4657554626465 14.8951177597046\\
36.5657539367676 14.8962450027466\\
36.6657524108887 14.8973865509033\\
36.765754699707 14.8985681533813\\
36.8657531738281 14.8997735977173\\
36.9657554626465 14.9010314941406\\
37.0657539367676 14.9073648452759\\
37.1539344787598 14.9112615585327\\
37.2244758605957 14.9100408554077\\
37.302074432373 14.904881477356\\
37.3874282836914 14.9054937362671\\
37.4813194274902 14.9221267700195\\
37.5215835571289 14.8703212738037\\
37.5658721923828 14.901478767395\\
37.585823059082 14.9153032302856\\
37.6024284362793 14.9171552658081\\
37.6206932067871 14.9087934494019\\
37.6407852172852 14.9088315963745\\
37.6628875732422 14.9089813232422\\
37.687198638916 14.9091501235962\\
37.713939666748 14.9093561172485\\
37.7433547973633 14.9096031188965\\
37.7757110595703 14.90989112854\\
37.8113059997559 14.9102230072021\\
37.8504600524902 14.9106016159058\\
37.8935279846191 14.9110298156738\\
37.9409027099609 14.9115076065063\\
37.9930152893066 14.9120388031006\\
38.0503387451172 14.9126262664795\\
38.1133918762207 14.913272857666\\
38.1827545166016 14.9139842987061\\
38.2590522766113 14.9147672653198\\
38.3429794311523 14.9248304367065\\
38.4352989196777 14.9079294204712\\
38.5352973937988 14.9177131652832\\
38.6352996826172 14.9214849472046\\
38.7352981567383 14.9269790649414\\
38.7853012084961 14.9785614013672\\
38.822696685791 14.9526319503784\\
38.8252944946289 14.9351177215576\\
38.8281555175781 14.9392414093018\\
38.8313026428223 14.9413003921509\\
38.8347625732422 14.9388408660889\\
38.838565826416 14.9328174591064\\
38.8427543640137 14.9243354797363\\
38.8473587036133 14.9157981872559\\
38.852424621582 14.9089088439941\\
38.8579978942871 14.9027090072632\\
38.864128112793 14.8959226608276\\
38.8708724975586 14.8875455856323\\
38.878288269043 14.8752603530884\\
38.8864479064941 14.8587512969971\\
38.8954238891602 14.8369483947754\\
38.9052963256836 14.8196353912354\\
38.9161567687988 14.801851272583\\
38.9281005859375 14.8035411834717\\
38.9412422180176 14.8052930831909\\
38.955696105957 14.8090343475342\\
38.9715957641602 14.8169059753418\\
38.989086151123 14.8354797363281\\
39.0083236694336 14.89293384552\\
39.029483795166 14.9465942382813\\
39.0527648925781 14.9545421600342\\
39.078369140625 14.937463760376\\
39.1065368652344 14.9392805099487\\
39.137523651123 14.9429740905762\\
39.1716041564941 14.9427919387817\\
39.2090950012207 14.9398365020752\\
39.2477378845215 14.9365653991699\\
39.2902450561523 14.9343252182007\\
39.3370056152344 14.9328317642212\\
39.3884391784668 14.9319725036621\\
39.4450187683105 14.9328527450562\\
39.4936180114746 14.9331951141357\\
39.5406875610352 14.9330749511719\\
39.5853233337402 14.9331750869751\\
39.6272773742676 14.9355363845825\\
39.6734275817871 14.935489654541\\
39.7241897583008 14.9341926574707\\
39.780029296875 14.931957244873\\
39.8414497375488 14.9322671890259\\
39.9090156555176 14.9327945709229\\
39.9833374023438 14.9333057403564\\
40.0650939941406 14.9340209960938\\
40.1550216674805 14.9370288848877\\
40.2539443969727 14.933632850647\\
40.353946685791 14.93922996521\\
40.4539451599121 14.9355688095093\\
40.5539436340332 14.9381160736084\\
40.5569686889648 15.0468902587891\\
40.5602951049805 14.8930931091309\\
40.5639533996582 14.9047975540161\\
40.5665168762207 14.963171005249\\
40.5693359375 14.9612560272217\\
40.572437286377 14.9421281814575\\
40.5758476257324 14.9405498504639\\
40.5796012878418 14.9412393569946\\
40.5837287902832 14.9413242340088\\
40.5882682800293 14.9413051605225\\
40.59326171875 14.9413003921509\\
40.5987548828125 14.9413022994995\\
40.6048011779785 14.9413089752197\\
40.6114463806152 14.9413204193115\\
40.6187591552734 14.9413385391235\\
40.6268005371094 14.9413642883301\\
40.6356506347656 14.9413995742798\\
40.6453819274902 14.9414443969727\\
40.6560897827148 14.9415016174316\\
40.6678657531738 14.9415721893311\\
40.6808204650879 14.9416580200195\\
40.695068359375 14.9417610168457\\
40.7107429504395 14.9418821334839\\
40.7279853820801 14.942024230957\\
40.7469482421875 14.9421882629395\\
40.7678108215332 14.9423770904541\\
40.7907600402832 14.9425916671753\\
40.8160057067871 14.9428339004517\\
40.8437728881836 14.9431056976318\\
40.8743171691895 14.9434099197388\\
40.9079170227051 14.9437475204468\\
40.9448776245117 14.9441223144531\\
40.9855308532715 14.9445362091064\\
41.0302505493164 14.944993019104\\
41.0794448852539 14.9454965591431\\
41.1335563659668 14.9460506439209\\
41.1930809020996 14.9466600418091\\
41.2585563659668 14.9473304748535\\
41.3305778503418 14.9480686187744\\
41.4098052978516 14.9488792419434\\
41.4969520568848 14.9517307281494\\
41.5928153991699 14.9486961364746\\
41.692813873291 14.9519386291504\\
41.7928161621094 14.9601202011108\\
41.812816619873 14.9618997573853\\
41.8206024169922 14.9866905212402\\
41.8281707763672 14.9902801513672\\
41.8364944458008 14.9876079559326\\
41.8444328308105 14.9844951629639\\
41.8531684875488 14.9839754104614\\
41.8627777099609 14.9846096038818\\
41.8733444213867 15.0077123641968\\
41.8849716186523 15.042350769043\\
41.8977584838867 15.0458459854126\\
41.9118270874023 15.0249519348145\\
41.9272994995117 15.0048379898071\\
41.9443206787109 14.9834451675415\\
41.9630432128906 14.9617710113525\\
41.9836387634277 14.9404754638672\\
42.0062942504883 14.9154348373413\\
42.0312156677246 14.886248588562\\
42.0477752685547 14.8740587234497\\
42.0659866333008 14.8753185272217\\
42.0860252380371 14.8866281509399\\
42.1080627441406 14.9066314697266\\
42.1323051452637 14.9163885116577\\
42.1589736938477 14.9527816772461\\
42.1883087158203 14.960976600647\\
42.2205772399902 14.9629468917847\\
42.2560729980469 14.9583158493042\\
42.2951164245605 14.9618625640869\\
42.3380661010742 14.9693574905396\\
42.3853073120117 14.9683485031128\\
42.4372749328613 14.9650659561157\\
42.494441986084 14.9670066833496\\
42.5416946411133 14.9568157196045\\
42.5936737060547 14.9582090377808\\
42.6379699707031 14.956018447876\\
42.6777877807617 14.9578504562378\\
42.711311340332 14.9650087356567\\
42.7481842041016 14.9664974212646\\
42.788745880127 14.96644115448\\
42.833366394043 14.965069770813\\
42.8824462890625 14.9657049179077\\
42.9364318847656 14.9663181304932\\
42.9958190917969 14.9668292999268\\
43.0611457824707 14.9676332473755\\
43.1330032348633 14.9689092636108\\
43.2120475769043 14.9712390899658\\
43.2989959716797 14.9754133224487\\
43.3946380615234 14.97984790802\\
43.4946365356445 14.9797563552856\\
43.5946388244629 14.9774351119995\\
43.694637298584 14.9767141342163\\
43.7946357727051 14.9762372970581\\
43.8946380615234 14.9761629104614\\
43.9946365356445 14.9775247573853\\
44.0946388244629 14.9784469604492\\
44.194637298584 14.9798679351807\\
44.2946357727051 14.9808006286621\\
44.3946380615234 14.9820547103882\\
44.4946365356445 14.983097076416\\
44.5946388244629 14.9842796325684\\
44.694637298584 14.985369682312\\
44.7946357727051 14.9865217208862\\
44.8946380615234 14.9876327514648\\
44.9946365356445 14.9887704849243\\
45.0946388244629 14.9899053573608\\
45.194637298584 14.9910192489624\\
45.2946357727051 14.9921436309814\\
45.3946380615234 14.9932727813721\\
45.4946365356445 14.9943990707397\\
45.5946388244629 14.9955272674561\\
45.694637298584 14.9966535568237\\
45.7946357727051 14.99778175354\\
45.8946380615234 14.998908996582\\
45.9946365356445 15.000036239624\\
46.0946388244629 15.0014657974243\\
46.194637298584 15.0025653839111\\
46.2946357727051 15.0037021636963\\
46.3946380615234 15.0048274993896\\
46.4946365356445 15.0059566497803\\
46.5946388244629 15.0070838928223\\
46.694637298584 15.0082120895386\\
46.7946357727051 15.0093393325806\\
46.8946380615234 15.0104675292969\\
46.9946365356445 15.0115947723389\\
47.0946388244629 15.0127229690552\\
47.194637298584 15.0138502120972\\
47.2946357727051 15.0149784088135\\
47.3946380615234 15.0161056518555\\
47.4946365356445 15.0172338485718\\
47.5946388244629 15.0182676315308\\
47.694637298584 15.0206108093262\\
47.7946357727051 15.0207405090332\\
47.8946380615234 15.0214710235596\\
47.9946365356445 15.0226926803589\\
48.0946388244629 15.0237770080566\\
48.194637298584 15.0249128341675\\
48.2946357727051 15.0260286331177\\
48.3946380615234 15.0271511077881\\
48.4946365356445 15.0282716751099\\
48.5946388244629 15.0293922424316\\
48.694637298584 15.0305442810059\\
48.7946357727051 15.0316886901855\\
48.8946380615234 15.0329189300537\\
48.9946365356445 15.034049987793\\
49.0946388244629 15.0351963043213\\
49.194637298584 15.0363779067993\\
49.2946357727051 15.0375843048096\\
49.3946380615234 15.0388402938843\\
49.4946365356445 15.0447454452515\\
49.5842514038086 15.0488681793213\\
49.6559371948242 15.0485153198242\\
49.7347946166992 15.0427618026733\\
49.8215370178223 15.0433874130249\\
49.9169502258301 15.0598392486572\\
49.9568328857422 15.0077390670776\\
50.0007057189941 15.039421081543\\
50.0206336975098 15.0530853271484\\
50.0371971130371 15.0549802780151\\
50.0554161071777 15.0467109680176\\
50.0754585266113 15.0467319488525\\
50.0975074768066 15.0468893051147\\
50.1217575073242 15.0470581054688\\
50.1484336853027 15.0472650527954\\
50.1777763366699 15.047511100769\\
50.2100563049316 15.0477981567383\\
50.2455635070801 15.0481300354004\\
50.2846183776855 15.048508644104\\
50.327579498291 15.0489358901978\\
50.3748397827148 15.0494136810303\\
50.4268226623535 15.0499439239502\\
50.4840087890625 15.0505313873291\\
50.5469093322754 15.0511770248413\\
50.6161003112793 15.0518884658813\\
50.692211151123 15.0526704788208\\
50.7759323120117 15.0627861022949\\
50.8680267333984 15.0457849502563\\
50.9680252075195 15.0556125640869\\
51.0680236816406 15.0593776702881\\
51.168025970459 15.0648803710938\\
51.2180252075195 15.116548538208\\
51.2553977966309 15.0912437438965\\
51.2579727172852 15.0742902755737\\
51.2608032226563 15.0771884918213\\
51.263916015625 15.0795803070068\\
51.2673416137695 15.0770597457886\\
51.2711067199707 15.0709066390991\\
51.2752532958984 15.0623970031738\\
51.2798080444336 15.0539112091064\\
51.2848243713379 15.0469694137573\\
51.290340423584 15.040843963623\\
51.2964057922363 15.0340385437012\\
51.3030815124512 15.0255680084229\\
51.3104209899902 15.0131368637085\\
51.3184967041016 14.9962282180786\\
51.3273811340332 14.9742994308472\\
51.3371505737305 14.9571199417114\\
51.347900390625 14.940544128418\\
51.3597221374512 14.9411754608154\\
51.372730255127 14.9427242279053\\
51.387035369873 14.9463291168213\\
51.4027709960938 14.9534101486206\\
51.4200820922852 14.9710655212402\\
51.4391212463379 15.0305023193359\\
51.4600677490234 15.0826597213745\\
51.4831085205078 15.0915117263794\\
51.5084533691406 15.0745086669922\\
51.5363311767578 15.075927734375\\
51.5669975280762 15.0801773071289\\
51.6007308959961 15.0809307098389\\
51.6378364562988 15.0782690048218\\
51.6764678955078 15.0748386383057\\
51.7189598083496 15.0725440979004\\
51.7657051086426 15.0709924697876\\
51.8171195983887 15.0698881149292\\
51.8736801147461 15.0701761245728\\
51.9213600158691 15.0718479156494\\
51.9687728881836 15.0711879730225\\
52.0141105651855 15.0710296630859\\
52.056568145752 15.0728902816772\\
52.103271484375 15.0737867355347\\
52.1546440124512 15.0724945068359\\
52.2111549377441 15.0699281692505\\
52.2733154296875 15.0702238082886\\
52.3416938781738 15.0707664489746\\
52.4169082641602 15.0712728500366\\
52.4996490478516 15.0719947814941\\
52.5906600952148 15.074990272522\\
52.6906585693359 15.0716514587402\\
52.790657043457 15.0772457122803\\
52.8906593322754 15.073579788208\\
52.9906578063965 15.0761108398438\\
52.9936103820801 15.1815595626831\\
52.9968528747559 15.0397243499756\\
53.0004234313965 15.0401020050049\\
53.0029792785645 15.0965013504028\\
53.0057907104492 15.1011781692505\\
53.0088844299316 15.081202507019\\
53.0122871398926 15.0781116485596\\
53.0160293579102 15.079122543335\\
53.0201454162598 15.0793619155884\\
53.0246734619141 15.0793313980103\\
53.0296516418457 15.0793190002441\\
53.0351333618164 15.0793209075928\\
53.0411605834961 15.079327583313\\
53.0477867126465 15.0793380737305\\
53.0550804138184 15.0793561935425\\
53.063102722168 15.079381942749\\
53.0719261169434 15.0794162750244\\
53.0816307067871 15.0794610977173\\
53.0923080444336 15.079517364502\\
53.1040496826172 15.0795869827271\\
53.1169700622559 15.0796728134155\\
53.1311798095703 15.0797748565674\\
53.1468124389648 15.0798959732056\\
53.164005279541 15.0800371170044\\
53.1829223632813 15.0802001953125\\
53.203727722168 15.0803880691528\\
53.2266120910645 15.080602645874\\
53.2517852783203 15.0808439254761\\
53.2794799804688 15.0811157226563\\
53.3099403381348 15.0814189910889\\
53.343448638916 15.0817565917969\\
53.3803062438965 15.0821304321289\\
53.4208488464355 15.0825433731079\\
53.465446472168 15.0829992294312\\
53.5145034790039 15.0835018157959\\
53.5684700012207 15.0840539932251\\
53.6278305053711 15.0846633911133\\
53.6931266784668 15.0853319168091\\
53.7649536132813 15.0860691070557\\
53.8439598083496 15.0868787765503\\
53.9308700561523 15.0897359848022\\
54.0264701843262 15.0866851806641\\
54.1264686584473 15.0899314880371\\
54.2264709472656 15.0981073379517\\
54.2464714050293 15.1001882553101\\
54.2542266845703 15.1247634887695\\
54.2617721557617 15.1285429000854\\
54.2700691223145 15.1257877349854\\
54.2779922485352 15.1225824356079\\
54.2867050170898 15.1219940185547\\
54.2962913513184 15.1225337982178\\
54.3068389892578 15.1454792022705\\
54.3184356689453 15.1801357269287\\
54.3311958312988 15.1838722229004\\
54.3452301025391 15.1629858016968\\
54.360668182373 15.1428747177124\\
54.3776512145996 15.1215333938599\\
54.3963317871094 15.0998964309692\\
54.4168815612793 15.0786294937134\\
54.4394836425781 15.0536165237427\\
54.4643478393555 15.0244607925415\\
54.4809799194336 15.0120620727539\\
54.4992752075195 15.0131940841675\\
54.519401550293 15.0246343612671\\
54.5415382385254 15.0447454452515\\
54.5658912658691 15.0545864105225\\
54.5926780700684 15.0911378860474\\
54.622142791748 15.0991878509521\\
54.6545562744141 15.1009874343872\\
54.6902084350586 15.0960216522217\\
54.7294273376465 15.100604057312\\
54.7725677490234 15.1065816879272\\
54.8200225830078 15.1069784164429\\
54.8722229003906 15.1028804779053\\
54.9296417236328 15.1044301986694\\
54.9768371582031 15.095965385437\\
55.028751373291 15.0951986312866\\
55.0732498168945 15.0939340591431\\
55.112865447998 15.096245765686\\
55.1461601257324 15.1035261154175\\
55.1827850341797 15.1047391891479\\
55.223072052002 15.1045246124268\\
55.2673873901367 15.103120803833\\
55.3161315917969 15.1037588119507\\
55.3697547912598 15.1043462753296\\
55.4287376403809 15.1048393249512\\
55.4936180114746 15.1056108474731\\
55.5649871826172 15.1068115234375\\
55.643497467041 15.1089820861816\\
55.7298545837402 15.1129627227783\\
55.8248481750488 15.1175813674927\\
55.9248466491699 15.1182641983032\\
56.024845123291 15.1156520843506\\
56.1248474121094 15.1149187088013\\
56.2248458862305 15.1144046783447\\
56.3248481750488 15.114185333252\\
56.4248466491699 15.1155347824097\\
56.524845123291 15.116455078125\\
56.6248474121094 15.1178770065308\\
56.7248458862305 15.1188097000122\\
56.8248481750488 15.1200647354126\\
56.9248466491699 15.1211080551147\\
57.024845123291 15.1222915649414\\
57.1248474121094 15.1233825683594\\
57.2248458862305 15.1245346069336\\
57.3248481750488 15.1256465911865\\
57.4248466491699 15.1267852783203\\
57.524845123291 15.1279058456421\\
57.6248474121094 15.1290531158447\\
57.7248458862305 15.1301603317261\\
57.8248481750488 15.131290435791\\
57.9248466491699 15.1324167251587\\
58.024845123291 15.1335458755493\\
58.1248474121094 15.1346731185913\\
58.2248458862305 15.1358022689819\\
58.3248481750488 15.1369295120239\\
58.4248466491699 15.1380586624146\\
58.524845123291 15.1394472122192\\
58.6248474121094 15.1405992507935\\
58.7248458862305 15.1417226791382\\
58.8248481750488 15.1428527832031\\
58.9248466491699 15.1439809799194\\
59.024845123291 15.1451091766357\\
59.1248474121094 15.1462383270264\\
59.2248458862305 15.1473665237427\\
59.3248481750488 15.148494720459\\
59.4248466491699 15.1496229171753\\
59.524845123291 15.1507520675659\\
59.6248474121094 15.1518802642822\\
59.7248458862305 15.1530084609985\\
59.8248481750488 15.1541366577148\\
59.9124221801758 15.1552743911743\\
60 15.1562433242798\\
60.0963325500488 15.1583948135376\\
60.1963348388672 15.1584644317627\\
60.2963333129883 15.1591835021973\\
60.3963356018066 15.1604099273682\\
60.4963340759277 15.1614933013916\\
60.5963325500488 15.1626310348511\\
60.6963348388672 15.1637477874756\\
60.7963333129883 15.1648712158203\\
60.8963356018066 15.1659927368164\\
60.9963340759277 15.1671142578125\\
61.0963325500488 15.1682500839233\\
61.1963348388672 15.1694021224976\\
61.2963333129883 15.1706085205078\\
61.3963356018066 15.1717510223389\\
61.4963340759277 15.1728992462158\\
61.5963325500488 15.1740713119507\\
61.6963348388672 15.1752653121948\\
61.7963333129883 15.1764984130859\\
61.8963356018066 15.1797571182251\\
61.9940414428711 15.1856994628906\\
62.0722236633301 15.1898422241211\\
62.1347732543945 15.1805095672607\\
62.2035751342773 15.1982555389404\\
62.2478866577148 15.1557388305664\\
62.296630859375 15.1706457138062\\
62.3256530761719 15.1918334960938\\
62.3496742248535 15.3321781158447\\
62.376091003418 15.0602235794067\\
62.4051551818848 15.1797170639038\\
62.4213905334473 15.187388420105\\
62.4340591430664 15.1877546310425\\
62.4479942321777 15.1843690872192\\
62.463321685791 15.1843900680542\\
62.4801826477051 15.1845111846924\\
62.4987297058105 15.1846332550049\\
62.5191307067871 15.1847791671753\\
62.5415687561035 15.1849517822266\\
62.5662574768066 15.1851539611816\\
62.5934104919434 15.1853885650635\\
62.6232795715332 15.1856594085693\\
62.656135559082 15.1859693527222\\
62.6922760009766 15.1863203048706\\
62.7320327758789 15.1867170333862\\
62.775764465332 15.1871604919434\\
62.8238677978516 15.1876525878906\\
62.8767852783203 15.1881971359253\\
62.9349899291992 15.1887969970703\\
62.9990196228027 15.1894559860229\\
63.069450378418 15.1901807785034\\
63.1469230651855 15.1909770965576\\
63.2321434020996 15.2012567520142\\
63.3258857727051 15.1839656829834\\
63.4258842468262 15.1939897537231\\
63.5258865356445 15.1994380950928\\
63.6258850097656 15.1999664306641\\
63.6758842468262 15.2432832717896\\
63.7138481140137 15.1929655075073\\
63.7273941040039 15.1611614227295\\
63.7422943115234 15.1784954071045\\
63.7586822509766 15.1800727844238\\
63.7767105102539 15.1584548950195\\
63.7965431213379 15.1632432937622\\
63.8183555603027 15.1798334121704\\
63.8423538208008 15.1834554672241\\
63.8687477111816 15.1802520751953\\
63.8977813720703 15.1795530319214\\
63.9297218322754 15.1765232086182\\
63.9648513793945 15.172022819519\\
64.0015869140625 15.1726913452148\\
64.022705078125 15.1782808303833\\
64.0459365844727 15.1949596405029\\
64.0714874267578 15.2004804611206\\
64.0996017456055 15.2078657150269\\
64.1305236816406 15.2162895202637\\
64.1645355224609 15.2150630950928\\
64.2019424438477 15.2106037139893\\
64.2431030273438 15.2095184326172\\
64.2883682250977 15.209246635437\\
64.3381729125977 15.209020614624\\
64.3929443359375 15.2090339660645\\
64.4477233886719 15.209981918335\\
64.4993057250977 15.2103309631348\\
64.5481796264648 15.2105007171631\\
64.5940780639648 15.2107782363892\\
64.6370315551758 15.212797164917\\
64.6842803955078 15.2132768630981\\
64.7362518310547 15.2117891311646\\
64.7934188842773 15.2098579406738\\
64.8563079833984 15.2101860046387\\
64.9254837036133 15.2107448577881\\
65.0015716552734 15.2112731933594\\
65.0852737426758 15.2120161056519\\
65.1773452758789 15.2148790359497\\
65.27734375 15.2118148803711\\
65.3773422241211 15.217173576355\\
65.4773483276367 15.2137603759766\\
65.5773468017578 15.2160024642944\\
65.5794677734375 15.3361549377441\\
65.5817947387695 15.2058362960815\\
65.5843658447266 15.1875171661377\\
65.5871810913086 15.2799234390259\\
65.5902862548828 15.1818199157715\\
65.5937042236328 15.1881170272827\\
65.5974578857422 15.227765083313\\
65.6015930175781 15.2298183441162\\
65.6061325073242 15.2200059890747\\
65.6111373901367 15.2190752029419\\
65.6166305541992 15.2194118499756\\
65.6226806640625 15.2194538116455\\
65.6293334960938 15.2194480895996\\
65.6366577148438 15.2194538116455\\
65.6447067260742 15.2194690704346\\
65.653564453125 15.2194938659668\\
65.6633071899414 15.2195291519165\\
65.6740188598633 15.219575881958\\
65.6858062744141 15.2196378707886\\
65.6987762451172 15.2197151184082\\
65.7130355834961 15.2198104858398\\
65.7287292480469 15.2199249267578\\
65.7459869384766 15.2200603485107\\
65.7649688720703 15.2202196121216\\
65.7858505249023 15.220404624939\\
65.8088226318359 15.2206153869629\\
65.8340911865234 15.2208557128906\\
65.8618850708008 15.2211265563965\\
65.892463684082 15.2214298248291\\
65.9260940551758 15.2217674255371\\
65.9630889892578 15.2221422195435\\
66.0037841796875 15.2225570678711\\
66.0485458374023 15.223014831543\\
66.097785949707 15.223518371582\\
66.1519470214844 15.2240734100342\\
66.2115325927734 15.224684715271\\
66.2770690917969 15.2253570556641\\
66.3491592407227 15.2260961532593\\
66.428466796875 15.2269096374512\\
66.5156936645508 15.2296752929688\\
66.6116485595703 15.2268133163452\\
66.7116470336914 15.2299671173096\\
66.811653137207 15.2393445968628\\
66.8316497802734 15.2513904571533\\
66.8391723632813 15.2814321517944\\
66.8464965820313 15.2861518859863\\
66.8545455932617 15.2822303771973\\
66.8622512817383 15.2785263061523\\
66.870735168457 15.2937593460083\\
66.8800659179688 15.3371496200562\\
66.8903274536133 15.3540134429932\\
66.9016189575195 15.3395051956177\\
66.9140319824219 15.3151388168335\\
66.9276962280273 15.2941694259644\\
66.9427185058594 15.270938873291\\
66.9592437744141 15.2459526062012\\
66.977424621582 15.219783782959\\
66.997428894043 15.180606842041\\
67.0194244384766 15.154088973999\\
67.0415725708008 15.1369504928589\\
67.0659408569336 15.1298866271973\\
67.0927429199219 15.1302528381348\\
67.1222229003906 15.1601886749268\\
67.154655456543 15.2067985534668\\
67.1903305053711 15.2832460403442\\
67.1995849609375 15.2770853042603\\
67.2097702026367 15.260082244873\\
67.2209701538086 15.2515735626221\\
67.233283996582 15.2481851577759\\
67.2468414306641 15.2468347549438\\
67.2617416381836 15.2463283538818\\
67.2781448364258 15.2463579177856\\
67.2961807250977 15.2471389770508\\
67.3160247802734 15.2479000091553\\
67.3378448486328 15.2431659698486\\
67.3618545532227 15.2461376190186\\
67.3882598876953 15.2437105178833\\
67.4173126220703 15.2449979782104\\
67.4492645263672 15.2441091537476\\
67.4844131469727 15.241644859314\\
67.523078918457 15.2398195266724\\
67.5656127929688 15.2400712966919\\
67.6123962402344 15.2388439178467\\
67.6579818725586 15.2366561889648\\
67.6993789672852 15.2350511550903\\
67.7355346679688 15.2368221282959\\
67.7753067016602 15.2397766113281\\
67.8190612792969 15.2418775558472\\
67.8671875 15.2425336837769\\
67.9201278686523 15.2445850372314\\
67.9783630371094 15.2447967529297\\
68.0421829223633 15.2472085952759\\
68.1070251464844 15.2499599456787\\
68.178352355957 15.2533721923828\\
68.2568130493164 15.256555557251\\
68.3431243896484 15.2541465759277\\
68.4380569458008 15.2535820007324\\
68.5380630493164 15.253116607666\\
68.6380615234375 15.2527008056641\\
68.7380599975586 15.2524404525757\\
68.8380584716797 15.2535133361816\\
68.9380569458008 15.2544803619385\\
69.0380630493164 15.2558641433716\\
69.1380615234375 15.2568206787109\\
69.2380599975586 15.2580595016479\\
69.3380584716797 15.2591123580933\\
69.4380569458008 15.2602891921997\\
69.5380630493164 15.2613849639893\\
69.6380615234375 15.2625341415405\\
69.7380599975586 15.2636480331421\\
69.8380584716797 15.2647857666016\\
69.9380569458008 15.265923500061\\
70.0380630493164 15.2670364379883\\
70.1380615234375 15.268162727356\\
70.2380599975586 15.2692928314209\\
70.3380584716797 15.2704200744629\\
70.4380569458008 15.2715492248535\\
70.5380630493164 15.2726774215698\\
70.6380615234375 15.2738056182861\\
70.7380599975586 15.2749338150024\\
70.8380584716797 15.2760620117188\\
70.9380569458008 15.2773408889771\\
71.0380630493164 15.2786331176758\\
71.1380615234375 15.2797193527222\\
71.2380599975586 15.2808609008789\\
71.3380584716797 15.2819852828979\\
71.4380569458008 15.2831153869629\\
71.5380630493164 15.2842435836792\\
71.6380615234375 15.2853727340698\\
71.7380599975586 15.2865009307861\\
71.8380584716797 15.2876300811768\\
71.9380569458008 15.2887582778931\\
72.0380630493164 15.2898874282837\\
72.1380615234375 15.291015625\\
72.2380599975586 15.2921438217163\\
72.3380584716797 15.2932729721069\\
72.4380569458008 15.2944002151489\\
72.5380630493164 15.2966642379761\\
72.6380615234375 15.2967672348022\\
72.7380599975586 15.2975215911865\\
72.8380584716797 15.2987375259399\\
72.9380569458008 15.299825668335\\
73.0380630493164 15.3009614944458\\
73.1380615234375 15.3020792007446\\
73.2380599975586 15.3032026290894\\
73.3380584716797 15.3043241500854\\
73.4380569458008 15.3054456710815\\
73.5380630493164 15.3065814971924\\
73.6380615234375 15.3077344894409\\
73.7380599975586 15.3089408874512\\
73.8380584716797 15.3100833892822\\
73.9380569458008 15.3112316131592\\
74.0380630493164 15.312403678894\\
74.1380615234375 15.3135967254639\\
74.2380599975586 15.3148288726807\\
74.3380584716797 15.3178853988647\\
74.4363632202148 15.3239889144897\\
74.5150299072266 15.328106880188\\
74.5779571533203 15.3192110061646\\
74.6471862792969 15.3195676803589\\
74.7233276367188 15.3201036453247\\
74.8070907592773 15.3207378387451\\
74.8992309570313 15.3373775482178\\
74.945930480957 15.2916307449341\\
74.9729614257813 15.3154029846191\\
74.9864807128906 15.3312788009644\\
75 15.3322019577026\\
};
\addlegendentry{6 "};

\addplot [
color=black,
dashed
]
table[row sep=crcr]{
0 14.700608253479\\
9.99999971718069e-10 14.700608253479\\
2.49999998480632e-09 14.700608253479\\
4.74999994892755e-09 14.700608253479\\
8.1250002281763e-09 14.700608253479\\
1.31874999809156e-08 14.700608253479\\
2.0781250498203e-08 14.700608253479\\
3.21718758300449e-08 14.700608253479\\
4.92578138278077e-08 14.700608253479\\
7.48867208244519e-08 14.700608253479\\
1.13330081319418e-07 14.700608253479\\
1.70995122061868e-07 14.700608253479\\
2.57492672517401e-07 14.700608253479\\
3.8723902662241e-07 14.700608253479\\
5.81858500936505e-07 14.700608253479\\
8.73787769251066e-07 14.700608253479\\
1.31168167172291e-06 14.700608253479\\
1.9685226106958e-06 14.700608253479\\
2.95378367809462e-06 14.700608253479\\
4.43167573394021e-06 14.700608253479\\
6.64851359033491e-06 14.7006092071533\\
9.97376992017962e-06 14.7006092071533\\
1.49616553244414e-05 14.7006092071533\\
2.24434825213393e-05 14.7006101608276\\
3.36662233166862e-05 14.700611114502\\
5.05003372381907e-05 14.7006120681763\\
7.57515081204474e-05 14.7006130218506\\
0.000113628258986864 14.7006149291992\\
0.000170443381648511 14.7006158828735\\
0.000255666091106832 14.7006177902222\\
0.000383500126190484 14.7006187438965\\
0.000575251178815961 14.7006187438965\\
0.000862877757754177 14.7006168365479\\
0.00129431765526533 14.7006139755249\\
0.00194147753063589 14.7006101608276\\
0.00291221728548408 14.7006063461304\\
0.00436832662671804 14.7006044387817\\
0.00597004732117057 14.7006063461304\\
0.00773194013163447 14.700608253479\\
0.009670021943748 14.7006092071533\\
0.0118019115179777 14.7006092071533\\
0.0141469910740852 14.7006092071533\\
0.0167265776544809 14.700608253479\\
0.0195641238242388 14.700608253479\\
0.0226854234933853 14.700608253479\\
0.0261188540607691 14.700608253479\\
0.0298956278711557 14.700608253479\\
0.0340500771999359 14.700608253479\\
0.0386199727654457 14.700608253479\\
0.0436468608677387 14.700608253479\\
0.0491764321923256 14.700608253479\\
0.0552589632570744 14.700608253479\\
0.0619497485458851 14.700608253479\\
0.0693096145987511 14.700608253479\\
0.0774054601788521 14.700608253479\\
0.0863108932971954 14.7006072998047\\
0.0961068719625473 14.7006092071533\\
0.106882445514202 14.7006092071533\\
0.118735581636429 14.7006092071533\\
0.13177402317524 14.7006092071533\\
0.146116316318512 14.7006092071533\\
0.161892831325531 14.7006092071533\\
0.179247006773949 14.7006092071533\\
0.198336601257324 14.7006092071533\\
0.21933513879776 14.7006092071533\\
0.242433547973633 14.700608253479\\
0.267841786146164 14.7006092071533\\
0.29579085111618 14.7006092071533\\
0.32653483748436 14.7006092071533\\
0.360353201627731 14.7006092071533\\
0.397553414106369 14.7006092071533\\
0.438473641872406 14.7006092071533\\
0.483485877513886 14.7006092071533\\
0.532999336719513 14.7006092071533\\
0.587464153766632 14.7006092071533\\
0.647375464439392 14.7006092071533\\
0.71327793598175 14.7006092071533\\
0.7857705950737 14.7006092071533\\
0.865512549877167 14.7006092071533\\
0.953228712081909 14.7006092071533\\
1.04971647262573 14.700608253479\\
1.14971649646759 14.700608253479\\
1.24971640110016 14.700608253479\\
1.34971642494202 14.700608253479\\
1.44971644878387 14.700608253479\\
1.54971647262573 14.700608253479\\
1.64971649646759 14.700608253479\\
1.74971640110016 14.700608253479\\
1.84971642494202 14.700608253479\\
1.94971644878387 14.700608253479\\
2.04971647262573 14.700608253479\\
2.1497163772583 14.700608253479\\
2.24971652030945 14.700608253479\\
2.34971642494202 14.700608253479\\
2.44971656799316 14.700608253479\\
2.54971647262573 14.700608253479\\
2.6497163772583 14.700608253479\\
2.74971652030945 14.700608253479\\
2.84971642494202 14.700608253479\\
2.94971656799316 14.700608253479\\
3.04971647262573 14.700608253479\\
3.1497163772583 14.700608253479\\
3.24971652030945 14.700608253479\\
3.34971642494202 14.700608253479\\
3.44971656799316 14.700608253479\\
3.54971647262573 14.700608253479\\
3.6497163772583 14.700608253479\\
3.74971652030945 14.700608253479\\
3.84971642494202 14.700608253479\\
3.94971656799316 14.700608253479\\
4.04971647262573 14.700608253479\\
4.1497163772583 14.700608253479\\
4.24971628189087 14.700608253479\\
4.3497166633606 14.700608253479\\
4.44971656799316 14.700608253479\\
4.54971647262573 14.700608253479\\
4.6497163772583 14.700608253479\\
4.74971628189087 14.700608253479\\
4.8497166633606 14.700608253479\\
4.94971656799316 14.700608253479\\
5.04971647262573 14.700608253479\\
5.1497163772583 14.700608253479\\
5.24971628189087 14.700608253479\\
5.3497166633606 14.700608253479\\
5.44971656799316 14.700608253479\\
5.54971647262573 14.700608253479\\
5.6497163772583 14.700608253479\\
5.74971628189087 14.700608253479\\
5.8497166633606 14.700608253479\\
5.94971656799316 14.700608253479\\
6.04971647262573 14.700608253479\\
6.1497163772583 14.7006072998047\\
6.24971628189087 14.7006063461304\\
6.3497166633606 14.7006053924561\\
6.44971656799316 14.7006072998047\\
6.54971647262573 14.7006092071533\\
6.6497163772583 14.7006101608276\\
6.74971628189087 14.7006101608276\\
6.8497166633606 14.700608253479\\
6.94971656799316 14.700608253479\\
7.04971647262573 14.700608253479\\
7.1497163772583 14.700608253479\\
7.24971628189087 14.700608253479\\
7.3497166633606 14.700608253479\\
7.44971656799316 14.700608253479\\
7.54971647262573 14.700608253479\\
7.6497163772583 14.700608253479\\
7.74971628189087 14.700608253479\\
7.8497166633606 14.700608253479\\
7.94971656799316 14.700608253479\\
8.04971599578857 14.700608253479\\
8.1497163772583 14.700608253479\\
8.24971675872803 14.700608253479\\
8.34971618652344 14.700608253479\\
8.44971656799316 14.700608253479\\
8.54971599578857 14.700608253479\\
8.6497163772583 14.700608253479\\
8.74971675872803 14.700608253479\\
8.84971618652344 14.700608253479\\
8.94971656799316 14.700608253479\\
9.04971599578857 14.700608253479\\
9.1497163772583 14.700608253479\\
9.24971675872803 14.700608253479\\
9.34971618652344 14.700608253479\\
9.44971656799316 14.700608253479\\
9.54971599578857 14.700608253479\\
9.6497163772583 14.700608253479\\
9.74971675872803 14.700608253479\\
9.84971618652344 14.700608253479\\
9.94971656799316 14.700608253479\\
10.0497159957886 14.700608253479\\
10.1497163772583 14.700608253479\\
10.249716758728 14.700608253479\\
10.3497161865234 14.700608253479\\
10.4497165679932 14.700608253479\\
10.5497159957886 14.700608253479\\
10.6497163772583 14.700608253479\\
10.749716758728 14.700608253479\\
10.8497161865234 14.700608253479\\
10.9497165679932 14.700608253479\\
11.0497159957886 14.700608253479\\
11.1497163772583 14.700608253479\\
11.249716758728 14.700608253479\\
11.3497161865234 14.700608253479\\
11.4497165679932 14.700608253479\\
11.5497159957886 14.700608253479\\
11.6497163772583 14.700608253479\\
11.749716758728 14.700608253479\\
11.8497161865234 14.700608253479\\
11.9497165679932 14.700608253479\\
12.0497159957886 14.700608253479\\
12.1497163772583 14.700608253479\\
12.249716758728 14.700608253479\\
12.347146987915 14.700608253479\\
12.4250946044922 14.7006092071533\\
12.4874591827393 14.7006101608276\\
12.5560598373413 14.700608253479\\
12.6315202713013 14.7006072998047\\
12.714527130127 14.700608253479\\
12.8058338165283 14.700608253479\\
12.8497476577759 14.7006101608276\\
12.8980522155762 14.7006072998047\\
12.9143037796021 14.700608253479\\
12.9277105331421 14.7006101608276\\
12.9424571990967 14.7006092071533\\
12.9586782455444 14.7006092071533\\
12.9765224456787 14.700608253479\\
12.9961500167847 14.700608253479\\
13.0177412033081 14.700608253479\\
13.0414915084839 14.700608253479\\
13.0676164627075 14.700608253479\\
13.0963535308838 14.700608253479\\
13.1279649734497 14.700608253479\\
13.1627368927002 14.700608253479\\
13.2009868621826 14.700608253479\\
13.2430610656738 14.700608253479\\
13.289342880249 14.700608253479\\
13.3402528762817 14.700608253479\\
13.3962545394897 14.700608253479\\
13.4578561782837 14.700608253479\\
13.5256175994873 14.700608253479\\
13.600154876709 14.700608253479\\
13.6821460723877 14.7006072998047\\
13.7723369598389 14.700608253479\\
13.871545791626 14.700608253479\\
13.9715461730957 14.7006368637085\\
14.0715456008911 14.7007579803467\\
14.1215467453003 14.7013540267944\\
14.1629066467285 14.7011709213257\\
14.2008075714111 14.7023859024048\\
14.202446937561 14.6913003921509\\
14.2042503356934 14.6935138702393\\
14.2062339782715 14.6986989974976\\
14.2084159851074 14.7013282775879\\
14.2108154296875 14.7016487121582\\
14.2134561538696 14.7021751403809\\
14.2163600921631 14.7022514343262\\
14.219554901123 14.7013835906982\\
14.223069190979 14.7017230987549\\
14.2269344329834 14.7015571594238\\
14.2311868667603 14.7012205123901\\
14.2358636856079 14.700927734375\\
14.2410087585449 14.7007541656494\\
14.2466678619385 14.7006578445435\\
14.252893447876 14.7005815505981\\
14.2597417831421 14.7005033493042\\
14.2672748565674 14.7004308700562\\
14.2755603790283 14.7003726959229\\
14.2846755981445 14.7003297805786\\
14.2947015762329 14.7003002166748\\
14.3057298660278 14.700288772583\\
14.3178615570068 14.7002992630005\\
14.3312063217163 14.700493812561\\
14.3458862304688 14.7005834579468\\
14.3620328903198 14.7005338668823\\
14.3797950744629 14.7004613876343\\
14.3993330001831 14.7004232406616\\
14.4208250045776 14.7004270553589\\
14.444465637207 14.7004508972168\\
14.4704713821411 14.7006902694702\\
14.4990768432617 14.7006578445435\\
14.5269069671631 14.7005643844604\\
14.5575199127197 14.700590133667\\
14.591194152832 14.7006015777588\\
14.6282358169556 14.7006092071533\\
14.668981552124 14.7006120681763\\
14.7138023376465 14.7006149291992\\
14.7631044387817 14.7006072998047\\
14.8173370361328 14.7006072998047\\
14.8769931793213 14.7006072998047\\
14.9244384765625 14.700611114502\\
14.971435546875 14.7006101608276\\
15.0163955688477 14.7006092071533\\
15.0584926605225 14.7006092071533\\
15.09974193573 14.700608253479\\
15.1451168060303 14.700608253479\\
15.1950283050537 14.7006092071533\\
15.2499313354492 14.700608253479\\
15.31032371521 14.7006092071533\\
15.3767566680908 14.700608253479\\
15.4498319625854 14.700608253479\\
15.5302152633667 14.700608253479\\
15.6186370849609 14.7006072998047\\
15.7159004211426 14.7006092071533\\
15.8159008026123 14.7006072998047\\
15.9159002304077 14.7006092071533\\
16.0159015655518 14.7006072998047\\
16.0210952758789 14.7006053924561\\
16.0268077850342 14.7006139755249\\
16.0330924987793 14.7006053924561\\
16.0382099151611 14.7006092071533\\
16.0438404083252 14.7006101608276\\
16.0500316619873 14.7006101608276\\
16.0568428039551 14.7006092071533\\
16.0643348693848 14.7006092071533\\
16.0725765228271 14.7006092071533\\
16.0816440582275 14.7006092071533\\
16.0916156768799 14.7006092071533\\
16.1025848388672 14.7006092071533\\
16.114652633667 14.700608253479\\
16.1279258728027 14.700608253479\\
16.1425266265869 14.700608253479\\
16.1585865020752 14.700608253479\\
16.1762542724609 14.700608253479\\
16.1956882476807 14.700608253479\\
16.2170658111572 14.700608253479\\
16.2405796051025 14.700608253479\\
16.2664451599121 14.700608253479\\
16.2948989868164 14.700608253479\\
16.3261966705322 14.700608253479\\
16.3606243133545 14.700608253479\\
16.398494720459 14.700608253479\\
16.4401531219482 14.700608253479\\
16.4859771728516 14.700608253479\\
16.5363826751709 14.700608253479\\
16.5918292999268 14.700608253479\\
16.6528205871582 14.700608253479\\
16.7199096679688 14.700608253479\\
16.7937088012695 14.700608253479\\
16.8748874664307 14.700608253479\\
16.9641857147217 14.7006072998047\\
17.0624122619629 14.7006092071533\\
17.1624126434326 14.7006072998047\\
17.2624130249023 14.700647354126\\
17.282413482666 14.7007713317871\\
17.2906188964844 14.7008666992188\\
17.2996425628662 14.7009029388428\\
17.308614730835 14.7009201049805\\
17.3184852600098 14.7009181976318\\
17.3293399810791 14.7008771896362\\
17.3412818908691 14.7008152008057\\
17.3544178009033 14.700119972229\\
17.3688678741455 14.6993179321289\\
17.3847618103027 14.7011165618896\\
17.4022445678711 14.7006988525391\\
17.4214763641357 14.7009925842285\\
17.4426326751709 14.7009172439575\\
17.4659042358398 14.7008438110352\\
17.4915008544922 14.700701713562\\
17.5196590423584 14.7005176544189\\
17.5506324768066 14.7004423141479\\
17.5847034454346 14.7004041671753\\
17.6221809387207 14.7004194259644\\
17.6634063720703 14.7004652023315\\
17.7087554931641 14.7005195617676\\
17.758638381958 14.7006034851074\\
17.8135089874268 14.7006072998047\\
17.8738670349121 14.7006187438965\\
17.9402618408203 14.700611114502\\
18.0132961273193 14.700611114502\\
18.0585689544678 14.7006206512451\\
18.1083717346191 14.7006187438965\\
18.1509475708008 14.7006216049194\\
18.1891002655029 14.700611114502\\
18.2310676574707 14.7005958557129\\
18.2772312164307 14.7005939483643\\
18.328010559082 14.7005977630615\\
18.3838691711426 14.7006130218506\\
18.4453125 14.7006072998047\\
18.5129013061523 14.700608253479\\
18.5872478485107 14.700608253479\\
18.6690311431885 14.700608253479\\
18.7589912414551 14.700608253479\\
18.857946395874 14.7006092071533\\
18.9579467773438 14.700608253479\\
19.0579471588135 14.700608253479\\
19.1579456329346 14.700608253479\\
19.2579460144043 14.700608253479\\
19.357946395874 14.7006092071533\\
19.4579467773438 14.700608253479\\
19.5579471588135 14.700608253479\\
19.6579456329346 14.700608253479\\
19.7579460144043 14.700608253479\\
19.857946395874 14.700608253479\\
19.9579467773438 14.700608253479\\
20.0579471588135 14.700608253479\\
20.1579456329346 14.700608253479\\
20.2579460144043 14.700608253479\\
20.357946395874 14.700608253479\\
20.4579467773438 14.700608253479\\
20.5579471588135 14.700608253479\\
20.6579456329346 14.700608253479\\
20.7579460144043 14.700608253479\\
20.857946395874 14.700608253479\\
20.9579467773438 14.700608253479\\
21.0579471588135 14.700608253479\\
21.1579456329346 14.700608253479\\
21.2579460144043 14.700608253479\\
21.357946395874 14.700608253479\\
21.4579467773438 14.700608253479\\
21.5579471588135 14.700608253479\\
21.6579456329346 14.700608253479\\
21.7579460144043 14.700608253479\\
21.857946395874 14.700608253479\\
21.9579467773438 14.700608253479\\
22.0579471588135 14.700608253479\\
22.1579456329346 14.700608253479\\
22.2579460144043 14.700608253479\\
22.357946395874 14.700608253479\\
22.4579467773438 14.700608253479\\
22.5579471588135 14.700608253479\\
22.6579456329346 14.700608253479\\
22.7579460144043 14.700608253479\\
22.857946395874 14.700608253479\\
22.9579467773438 14.700608253479\\
23.0579471588135 14.700608253479\\
23.1579456329346 14.700608253479\\
23.2579460144043 14.700608253479\\
23.357946395874 14.700608253479\\
23.4579467773438 14.700608253479\\
23.5579471588135 14.700608253479\\
23.6579456329346 14.700608253479\\
23.7579460144043 14.700608253479\\
23.857946395874 14.700608253479\\
23.9579467773438 14.700608253479\\
24.0579471588135 14.700608253479\\
24.1579456329346 14.700608253479\\
24.2579460144043 14.700608253479\\
24.357946395874 14.700608253479\\
24.4579467773438 14.700608253479\\
24.5579471588135 14.7006072998047\\
24.6579456329346 14.7006072998047\\
24.7421741485596 14.7006072998047\\
24.8095417022705 14.700608253479\\
24.8836441040039 14.7006072998047\\
24.9651584625244 14.7006072998047\\
25.0548248291016 14.7006072998047\\
25.0960559844971 14.700608253479\\
25.1414089202881 14.7006025314331\\
25.1613826751709 14.7006225585938\\
25.1780624389648 14.7006130218506\\
25.1964111328125 14.7006101608276\\
25.2165927886963 14.7006092071533\\
25.2387943267822 14.7006092071533\\
25.2632160186768 14.7006092071533\\
25.2900791168213 14.7006092071533\\
25.3196296691895 14.7006092071533\\
25.3521347045898 14.7006092071533\\
25.3878898620605 14.7006092071533\\
25.4272212982178 14.7006092071533\\
25.4704837799072 14.7006092071533\\
25.5180740356445 14.7006092071533\\
25.5704231262207 14.7006092071533\\
25.6280078887939 14.7006092071533\\
25.6913509368896 14.7006092071533\\
25.7610282897949 14.7006092071533\\
25.8376731872559 14.7006092071533\\
25.9219818115234 14.700608253479\\
26.014720916748 14.7006101608276\\
26.1147212982178 14.700608253479\\
26.2147216796875 14.7006273269653\\
26.3147220611572 14.7008085250854\\
26.3647212982178 14.7019119262695\\
26.4022121429443 14.7718486785889\\
26.4049663543701 14.8119611740112\\
26.407995223999 14.8233222961426\\
26.4113292694092 14.8253383636475\\
26.4149932861328 14.8145122528076\\
26.4190254211426 14.8154487609863\\
26.4234619140625 14.8059749603271\\
26.4283409118652 14.7943620681763\\
26.4337062835693 14.7801971435547\\
26.4396114349365 14.7647943496704\\
26.4461040496826 14.7497911453247\\
26.4532470703125 14.7358055114746\\
26.4611053466797 14.7235813140869\\
26.469747543335 14.714319229126\\
26.4792556762695 14.7066612243652\\
26.4897136688232 14.7017517089844\\
26.5012187957764 14.7007818222046\\
26.5138721466064 14.7003793716431\\
26.5277919769287 14.7002058029175\\
26.5431041717529 14.7002048492432\\
26.5599479675293 14.7002019882202\\
26.5784759521484 14.7001724243164\\
26.59885597229 14.7006320953369\\
26.6212730407715 14.7006416320801\\
26.6459331512451 14.7006597518921\\
26.673059463501 14.7006225585938\\
26.7028980255127 14.7006015777588\\
26.7357215881348 14.7006340026855\\
26.7718257904053 14.7006464004517\\
26.8115406036377 14.7006464004517\\
26.8552284240723 14.7006568908691\\
26.903284072876 14.700647354126\\
26.9561443328857 14.7006387710571\\
27.0142917633057 14.7006359100342\\
27.0605869293213 14.7006397247314\\
27.1115131378174 14.7006244659424\\
27.1556930541992 14.7006216049194\\
27.1975421905518 14.7006139755249\\
27.2435741424561 14.7006196975708\\
27.29421043396 14.7006187438965\\
27.3499088287354 14.7006168365479\\
27.4111785888672 14.7006168365479\\
27.4785747528076 14.7006158828735\\
27.5527095794678 14.7006158828735\\
27.6342601776123 14.7006149291992\\
27.7239646911621 14.7006139755249\\
27.8226375579834 14.7006149291992\\
27.9226379394531 14.7006130218506\\
28.0226383209229 14.7006139755249\\
28.1226387023926 14.7006130218506\\
28.1260833740234 14.7006034851074\\
28.1298713684082 14.7006311416626\\
28.1340389251709 14.7006044387817\\
28.1376724243164 14.7006072998047\\
28.1416683197021 14.700626373291\\
28.1460647583008 14.7006063461304\\
28.1490650177002 14.7006139755249\\
28.1523666381836 14.7006168365479\\
28.1559963226318 14.7006149291992\\
28.1599903106689 14.7006139755249\\
28.1643829345703 14.7006149291992\\
28.1692161560059 14.7006139755249\\
28.1745319366455 14.7006139755249\\
28.1803779602051 14.7006139755249\\
28.1868114471436 14.7006139755249\\
28.1938858032227 14.7006139755249\\
28.2016677856445 14.7006139755249\\
28.2102298736572 14.7006139755249\\
28.2196464538574 14.7006139755249\\
28.2300052642822 14.7006139755249\\
28.2413997650146 14.7006139755249\\
28.2539348602295 14.7006130218506\\
28.2677211761475 14.7006130218506\\
28.2828884124756 14.7006130218506\\
28.2995700836182 14.7006130218506\\
28.3179225921631 14.7006130218506\\
28.3381080627441 14.7006130218506\\
28.3603134155273 14.7006130218506\\
28.3847389221191 14.7006130218506\\
28.4116058349609 14.7006130218506\\
28.4411602020264 14.7006130218506\\
28.4736709594727 14.7006130218506\\
28.5094318389893 14.7006130218506\\
28.5487689971924 14.7006130218506\\
28.592041015625 14.7006130218506\\
28.6396389007568 14.7006130218506\\
28.6919975280762 14.7006130218506\\
28.7495918273926 14.7006130218506\\
28.8129444122314 14.7006130218506\\
28.8826332092285 14.7006130218506\\
28.9592895507813 14.7006130218506\\
29.0436134338379 14.7006120681763\\
29.1363677978516 14.700611114502\\
29.2363681793213 14.7006130218506\\
29.336368560791 14.700611114502\\
29.4363689422607 14.7006616592407\\
29.4563694000244 14.7008714675903\\
29.4642658233643 14.7011051177979\\
29.471960067749 14.7012243270874\\
29.4804267883301 14.701301574707\\
29.4885120391846 14.7013816833496\\
29.497407913208 14.7015314102173\\
29.5071926116943 14.7072830200195\\
29.5179557800293 14.7878923416138\\
29.5297946929932 14.8606586456299\\
29.542818069458 14.8496875762939\\
29.5571441650391 14.7831888198853\\
29.5729026794434 14.7781105041504\\
29.5902366638184 14.7678632736206\\
29.6093044281006 14.7566080093384\\
29.630277633667 14.7458744049072\\
29.6533489227295 14.7308502197266\\
29.6787281036377 14.7051200866699\\
29.6999015808105 14.7015686035156\\
29.7231941223145 14.7206888198853\\
29.7488136291504 14.7402534484863\\
29.7769947052002 14.7561521530151\\
29.8079967498779 14.7583999633789\\
29.8420963287354 14.7622289657593\\
29.8796062469482 14.756534576416\\
29.9208679199219 14.7543144226074\\
29.9662551879883 14.7282600402832\\
30.0161819458008 14.7518396377563\\
30.0711002349854 14.7315187454224\\
30.1315116882324 14.7306890487671\\
30.1804618835449 14.7237243652344\\
30.2286376953125 14.7165279388428\\
30.2746448516846 14.7108926773071\\
30.3156337738037 14.7051124572754\\
30.3508148193359 14.7023010253906\\
30.3895168304443 14.7014827728271\\
30.4320869445801 14.701265335083\\
30.4789142608643 14.7012939453125\\
30.530424118042 14.7014284133911\\
30.587085723877 14.7017469406128\\
30.6491241455078 14.7024850845337\\
30.7173671722412 14.7041082382202\\
30.7924346923828 14.7077054977417\\
30.8750076293945 14.7145433425903\\
30.9658393859863 14.7239513397217\\
31.0657539367676 14.7291631698608\\
31.1657543182373 14.7319641113281\\
31.265754699707 14.7328395843506\\
31.3657531738281 14.7328815460205\\
31.4657535552979 14.7326364517212\\
31.5657539367676 14.7332887649536\\
31.6657543182373 14.73362159729\\
31.765754699707 14.734356880188\\
31.8657531738281 14.7347278594971\\
31.9657535552979 14.7353401184082\\
32.0657539367676 14.7357931137085\\
32.1657524108887 14.7363510131836\\
32.265754699707 14.7368392944336\\
32.3657531738281 14.7373743057251\\
32.4657554626465 14.7378787994385\\
32.5657539367676 14.7384033203125\\
32.6657524108887 14.7389154434204\\
32.765754699707 14.7394399642944\\
32.8657531738281 14.7399568557739\\
32.9657554626465 14.7404823303223\\
33.0657539367676 14.7410039901733\\
33.1657524108887 14.7415285110474\\
33.265754699707 14.7420511245728\\
33.3657531738281 14.7425746917725\\
33.4657554626465 14.7430973052979\\
33.5657539367676 14.7436208724976\\
33.6657524108887 14.7444610595703\\
33.765754699707 14.7450542449951\\
33.8657531738281 14.7456912994385\\
33.9657554626465 14.7463121414185\\
34.0657539367676 14.7469358444214\\
34.1657524108887 14.7475576400757\\
34.265754699707 14.7481775283813\\
34.3657531738281 14.7487964630127\\
34.4657554626465 14.7494144439697\\
34.5657539367676 14.7500305175781\\
34.6657524108887 14.7506456375122\\
34.765754699707 14.751259803772\\
34.8657531738281 14.7518720626831\\
34.9657554626465 14.7524833679199\\
35.0657539367676 14.7530946731567\\
35.1657524108887 14.7535905838013\\
35.265754699707 14.7554092407227\\
35.3657531738281 14.7553234100342\\
35.4657554626465 14.7555627822876\\
35.5657539367676 14.7561845779419\\
35.6657524108887 14.7566947937012\\
35.765754699707 14.7572450637817\\
35.8657531738281 14.7577781677246\\
35.9657554626465 14.7583150863647\\
36.0657539367676 14.7588481903076\\
36.1657524108887 14.7593793869019\\
36.265754699707 14.7599430084229\\
36.3657531738281 14.7605028152466\\
36.4657554626465 14.761157989502\\
36.5657539367676 14.761736869812\\
36.6657524108887 14.7623310089111\\
36.765754699707 14.7629680633545\\
36.8657531738281 14.7636432647705\\
36.9657554626465 14.7643909454346\\
37.0657539367676 14.770245552063\\
37.1539344787598 14.7746934890747\\
37.2244758605957 14.7737627029419\\
37.302074432373 14.7683973312378\\
37.3874282836914 14.7687759399414\\
37.4813194274902 14.7852363586426\\
37.5215835571289 14.7328824996948\\
37.5658721923828 14.7641773223877\\
37.585823059082 14.7777004241943\\
37.6024284362793 14.7794485092163\\
37.6206932067871 14.7710819244385\\
37.6407852172852 14.7711410522461\\
37.6628875732422 14.7712821960449\\
37.687198638916 14.7714490890503\\
37.713939666748 14.7716541290283\\
37.7433547973633 14.7718982696533\\
37.7757110595703 14.7721843719482\\
37.8113059997559 14.7725143432617\\
37.8504600524902 14.7728910446167\\
37.8935279846191 14.7733163833618\\
37.9409027099609 14.7737913131714\\
37.9930152893066 14.7743186950684\\
38.0503387451172 14.77490234375\\
38.1133918762207 14.7755460739136\\
38.1827545166016 14.776252746582\\
38.2590522766113 14.7770299911499\\
38.3429794311523 14.787088394165\\
38.4352989196777 14.7700834274292\\
38.5352973937988 14.7799234390259\\
38.6352996826172 14.7833061218262\\
38.7352981567383 14.7888202667236\\
38.7853012084961 14.8490209579468\\
38.822696685791 14.8402452468872\\
38.8252944946289 14.8315048217773\\
38.8281555175781 14.8351354598999\\
38.8313026428223 14.836877822876\\
38.8347625732422 14.8348178863525\\
38.838565826416 14.8292722702026\\
38.8427543640137 14.8212518692017\\
38.8473587036133 14.8131685256958\\
38.852424621582 14.8066940307617\\
38.8579978942871 14.8009166717529\\
38.864128112793 14.7944526672363\\
38.8708724975586 14.7863969802856\\
38.878288269043 14.7744235992432\\
38.8864479064941 14.7582206726074\\
38.8954238891602 14.7366762161255\\
38.9052963256836 14.719554901123\\
38.9161567687988 14.7017278671265\\
38.9281005859375 14.7027750015259\\
38.9412422180176 14.7026071548462\\
38.955696105957 14.702974319458\\
38.9715957641602 14.7057390213013\\
38.989086151123 14.7133102416992\\
39.0083236694336 14.7705984115601\\
39.029483795166 14.816086769104\\
39.0527648925781 14.8317432403564\\
39.078369140625 14.8151168823242\\
39.1065368652344 14.8167133331299\\
39.137523651123 14.8195447921753\\
39.1716041564941 14.8183031082153\\
39.2090950012207 14.8141479492188\\
39.2477378845215 14.8095951080322\\
39.2902450561523 14.8061285018921\\
39.3370056152344 14.803394317627\\
39.3884391784668 14.8012762069702\\
39.4450187683105 14.8008728027344\\
39.4936180114746 14.800048828125\\
39.5406875610352 14.7990961074829\\
39.5853233337402 14.7984828948975\\
39.6272773742676 14.8002452850342\\
39.6734275817871 14.7998046875\\
39.7241897583008 14.7981824874878\\
39.780029296875 14.7956371307373\\
39.8414497375488 14.7956991195679\\
39.9090156555176 14.7960157394409\\
39.9833374023438 14.7963647842407\\
40.0650939941406 14.7969455718994\\
40.1550216674805 14.7998428344727\\
40.2539443969727 14.7963180541992\\
40.353946685791 14.8018598556519\\
40.4539451599121 14.7981081008911\\
40.5539436340332 14.8006324768066\\
40.5569686889648 14.9116926193237\\
40.5602951049805 14.7486228942871\\
40.5639533996582 14.7699823379517\\
40.5665168762207 14.8269119262695\\
40.5693359375 14.8209218978882\\
40.572437286377 14.8034687042236\\
40.5758476257324 14.8027648925781\\
40.5796012878418 14.8030767440796\\
40.5837287902832 14.8031091690063\\
40.5882682800293 14.8030986785889\\
40.59326171875 14.8030958175659\\
40.5987548828125 14.8030977249146\\
40.6048011779785 14.8031044006348\\
40.6114463806152 14.8031158447266\\
40.6187591552734 14.8031339645386\\
40.6268005371094 14.8031597137451\\
40.6356506347656 14.8031949996948\\
40.6453819274902 14.8032398223877\\
40.6560897827148 14.8032970428467\\
40.6678657531738 14.8033676147461\\
40.6808204650879 14.8034534454346\\
40.695068359375 14.8035554885864\\
40.7107429504395 14.8036766052246\\
40.7279853820801 14.8038187026978\\
40.7469482421875 14.8039827346802\\
40.7678108215332 14.8041715621948\\
40.7907600402832 14.8043851852417\\
40.8160057067871 14.8046274185181\\
40.8437728881836 14.8048992156982\\
40.8743171691895 14.8052034378052\\
40.9079170227051 14.8055410385132\\
40.9448776245117 14.8059158325195\\
40.9855308532715 14.8063287734985\\
41.0302505493164 14.8067855834961\\
41.0794448852539 14.8072881698608\\
41.1335563659668 14.8078413009644\\
41.1930809020996 14.8084506988525\\
41.2585563659668 14.8091201782227\\
41.3305778503418 14.8098573684692\\
41.4098052978516 14.8106679916382\\
41.4969520568848 14.8135185241699\\
41.5928153991699 14.810471534729\\
41.692813873291 14.8137235641479\\
41.7928161621094 14.8218679428101\\
41.812816619873 14.8248853683472\\
41.8206024169922 14.8512096405029\\
41.8281707763672 14.8550987243652\\
41.8364944458008 14.8533248901367\\
41.8444328308105 14.8513402938843\\
41.8531684875488 14.8519992828369\\
41.8627777099609 14.8541059494019\\
41.8733444213867 14.8790597915649\\
41.8849716186523 14.9155511856079\\
41.8977584838867 14.9205255508423\\
41.9118270874023 14.9010696411133\\
41.9272994995117 14.8827133178711\\
41.9443206787109 14.8632774353027\\
41.9630432128906 14.8437042236328\\
41.9836387634277 14.8245420455933\\
42.0062942504883 14.8015394210815\\
42.0312156677246 14.7743062973022\\
42.0477752685547 14.7639064788818\\
42.0659866333008 14.7660207748413\\
42.0860252380371 14.7777967453003\\
42.1080627441406 14.7975549697876\\
42.1323051452637 14.8054332733154\\
42.1589736938477 14.8388872146606\\
42.1883087158203 14.8449678421021\\
42.2205772399902 14.8446197509766\\
42.2560729980469 14.8375616073608\\
42.2951164245605 14.8386783599854\\
42.3380661010742 14.8430738449097\\
42.3853073120117 14.8413372039795\\
42.4372749328613 14.8359746932983\\
42.494441986084 14.8365621566772\\
42.5416946411133 14.824836730957\\
42.5936737060547 14.8252792358398\\
42.6379699707031 14.8220634460449\\
42.6777877807617 14.8231067657471\\
42.711311340332 14.8295707702637\\
42.7481842041016 14.8306617736816\\
42.788745880127 14.8303022384644\\
42.833366394043 14.8286962509155\\
42.8824462890625 14.8291282653809\\
42.9364318847656 14.8295612335205\\
42.9958190917969 14.8299245834351\\
43.0611457824707 14.8306007385254\\
43.1330032348633 14.83176612854\\
43.2120475769043 14.8340005874634\\
43.2989959716797 14.8380918502808\\
43.3946380615234 14.8424549102783\\
43.4946365356445 14.842303276062\\
43.5946388244629 14.8399333953857\\
43.694637298584 14.8391752243042\\
43.7946357727051 14.8386688232422\\
43.8946380615234 14.8385705947876\\
43.9946365356445 14.8399133682251\\
44.0946388244629 14.8408193588257\\
44.194637298584 14.8422241210938\\
44.2946357727051 14.8431415557861\\
44.3946380615234 14.8443813323975\\
44.4946365356445 14.8454084396362\\
44.5946388244629 14.8465776443481\\
44.694637298584 14.8476533889771\\
44.7946357727051 14.8487911224365\\
44.8946380615234 14.8498878479004\\
44.9946365356445 14.8510122299194\\
45.0946388244629 14.8521337509155\\
45.194637298584 14.8532409667969\\
45.2946357727051 14.8543586730957\\
45.3946380615234 14.8554811477661\\
45.4946365356445 14.8566007614136\\
45.5946388244629 14.8577222824097\\
45.694637298584 14.8588428497314\\
45.7946357727051 14.8599634170532\\
45.8946380615234 14.861083984375\\
45.9946365356445 14.8622055053711\\
46.0946388244629 14.8636274337769\\
46.194637298584 14.8647212982178\\
46.2946357727051 14.8658514022827\\
46.3946380615234 14.8669700622559\\
46.4946365356445 14.8680925369263\\
46.5946388244629 14.8692140579224\\
46.694637298584 14.8703355789185\\
46.7946357727051 14.8714570999146\\
46.8946380615234 14.8725786209106\\
46.9946365356445 14.8737001419067\\
47.0946388244629 14.8748216629028\\
47.194637298584 14.8759431838989\\
47.2946357727051 14.877064704895\\
47.3946380615234 14.8781862258911\\
47.4946365356445 14.8793077468872\\
47.5946388244629 14.8803367614746\\
47.694637298584 14.8826732635498\\
47.7946357727051 14.8827972412109\\
47.8946380615234 14.8835229873657\\
47.9946365356445 14.8847389221191\\
48.0946388244629 14.885817527771\\
48.194637298584 14.8869485855103\\
48.2946357727051 14.8880596160889\\
48.3946380615234 14.8891773223877\\
48.4946365356445 14.8902921676636\\
48.5946388244629 14.8914089202881\\
48.694637298584 14.8925561904907\\
48.7946357727051 14.8936958312988\\
48.8946380615234 14.8949213027954\\
48.9946365356445 14.8960485458374\\
49.0946388244629 14.8971891403198\\
49.194637298584 14.8983669281006\\
49.2946357727051 14.8995685577393\\
49.3946380615234 14.9008207321167\\
49.4946365356445 14.9067211151123\\
49.5842514038086 14.9108409881592\\
49.6559371948242 14.9104833602905\\
49.7347946166992 14.9047269821167\\
49.8215370178223 14.9053497314453\\
49.9169502258301 14.9217987060547\\
49.9568328857422 14.8696908950806\\
50.0007057189941 14.9013805389404\\
50.0206336975098 14.9150381088257\\
50.0371971130371 14.9169282913208\\
50.0554161071777 14.9086589813232\\
50.0754585266113 14.9086856842041\\
50.0975074768066 14.908839225769\\
50.1217575073242 14.9090070724487\\
50.1484336853027 14.9092130661011\\
50.1777763366699 14.9094581604004\\
50.2100563049316 14.9097452163696\\
50.2455635070801 14.9100761413574\\
50.2846183776855 14.9104537963867\\
50.327579498291 14.9108791351318\\
50.3748397827148 14.91135597229\\
50.4268226623535 14.9118852615356\\
50.4840087890625 14.9124708175659\\
50.5469093322754 14.9131155014038\\
50.6161003112793 14.9138240814209\\
50.692211151123 14.9146041870117\\
50.7759323120117 14.9247179031372\\
50.8680267333984 14.9077129364014\\
50.9680252075195 14.9175395965576\\
51.0680236816406 14.9213008880615\\
51.168025970459 14.9267997741699\\
51.2180252075195 14.9784574508667\\
51.2553977966309 14.9531364440918\\
51.2579727172852 14.9359560012817\\
51.2608032226563 14.9393444061279\\
51.263916015625 14.9414644241333\\
51.2673416137695 14.9389085769653\\
51.2711067199707 14.9327754974365\\
51.2752532958984 14.9242782592773\\
51.2798080444336 14.915807723999\\
51.2848243713379 14.9088706970215\\
51.290340423584 14.9027404785156\\
51.2964057922363 14.8959302902222\\
51.3030815124512 14.8874568939209\\
51.3104209899902 14.8750219345093\\
51.3184967041016 14.8581142425537\\
51.3273811340332 14.8361854553223\\
51.3371505737305 14.8190174102783\\
51.347900390625 14.802435874939\\
51.3597221374512 14.8030786514282\\
51.372730255127 14.8046131134033\\
51.387035369873 14.8082180023193\\
51.4027709960938 14.8152990341187\\
51.4200820922852 14.8329544067383\\
51.4391212463379 14.8923940658569\\
51.4600677490234 14.9445381164551\\
51.4831085205078 14.9533796310425\\
51.5084533691406 14.9363822937012\\
51.5363311767578 14.9378061294556\\
51.5669975280762 14.9420528411865\\
51.6007308959961 14.9428033828735\\
51.6378364562988 14.9401407241821\\
51.6764678955078 14.9367094039917\\
51.7189598083496 14.9344129562378\\
51.7657051086426 14.9328603744507\\
51.8171195983887 14.9317531585693\\
51.8736801147461 14.9320392608643\\
51.9213600158691 14.9337091445923\\
51.9687728881836 14.9330472946167\\
52.0141105651855 14.9328880310059\\
52.056568145752 14.9347467422485\\
52.103271484375 14.9356412887573\\
52.1546440124512 14.9343481063843\\
52.2111549377441 14.9317798614502\\
52.2733154296875 14.9320735931396\\
52.3416938781738 14.9326143264771\\
52.4169082641602 14.9331188201904\\
52.4996490478516 14.9338397979736\\
52.5906600952148 14.9368324279785\\
52.6906585693359 14.9334917068481\\
52.790657043457 14.9390830993652\\
52.8906593322754 14.9354152679443\\
52.9906578063965 14.9379434585571\\
52.9936103820801 15.04443359375\\
52.9968528747559 14.8994569778442\\
53.0004234313965 14.9029769897461\\
53.0029792785645 14.9590759277344\\
53.0057907104492 14.9623785018921\\
53.0088844299316 14.9428062438965\\
53.0122871398926 14.9400806427002\\
53.0160293579102 14.9409799575806\\
53.0201454162598 14.9411869049072\\
53.0246734619141 14.9411592483521\\
53.0296516418457 14.9411487579346\\
53.0351333618164 14.9411497116089\\
53.0411605834961 14.9411563873291\\
53.0477867126465 14.9411678314209\\
53.0550804138184 14.9411849975586\\
53.063102722168 14.9412107467651\\
53.0719261169434 14.9412441253662\\
53.0816307067871 14.9412889480591\\
53.0923080444336 14.9413452148438\\
53.1040496826172 14.9414148330688\\
53.1169700622559 14.9415006637573\\
53.1311798095703 14.9416017532349\\
53.1468124389648 14.941722869873\\
53.164005279541 14.9418640136719\\
53.1829223632813 14.94202709198\\
53.203727722168 14.942214012146\\
53.2266120910645 14.9424276351929\\
53.2517852783203 14.9426689147949\\
53.2794799804688 14.9429397583008\\
53.3099403381348 14.9432430267334\\
53.343448638916 14.9435796737671\\
53.3803062438965 14.9439525604248\\
53.4208488464355 14.9443655014038\\
53.465446472168 14.9448204040527\\
53.5145034790039 14.9453220367432\\
53.5684700012207 14.945873260498\\
53.6278305053711 14.9464807510376\\
53.6931266784668 14.9471492767334\\
53.7649536132813 14.9478845596313\\
53.8439598083496 14.9486932754517\\
53.9308700561523 14.951548576355\\
54.0264701843262 14.9484958648682\\
54.1264686584473 14.9517402648926\\
54.2264709472656 14.9599142074585\\
54.2464714050293 14.961986541748\\
54.2542266845703 14.9865913391113\\
54.2617721557617 14.9903116226196\\
54.2700691223145 14.9875755310059\\
54.2779922485352 14.9843769073486\\
54.2867050170898 14.9837923049927\\
54.2962913513184 14.9843301773071\\
54.3068389892578 15.0072889328003\\
54.3184356689453 15.041934967041\\
54.3311958312988 15.0456485748291\\
54.3452301025391 15.0247688293457\\
54.360668182373 15.0046682357788\\
54.3776512145996 14.983325958252\\
54.3963317871094 14.9616899490356\\
54.4168815612793 14.9404220581055\\
54.4394836425781 14.9154081344604\\
54.4643478393555 14.8862524032593\\
54.4809799194336 14.8738555908203\\
54.4992752075195 14.8749895095825\\
54.519401550293 14.8864278793335\\
54.5415382385254 14.9065370559692\\
54.5658912658691 14.916374206543\\
54.5926780700684 14.9529294967651\\
54.622142791748 14.9609718322754\\
54.6545562744141 14.9627733230591\\
54.6902084350586 14.9578075408936\\
54.7294273376465 14.9623889923096\\
54.7725677490234 14.9683647155762\\
54.8200225830078 14.9687585830688\\
54.8722229003906 14.9646596908569\\
54.9296417236328 14.9662065505981\\
54.9768371582031 14.9577407836914\\
55.028751373291 14.9569721221924\\
55.0732498168945 14.9557065963745\\
55.112865447998 14.9580173492432\\
55.1461601257324 14.9652967453003\\
55.1827850341797 14.9665079116821\\
55.223072052002 14.9662923812866\\
55.2673873901367 14.9648876190186\\
55.3161315917969 14.9655256271362\\
55.3697547912598 14.9661121368408\\
55.4287376403809 14.9666032791138\\
55.4936180114746 14.9673738479614\\
55.5649871826172 14.9685735702515\\
55.643497467041 14.970742225647\\
55.7298545837402 14.9747219085693\\
55.8248481750488 14.9793386459351\\
55.9248466491699 14.9800205230713\\
56.024845123291 14.9774055480957\\
56.1248474121094 14.9766712188721\\
56.2248458862305 14.9761552810669\\
56.3248481750488 14.9759340286255\\
56.4248466491699 14.9772815704346\\
56.524845123291 14.9781999588013\\
56.6248474121094 14.9796209335327\\
56.7248458862305 14.9805517196655\\
56.8248481750488 14.9818058013916\\
56.9248466491699 14.9828472137451\\
57.024845123291 14.9840288162231\\
57.1248474121094 14.9851179122925\\
57.2248458862305 14.9862689971924\\
57.3248481750488 14.9873790740967\\
57.4248466491699 14.9885158538818\\
57.524845123291 14.9896354675293\\
57.6248474121094 14.9907817840576\\
57.7248458862305 14.9918870925903\\
57.8248481750488 14.993016242981\\
57.9248466491699 14.994140625\\
58.024845123291 14.9952688217163\\
58.1248474121094 14.996395111084\\
58.2248458862305 14.997522354126\\
58.3248481750488 14.9986486434937\\
58.4248466491699 14.9997749328613\\
58.524845123291 15.0011625289917\\
58.6248474121094 15.0023136138916\\
58.7248458862305 15.003436088562\\
58.8248481750488 15.0045642852783\\
58.9248466491699 15.005690574646\\
59.024845123291 15.006817817688\\
59.1248474121094 15.00794506073\\
59.2248458862305 15.009072303772\\
59.3248481750488 15.010199546814\\
59.4248466491699 15.011326789856\\
59.524845123291 15.0124540328979\\
59.6248474121094 15.0135803222656\\
59.7248458862305 15.0147075653076\\
59.8248481750488 15.0158348083496\\
59.9124221801758 15.0169715881348\\
60 15.0179386138916\\
60.0963325500488 15.0200901031494\\
60.1963348388672 15.0201578140259\\
60.2963333129883 15.0208759307861\\
60.3963356018066 15.0221004486084\\
60.4963340759277 15.0231828689575\\
60.5963325500488 15.0243186950684\\
60.6963348388672 15.0254344940186\\
60.7963333129883 15.026556968689\\
60.8963356018066 15.0276775360107\\
60.9963340759277 15.0287981033325\\
61.0963325500488 15.029932975769\\
61.1963348388672 15.0310831069946\\
61.2963333129883 15.0322885513306\\
61.3963356018066 15.0334300994873\\
61.4963340759277 15.0345764160156\\
61.5963325500488 15.0357484817505\\
61.6963348388672 15.036940574646\\
61.7963333129883 15.0381727218628\\
61.8963356018066 15.0414304733276\\
61.9940414428711 15.0473718643188\\
62.0722236633301 15.051513671875\\
62.1347732543945 15.042179107666\\
62.2035751342773 15.0599250793457\\
62.2478866577148 15.0174055099487\\
62.296630859375 15.0323143005371\\
62.3256530761719 15.0535020828247\\
62.3496742248535 15.1938543319702\\
62.376091003418 14.9218626022339\\
62.4051551818848 15.0414056777954\\
62.4213905334473 15.0490446090698\\
62.4340591430664 15.0494184494019\\
62.4479942321777 15.0460338592529\\
62.463321685791 15.0460567474365\\
62.4801826477051 15.0461769104004\\
62.4987297058105 15.0462989807129\\
62.5191307067871 15.0464448928833\\
62.5415687561035 15.0466175079346\\
62.5662574768066 15.0468187332153\\
62.5934104919434 15.0470533370972\\
62.6232795715332 15.047324180603\\
62.656135559082 15.0476331710815\\
62.6922760009766 15.04798412323\\
62.7320327758789 15.0483798980713\\
62.775764465332 15.0488233566284\\
62.8238677978516 15.0493154525757\\
62.8767852783203 15.049859046936\\
62.9349899291992 15.0504579544067\\
62.9990196228027 15.0511169433594\\
63.069450378418 15.0518407821655\\
63.1469230651855 15.0526361465454\\
63.2321434020996 15.062915802002\\
63.3258857727051 15.0456228256226\\
63.4258842468262 15.0556468963623\\
63.5258865356445 15.0610942840576\\
63.6258850097656 15.0616216659546\\
63.6758842468262 15.1049375534058\\
63.7138481140137 15.0546159744263\\
63.7273941040039 15.0228109359741\\
63.7422943115234 15.0401582717896\\
63.7586822509766 15.0417232513428\\
63.7767105102539 15.0201044082642\\
63.7965431213379 15.0248994827271\\
63.8183555603027 15.0414867401123\\
63.8423538208008 15.0451059341431\\
63.8687477111816 15.0419034957886\\
63.8977813720703 15.0412044525146\\
63.9297218322754 15.0381746292114\\
63.9648513793945 15.033673286438\\
64.0015869140625 15.0343418121338\\
64.022705078125 15.0399312973022\\
64.0459365844727 15.0566110610962\\
64.0714874267578 15.0621290206909\\
64.0996017456055 15.0695152282715\\
64.1305236816406 15.0779390335083\\
64.1645355224609 15.0767116546631\\
64.2019424438477 15.0722513198853\\
64.2431030273438 15.0711669921875\\
64.2883682250977 15.070894241333\\
64.3381729125977 15.0706672668457\\
64.3929443359375 15.0706806182861\\
64.4477233886719 15.0716276168823\\
64.4993057250977 15.0719766616821\\
64.5481796264648 15.0721464157104\\
64.5940780639648 15.0724229812622\\
64.6370315551758 15.07444190979\\
64.6842803955078 15.0749206542969\\
64.7362518310547 15.0734329223633\\
64.7934188842773 15.0715017318726\\
64.8563079833984 15.0718288421631\\
64.9254837036133 15.0723867416382\\
65.0015716552734 15.0729150772095\\
65.0852737426758 15.0736570358276\\
65.1773452758789 15.0765190124512\\
65.27734375 15.0734539031982\\
65.3773422241211 15.0788116455078\\
65.4773483276367 15.0753974914551\\
65.5773468017578 15.0776395797729\\
65.5794677734375 15.1987447738647\\
65.5817947387695 15.0657567977905\\
65.5843658447266 15.0498218536377\\
65.5871810913086 15.1420660018921\\
65.5902862548828 15.0427160263062\\
65.5937042236328 15.0500993728638\\
65.5974578857422 15.0894842147827\\
65.6015930175781 15.0913667678833\\
65.6061325073242 15.0816202163696\\
65.6111373901367 15.0807237625122\\
65.6166305541992 15.0810480117798\\
65.6226806640625 15.0810880661011\\
65.6293334960938 15.0810823440552\\
65.6366577148438 15.0810890197754\\
65.6447067260742 15.0811042785645\\
65.653564453125 15.0811281204224\\
65.6633071899414 15.0811634063721\\
65.6740188598633 15.0812110900879\\
65.6858062744141 15.0812721252441\\
65.6987762451172 15.0813493728638\\
65.7130355834961 15.0814447402954\\
65.7287292480469 15.0815591812134\\
65.7459869384766 15.0816946029663\\
65.7649688720703 15.0818538665771\\
65.7858505249023 15.0820379257202\\
65.8088226318359 15.0822496414185\\
65.8340911865234 15.0824890136719\\
65.8618850708008 15.0827598571777\\
65.892463684082 15.083062171936\\
65.9260940551758 15.083399772644\\
65.9630889892578 15.0837745666504\\
66.0037841796875 15.0841884613037\\
66.0485458374023 15.0846462249756\\
66.097785949707 15.0851497650146\\
66.1519470214844 15.0857048034668\\
66.2115325927734 15.0863151550293\\
66.2770690917969 15.086986541748\\
66.3491592407227 15.0877256393433\\
66.428466796875 15.0885381698608\\
66.5156936645508 15.0913028717041\\
66.6116485595703 15.0884408950806\\
66.7116470336914 15.0915937423706\\
66.811653137207 15.1009702682495\\
66.8316497802734 15.11301612854\\
66.8391723632813 15.1430721282959\\
66.8464965820313 15.1477613449097\\
66.8545455932617 15.1438503265381\\
66.8622512817383 15.1401510238647\\
66.870735168457 15.1553916931152\\
66.8800659179688 15.1987838745117\\
66.8903274536133 15.2156276702881\\
66.9016189575195 15.2011203765869\\
66.9140319824219 15.1767606735229\\
66.9276962280273 15.1557941436768\\
66.9427185058594 15.1325626373291\\
66.9592437744141 15.1075763702393\\
66.977424621582 15.0814075469971\\
66.997428894043 15.0422296524048\\
67.0194244384766 15.0157146453857\\
67.0415725708008 14.9985752105713\\
67.0659408569336 14.9915113449097\\
67.0927429199219 14.9918766021729\\
67.1222229003906 15.0218133926392\\
67.154655456543 15.0684223175049\\
67.1903305053711 15.144868850708\\
67.1995849609375 15.1386976242065\\
67.2097702026367 15.1217012405396\\
67.2209701538086 15.1131982803345\\
67.233283996582 15.1098079681396\\
67.2468414306641 15.1084575653076\\
67.2617416381836 15.1079502105713\\
67.2781448364258 15.1079797744751\\
67.2961807250977 15.1087608337402\\
67.3160247802734 15.1095209121704\\
67.3378448486328 15.1047868728638\\
67.3618545532227 15.1077585220337\\
67.3882598876953 15.1053314208984\\
67.4173126220703 15.1066179275513\\
67.4492645263672 15.1057300567627\\
67.4844131469727 15.1032648086548\\
67.523078918457 15.1014394760132\\
67.5656127929688 15.1016902923584\\
67.6123962402344 15.1004629135132\\
67.6579818725586 15.0982751846313\\
67.6993789672852 15.0966701507568\\
67.7355346679688 15.0984401702881\\
67.7753067016602 15.1013946533203\\
67.8190612792969 15.103494644165\\
67.8671875 15.1041507720947\\
67.9201278686523 15.1062021255493\\
67.9783630371094 15.1064128875732\\
68.0421829223633 15.1088247299194\\
68.1070251464844 15.1115751266479\\
68.178352355957 15.1149873733521\\
68.2568130493164 15.1181697845459\\
68.3431243896484 15.1157608032227\\
68.4380569458008 15.115195274353\\
68.5380630493164 15.1147289276123\\
68.6380615234375 15.114312171936\\
68.7380599975586 15.1140518188477\\
68.8380584716797 15.1151237487793\\
68.9380569458008 15.1160898208618\\
69.0380630493164 15.1174726486206\\
69.1380615234375 15.1184282302856\\
69.2380599975586 15.1196670532227\\
69.3380584716797 15.1207189559937\\
69.4380569458008 15.1218948364258\\
69.5380630493164 15.1229906082153\\
69.6380615234375 15.1241388320923\\
69.7380599975586 15.1252527236938\\
69.8380584716797 15.126389503479\\
69.9380569458008 15.1275262832642\\
70.0380630493164 15.1286392211914\\
70.1380615234375 15.1297645568848\\
70.2380599975586 15.1308937072754\\
70.3380584716797 15.1320199966431\\
70.4380569458008 15.1331481933594\\
70.5380630493164 15.1342754364014\\
70.6380615234375 15.1354036331177\\
70.7380599975586 15.136531829834\\
70.8380584716797 15.137659072876\\
70.9380569458008 15.13893699646\\
71.0380630493164 15.1402292251587\\
71.1380615234375 15.1413145065308\\
71.2380599975586 15.1424551010132\\
71.3380584716797 15.1435794830322\\
71.4380569458008 15.1447086334229\\
71.5380630493164 15.1458368301392\\
71.6380615234375 15.1469650268555\\
71.7380599975586 15.1480922698975\\
71.8380584716797 15.1492204666138\\
71.9380569458008 15.1503486633301\\
72.0380630493164 15.1514768600464\\
72.1380615234375 15.1526050567627\\
72.2380599975586 15.153733253479\\
72.3380584716797 15.154860496521\\
72.4380569458008 15.155987739563\\
72.5380630493164 15.1582508087158\\
72.6380615234375 15.1583528518677\\
72.7380599975586 15.159107208252\\
72.8380584716797 15.1603221893311\\
72.9380569458008 15.1614103317261\\
73.0380630493164 15.1625452041626\\
73.1380615234375 15.1636619567871\\
73.2380599975586 15.1647853851318\\
73.3380584716797 15.1659069061279\\
73.4380569458008 15.1670274734497\\
73.5380630493164 15.1681623458862\\
73.6380615234375 15.1693153381348\\
73.7380599975586 15.1705207824707\\
73.8380584716797 15.1716632843018\\
73.9380569458008 15.1728115081787\\
74.0380630493164 15.1739826202393\\
74.1380615234375 15.1751747131348\\
74.2380599975586 15.1764068603516\\
74.3380584716797 15.1794633865356\\
74.4363632202148 15.1855659484863\\
74.5150299072266 15.1896829605103\\
74.5779571533203 15.1807870864868\\
74.6471862792969 15.1811437606812\\
74.7233276367188 15.1816787719727\\
74.8070907592773 15.1823129653931\\
74.8992309570313 15.1989517211914\\
74.945930480957 15.1532049179077\\
74.9729614257813 15.1769781112671\\
74.9864807128906 15.192852973938\\
75 15.1937732696533\\
};
\addlegendentry{10 "};

\addplot [
color=black,
dotted
]
table[row sep=crcr]{
0 14.7005214691162\\
9.99999971718069e-10 14.7005214691162\\
2.49999998480632e-09 14.7005214691162\\
4.74999994892755e-09 14.7005214691162\\
8.1250002281763e-09 14.7005214691162\\
1.31874999809156e-08 14.7005214691162\\
2.0781250498203e-08 14.7005214691162\\
3.21718758300449e-08 14.7005214691162\\
4.92578138278077e-08 14.7005214691162\\
7.48867208244519e-08 14.7005214691162\\
1.13330081319418e-07 14.7005214691162\\
1.70995122061868e-07 14.7005214691162\\
2.57492672517401e-07 14.7005214691162\\
3.8723902662241e-07 14.7005214691162\\
5.81858500936505e-07 14.7005214691162\\
8.73787769251066e-07 14.7005214691162\\
1.31168167172291e-06 14.7005214691162\\
1.9685226106958e-06 14.7005214691162\\
2.95378367809462e-06 14.7005214691162\\
4.43167573394021e-06 14.7005214691162\\
6.64851359033491e-06 14.7005214691162\\
9.97376992017962e-06 14.7005214691162\\
1.49616553244414e-05 14.7005224227905\\
2.24434825213393e-05 14.7005224227905\\
3.36662233166862e-05 14.7005233764648\\
5.05003372381907e-05 14.7005233764648\\
7.57515081204474e-05 14.7005243301392\\
0.000113628258986864 14.7005252838135\\
0.000170443381648511 14.7005262374878\\
0.000255666091106832 14.7005271911621\\
0.000383500126190484 14.7005281448364\\
0.000575251178815961 14.7005290985107\\
0.000862877757754177 14.7005290985107\\
0.00129431765526533 14.7005281448364\\
0.00194147753063589 14.7005243301392\\
0.00291221728548408 14.7005195617676\\
0.00436832662671804 14.7005176544189\\
0.00597004732117057 14.7005195617676\\
0.00773194013163447 14.7005214691162\\
0.009670021943748 14.7005224227905\\
0.0118019115179777 14.7005224227905\\
0.0141469910740852 14.7005214691162\\
0.0167265776544809 14.7005214691162\\
0.0195641238242388 14.7005214691162\\
0.0226854234933853 14.7005214691162\\
0.0261188540607691 14.7005214691162\\
0.0298956278711557 14.7005214691162\\
0.0340500771999359 14.7005214691162\\
0.0386199727654457 14.7005214691162\\
0.0436468608677387 14.7005214691162\\
0.0491764321923256 14.7005214691162\\
0.0552589632570744 14.7005214691162\\
0.0619497485458851 14.7005214691162\\
0.0693096145987511 14.7005214691162\\
0.0774054601788521 14.7005214691162\\
0.0863108932971954 14.7005205154419\\
0.0961068719625473 14.7005224227905\\
0.106882445514202 14.7005224227905\\
0.118735581636429 14.7005214691162\\
0.13177402317524 14.7005214691162\\
0.146116316318512 14.7005214691162\\
0.161892831325531 14.7005214691162\\
0.179247006773949 14.7005214691162\\
0.198336601257324 14.7005214691162\\
0.21933513879776 14.7005214691162\\
0.242433547973633 14.7005214691162\\
0.267841786146164 14.7005214691162\\
0.29579085111618 14.7005214691162\\
0.32653483748436 14.7005214691162\\
0.360353201627731 14.7005214691162\\
0.397553414106369 14.7005214691162\\
0.438473641872406 14.7005214691162\\
0.483485877513886 14.7005214691162\\
0.532999336719513 14.7005214691162\\
0.587464153766632 14.7005214691162\\
0.647375464439392 14.7005214691162\\
0.71327793598175 14.7005214691162\\
0.7857705950737 14.7005214691162\\
0.865512549877167 14.7005214691162\\
0.953228712081909 14.7005214691162\\
1.04971647262573 14.7005214691162\\
1.14971649646759 14.7005214691162\\
1.24971640110016 14.7005214691162\\
1.34971642494202 14.7005214691162\\
1.44971644878387 14.7005214691162\\
1.54971647262573 14.7005214691162\\
1.64971649646759 14.7005214691162\\
1.74971640110016 14.7005214691162\\
1.84971642494202 14.7005214691162\\
1.94971644878387 14.7005214691162\\
2.04971647262573 14.7005214691162\\
2.1497163772583 14.7005214691162\\
2.24971652030945 14.7005214691162\\
2.34971642494202 14.7005214691162\\
2.44971656799316 14.7005214691162\\
2.54971647262573 14.7005214691162\\
2.6497163772583 14.7005214691162\\
2.74971652030945 14.7005214691162\\
2.84971642494202 14.7005214691162\\
2.94971656799316 14.7005214691162\\
3.04971647262573 14.7005214691162\\
3.1497163772583 14.7005214691162\\
3.24971652030945 14.7005214691162\\
3.34971642494202 14.7005214691162\\
3.44971656799316 14.7005214691162\\
3.54971647262573 14.7005214691162\\
3.6497163772583 14.7005214691162\\
3.74971652030945 14.7005214691162\\
3.84971642494202 14.7005214691162\\
3.94971656799316 14.7005214691162\\
4.04971647262573 14.7005214691162\\
4.1497163772583 14.7005214691162\\
4.24971628189087 14.7005214691162\\
4.3497166633606 14.7005214691162\\
4.44971656799316 14.7005214691162\\
4.54971647262573 14.7005214691162\\
4.6497163772583 14.7005214691162\\
4.74971628189087 14.7005214691162\\
4.8497166633606 14.7005214691162\\
4.94971656799316 14.7005214691162\\
5.04971647262573 14.7005214691162\\
5.1497163772583 14.7005214691162\\
5.24971628189087 14.7005214691162\\
5.3497166633606 14.7005214691162\\
5.44971656799316 14.7005214691162\\
5.54971647262573 14.7005214691162\\
5.6497163772583 14.7005214691162\\
5.74971628189087 14.7005214691162\\
5.8497166633606 14.7005214691162\\
5.94971656799316 14.7005214691162\\
6.04971647262573 14.7005214691162\\
6.1497163772583 14.7005205154419\\
6.24971628189087 14.7005195617676\\
6.3497166633606 14.7005186080933\\
6.44971656799316 14.7005205154419\\
6.54971647262573 14.7005224227905\\
6.6497163772583 14.7005233764648\\
6.74971628189087 14.7005224227905\\
6.8497166633606 14.7005214691162\\
6.94971656799316 14.7005214691162\\
7.04971647262573 14.7005214691162\\
7.1497163772583 14.7005214691162\\
7.24971628189087 14.7005214691162\\
7.3497166633606 14.7005214691162\\
7.44971656799316 14.7005214691162\\
7.54971647262573 14.7005214691162\\
7.6497163772583 14.7005214691162\\
7.74971628189087 14.7005214691162\\
7.8497166633606 14.7005214691162\\
7.94971656799316 14.7005214691162\\
8.04971599578857 14.7005214691162\\
8.1497163772583 14.7005214691162\\
8.24971675872803 14.7005214691162\\
8.34971618652344 14.7005214691162\\
8.44971656799316 14.7005214691162\\
8.54971599578857 14.7005214691162\\
8.6497163772583 14.7005214691162\\
8.74971675872803 14.7005214691162\\
8.84971618652344 14.7005214691162\\
8.94971656799316 14.7005214691162\\
9.04971599578857 14.7005214691162\\
9.1497163772583 14.7005214691162\\
9.24971675872803 14.7005214691162\\
9.34971618652344 14.7005214691162\\
9.44971656799316 14.7005214691162\\
9.54971599578857 14.7005214691162\\
9.6497163772583 14.7005214691162\\
9.74971675872803 14.7005214691162\\
9.84971618652344 14.7005214691162\\
9.94971656799316 14.7005214691162\\
10.0497159957886 14.7005214691162\\
10.1497163772583 14.7005214691162\\
10.249716758728 14.7005214691162\\
10.3497161865234 14.7005214691162\\
10.4497165679932 14.7005214691162\\
10.5497159957886 14.7005214691162\\
10.6497163772583 14.7005214691162\\
10.749716758728 14.7005214691162\\
10.8497161865234 14.7005214691162\\
10.9497165679932 14.7005214691162\\
11.0497159957886 14.7005214691162\\
11.1497163772583 14.7005214691162\\
11.249716758728 14.7005214691162\\
11.3497161865234 14.7005214691162\\
11.4497165679932 14.7005214691162\\
11.5497159957886 14.7005214691162\\
11.6497163772583 14.7005214691162\\
11.749716758728 14.7005214691162\\
11.8497161865234 14.7005214691162\\
11.9497165679932 14.7005214691162\\
12.0497159957886 14.7005214691162\\
12.1497163772583 14.7005214691162\\
12.249716758728 14.7005214691162\\
12.347146987915 14.7005214691162\\
12.4250946044922 14.7005214691162\\
12.4874591827393 14.7005224227905\\
12.5560598373413 14.7005214691162\\
12.6315202713013 14.7005205154419\\
12.714527130127 14.7005214691162\\
12.8058338165283 14.7005214691162\\
12.8497476577759 14.7005224227905\\
12.8980522155762 14.7005205154419\\
12.9143037796021 14.7005214691162\\
12.9277105331421 14.7005224227905\\
12.9424571990967 14.7005224227905\\
12.9586782455444 14.7005214691162\\
12.9765224456787 14.7005214691162\\
12.9961500167847 14.7005214691162\\
13.0177412033081 14.7005214691162\\
13.0414915084839 14.7005214691162\\
13.0676164627075 14.7005214691162\\
13.0963535308838 14.7005214691162\\
13.1279649734497 14.7005214691162\\
13.1627368927002 14.7005214691162\\
13.2009868621826 14.7005214691162\\
13.2430610656738 14.7005214691162\\
13.289342880249 14.7005214691162\\
13.3402528762817 14.7005214691162\\
13.3962545394897 14.7005214691162\\
13.4578561782837 14.7005214691162\\
13.5256175994873 14.7005214691162\\
13.600154876709 14.7005214691162\\
13.6821460723877 14.7005205154419\\
13.7723369598389 14.7005214691162\\
13.871545791626 14.7005214691162\\
13.9715461730957 14.7005462646484\\
14.0715456008911 14.7006502151489\\
14.1215467453003 14.7011623382568\\
14.1629066467285 14.7010250091553\\
14.2008075714111 14.6995153427124\\
14.202446937561 14.6928110122681\\
14.2042503356934 14.6941957473755\\
14.2062339782715 14.6987857818604\\
14.2084159851074 14.7011861801147\\
14.2108154296875 14.7014493942261\\
14.2134561538696 14.7018566131592\\
14.2163600921631 14.7019338607788\\
14.219554901123 14.7011957168579\\
14.223069190979 14.7014713287354\\
14.2269344329834 14.7013387680054\\
14.2311868667603 14.7010488510132\\
14.2358636856079 14.7007961273193\\
14.2410087585449 14.7006464004517\\
14.2466678619385 14.7005634307861\\
14.252893447876 14.7004985809326\\
14.2597417831421 14.7004308700562\\
14.2672748565674 14.7003688812256\\
14.2755603790283 14.7003192901611\\
14.2846755981445 14.7002840042114\\
14.2947015762329 14.7002601623535\\
14.3057298660278 14.7002515792847\\
14.3178615570068 14.7002611160278\\
14.3312063217163 14.7004261016846\\
14.3458862304688 14.7005033493042\\
14.3620328903198 14.7004623413086\\
14.3797950744629 14.700400352478\\
14.3993330001831 14.7003679275513\\
14.4208250045776 14.7003698348999\\
14.444465637207 14.7003870010376\\
14.4704713821411 14.7005910873413\\
14.4990768432617 14.7005634307861\\
14.5269069671631 14.7004833221436\\
14.5575199127197 14.7005062103271\\
14.591194152832 14.7005157470703\\
14.6282358169556 14.7005224227905\\
14.668981552124 14.7005252838135\\
14.7138023376465 14.7005271911621\\
14.7631044387817 14.7005214691162\\
14.8173370361328 14.7005224227905\\
14.8769931793213 14.7005224227905\\
14.9244384765625 14.7005252838135\\
14.971435546875 14.7005233764648\\
15.0163955688477 14.7005224227905\\
15.0584926605225 14.7005224227905\\
15.09974193573 14.7005214691162\\
15.1451168060303 14.7005214691162\\
15.1950283050537 14.7005224227905\\
15.2499313354492 14.7005214691162\\
15.31032371521 14.7005224227905\\
15.3767566680908 14.7005214691162\\
15.4498319625854 14.7005214691162\\
15.5302152633667 14.7005214691162\\
15.6186370849609 14.7005205154419\\
15.7159004211426 14.7005214691162\\
15.8159008026123 14.7005205154419\\
15.9159002304077 14.7005224227905\\
16.0159015655518 14.7005205154419\\
16.0210952758789 14.7005186080933\\
16.0268077850342 14.7005262374878\\
16.0330924987793 14.7005195617676\\
16.0382099151611 14.7005224227905\\
16.0438404083252 14.7005233764648\\
16.0500316619873 14.7005224227905\\
16.0568428039551 14.7005224227905\\
16.0643348693848 14.7005224227905\\
16.0725765228271 14.7005224227905\\
16.0816440582275 14.7005224227905\\
16.0916156768799 14.7005224227905\\
16.1025848388672 14.7005214691162\\
16.114652633667 14.7005214691162\\
16.1279258728027 14.7005214691162\\
16.1425266265869 14.7005214691162\\
16.1585865020752 14.7005214691162\\
16.1762542724609 14.7005214691162\\
16.1956882476807 14.7005214691162\\
16.2170658111572 14.7005214691162\\
16.2405796051025 14.7005214691162\\
16.2664451599121 14.7005214691162\\
16.2948989868164 14.7005214691162\\
16.3261966705322 14.7005214691162\\
16.3606243133545 14.7005214691162\\
16.398494720459 14.7005214691162\\
16.4401531219482 14.7005214691162\\
16.4859771728516 14.7005214691162\\
16.5363826751709 14.7005214691162\\
16.5918292999268 14.7005214691162\\
16.6528205871582 14.7005214691162\\
16.7199096679688 14.7005214691162\\
16.7937088012695 14.7005214691162\\
16.8748874664307 14.7005214691162\\
16.9641857147217 14.7005205154419\\
17.0624122619629 14.7005214691162\\
17.1624126434326 14.7005205154419\\
17.2624130249023 14.7005548477173\\
17.282413482666 14.7006616592407\\
17.2906188964844 14.7007436752319\\
17.2996425628662 14.7007751464844\\
17.308614730835 14.7007913589478\\
17.3184852600098 14.7007904052734\\
17.3293399810791 14.7007570266724\\
17.3412818908691 14.7007074356079\\
17.3544178009033 14.7001132965088\\
17.3688678741455 14.6994142532349\\
17.3847618103027 14.7009563446045\\
17.4022445678711 14.7005996704102\\
17.4214763641357 14.7008581161499\\
17.4426326751709 14.7007923126221\\
17.4659042358398 14.7007246017456\\
17.4915008544922 14.7005987167358\\
17.5196590423584 14.7004423141479\\
17.5506324768066 14.7003784179688\\
17.5847034454346 14.7003488540649\\
17.6221809387207 14.700364112854\\
17.6634063720703 14.7004022598267\\
17.7087554931641 14.7004470825195\\
17.758638381958 14.7005138397217\\
17.8135089874268 14.7005186080933\\
17.8738670349121 14.7005281448364\\
17.9402618408203 14.7005233764648\\
18.0132961273193 14.7005233764648\\
18.0585689544678 14.7005319595337\\
18.1083717346191 14.7005300521851\\
18.1509475708008 14.7005319595337\\
18.1891002655029 14.7005233764648\\
18.2310676574707 14.7005100250244\\
18.2772312164307 14.7005090713501\\
18.328010559082 14.700511932373\\
18.3838691711426 14.7005252838135\\
18.4453125 14.7005205154419\\
18.5129013061523 14.7005205154419\\
18.5872478485107 14.7005214691162\\
18.6690311431885 14.7005214691162\\
18.7589912414551 14.7005214691162\\
18.857946395874 14.7005214691162\\
18.9579467773438 14.7005214691162\\
19.0579471588135 14.7005214691162\\
19.1579456329346 14.7005214691162\\
19.2579460144043 14.7005214691162\\
19.357946395874 14.7005214691162\\
19.4579467773438 14.7005214691162\\
19.5579471588135 14.7005214691162\\
19.6579456329346 14.7005214691162\\
19.7579460144043 14.7005214691162\\
19.857946395874 14.7005214691162\\
19.9579467773438 14.7005214691162\\
20.0579471588135 14.7005214691162\\
20.1579456329346 14.7005214691162\\
20.2579460144043 14.7005214691162\\
20.357946395874 14.7005214691162\\
20.4579467773438 14.7005214691162\\
20.5579471588135 14.7005214691162\\
20.6579456329346 14.7005214691162\\
20.7579460144043 14.7005214691162\\
20.857946395874 14.7005214691162\\
20.9579467773438 14.7005214691162\\
21.0579471588135 14.7005214691162\\
21.1579456329346 14.7005214691162\\
21.2579460144043 14.7005214691162\\
21.357946395874 14.7005214691162\\
21.4579467773438 14.7005214691162\\
21.5579471588135 14.7005214691162\\
21.6579456329346 14.7005214691162\\
21.7579460144043 14.7005214691162\\
21.857946395874 14.7005214691162\\
21.9579467773438 14.7005214691162\\
22.0579471588135 14.7005214691162\\
22.1579456329346 14.7005214691162\\
22.2579460144043 14.7005214691162\\
22.357946395874 14.7005214691162\\
22.4579467773438 14.7005214691162\\
22.5579471588135 14.7005214691162\\
22.6579456329346 14.7005214691162\\
22.7579460144043 14.7005214691162\\
22.857946395874 14.7005214691162\\
22.9579467773438 14.7005214691162\\
23.0579471588135 14.7005214691162\\
23.1579456329346 14.7005214691162\\
23.2579460144043 14.7005214691162\\
23.357946395874 14.7005214691162\\
23.4579467773438 14.7005214691162\\
23.5579471588135 14.7005214691162\\
23.6579456329346 14.7005214691162\\
23.7579460144043 14.7005214691162\\
23.857946395874 14.7005214691162\\
23.9579467773438 14.7005214691162\\
24.0579471588135 14.7005205154419\\
24.1579456329346 14.7005214691162\\
24.2579460144043 14.7005214691162\\
24.357946395874 14.7005214691162\\
24.4579467773438 14.7005214691162\\
24.5579471588135 14.7005214691162\\
24.6579456329346 14.7005214691162\\
24.7421741485596 14.7005214691162\\
24.8095417022705 14.7005224227905\\
24.8836441040039 14.7005214691162\\
24.9651584625244 14.7005205154419\\
25.0548248291016 14.7005205154419\\
25.0960559844971 14.7005224227905\\
25.1414089202881 14.7005157470703\\
25.1613826751709 14.7005319595337\\
25.1780624389648 14.7005243301392\\
25.1964111328125 14.7005224227905\\
25.2165927886963 14.7005214691162\\
25.2387943267822 14.7005214691162\\
25.2632160186768 14.7005214691162\\
25.2900791168213 14.7005214691162\\
25.3196296691895 14.7005214691162\\
25.3521347045898 14.7005214691162\\
25.3878898620605 14.7005214691162\\
25.4272212982178 14.7005214691162\\
25.4704837799072 14.7005214691162\\
25.5180740356445 14.7005214691162\\
25.5704231262207 14.7005214691162\\
25.6280078887939 14.7005214691162\\
25.6913509368896 14.7005214691162\\
25.7610282897949 14.7005214691162\\
25.8376731872559 14.7005214691162\\
25.9219818115234 14.7005205154419\\
26.014720916748 14.7005214691162\\
26.1147212982178 14.7005205154419\\
26.2147216796875 14.7005367279053\\
26.3147220611572 14.7006893157959\\
26.3647212982178 14.7015781402588\\
26.4022121429443 14.7020950317383\\
26.4049663543701 14.6935615539551\\
26.407995223999 14.6975154876709\\
26.4113292694092 14.6994028091431\\
26.4149932861328 14.7014999389648\\
26.4190254211426 14.7016754150391\\
26.4234619140625 14.7016935348511\\
26.4283409118652 14.701605796814\\
26.4337062835693 14.701379776001\\
26.4396114349365 14.7010850906372\\
26.4461040496826 14.7007942199707\\
26.4532470703125 14.7005424499512\\
26.4611053466797 14.7003622055054\\
26.469747543335 14.7002143859863\\
26.4792556762695 14.6998815536499\\
26.4897136688232 14.6996459960938\\
26.5012187957764 14.6999731063843\\
26.5138721466064 14.7000875473022\\
26.5277919769287 14.7001256942749\\
26.5431041717529 14.7001686096191\\
26.5599479675293 14.7001810073853\\
26.5784759521484 14.700156211853\\
26.59885597229 14.700439453125\\
26.6212730407715 14.7004804611206\\
26.6459331512451 14.700532913208\\
26.673059463501 14.7005167007446\\
26.7028980255127 14.700514793396\\
26.7357215881348 14.7005186080933\\
26.7718257904053 14.7005252838135\\
26.8115406036377 14.7005281448364\\
26.8552284240723 14.7005281448364\\
26.903284072876 14.7005252838135\\
26.9561443328857 14.7005214691162\\
27.0142917633057 14.7005224227905\\
27.0605869293213 14.7005281448364\\
27.1115131378174 14.7005252838135\\
27.1556930541992 14.7005243301392\\
27.1975421905518 14.7005233764648\\
27.2435741424561 14.7005233764648\\
27.29421043396 14.7005233764648\\
27.3499088287354 14.7005224227905\\
27.4111785888672 14.7005233764648\\
27.4785747528076 14.7005224227905\\
27.5527095794678 14.7005224227905\\
27.6342601776123 14.7005224227905\\
27.7239646911621 14.7005214691162\\
27.8226375579834 14.7005224227905\\
27.9226379394531 14.7005214691162\\
28.0226383209229 14.7005224227905\\
28.1226387023926 14.7005214691162\\
28.1260833740234 14.7005138397217\\
28.1298713684082 14.7005367279053\\
28.1340389251709 14.700514793396\\
28.1376724243164 14.7005167007446\\
28.1416683197021 14.700532913208\\
28.1460647583008 14.7005167007446\\
28.1490650177002 14.7005224227905\\
28.1523666381836 14.7005252838135\\
28.1559963226318 14.7005233764648\\
28.1599903106689 14.7005233764648\\
28.1643829345703 14.7005233764648\\
28.1692161560059 14.7005233764648\\
28.1745319366455 14.7005233764648\\
28.1803779602051 14.7005233764648\\
28.1868114471436 14.7005224227905\\
28.1938858032227 14.7005224227905\\
28.2016677856445 14.7005224227905\\
28.2102298736572 14.7005224227905\\
28.2196464538574 14.7005224227905\\
28.2300052642822 14.7005224227905\\
28.2413997650146 14.7005224227905\\
28.2539348602295 14.7005224227905\\
28.2677211761475 14.7005224227905\\
28.2828884124756 14.7005224227905\\
28.2995700836182 14.7005224227905\\
28.3179225921631 14.7005224227905\\
28.3381080627441 14.7005224227905\\
28.3603134155273 14.7005224227905\\
28.3847389221191 14.7005224227905\\
28.4116058349609 14.7005224227905\\
28.4411602020264 14.7005224227905\\
28.4736709594727 14.7005224227905\\
28.5094318389893 14.7005224227905\\
28.5487689971924 14.7005224227905\\
28.592041015625 14.7005224227905\\
28.6396389007568 14.7005224227905\\
28.6919975280762 14.7005224227905\\
28.7495918273926 14.7005224227905\\
28.8129444122314 14.7005224227905\\
28.8826332092285 14.7005224227905\\
28.9592895507813 14.7005224227905\\
29.0436134338379 14.7005224227905\\
29.1363677978516 14.7005214691162\\
29.2363681793213 14.7005224227905\\
29.336368560791 14.7005214691162\\
29.4363689422607 14.7005615234375\\
29.4563694000244 14.7007255554199\\
29.4642658233643 14.7009019851685\\
29.471960067749 14.7009897232056\\
29.4804267883301 14.70103931427\\
29.4885120391846 14.7010803222656\\
29.497407913208 14.7010440826416\\
29.5071926116943 14.7008237838745\\
29.5179557800293 14.6983680725098\\
29.5297946929932 14.7006578445435\\
29.542818069458 14.7006225585938\\
29.5571441650391 14.701247215271\\
29.5729026794434 14.7011823654175\\
29.5902366638184 14.700611114502\\
29.6093044281006 14.7008380889893\\
29.630277633667 14.7004909515381\\
29.6533489227295 14.700457572937\\
29.6787281036377 14.7002973556519\\
29.6999015808105 14.7002630233765\\
29.7231941223145 14.7000732421875\\
29.7488136291504 14.6999292373657\\
29.7769947052002 14.7003746032715\\
29.8079967498779 14.7005567550659\\
29.8420963287354 14.7005100250244\\
29.8796062469482 14.7005405426025\\
29.9208679199219 14.7005271911621\\
29.9662551879883 14.7005548477173\\
30.0161819458008 14.7004346847534\\
30.0711002349854 14.7006025314331\\
30.1315116882324 14.7005386352539\\
30.1804618835449 14.7005319595337\\
30.2286376953125 14.7005290985107\\
30.2746448516846 14.7005214691162\\
30.3156337738037 14.7005252838135\\
30.3508148193359 14.7005128860474\\
30.3895168304443 14.7005081176758\\
30.4320869445801 14.7005128860474\\
30.4789142608643 14.7005167007446\\
30.530424118042 14.7005186080933\\
30.587085723877 14.7005195617676\\
30.6491241455078 14.7005195617676\\
30.7173671722412 14.7005195617676\\
30.7924346923828 14.7005224227905\\
30.8750076293945 14.7005205154419\\
30.9658393859863 14.7005205154419\\
31.0657539367676 14.7005214691162\\
31.1657543182373 14.7005224227905\\
31.265754699707 14.7005224227905\\
31.3657531738281 14.7005233764648\\
31.4657535552979 14.7005224227905\\
31.5657539367676 14.7005224227905\\
31.6657543182373 14.7005224227905\\
31.765754699707 14.7005224227905\\
31.8657531738281 14.7005224227905\\
31.9657535552979 14.7005224227905\\
32.0657539367676 14.7005224227905\\
32.1657524108887 14.7005224227905\\
32.265754699707 14.7005224227905\\
32.3657531738281 14.7005224227905\\
32.4657554626465 14.7005224227905\\
32.5657539367676 14.7005224227905\\
32.6657524108887 14.7005224227905\\
32.765754699707 14.7005224227905\\
32.8657531738281 14.7005224227905\\
32.9657554626465 14.7005224227905\\
33.0657539367676 14.7005224227905\\
33.1657524108887 14.7005224227905\\
33.265754699707 14.7005224227905\\
33.3657531738281 14.7005224227905\\
33.4657554626465 14.7005224227905\\
33.5657539367676 14.7005224227905\\
33.6657524108887 14.7005224227905\\
33.765754699707 14.7005224227905\\
33.8657531738281 14.7005224227905\\
33.9657554626465 14.7005224227905\\
34.0657539367676 14.7005224227905\\
34.1657524108887 14.7005224227905\\
34.265754699707 14.7005224227905\\
34.3657531738281 14.7005224227905\\
34.4657554626465 14.7005224227905\\
34.5657539367676 14.7005224227905\\
34.6657524108887 14.7005224227905\\
34.765754699707 14.7005224227905\\
34.8657531738281 14.7005224227905\\
34.9657554626465 14.7005224227905\\
35.0657539367676 14.7005224227905\\
35.1657524108887 14.7005224227905\\
35.265754699707 14.7005224227905\\
35.3657531738281 14.7005224227905\\
35.4657554626465 14.7005224227905\\
35.5657539367676 14.7005224227905\\
35.6657524108887 14.7005224227905\\
35.765754699707 14.7005224227905\\
35.8657531738281 14.7005224227905\\
35.9657554626465 14.7005224227905\\
36.0657539367676 14.7005224227905\\
36.1657524108887 14.7005224227905\\
36.265754699707 14.7005224227905\\
36.3657531738281 14.7005224227905\\
36.4657554626465 14.7005224227905\\
36.5657539367676 14.7005224227905\\
36.6657524108887 14.7005224227905\\
36.765754699707 14.7005224227905\\
36.8657531738281 14.7005214691162\\
36.9657554626465 14.7005214691162\\
37.0657539367676 14.7005214691162\\
37.1539344787598 14.7005214691162\\
37.2244758605957 14.7005214691162\\
37.302074432373 14.7005214691162\\
37.3874282836914 14.7005214691162\\
37.4813194274902 14.7005214691162\\
37.5215835571289 14.7005214691162\\
37.5658721923828 14.7005176544189\\
37.585823059082 14.7005319595337\\
37.6024284362793 14.7005262374878\\
37.6206932067871 14.7005243301392\\
37.6407852172852 14.7005233764648\\
37.6628875732422 14.7005233764648\\
37.687198638916 14.7005233764648\\
37.713939666748 14.7005233764648\\
37.7433547973633 14.7005233764648\\
37.7757110595703 14.7005233764648\\
37.8113059997559 14.7005233764648\\
37.8504600524902 14.7005233764648\\
37.8935279846191 14.7005233764648\\
37.9409027099609 14.7005233764648\\
37.9930152893066 14.7005233764648\\
38.0503387451172 14.7005233764648\\
38.1133918762207 14.7005233764648\\
38.1827545166016 14.7005233764648\\
38.2590522766113 14.7005233764648\\
38.3429794311523 14.7005224227905\\
38.4352989196777 14.7005233764648\\
38.5352973937988 14.7005224227905\\
38.6352996826172 14.7005376815796\\
38.7352981567383 14.7006978988647\\
38.7853012084961 14.7016763687134\\
38.822696685791 14.7839202880859\\
38.8252944946289 14.8122444152832\\
38.8281555175781 14.8208847045898\\
38.8313026428223 14.8218469619751\\
38.8347625732422 14.8179893493652\\
38.838565826416 14.8110094070435\\
38.8427543640137 14.8019409179688\\
38.8473587036133 14.790675163269\\
38.852424621582 14.7774524688721\\
38.8579978942871 14.7634496688843\\
38.864128112793 14.7495727539063\\
38.8708724975586 14.736439704895\\
38.878288269043 14.7252635955811\\
38.8864479064941 14.7160091400146\\
38.8954238891602 14.7048768997192\\
38.9052963256836 14.70188331604\\
38.9161567687988 14.6983852386475\\
38.9281005859375 14.6999492645264\\
38.9412422180176 14.7006044387817\\
38.955696105957 14.7006597518921\\
38.9715957641602 14.7008104324341\\
38.989086151123 14.7003049850464\\
39.0083236694336 14.69970703125\\
39.029483795166 14.7006921768188\\
39.0527648925781 14.7005453109741\\
39.078369140625 14.7005472183228\\
39.1065368652344 14.7005138397217\\
39.137523651123 14.7005186080933\\
39.1716041564941 14.7005157470703\\
39.2090950012207 14.7005424499512\\
39.2477378845215 14.7005443572998\\
39.2902450561523 14.7005395889282\\
39.3370056152344 14.7005472183228\\
39.3884391784668 14.7005414962769\\
39.4450187683105 14.7005405426025\\
39.4936180114746 14.7005338668823\\
39.5406875610352 14.7005310058594\\
39.5853233337402 14.7005290985107\\
39.6272773742676 14.7005262374878\\
39.6734275817871 14.7005281448364\\
39.7241897583008 14.7005281448364\\
39.780029296875 14.7005281448364\\
39.8414497375488 14.7005281448364\\
39.9090156555176 14.7005281448364\\
39.9833374023438 14.7005281448364\\
40.0650939941406 14.7005281448364\\
40.1550216674805 14.7005271911621\\
40.2539443969727 14.7005281448364\\
40.353946685791 14.7005262374878\\
40.4539451599121 14.7005281448364\\
40.5539436340332 14.7005262374878\\
40.5569686889648 14.7005081176758\\
40.5602951049805 14.7005605697632\\
40.5639533996582 14.7004127502441\\
40.5665168762207 14.7006187438965\\
40.5693359375 14.7005729675293\\
40.572437286377 14.7005319595337\\
40.5758476257324 14.7005243301392\\
40.5796012878418 14.7005271911621\\
40.5837287902832 14.7005281448364\\
40.5882682800293 14.7005281448364\\
40.59326171875 14.7005281448364\\
40.5987548828125 14.7005281448364\\
40.6048011779785 14.7005281448364\\
40.6114463806152 14.7005271911621\\
40.6187591552734 14.7005271911621\\
40.6268005371094 14.7005271911621\\
40.6356506347656 14.7005271911621\\
40.6453819274902 14.7005271911621\\
40.6560897827148 14.7005271911621\\
40.6678657531738 14.7005271911621\\
40.6808204650879 14.7005271911621\\
40.695068359375 14.7005271911621\\
40.7107429504395 14.7005271911621\\
40.7279853820801 14.7005271911621\\
40.7469482421875 14.7005271911621\\
40.7678108215332 14.7005271911621\\
40.7907600402832 14.7005271911621\\
40.8160057067871 14.7005271911621\\
40.8437728881836 14.7005271911621\\
40.8743171691895 14.7005262374878\\
40.9079170227051 14.7005262374878\\
40.9448776245117 14.7005262374878\\
40.9855308532715 14.7005262374878\\
41.0302505493164 14.7005262374878\\
41.0794448852539 14.7005262374878\\
41.1335563659668 14.7005262374878\\
41.1930809020996 14.7005262374878\\
41.2585563659668 14.7005262374878\\
41.3305778503418 14.7005262374878\\
41.4098052978516 14.7005262374878\\
41.4969520568848 14.7005252838135\\
41.5928153991699 14.7005271911621\\
41.692813873291 14.7005252838135\\
41.7928161621094 14.7005701065063\\
41.812816619873 14.7007617950439\\
41.8206024169922 14.7009801864624\\
41.8281707763672 14.7010908126831\\
41.8364944458008 14.7011623382568\\
41.8444328308105 14.7012376785278\\
41.8531684875488 14.7012996673584\\
41.8627777099609 14.7061986923218\\
41.8733444213867 14.7791528701782\\
41.8849716186523 14.8718585968018\\
41.8977584838867 14.8491659164429\\
41.9118270874023 14.7871656417847\\
41.9272994995117 14.7786550521851\\
41.9443206787109 14.7676286697388\\
41.9630432128906 14.7564706802368\\
41.9836387634277 14.7456359863281\\
42.0062942504883 14.7246732711792\\
42.0312156677246 14.7029867172241\\
42.0477752685547 14.7017736434937\\
42.0659866333008 14.7171010971069\\
42.0860252380371 14.736870765686\\
42.1080627441406 14.7581605911255\\
42.1323051452637 14.7480888366699\\
42.1589736938477 14.7547206878662\\
42.1883087158203 14.7596273422241\\
42.2205772399902 14.7571430206299\\
42.2560729980469 14.7473649978638\\
42.2951164245605 14.7452325820923\\
42.3380661010742 14.7416925430298\\
42.3853073120117 14.7361307144165\\
42.4372749328613 14.7296190261841\\
42.494441986084 14.7303924560547\\
42.5416946411133 14.7158555984497\\
42.5936737060547 14.7134552001953\\
42.6379699707031 14.7056789398193\\
42.6777877807617 14.7021970748901\\
42.711311340332 14.7010383605957\\
42.7481842041016 14.7007741928101\\
42.788745880127 14.7007122039795\\
42.833366394043 14.700719833374\\
42.8824462890625 14.7007532119751\\
42.9364318847656 14.7008361816406\\
42.9958190917969 14.7010278701782\\
43.0611457824707 14.7014636993408\\
43.1330032348633 14.7025289535522\\
43.2120475769043 14.7052059173584\\
43.2989959716797 14.7113342285156\\
43.3946380615234 14.7210845947266\\
43.4946365356445 14.7289028167725\\
43.5946388244629 14.7317008972168\\
43.694637298584 14.7327709197998\\
43.7946357727051 14.7328157424927\\
43.8946380615234 14.7325105667114\\
43.9946365356445 14.7331809997559\\
44.0946388244629 14.7335014343262\\
44.194637298584 14.7342443466187\\
44.2946357727051 14.7346115112305\\
44.3946380615234 14.7352256774902\\
44.4946365356445 14.7356767654419\\
44.5946388244629 14.7362365722656\\
44.694637298584 14.7367238998413\\
44.7946357727051 14.7372598648071\\
44.8946380615234 14.7377634048462\\
44.9946365356445 14.7382888793945\\
45.0946388244629 14.7388038635254\\
45.194637298584 14.7393274307251\\
45.2946357727051 14.7398481369019\\
45.3946380615234 14.7403726577759\\
45.4946365356445 14.740894317627\\
45.5946388244629 14.7414178848267\\
45.694637298584 14.7419404983521\\
45.7946357727051 14.7424640655518\\
45.8946380615234 14.7429866790771\\
45.9946365356445 14.7435102462769\\
46.0946388244629 14.7443342208862\\
46.194637298584 14.7449417114258\\
46.2946357727051 14.7455739974976\\
46.3946380615234 14.7461967468262\\
46.4946365356445 14.7468194961548\\
46.5946388244629 14.7474412918091\\
46.694637298584 14.7480611801147\\
46.7946357727051 14.7486801147461\\
46.8946380615234 14.7492971420288\\
46.9946365356445 14.7499132156372\\
47.0946388244629 14.750527381897\\
47.194637298584 14.7511415481567\\
47.2946357727051 14.7517538070679\\
47.3946380615234 14.7523651123047\\
47.4946365356445 14.7529754638672\\
47.5946388244629 14.753490447998\\
47.694637298584 14.7552862167358\\
47.7946357727051 14.7552089691162\\
47.8946380615234 14.755446434021\\
47.9946365356445 14.7560691833496\\
48.0946388244629 14.7565793991089\\
48.194637298584 14.7571296691895\\
48.2946357727051 14.7576627731323\\
48.3946380615234 14.7581996917725\\
48.4946365356445 14.7587327957153\\
48.5946388244629 14.7592649459839\\
48.694637298584 14.7598257064819\\
48.7946357727051 14.76038646698\\
48.8946380615234 14.7610359191895\\
48.9946365356445 14.7616147994995\\
49.0946388244629 14.7622079849243\\
49.194637298584 14.7628412246704\\
49.2946357727051 14.7635126113892\\
49.3946380615234 14.764253616333\\
49.4946365356445 14.7696723937988\\
49.5842514038086 14.7742719650269\\
49.6559371948242 14.7742414474487\\
49.7347946166992 14.7682943344116\\
49.8215370178223 14.7686758041382\\
49.9169502258301 14.7849473953247\\
49.9568328857422 14.7322778701782\\
50.0007057189941 14.7641191482544\\
50.0206336975098 14.7774715423584\\
50.0371971130371 14.7792539596558\\
50.0554161071777 14.7709798812866\\
50.0754585266113 14.771035194397\\
50.0975074768066 14.7711772918701\\
50.1217575073242 14.7713441848755\\
50.1484336853027 14.7715482711792\\
50.1777763366699 14.7717924118042\\
50.2100563049316 14.7720775604248\\
50.2455635070801 14.772406578064\\
50.2846183776855 14.7727813720703\\
50.327579498291 14.7732048034668\\
50.3748397827148 14.7736787796021\\
50.4268226623535 14.7742052078247\\
50.4840087890625 14.7747869491577\\
50.5469093322754 14.7754287719727\\
50.6161003112793 14.7761335372925\\
50.692211151123 14.7769088745117\\
50.7759323120117 14.7870187759399\\
50.8680267333984 14.7699089050293\\
50.9680252075195 14.7797927856445\\
51.0680236816406 14.783164024353\\
51.168025970459 14.788685798645\\
51.2180252075195 14.8489780426025\\
51.2553977966309 14.8408527374268\\
51.2579727172852 14.8321266174316\\
51.2608032226563 14.835545539856\\
51.263916015625 14.8370866775513\\
51.2673416137695 14.834867477417\\
51.2711067199707 14.8292350769043\\
51.2752532958984 14.8212156295776\\
51.2798080444336 14.8132123947144\\
51.2848243713379 14.8067646026611\\
51.290340423584 14.8009672164917\\
51.2964057922363 14.7944812774658\\
51.3030815124512 14.7863349914551\\
51.3104209899902 14.7742118835449\\
51.3184967041016 14.7576150894165\\
51.3273811340332 14.7359523773193\\
51.3371505737305 14.7189950942993\\
51.347900390625 14.702374458313\\
51.3597221374512 14.7024259567261\\
51.372730255127 14.7020959854126\\
51.387035369873 14.7024116516113\\
51.4027709960938 14.7044887542725\\
51.4200820922852 14.7111415863037\\
51.4391212463379 14.7704486846924\\
51.4600677490234 14.8143939971924\\
51.4831085205078 14.830924987793\\
51.5084533691406 14.8143577575684\\
51.5363311767578 14.8155975341797\\
51.5669975280762 14.8189573287964\\
51.6007308959961 14.8186197280884\\
51.6378364562988 14.814754486084\\
51.6764678955078 14.8100414276123\\
51.7189598083496 14.8064975738525\\
51.7657051086426 14.8036890029907\\
51.8171195983887 14.8013172149658\\
51.8736801147461 14.8002977371216\\
51.9213600158691 14.8007049560547\\
51.9687728881836 14.7992343902588\\
52.0141105651855 14.7983493804932\\
52.056568145752 14.7995862960815\\
52.103271484375 14.8000469207764\\
52.1546440124512 14.7984180450439\\
52.2111549377441 14.7955274581909\\
52.2733154296875 14.7955636978149\\
52.3416938781738 14.7958850860596\\
52.4169082641602 14.7962236404419\\
52.4996490478516 14.7968072891235\\
52.5906600952148 14.7996883392334\\
52.6906585693359 14.796217918396\\
52.790657043457 14.8017539978027\\
52.8906593322754 14.7979955673218\\
52.9906578063965 14.8005018234253\\
52.9936103820801 14.9109401702881\\
52.9968528747559 14.7518138885498\\
53.0004234313965 14.7696018218994\\
53.0029792785645 14.8241014480591\\
53.0057907104492 14.8213682174683\\
53.0088844299316 14.803689956665\\
53.0122871398926 14.8025331497192\\
53.0160293579102 14.8029232025146\\
53.0201454162598 14.8030042648315\\
53.0246734619141 14.8029899597168\\
53.0296516418457 14.8029842376709\\
53.0351333618164 14.8029861450195\\
53.0411605834961 14.8029918670654\\
53.0477867126465 14.8030033111572\\
53.0550804138184 14.8030204772949\\
53.063102722168 14.8030462265015\\
53.0719261169434 14.8030805587769\\
53.0816307067871 14.8031244277954\\
53.0923080444336 14.8031816482544\\
53.1040496826172 14.8032512664795\\
53.1169700622559 14.8033361434937\\
53.1311798095703 14.8034372329712\\
53.1468124389648 14.8035583496094\\
53.164005279541 14.8036994934082\\
53.1829223632813 14.8038625717163\\
53.203727722168 14.8040494918823\\
53.2266120910645 14.8042631149292\\
53.2517852783203 14.8045043945313\\
53.2794799804688 14.8047752380371\\
53.3099403381348 14.8050775527954\\
53.343448638916 14.8054141998291\\
53.3803062438965 14.8057870864868\\
53.4208488464355 14.8062000274658\\
53.465446472168 14.8066549301147\\
53.5145034790039 14.8071556091309\\
53.5684700012207 14.8077077865601\\
53.6278305053711 14.8083143234253\\
53.6931266784668 14.8089828491211\\
53.7649536132813 14.8097171783447\\
53.8439598083496 14.810525894165\\
53.9308700561523 14.8133811950684\\
54.0264701843262 14.8103151321411\\
54.1264686584473 14.8135709762573\\
54.2264709472656 14.8217077255249\\
54.2464714050293 14.8250160217285\\
54.2542266845703 14.8512182235718\\
54.2617721557617 14.8551111221313\\
54.2700691223145 14.8533182144165\\
54.2779922485352 14.8512687683105\\
54.2867050170898 14.8518743515015\\
54.2962913513184 14.8538846969604\\
54.3068389892578 14.8787212371826\\
54.3184356689453 14.9152097702026\\
54.3311958312988 14.9203720092773\\
54.3452301025391 14.9009389877319\\
54.360668182373 14.882604598999\\
54.3776512145996 14.8632125854492\\
54.3963317871094 14.8436717987061\\
54.4168815612793 14.8245344161987\\
54.4394836425781 14.8015565872192\\
54.4643478393555 14.7743549346924\\
54.4809799194336 14.7637548446655\\
54.4992752075195 14.7657690048218\\
54.519401550293 14.777645111084\\
54.5415382385254 14.7975168228149\\
54.5658912658691 14.8054533004761\\
54.5926780700684 14.839054107666\\
54.622142791748 14.8449716567993\\
54.6545562744141 14.8444499969482\\
54.6902084350586 14.8370504379272\\
54.7294273376465 14.8392009735107\\
54.7725677490234 14.8420810699463\\
54.8200225830078 14.8417549133301\\
54.8722229003906 14.8352403640747\\
54.9296417236328 14.8360805511475\\
54.9768371582031 14.8257675170898\\
55.028751373291 14.824049949646\\
55.0732498168945 14.8217725753784\\
55.112865447998 14.8232889175415\\
55.1461601257324 14.8298788070679\\
55.1827850341797 14.83069896698\\
55.223072052002 14.8301868438721\\
55.2673873901367 14.8285522460938\\
55.3161315917969 14.828987121582\\
55.3697547912598 14.8293952941895\\
55.4287376403809 14.829740524292\\
55.4936180114746 14.8303833007813\\
55.5649871826172 14.8314733505249\\
55.643497467041 14.8335475921631\\
55.7298545837402 14.8374443054199\\
55.8248481750488 14.8419904708862\\
55.9248466491699 14.8426103591919\\
56.024845123291 14.8399467468262\\
56.1248474121094 14.8391752243042\\
56.2248458862305 14.8386297225952\\
56.3248481750488 14.8383846282959\\
56.4248466491699 14.839714050293\\
56.524845123291 14.8406162261963\\
56.6248474121094 14.8420209884644\\
56.7248458862305 14.8429374694824\\
56.8248481750488 14.8441762924194\\
56.9248466491699 14.8452033996582\\
57.024845123291 14.8463716506958\\
57.1248474121094 14.8474473953247\\
57.2248458862305 14.8485841751099\\
57.3248481750488 14.8496809005737\\
57.4248466491699 14.8508052825928\\
57.524845123291 14.8519115447998\\
57.6248474121094 14.8530445098877\\
57.7248458862305 14.8541440963745\\
57.8248481750488 14.8552665710449\\
57.9248466491699 14.8563852310181\\
58.024845123291 14.8575067520142\\
58.1248474121094 14.8586263656616\\
58.2248458862305 14.8597478866577\\
58.3248481750488 14.8608675003052\\
58.4248466491699 14.861988067627\\
58.524845123291 14.8633699417114\\
58.6248474121094 14.8645133972168\\
58.7248458862305 14.8656301498413\\
58.8248481750488 14.8667526245117\\
58.9248466491699 14.8678731918335\\
59.024845123291 14.8689937591553\\
59.1248474121094 14.8701152801514\\
59.2248458862305 14.8712358474731\\
59.3248481750488 14.8723573684692\\
59.4248466491699 14.873477935791\\
59.524845123291 14.8745994567871\\
59.6248474121094 14.8757209777832\\
59.7248458862305 14.876841545105\\
59.8248481750488 14.8779630661011\\
59.9124221801758 14.8790941238403\\
60 14.8800573348999\\
60.0963325500488 14.8822031021118\\
60.1963348388672 14.8822650909424\\
60.2963333129883 14.8829784393311\\
60.3963356018066 14.8841981887817\\
60.4963340759277 14.8852758407593\\
60.5963325500488 14.8864068984985\\
60.6963348388672 14.8875169754028\\
60.7963333129883 14.8886346817017\\
60.8963356018066 14.8897504806519\\
60.9963340759277 14.8908662796021\\
61.0963325500488 14.891996383667\\
61.1963348388672 14.8931427001953\\
61.2963333129883 14.8943433761597\\
61.3963356018066 14.8954811096191\\
61.4963340759277 14.8966236114502\\
61.5963325500488 14.8977899551392\\
61.6963348388672 14.8989782333374\\
61.7963333129883 14.9002065658569\\
61.8963356018066 14.9034605026245\\
61.9940414428711 14.9093980789185\\
62.0722236633301 14.913535118103\\
62.1347732543945 14.9041967391968\\
62.2035751342773 14.9219427108765\\
62.2478866577148 14.8794136047363\\
62.296630859375 14.8943328857422\\
62.3256530761719 14.9155149459839\\
62.3496742248535 15.0559053421021\\
62.376091003418 14.7837495803833\\
62.4051551818848 14.9035120010376\\
62.4213905334473 14.9110050201416\\
62.4340591430664 14.911416053772\\
62.4479942321777 14.9080333709717\\
62.463321685791 14.9080629348755\\
62.4801826477051 14.9081792831421\\
62.4987297058105 14.9083013534546\\
62.5191307067871 14.9084463119507\\
62.5415687561035 14.908618927002\\
62.5662574768066 14.9088201522827\\
62.5934104919434 14.9090538024902\\
62.6232795715332 14.9093227386475\\
62.656135559082 14.909631729126\\
62.6922760009766 14.9099817276001\\
62.7320327758789 14.9103765487671\\
62.775764465332 14.9108180999756\\
62.8238677978516 14.9113092422485\\
62.8767852783203 14.9118518829346\\
62.9349899291992 14.9124488830566\\
62.9990196228027 14.913106918335\\
63.069450378418 14.9138288497925\\
63.1469230651855 14.9146223068237\\
63.2321434020996 14.9248991012573\\
63.3258857727051 14.9076042175293\\
63.4258842468262 14.9176263809204\\
63.5258865356445 14.9230699539185\\
63.6258850097656 14.9235849380493\\
63.6758842468262 14.9668941497803\\
63.7138481140137 14.9165487289429\\
63.7273941040039 14.8847370147705\\
63.7422943115234 14.9021396636963\\
63.7586822509766 14.9036493301392\\
63.7767105102539 14.8820295333862\\
63.7965431213379 14.8868494033813\\
63.8183555603027 14.9034280776978\\
63.8423538208008 14.9070377349854\\
63.8687477111816 14.9038362503052\\
63.8977813720703 14.9031391143799\\
63.9297218322754 14.900107383728\\
63.9648513793945 14.8956050872803\\
64.0015869140625 14.8962736129761\\
64.022705078125 14.9018630981445\\
64.0459365844727 14.9185428619385\\
64.0714874267578 14.9240531921387\\
64.0996017456055 14.9314413070679\\
64.1305236816406 14.9398632049561\\
64.1645355224609 14.9386329650879\\
64.2019424438477 14.9341726303101\\
64.2431030273438 14.933087348938\\
64.2883682250977 14.9328126907349\\
64.3381729125977 14.9325838088989\\
64.3929443359375 14.9325952529907\\
64.4477233886719 14.9335403442383\\
64.4993057250977 14.9338874816895\\
64.5481796264648 14.9340543746948\\
64.5940780639648 14.9343299865723\\
64.6370315551758 14.9363470077515\\
64.6842803955078 14.936824798584\\
64.7362518310547 14.9353351593018\\
64.7934188842773 14.9334020614624\\
64.8563079833984 14.9337282180786\\
64.9254837036133 14.9342842102051\\
65.0015716552734 14.9348106384277\\
65.0852737426758 14.9355506896973\\
65.1773452758789 14.9384107589722\\
65.27734375 14.9353437423706\\
65.3773422241211 14.9406986236572\\
65.4773483276367 14.9372816085815\\
65.5773468017578 14.9395217895508\\
65.5794677734375 15.064887046814\\
65.5817947387695 14.9199590682983\\
65.5843658447266 14.9147567749023\\
65.5871810913086 15.0061511993408\\
65.5902862548828 14.9013032913208\\
65.5937042236328 14.9135456085205\\
65.5974578857422 14.9517202377319\\
65.6015930175781 14.95285987854\\
65.6061325073242 14.9434089660645\\
65.6111373901367 14.9426651000977\\
65.6166305541992 14.9429340362549\\
65.6226806640625 14.9429655075073\\
65.6293334960938 14.9429616928101\\
65.6366577148438 14.942967414856\\
65.6447067260742 14.942982673645\\
65.653564453125 14.9430065155029\\
65.6633071899414 14.9430418014526\\
65.6740188598633 14.9430885314941\\
65.6858062744141 14.9431505203247\\
65.6987762451172 14.94322681427\\
65.7130355834961 14.9433221817017\\
65.7287292480469 14.9434356689453\\
65.7459869384766 14.9435710906982\\
65.7649688720703 14.9437303543091\\
65.7858505249023 14.9439144134521\\
65.8088226318359 14.9441251754761\\
65.8340911865234 14.9443645477295\\
65.8618850708008 14.944634437561\\
65.892463684082 14.9449367523193\\
65.9260940551758 14.945273399353\\
65.9630889892578 14.9456472396851\\
66.0037841796875 14.9460611343384\\
66.0485458374023 14.9465169906616\\
66.097785949707 14.9470205307007\\
66.1519470214844 14.9475746154785\\
66.2115325927734 14.9481840133667\\
66.2770690917969 14.9488544464111\\
66.3491592407227 14.9495916366577\\
66.428466796875 14.950403213501\\
66.5156936645508 14.9531660079956\\
66.6116485595703 14.9503021240234\\
66.7116470336914 14.9534530639648\\
66.811653137207 14.9628276824951\\
66.8316497802734 14.9748697280884\\
66.8391723632813 15.0049858093262\\
66.8464965820313 15.0095405578613\\
66.8545455932617 15.0056753158569\\
66.8622512817383 15.0019960403442\\
66.870735168457 15.017276763916\\
66.8800659179688 15.0606737136841\\
66.8903274536133 15.077428817749\\
66.9016189575195 15.0629272460938\\
66.9140319824219 15.0385961532593\\
66.9276962280273 15.0176420211792\\
66.9427185058594 14.99440574646\\
66.9592437744141 14.9694194793701\\
66.977424621582 14.9432516098022\\
66.997428894043 14.9040689468384\\
67.0194244384766 14.8775634765625\\
67.0415725708008 14.8604221343994\\
67.0659408569336 14.8533573150635\\
67.0927429199219 14.853720664978\\
67.1222229003906 14.8836603164673\\
67.154655456543 14.9302635192871\\
67.1903305053711 15.0067100524902\\
67.1995849609375 15.0003862380981\\
67.2097702026367 14.9834222793579\\
67.2209701538086 14.9749460220337\\
67.233283996582 14.9715509414673\\
67.2468414306641 14.9701957702637\\
67.2617416381836 14.969687461853\\
67.2781448364258 14.9697160720825\\
67.2961807250977 14.9704971313477\\
67.3160247802734 14.9712572097778\\
67.3378448486328 14.9665203094482\\
67.3618545532227 14.9694957733154\\
67.3882598876953 14.9670648574829\\
67.4173126220703 14.9683532714844\\
67.4492645263672 14.9674634933472\\
67.4844131469727 14.9649982452393\\
67.523078918457 14.9631719589233\\
67.5656127929688 14.9634237289429\\
67.6123962402344 14.962194442749\\
67.6579818725586 14.9600067138672\\
67.6993789672852 14.9584007263184\\
67.7355346679688 14.9601707458496\\
67.7753067016602 14.9631242752075\\
67.8190612792969 14.9652233123779\\
67.8671875 14.9658784866333\\
67.9201278686523 14.9679298400879\\
67.9783630371094 14.9681406021118\\
68.0421829223633 14.9705514907837\\
68.1070251464844 14.9733009338379\\
68.178352355957 14.9767122268677\\
68.2568130493164 14.9798946380615\\
68.3431243896484 14.9774837493896\\
68.4380569458008 14.97691822052\\
68.5380630493164 14.976450920105\\
68.6380615234375 14.9760332107544\\
68.7380599975586 14.9757709503174\\
68.8380584716797 14.9768419265747\\
68.9380569458008 14.9778070449829\\
69.0380630493164 14.9791898727417\\
69.1380615234375 14.9801445007324\\
69.2380599975586 14.9813814163208\\
69.3380584716797 14.9824333190918\\
69.4380569458008 14.9836082458496\\
69.5380630493164 14.9847021102905\\
69.6380615234375 14.9858503341675\\
69.7380599975586 14.9869623184204\\
69.8380584716797 14.9880990982056\\
69.9380569458008 14.9892349243164\\
70.0380630493164 14.9903469085693\\
70.1380615234375 14.9914712905884\\
70.2380599975586 14.9925994873047\\
70.3380584716797 14.993724822998\\
70.4380569458008 14.9948530197144\\
70.5380630493164 14.995979309082\\
70.6380615234375 14.997106552124\\
70.7380599975586 14.9982328414917\\
70.8380584716797 14.9993600845337\\
70.9380569458008 15.0006370544434\\
71.0380630493164 15.0019283294678\\
71.1380615234375 15.0030126571655\\
71.2380599975586 15.0041522979736\\
71.3380584716797 15.0052757263184\\
71.4380569458008 15.0064039230347\\
71.5380630493164 15.0075311660767\\
71.6380615234375 15.0086584091187\\
71.7380599975586 15.0097856521606\\
71.8380584716797 15.0109128952026\\
71.9380569458008 15.0120401382446\\
72.0380630493164 15.0131673812866\\
72.1380615234375 15.0142946243286\\
72.2380599975586 15.0154218673706\\
72.3380584716797 15.0165481567383\\
72.4380569458008 15.017674446106\\
72.5380630493164 15.0199365615845\\
72.6380615234375 15.0200386047363\\
72.7380599975586 15.0207920074463\\
72.8380584716797 15.0220060348511\\
72.9380569458008 15.0230932235718\\
73.0380630493164 15.024227142334\\
73.1380615234375 15.0253438949585\\
73.2380599975586 15.0264663696289\\
73.3380584716797 15.0275859832764\\
73.4380569458008 15.0287065505981\\
73.5380630493164 15.0298404693604\\
73.6380615234375 15.0309925079346\\
73.7380599975586 15.0321969985962\\
73.8380584716797 15.0333395004272\\
73.9380569458008 15.0344858169556\\
74.0380630493164 15.0356569290161\\
74.1380615234375 15.0368480682373\\
74.2380599975586 15.0380792617798\\
74.3380584716797 15.0411348342896\\
74.4363632202148 15.0472364425659\\
74.5150299072266 15.0513534545898\\
74.5779571533203 15.0424566268921\\
74.6471862792969 15.0428123474121\\
74.7233276367188 15.0433473587036\\
74.8070907592773 15.0439805984497\\
74.8992309570313 15.060619354248\\
74.945930480957 15.0148687362671\\
74.9729614257813 15.0386505126953\\
74.9864807128906 15.0545206069946\\
75 15.0554313659668\\
};
\addlegendentry{14 "};

\addplot [
color=green,
solid
]
table[row sep=crcr]{
0 14.7004346847534\\
9.99999971718069e-10 14.7004346847534\\
2.49999998480632e-09 14.7004346847534\\
4.74999994892755e-09 14.7004346847534\\
8.1250002281763e-09 14.7004346847534\\
1.31874999809156e-08 14.7004346847534\\
2.0781250498203e-08 14.7004346847534\\
3.21718758300449e-08 14.7004346847534\\
4.92578138278077e-08 14.7004346847534\\
7.48867208244519e-08 14.7004346847534\\
1.13330081319418e-07 14.7004346847534\\
1.70995122061868e-07 14.7004346847534\\
2.57492672517401e-07 14.7004346847534\\
3.8723902662241e-07 14.7004346847534\\
5.81858500936505e-07 14.7004346847534\\
8.73787769251066e-07 14.7004346847534\\
1.31168167172291e-06 14.7004346847534\\
1.9685226106958e-06 14.7004346847534\\
2.95378367809462e-06 14.7004346847534\\
4.43167573394021e-06 14.7004346847534\\
6.64851359033491e-06 14.7004346847534\\
9.97376992017962e-06 14.7004346847534\\
1.49616553244414e-05 14.7004346847534\\
2.24434825213393e-05 14.7004346847534\\
3.36662233166862e-05 14.7004346847534\\
5.05003372381907e-05 14.7004356384277\\
7.57515081204474e-05 14.7004356384277\\
0.000113628258986864 14.7004356384277\\
0.000170443381648511 14.7004365921021\\
0.000255666091106832 14.7004375457764\\
0.000383500126190484 14.7004384994507\\
0.000575251178815961 14.7004404067993\\
0.000862877757754177 14.7004413604736\\
0.00129431765526533 14.7004413604736\\
0.00194147753063589 14.7004375457764\\
0.00291221728548408 14.7004327774048\\
0.00436832662671804 14.7004318237305\\
0.00597004732117057 14.7004327774048\\
0.00773194013163447 14.7004346847534\\
0.009670021943748 14.7004356384277\\
0.0118019115179777 14.7004356384277\\
0.0141469910740852 14.7004346847534\\
0.0167265776544809 14.7004346847534\\
0.0195641238242388 14.7004346847534\\
0.0226854234933853 14.7004346847534\\
0.0261188540607691 14.7004346847534\\
0.0298956278711557 14.7004346847534\\
0.0340500771999359 14.7004346847534\\
0.0386199727654457 14.7004346847534\\
0.0436468608677387 14.7004346847534\\
0.0491764321923256 14.7004346847534\\
0.0552589632570744 14.7004346847534\\
0.0619497485458851 14.7004346847534\\
0.0693096145987511 14.7004346847534\\
0.0774054601788521 14.7004346847534\\
0.0863108932971954 14.7004337310791\\
0.0961068719625473 14.7004346847534\\
0.106882445514202 14.7004346847534\\
0.118735581636429 14.7004346847534\\
0.13177402317524 14.7004346847534\\
0.146116316318512 14.7004346847534\\
0.161892831325531 14.7004346847534\\
0.179247006773949 14.7004346847534\\
0.198336601257324 14.7004346847534\\
0.21933513879776 14.7004346847534\\
0.242433547973633 14.7004346847534\\
0.267841786146164 14.7004346847534\\
0.29579085111618 14.7004346847534\\
0.32653483748436 14.7004346847534\\
0.360353201627731 14.7004346847534\\
0.397553414106369 14.7004346847534\\
0.438473641872406 14.7004346847534\\
0.483485877513886 14.7004346847534\\
0.532999336719513 14.7004346847534\\
0.587464153766632 14.7004346847534\\
0.647375464439392 14.7004346847534\\
0.71327793598175 14.7004346847534\\
0.7857705950737 14.7004346847534\\
0.865512549877167 14.7004346847534\\
0.953228712081909 14.7004346847534\\
1.04971647262573 14.7004346847534\\
1.14971649646759 14.7004346847534\\
1.24971640110016 14.7004346847534\\
1.34971642494202 14.7004346847534\\
1.44971644878387 14.7004346847534\\
1.54971647262573 14.7004346847534\\
1.64971649646759 14.7004346847534\\
1.74971640110016 14.7004346847534\\
1.84971642494202 14.7004346847534\\
1.94971644878387 14.7004346847534\\
2.04971647262573 14.7004346847534\\
2.1497163772583 14.7004346847534\\
2.24971652030945 14.7004346847534\\
2.34971642494202 14.7004346847534\\
2.44971656799316 14.7004346847534\\
2.54971647262573 14.7004346847534\\
2.6497163772583 14.7004346847534\\
2.74971652030945 14.7004346847534\\
2.84971642494202 14.7004346847534\\
2.94971656799316 14.7004346847534\\
3.04971647262573 14.7004346847534\\
3.1497163772583 14.7004346847534\\
3.24971652030945 14.7004346847534\\
3.34971642494202 14.7004346847534\\
3.44971656799316 14.7004346847534\\
3.54971647262573 14.7004346847534\\
3.6497163772583 14.7004346847534\\
3.74971652030945 14.7004346847534\\
3.84971642494202 14.7004346847534\\
3.94971656799316 14.7004346847534\\
4.04971647262573 14.7004346847534\\
4.1497163772583 14.7004346847534\\
4.24971628189087 14.7004346847534\\
4.3497166633606 14.7004346847534\\
4.44971656799316 14.7004346847534\\
4.54971647262573 14.7004346847534\\
4.6497163772583 14.7004346847534\\
4.74971628189087 14.7004346847534\\
4.8497166633606 14.7004346847534\\
4.94971656799316 14.7004346847534\\
5.04971647262573 14.7004346847534\\
5.1497163772583 14.7004346847534\\
5.24971628189087 14.7004346847534\\
5.3497166633606 14.7004346847534\\
5.44971656799316 14.7004346847534\\
5.54971647262573 14.7004346847534\\
5.6497163772583 14.7004346847534\\
5.74971628189087 14.7004346847534\\
5.8497166633606 14.7004346847534\\
5.94971656799316 14.7004346847534\\
6.04971647262573 14.7004346847534\\
6.1497163772583 14.7004337310791\\
6.24971628189087 14.7004327774048\\
6.3497166633606 14.7004327774048\\
6.44971656799316 14.7004337310791\\
6.54971647262573 14.7004356384277\\
6.6497163772583 14.7004356384277\\
6.74971628189087 14.7004356384277\\
6.8497166633606 14.7004346847534\\
6.94971656799316 14.7004346847534\\
7.04971647262573 14.7004346847534\\
7.1497163772583 14.7004346847534\\
7.24971628189087 14.7004346847534\\
7.3497166633606 14.7004346847534\\
7.44971656799316 14.7004346847534\\
7.54971647262573 14.7004346847534\\
7.6497163772583 14.7004346847534\\
7.74971628189087 14.7004346847534\\
7.8497166633606 14.7004346847534\\
7.94971656799316 14.7004346847534\\
8.04971599578857 14.7004346847534\\
8.1497163772583 14.7004346847534\\
8.24971675872803 14.7004346847534\\
8.34971618652344 14.7004346847534\\
8.44971656799316 14.7004346847534\\
8.54971599578857 14.7004346847534\\
8.6497163772583 14.7004346847534\\
8.74971675872803 14.7004346847534\\
8.84971618652344 14.7004346847534\\
8.94971656799316 14.7004346847534\\
9.04971599578857 14.7004346847534\\
9.1497163772583 14.7004346847534\\
9.24971675872803 14.7004346847534\\
9.34971618652344 14.7004346847534\\
9.44971656799316 14.7004346847534\\
9.54971599578857 14.7004346847534\\
9.6497163772583 14.7004346847534\\
9.74971675872803 14.7004346847534\\
9.84971618652344 14.7004346847534\\
9.94971656799316 14.7004346847534\\
10.0497159957886 14.7004346847534\\
10.1497163772583 14.7004346847534\\
10.249716758728 14.7004346847534\\
10.3497161865234 14.7004346847534\\
10.4497165679932 14.7004346847534\\
10.5497159957886 14.7004346847534\\
10.6497163772583 14.7004346847534\\
10.749716758728 14.7004346847534\\
10.8497161865234 14.7004346847534\\
10.9497165679932 14.7004346847534\\
11.0497159957886 14.7004346847534\\
11.1497163772583 14.7004346847534\\
11.249716758728 14.7004346847534\\
11.3497161865234 14.7004346847534\\
11.4497165679932 14.7004346847534\\
11.5497159957886 14.7004346847534\\
11.6497163772583 14.7004346847534\\
11.749716758728 14.7004346847534\\
11.8497161865234 14.7004346847534\\
11.9497165679932 14.7004346847534\\
12.0497159957886 14.7004346847534\\
12.1497163772583 14.7004346847534\\
12.249716758728 14.7004346847534\\
12.347146987915 14.7004346847534\\
12.4250946044922 14.7004346847534\\
12.4874591827393 14.7004356384277\\
12.5560598373413 14.7004346847534\\
12.6315202713013 14.7004337310791\\
12.714527130127 14.7004346847534\\
12.8058338165283 14.7004346847534\\
12.8497476577759 14.7004356384277\\
12.8980522155762 14.7004337310791\\
12.9143037796021 14.7004346847534\\
12.9277105331421 14.7004356384277\\
12.9424571990967 14.7004356384277\\
12.9586782455444 14.7004346847534\\
12.9765224456787 14.7004346847534\\
12.9961500167847 14.7004346847534\\
13.0177412033081 14.7004346847534\\
13.0414915084839 14.7004346847534\\
13.0676164627075 14.7004346847534\\
13.0963535308838 14.7004346847534\\
13.1279649734497 14.7004346847534\\
13.1627368927002 14.7004346847534\\
13.2009868621826 14.7004346847534\\
13.2430610656738 14.7004346847534\\
13.289342880249 14.7004346847534\\
13.3402528762817 14.7004346847534\\
13.3962545394897 14.7004346847534\\
13.4578561782837 14.7004346847534\\
13.5256175994873 14.7004346847534\\
13.600154876709 14.7004346847534\\
13.6821460723877 14.7004337310791\\
13.7723369598389 14.7004346847534\\
13.871545791626 14.7004346847534\\
13.9715461730957 14.7004547119141\\
14.0715456008911 14.7005414962769\\
14.1215467453003 14.7009677886963\\
14.1629066467285 14.7008543014526\\
14.2008075714111 14.6995964050293\\
14.202446937561 14.6941738128662\\
14.2042503356934 14.6949996948242\\
14.2062339782715 14.6989135742188\\
14.2084159851074 14.7010278701782\\
14.2108154296875 14.7012395858765\\
14.2134561538696 14.7015466690063\\
14.2163600921631 14.7016143798828\\
14.219554901123 14.7010021209717\\
14.223069190979 14.7012205123901\\
14.2269344329834 14.701117515564\\
14.2311868667603 14.7008743286133\\
14.2358636856079 14.7006635665894\\
14.2410087585449 14.7005386352539\\
14.2466678619385 14.7004699707031\\
14.252893447876 14.7004146575928\\
14.2597417831421 14.7003593444824\\
14.2672748565674 14.7003078460693\\
14.2755603790283 14.7002668380737\\
14.2846755981445 14.7002372741699\\
14.2947015762329 14.7002172470093\\
14.3057298660278 14.7002105712891\\
14.3178615570068 14.7002182006836\\
14.3312063217163 14.7003555297852\\
14.3458862304688 14.7004194259644\\
14.3620328903198 14.7003860473633\\
14.3797950744629 14.7003345489502\\
14.3993330001831 14.700306892395\\
14.4208250045776 14.7003078460693\\
14.444465637207 14.7003221511841\\
14.4704713821411 14.7004919052124\\
14.4990768432617 14.7004690170288\\
14.5269069671631 14.700403213501\\
14.5575199127197 14.700421333313\\
14.591194152832 14.7004299163818\\
14.6282358169556 14.7004356384277\\
14.668981552124 14.7004375457764\\
14.7138023376465 14.700439453125\\
14.7631044387817 14.7004346847534\\
14.8173370361328 14.7004346847534\\
14.8769931793213 14.7004346847534\\
14.9244384765625 14.7004375457764\\
14.971435546875 14.7004356384277\\
15.0163955688477 14.7004356384277\\
15.0584926605225 14.7004346847534\\
15.09974193573 14.7004346847534\\
15.1451168060303 14.7004346847534\\
15.1950283050537 14.7004346847534\\
15.2499313354492 14.7004346847534\\
15.31032371521 14.7004356384277\\
15.3767566680908 14.7004346847534\\
15.4498319625854 14.7004346847534\\
15.5302152633667 14.7004346847534\\
15.6186370849609 14.7004337310791\\
15.7159004211426 14.7004346847534\\
15.8159008026123 14.7004337310791\\
15.9159002304077 14.7004346847534\\
16.0159015655518 14.7004337310791\\
16.0210952758789 14.7004318237305\\
16.0268077850342 14.7004384994507\\
16.0330924987793 14.7004327774048\\
16.0382099151611 14.7004356384277\\
16.0438404083252 14.7004356384277\\
16.0500316619873 14.7004356384277\\
16.0568428039551 14.7004356384277\\
16.0643348693848 14.7004356384277\\
16.0725765228271 14.7004356384277\\
16.0816440582275 14.7004346847534\\
16.0916156768799 14.7004346847534\\
16.1025848388672 14.7004346847534\\
16.114652633667 14.7004346847534\\
16.1279258728027 14.7004346847534\\
16.1425266265869 14.7004346847534\\
16.1585865020752 14.7004346847534\\
16.1762542724609 14.7004346847534\\
16.1956882476807 14.7004346847534\\
16.2170658111572 14.7004346847534\\
16.2405796051025 14.7004346847534\\
16.2664451599121 14.7004346847534\\
16.2948989868164 14.7004346847534\\
16.3261966705322 14.7004346847534\\
16.3606243133545 14.7004346847534\\
16.398494720459 14.7004346847534\\
16.4401531219482 14.7004346847534\\
16.4859771728516 14.7004346847534\\
16.5363826751709 14.7004346847534\\
16.5918292999268 14.7004346847534\\
16.6528205871582 14.7004346847534\\
16.7199096679688 14.7004346847534\\
16.7937088012695 14.7004346847534\\
16.8748874664307 14.7004346847534\\
16.9641857147217 14.7004337310791\\
17.0624122619629 14.7004346847534\\
17.1624126434326 14.7004337310791\\
17.2624130249023 14.7004623413086\\
17.282413482666 14.70055103302\\
17.2906188964844 14.7006196975708\\
17.2996425628662 14.7006464004517\\
17.308614730835 14.7006597518921\\
17.3184852600098 14.7006587982178\\
17.3293399810791 14.7006311416626\\
17.3412818908691 14.7005891799927\\
17.3544178009033 14.7000951766968\\
17.3688678741455 14.6995115280151\\
17.3847618103027 14.7007961273193\\
17.4022445678711 14.7005004882813\\
17.4214763641357 14.7007150650024\\
17.4426326751709 14.7006597518921\\
17.4659042358398 14.7006034851074\\
17.4915008544922 14.7004995346069\\
17.5196590423584 14.7003698348999\\
17.5506324768066 14.7003173828125\\
17.5847034454346 14.700291633606\\
17.6221809387207 14.7003040313721\\
17.6634063720703 14.7003355026245\\
17.7087554931641 14.7003726959229\\
17.758638381958 14.7004280090332\\
17.8135089874268 14.7004318237305\\
17.8738670349121 14.7004404067993\\
17.9402618408203 14.7004365921021\\
18.0132961273193 14.7004365921021\\
18.0585689544678 14.7004432678223\\
18.1083717346191 14.7004413604736\\
18.1509475708008 14.7004432678223\\
18.1891002655029 14.7004356384277\\
18.2310676574707 14.7004251480103\\
18.2772312164307 14.7004241943359\\
18.328010559082 14.7004270553589\\
18.3838691711426 14.7004384994507\\
18.4453125 14.7004337310791\\
18.5129013061523 14.7004337310791\\
18.5872478485107 14.7004346847534\\
18.6690311431885 14.7004346847534\\
18.7589912414551 14.7004346847534\\
18.857946395874 14.7004346847534\\
18.9579467773438 14.7004346847534\\
19.0579471588135 14.7004346847534\\
19.1579456329346 14.7004346847534\\
19.2579460144043 14.7004346847534\\
19.357946395874 14.7004346847534\\
19.4579467773438 14.7004346847534\\
19.5579471588135 14.7004346847534\\
19.6579456329346 14.7004346847534\\
19.7579460144043 14.7004346847534\\
19.857946395874 14.7004346847534\\
19.9579467773438 14.7004346847534\\
20.0579471588135 14.7004346847534\\
20.1579456329346 14.7004346847534\\
20.2579460144043 14.7004346847534\\
20.357946395874 14.7004346847534\\
20.4579467773438 14.7004346847534\\
20.5579471588135 14.7004346847534\\
20.6579456329346 14.7004346847534\\
20.7579460144043 14.7004346847534\\
20.857946395874 14.7004346847534\\
20.9579467773438 14.7004346847534\\
21.0579471588135 14.7004346847534\\
21.1579456329346 14.7004346847534\\
21.2579460144043 14.7004346847534\\
21.357946395874 14.7004346847534\\
21.4579467773438 14.7004346847534\\
21.5579471588135 14.7004346847534\\
21.6579456329346 14.7004346847534\\
21.7579460144043 14.7004346847534\\
21.857946395874 14.7004346847534\\
21.9579467773438 14.7004346847534\\
22.0579471588135 14.7004346847534\\
22.1579456329346 14.7004346847534\\
22.2579460144043 14.7004346847534\\
22.357946395874 14.7004346847534\\
22.4579467773438 14.7004346847534\\
22.5579471588135 14.7004346847534\\
22.6579456329346 14.7004346847534\\
22.7579460144043 14.7004346847534\\
22.857946395874 14.7004346847534\\
22.9579467773438 14.7004346847534\\
23.0579471588135 14.7004346847534\\
23.1579456329346 14.7004346847534\\
23.2579460144043 14.7004346847534\\
23.357946395874 14.7004346847534\\
23.4579467773438 14.7004346847534\\
23.5579471588135 14.7004346847534\\
23.6579456329346 14.7004346847534\\
23.7579460144043 14.7004346847534\\
23.857946395874 14.7004346847534\\
23.9579467773438 14.7004346847534\\
24.0579471588135 14.7004337310791\\
24.1579456329346 14.7004346847534\\
24.2579460144043 14.7004346847534\\
24.357946395874 14.7004346847534\\
24.4579467773438 14.7004346847534\\
24.5579471588135 14.7004346847534\\
24.6579456329346 14.7004346847534\\
24.7421741485596 14.7004346847534\\
24.8095417022705 14.7004356384277\\
24.8836441040039 14.7004346847534\\
24.9651584625244 14.7004337310791\\
25.0548248291016 14.7004337310791\\
25.0960559844971 14.7004356384277\\
25.1414089202881 14.7004299163818\\
25.1613826751709 14.7004432678223\\
25.1780624389648 14.7004365921021\\
25.1964111328125 14.7004356384277\\
25.2165927886963 14.7004346847534\\
25.2387943267822 14.7004346847534\\
25.2632160186768 14.7004346847534\\
25.2900791168213 14.7004346847534\\
25.3196296691895 14.7004346847534\\
25.3521347045898 14.7004346847534\\
25.3878898620605 14.7004346847534\\
25.4272212982178 14.7004346847534\\
25.4704837799072 14.7004346847534\\
25.5180740356445 14.7004346847534\\
25.5704231262207 14.7004346847534\\
25.6280078887939 14.7004346847534\\
25.6913509368896 14.7004346847534\\
25.7610282897949 14.7004346847534\\
25.8376731872559 14.7004346847534\\
25.9219818115234 14.7004337310791\\
26.014720916748 14.7004346847534\\
26.1147212982178 14.7004337310791\\
26.2147216796875 14.7004470825195\\
26.3147220611572 14.7005739212036\\
26.3647212982178 14.7013168334961\\
26.4022121429443 14.7001361846924\\
26.4049663543701 14.6947832107544\\
26.407995223999 14.6978845596313\\
26.4113292694092 14.6995315551758\\
26.4149932861328 14.7012557983398\\
26.4190254211426 14.7013998031616\\
26.4234619140625 14.7014007568359\\
26.4283409118652 14.7013244628906\\
26.4337062835693 14.7011318206787\\
26.4396114349365 14.7008857727051\\
26.4461040496826 14.7006425857544\\
26.4532470703125 14.7004308700562\\
26.4611053466797 14.7002840042114\\
26.469747543335 14.7001647949219\\
26.4792556762695 14.6999044418335\\
26.4897136688232 14.6997222900391\\
26.5012187957764 14.6999931335449\\
26.5138721466064 14.7000913619995\\
26.5277919769287 14.7001256942749\\
26.5431041717529 14.700159072876\\
26.5599479675293 14.7001667022705\\
26.5784759521484 14.7001438140869\\
26.59885597229 14.7003698348999\\
26.6212730407715 14.700400352478\\
26.6459331512451 14.7004423141479\\
26.673059463501 14.7004289627075\\
26.7028980255127 14.7004270553589\\
26.7357215881348 14.7004299163818\\
26.7718257904053 14.7004365921021\\
26.8115406036377 14.7004384994507\\
26.8552284240723 14.7004384994507\\
26.903284072876 14.7004356384277\\
26.9561443328857 14.7004346847534\\
27.0142917633057 14.7004356384277\\
27.0605869293213 14.7004384994507\\
27.1115131378174 14.7004356384277\\
27.1556930541992 14.7004356384277\\
27.1975421905518 14.7004346847534\\
27.2435741424561 14.7004346847534\\
27.29421043396 14.7004346847534\\
27.3499088287354 14.7004346847534\\
27.4111785888672 14.7004346847534\\
27.4785747528076 14.7004346847534\\
27.5527095794678 14.7004346847534\\
27.6342601776123 14.7004346847534\\
27.7239646911621 14.7004337310791\\
27.8226375579834 14.7004346847534\\
27.9226379394531 14.7004337310791\\
28.0226383209229 14.7004346847534\\
28.1226387023926 14.7004337310791\\
28.1260833740234 14.7004280090332\\
28.1298713684082 14.7004461288452\\
28.1340389251709 14.7004289627075\\
28.1376724243164 14.7004299163818\\
28.1416683197021 14.7004432678223\\
28.1460647583008 14.7004299163818\\
28.1490650177002 14.7004346847534\\
28.1523666381836 14.7004375457764\\
28.1559963226318 14.7004356384277\\
28.1599903106689 14.7004346847534\\
28.1643829345703 14.7004356384277\\
28.1692161560059 14.7004346847534\\
28.1745319366455 14.7004346847534\\
28.1803779602051 14.7004346847534\\
28.1868114471436 14.7004346847534\\
28.1938858032227 14.7004346847534\\
28.2016677856445 14.7004346847534\\
28.2102298736572 14.7004346847534\\
28.2196464538574 14.7004346847534\\
28.2300052642822 14.7004346847534\\
28.2413997650146 14.7004346847534\\
28.2539348602295 14.7004346847534\\
28.2677211761475 14.7004346847534\\
28.2828884124756 14.7004346847534\\
28.2995700836182 14.7004346847534\\
28.3179225921631 14.7004346847534\\
28.3381080627441 14.7004346847534\\
28.3603134155273 14.7004346847534\\
28.3847389221191 14.7004346847534\\
28.4116058349609 14.7004346847534\\
28.4411602020264 14.7004346847534\\
28.4736709594727 14.7004346847534\\
28.5094318389893 14.7004346847534\\
28.5487689971924 14.7004346847534\\
28.592041015625 14.7004346847534\\
28.6396389007568 14.7004346847534\\
28.6919975280762 14.7004346847534\\
28.7495918273926 14.7004346847534\\
28.8129444122314 14.7004346847534\\
28.8826332092285 14.7004346847534\\
28.9592895507813 14.7004346847534\\
29.0436134338379 14.7004346847534\\
29.1363677978516 14.7004337310791\\
29.2363681793213 14.7004346847534\\
29.336368560791 14.7004337310791\\
29.4363689422607 14.7004671096802\\
29.4563694000244 14.7006034851074\\
29.4642658233643 14.7007513046265\\
29.471960067749 14.7008256912231\\
29.4804267883301 14.7008686065674\\
29.4885120391846 14.70090675354\\
29.497407913208 14.7008819580078\\
29.5071926116943 14.7007055282593\\
29.5179557800293 14.6986379623413\\
29.5297946929932 14.7005443572998\\
29.542818069458 14.7005176544189\\
29.5571441650391 14.7010450363159\\
29.5729026794434 14.7009868621826\\
29.5902366638184 14.7005062103271\\
29.6093044281006 14.7006874084473\\
29.630277633667 14.7003898620605\\
29.6533489227295 14.7003583908081\\
29.6787281036377 14.700234413147\\
29.6999015808105 14.7002286911011\\
29.7231941223145 14.7000913619995\\
29.7488136291504 14.6999702453613\\
29.7769947052002 14.7003145217896\\
29.8079967498779 14.7004585266113\\
29.8420963287354 14.700421333313\\
29.8796062469482 14.7004451751709\\
29.9208679199219 14.7004356384277\\
29.9662551879883 14.7004594802856\\
30.0161819458008 14.7003631591797\\
30.0711002349854 14.7005004882813\\
30.1315116882324 14.7004489898682\\
30.1804618835449 14.7004432678223\\
30.2286376953125 14.7004404067993\\
30.2746448516846 14.7004346847534\\
30.3156337738037 14.7004375457764\\
30.3508148193359 14.7004270553589\\
30.3895168304443 14.7004232406616\\
30.4320869445801 14.7004270553589\\
30.4789142608643 14.7004308700562\\
30.530424118042 14.7004327774048\\
30.587085723877 14.7004337310791\\
30.6491241455078 14.7004337310791\\
30.7173671722412 14.7004337310791\\
30.7924346923828 14.7004346847534\\
30.8750076293945 14.7004337310791\\
30.9658393859863 14.7004337310791\\
31.0657539367676 14.7004346847534\\
31.1657543182373 14.7004346847534\\
31.265754699707 14.7004356384277\\
31.3657531738281 14.7004356384277\\
31.4657535552979 14.7004356384277\\
31.5657539367676 14.7004346847534\\
31.6657543182373 14.7004346847534\\
31.765754699707 14.7004346847534\\
31.8657531738281 14.7004346847534\\
31.9657535552979 14.7004346847534\\
32.0657539367676 14.7004346847534\\
32.1657524108887 14.7004346847534\\
32.265754699707 14.7004346847534\\
32.3657531738281 14.7004346847534\\
32.4657554626465 14.7004346847534\\
32.5657539367676 14.7004346847534\\
32.6657524108887 14.7004346847534\\
32.765754699707 14.7004346847534\\
32.8657531738281 14.7004346847534\\
32.9657554626465 14.7004346847534\\
33.0657539367676 14.7004346847534\\
33.1657524108887 14.7004346847534\\
33.265754699707 14.7004346847534\\
33.3657531738281 14.7004346847534\\
33.4657554626465 14.7004346847534\\
33.5657539367676 14.7004346847534\\
33.6657524108887 14.7004346847534\\
33.765754699707 14.7004346847534\\
33.8657531738281 14.7004346847534\\
33.9657554626465 14.7004346847534\\
34.0657539367676 14.7004346847534\\
34.1657524108887 14.7004346847534\\
34.265754699707 14.7004346847534\\
34.3657531738281 14.7004346847534\\
34.4657554626465 14.7004346847534\\
34.5657539367676 14.7004346847534\\
34.6657524108887 14.7004346847534\\
34.765754699707 14.7004346847534\\
34.8657531738281 14.7004346847534\\
34.9657554626465 14.7004346847534\\
35.0657539367676 14.7004346847534\\
35.1657524108887 14.7004346847534\\
35.265754699707 14.7004346847534\\
35.3657531738281 14.7004346847534\\
35.4657554626465 14.7004346847534\\
35.5657539367676 14.7004346847534\\
35.6657524108887 14.7004346847534\\
35.765754699707 14.7004346847534\\
35.8657531738281 14.7004346847534\\
35.9657554626465 14.7004346847534\\
36.0657539367676 14.7004346847534\\
36.1657524108887 14.7004346847534\\
36.265754699707 14.7004346847534\\
36.3657531738281 14.7004346847534\\
36.4657554626465 14.7004346847534\\
36.5657539367676 14.7004346847534\\
36.6657524108887 14.7004346847534\\
36.765754699707 14.7004346847534\\
36.8657531738281 14.7004346847534\\
36.9657554626465 14.7004346847534\\
37.0657539367676 14.7004346847534\\
37.1539344787598 14.7004346847534\\
37.2244758605957 14.7004356384277\\
37.302074432373 14.7004356384277\\
37.3874282836914 14.7004346847534\\
37.4813194274902 14.7004346847534\\
37.5215835571289 14.7004356384277\\
37.5658721923828 14.7004308700562\\
37.585823059082 14.7004423141479\\
37.6024284362793 14.7004375457764\\
37.6206932067871 14.7004356384277\\
37.6407852172852 14.7004356384277\\
37.6628875732422 14.7004356384277\\
37.687198638916 14.7004346847534\\
37.713939666748 14.7004346847534\\
37.7433547973633 14.7004346847534\\
37.7757110595703 14.7004346847534\\
37.8113059997559 14.7004346847534\\
37.8504600524902 14.7004346847534\\
37.8935279846191 14.7004346847534\\
37.9409027099609 14.7004346847534\\
37.9930152893066 14.7004346847534\\
38.0503387451172 14.7004346847534\\
38.1133918762207 14.7004346847534\\
38.1827545166016 14.7004346847534\\
38.2590522766113 14.7004346847534\\
38.3429794311523 14.7004337310791\\
38.4352989196777 14.7004356384277\\
38.5352973937988 14.7004346847534\\
38.6352996826172 14.7004470825195\\
38.7352981567383 14.7005777359009\\
38.7853012084961 14.7013368606567\\
38.822696685791 14.7021312713623\\
38.8252944946289 14.6969156265259\\
38.8281555175781 14.6982412338257\\
38.8313026428223 14.6990566253662\\
38.8347625732422 14.7018222808838\\
38.838565826416 14.701379776001\\
38.8427543640137 14.7012977600098\\
38.8473587036133 14.7012424468994\\
38.852424621582 14.7010612487793\\
38.8579978942871 14.7008295059204\\
38.864128112793 14.7006340026855\\
38.8708724975586 14.7004661560059\\
38.878288269043 14.70032787323\\
38.8864479064941 14.7002439498901\\
38.8954238891602 14.6996164321899\\
38.9052963256836 14.6998357772827\\
38.9161567687988 14.698881149292\\
38.9281005859375 14.6999797821045\\
38.9412422180176 14.7003774642944\\
38.955696105957 14.7005062103271\\
38.9715957641602 14.7006330490112\\
38.989086151123 14.700252532959\\
39.0083236694336 14.6997528076172\\
39.029483795166 14.7005481719971\\
39.0527648925781 14.7004165649414\\
39.078369140625 14.7004442214966\\
39.1065368652344 14.7004241943359\\
39.137523651123 14.7004299163818\\
39.1716041564941 14.7004337310791\\
39.2090950012207 14.7004365921021\\
39.2477378845215 14.700439453125\\
39.2902450561523 14.7004384994507\\
39.3370056152344 14.7004375457764\\
39.3884391784668 14.7004346847534\\
39.4450187683105 14.7004346847534\\
39.4936180114746 14.7004384994507\\
39.5406875610352 14.7004375457764\\
39.5853233337402 14.7004375457764\\
39.6272773742676 14.7004365921021\\
39.6734275817871 14.7004365921021\\
39.7241897583008 14.7004365921021\\
39.780029296875 14.7004365921021\\
39.8414497375488 14.7004365921021\\
39.9090156555176 14.7004365921021\\
39.9833374023438 14.7004365921021\\
40.0650939941406 14.7004365921021\\
40.1550216674805 14.7004356384277\\
40.2539443969727 14.7004365921021\\
40.353946685791 14.7004356384277\\
40.4539451599121 14.7004365921021\\
40.5539436340332 14.7004356384277\\
40.5569686889648 14.700421333313\\
40.5602951049805 14.7004623413086\\
40.5639533996582 14.7003479003906\\
40.5665168762207 14.700511932373\\
40.5693359375 14.7004690170288\\
40.572437286377 14.700439453125\\
40.5758476257324 14.7004337310791\\
40.5796012878418 14.7004365921021\\
40.5837287902832 14.7004375457764\\
40.5882682800293 14.7004365921021\\
40.59326171875 14.7004365921021\\
40.5987548828125 14.7004365921021\\
40.6048011779785 14.7004365921021\\
40.6114463806152 14.7004365921021\\
40.6187591552734 14.7004365921021\\
40.6268005371094 14.7004365921021\\
40.6356506347656 14.7004365921021\\
40.6453819274902 14.7004365921021\\
40.6560897827148 14.7004365921021\\
40.6678657531738 14.7004365921021\\
40.6808204650879 14.7004365921021\\
40.695068359375 14.7004356384277\\
40.7107429504395 14.7004356384277\\
40.7279853820801 14.7004356384277\\
40.7469482421875 14.7004356384277\\
40.7678108215332 14.7004356384277\\
40.7907600402832 14.7004356384277\\
40.8160057067871 14.7004356384277\\
40.8437728881836 14.7004356384277\\
40.8743171691895 14.7004356384277\\
40.9079170227051 14.7004356384277\\
40.9448776245117 14.7004356384277\\
40.9855308532715 14.7004356384277\\
41.0302505493164 14.7004356384277\\
41.0794448852539 14.7004356384277\\
41.1335563659668 14.7004356384277\\
41.1930809020996 14.7004356384277\\
41.2585563659668 14.7004356384277\\
41.3305778503418 14.7004356384277\\
41.4098052978516 14.7004356384277\\
41.4969520568848 14.7004346847534\\
41.5928153991699 14.7004365921021\\
41.692813873291 14.7004356384277\\
41.7928161621094 14.7004699707031\\
41.812816619873 14.7006139755249\\
41.8206024169922 14.7007713317871\\
41.8281707763672 14.7008485794067\\
41.8364944458008 14.700891494751\\
41.8444328308105 14.7009286880493\\
41.8531684875488 14.7008972167969\\
41.8627777099609 14.7007284164429\\
41.8733444213867 14.6987838745117\\
41.8849716186523 14.700213432312\\
41.8977584838867 14.7005949020386\\
41.9118270874023 14.7010364532471\\
41.9272994995117 14.7010231018066\\
41.9443206787109 14.7005071640015\\
41.9630432128906 14.7007131576538\\
41.9836387634277 14.7004089355469\\
42.0062942504883 14.7004680633545\\
42.0312156677246 14.7001523971558\\
42.0477752685547 14.7001934051514\\
42.0659866333008 14.7000703811646\\
42.0860252380371 14.699782371521\\
42.1080627441406 14.700216293335\\
42.1323051452637 14.7003765106201\\
42.1589736938477 14.7004451751709\\
42.1883087158203 14.7004327774048\\
42.2205772399902 14.7004470825195\\
42.2560729980469 14.7004537582397\\
42.2951164245605 14.7004384994507\\
42.3380661010742 14.700439453125\\
42.3853073120117 14.7004413604736\\
42.4372749328613 14.7004442214966\\
42.494441986084 14.7004356384277\\
42.5416946411133 14.7004499435425\\
42.5936737060547 14.7004632949829\\
42.6379699707031 14.7004384994507\\
42.6777877807617 14.7004241943359\\
42.711311340332 14.7004222869873\\
42.7481842041016 14.7004251480103\\
42.788745880127 14.7004289627075\\
42.833366394043 14.7004318237305\\
42.8824462890625 14.7004337310791\\
42.9364318847656 14.7004337310791\\
42.9958190917969 14.7004346847534\\
43.0611457824707 14.7004346847534\\
43.1330032348633 14.7004346847534\\
43.2120475769043 14.7004337310791\\
43.2989959716797 14.7004356384277\\
43.3946380615234 14.7004337310791\\
43.4946365356445 14.7004346847534\\
43.5946388244629 14.7004356384277\\
43.694637298584 14.7004356384277\\
43.7946357727051 14.7004365921021\\
43.8946380615234 14.7004356384277\\
43.9946365356445 14.7004356384277\\
44.0946388244629 14.7004356384277\\
44.194637298584 14.7004356384277\\
44.2946357727051 14.7004356384277\\
44.3946380615234 14.7004356384277\\
44.4946365356445 14.7004356384277\\
44.5946388244629 14.7004356384277\\
44.694637298584 14.7004356384277\\
44.7946357727051 14.7004356384277\\
44.8946380615234 14.7004356384277\\
44.9946365356445 14.7004356384277\\
45.0946388244629 14.7004356384277\\
45.194637298584 14.7004356384277\\
45.2946357727051 14.7004356384277\\
45.3946380615234 14.7004356384277\\
45.4946365356445 14.7004356384277\\
45.5946388244629 14.7004356384277\\
45.694637298584 14.7004356384277\\
45.7946357727051 14.7004356384277\\
45.8946380615234 14.7004356384277\\
45.9946365356445 14.7004356384277\\
46.0946388244629 14.7004356384277\\
46.194637298584 14.7004356384277\\
46.2946357727051 14.7004356384277\\
46.3946380615234 14.7004356384277\\
46.4946365356445 14.7004356384277\\
46.5946388244629 14.7004356384277\\
46.694637298584 14.7004356384277\\
46.7946357727051 14.7004356384277\\
46.8946380615234 14.7004356384277\\
46.9946365356445 14.7004356384277\\
47.0946388244629 14.7004356384277\\
47.194637298584 14.7004356384277\\
47.2946357727051 14.7004356384277\\
47.3946380615234 14.7004356384277\\
47.4946365356445 14.7004356384277\\
47.5946388244629 14.7004356384277\\
47.694637298584 14.7004356384277\\
47.7946357727051 14.7004356384277\\
47.8946380615234 14.7004356384277\\
47.9946365356445 14.7004356384277\\
48.0946388244629 14.7004356384277\\
48.194637298584 14.7004356384277\\
48.2946357727051 14.7004356384277\\
48.3946380615234 14.7004356384277\\
48.4946365356445 14.7004356384277\\
48.5946388244629 14.7004356384277\\
48.694637298584 14.7004356384277\\
48.7946357727051 14.7004356384277\\
48.8946380615234 14.7004356384277\\
48.9946365356445 14.7004356384277\\
49.0946388244629 14.7004356384277\\
49.194637298584 14.7004356384277\\
49.2946357727051 14.7004356384277\\
49.3946380615234 14.7004346847534\\
49.4946365356445 14.7004346847534\\
49.5842514038086 14.7004346847534\\
49.6559371948242 14.7004346847534\\
49.7347946166992 14.7004346847534\\
49.8215370178223 14.7004346847534\\
49.9169502258301 14.7004346847534\\
49.9568328857422 14.7004346847534\\
50.0007057189941 14.7004318237305\\
50.0206336975098 14.7004442214966\\
50.0371971130371 14.7004384994507\\
50.0554161071777 14.7004375457764\\
50.0754585266113 14.7004365921021\\
50.0975074768066 14.7004365921021\\
50.1217575073242 14.7004365921021\\
50.1484336853027 14.7004365921021\\
50.1777763366699 14.7004365921021\\
50.2100563049316 14.7004365921021\\
50.2455635070801 14.7004365921021\\
50.2846183776855 14.7004365921021\\
50.327579498291 14.7004365921021\\
50.3748397827148 14.7004365921021\\
50.4268226623535 14.7004365921021\\
50.4840087890625 14.7004365921021\\
50.5469093322754 14.7004365921021\\
50.6161003112793 14.7004365921021\\
50.692211151123 14.7004365921021\\
50.7759323120117 14.7004356384277\\
50.8680267333984 14.7004365921021\\
50.9680252075195 14.7004356384277\\
51.0680236816406 14.7004489898682\\
51.168025970459 14.7005825042725\\
51.2180252075195 14.7014131546021\\
51.2553977966309 14.785304069519\\
51.2579727172852 14.8123378753662\\
51.2608032226563 14.8206787109375\\
51.263916015625 14.8214416503906\\
51.2673416137695 14.8174848556519\\
51.2711067199707 14.8104915618896\\
51.2752532958984 14.8014631271362\\
51.2798080444336 14.7902717590332\\
51.2848243713379 14.7771835327148\\
51.290340423584 14.7633266448975\\
51.2964057922363 14.7495574951172\\
51.3030815124512 14.7365083694458\\
51.3104209899902 14.7253751754761\\
51.3184967041016 14.7156581878662\\
51.3273811340332 14.7047271728516\\
51.3371505737305 14.7018232345581\\
51.347900390625 14.6987438201904\\
51.3597221374512 14.699951171875\\
51.372730255127 14.7003555297852\\
51.387035369873 14.700532913208\\
51.4027709960938 14.7006788253784\\
51.4200820922852 14.7003927230835\\
51.4391212463379 14.6997137069702\\
51.4600677490234 14.7005624771118\\
51.4831085205078 14.7004566192627\\
51.5084533691406 14.7004566192627\\
51.5363311767578 14.7004270553589\\
51.5669975280762 14.7004299163818\\
51.6007308959961 14.7004280090332\\
51.6378364562988 14.7004518508911\\
51.6764678955078 14.7004547119141\\
51.7189598083496 14.7004508972168\\
51.7657051086426 14.7004594802856\\
51.8171195983887 14.7004537582397\\
51.8736801147461 14.7004480361938\\
51.9213600158691 14.7004470825195\\
51.9687728881836 14.7004432678223\\
52.0141105651855 14.7004423141479\\
52.056568145752 14.700439453125\\
52.103271484375 14.7004404067993\\
52.1546440124512 14.7004413604736\\
52.2111549377441 14.7004404067993\\
52.2733154296875 14.7004413604736\\
52.3416938781738 14.7004404067993\\
52.4169082641602 14.7004404067993\\
52.4996490478516 14.7004404067993\\
52.5906600952148 14.700439453125\\
52.6906585693359 14.7004404067993\\
52.790657043457 14.700439453125\\
52.8906593322754 14.7004404067993\\
52.9906578063965 14.700439453125\\
52.9936103820801 14.70041847229\\
52.9968528747559 14.700478553772\\
53.0004234313965 14.7003440856934\\
53.0029792785645 14.7005090713501\\
53.0057907104492 14.700475692749\\
53.0088844299316 14.7004451751709\\
53.0122871398926 14.7004375457764\\
53.0160293579102 14.7004404067993\\
53.0201454162598 14.7004404067993\\
53.0246734619141 14.7004404067993\\
53.0296516418457 14.7004404067993\\
53.0351333618164 14.7004404067993\\
53.0411605834961 14.7004404067993\\
53.0477867126465 14.7004404067993\\
53.0550804138184 14.7004404067993\\
53.063102722168 14.7004404067993\\
53.0719261169434 14.7004404067993\\
53.0816307067871 14.7004404067993\\
53.0923080444336 14.7004404067993\\
53.1040496826172 14.700439453125\\
53.1169700622559 14.700439453125\\
53.1311798095703 14.700439453125\\
53.1468124389648 14.700439453125\\
53.164005279541 14.700439453125\\
53.1829223632813 14.700439453125\\
53.203727722168 14.700439453125\\
53.2266120910645 14.700439453125\\
53.2517852783203 14.700439453125\\
53.2794799804688 14.700439453125\\
53.3099403381348 14.700439453125\\
53.343448638916 14.700439453125\\
53.3803062438965 14.700439453125\\
53.4208488464355 14.700439453125\\
53.465446472168 14.700439453125\\
53.5145034790039 14.700439453125\\
53.5684700012207 14.700439453125\\
53.6278305053711 14.700439453125\\
53.6931266784668 14.700439453125\\
53.7649536132813 14.700439453125\\
53.8439598083496 14.700439453125\\
53.9308700561523 14.7004384994507\\
54.0264701843262 14.700439453125\\
54.1264686584473 14.7004384994507\\
54.2264709472656 14.700475692749\\
54.2464714050293 14.7006416320801\\
54.2542266845703 14.7008295059204\\
54.2617721557617 14.7009239196777\\
54.2700691223145 14.7009859085083\\
54.2779922485352 14.7010536193848\\
54.2867050170898 14.7011156082153\\
54.2962913513184 14.7058486938477\\
54.3068389892578 14.7788057327271\\
54.3184356689453 14.8718605041504\\
54.3311958312988 14.8496513366699\\
54.3452301025391 14.7873611450195\\
54.360668182373 14.7785787582397\\
54.3776512145996 14.7675104141235\\
54.3963317871094 14.7563676834106\\
54.4168815612793 14.7455711364746\\
54.4394836425781 14.7246408462524\\
54.4643478393555 14.7029705047607\\
54.4809799194336 14.701699256897\\
54.4992752075195 14.7168884277344\\
54.519401550293 14.7367563247681\\
54.5415382385254 14.7580966949463\\
54.5658912658691 14.7480096817017\\
54.5926780700684 14.7547235488892\\
54.622142791748 14.7595672607422\\
54.6545562744141 14.7569274902344\\
54.6902084350586 14.7467765808105\\
54.7294273376465 14.7456140518188\\
54.7725677490234 14.7405414581299\\
54.8200225830078 14.7364912033081\\
54.8722229003906 14.7288122177124\\
54.9296417236328 14.729772567749\\
54.9768371582031 14.7165079116821\\
55.028751373291 14.712272644043\\
55.0732498168945 14.7051200866699\\
55.112865447998 14.701922416687\\
55.1461601257324 14.7008924484253\\
55.1827850341797 14.7006616592407\\
55.223072052002 14.7006072998047\\
55.2673873901367 14.7006149291992\\
55.3161315917969 14.7006435394287\\
55.3697547912598 14.7007179260254\\
55.4287376403809 14.7008876800537\\
55.4936180114746 14.7012748718262\\
55.5649871826172 14.7022199630737\\
55.643497467041 14.7046184539795\\
55.7298545837402 14.7102718353271\\
55.8248481750488 14.719780921936\\
55.9248466491699 14.7283134460449\\
56.024845123291 14.7314033508301\\
56.1248474121094 14.7326459884644\\
56.2248458862305 14.7327632904053\\
56.3248481750488 14.7323789596558\\
56.4248466491699 14.73304271698\\
56.524845123291 14.7333612442017\\
56.6248474121094 14.7341041564941\\
56.7248458862305 14.7344703674316\\
56.8248481750488 14.7350854873657\\
56.9248466491699 14.7355365753174\\
57.024845123291 14.7360963821411\\
57.1248474121094 14.7365837097168\\
57.2248458862305 14.7371196746826\\
57.3248481750488 14.7376232147217\\
57.4248466491699 14.73814868927\\
57.524845123291 14.7386598587036\\
57.6248474121094 14.7391843795776\\
57.7248458862305 14.7397012710571\\
57.8248481750488 14.7402267456055\\
57.9248466491699 14.7407484054565\\
58.024845123291 14.7412719726563\\
58.1248474121094 14.7417945861816\\
58.2248458862305 14.7423181533813\\
58.3248481750488 14.7428407669067\\
58.4248466491699 14.7433643341064\\
58.524845123291 14.7441482543945\\
58.6248474121094 14.7447919845581\\
58.7248458862305 14.7454137802124\\
58.8248481750488 14.746039390564\\
58.9248466491699 14.7466621398926\\
59.024845123291 14.7472839355469\\
59.1248474121094 14.7479038238525\\
59.2248458862305 14.7485227584839\\
59.3248481750488 14.7491397857666\\
59.4248466491699 14.749755859375\\
59.524845123291 14.7503709793091\\
59.6248474121094 14.7509841918945\\
59.7248458862305 14.75159740448\\
59.8248481750488 14.7522087097168\\
59.9124221801758 14.7528276443481\\
60 14.7533445358276\\
60.0963325500488 14.7550392150879\\
60.1963348388672 14.7549152374268\\
60.2963333129883 14.7551441192627\\
60.3963356018066 14.7557716369629\\
60.4963340759277 14.7562818527222\\
60.5963325500488 14.7568340301514\\
60.6963348388672 14.7573680877686\\
60.7963333129883 14.7579069137573\\
60.8963356018066 14.7584409713745\\
60.9963340759277 14.7589750289917\\
61.0963325500488 14.7595205307007\\
61.1963348388672 14.760085105896\\
61.2963333129883 14.7607097625732\\
61.3963356018066 14.7612915039063\\
61.4963340759277 14.7618818283081\\
61.5963325500488 14.7625017166138\\
61.6963348388672 14.7631530761719\\
61.7963333129883 14.7638607025146\\
61.8963356018066 14.7666187286377\\
61.9940414428711 14.7725372314453\\
62.0722236633301 14.7772302627563\\
62.1347732543945 14.7678384780884\\
62.2035751342773 14.7853899002075\\
62.2478866577148 14.7423706054688\\
62.296630859375 14.7573757171631\\
62.3256530761719 14.7781038284302\\
62.3496742248535 14.9185819625854\\
62.376091003418 14.6455879211426\\
62.4051551818848 14.7663469314575\\
62.4213905334473 14.7731399536133\\
62.4340591430664 14.7736177444458\\
62.4479942321777 14.7702512741089\\
62.463321685791 14.7703065872192\\
62.4801826477051 14.770414352417\\
62.4987297058105 14.7705354690552\\
62.5191307067871 14.7706804275513\\
62.5415687561035 14.7708520889282\\
62.5662574768066 14.7710523605347\\
62.5934104919434 14.7712850570679\\
62.6232795715332 14.7715530395508\\
62.656135559082 14.771861076355\\
62.6922760009766 14.7722101211548\\
62.7320327758789 14.7726030349731\\
62.775764465332 14.7730436325073\\
62.8238677978516 14.7735319137573\\
62.8767852783203 14.7740726470947\\
62.9349899291992 14.7746677398682\\
62.9990196228027 14.7753229141235\\
63.069450378418 14.7760419845581\\
63.1469230651855 14.7768335342407\\
63.2321434020996 14.7871074676514\\
63.3258857727051 14.7697143554688\\
63.4258842468262 14.7797985076904\\
63.5258865356445 14.7849702835083\\
63.6258850097656 14.7854433059692\\
63.6758842468262 14.836443901062\\
63.7138481140137 14.8006286621094\\
63.7273941040039 14.7794408798218\\
63.7422943115234 14.7986993789673\\
63.7586822509766 14.8014860153198\\
63.7767105102539 14.7805910110474\\
63.7965431213379 14.7852239608765\\
63.8183555603027 14.8000230789185\\
63.8423538208008 14.801703453064\\
63.8687477111816 14.7968225479126\\
63.8977813720703 14.7944383621216\\
63.9297218322754 14.7894487380981\\
63.9648513793945 14.7829494476318\\
64.0015869140625 14.7817621231079\\
64.022705078125 14.7849435806274\\
64.0459365844727 14.7998704910278\\
64.0714874267578 14.8038244247437\\
64.0996017456055 14.8095560073853\\
64.1305236816406 14.8164434432983\\
64.1645355224609 14.8141679763794\\
64.2019424438477 14.8086996078491\\
64.2431030273438 14.8064050674438\\
64.2883682250977 14.8048639297485\\
64.3381729125977 14.8033571243286\\
64.3929443359375 14.8020734786987\\
64.4477233886719 14.8017148971558\\
64.4993057250977 14.8009595870972\\
64.5481796264648 14.8002328872681\\
64.5940780639648 14.7997608184814\\
64.6370315551758 14.8011608123779\\
64.6842803955078 14.801215171814\\
64.7362518310547 14.799391746521\\
64.7934188842773 14.7971467971802\\
64.8563079833984 14.7972230911255\\
64.9254837036133 14.7975673675537\\
65.0015716552734 14.7979307174683\\
65.0852737426758 14.7985363006592\\
65.1773452758789 14.8012857437134\\
65.27734375 14.7980928421021\\
65.3773422241211 14.8033933639526\\
65.4773483276367 14.7998895645142\\
65.5773468017578 14.8021068572998\\
65.5794677734375 14.9397068023682\\
65.5817947387695 14.7578678131104\\
65.5843658447266 14.7868309020996\\
65.5871810913086 14.8740196228027\\
65.5902862548828 14.7531795501709\\
65.5937042236328 14.7803544998169\\
65.5974578857422 14.8144588470459\\
65.6015930175781 14.8135328292847\\
65.6061325073242 14.8049659729004\\
65.6111373901367 14.8046598434448\\
65.6166305541992 14.8047647476196\\
65.6226806640625 14.8047742843628\\
65.6293334960938 14.8047733306885\\
65.6366577148438 14.804780960083\\
65.6447067260742 14.8047962188721\\
65.653564453125 14.80482006073\\
65.6633071899414 14.8048553466797\\
65.6740188598633 14.8049020767212\\
65.6858062744141 14.8049631118774\\
65.6987762451172 14.8050403594971\\
65.7130355834961 14.8051347732544\\
65.7287292480469 14.8052492141724\\
65.7459869384766 14.8053846359253\\
65.7649688720703 14.8055438995361\\
65.7858505249023 14.8057279586792\\
65.8088226318359 14.8059387207031\\
65.8340911865234 14.8061780929565\\
65.8618850708008 14.8064479827881\\
65.892463684082 14.8067502975464\\
65.9260940551758 14.8070878982544\\
65.9630889892578 14.8074617385864\\
66.0037841796875 14.8078756332397\\
66.0485458374023 14.808331489563\\
66.097785949707 14.8088350296021\\
66.1519470214844 14.8093891143799\\
66.2115325927734 14.8099985122681\\
66.2770690917969 14.8106689453125\\
66.3491592407227 14.8114070892334\\
66.428466796875 14.8122186660767\\
66.5156936645508 14.8149814605713\\
66.6116485595703 14.8121070861816\\
66.7116470336914 14.8152685165405\\
66.811653137207 14.8246393203735\\
66.8316497802734 14.8383340835571\\
66.8391723632813 14.8705806732178\\
66.8464965820313 14.8753366470337\\
66.8545455932617 14.8727407455444\\
66.8622512817383 14.8705987930298\\
66.870735168457 14.8875360488892\\
66.8800659179688 14.9326839447021\\
66.8903274536133 14.9507627487183\\
66.9016189575195 14.9377384185791\\
66.9140319824219 14.9151763916016\\
66.9276962280273 14.8963766098022\\
66.9427185058594 14.8754615783691\\
66.9592437744141 14.8529281616211\\
66.977424621582 14.8292407989502\\
66.997428894043 14.7924518585205\\
67.0194244384766 14.768310546875\\
67.0415725708008 14.7532615661621\\
67.0659408569336 14.7476253509521\\
67.0927429199219 14.748740196228\\
67.1222229003906 14.7785997390747\\
67.154655456543 14.8226356506348\\
67.1903305053711 14.8924341201782\\
67.1995849609375 14.8809413909912\\
67.2097702026367 14.8653688430786\\
67.2209701538086 14.8561859130859\\
67.233283996582 14.8519468307495\\
67.2468414306641 14.8496732711792\\
67.2617416381836 14.8481760025024\\
67.2781448364258 14.8471488952637\\
67.2961807250977 14.8468217849731\\
67.3160247802734 14.8464612960815\\
67.3378448486328 14.8407402038574\\
67.3618545532227 14.8428916931152\\
67.3882598876953 14.8396091461182\\
67.4173126220703 14.8401174545288\\
67.4492645263672 14.838418006897\\
67.4844131469727 14.8350276947021\\
67.523078918457 14.8322305679321\\
67.5656127929688 14.8315010070801\\
67.6123962402344 14.8293342590332\\
67.6579818725586 14.8262252807617\\
67.6993789672852 14.8238000869751\\
67.7355346679688 14.8248710632324\\
67.7753067016602 14.8273591995239\\
67.8190612792969 14.8291044235229\\
67.8671875 14.829493522644\\
67.9201278686523 14.8313245773315\\
67.9783630371094 14.831335067749\\
68.0421829223633 14.833592414856\\
68.1070251464844 14.8362102508545\\
68.178352355957 14.8395175933838\\
68.2568130493164 14.8426094055176\\
68.3431243896484 14.8401212692261\\
68.4380569458008 14.8394899368286\\
68.5380630493164 14.8389663696289\\
68.6380615234375 14.8385028839111\\
68.7380599975586 14.8382053375244\\
68.8380584716797 14.8392477035522\\
68.9380569458008 14.8401908874512\\
69.0380630493164 14.8415546417236\\
69.1380615234375 14.8424921035767\\
69.2380599975586 14.843713760376\\
69.3380584716797 14.8447504043579\\
69.4380569458008 14.8459119796753\\
69.5380630493164 14.8469915390015\\
69.6380615234375 14.8481254577637\\
69.7380599975586 14.8492240905762\\
69.8380584716797 14.8503465652466\\
69.9380569458008 14.8514699935913\\
70.0380630493164 14.852575302124\\
70.1380615234375 14.8536939620972\\
70.2380599975586 14.8548154830933\\
70.3380584716797 14.8559350967407\\
70.4380569458008 14.8570566177368\\
70.5380630493164 14.8581771850586\\
70.6380615234375 14.8592977523804\\
70.7380599975586 14.8604183197021\\
70.8380584716797 14.8615398406982\\
70.9380569458008 14.8628101348877\\
71.0380630493164 14.8640956878662\\
71.1380615234375 14.8651733398438\\
71.2380599975586 14.866307258606\\
71.3380584716797 14.8674249649048\\
71.4380569458008 14.8685474395752\\
71.5380630493164 14.869668006897\\
71.6380615234375 14.8707895278931\\
71.7380599975586 14.8719110488892\\
71.8380584716797 14.8730325698853\\
71.9380569458008 14.874153137207\\
72.0380630493164 14.8752746582031\\
72.1380615234375 14.8763961791992\\
72.2380599975586 14.8775177001953\\
72.3380584716797 14.8786392211914\\
72.4380569458008 14.8797597885132\\
72.5380630493164 14.8820171356201\\
72.6380615234375 14.8821134567261\\
72.7380599975586 14.8828620910645\\
72.8380584716797 14.8840703964233\\
72.9380569458008 14.8851528167725\\
73.0380630493164 14.8862819671631\\
73.1380615234375 14.887393951416\\
73.2380599975586 14.8885116577148\\
73.3380584716797 14.889627456665\\
73.4380569458008 14.8907432556152\\
73.5380630493164 14.8918733596802\\
73.6380615234375 14.8930206298828\\
73.7380599975586 14.8942213058472\\
73.8380584716797 14.8953590393066\\
73.9380569458008 14.8965015411377\\
74.0380630493164 14.897668838501\\
74.1380615234375 14.8988561630249\\
74.2380599975586 14.9000835418701\\
74.3380584716797 14.9031352996826\\
74.4363632202148 14.909234046936\\
74.5150299072266 14.9133462905884\\
74.5779571533203 14.9044456481934\\
74.6471862792969 14.9048004150391\\
74.7233276367188 14.9053325653076\\
74.8070907592773 14.9059629440308\\
74.8992309570313 14.9225997924805\\
74.945930480957 14.8768405914307\\
74.9729614257813 14.9006414413452\\
74.9864807128906 14.9165000915527\\
75 14.9173889160156\\
};
\addlegendentry{18 "};

\addplot [
color=green,
dash pattern=on 1pt off 3pt on 3pt off 3pt
]
table[row sep=crcr]{
0 14.7003479003906\\
9.99999971718069e-10 14.7003479003906\\
2.49999998480632e-09 14.7003479003906\\
4.74999994892755e-09 14.7003479003906\\
8.1250002281763e-09 14.7003479003906\\
1.31874999809156e-08 14.7003479003906\\
2.0781250498203e-08 14.7003479003906\\
3.21718758300449e-08 14.7003479003906\\
4.92578138278077e-08 14.7003479003906\\
7.48867208244519e-08 14.7003479003906\\
1.13330081319418e-07 14.7003479003906\\
1.70995122061868e-07 14.7003479003906\\
2.57492672517401e-07 14.7003479003906\\
3.8723902662241e-07 14.7003479003906\\
5.81858500936505e-07 14.7003479003906\\
8.73787769251066e-07 14.7003479003906\\
1.31168167172291e-06 14.7003479003906\\
1.9685226106958e-06 14.7003479003906\\
2.95378367809462e-06 14.7003479003906\\
4.43167573394021e-06 14.7003479003906\\
6.64851359033491e-06 14.7003479003906\\
9.97376992017962e-06 14.7003479003906\\
1.49616553244414e-05 14.7003479003906\\
2.24434825213393e-05 14.7003479003906\\
3.36662233166862e-05 14.7003479003906\\
5.05003372381907e-05 14.7003479003906\\
7.57515081204474e-05 14.7003479003906\\
0.000113628258986864 14.7003479003906\\
0.000170443381648511 14.7003479003906\\
0.000255666091106832 14.7003488540649\\
0.000383500126190484 14.7003498077393\\
0.000575251178815961 14.7003507614136\\
0.000862877757754177 14.7003536224365\\
0.00129431765526533 14.7003536224365\\
0.00194147753063589 14.7003507614136\\
0.00291221728548408 14.7003469467163\\
0.00436832662671804 14.7003450393677\\
0.00597004732117057 14.700345993042\\
0.00773194013163447 14.7003479003906\\
0.009670021943748 14.7003479003906\\
0.0118019115179777 14.7003479003906\\
0.0141469910740852 14.7003479003906\\
0.0167265776544809 14.7003479003906\\
0.0195641238242388 14.7003479003906\\
0.0226854234933853 14.7003479003906\\
0.0261188540607691 14.7003479003906\\
0.0298956278711557 14.7003479003906\\
0.0340500771999359 14.7003479003906\\
0.0386199727654457 14.7003479003906\\
0.0436468608677387 14.7003479003906\\
0.0491764321923256 14.7003479003906\\
0.0552589632570744 14.7003479003906\\
0.0619497485458851 14.7003479003906\\
0.0693096145987511 14.7003479003906\\
0.0774054601788521 14.7003479003906\\
0.0863108932971954 14.7003469467163\\
0.0961068719625473 14.7003479003906\\
0.106882445514202 14.7003479003906\\
0.118735581636429 14.7003479003906\\
0.13177402317524 14.7003479003906\\
0.146116316318512 14.7003479003906\\
0.161892831325531 14.7003479003906\\
0.179247006773949 14.7003479003906\\
0.198336601257324 14.7003479003906\\
0.21933513879776 14.7003479003906\\
0.242433547973633 14.7003479003906\\
0.267841786146164 14.7003479003906\\
0.29579085111618 14.7003479003906\\
0.32653483748436 14.7003479003906\\
0.360353201627731 14.7003479003906\\
0.397553414106369 14.7003479003906\\
0.438473641872406 14.7003479003906\\
0.483485877513886 14.7003479003906\\
0.532999336719513 14.7003479003906\\
0.587464153766632 14.7003479003906\\
0.647375464439392 14.7003479003906\\
0.71327793598175 14.7003479003906\\
0.7857705950737 14.7003479003906\\
0.865512549877167 14.7003479003906\\
0.953228712081909 14.7003479003906\\
1.04971647262573 14.7003479003906\\
1.14971649646759 14.7003479003906\\
1.24971640110016 14.7003479003906\\
1.34971642494202 14.7003479003906\\
1.44971644878387 14.7003479003906\\
1.54971647262573 14.7003479003906\\
1.64971649646759 14.7003479003906\\
1.74971640110016 14.7003479003906\\
1.84971642494202 14.7003479003906\\
1.94971644878387 14.7003479003906\\
2.04971647262573 14.7003479003906\\
2.1497163772583 14.7003479003906\\
2.24971652030945 14.7003479003906\\
2.34971642494202 14.7003479003906\\
2.44971656799316 14.7003479003906\\
2.54971647262573 14.7003479003906\\
2.6497163772583 14.7003479003906\\
2.74971652030945 14.7003479003906\\
2.84971642494202 14.7003479003906\\
2.94971656799316 14.7003479003906\\
3.04971647262573 14.7003479003906\\
3.1497163772583 14.7003479003906\\
3.24971652030945 14.7003479003906\\
3.34971642494202 14.7003479003906\\
3.44971656799316 14.7003479003906\\
3.54971647262573 14.7003479003906\\
3.6497163772583 14.7003479003906\\
3.74971652030945 14.7003479003906\\
3.84971642494202 14.7003479003906\\
3.94971656799316 14.7003479003906\\
4.04971647262573 14.7003479003906\\
4.1497163772583 14.7003479003906\\
4.24971628189087 14.7003479003906\\
4.3497166633606 14.7003479003906\\
4.44971656799316 14.7003479003906\\
4.54971647262573 14.7003479003906\\
4.6497163772583 14.7003479003906\\
4.74971628189087 14.7003479003906\\
4.8497166633606 14.7003479003906\\
4.94971656799316 14.7003479003906\\
5.04971647262573 14.7003479003906\\
5.1497163772583 14.7003479003906\\
5.24971628189087 14.7003479003906\\
5.3497166633606 14.7003479003906\\
5.44971656799316 14.7003479003906\\
5.54971647262573 14.7003479003906\\
5.6497163772583 14.7003479003906\\
5.74971628189087 14.7003479003906\\
5.8497166633606 14.7003479003906\\
5.94971656799316 14.7003479003906\\
6.04971647262573 14.7003479003906\\
6.1497163772583 14.7003469467163\\
6.24971628189087 14.700345993042\\
6.3497166633606 14.700345993042\\
6.44971656799316 14.7003469467163\\
6.54971647262573 14.7003479003906\\
6.6497163772583 14.7003488540649\\
6.74971628189087 14.7003488540649\\
6.8497166633606 14.7003479003906\\
6.94971656799316 14.7003479003906\\
7.04971647262573 14.7003479003906\\
7.1497163772583 14.7003479003906\\
7.24971628189087 14.7003479003906\\
7.3497166633606 14.7003479003906\\
7.44971656799316 14.7003479003906\\
7.54971647262573 14.7003479003906\\
7.6497163772583 14.7003479003906\\
7.74971628189087 14.7003479003906\\
7.8497166633606 14.7003479003906\\
7.94971656799316 14.7003479003906\\
8.04971599578857 14.7003479003906\\
8.1497163772583 14.7003479003906\\
8.24971675872803 14.7003479003906\\
8.34971618652344 14.7003479003906\\
8.44971656799316 14.7003479003906\\
8.54971599578857 14.7003479003906\\
8.6497163772583 14.7003479003906\\
8.74971675872803 14.7003479003906\\
8.84971618652344 14.7003479003906\\
8.94971656799316 14.7003479003906\\
9.04971599578857 14.7003479003906\\
9.1497163772583 14.7003479003906\\
9.24971675872803 14.7003479003906\\
9.34971618652344 14.7003479003906\\
9.44971656799316 14.7003479003906\\
9.54971599578857 14.7003479003906\\
9.6497163772583 14.7003479003906\\
9.74971675872803 14.7003479003906\\
9.84971618652344 14.7003479003906\\
9.94971656799316 14.7003479003906\\
10.0497159957886 14.7003479003906\\
10.1497163772583 14.7003479003906\\
10.249716758728 14.7003479003906\\
10.3497161865234 14.7003479003906\\
10.4497165679932 14.7003479003906\\
10.5497159957886 14.7003479003906\\
10.6497163772583 14.7003479003906\\
10.749716758728 14.7003479003906\\
10.8497161865234 14.7003479003906\\
10.9497165679932 14.7003479003906\\
11.0497159957886 14.7003479003906\\
11.1497163772583 14.7003479003906\\
11.249716758728 14.7003479003906\\
11.3497161865234 14.7003479003906\\
11.4497165679932 14.7003479003906\\
11.5497159957886 14.7003479003906\\
11.6497163772583 14.7003479003906\\
11.749716758728 14.7003479003906\\
11.8497161865234 14.7003479003906\\
11.9497165679932 14.7003479003906\\
12.0497159957886 14.7003479003906\\
12.1497163772583 14.7003479003906\\
12.249716758728 14.7003479003906\\
12.347146987915 14.7003479003906\\
12.4250946044922 14.7003479003906\\
12.4874591827393 14.7003488540649\\
12.5560598373413 14.7003479003906\\
12.6315202713013 14.7003469467163\\
12.714527130127 14.7003479003906\\
12.8058338165283 14.7003479003906\\
12.8497476577759 14.7003488540649\\
12.8980522155762 14.7003469467163\\
12.9143037796021 14.7003479003906\\
12.9277105331421 14.7003488540649\\
12.9424571990967 14.7003479003906\\
12.9586782455444 14.7003479003906\\
12.9765224456787 14.7003479003906\\
12.9961500167847 14.7003479003906\\
13.0177412033081 14.7003479003906\\
13.0414915084839 14.7003479003906\\
13.0676164627075 14.7003479003906\\
13.0963535308838 14.7003479003906\\
13.1279649734497 14.7003479003906\\
13.1627368927002 14.7003479003906\\
13.2009868621826 14.7003479003906\\
13.2430610656738 14.7003479003906\\
13.289342880249 14.7003479003906\\
13.3402528762817 14.7003479003906\\
13.3962545394897 14.7003479003906\\
13.4578561782837 14.7003479003906\\
13.5256175994873 14.7003479003906\\
13.600154876709 14.7003479003906\\
13.6821460723877 14.7003469467163\\
13.7723369598389 14.7003479003906\\
13.871545791626 14.7003469467163\\
13.9715461730957 14.700364112854\\
14.0715456008911 14.7004337310791\\
14.1215467453003 14.7007741928101\\
14.1629066467285 14.70068359375\\
14.2008075714111 14.6996774673462\\
14.202446937561 14.69544506073\\
14.2042503356934 14.695897102356\\
14.2062339782715 14.6990795135498\\
14.2084159851074 14.7008476257324\\
14.2108154296875 14.7010135650635\\
14.2134561538696 14.7012376785278\\
14.2163600921631 14.7012929916382\\
14.219554901123 14.7008047103882\\
14.223069190979 14.7009725570679\\
14.2269344329834 14.7008953094482\\
14.2311868667603 14.7006998062134\\
14.2358636856079 14.7005310058594\\
14.2410087585449 14.7004308700562\\
14.2466678619385 14.7003755569458\\
14.252893447876 14.7003316879272\\
14.2597417831421 14.7002878189087\\
14.2672748565674 14.7002468109131\\
14.2755603790283 14.700213432312\\
14.2846755981445 14.7001895904541\\
14.2947015762329 14.700174331665\\
14.3057298660278 14.7001686096191\\
14.3178615570068 14.700174331665\\
14.3312063217163 14.7002849578857\\
14.3458862304688 14.7003355026245\\
14.3620328903198 14.7003087997437\\
14.3797950744629 14.7002668380737\\
14.3993330001831 14.7002449035645\\
14.4208250045776 14.7002468109131\\
14.444465637207 14.7002573013306\\
14.4704713821411 14.7003936767578\\
14.4990768432617 14.7003755569458\\
14.5269069671631 14.7003221511841\\
14.5575199127197 14.7003374099731\\
14.591194152832 14.7003440856934\\
14.6282358169556 14.7003479003906\\
14.668981552124 14.7003498077393\\
14.7138023376465 14.7003517150879\\
14.7631044387817 14.7003479003906\\
14.8173370361328 14.7003479003906\\
14.8769931793213 14.7003479003906\\
14.9244384765625 14.7003498077393\\
14.971435546875 14.7003488540649\\
15.0163955688477 14.7003479003906\\
15.0584926605225 14.7003479003906\\
15.09974193573 14.7003479003906\\
15.1451168060303 14.7003479003906\\
15.1950283050537 14.7003479003906\\
15.2499313354492 14.7003479003906\\
15.31032371521 14.7003479003906\\
15.3767566680908 14.7003479003906\\
15.4498319625854 14.7003479003906\\
15.5302152633667 14.7003479003906\\
15.6186370849609 14.7003469467163\\
15.7159004211426 14.7003479003906\\
15.8159008026123 14.7003469467163\\
15.9159002304077 14.7003479003906\\
16.0159015655518 14.7003469467163\\
16.0210952758789 14.700345993042\\
16.0268077850342 14.7003507614136\\
16.0330924987793 14.700345993042\\
16.0382099151611 14.7003479003906\\
16.0438404083252 14.7003488540649\\
16.0500316619873 14.7003488540649\\
16.0568428039551 14.7003479003906\\
16.0643348693848 14.7003479003906\\
16.0725765228271 14.7003479003906\\
16.0816440582275 14.7003479003906\\
16.0916156768799 14.7003479003906\\
16.1025848388672 14.7003479003906\\
16.114652633667 14.7003479003906\\
16.1279258728027 14.7003479003906\\
16.1425266265869 14.7003479003906\\
16.1585865020752 14.7003479003906\\
16.1762542724609 14.7003479003906\\
16.1956882476807 14.7003479003906\\
16.2170658111572 14.7003479003906\\
16.2405796051025 14.7003479003906\\
16.2664451599121 14.7003479003906\\
16.2948989868164 14.7003479003906\\
16.3261966705322 14.7003479003906\\
16.3606243133545 14.7003479003906\\
16.398494720459 14.7003479003906\\
16.4401531219482 14.7003479003906\\
16.4859771728516 14.7003479003906\\
16.5363826751709 14.7003479003906\\
16.5918292999268 14.7003479003906\\
16.6528205871582 14.7003479003906\\
16.7199096679688 14.7003479003906\\
16.7937088012695 14.7003479003906\\
16.8748874664307 14.7003479003906\\
16.9641857147217 14.7003469467163\\
17.0624122619629 14.7003479003906\\
17.1624126434326 14.7003469467163\\
17.2624130249023 14.7003698348999\\
17.282413482666 14.7004413604736\\
17.2906188964844 14.7004957199097\\
17.2996425628662 14.7005167007446\\
17.308614730835 14.7005271911621\\
17.3184852600098 14.7005271911621\\
17.3293399810791 14.7005052566528\\
17.3412818908691 14.7004718780518\\
17.3544178009033 14.7000761032104\\
17.3688678741455 14.6996097564697\\
17.3847618103027 14.7006368637085\\
17.4022445678711 14.700400352478\\
17.4214763641357 14.700572013855\\
17.4426326751709 14.7005281448364\\
17.4659042358398 14.7004833221436\\
17.4915008544922 14.7003993988037\\
17.5196590423584 14.7002954483032\\
17.5506324768066 14.7002534866333\\
17.5847034454346 14.7002334594727\\
17.6221809387207 14.7002439498901\\
17.6634063720703 14.7002687454224\\
17.7087554931641 14.7002983093262\\
17.758638381958 14.7003421783447\\
17.8135089874268 14.700345993042\\
17.8738670349121 14.7003526687622\\
17.9402618408203 14.7003488540649\\
18.0132961273193 14.7003488540649\\
18.0585689544678 14.7003545761108\\
18.1083717346191 14.7003536224365\\
18.1509475708008 14.7003545761108\\
18.1891002655029 14.7003488540649\\
18.2310676574707 14.7003402709961\\
18.2772312164307 14.7003393173218\\
18.328010559082 14.7003421783447\\
18.3838691711426 14.7003507614136\\
18.4453125 14.7003469467163\\
18.5129013061523 14.7003469467163\\
18.5872478485107 14.7003479003906\\
18.6690311431885 14.7003479003906\\
18.7589912414551 14.7003479003906\\
18.857946395874 14.7003479003906\\
18.9579467773438 14.7003479003906\\
19.0579471588135 14.7003479003906\\
19.1579456329346 14.7003479003906\\
19.2579460144043 14.7003479003906\\
19.357946395874 14.7003479003906\\
19.4579467773438 14.7003479003906\\
19.5579471588135 14.7003479003906\\
19.6579456329346 14.7003479003906\\
19.7579460144043 14.7003479003906\\
19.857946395874 14.7003479003906\\
19.9579467773438 14.7003479003906\\
20.0579471588135 14.7003479003906\\
20.1579456329346 14.7003479003906\\
20.2579460144043 14.7003479003906\\
20.357946395874 14.7003479003906\\
20.4579467773438 14.7003479003906\\
20.5579471588135 14.7003479003906\\
20.6579456329346 14.7003479003906\\
20.7579460144043 14.7003479003906\\
20.857946395874 14.7003479003906\\
20.9579467773438 14.7003479003906\\
21.0579471588135 14.7003479003906\\
21.1579456329346 14.7003479003906\\
21.2579460144043 14.7003479003906\\
21.357946395874 14.7003479003906\\
21.4579467773438 14.7003479003906\\
21.5579471588135 14.7003479003906\\
21.6579456329346 14.7003479003906\\
21.7579460144043 14.7003479003906\\
21.857946395874 14.7003479003906\\
21.9579467773438 14.7003479003906\\
22.0579471588135 14.7003479003906\\
22.1579456329346 14.7003479003906\\
22.2579460144043 14.7003479003906\\
22.357946395874 14.7003479003906\\
22.4579467773438 14.7003479003906\\
22.5579471588135 14.7003479003906\\
22.6579456329346 14.7003479003906\\
22.7579460144043 14.7003479003906\\
22.857946395874 14.7003479003906\\
22.9579467773438 14.7003479003906\\
23.0579471588135 14.7003479003906\\
23.1579456329346 14.7003479003906\\
23.2579460144043 14.7003479003906\\
23.357946395874 14.7003479003906\\
23.4579467773438 14.7003479003906\\
23.5579471588135 14.7003479003906\\
23.6579456329346 14.7003479003906\\
23.7579460144043 14.7003479003906\\
23.857946395874 14.7003479003906\\
23.9579467773438 14.7003479003906\\
24.0579471588135 14.7003469467163\\
24.1579456329346 14.7003479003906\\
24.2579460144043 14.7003479003906\\
24.357946395874 14.7003479003906\\
24.4579467773438 14.7003479003906\\
24.5579471588135 14.7003479003906\\
24.6579456329346 14.7003479003906\\
24.7421741485596 14.7003479003906\\
24.8095417022705 14.7003479003906\\
24.8836441040039 14.7003479003906\\
24.9651584625244 14.7003469467163\\
25.0548248291016 14.7003469467163\\
25.0960559844971 14.7003479003906\\
25.1414089202881 14.7003440856934\\
25.1613826751709 14.7003545761108\\
25.1780624389648 14.7003498077393\\
25.1964111328125 14.7003479003906\\
25.2165927886963 14.7003479003906\\
25.2387943267822 14.7003479003906\\
25.2632160186768 14.7003479003906\\
25.2900791168213 14.7003479003906\\
25.3196296691895 14.7003479003906\\
25.3521347045898 14.7003479003906\\
25.3878898620605 14.7003479003906\\
25.4272212982178 14.7003479003906\\
25.4704837799072 14.7003479003906\\
25.5180740356445 14.7003479003906\\
25.5704231262207 14.7003479003906\\
25.6280078887939 14.7003479003906\\
25.6913509368896 14.7003479003906\\
25.7610282897949 14.7003479003906\\
25.8376731872559 14.7003479003906\\
25.9219818115234 14.7003469467163\\
26.014720916748 14.7003479003906\\
26.1147212982178 14.7003469467163\\
26.2147216796875 14.7003574371338\\
26.3147220611572 14.7004594802856\\
26.3647212982178 14.7010536193848\\
26.4022121429443 14.7001094818115\\
26.4049663543701 14.6958637237549\\
26.407995223999 14.6982536315918\\
26.4113292694092 14.699631690979\\
26.4149932861328 14.7010059356689\\
26.4190254211426 14.7011260986328\\
26.4234619140625 14.7011213302612\\
26.4283409118652 14.701060295105\\
26.4337062835693 14.70090675354\\
26.4396114349365 14.7007112503052\\
26.4461040496826 14.7005176544189\\
26.4532470703125 14.7003507614136\\
26.4611053466797 14.700234413147\\
26.469747543335 14.7001419067383\\
26.4792556762695 14.6999340057373\\
26.4897136688232 14.6997871398926\\
26.5012187957764 14.7000007629395\\
26.5138721466064 14.7000780105591\\
26.5277919769287 14.7001037597656\\
26.5431041717529 14.7001285552979\\
26.5599479675293 14.7001342773438\\
26.5784759521484 14.7001152038574\\
26.59885597229 14.7002954483032\\
26.6212730407715 14.7003192901611\\
26.6459331512451 14.7003526687622\\
26.673059463501 14.7003421783447\\
26.7028980255127 14.7003412246704\\
26.7357215881348 14.7003440856934\\
26.7718257904053 14.7003488540649\\
26.8115406036377 14.7003507614136\\
26.8552284240723 14.7003507614136\\
26.903284072876 14.7003488540649\\
26.9561443328857 14.7003479003906\\
27.0142917633057 14.7003479003906\\
27.0605869293213 14.7003507614136\\
27.1115131378174 14.7003488540649\\
27.1556930541992 14.7003479003906\\
27.1975421905518 14.7003479003906\\
27.2435741424561 14.7003479003906\\
27.29421043396 14.7003479003906\\
27.3499088287354 14.7003479003906\\
27.4111785888672 14.7003479003906\\
27.4785747528076 14.7003479003906\\
27.5527095794678 14.7003479003906\\
27.6342601776123 14.7003479003906\\
27.7239646911621 14.7003469467163\\
27.8226375579834 14.7003479003906\\
27.9226379394531 14.7003469467163\\
28.0226383209229 14.7003479003906\\
28.1226387023926 14.7003469467163\\
28.1260833740234 14.7003421783447\\
28.1298713684082 14.7003564834595\\
28.1340389251709 14.700343132019\\
28.1376724243164 14.7003440856934\\
28.1416683197021 14.7003545761108\\
28.1460647583008 14.7003440856934\\
28.1490650177002 14.7003469467163\\
28.1523666381836 14.7003498077393\\
28.1559963226318 14.7003479003906\\
28.1599903106689 14.7003479003906\\
28.1643829345703 14.7003479003906\\
28.1692161560059 14.7003479003906\\
28.1745319366455 14.7003479003906\\
28.1803779602051 14.7003479003906\\
28.1868114471436 14.7003479003906\\
28.1938858032227 14.7003479003906\\
28.2016677856445 14.7003479003906\\
28.2102298736572 14.7003479003906\\
28.2196464538574 14.7003479003906\\
28.2300052642822 14.7003479003906\\
28.2413997650146 14.7003479003906\\
28.2539348602295 14.7003479003906\\
28.2677211761475 14.7003479003906\\
28.2828884124756 14.7003479003906\\
28.2995700836182 14.7003479003906\\
28.3179225921631 14.7003469467163\\
28.3381080627441 14.7003469467163\\
28.3603134155273 14.7003469467163\\
28.3847389221191 14.7003469467163\\
28.4116058349609 14.7003469467163\\
28.4411602020264 14.7003469467163\\
28.4736709594727 14.7003469467163\\
28.5094318389893 14.7003469467163\\
28.5487689971924 14.7003469467163\\
28.592041015625 14.7003469467163\\
28.6396389007568 14.7003469467163\\
28.6919975280762 14.7003469467163\\
28.7495918273926 14.7003469467163\\
28.8129444122314 14.7003469467163\\
28.8826332092285 14.7003469467163\\
28.9592895507813 14.7003469467163\\
29.0436134338379 14.7003469467163\\
29.1363677978516 14.7003469467163\\
29.2363681793213 14.7003479003906\\
29.336368560791 14.7003469467163\\
29.4363689422607 14.7003736495972\\
29.4563694000244 14.7004823684692\\
29.4642658233643 14.7006006240845\\
29.471960067749 14.7006597518921\\
29.4804267883301 14.7006950378418\\
29.4885120391846 14.7007246017456\\
29.497407913208 14.7007055282593\\
29.5071926116943 14.7005643844604\\
29.5179557800293 14.6989126205444\\
29.5297946929932 14.7004337310791\\
29.542818069458 14.7004146575928\\
29.5571441650391 14.7008352279663\\
29.5729026794434 14.7007894515991\\
29.5902366638184 14.7004041671753\\
29.6093044281006 14.70055103302\\
29.630277633667 14.7003154754639\\
29.6533489227295 14.7002925872803\\
29.6787281036377 14.7001953125\\
29.6999015808105 14.7001905441284\\
29.7231941223145 14.7000780105591\\
29.7488136291504 14.6999788284302\\
29.7769947052002 14.7002515792847\\
29.8079967498779 14.700366973877\\
29.8420963287354 14.7003374099731\\
29.8796062469482 14.7003555297852\\
29.9208679199219 14.7003479003906\\
29.9662551879883 14.700366973877\\
30.0161819458008 14.7002906799316\\
30.0711002349854 14.700400352478\\
30.1315116882324 14.7003583908081\\
30.1804618835449 14.7003545761108\\
30.2286376953125 14.7003517150879\\
30.2746448516846 14.7003469467163\\
30.3156337738037 14.7003498077393\\
30.3508148193359 14.7003412246704\\
30.3895168304443 14.7003383636475\\
30.4320869445801 14.7003421783447\\
30.4789142608643 14.7003450393677\\
30.530424118042 14.700345993042\\
30.587085723877 14.7003469467163\\
30.6491241455078 14.7003469467163\\
30.7173671722412 14.7003469467163\\
30.7924346923828 14.7003479003906\\
30.8750076293945 14.7003469467163\\
30.9658393859863 14.7003469467163\\
31.0657539367676 14.7003479003906\\
31.1657543182373 14.7003479003906\\
31.265754699707 14.7003479003906\\
31.3657531738281 14.7003488540649\\
31.4657535552979 14.7003479003906\\
31.5657539367676 14.7003479003906\\
31.6657543182373 14.7003479003906\\
31.765754699707 14.7003479003906\\
31.8657531738281 14.7003479003906\\
31.9657535552979 14.7003479003906\\
32.0657539367676 14.7003479003906\\
32.1657524108887 14.7003479003906\\
32.265754699707 14.7003479003906\\
32.3657531738281 14.7003479003906\\
32.4657554626465 14.7003479003906\\
32.5657539367676 14.7003479003906\\
32.6657524108887 14.7003479003906\\
32.765754699707 14.7003479003906\\
32.8657531738281 14.7003479003906\\
32.9657554626465 14.7003479003906\\
33.0657539367676 14.7003479003906\\
33.1657524108887 14.7003479003906\\
33.265754699707 14.7003479003906\\
33.3657531738281 14.7003479003906\\
33.4657554626465 14.7003479003906\\
33.5657539367676 14.7003479003906\\
33.6657524108887 14.7003479003906\\
33.765754699707 14.7003479003906\\
33.8657531738281 14.7003479003906\\
33.9657554626465 14.7003479003906\\
34.0657539367676 14.7003479003906\\
34.1657524108887 14.7003479003906\\
34.265754699707 14.7003479003906\\
34.3657531738281 14.7003479003906\\
34.4657554626465 14.7003479003906\\
34.5657539367676 14.7003479003906\\
34.6657524108887 14.7003479003906\\
34.765754699707 14.7003479003906\\
34.8657531738281 14.7003479003906\\
34.9657554626465 14.7003479003906\\
35.0657539367676 14.7003479003906\\
35.1657524108887 14.7003479003906\\
35.265754699707 14.7003479003906\\
35.3657531738281 14.7003479003906\\
35.4657554626465 14.7003479003906\\
35.5657539367676 14.7003479003906\\
35.6657524108887 14.7003479003906\\
35.765754699707 14.7003479003906\\
35.8657531738281 14.7003479003906\\
35.9657554626465 14.7003479003906\\
36.0657539367676 14.7003479003906\\
36.1657524108887 14.7003479003906\\
36.265754699707 14.7003479003906\\
36.3657531738281 14.7003479003906\\
36.4657554626465 14.7003479003906\\
36.5657539367676 14.7003479003906\\
36.6657524108887 14.7003479003906\\
36.765754699707 14.7003479003906\\
36.8657531738281 14.7003479003906\\
36.9657554626465 14.7003479003906\\
37.0657539367676 14.7003479003906\\
37.1539344787598 14.7003479003906\\
37.2244758605957 14.7003488540649\\
37.302074432373 14.7003479003906\\
37.3874282836914 14.7003479003906\\
37.4813194274902 14.7003469467163\\
37.5215835571289 14.7003488540649\\
37.5658721923828 14.7003450393677\\
37.585823059082 14.7003545761108\\
37.6024284362793 14.7003498077393\\
37.6206932067871 14.7003488540649\\
37.6407852172852 14.7003479003906\\
37.6628875732422 14.7003479003906\\
37.687198638916 14.7003479003906\\
37.713939666748 14.7003479003906\\
37.7433547973633 14.7003479003906\\
37.7757110595703 14.7003479003906\\
37.8113059997559 14.7003479003906\\
37.8504600524902 14.7003479003906\\
37.8935279846191 14.7003479003906\\
37.9409027099609 14.7003479003906\\
37.9930152893066 14.7003479003906\\
38.0503387451172 14.7003479003906\\
38.1133918762207 14.7003479003906\\
38.1827545166016 14.7003479003906\\
38.2590522766113 14.7003479003906\\
38.3429794311523 14.7003469467163\\
38.4352989196777 14.7003479003906\\
38.5352973937988 14.7003479003906\\
38.6352996826172 14.7003574371338\\
38.7352981567383 14.7004623413086\\
38.7853012084961 14.7010707855225\\
38.822696685791 14.7000131607056\\
38.8252944946289 14.6976318359375\\
38.8281555175781 14.6985931396484\\
38.8313026428223 14.699257850647\\
38.8347625732422 14.7014503479004\\
38.838565826416 14.7011079788208\\
38.8427543640137 14.701021194458\\
38.8473587036133 14.7009754180908\\
38.852424621582 14.7008266448975\\
38.8579978942871 14.7006378173828\\
38.864128112793 14.7004814147949\\
38.8708724975586 14.7003488540649\\
38.878288269043 14.7002372741699\\
38.8864479064941 14.7001705169678\\
38.8954238891602 14.6996898651123\\
38.9052963256836 14.6998748779297\\
38.9161567687988 14.6991586685181\\
38.9281005859375 14.7000303268433\\
38.9412422180176 14.7002716064453\\
38.955696105957 14.7004041671753\\
38.9715957641602 14.7005033493042\\
38.989086151123 14.700216293335\\
39.0083236694336 14.6998300552368\\
39.029483795166 14.7004337310791\\
39.0527648925781 14.7003335952759\\
39.078369140625 14.7003555297852\\
39.1065368652344 14.7003383636475\\
39.137523651123 14.7003421783447\\
39.1716041564941 14.7003450393677\\
39.2090950012207 14.7003488540649\\
39.2477378845215 14.7003507614136\\
39.2902450561523 14.7003498077393\\
39.3370056152344 14.7003488540649\\
39.3884391784668 14.7003479003906\\
39.4450187683105 14.7003498077393\\
39.4936180114746 14.7003498077393\\
39.5406875610352 14.7003488540649\\
39.5853233337402 14.7003488540649\\
39.6272773742676 14.7003479003906\\
39.6734275817871 14.7003479003906\\
39.7241897583008 14.7003479003906\\
39.780029296875 14.7003479003906\\
39.8414497375488 14.7003479003906\\
39.9090156555176 14.7003479003906\\
39.9833374023438 14.7003479003906\\
40.0650939941406 14.7003479003906\\
40.1550216674805 14.7003479003906\\
40.2539443969727 14.7003479003906\\
40.353946685791 14.7003479003906\\
40.4539451599121 14.7003479003906\\
40.5539436340332 14.7003479003906\\
40.5569686889648 14.7003364562988\\
40.5602951049805 14.7003688812256\\
40.5639533996582 14.7002782821655\\
40.5665168762207 14.7004060745239\\
40.5693359375 14.7003755569458\\
40.572437286377 14.7003507614136\\
40.5758476257324 14.700345993042\\
40.5796012878418 14.7003479003906\\
40.5837287902832 14.7003488540649\\
40.5882682800293 14.7003488540649\\
40.59326171875 14.7003488540649\\
40.5987548828125 14.7003488540649\\
40.6048011779785 14.7003488540649\\
40.6114463806152 14.7003488540649\\
40.6187591552734 14.7003488540649\\
40.6268005371094 14.7003479003906\\
40.6356506347656 14.7003479003906\\
40.6453819274902 14.7003479003906\\
40.6560897827148 14.7003479003906\\
40.6678657531738 14.7003479003906\\
40.6808204650879 14.7003479003906\\
40.695068359375 14.7003479003906\\
40.7107429504395 14.7003479003906\\
40.7279853820801 14.7003479003906\\
40.7469482421875 14.7003479003906\\
40.7678108215332 14.7003479003906\\
40.7907600402832 14.7003479003906\\
40.8160057067871 14.7003479003906\\
40.8437728881836 14.7003479003906\\
40.8743171691895 14.7003479003906\\
40.9079170227051 14.7003479003906\\
40.9448776245117 14.7003479003906\\
40.9855308532715 14.7003479003906\\
41.0302505493164 14.7003479003906\\
41.0794448852539 14.7003479003906\\
41.1335563659668 14.7003479003906\\
41.1930809020996 14.7003479003906\\
41.2585563659668 14.7003479003906\\
41.3305778503418 14.7003479003906\\
41.4098052978516 14.7003479003906\\
41.4969520568848 14.7003469467163\\
41.5928153991699 14.7003479003906\\
41.692813873291 14.7003479003906\\
41.7928161621094 14.7003746032715\\
41.812816619873 14.7004899978638\\
41.8206024169922 14.7006168365479\\
41.8281707763672 14.7006797790527\\
41.8364944458008 14.7007160186768\\
41.8444328308105 14.7007493972778\\
41.8531684875488 14.7007293701172\\
41.8627777099609 14.7006015777588\\
41.8733444213867 14.6990594863892\\
41.8849716186523 14.700174331665\\
41.8977584838867 14.700478553772\\
41.9118270874023 14.7008361816406\\
41.9272994995117 14.7008180618286\\
41.9443206787109 14.7004022598267\\
41.9630432128906 14.7005605697632\\
41.9836387634277 14.700306892395\\
42.0062942504883 14.700345993042\\
42.0312156677246 14.7001142501831\\
42.0477752685547 14.7001657485962\\
42.0659866333008 14.700083732605\\
42.0860252380371 14.6998643875122\\
42.1080627441406 14.7001838684082\\
42.1323051452637 14.7003002166748\\
42.1589736938477 14.7003507614136\\
42.1883087158203 14.7003412246704\\
42.2205772399902 14.7003536224365\\
42.2560729980469 14.7003583908081\\
42.2951164245605 14.7003479003906\\
42.3380661010742 14.7003488540649\\
42.3853073120117 14.7003517150879\\
42.4372749328613 14.7003545761108\\
42.494441986084 14.7003479003906\\
42.5416946411133 14.7003593444824\\
42.5936737060547 14.7003707885742\\
42.6379699707031 14.7003507614136\\
42.6777877807617 14.7003393173218\\
42.711311340332 14.7003374099731\\
42.7481842041016 14.7003402709961\\
42.788745880127 14.7003440856934\\
42.833366394043 14.700345993042\\
42.8824462890625 14.7003479003906\\
42.9364318847656 14.7003479003906\\
42.9958190917969 14.7003479003906\\
43.0611457824707 14.7003479003906\\
43.1330032348633 14.7003479003906\\
43.2120475769043 14.7003469467163\\
43.2989959716797 14.7003488540649\\
43.3946380615234 14.7003469467163\\
43.4946365356445 14.7003479003906\\
43.5946388244629 14.7003488540649\\
43.694637298584 14.7003488540649\\
43.7946357727051 14.7003488540649\\
43.8946380615234 14.7003488540649\\
43.9946365356445 14.7003488540649\\
44.0946388244629 14.7003488540649\\
44.194637298584 14.7003488540649\\
44.2946357727051 14.7003488540649\\
44.3946380615234 14.7003488540649\\
44.4946365356445 14.7003488540649\\
44.5946388244629 14.7003488540649\\
44.694637298584 14.7003488540649\\
44.7946357727051 14.7003488540649\\
44.8946380615234 14.7003488540649\\
44.9946365356445 14.7003488540649\\
45.0946388244629 14.7003488540649\\
45.194637298584 14.7003488540649\\
45.2946357727051 14.7003488540649\\
45.3946380615234 14.7003479003906\\
45.4946365356445 14.7003479003906\\
45.5946388244629 14.7003479003906\\
45.694637298584 14.7003479003906\\
45.7946357727051 14.7003479003906\\
45.8946380615234 14.7003479003906\\
45.9946365356445 14.7003479003906\\
46.0946388244629 14.7003479003906\\
46.194637298584 14.7003479003906\\
46.2946357727051 14.7003479003906\\
46.3946380615234 14.7003479003906\\
46.4946365356445 14.7003479003906\\
46.5946388244629 14.7003479003906\\
46.694637298584 14.7003479003906\\
46.7946357727051 14.7003479003906\\
46.8946380615234 14.7003479003906\\
46.9946365356445 14.7003479003906\\
47.0946388244629 14.7003479003906\\
47.194637298584 14.7003479003906\\
47.2946357727051 14.7003479003906\\
47.3946380615234 14.7003479003906\\
47.4946365356445 14.7003479003906\\
47.5946388244629 14.7003479003906\\
47.694637298584 14.7003479003906\\
47.7946357727051 14.7003479003906\\
47.8946380615234 14.7003479003906\\
47.9946365356445 14.7003479003906\\
48.0946388244629 14.7003479003906\\
48.194637298584 14.7003479003906\\
48.2946357727051 14.7003479003906\\
48.3946380615234 14.7003479003906\\
48.4946365356445 14.7003479003906\\
48.5946388244629 14.7003479003906\\
48.694637298584 14.7003479003906\\
48.7946357727051 14.7003479003906\\
48.8946380615234 14.7003488540649\\
48.9946365356445 14.7003479003906\\
49.0946388244629 14.7003479003906\\
49.194637298584 14.7003479003906\\
49.2946357727051 14.7003479003906\\
49.3946380615234 14.7003479003906\\
49.4946365356445 14.7003479003906\\
49.5842514038086 14.7003479003906\\
49.6559371948242 14.7003488540649\\
49.7347946166992 14.7003488540649\\
49.8215370178223 14.7003479003906\\
49.9169502258301 14.7003479003906\\
49.9568328857422 14.7003488540649\\
50.0007057189941 14.7003450393677\\
50.0206336975098 14.7003545761108\\
50.0371971130371 14.7003507614136\\
50.0554161071777 14.7003488540649\\
50.0754585266113 14.7003488540649\\
50.0975074768066 14.7003488540649\\
50.1217575073242 14.7003479003906\\
50.1484336853027 14.7003479003906\\
50.1777763366699 14.7003479003906\\
50.2100563049316 14.7003479003906\\
50.2455635070801 14.7003479003906\\
50.2846183776855 14.7003479003906\\
50.327579498291 14.7003479003906\\
50.3748397827148 14.7003479003906\\
50.4268226623535 14.7003479003906\\
50.4840087890625 14.7003479003906\\
50.5469093322754 14.7003479003906\\
50.6161003112793 14.7003479003906\\
50.692211151123 14.7003479003906\\
50.7759323120117 14.7003479003906\\
50.8680267333984 14.7003488540649\\
50.9680252075195 14.7003479003906\\
51.0680236816406 14.7003574371338\\
51.168025970459 14.7004623413086\\
51.2180252075195 14.7010726928711\\
51.2553977966309 14.7021503448486\\
51.2579727172852 14.6976871490479\\
51.2608032226563 14.698637008667\\
51.263916015625 14.6992158889771\\
51.2673416137695 14.7015361785889\\
51.2711067199707 14.7010831832886\\
51.2752532958984 14.7010354995728\\
51.2798080444336 14.7009935379028\\
51.2848243713379 14.7008466720581\\
51.290340423584 14.700665473938\\
51.2964057922363 14.700511932373\\
51.3030815124512 14.7003793716431\\
51.3104209899902 14.7002696990967\\
51.3184967041016 14.7002229690552\\
51.3273811340332 14.699670791626\\
51.3371505737305 14.6998510360718\\
51.347900390625 14.6991453170776\\
51.3597221374512 14.6999034881592\\
51.372730255127 14.7002286911011\\
51.387035369873 14.7003879547119\\
51.4027709960938 14.7005043029785\\
51.4200820922852 14.7003011703491\\
51.4391212463379 14.6997737884521\\
51.4600677490234 14.7004270553589\\
51.4831085205078 14.7003297805786\\
51.5084533691406 14.7003545761108\\
51.5363311767578 14.7003393173218\\
51.5669975280762 14.700343132019\\
51.6007308959961 14.700345993042\\
51.6378364562988 14.7003498077393\\
51.6764678955078 14.7003517150879\\
51.7189598083496 14.7003507614136\\
51.7657051086426 14.7003498077393\\
51.8171195983887 14.7003479003906\\
51.8736801147461 14.700343132019\\
51.9213600158691 14.7003517150879\\
51.9687728881836 14.7003507614136\\
52.0141105651855 14.7003507614136\\
52.056568145752 14.7003498077393\\
52.103271484375 14.7003498077393\\
52.1546440124512 14.7003488540649\\
52.2111549377441 14.7003488540649\\
52.2733154296875 14.7003498077393\\
52.3416938781738 14.7003488540649\\
52.4169082641602 14.7003488540649\\
52.4996490478516 14.7003488540649\\
52.5906600952148 14.7003488540649\\
52.6906585693359 14.7003488540649\\
52.790657043457 14.7003488540649\\
52.8906593322754 14.7003488540649\\
52.9906578063965 14.7003488540649\\
52.9936103820801 14.7003335952759\\
52.9968528747559 14.7003774642944\\
53.0004234313965 14.7002782821655\\
53.0029792785645 14.7004041671753\\
53.0057907104492 14.7003717422485\\
53.0088844299316 14.7003526687622\\
53.0122871398926 14.7003469467163\\
53.0160293579102 14.7003488540649\\
53.0201454162598 14.7003498077393\\
53.0246734619141 14.7003498077393\\
53.0296516418457 14.7003498077393\\
53.0351333618164 14.7003498077393\\
53.0411605834961 14.7003498077393\\
53.0477867126465 14.7003498077393\\
53.0550804138184 14.7003498077393\\
53.063102722168 14.7003488540649\\
53.0719261169434 14.7003488540649\\
53.0816307067871 14.7003488540649\\
53.0923080444336 14.7003488540649\\
53.1040496826172 14.7003488540649\\
53.1169700622559 14.7003488540649\\
53.1311798095703 14.7003488540649\\
53.1468124389648 14.7003488540649\\
53.164005279541 14.7003488540649\\
53.1829223632813 14.7003488540649\\
53.203727722168 14.7003488540649\\
53.2266120910645 14.7003488540649\\
53.2517852783203 14.7003488540649\\
53.2794799804688 14.7003488540649\\
53.3099403381348 14.7003488540649\\
53.343448638916 14.7003488540649\\
53.3803062438965 14.7003488540649\\
53.4208488464355 14.7003488540649\\
53.465446472168 14.7003488540649\\
53.5145034790039 14.7003488540649\\
53.5684700012207 14.7003488540649\\
53.6278305053711 14.7003488540649\\
53.6931266784668 14.7003488540649\\
53.7649536132813 14.7003488540649\\
53.8439598083496 14.7003488540649\\
53.9308700561523 14.7003479003906\\
54.0264701843262 14.7003488540649\\
54.1264686584473 14.7003479003906\\
54.2264709472656 14.7003755569458\\
54.2464714050293 14.7004928588867\\
54.2542266845703 14.7006187438965\\
54.2617721557617 14.7006797790527\\
54.2700691223145 14.7007141113281\\
54.2779922485352 14.7007436752319\\
54.2867050170898 14.7007160186768\\
54.2962913513184 14.7005825042725\\
54.3068389892578 14.699013710022\\
54.3184356689453 14.7001657485962\\
54.3311958312988 14.7004709243774\\
54.3452301025391 14.700831413269\\
54.360668182373 14.7008171081543\\
54.3776512145996 14.7004098892212\\
54.3963317871094 14.7005748748779\\
54.4168815612793 14.7003307342529\\
54.4394836425781 14.7003803253174\\
54.4643478393555 14.7001209259033\\
54.4809799194336 14.7001485824585\\
54.4992752075195 14.7000503540039\\
54.519401550293 14.6998147964478\\
54.5415382385254 14.7001714706421\\
54.5658912658691 14.7003021240234\\
54.5926780700684 14.7003574371338\\
54.622142791748 14.7003479003906\\
54.6545562744141 14.7003593444824\\
54.6902084350586 14.700364112854\\
54.7294273376465 14.7003507614136\\
54.7725677490234 14.7003536224365\\
54.8200225830078 14.7003517150879\\
54.8722229003906 14.7003564834595\\
54.9296417236328 14.7003488540649\\
54.9768371582031 14.700382232666\\
55.028751373291 14.7003412246704\\
55.0732498168945 14.7003545761108\\
55.112865447998 14.7003421783447\\
55.1461601257324 14.7003393173218\\
55.1827850341797 14.7003412246704\\
55.223072052002 14.7003440856934\\
55.2673873901367 14.700345993042\\
55.3161315917969 14.7003469467163\\
55.3697547912598 14.7003469467163\\
55.4287376403809 14.7003479003906\\
55.4936180114746 14.7003479003906\\
55.5649871826172 14.7003479003906\\
55.643497467041 14.7003469467163\\
55.7298545837402 14.7003488540649\\
55.8248481750488 14.7003469467163\\
55.9248466491699 14.7003479003906\\
56.024845123291 14.7003488540649\\
56.1248474121094 14.7003488540649\\
56.2248458862305 14.7003488540649\\
56.3248481750488 14.7003488540649\\
56.4248466491699 14.7003488540649\\
56.524845123291 14.7003488540649\\
56.6248474121094 14.7003488540649\\
56.7248458862305 14.7003488540649\\
56.8248481750488 14.7003488540649\\
56.9248466491699 14.7003488540649\\
57.024845123291 14.7003488540649\\
57.1248474121094 14.7003488540649\\
57.2248458862305 14.7003488540649\\
57.3248481750488 14.7003488540649\\
57.4248466491699 14.7003488540649\\
57.524845123291 14.7003488540649\\
57.6248474121094 14.7003488540649\\
57.7248458862305 14.7003488540649\\
57.8248481750488 14.7003488540649\\
57.9248466491699 14.7003488540649\\
58.024845123291 14.7003479003906\\
58.1248474121094 14.7003479003906\\
58.2248458862305 14.7003479003906\\
58.3248481750488 14.7003479003906\\
58.4248466491699 14.7003479003906\\
58.524845123291 14.7003479003906\\
58.6248474121094 14.7003479003906\\
58.7248458862305 14.7003479003906\\
58.8248481750488 14.7003479003906\\
58.9248466491699 14.7003479003906\\
59.024845123291 14.7003479003906\\
59.1248474121094 14.7003479003906\\
59.2248458862305 14.7003479003906\\
59.3248481750488 14.7003479003906\\
59.4248466491699 14.7003479003906\\
59.524845123291 14.7003479003906\\
59.6248474121094 14.7003479003906\\
59.7248458862305 14.7003479003906\\
59.8248481750488 14.7003479003906\\
59.9124221801758 14.7003479003906\\
60 14.7003479003906\\
60.0963325500488 14.7003479003906\\
60.1963348388672 14.7003479003906\\
60.2963333129883 14.7003479003906\\
60.3963356018066 14.7003479003906\\
60.4963340759277 14.7003479003906\\
60.5963325500488 14.7003479003906\\
60.6963348388672 14.7003479003906\\
60.7963333129883 14.7003479003906\\
60.8963356018066 14.7003479003906\\
60.9963340759277 14.7003479003906\\
61.0963325500488 14.7003479003906\\
61.1963348388672 14.7003479003906\\
61.2963333129883 14.7003479003906\\
61.3963356018066 14.7003479003906\\
61.4963340759277 14.7003479003906\\
61.5963325500488 14.7003479003906\\
61.6963348388672 14.7003479003906\\
61.7963333129883 14.7003479003906\\
61.8963356018066 14.7003479003906\\
61.9940414428711 14.7003479003906\\
62.0722236633301 14.7003479003906\\
62.1347732543945 14.7003479003906\\
62.2035751342773 14.700345993042\\
62.2478866577148 14.7003479003906\\
62.296630859375 14.700345993042\\
62.3256530761719 14.7003498077393\\
62.3496742248535 14.7003479003906\\
62.376091003418 14.7003469467163\\
62.4051551818848 14.700343132019\\
62.4213905334473 14.7003574371338\\
62.4340591430664 14.7003517150879\\
62.4479942321777 14.7003498077393\\
62.463321685791 14.7003488540649\\
62.4801826477051 14.7003488540649\\
62.4987297058105 14.7003488540649\\
62.5191307067871 14.7003488540649\\
62.5415687561035 14.7003488540649\\
62.5662574768066 14.7003488540649\\
62.5934104919434 14.7003488540649\\
62.6232795715332 14.7003488540649\\
62.656135559082 14.7003488540649\\
62.6922760009766 14.7003488540649\\
62.7320327758789 14.7003488540649\\
62.775764465332 14.7003488540649\\
62.8238677978516 14.7003488540649\\
62.8767852783203 14.7003498077393\\
62.9349899291992 14.7003498077393\\
62.9990196228027 14.7003498077393\\
63.069450378418 14.7003498077393\\
63.1469230651855 14.7003498077393\\
63.2321434020996 14.7003488540649\\
63.3258857727051 14.7003498077393\\
63.4258842468262 14.7003488540649\\
63.5258865356445 14.7003688812256\\
63.6258850097656 14.7004365921021\\
63.6758842468262 14.7009696960449\\
63.7138481140137 14.701434135437\\
63.7273941040039 14.7902860641479\\
63.7422943115234 14.8020162582397\\
63.7586822509766 14.7965803146362\\
63.7767105102539 14.7740345001221\\
63.7965431213379 14.7503843307495\\
63.8183555603027 14.7378845214844\\
63.8423538208008 14.7301378250122\\
63.8687477111816 14.7242231369019\\
63.8977813720703 14.7234687805176\\
63.9297218322754 14.7159404754639\\
63.9648513793945 14.7040414810181\\
64.0015869140625 14.700704574585\\
64.022705078125 14.7003011703491\\
64.0459365844727 14.7003479003906\\
64.0714874267578 14.7003374099731\\
64.0996017456055 14.7003383636475\\
64.1305236816406 14.7003412246704\\
64.1645355224609 14.7003479003906\\
64.2019424438477 14.7003488540649\\
64.2431030273438 14.7003479003906\\
64.2883682250977 14.7003469467163\\
64.3381729125977 14.7003469467163\\
64.3929443359375 14.700345993042\\
64.4477233886719 14.7003469467163\\
64.4993057250977 14.7003469467163\\
64.5481796264648 14.7003469467163\\
64.5940780639648 14.7003469467163\\
64.6370315551758 14.700345993042\\
64.6842803955078 14.700345993042\\
64.7362518310547 14.7003469467163\\
64.7934188842773 14.7003479003906\\
64.8563079833984 14.7003479003906\\
64.9254837036133 14.7003479003906\\
65.0015716552734 14.7003488540649\\
65.0852737426758 14.7003488540649\\
65.1773452758789 14.7003479003906\\
65.27734375 14.7003488540649\\
65.3773422241211 14.7003479003906\\
65.4773483276367 14.7003498077393\\
65.5773468017578 14.7003479003906\\
65.5794677734375 14.7003078460693\\
65.5817947387695 14.70041847229\\
65.5843658447266 14.7002296447754\\
65.5871810913086 14.7003955841064\\
65.5902862548828 14.7003965377808\\
65.5937042236328 14.7002897262573\\
65.5974578857422 14.7003765106201\\
65.6015930175781 14.7003622055054\\
65.6061325073242 14.7003507614136\\
65.6111373901367 14.7003498077393\\
65.6166305541992 14.7003498077393\\
65.6226806640625 14.7003498077393\\
65.6293334960938 14.7003498077393\\
65.6366577148438 14.7003498077393\\
65.6447067260742 14.7003498077393\\
65.653564453125 14.7003498077393\\
65.6633071899414 14.7003498077393\\
65.6740188598633 14.7003498077393\\
65.6858062744141 14.7003488540649\\
65.6987762451172 14.7003488540649\\
65.7130355834961 14.7003488540649\\
65.7287292480469 14.7003488540649\\
65.7459869384766 14.7003488540649\\
65.7649688720703 14.7003488540649\\
65.7858505249023 14.7003488540649\\
65.8088226318359 14.7003488540649\\
65.8340911865234 14.7003488540649\\
65.8618850708008 14.7003488540649\\
65.892463684082 14.7003488540649\\
65.9260940551758 14.7003488540649\\
65.9630889892578 14.7003488540649\\
66.0037841796875 14.7003488540649\\
66.0485458374023 14.7003488540649\\
66.097785949707 14.7003488540649\\
66.1519470214844 14.7003488540649\\
66.2115325927734 14.7003488540649\\
66.2770690917969 14.7003488540649\\
66.3491592407227 14.7003488540649\\
66.428466796875 14.7003488540649\\
66.5156936645508 14.7003479003906\\
66.6116485595703 14.7003498077393\\
66.7116470336914 14.7003479003906\\
66.811653137207 14.7003841400146\\
66.8316497802734 14.7006282806396\\
66.8391723632813 14.7008333206177\\
66.8464965820313 14.7009515762329\\
66.8545455932617 14.7010726928711\\
66.8622512817383 14.7037649154663\\
66.870735168457 14.7645597457886\\
66.8800659179688 14.88401222229\\
66.8903274536133 14.8875246047974\\
66.9016189575195 14.8243007659912\\
66.9140319824219 14.7807149887085\\
66.9276962280273 14.7898187637329\\
66.9427185058594 14.7792625427246\\
66.9592437744141 14.765793800354\\
66.977424621582 14.7513475418091\\
66.997428894043 14.7169055938721\\
67.0194244384766 14.7028465270996\\
67.0415725708008 14.7005729675293\\
67.0659408569336 14.7002840042114\\
67.0927429199219 14.7012214660645\\
67.1222229003906 14.7350797653198\\
67.154655456543 14.7727890014648\\
67.1903305053711 14.7879161834717\\
67.1995849609375 14.7818479537964\\
67.2097702026367 14.7694807052612\\
67.2209701538086 14.7620239257813\\
67.233283996582 14.7587099075317\\
67.2468414306641 14.7567577362061\\
67.2617416381836 14.7551498413086\\
67.2781448364258 14.7534780502319\\
67.2961807250977 14.7516660690308\\
67.3160247802734 14.7470388412476\\
67.3378448486328 14.7377843856812\\
67.3618545532227 14.7399091720581\\
67.3882598876953 14.7354564666748\\
67.4173126220703 14.7364654541016\\
67.4492645263672 14.7367925643921\\
67.4844131469727 14.732813835144\\
67.523078918457 14.7283630371094\\
67.5656127929688 14.7250785827637\\
67.6123962402344 14.7210493087769\\
67.6579818725586 14.7155752182007\\
67.6993789672852 14.7099475860596\\
67.7355346679688 14.7048835754395\\
67.7753067016602 14.7029294967651\\
67.8190612792969 14.7023687362671\\
67.8671875 14.7024431228638\\
67.9201278686523 14.7029361724854\\
67.9783630371094 14.7038545608521\\
68.0421829223633 14.7059469223022\\
68.1070251464844 14.7100410461426\\
68.178352355957 14.7163944244385\\
68.2568130493164 14.724552154541\\
68.3431243896484 14.7282743453979\\
68.4380569458008 14.7308950424194\\
68.5380630493164 14.7322807312012\\
68.6380615234375 14.7326211929321\\
68.7380599975586 14.7323122024536\\
68.8380584716797 14.7327833175659\\
68.9380569458008 14.7331352233887\\
69.0380630493164 14.7338466644287\\
69.1380615234375 14.7342290878296\\
69.2380599975586 14.7348299026489\\
69.3380584716797 14.7352876663208\\
69.4380569458008 14.7358417510986\\
69.5380630493164 14.7363319396973\\
69.6380615234375 14.7368640899658\\
69.7380599975586 14.7373695373535\\
69.8380584716797 14.7378931045532\\
69.9380569458008 14.7384090423584\\
70.0380630493164 14.7389307022095\\
70.1380615234375 14.7394523620605\\
70.2380599975586 14.7399759292603\\
70.3380584716797 14.7404985427856\\
70.4380569458008 14.741021156311\\
70.5380630493164 14.7415437698364\\
70.6380615234375 14.7420673370361\\
70.7380599975586 14.7425899505615\\
70.8380584716797 14.7431125640869\\
70.9380569458008 14.7437858581543\\
71.0380630493164 14.7445287704468\\
71.1380615234375 14.7451238632202\\
71.2380599975586 14.7457580566406\\
71.3380584716797 14.7463788986206\\
71.4380569458008 14.7470006942749\\
71.5380630493164 14.7476215362549\\
71.6380615234375 14.7482404708862\\
71.7380599975586 14.7488584518433\\
71.8380584716797 14.7494745254517\\
71.9380569458008 14.7500896453857\\
72.0380630493164 14.7507038116455\\
72.1380615234375 14.7513160705566\\
72.2380599975586 14.7519283294678\\
72.3380584716797 14.7525386810303\\
72.4380569458008 14.7531461715698\\
72.5380630493164 14.7548904418945\\
72.6380615234375 14.7547874450684\\
72.7380599975586 14.7550449371338\\
72.8380584716797 14.7556629180908\\
72.9380569458008 14.7561769485474\\
73.0380630493164 14.7567272186279\\
73.1380615234375 14.7572622299194\\
73.2380599975586 14.7578001022339\\
73.3380584716797 14.7583351135254\\
73.4380569458008 14.7588682174683\\
73.5380630493164 14.7594137191772\\
73.6380615234375 14.7599782943726\\
73.7380599975586 14.7606029510498\\
73.8380584716797 14.7611846923828\\
73.9380569458008 14.7617750167847\\
74.0380630493164 14.762393951416\\
74.1380615234375 14.7630443572998\\
74.2380599975586 14.7637500762939\\
74.3380584716797 14.7663059234619\\
74.4363632202148 14.7723417282104\\
74.5150299072266 14.7770299911499\\
74.5779571533203 14.7680969238281\\
74.6471862792969 14.7682504653931\\
74.7233276367188 14.7686100006104\\
74.8070907592773 14.7690992355347\\
74.8992309570313 14.7856187820435\\
74.945930480957 14.7393989562988\\
74.9729614257813 14.7633609771729\\
74.9864807128906 14.7790222167969\\
75 14.7797803878784\\
};
\addlegendentry{22 "};

\end{axis}
\end{tikzpicture}%
\caption{Filling problem with \dtmax{} = \expneg{1.0}{1}{[s]} with the linear solver.}
\label{fig:linFill1em1}
\end{figure}

The solution at this \dtmax{} is rather noisy and when the actual timestep sizes are examined, \dtmax{} is infrequently reached due to restrictive \dtcrnt{} limits.
The measure of error, $e_{p}$, for this simulation was \expneg{7.5}{2}{[psia]}.

\begin{figure}[h!tb]
\centering
% This file was created by matlab2tikz v0.4.3.
% Copyright (c) 2008--2013, Nico Schlömer <nico.schloemer@gmail.com>
% All rights reserved.
% 
\tikzsetnextfilename{plots/vmpLinear5em2_eps}
\begin{tikzpicture}

\begin{axis}[%
width=0.8\textwidth,
%height=0.630967741935484\textwidth,
height=0.491294629700995\textwidth,
scale only axis,
xmin=0,
xmax=75,
xlabel={Time [s]},
ymin=14.6,
ymax=15.6,
ylabel={Pressure [psia]},
legend style={at={(0.03,0.97)},anchor=north west,draw=black,fill=white,legend cell align=left}
]
\addplot [
color=black,
solid
]
table[row sep=crcr]{
0 14.7007818222046\\
9.99999971718069e-10 14.7007818222046\\
2.49999998480632e-09 14.7007818222046\\
4.74999994892755e-09 14.7007818222046\\
8.1250002281763e-09 14.7007818222046\\
1.31874999809156e-08 14.7007818222046\\
2.0781250498203e-08 14.7007818222046\\
3.21718758300449e-08 14.7007818222046\\
4.92578138278077e-08 14.7007818222046\\
7.48867208244519e-08 14.7007818222046\\
1.13330081319418e-07 14.7007818222046\\
1.70995122061868e-07 14.7007818222046\\
2.57492672517401e-07 14.7007818222046\\
3.8723902662241e-07 14.7007818222046\\
5.81858500936505e-07 14.7007818222046\\
8.73787769251066e-07 14.7007818222046\\
1.31168167172291e-06 14.7007818222046\\
1.9685226106958e-06 14.7007827758789\\
2.95378367809462e-06 14.7007827758789\\
4.43167573394021e-06 14.7007827758789\\
6.64851359033491e-06 14.7007827758789\\
9.97376992017962e-06 14.7007837295532\\
1.49616553244414e-05 14.7007846832275\\
2.24434825213393e-05 14.7007856369019\\
3.36662233166862e-05 14.7007865905762\\
5.05003372381907e-05 14.7007884979248\\
7.57515081204474e-05 14.7007904052734\\
0.000113628258986864 14.7007932662964\\
0.000170443381648511 14.700795173645\\
0.000255666091106832 14.7007970809937\\
0.000383500126190484 14.7007970809937\\
0.000575251178815961 14.7007942199707\\
0.000862877757754177 14.7007904052734\\
0.00129431765526533 14.7007865905762\\
0.00194147753063589 14.7007837295532\\
0.00291221728548408 14.700779914856\\
0.00436832662671804 14.7007780075073\\
0.00597004732117057 14.700779914856\\
0.00773194013163447 14.7007827758789\\
0.009670021943748 14.7007837295532\\
0.0118019115179777 14.7007837295532\\
0.0141469910740852 14.7007827758789\\
0.0167265776544809 14.7007827758789\\
0.0195641238242388 14.7007827758789\\
0.0226854234933853 14.7007827758789\\
0.0261188540607691 14.7007827758789\\
0.0298956278711557 14.7007827758789\\
0.0340500771999359 14.7007827758789\\
0.0386199727654457 14.7007827758789\\
0.0436468608677387 14.7007827758789\\
0.0491764321923256 14.7007827758789\\
0.0552589632570744 14.7007827758789\\
0.0619497485458851 14.7007827758789\\
0.0693096145987511 14.7007827758789\\
0.0774054601788521 14.7007827758789\\
0.0863108932971954 14.700779914856\\
0.0961068719625473 14.7007827758789\\
0.106882445514202 14.7007827758789\\
0.118735581636429 14.7007827758789\\
0.13177402317524 14.7007827758789\\
0.146116316318512 14.7007827758789\\
0.161892831325531 14.7007827758789\\
0.179247006773949 14.7007827758789\\
0.198336601257324 14.7007827758789\\
0.21933513879776 14.7007827758789\\
0.242433547973633 14.7007827758789\\
0.267841786146164 14.7007827758789\\
0.29579085111618 14.7007827758789\\
0.32653483748436 14.7007827758789\\
0.360353201627731 14.7007827758789\\
0.397553414106369 14.7007837295532\\
0.438473641872406 14.7007837295532\\
0.483485877513886 14.7007837295532\\
0.532999336719513 14.7007837295532\\
0.582999348640442 14.7007837295532\\
0.632999360561371 14.7007837295532\\
0.6829993724823 14.7007837295532\\
0.732999384403229 14.7007837295532\\
0.782999336719513 14.7007837295532\\
0.832999348640442 14.7007837295532\\
0.882999360561371 14.7007837295532\\
0.9329993724823 14.7007837295532\\
0.982999384403229 14.7007837295532\\
1.03299939632416 14.7007827758789\\
1.08299934864044 14.7007827758789\\
1.13299930095673 14.7007827758789\\
1.1829993724823 14.7007827758789\\
1.23299932479858 14.7007827758789\\
1.28299939632416 14.7007827758789\\
1.33299934864044 14.7007827758789\\
1.38299930095673 14.7007827758789\\
1.4329993724823 14.7007827758789\\
1.48299932479858 14.7007827758789\\
1.53299939632416 14.7007827758789\\
1.58299934864044 14.7007827758789\\
1.63299930095673 14.7007827758789\\
1.6829993724823 14.7007827758789\\
1.73299932479858 14.7007827758789\\
1.78299939632416 14.7007827758789\\
1.83299934864044 14.7007827758789\\
1.88299930095673 14.7007827758789\\
1.9329993724823 14.7007827758789\\
1.98299932479858 14.7007827758789\\
2.03299927711487 14.7007827758789\\
2.08299946784973 14.7007827758789\\
2.13299942016602 14.7007827758789\\
2.1829993724823 14.7007827758789\\
2.23299932479858 14.7007827758789\\
2.28299927711487 14.7007827758789\\
2.33299946784973 14.7007827758789\\
2.38299942016602 14.7007827758789\\
2.4329993724823 14.7007827758789\\
2.48299932479858 14.7007827758789\\
2.53299927711487 14.7007827758789\\
2.58299946784973 14.7007827758789\\
2.63299942016602 14.7007827758789\\
2.6829993724823 14.7007827758789\\
2.73299932479858 14.7007827758789\\
2.78299927711487 14.7007827758789\\
2.83299946784973 14.7007827758789\\
2.88299942016602 14.7007827758789\\
2.9329993724823 14.7007827758789\\
2.98299932479858 14.7007827758789\\
3.03299927711487 14.7007827758789\\
3.08299946784973 14.7007827758789\\
3.13299942016602 14.7007827758789\\
3.1829993724823 14.7007827758789\\
3.23299932479858 14.7007827758789\\
3.28299927711487 14.7007827758789\\
3.33299946784973 14.7007827758789\\
3.38299942016602 14.7007827758789\\
3.4329993724823 14.7007827758789\\
3.48299932479858 14.7007827758789\\
3.53299927711487 14.7007827758789\\
3.58299946784973 14.7007827758789\\
3.63299942016602 14.7007827758789\\
3.6829993724823 14.7007827758789\\
3.73299932479858 14.7007827758789\\
3.78299927711487 14.7007827758789\\
3.83299946784973 14.7007827758789\\
3.88299942016602 14.7007827758789\\
3.9329993724823 14.7007827758789\\
3.98299932479858 14.7007827758789\\
4.03299951553345 14.7007827758789\\
4.08299922943115 14.7007827758789\\
4.13299942016602 14.7007827758789\\
4.18299913406372 14.7007827758789\\
4.23299932479858 14.7007827758789\\
4.28299951553345 14.7007827758789\\
4.33299922943115 14.7007827758789\\
4.38299942016602 14.7007827758789\\
4.43299913406372 14.7007827758789\\
4.48299932479858 14.7007827758789\\
4.53299951553345 14.7007827758789\\
4.58299922943115 14.7007827758789\\
4.63299942016602 14.7007827758789\\
4.68299913406372 14.7007827758789\\
4.73299932479858 14.7007827758789\\
4.78299951553345 14.7007827758789\\
4.83299922943115 14.7007827758789\\
4.88299942016602 14.7007827758789\\
4.93299913406372 14.7007827758789\\
4.98299932479858 14.7007827758789\\
5.03299951553345 14.7007827758789\\
5.08299922943115 14.7007827758789\\
5.13299942016602 14.7007827758789\\
5.18299913406372 14.7007827758789\\
5.23299932479858 14.7007827758789\\
5.28299951553345 14.7007827758789\\
5.33299922943115 14.7007827758789\\
5.38299942016602 14.7007827758789\\
5.43299913406372 14.7007827758789\\
5.48299932479858 14.7007837295532\\
5.53299951553345 14.7007837295532\\
5.58299922943115 14.7007837295532\\
5.63299942016602 14.7007856369019\\
5.68299913406372 14.7007865905762\\
5.73299932479858 14.7007913589478\\
5.78299951553345 14.7008008956909\\
5.83299922943115 14.7008218765259\\
5.88299942016602 14.7008686065674\\
5.93299913406372 14.7009773254395\\
5.98299932479858 14.7012338638306\\
6.03299951553345 14.7018346786499\\
6.08299922943115 14.7031946182251\\
6.13299942016602 14.7060613632202\\
6.18299913406372 14.7113580703735\\
6.23299932479858 14.719313621521\\
6.28299951553345 14.7358856201172\\
6.33299922943115 14.7318172454834\\
6.38299942016602 14.7319145202637\\
6.43299913406372 14.732569694519\\
6.48299932479858 14.732780456543\\
6.53299951553345 14.7329292297363\\
6.58299922943115 14.7330045700073\\
6.63299942016602 14.7329235076904\\
6.68299913406372 14.7327470779419\\
6.73299932479858 14.7329006195068\\
6.78299951553345 14.7331695556641\\
6.83299922943115 14.7334308624268\\
6.88299942016602 14.7336978912354\\
6.93299913406372 14.7339630126953\\
6.98299932479858 14.7342290878296\\
7.03299951553345 14.7344942092896\\
7.08299922943115 14.7347593307495\\
7.13299942016602 14.7350244522095\\
7.18299913406372 14.7352905273438\\
7.23299932479858 14.7355556488037\\
7.28299951553345 14.7358207702637\\
7.33299922943115 14.7360868453979\\
7.38299942016602 14.7363519668579\\
7.43299913406372 14.7366180419922\\
7.48299932479858 14.7368831634521\\
7.53299951553345 14.7371482849121\\
7.58299922943115 14.7374143600464\\
7.63299942016602 14.7376794815063\\
7.68299913406372 14.7379455566406\\
7.73299932479858 14.7382106781006\\
7.78299951553345 14.7384767532349\\
7.83299922943115 14.7387418746948\\
7.88299942016602 14.7390079498291\\
7.93299913406372 14.7392730712891\\
7.98299932479858 14.7395391464233\\
8.03299903869629 14.7398052215576\\
8.08299922943115 14.7400703430176\\
8.13299942016602 14.7403364181519\\
8.18299961090088 14.7406024932861\\
8.23299980163574 14.7408676147461\\
8.28299903869629 14.7411336898804\\
8.33299922943115 14.7413997650146\\
8.38299942016602 14.7416658401489\\
8.43299961090088 14.7419319152832\\
8.48299980163574 14.7421979904175\\
8.53299903869629 14.7424640655518\\
8.58299922943115 14.742730140686\\
8.63299942016602 14.7429962158203\\
8.68299961090088 14.7432622909546\\
8.73299980163574 14.7435283660889\\
8.78299903869629 14.7437944412231\\
8.83299922943115 14.7442302703857\\
8.88299942016602 14.7446527481079\\
8.93299961090088 14.7449893951416\\
8.98299980163574 14.7453165054321\\
9.03299903869629 14.7456398010254\\
9.08299922943115 14.7459630966187\\
9.13299942016602 14.7462854385376\\
9.18299961090088 14.7466077804565\\
9.23299980163574 14.7469291687012\\
9.28299903869629 14.7472505569458\\
9.33299922943115 14.7475709915161\\
9.38299942016602 14.7478914260864\\
9.43299961090088 14.7482118606567\\
9.48299980163574 14.7485313415527\\
9.53299903869629 14.748851776123\\
9.58299922943115 14.749171257019\\
9.63299942016602 14.7494897842407\\
9.68299961090088 14.7498092651367\\
9.73299980163574 14.7501277923584\\
9.78299903869629 14.7504472732544\\
9.83299922943115 14.7507658004761\\
9.88299942016602 14.7510843276978\\
9.93299961090088 14.7514019012451\\
9.98299980163574 14.7517204284668\\
10.0329990386963 14.7520389556885\\
10.0829992294312 14.7523565292358\\
10.132999420166 14.7526750564575\\
10.1829996109009 14.7529935836792\\
10.2329998016357 14.7533111572266\\
10.2829990386963 14.7536182403564\\
10.3329992294312 14.7538833618164\\
10.382999420166 14.7539463043213\\
10.4329996109009 14.7542095184326\\
10.4829998016357 14.754469871521\\
10.5329990386963 14.7547359466553\\
10.5829992294312 14.7549991607666\\
10.632999420166 14.7552623748779\\
10.6829996109009 14.7555236816406\\
10.7329998016357 14.7557849884033\\
10.7829990386963 14.7560443878174\\
10.8329992294312 14.7563028335571\\
10.882999420166 14.7565603256226\\
10.9329996109009 14.7568159103394\\
10.9829998016357 14.7570695877075\\
11.0329990386963 14.7573223114014\\
11.0829992294312 14.7575740814209\\
11.132999420166 14.7578229904175\\
11.1829996109009 14.7580699920654\\
11.2329998016357 14.7583150863647\\
11.2829990386963 14.7585582733154\\
11.3329992294312 14.7587985992432\\
11.382999420166 14.7590579986572\\
11.4329996109009 14.7593879699707\\
11.4829998016357 14.7596836090088\\
11.5329990386963 14.7599534988403\\
11.5829992294312 14.7604160308838\\
11.632999420166 14.7607583999634\\
11.6829996109009 14.7610578536987\\
11.7329998016357 14.7613754272461\\
11.7829990386963 14.7616958618164\\
11.8329992294312 14.7620239257813\\
11.882999420166 14.7623605728149\\
11.9329996109009 14.7627077102661\\
11.9829998016357 14.7630672454834\\
12.0329990386963 14.7634429931641\\
12.0829992294312 14.7638387680054\\
12.132999420166 14.7642621994019\\
12.1829996109009 14.7651968002319\\
12.2329998016357 14.770655632019\\
12.2829990386963 14.772385597229\\
12.3329992294312 14.7741975784302\\
12.382999420166 14.7773275375366\\
12.4329996109009 14.7730026245117\\
12.4829998016357 14.7684373855591\\
12.5329990386963 14.7686653137207\\
12.5829992294312 14.7689294815063\\
12.632999420166 14.7692222595215\\
12.6829996109009 14.7695665359497\\
12.7329998016357 14.7699575424194\\
12.7829990386963 14.7703914642334\\
12.8329992294312 14.7708559036255\\
12.882999420166 14.7713422775269\\
12.9329996109009 14.771842956543\\
12.9829998016357 14.7723531723022\\
13.0329990386963 14.7728719711304\\
13.0829992294312 14.7733955383301\\
13.132999420166 14.7739248275757\\
13.1829996109009 14.7744579315186\\
13.2329998016357 14.7749967575073\\
13.2829990386963 14.7755346298218\\
13.3329992294312 14.7760763168335\\
13.382999420166 14.7766199111938\\
13.4329996109009 14.7771663665771\\
13.4829998016357 14.7777156829834\\
13.5329990386963 14.7782669067383\\
13.5829992294312 14.7788190841675\\
13.632999420166 14.779372215271\\
13.6829996109009 14.7799263000488\\
13.7329998016357 14.7804803848267\\
13.7829990386963 14.7810344696045\\
13.8329992294312 14.7815895080566\\
13.882999420166 14.7821435928345\\
13.9329996109009 14.7826976776123\\
13.9829998016357 14.7832527160645\\
14.0329990386963 14.7838068008423\\
14.0829992294312 14.7843618392944\\
14.132999420166 14.7849168777466\\
14.1829996109009 14.7854709625244\\
14.2329998016357 14.7860260009766\\
14.2829990386963 14.7865810394287\\
14.3329992294312 14.7871360778809\\
14.382999420166 14.787691116333\\
14.4329996109009 14.7882461547852\\
14.4829998016357 14.7888011932373\\
14.5329990386963 14.7893571853638\\
14.5829992294312 14.7899122238159\\
14.632999420166 14.7904682159424\\
14.6829996109009 14.7910232543945\\
14.7329998016357 14.791579246521\\
14.7829990386963 14.7921342849731\\
14.8329992294312 14.7926902770996\\
14.882999420166 14.7932462692261\\
14.9329996109009 14.7938022613525\\
14.9829998016357 14.794358253479\\
15.0329990386963 14.7949151992798\\
15.0829992294312 14.7954711914063\\
15.132999420166 14.7960271835327\\
15.1829996109009 14.7965841293335\\
15.2329998016357 14.79714012146\\
15.2829990386963 14.7976970672607\\
15.3329992294312 14.7982654571533\\
15.382999420166 14.7988080978394\\
15.4329996109009 14.7993659973145\\
15.4829998016357 14.7999248504639\\
15.5329990386963 14.8004846572876\\
15.5829992294312 14.801043510437\\
15.632999420166 14.8016033172607\\
15.6829996109009 14.8021631240845\\
15.7329998016357 14.8027229309082\\
15.7829990386963 14.8032817840576\\
15.8329992294312 14.8038415908813\\
15.882999420166 14.8044013977051\\
15.9329996109009 14.8049612045288\\
15.9829998016357 14.8055210113525\\
16.0329990386963 14.8060808181763\\
16.0830001831055 14.806640625\\
16.132999420166 14.8072004318237\\
16.1829986572266 14.8077611923218\\
16.2329998016357 14.8083209991455\\
16.2829990386963 14.8088808059692\\
16.3330001831055 14.809440612793\\
16.382999420166 14.810001373291\\
16.4329986572266 14.8105611801147\\
16.4829998016357 14.8111219406128\\
16.5329990386963 14.8116817474365\\
16.5830001831055 14.8122425079346\\
16.632999420166 14.8128023147583\\
16.6829986572266 14.8133630752563\\
16.7329998016357 14.8139228820801\\
16.7829990386963 14.8144836425781\\
16.8330001831055 14.8150444030762\\
16.882999420166 14.8156051635742\\
16.9329986572266 14.8161659240723\\
16.9829998016357 14.816725730896\\
17.0329990386963 14.817286491394\\
17.0830001831055 14.8178472518921\\
17.132999420166 14.8184080123901\\
17.1829986572266 14.8189687728882\\
17.2329998016357 14.8195304870605\\
17.2829990386963 14.8200912475586\\
17.3330001831055 14.8206520080566\\
17.382999420166 14.8212127685547\\
17.4329986572266 14.8217744827271\\
17.4829998016357 14.8223352432251\\
17.5329990386963 14.8228960037231\\
17.5830001831055 14.8234577178955\\
17.632999420166 14.8240184783936\\
17.6829986572266 14.8245801925659\\
17.7329998016357 14.825140953064\\
17.7829990386963 14.8257026672363\\
17.8330001831055 14.8262643814087\\
17.882999420166 14.8268270492554\\
17.9329986572266 14.8273887634277\\
17.9829998016357 14.8279552459717\\
18.0329990386963 14.8285255432129\\
18.0830001831055 14.8291053771973\\
18.132999420166 14.8297109603882\\
18.1829986572266 14.830376625061\\
18.2329998016357 14.8311824798584\\
18.2829990386963 14.8323011398315\\
18.3330001831055 14.8340578079224\\
18.382999420166 14.836895942688\\
18.4329986572266 14.8409700393677\\
18.4829998016357 14.8454065322876\\
18.5329990386963 14.8538360595703\\
18.5830001831055 14.841178894043\\
18.632999420166 14.8400831222534\\
18.6829986572266 14.8397035598755\\
18.7329998016357 14.8393955230713\\
18.7829990386963 14.8392553329468\\
18.8330001831055 14.8391199111938\\
18.882999420166 14.8394784927368\\
18.9329986572266 14.8391151428223\\
18.9829998016357 14.8396854400635\\
19.0329990386963 14.8404207229614\\
19.0830001831055 14.8410863876343\\
19.132999420166 14.8414373397827\\
19.1829986572266 14.8420810699463\\
19.2329998016357 14.842604637146\\
19.2829990386963 14.8431854248047\\
19.3330001831055 14.8437395095825\\
19.382999420166 14.8443059921265\\
19.4329986572266 14.8448667526245\\
19.4829998016357 14.8454313278198\\
19.5329990386963 14.8459930419922\\
19.5830001831055 14.8465566635132\\
19.632999420166 14.8471193313599\\
19.6829986572266 14.8476819992065\\
19.7329998016357 14.8482446670532\\
19.7829990386963 14.8488082885742\\
19.8330001831055 14.8493709564209\\
19.882999420166 14.8499336242676\\
19.9329986572266 14.8504972457886\\
19.9829998016357 14.8510599136353\\
20.0329990386963 14.8516235351563\\
20.0830001831055 14.8521862030029\\
20.132999420166 14.8527498245239\\
20.1829986572266 14.8533124923706\\
20.2329998016357 14.8538761138916\\
20.2829990386963 14.8544387817383\\
20.3330001831055 14.8550024032593\\
20.382999420166 14.8555660247803\\
20.4329986572266 14.8561296463013\\
20.4829998016357 14.8566923141479\\
20.5329990386963 14.8572559356689\\
20.5830001831055 14.8578195571899\\
20.632999420166 14.8583831787109\\
20.6829986572266 14.8589468002319\\
20.7329998016357 14.8595104217529\\
20.7829990386963 14.8600740432739\\
20.8330001831055 14.8606376647949\\
20.882999420166 14.8612012863159\\
20.9329986572266 14.8617649078369\\
20.9829998016357 14.8623285293579\\
21.0329990386963 14.8628921508789\\
21.0830001831055 14.8636484146118\\
21.132999420166 14.8643102645874\\
21.1829986572266 14.8648853302002\\
21.2329998016357 14.8654527664185\\
21.2829990386963 14.8660163879395\\
21.3330001831055 14.8665809631348\\
21.382999420166 14.8671455383301\\
21.4329986572266 14.8677091598511\\
21.4829998016357 14.8682737350464\\
21.5329990386963 14.8688383102417\\
21.5830001831055 14.869402885437\\
21.632999420166 14.8699674606323\\
21.6829986572266 14.8705320358276\\
21.7329998016357 14.8710966110229\\
21.7829990386963 14.8716621398926\\
21.8330001831055 14.8722267150879\\
21.882999420166 14.8727922439575\\
21.9329986572266 14.8733577728271\\
21.9829998016357 14.8739223480225\\
22.0329990386963 14.8744878768921\\
22.0830001831055 14.875054359436\\
22.132999420166 14.8756198883057\\
22.1829986572266 14.8761854171753\\
22.2329998016357 14.8767518997192\\
22.2829990386963 14.8773174285889\\
22.3330001831055 14.8778839111328\\
22.382999420166 14.8784503936768\\
22.4329986572266 14.8790168762207\\
22.4829998016357 14.8795833587646\\
22.5329990386963 14.8801431655884\\
22.5830001831055 14.8806686401367\\
22.632999420166 14.8809804916382\\
22.6829986572266 14.8815402984619\\
22.7329998016357 14.8820934295654\\
22.7829990386963 14.8826551437378\\
22.8330001831055 14.8832149505615\\
22.882999420166 14.8837747573853\\
22.9329986572266 14.8843336105347\\
22.9829998016357 14.8848934173584\\
23.0329990386963 14.8854522705078\\
23.0830001831055 14.8860111236572\\
23.132999420166 14.8865690231323\\
23.1829986572266 14.8871269226074\\
23.2329998016357 14.8876848220825\\
23.2829990386963 14.8882427215576\\
23.3330001831055 14.8887996673584\\
23.382999420166 14.8893566131592\\
23.4329986572266 14.8899126052856\\
23.4829998016357 14.8904685974121\\
23.5329990386963 14.8910245895386\\
23.5830001831055 14.8915796279907\\
23.632999420166 14.8921527862549\\
23.6829986572266 14.8927812576294\\
23.7329998016357 14.8933725357056\\
23.7829990386963 14.8939409255981\\
23.8330001831055 14.8946504592896\\
23.882999420166 14.8952522277832\\
23.9329986572266 14.8958168029785\\
23.9829998016357 14.8963966369629\\
24.0329990386963 14.8969774246216\\
24.0830001831055 14.8975639343262\\
24.132999420166 14.8981552124023\\
24.1829986572266 14.8987531661987\\
24.2329998016357 14.8993587493896\\
24.2829990386963 14.899974822998\\
24.3330001831055 14.9006052017212\\
24.382999420166 14.901252746582\\
24.4329986572266 14.9022483825684\\
24.4829998016357 14.9080715179443\\
24.5329990386963 14.909327507019\\
24.5830001831055 14.9107666015625\\
24.632999420166 14.9123458862305\\
24.6829986572266 14.9096565246582\\
24.7329998016357 14.9049825668335\\
24.7829990386963 14.9054279327393\\
24.8330001831055 14.9057788848877\\
24.882999420166 14.9061832427979\\
24.9329986572266 14.9066190719604\\
24.9829998016357 14.9070892333984\\
25.0329990386963 14.9075908660889\\
25.0830001831055 14.9081144332886\\
25.132999420166 14.9086513519287\\
25.1829986572266 14.9091949462891\\
25.2329998016357 14.909743309021\\
25.2829990386963 14.9102945327759\\
25.3330001831055 14.9108476638794\\
25.382999420166 14.9114017486572\\
25.4329986572266 14.9119567871094\\
25.4829998016357 14.9125137329102\\
25.5329990386963 14.9130725860596\\
25.5830001831055 14.9136304855347\\
25.632999420166 14.9141874313354\\
25.6829986572266 14.9147453308105\\
25.7329998016357 14.91530418396\\
25.7829990386963 14.9158639907837\\
25.8330001831055 14.9164247512817\\
25.882999420166 14.9169855117798\\
25.9329986572266 14.9175481796265\\
25.9829998016357 14.9181108474731\\
26.0329990386963 14.9186744689941\\
26.0830001831055 14.9192390441895\\
26.132999420166 14.9198026657104\\
26.1829986572266 14.9203662872314\\
26.2329998016357 14.9209299087524\\
26.2829990386963 14.9214944839478\\
26.3330001831055 14.9220581054688\\
26.382999420166 14.9226226806641\\
26.4329986572266 14.9231863021851\\
26.4829998016357 14.9237499237061\\
26.5329990386963 14.9243144989014\\
26.5830001831055 14.9248781204224\\
26.632999420166 14.9254417419434\\
26.6829986572266 14.9260063171387\\
26.7329998016357 14.9265699386597\\
26.7829990386963 14.9271335601807\\
26.8330001831055 14.927698135376\\
26.882999420166 14.928261756897\\
26.9329986572266 14.928825378418\\
26.9829998016357 14.9293899536133\\
27.0329990386963 14.9299535751343\\
27.0830001831055 14.9305171966553\\
27.132999420166 14.9310817718506\\
27.1829986572266 14.9316453933716\\
27.2329998016357 14.9322090148926\\
27.2829990386963 14.9327735900879\\
27.3330001831055 14.9333372116089\\
27.382999420166 14.9339008331299\\
27.4329986572266 14.9344654083252\\
27.4829998016357 14.9350290298462\\
27.5329990386963 14.9355926513672\\
27.5830001831055 14.9361572265625\\
27.632999420166 14.9367208480835\\
27.6829986572266 14.9372844696045\\
27.7329998016357 14.9378490447998\\
27.7829990386963 14.9384126663208\\
27.8330001831055 14.9389762878418\\
27.882999420166 14.9395399093628\\
27.9329986572266 14.9401044845581\\
27.9829998016357 14.9406681060791\\
28.0329990386963 14.9412317276001\\
28.0830001831055 14.9417963027954\\
28.132999420166 14.9423599243164\\
28.1829986572266 14.9429378509521\\
28.2329998016357 14.9434814453125\\
28.2829990386963 14.9440441131592\\
28.3330001831055 14.9446077346802\\
28.382999420166 14.9451723098755\\
28.4329986572266 14.9457359313965\\
28.4829998016357 14.9462995529175\\
28.5329990386963 14.9468641281128\\
28.5830001831055 14.9474277496338\\
28.632999420166 14.9479923248291\\
28.6829986572266 14.9485559463501\\
28.7329998016357 14.9491195678711\\
28.7829990386963 14.9496841430664\\
28.8330001831055 14.9502477645874\\
28.882999420166 14.9508123397827\\
28.9329986572266 14.9513759613037\\
28.9829998016357 14.9519395828247\\
29.0329990386963 14.95250415802\\
29.0830001831055 14.953067779541\\
29.132999420166 14.9536323547363\\
29.1829986572266 14.9541959762573\\
29.2329998016357 14.9547595977783\\
29.2829990386963 14.9553241729736\\
29.3330001831055 14.9558877944946\\
29.382999420166 14.9564523696899\\
29.4329986572266 14.9570159912109\\
29.4829998016357 14.9575796127319\\
29.5329990386963 14.9581441879272\\
29.5830001831055 14.9587078094482\\
29.632999420166 14.9592723846436\\
29.6829986572266 14.9598360061646\\
29.7329998016357 14.9604005813599\\
29.7829990386963 14.9609642028809\\
29.8330001831055 14.9615278244019\\
29.882999420166 14.9620923995972\\
29.9329986572266 14.9626560211182\\
29.9829998016357 14.9632205963135\\
30.0329990386963 14.9637851715088\\
30.0830001831055 14.9643497467041\\
30.132999420166 14.9649152755737\\
30.1829986572266 14.965479850769\\
30.2329998016357 14.9660501480103\\
30.2829990386963 14.9666261672974\\
30.3330001831055 14.9672174453735\\
30.382999420166 14.9678449630737\\
30.4329986572266 14.9685583114624\\
30.4829998016357 14.9694700241089\\
30.5329990386963 14.9708194732666\\
30.5830001831055 14.9729900360107\\
30.632999420166 14.9763927459717\\
30.6829986572266 14.9808139801025\\
30.7329998016357 14.9848794937134\\
30.7829990386963 14.9890556335449\\
30.8330001831055 14.9784059524536\\
30.882999420166 14.9777975082397\\
30.9329986572266 14.9774398803711\\
30.9829998016357 14.977201461792\\
31.0329990386963 14.9767684936523\\
31.0830001831055 14.9774942398071\\
31.132999420166 14.9769878387451\\
31.1829986572266 14.9775276184082\\
31.2329998016357 14.9778909683228\\
31.2829990386963 14.9785318374634\\
31.3330001831055 14.9790601730347\\
31.382999420166 14.9796419143677\\
31.4329986572266 14.9801979064941\\
31.4829998016357 14.9807653427124\\
31.5329990386963 14.9813280105591\\
31.5830001831055 14.9818935394287\\
31.632999420166 14.9824571609497\\
31.6829986572266 14.983021736145\\
31.7329998016357 14.983585357666\\
31.7829990386963 14.9841499328613\\
31.8330001831055 14.9847145080566\\
31.882999420166 14.9852781295776\\
31.9329986572266 14.9858427047729\\
31.9829998016357 14.9864072799683\\
32.0330009460449 14.9869709014893\\
32.0830001831055 14.9875354766846\\
32.132999420166 14.9881000518799\\
32.1829986572266 14.9886646270752\\
32.2329978942871 14.9892292022705\\
32.2830009460449 14.9897928237915\\
32.3330001831055 14.9903573989868\\
32.382999420166 14.9909219741821\\
32.4329986572266 14.9914855957031\\
32.4829978942871 14.9920501708984\\
32.5330009460449 14.9926147460938\\
32.5830001831055 14.9931793212891\\
32.632999420166 14.9937438964844\\
32.6829986572266 14.9943084716797\\
32.7329978942871 14.994873046875\\
32.7830009460449 14.9954376220703\\
32.8330001831055 14.9960021972656\\
32.882999420166 14.9965667724609\\
32.9329986572266 14.9971313476563\\
32.9829978942871 14.9976959228516\\
33.0330009460449 14.9982604980469\\
33.0830001831055 14.9988250732422\\
33.132999420166 14.9993896484375\\
33.1829986572266 14.9999542236328\\
33.2329978942871 15.0005187988281\\
33.2830009460449 15.0011043548584\\
33.3330001831055 15.001932144165\\
33.382999420166 15.0025043487549\\
33.4329986572266 15.0030794143677\\
33.4829978942871 15.0036430358887\\
33.5330009460449 15.0042085647583\\
33.5830001831055 15.0047731399536\\
33.632999420166 15.0053386688232\\
33.6829986572266 15.0059032440186\\
33.7329978942871 15.0064687728882\\
33.7830009460449 15.0070333480835\\
33.8330001831055 15.0075988769531\\
33.882999420166 15.0081644058228\\
33.9329986572266 15.0087299346924\\
33.9829978942871 15.009295463562\\
34.0330009460449 15.0098609924316\\
34.0830001831055 15.0104265213013\\
34.132999420166 15.0109930038452\\
34.1829986572266 15.0115585327148\\
34.2329978942871 15.0121250152588\\
34.2830009460449 15.0126905441284\\
34.3330001831055 15.0132570266724\\
34.382999420166 15.0138235092163\\
34.4329986572266 15.0143899917603\\
34.4829978942871 15.0149564743042\\
34.5330009460449 15.0155229568481\\
34.5830001831055 15.0160894393921\\
34.632999420166 15.0166568756104\\
34.6829986572266 15.0172233581543\\
34.7329978942871 15.0177907943726\\
34.7830009460449 15.0183401107788\\
34.8330001831055 15.0188236236572\\
34.882999420166 15.0191688537598\\
34.9329986572266 15.01975440979\\
34.9829978942871 15.0203056335449\\
35.0330009460449 15.0208692550659\\
35.0830001831055 15.0214281082153\\
35.132999420166 15.0219888687134\\
35.1829986572266 15.0225486755371\\
35.2329978942871 15.0231084823608\\
35.2830009460449 15.0236682891846\\
35.3330001831055 15.0242280960083\\
35.382999420166 15.0247869491577\\
35.4329986572266 15.0253458023071\\
35.4829978942871 15.0259046554565\\
35.5330009460449 15.0264625549316\\
35.5830001831055 15.0270204544067\\
35.632999420166 15.0275783538818\\
35.6829986572266 15.0281352996826\\
35.7329978942871 15.0286922454834\\
35.7830009460449 15.0292491912842\\
35.8330001831055 15.0298051834106\\
35.882999420166 15.0303964614868\\
35.9329986572266 15.0310096740723\\
35.9829978942871 15.0315914154053\\
36.0330009460449 15.0321598052979\\
36.0830001831055 15.0328788757324\\
36.132999420166 15.0334663391113\\
36.1829986572266 15.0340223312378\\
36.2329978942871 15.0346012115479\\
36.2830009460449 15.0351829528809\\
36.3330001831055 15.0357704162598\\
36.382999420166 15.0363626480103\\
36.4329986572266 15.036961555481\\
36.4829978942871 15.0375690460205\\
36.5330009460449 15.0381870269775\\
36.5830001831055 15.0388202667236\\
36.632999420166 15.0394716262817\\
36.6829986572266 15.0416326522827\\
36.7329978942871 15.0463647842407\\
36.7830009460449 15.0476837158203\\
36.8330001831055 15.0491361618042\\
36.882999420166 15.0507764816284\\
36.9329986572266 15.0458250045776\\
36.9829978942871 15.0431499481201\\
37.0330009460449 15.0435962677002\\
37.0830001831055 15.0439472198486\\
37.132999420166 15.0443706512451\\
37.1829986572266 15.0448122024536\\
37.2329978942871 15.0452919006348\\
37.2830009460449 15.0457992553711\\
37.3330001831055 15.0463266372681\\
37.382999420166 15.0468654632568\\
37.4329986572266 15.0474109649658\\
37.4829978942871 15.0479602813721\\
37.5330009460449 15.048511505127\\
37.5830001831055 15.0490646362305\\
37.632999420166 15.0496187210083\\
37.6829986572266 15.0501747131348\\
37.7329978942871 15.0507316589355\\
37.7830009460449 15.0512914657593\\
37.8330001831055 15.0518493652344\\
37.882999420166 15.0524063110352\\
37.9329986572266 15.0529651641846\\
37.9829978942871 15.053524017334\\
38.0330009460449 15.0540838241577\\
38.0830001831055 15.0546445846558\\
38.132999420166 15.0552053451538\\
38.1829986572266 15.0557680130005\\
38.2329978942871 15.0563306808472\\
38.2830009460449 15.0568943023682\\
38.3330001831055 15.0574579238892\\
38.382999420166 15.0580224990845\\
38.4329986572266 15.0585861206055\\
38.4829978942871 15.0591506958008\\
38.5330009460449 15.0597143173218\\
38.5830001831055 15.0602788925171\\
38.632999420166 15.0608425140381\\
38.6829986572266 15.0614070892334\\
38.7329978942871 15.0619707107544\\
38.7830009460449 15.0625352859497\\
38.8330001831055 15.063099861145\\
38.882999420166 15.063663482666\\
38.9329986572266 15.0642280578613\\
38.9829978942871 15.0647916793823\\
39.0330009460449 15.0653562545776\\
39.0830001831055 15.0659198760986\\
39.132999420166 15.0664844512939\\
39.1829986572266 15.0670480728149\\
39.2329978942871 15.0676126480103\\
39.2830009460449 15.0681762695313\\
39.3330001831055 15.0687408447266\\
39.382999420166 15.0693044662476\\
39.4329986572266 15.0698680877686\\
39.4829978942871 15.0704326629639\\
39.5330009460449 15.0709962844849\\
39.5830001831055 15.0715608596802\\
39.632999420166 15.0721244812012\\
39.6829986572266 15.0726890563965\\
39.7329978942871 15.0732526779175\\
39.7830009460449 15.0738172531128\\
39.8330001831055 15.0743808746338\\
39.882999420166 15.0749454498291\\
39.9329986572266 15.0755090713501\\
39.9829978942871 15.0760736465454\\
40.0330009460449 15.0766372680664\\
40.0830001831055 15.0772008895874\\
40.132999420166 15.0777654647827\\
40.1829986572266 15.0783290863037\\
40.2329978942871 15.078893661499\\
40.2830009460449 15.07945728302\\
40.3330001831055 15.0800218582153\\
40.382999420166 15.0805854797363\\
40.4329986572266 15.0811500549316\\
40.4829978942871 15.0817136764526\\
40.5330009460449 15.0822782516479\\
40.5830001831055 15.0828418731689\\
40.632999420166 15.0834054946899\\
40.6829986572266 15.0839700698853\\
40.7329978942871 15.0845336914063\\
40.7830009460449 15.085111618042\\
40.8330001831055 15.0856561660767\\
40.882999420166 15.0862188339233\\
40.9329986572266 15.0867824554443\\
40.9829978942871 15.0873470306396\\
41.0330009460449 15.0879106521606\\
41.0830001831055 15.088475227356\\
41.132999420166 15.089038848877\\
41.1829986572266 15.0896034240723\\
41.2329978942871 15.0901670455933\\
41.2830009460449 15.0907316207886\\
41.3330001831055 15.0912952423096\\
41.382999420166 15.0918598175049\\
41.4329986572266 15.0924243927002\\
41.4829978942871 15.0929880142212\\
41.5330009460449 15.0935525894165\\
41.5830001831055 15.0941162109375\\
41.632999420166 15.0946807861328\\
41.6829986572266 15.0952444076538\\
41.7329978942871 15.0958089828491\\
41.7830009460449 15.0963726043701\\
41.8330001831055 15.0969371795654\\
41.882999420166 15.0975008010864\\
41.9329986572266 15.0980653762817\\
41.9829978942871 15.0986289978027\\
42.0330009460449 15.099193572998\\
42.0830001831055 15.099757194519\\
42.132999420166 15.1003217697144\\
42.1829986572266 15.1008863449097\\
42.2329978942871 15.1014499664307\\
42.2830009460449 15.102014541626\\
42.3330001831055 15.1025791168213\\
42.382999420166 15.1031446456909\\
42.4329986572266 15.1037101745605\\
42.4829978942871 15.1042814254761\\
42.5330009460449 15.1048583984375\\
42.5830001831055 15.1054515838623\\
42.632999420166 15.1060857772827\\
42.6829986572266 15.1068124771118\\
42.7329978942871 15.1077547073364\\
42.7830009460449 15.1091680526733\\
42.8330001831055 15.1114482879639\\
42.882999420166 15.1149787902832\\
42.9329986572266 15.1194429397583\\
42.9829978942871 15.1233673095703\\
43.0330009460449 15.1265230178833\\
43.0830001831055 15.1164398193359\\
43.132999420166 15.1159381866455\\
43.1829986572266 15.1155824661255\\
43.2329978942871 15.1153631210327\\
43.2830009460449 15.1150007247925\\
43.3330001831055 15.1156139373779\\
43.382999420166 15.1151533126831\\
43.4329986572266 15.1157827377319\\
43.4829978942871 15.116114616394\\
43.5330009460449 15.1167697906494\\
43.5830001831055 15.1172914505005\\
43.632999420166 15.1178760528564\\
43.6829986572266 15.1184310913086\\
43.7329978942871 15.1189994812012\\
43.7830009460449 15.1195611953735\\
43.8330001831055 15.1201267242432\\
43.882999420166 15.1206903457642\\
43.9329986572266 15.1212549209595\\
43.9829978942871 15.1218194961548\\
44.0330009460449 15.1223840713501\\
44.0830001831055 15.1229486465454\\
44.132999420166 15.1235122680664\\
44.1829986572266 15.1240768432617\\
44.2329978942871 15.124641418457\\
44.2830009460449 15.1252059936523\\
44.3330001831055 15.1257705688477\\
44.382999420166 15.126335144043\\
44.4329986572266 15.126898765564\\
44.4829978942871 15.1274633407593\\
44.5330009460449 15.1280279159546\\
44.5830001831055 15.1285924911499\\
44.632999420166 15.1291570663452\\
44.6829986572266 15.1297216415405\\
44.7329978942871 15.1302862167358\\
44.7830009460449 15.1308507919312\\
44.8330001831055 15.1314153671265\\
44.882999420166 15.1319799423218\\
44.9329986572266 15.1325445175171\\
44.9829978942871 15.1331090927124\\
45.0330009460449 15.1336736679077\\
45.0830001831055 15.134238243103\\
45.132999420166 15.1348028182983\\
45.1829986572266 15.1353673934937\\
45.2329978942871 15.135931968689\\
45.2830009460449 15.1364965438843\\
45.3330001831055 15.1370611190796\\
45.382999420166 15.1376256942749\\
45.4329986572266 15.1381912231445\\
45.4829978942871 15.1387557983398\\
45.5330009460449 15.1393613815308\\
45.5830001831055 15.1401691436768\\
45.632999420166 15.1407403945923\\
45.6829986572266 15.1413145065308\\
45.7329978942871 15.1418790817261\\
45.7830009460449 15.1424436569214\\
45.8330001831055 15.143009185791\\
45.882999420166 15.1435737609863\\
45.9329986572266 15.144139289856\\
45.9829978942871 15.1447048187256\\
46.0330009460449 15.1452703475952\\
46.0830001831055 15.1458358764648\\
46.132999420166 15.1464014053345\\
46.1829986572266 15.1469669342041\\
46.2329978942871 15.1475324630737\\
46.2830009460449 15.1480979919434\\
46.3330001831055 15.148663520813\\
46.382999420166 15.1492300033569\\
46.4329986572266 15.1497955322266\\
46.4829978942871 15.1503620147705\\
46.5330009460449 15.1509284973145\\
46.5830001831055 15.1514940261841\\
46.632999420166 15.152060508728\\
46.6829986572266 15.152626991272\\
46.7329978942871 15.1531944274902\\
46.7830009460449 15.1537609100342\\
46.8330001831055 15.1543273925781\\
46.882999420166 15.1548948287964\\
46.9329986572266 15.1554622650146\\
46.9829978942871 15.1560297012329\\
47.0330009460449 15.1565780639648\\
47.0830001831055 15.1570491790771\\
47.132999420166 15.1574115753174\\
47.1829986572266 15.1579942703247\\
47.2329978942871 15.1585474014282\\
47.2830009460449 15.1591100692749\\
47.3330001831055 15.1596698760986\\
47.382999420166 15.1602306365967\\
47.4329986572266 15.1607913970947\\
47.4829978942871 15.1613512039185\\
47.5330009460449 15.1619110107422\\
47.5830001831055 15.1624708175659\\
47.632999420166 15.1630306243896\\
47.6829986572266 15.1635894775391\\
47.7329978942871 15.1641483306885\\
47.7830009460449 15.1647071838379\\
47.8330001831055 15.1652660369873\\
47.882999420166 15.1658239364624\\
47.9329986572266 15.1663818359375\\
47.9829978942871 15.1669387817383\\
48.0330009460449 15.1674957275391\\
48.0830001831055 15.1680526733398\\
48.132999420166 15.1686458587646\\
48.1829986572266 15.1692543029785\\
48.2329978942871 15.1698341369629\\
48.2830009460449 15.1704025268555\\
48.3330001831055 15.1711149215698\\
48.382999420166 15.1716995239258\\
48.4329986572266 15.1722555160522\\
48.4829978942871 15.1728343963623\\
48.5330009460449 15.1734161376953\\
48.5830001831055 15.1740036010742\\
48.632999420166 15.1745958328247\\
48.6829986572266 15.1751947402954\\
48.7329978942871 15.175802230835\\
48.7830009460449 15.176420211792\\
48.8330001831055 15.1770534515381\\
48.882999420166 15.1777048110962\\
48.9329986572266 15.1800498962402\\
48.9829978942871 15.184609413147\\
49.0330009460449 15.1859378814697\\
49.0830001831055 15.1873912811279\\
49.132999420166 15.1890392303467\\
49.1829986572266 15.1836700439453\\
49.2329978942871 15.1813821792603\\
49.2830009460449 15.1818265914917\\
49.3330001831055 15.1821813583374\\
49.382999420166 15.1826047897339\\
49.4329986572266 15.183048248291\\
49.4829978942871 15.1835289001465\\
49.5330009460449 15.1840381622314\\
49.5830001831055 15.1845664978027\\
49.632999420166 15.1851053237915\\
49.6829986572266 15.1856508255005\\
49.7329978942871 15.1862001419067\\
49.7830009460449 15.1867513656616\\
49.8330001831055 15.1873044967651\\
49.882999420166 15.1878595352173\\
49.9329986572266 15.1884155273438\\
49.9829978942871 15.1889724731445\\
50.0330009460449 15.1895313262939\\
50.0830001831055 15.1900901794434\\
50.132999420166 15.1906480789185\\
50.1829986572266 15.1912059783936\\
50.2329978942871 15.191764831543\\
50.2830009460449 15.1923246383667\\
50.3330001831055 15.1928863525391\\
50.382999420166 15.1934471130371\\
50.4329986572266 15.1940088272095\\
50.4829978942871 15.1945714950562\\
50.5330009460449 15.1951351165771\\
50.5830001831055 15.1956996917725\\
50.632999420166 15.1962633132935\\
50.6829986572266 15.1968278884888\\
50.7329978942871 15.1973915100098\\
50.7830009460449 15.1979560852051\\
50.8330001831055 15.1985197067261\\
50.882999420166 15.1990842819214\\
50.9329986572266 15.1996479034424\\
50.9829978942871 15.2002124786377\\
51.0330009460449 15.200777053833\\
51.0830001831055 15.201340675354\\
51.132999420166 15.2019052505493\\
51.1829986572266 15.2024688720703\\
51.2329978942871 15.2030334472656\\
51.2830009460449 15.2035970687866\\
51.3330001831055 15.2041616439819\\
51.382999420166 15.2047262191772\\
51.4329986572266 15.2052898406982\\
51.4829978942871 15.2058544158936\\
51.5330009460449 15.2064180374146\\
51.5830001831055 15.2069826126099\\
51.632999420166 15.2075462341309\\
51.6829986572266 15.2081108093262\\
51.7329978942871 15.2086744308472\\
51.7830009460449 15.2092390060425\\
51.8330001831055 15.2098026275635\\
51.882999420166 15.2103672027588\\
51.9329986572266 15.2109308242798\\
51.9829978942871 15.2114953994751\\
52.0330009460449 15.2120590209961\\
52.0830001831055 15.2126235961914\\
52.132999420166 15.2131872177124\\
52.1829986572266 15.2137517929077\\
52.2329978942871 15.2143154144287\\
52.2830009460449 15.214879989624\\
52.3330001831055 15.215443611145\\
52.382999420166 15.2160081863403\\
52.4329986572266 15.2165718078613\\
52.4829978942871 15.2171363830566\\
52.5330009460449 15.2177000045776\\
52.5830001831055 15.2182645797729\\
52.632999420166 15.2188282012939\\
52.6829986572266 15.2193927764893\\
52.7329978942871 15.2199563980103\\
52.7830009460449 15.2205209732056\\
52.8330001831055 15.2210845947266\\
52.882999420166 15.2216491699219\\
52.9329986572266 15.2222127914429\\
52.9829978942871 15.2227773666382\\
53.0330009460449 15.2233409881592\\
53.0830001831055 15.2239055633545\\
53.132999420166 15.2244691848755\\
53.1829986572266 15.2250337600708\\
53.2329978942871 15.2256116867065\\
53.2830009460449 15.2261562347412\\
53.3330001831055 15.2267189025879\\
53.382999420166 15.2272825241089\\
53.4329986572266 15.2278470993042\\
53.4829978942871 15.2284107208252\\
53.5330009460449 15.2289752960205\\
53.5830001831055 15.2295398712158\\
53.632999420166 15.2301034927368\\
53.6829986572266 15.2306680679321\\
53.7329978942871 15.2312316894531\\
53.7830009460449 15.2317962646484\\
53.8330001831055 15.2323598861694\\
53.882999420166 15.2329244613647\\
53.9329986572266 15.2334880828857\\
53.9829978942871 15.2340526580811\\
54.0330009460449 15.2346172332764\\
54.0830001831055 15.2351808547974\\
54.132999420166 15.2357454299927\\
54.1829986572266 15.2363090515137\\
54.2329978942871 15.236873626709\\
54.2830009460449 15.23743724823\\
54.3330001831055 15.2380018234253\\
54.382999420166 15.2385663986206\\
54.4329986572266 15.2391300201416\\
54.4829978942871 15.2396945953369\\
54.5330009460449 15.2402591705322\\
54.5830001831055 15.2408237457275\\
54.632999420166 15.2413892745972\\
54.6829986572266 15.2419548034668\\
54.7329978942871 15.2425260543823\\
54.7830009460449 15.2431030273438\\
54.8330001831055 15.2436962127686\\
54.882999420166 15.2443294525146\\
54.9329986572266 15.2450561523438\\
54.9829978942871 15.2459955215454\\
55.0330009460449 15.2474031448364\\
55.0830001831055 15.2496757507324\\
55.132999420166 15.2531967163086\\
55.1829986572266 15.2576560974121\\
55.2329978942871 15.2615919113159\\
55.2830009460449 15.2648315429688\\
55.3330001831055 15.2547044754028\\
55.382999420166 15.2541885375977\\
55.4329986572266 15.2538328170776\\
55.4829978942871 15.2536144256592\\
55.5330009460449 15.253249168396\\
55.5830001831055 15.2538652420044\\
55.632999420166 15.2534046173096\\
55.6829986572266 15.2540292739868\\
55.7329978942871 15.2543601989746\\
55.7830009460449 15.2550163269043\\
55.8330001831055 15.2555370330811\\
55.882999420166 15.256121635437\\
55.9329986572266 15.2566766738892\\
55.9829978942871 15.2572450637817\\
56.0330009460449 15.2578077316284\\
56.0830001831055 15.258373260498\\
56.132999420166 15.258936882019\\
56.1829986572266 15.2595014572144\\
56.2329978942871 15.2600660324097\\
56.2830009460449 15.260630607605\\
56.3330001831055 15.261194229126\\
56.382999420166 15.2617588043213\\
56.4329986572266 15.2623233795166\\
56.4829978942871 15.2628879547119\\
56.5330009460449 15.2634525299072\\
56.5830001831055 15.2640171051025\\
56.632999420166 15.2645807266235\\
56.6829986572266 15.2651453018188\\
56.7329978942871 15.2657098770142\\
56.7830009460449 15.2662744522095\\
56.8330001831055 15.2668390274048\\
56.882999420166 15.2674036026001\\
56.9329986572266 15.2679681777954\\
56.9829978942871 15.2685327529907\\
57.0330009460449 15.269097328186\\
57.0830001831055 15.2696619033813\\
57.132999420166 15.2702264785767\\
57.1829986572266 15.270791053772\\
57.2329978942871 15.2713556289673\\
57.2830009460449 15.2719202041626\\
57.3330001831055 15.2724847793579\\
57.382999420166 15.2730493545532\\
57.4329986572266 15.2736139297485\\
57.4829978942871 15.2741794586182\\
57.5330009460449 15.2747440338135\\
57.5830001831055 15.2753086090088\\
57.632999420166 15.2758731842041\\
57.6829986572266 15.2764377593994\\
57.7329978942871 15.277003288269\\
57.7830009460449 15.2776031494141\\
57.8330001831055 15.278416633606\\
57.882999420166 15.2789869308472\\
57.9329986572266 15.2795610427856\\
57.9829978942871 15.2801246643066\\
58.0330009460449 15.2806901931763\\
58.0830001831055 15.2812557220459\\
58.132999420166 15.2818202972412\\
58.1829986572266 15.2823858261108\\
58.2329978942871 15.2829513549805\\
58.2830009460449 15.2835168838501\\
58.3330001831055 15.2840824127197\\
58.382999420166 15.2846479415894\\
58.4329986572266 15.285213470459\\
58.4829978942871 15.2857789993286\\
58.5330009460449 15.2863445281982\\
58.5830001831055 15.2869110107422\\
58.632999420166 15.2874765396118\\
58.6829986572266 15.2880430221558\\
58.7329978942871 15.2886085510254\\
58.7830009460449 15.2891750335693\\
58.8330001831055 15.2897415161133\\
58.882999420166 15.2903079986572\\
58.9329986572266 15.2908744812012\\
58.9829978942871 15.2914409637451\\
59.0330009460449 15.2920083999634\\
59.0830001831055 15.2925748825073\\
59.132999420166 15.2931423187256\\
59.1829986572266 15.2937097549438\\
59.2329978942871 15.2942771911621\\
59.2830009460449 15.2948274612427\\
59.3330001831055 15.2953062057495\\
59.382999420166 15.2956609725952\\
59.4329986572266 15.2962446212769\\
59.4829978942871 15.2967967987061\\
59.5330009460449 15.2973604202271\\
59.5830001831055 15.2979202270508\\
59.632999420166 15.2984809875488\\
59.6829986572266 15.2990417480469\\
59.7329978942871 15.2996015548706\\
59.7830009460449 15.3001613616943\\
59.8330001831055 15.3007211685181\\
59.882999420166 15.3012809753418\\
59.9329986572266 15.3018407821655\\
59.9664993286133 15.3023452758789\\
60 15.3027782440186\\
60.0368499755859 15.3031597137451\\
60.0773849487305 15.3035717010498\\
60.1219749450684 15.3040246963501\\
60.1710243225098 15.3045225143433\\
60.2210235595703 15.3050584793091\\
60.2710227966309 15.3056154251099\\
60.3210220336914 15.306173324585\\
60.371021270752 15.3067474365234\\
60.4210243225098 15.3073616027832\\
60.4710235595703 15.3079481124878\\
60.5210227966309 15.3085165023804\\
60.5710220336914 15.3091974258423\\
60.621021270752 15.3097925186157\\
60.6710243225098 15.3103590011597\\
60.7210235595703 15.3109378814697\\
60.7710227966309 15.3115186691284\\
60.8210220336914 15.3121042251587\\
60.871021270752 15.3126945495605\\
60.9210243225098 15.3132915496826\\
60.9710235595703 15.3138961791992\\
61.0210227966309 15.3145112991333\\
61.0710220336914 15.3151397705078\\
61.121021270752 15.31578540802\\
61.1710243225098 15.3166475296021\\
61.2210235595703 15.3225688934326\\
61.2710227966309 15.3238210678101\\
61.3210220336914 15.3252477645874\\
61.371021270752 15.326807975769\\
61.4210243225098 15.3241987228394\\
61.4710235595703 15.3195514678955\\
61.5210227966309 15.319993019104\\
61.5710220336914 15.3203506469727\\
61.621021270752 15.3207530975342\\
61.6710243225098 15.3211908340454\\
61.7210235595703 15.3216609954834\\
61.7710227966309 15.3221626281738\\
61.8210220336914 15.3226861953735\\
61.871021270752 15.3232231140137\\
61.9210243225098 15.3237676620483\\
61.9710235595703 15.3243160247803\\
62.0210227966309 15.3248672485352\\
62.0710220336914 15.3254203796387\\
62.121021270752 15.3259744644165\\
62.1710243225098 15.326530456543\\
62.2210235595703 15.3270874023438\\
62.2710227966309 15.3276453018188\\
62.3210220336914 15.3282051086426\\
62.371021270752 15.3287630081177\\
62.4210243225098 15.3293209075928\\
62.4710235595703 15.3298797607422\\
62.5210227966309 15.3304395675659\\
62.5710220336914 15.331000328064\\
62.621021270752 15.331561088562\\
62.6710243225098 15.3321228027344\\
62.7210235595703 15.3326854705811\\
62.7710227966309 15.3332481384277\\
62.8210220336914 15.3338117599487\\
62.871021270752 15.334376335144\\
62.9210243225098 15.334939956665\\
62.9710235595703 15.3355045318604\\
63.0210227966309 15.3360681533813\\
63.0710220336914 15.3366327285767\\
63.121021270752 15.337197303772\\
63.1710243225098 15.337760925293\\
63.2210235595703 15.3383255004883\\
63.2710227966309 15.3388900756836\\
63.3210220336914 15.3394536972046\\
63.371021270752 15.3400182723999\\
63.4210243225098 15.3405818939209\\
63.4710235595703 15.3411464691162\\
63.5210227966309 15.3417110443115\\
63.5710220336914 15.3422746658325\\
63.621021270752 15.3428392410278\\
63.6710243225098 15.3434028625488\\
63.7210235595703 15.3439674377441\\
63.7710227966309 15.3445310592651\\
63.8210220336914 15.3450956344604\\
63.871021270752 15.3456592559814\\
63.9210243225098 15.3462238311768\\
63.9710235595703 15.3467884063721\\
64.0210189819336 15.3473520278931\\
64.0710220336914 15.3479166030884\\
64.1210250854492 15.3484802246094\\
64.1710205078125 15.3490447998047\\
64.2210235595703 15.3496084213257\\
64.2710189819336 15.350172996521\\
64.3210220336914 15.350736618042\\
64.3710250854492 15.3513011932373\\
64.4210205078125 15.3518648147583\\
64.4710235595703 15.3524293899536\\
64.5210189819336 15.3529930114746\\
64.5710220336914 15.3535575866699\\
64.6210250854492 15.3541212081909\\
64.6710205078125 15.3546857833862\\
64.7210235595703 15.3552503585815\\
64.7710189819336 15.3558139801025\\
64.8210220336914 15.3563785552979\\
64.8710250854492 15.3569421768188\\
64.9210205078125 15.3575067520142\\
64.9710235595703 15.3580703735352\\
65.0210189819336 15.3586349487305\\
65.0710220336914 15.3591985702515\\
65.1210250854492 15.3597631454468\\
65.1710205078125 15.3603267669678\\
65.2210235595703 15.3608913421631\\
65.2710189819336 15.3614549636841\\
65.3210220336914 15.3620195388794\\
65.3710250854492 15.3625831604004\\
65.4210205078125 15.3631477355957\\
65.4710235595703 15.3637113571167\\
65.5210189819336 15.364275932312\\
65.5710220336914 15.364839553833\\
65.6210250854492 15.3654041290283\\
65.6710205078125 15.3659811019897\\
65.7210235595703 15.3665266036987\\
65.7710189819336 15.3670892715454\\
65.8210220336914 15.3676528930664\\
65.8710250854492 15.3682174682617\\
65.9210205078125 15.3687810897827\\
65.9710235595703 15.369345664978\\
66.0210189819336 15.3699102401733\\
66.0710220336914 15.3704738616943\\
66.1210250854492 15.3710384368896\\
66.1710205078125 15.3716020584106\\
66.2210235595703 15.372166633606\\
66.2710189819336 15.3727312088013\\
66.3210220336914 15.3732948303223\\
66.3710250854492 15.3738594055176\\
66.4210205078125 15.3744230270386\\
66.4710235595703 15.3749876022339\\
66.5210189819336 15.3755512237549\\
66.5710220336914 15.3761157989502\\
66.6210250854492 15.3766803741455\\
66.6710205078125 15.3772439956665\\
66.7210235595703 15.3778085708618\\
66.7710189819336 15.3783731460571\\
66.8210220336914 15.3789377212524\\
66.8710250854492 15.3795032501221\\
66.9210205078125 15.3800678253174\\
66.9710235595703 15.3806381225586\\
67.0210189819336 15.3812122344971\\
67.0710220336914 15.3817987442017\\
67.1210250854492 15.3824157714844\\
67.1710205078125 15.3831052780151\\
67.2210235595703 15.3839616775513\\
67.2710189819336 15.3851890563965\\
67.3210220336914 15.3871479034424\\
67.3710250854492 15.3902730941772\\
67.4210205078125 15.394552230835\\
67.4710235595703 15.3988475799561\\
67.5210189819336 15.4051685333252\\
67.5710220336914 15.3934564590454\\
67.6210250854492 15.3925790786743\\
67.6710205078125 15.3922109603882\\
67.7210235595703 15.3919515609741\\
67.7710189819336 15.3913669586182\\
67.8210220336914 15.39231300354\\
67.8710250854492 15.3917684555054\\
67.9210205078125 15.3920783996582\\
67.9710235595703 15.3925065994263\\
68.0210189819336 15.3931150436401\\
68.0710220336914 15.3936595916748\\
68.1210250854492 15.3942337036133\\
68.1710205078125 15.394793510437\\
68.2210235595703 15.395359992981\\
68.2710189819336 15.3959226608276\\
68.3210220336914 15.3964872360229\\
68.3710250854492 15.3970518112183\\
68.4210205078125 15.3976163864136\\
68.4710235595703 15.3981809616089\\
68.5210189819336 15.3987445831299\\
68.5710220336914 15.3993091583252\\
68.6210250854492 15.3998737335205\\
68.6710205078125 15.4004383087158\\
68.7210235595703 15.4010028839111\\
68.7710189819336 15.4015674591064\\
68.8210220336914 15.4021320343018\\
68.8710250854492 15.4026956558228\\
68.9210205078125 15.4032602310181\\
68.9710235595703 15.4038248062134\\
69.0210189819336 15.4043893814087\\
69.0710220336914 15.404953956604\\
69.1210250854492 15.4055185317993\\
69.1710205078125 15.4060831069946\\
69.2210235595703 15.4066476821899\\
69.2710189819336 15.4072122573853\\
69.3210220336914 15.4077768325806\\
69.3710250854492 15.4083414077759\\
69.4210205078125 15.4089059829712\\
69.4710235595703 15.4094705581665\\
69.5210189819336 15.4100351333618\\
69.5710220336914 15.4105997085571\\
69.6210250854492 15.4111642837524\\
69.6710205078125 15.4117298126221\\
69.7210235595703 15.4122943878174\\
69.7710189819336 15.4128589630127\\
69.8210220336914 15.413423538208\\
69.8710250854492 15.4139881134033\\
69.9210205078125 15.4145536422729\\
69.9710235595703 15.4151182174683\\
70.0210189819336 15.4156827926636\\
70.0710220336914 15.4164791107178\\
70.1210250854492 15.4170989990234\\
70.1710205078125 15.4176740646362\\
70.2210235595703 15.4182395935059\\
70.2710189819336 15.4188051223755\\
70.3210220336914 15.4193696975708\\
70.3710250854492 15.4199352264404\\
70.4210205078125 15.4205007553101\\
70.4710235595703 15.4210653305054\\
70.5210189819336 15.421630859375\\
70.5710220336914 15.4221963882446\\
70.6210250854492 15.4227619171143\\
70.6710205078125 15.4233274459839\\
70.7210235595703 15.4238939285278\\
70.7710189819336 15.4244594573975\\
70.8210220336914 15.4250249862671\\
70.8710250854492 15.425591468811\\
70.9210205078125 15.4261569976807\\
70.9710235595703 15.4267234802246\\
71.0210189819336 15.4272899627686\\
71.0710220336914 15.4278564453125\\
71.1210250854492 15.4284229278564\\
71.1710205078125 15.4289894104004\\
71.2210235595703 15.4295558929443\\
71.2710189819336 15.4301223754883\\
71.3210220336914 15.4306898117065\\
71.3710250854492 15.4312562942505\\
71.4210205078125 15.4318237304688\\
71.4710235595703 15.432391166687\\
71.5210189819336 15.4329528808594\\
71.5710220336914 15.4334907531738\\
71.6210250854492 15.4337921142578\\
71.6710205078125 15.4343614578247\\
71.7210235595703 15.4349145889282\\
71.7710189819336 15.4354772567749\\
71.8210220336914 15.4360380172729\\
71.8710250854492 15.436598777771\\
71.9210205078125 15.437159538269\\
71.9710235595703 15.4377202987671\\
72.0210189819336 15.4382801055908\\
72.0710220336914 15.4388399124146\\
72.1210250854492 15.4393997192383\\
72.1710205078125 15.439959526062\\
72.2210235595703 15.4405193328857\\
72.2710189819336 15.4410781860352\\
72.3210220336914 15.4416379928589\\
72.3710250854492 15.442195892334\\
72.4210205078125 15.4427547454834\\
72.4710235595703 15.4433126449585\\
72.5210189819336 15.4438705444336\\
72.5710220336914 15.4444284439087\\
72.6210250854492 15.4449977874756\\
72.6710205078125 15.4456119537354\\
72.7210235595703 15.4461994171143\\
72.7710189819336 15.4467678070068\\
72.8210220336914 15.4474382400513\\
72.8710250854492 15.4480361938477\\
72.9210205078125 15.4486045837402\\
72.9710235595703 15.4491834640503\\
73.0210189819336 15.449764251709\\
73.0710220336914 15.4503498077393\\
73.1210250854492 15.4509391784668\\
73.1710205078125 15.4515352249146\\
73.2210235595703 15.4521398544312\\
73.2710189819336 15.4527530670166\\
73.3210220336914 15.4533796310425\\
73.3710250854492 15.4540243148804\\
73.4210205078125 15.4546909332275\\
73.4710235595703 15.4605655670166\\
73.5210189819336 15.4619998931885\\
73.5710220336914 15.4634218215942\\
73.6210250854492 15.4649629592896\\
73.6710205078125 15.463077545166\\
73.7210235595703 15.4578285217285\\
73.7710189819336 15.4582452774048\\
73.8210220336914 15.4586153030396\\
73.8710250854492 15.4590129852295\\
73.9210205078125 15.4594488143921\\
73.9710235595703 15.4599170684814\\
74.0210189819336 15.4604167938232\\
74.0710220336914 15.4609394073486\\
74.1210250854492 15.4614753723145\\
74.1710205078125 15.4620199203491\\
74.2210235595703 15.4625682830811\\
74.2710189819336 15.4631185531616\\
74.3210220336914 15.4636716842651\\
74.3710250854492 15.464225769043\\
74.4210205078125 15.4647808074951\\
74.4710235595703 15.4653377532959\\
74.5210189819336 15.465895652771\\
74.5710220336914 15.466456413269\\
74.6210250854492 15.4670133590698\\
74.6710205078125 15.4675712585449\\
74.7210235595703 15.4681301116943\\
74.7710189819336 15.4686899185181\\
74.8210220336914 15.4692506790161\\
74.8710250854492 15.4698114395142\\
74.9210205078125 15.4703731536865\\
74.9605102539063 15.4709367752075\\
75 15.4713802337646\\
};
\addlegendentry{2"};

\addplot [
color=black,
dash pattern=on 1pt off 3pt on 3pt off 3pt
]
table[row sep=crcr]{
0 14.7006950378418\\
9.99999971718069e-10 14.7006950378418\\
2.49999998480632e-09 14.7006950378418\\
4.74999994892755e-09 14.7006950378418\\
8.1250002281763e-09 14.7006950378418\\
1.31874999809156e-08 14.7006950378418\\
2.0781250498203e-08 14.7006950378418\\
3.21718758300449e-08 14.7006950378418\\
4.92578138278077e-08 14.7006950378418\\
7.48867208244519e-08 14.7006950378418\\
1.13330081319418e-07 14.7006950378418\\
1.70995122061868e-07 14.7006950378418\\
2.57492672517401e-07 14.7006950378418\\
3.8723902662241e-07 14.7006950378418\\
5.81858500936505e-07 14.7006950378418\\
8.73787769251066e-07 14.7006950378418\\
1.31168167172291e-06 14.7006950378418\\
1.9685226106958e-06 14.7006950378418\\
2.95378367809462e-06 14.7006959915161\\
4.43167573394021e-06 14.7006959915161\\
6.64851359033491e-06 14.7006959915161\\
9.97376992017962e-06 14.7006959915161\\
1.49616553244414e-05 14.7006969451904\\
2.24434825213393e-05 14.7006978988647\\
3.36662233166862e-05 14.7006988525391\\
5.05003372381907e-05 14.7006998062134\\
7.57515081204474e-05 14.700701713562\\
0.000113628258986864 14.7007036209106\\
0.000170443381648511 14.7007055282593\\
0.000255666091106832 14.7007074356079\\
0.000383500126190484 14.7007083892822\\
0.000575251178815961 14.7007074356079\\
0.000862877757754177 14.7007036209106\\
0.00129431765526533 14.7007007598877\\
0.00194147753063589 14.7006969451904\\
0.00291221728548408 14.7006931304932\\
0.00436832662671804 14.7006912231445\\
0.00597004732117057 14.7006931304932\\
0.00773194013163447 14.7006950378418\\
0.009670021943748 14.7006959915161\\
0.0118019115179777 14.7006959915161\\
0.0141469910740852 14.7006959915161\\
0.0167265776544809 14.7006959915161\\
0.0195641238242388 14.7006959915161\\
0.0226854234933853 14.7006959915161\\
0.0261188540607691 14.7006959915161\\
0.0298956278711557 14.7006959915161\\
0.0340500771999359 14.7006959915161\\
0.0386199727654457 14.7006950378418\\
0.0436468608677387 14.7006950378418\\
0.0491764321923256 14.7006950378418\\
0.0552589632570744 14.7006950378418\\
0.0619497485458851 14.7006950378418\\
0.0693096145987511 14.7006959915161\\
0.0774054601788521 14.7006959915161\\
0.0863108932971954 14.7006931304932\\
0.0961068719625473 14.7006959915161\\
0.106882445514202 14.7006959915161\\
0.118735581636429 14.7006959915161\\
0.13177402317524 14.7006959915161\\
0.146116316318512 14.7006959915161\\
0.161892831325531 14.7006959915161\\
0.179247006773949 14.7006959915161\\
0.198336601257324 14.7006959915161\\
0.21933513879776 14.7006959915161\\
0.242433547973633 14.7006950378418\\
0.267841786146164 14.7006959915161\\
0.29579085111618 14.7006959915161\\
0.32653483748436 14.7006959915161\\
0.360353201627731 14.7006959915161\\
0.397553414106369 14.7006959915161\\
0.438473641872406 14.7006959915161\\
0.483485877513886 14.7006959915161\\
0.532999336719513 14.7006959915161\\
0.582999348640442 14.7006959915161\\
0.632999360561371 14.7006959915161\\
0.6829993724823 14.7006959915161\\
0.732999384403229 14.7006959915161\\
0.782999336719513 14.7006959915161\\
0.832999348640442 14.7006959915161\\
0.882999360561371 14.7006959915161\\
0.9329993724823 14.7006959915161\\
0.982999384403229 14.7006959915161\\
1.03299939632416 14.7006950378418\\
1.08299934864044 14.7006950378418\\
1.13299930095673 14.7006950378418\\
1.1829993724823 14.7006950378418\\
1.23299932479858 14.7006950378418\\
1.28299939632416 14.7006950378418\\
1.33299934864044 14.7006950378418\\
1.38299930095673 14.7006950378418\\
1.4329993724823 14.7006950378418\\
1.48299932479858 14.7006950378418\\
1.53299939632416 14.7006950378418\\
1.58299934864044 14.7006950378418\\
1.63299930095673 14.7006950378418\\
1.6829993724823 14.7006950378418\\
1.73299932479858 14.7006950378418\\
1.78299939632416 14.7006950378418\\
1.83299934864044 14.7006950378418\\
1.88299930095673 14.7006950378418\\
1.9329993724823 14.7006950378418\\
1.98299932479858 14.7006950378418\\
2.03299927711487 14.7006950378418\\
2.08299946784973 14.7006950378418\\
2.13299942016602 14.7006950378418\\
2.1829993724823 14.7006950378418\\
2.23299932479858 14.7006950378418\\
2.28299927711487 14.7006950378418\\
2.33299946784973 14.7006950378418\\
2.38299942016602 14.7006950378418\\
2.4329993724823 14.7006950378418\\
2.48299932479858 14.7006950378418\\
2.53299927711487 14.7006950378418\\
2.58299946784973 14.7006950378418\\
2.63299942016602 14.7006950378418\\
2.6829993724823 14.7006950378418\\
2.73299932479858 14.7006950378418\\
2.78299927711487 14.7006950378418\\
2.83299946784973 14.7006950378418\\
2.88299942016602 14.7006950378418\\
2.9329993724823 14.7006950378418\\
2.98299932479858 14.7006950378418\\
3.03299927711487 14.7006950378418\\
3.08299946784973 14.7006950378418\\
3.13299942016602 14.7006950378418\\
3.1829993724823 14.7006950378418\\
3.23299932479858 14.7006950378418\\
3.28299927711487 14.7006950378418\\
3.33299946784973 14.7006950378418\\
3.38299942016602 14.7006950378418\\
3.4329993724823 14.7006950378418\\
3.48299932479858 14.7006950378418\\
3.53299927711487 14.7006950378418\\
3.58299946784973 14.7006950378418\\
3.63299942016602 14.7006950378418\\
3.6829993724823 14.7006950378418\\
3.73299932479858 14.7006950378418\\
3.78299927711487 14.7006950378418\\
3.83299946784973 14.7006950378418\\
3.88299942016602 14.7006950378418\\
3.9329993724823 14.7006950378418\\
3.98299932479858 14.7006950378418\\
4.03299951553345 14.7006950378418\\
4.08299922943115 14.7006950378418\\
4.13299942016602 14.7006950378418\\
4.18299913406372 14.7006950378418\\
4.23299932479858 14.7006950378418\\
4.28299951553345 14.7006950378418\\
4.33299922943115 14.7006950378418\\
4.38299942016602 14.7006950378418\\
4.43299913406372 14.7006950378418\\
4.48299932479858 14.7006950378418\\
4.53299951553345 14.7006950378418\\
4.58299922943115 14.7006950378418\\
4.63299942016602 14.7006950378418\\
4.68299913406372 14.7006950378418\\
4.73299932479858 14.7006950378418\\
4.78299951553345 14.7006950378418\\
4.83299922943115 14.7006950378418\\
4.88299942016602 14.7006950378418\\
4.93299913406372 14.7006950378418\\
4.98299932479858 14.7006950378418\\
5.03299951553345 14.7006950378418\\
5.08299922943115 14.7006950378418\\
5.13299942016602 14.7006950378418\\
5.18299913406372 14.7006950378418\\
5.23299932479858 14.7006950378418\\
5.28299951553345 14.7006950378418\\
5.33299922943115 14.7006950378418\\
5.38299942016602 14.7006950378418\\
5.43299913406372 14.7006950378418\\
5.48299932479858 14.7006950378418\\
5.53299951553345 14.7006950378418\\
5.58299922943115 14.7006950378418\\
5.63299942016602 14.7006950378418\\
5.68299913406372 14.7006950378418\\
5.73299932479858 14.7006950378418\\
5.78299951553345 14.7006950378418\\
5.83299922943115 14.7006950378418\\
5.88299942016602 14.7006950378418\\
5.93299913406372 14.7006950378418\\
5.98299932479858 14.7006950378418\\
6.03299951553345 14.7006950378418\\
6.08299922943115 14.7006940841675\\
6.13299942016602 14.7006931304932\\
6.18299913406372 14.7006921768188\\
6.23299932479858 14.7006902694702\\
6.28299951553345 14.7006845474243\\
6.33299922943115 14.700701713562\\
6.38299942016602 14.7006959915161\\
6.43299913406372 14.7006959915161\\
6.48299932479858 14.7006959915161\\
6.53299951553345 14.7006959915161\\
6.58299922943115 14.7006988525391\\
6.63299942016602 14.7006978988647\\
6.68299913406372 14.7006959915161\\
6.73299932479858 14.7006950378418\\
6.78299951553345 14.7006950378418\\
6.83299922943115 14.7006950378418\\
6.88299942016602 14.7006950378418\\
6.93299913406372 14.7006950378418\\
6.98299932479858 14.7006950378418\\
7.03299951553345 14.7006950378418\\
7.08299922943115 14.7006950378418\\
7.13299942016602 14.7006950378418\\
7.18299913406372 14.7006950378418\\
7.23299932479858 14.7006950378418\\
7.28299951553345 14.7006950378418\\
7.33299922943115 14.7006950378418\\
7.38299942016602 14.7006950378418\\
7.43299913406372 14.7006950378418\\
7.48299932479858 14.7006950378418\\
7.53299951553345 14.7006950378418\\
7.58299922943115 14.7006950378418\\
7.63299942016602 14.7006950378418\\
7.68299913406372 14.7006950378418\\
7.73299932479858 14.7006950378418\\
7.78299951553345 14.7006950378418\\
7.83299922943115 14.7006950378418\\
7.88299942016602 14.7006950378418\\
7.93299913406372 14.7006950378418\\
7.98299932479858 14.7006950378418\\
8.03299903869629 14.7006950378418\\
8.08299922943115 14.7006950378418\\
8.13299942016602 14.7006950378418\\
8.18299961090088 14.7006950378418\\
8.23299980163574 14.7006950378418\\
8.28299903869629 14.7006950378418\\
8.33299922943115 14.7006950378418\\
8.38299942016602 14.7006950378418\\
8.43299961090088 14.7006950378418\\
8.48299980163574 14.7006950378418\\
8.53299903869629 14.7006950378418\\
8.58299922943115 14.7006950378418\\
8.63299942016602 14.7006950378418\\
8.68299961090088 14.7006950378418\\
8.73299980163574 14.7006950378418\\
8.78299903869629 14.7006950378418\\
8.83299922943115 14.7006950378418\\
8.88299942016602 14.7006950378418\\
8.93299961090088 14.7006950378418\\
8.98299980163574 14.7006950378418\\
9.03299903869629 14.7006950378418\\
9.08299922943115 14.7006950378418\\
9.13299942016602 14.7006950378418\\
9.18299961090088 14.7006950378418\\
9.23299980163574 14.7006950378418\\
9.28299903869629 14.7006950378418\\
9.33299922943115 14.7006950378418\\
9.38299942016602 14.7006950378418\\
9.43299961090088 14.7006950378418\\
9.48299980163574 14.7006950378418\\
9.53299903869629 14.7006950378418\\
9.58299922943115 14.7006950378418\\
9.63299942016602 14.7006950378418\\
9.68299961090088 14.7006950378418\\
9.73299980163574 14.7006950378418\\
9.78299903869629 14.7006950378418\\
9.83299922943115 14.7006950378418\\
9.88299942016602 14.7006950378418\\
9.93299961090088 14.7006950378418\\
9.98299980163574 14.7006950378418\\
10.0329990386963 14.7006950378418\\
10.0829992294312 14.7006950378418\\
10.132999420166 14.7006950378418\\
10.1829996109009 14.7006950378418\\
10.2329998016357 14.7006950378418\\
10.2829990386963 14.7006950378418\\
10.3329992294312 14.7006950378418\\
10.382999420166 14.7006950378418\\
10.4329996109009 14.7006950378418\\
10.4829998016357 14.7006950378418\\
10.5329990386963 14.7006950378418\\
10.5829992294312 14.7006950378418\\
10.632999420166 14.7006950378418\\
10.6829996109009 14.7006950378418\\
10.7329998016357 14.7006950378418\\
10.7829990386963 14.7006950378418\\
10.8329992294312 14.7006950378418\\
10.882999420166 14.7006950378418\\
10.9329996109009 14.7006950378418\\
10.9829998016357 14.7006950378418\\
11.0329990386963 14.7006950378418\\
11.0829992294312 14.7006950378418\\
11.132999420166 14.7006950378418\\
11.1829996109009 14.7006950378418\\
11.2329998016357 14.7006950378418\\
11.2829990386963 14.7006950378418\\
11.3329992294312 14.7006950378418\\
11.382999420166 14.7006950378418\\
11.4329996109009 14.7006950378418\\
11.4829998016357 14.7006950378418\\
11.5329990386963 14.7006950378418\\
11.5829992294312 14.7006959915161\\
11.632999420166 14.7006959915161\\
11.6829996109009 14.7006959915161\\
11.7329998016357 14.7006950378418\\
11.7829990386963 14.7006950378418\\
11.8329992294312 14.7006950378418\\
11.882999420166 14.7006950378418\\
11.9329996109009 14.7006950378418\\
11.9829998016357 14.7006950378418\\
12.0329990386963 14.7006950378418\\
12.0829992294312 14.7006950378418\\
12.132999420166 14.7006950378418\\
12.1829996109009 14.7006950378418\\
12.2329998016357 14.7006950378418\\
12.2829990386963 14.7006950378418\\
12.3329992294312 14.7006950378418\\
12.382999420166 14.7006950378418\\
12.4329996109009 14.7006950378418\\
12.4829998016357 14.7006950378418\\
12.5329990386963 14.7006931304932\\
12.5829992294312 14.7006950378418\\
12.632999420166 14.7006950378418\\
12.6829996109009 14.7006969451904\\
12.7329998016357 14.7006959915161\\
12.7829990386963 14.7006959915161\\
12.8329992294312 14.7006959915161\\
12.882999420166 14.7006959915161\\
12.9329996109009 14.7006959915161\\
12.9829998016357 14.7006959915161\\
13.0329990386963 14.7006959915161\\
13.0829992294312 14.7006959915161\\
13.132999420166 14.7006959915161\\
13.1829996109009 14.7006959915161\\
13.2329998016357 14.7006959915161\\
13.2829990386963 14.7006959915161\\
13.3329992294312 14.7006959915161\\
13.382999420166 14.7006959915161\\
13.4329996109009 14.7006959915161\\
13.4829998016357 14.7006959915161\\
13.5329990386963 14.7006959915161\\
13.5829992294312 14.7006959915161\\
13.632999420166 14.7006959915161\\
13.6829996109009 14.7006959915161\\
13.7329998016357 14.7006959915161\\
13.7829990386963 14.7006959915161\\
13.8329992294312 14.7006959915161\\
13.882999420166 14.7006959915161\\
13.9329996109009 14.7006959915161\\
13.9829998016357 14.7006959915161\\
14.0329990386963 14.7006959915161\\
14.0829992294312 14.7006959915161\\
14.132999420166 14.7006959915161\\
14.1829996109009 14.7006959915161\\
14.2329998016357 14.7006959915161\\
14.2829990386963 14.7006959915161\\
14.3329992294312 14.7006959915161\\
14.382999420166 14.7006959915161\\
14.4329996109009 14.7006959915161\\
14.4829998016357 14.7006959915161\\
14.5329990386963 14.7006959915161\\
14.5829992294312 14.7006959915161\\
14.632999420166 14.7006959915161\\
14.6829996109009 14.7006959915161\\
14.7329998016357 14.7006959915161\\
14.7829990386963 14.7006959915161\\
14.8329992294312 14.7006959915161\\
14.882999420166 14.7006959915161\\
14.9329996109009 14.7006959915161\\
14.9829998016357 14.7006959915161\\
15.0329990386963 14.7006959915161\\
15.0829992294312 14.7006959915161\\
15.132999420166 14.7006959915161\\
15.1829996109009 14.7006959915161\\
15.2329998016357 14.7006959915161\\
15.2829990386963 14.7006959915161\\
15.3329992294312 14.7006969451904\\
15.382999420166 14.7006959915161\\
15.4329996109009 14.7006959915161\\
15.4829998016357 14.7006959915161\\
15.5329990386963 14.7006959915161\\
15.5829992294312 14.7006959915161\\
15.632999420166 14.7006959915161\\
15.6829996109009 14.7006959915161\\
15.7329998016357 14.7006959915161\\
15.7829990386963 14.7006959915161\\
15.8329992294312 14.7006959915161\\
15.882999420166 14.7006959915161\\
15.9329996109009 14.7006959915161\\
15.9829998016357 14.7006959915161\\
16.0329990386963 14.7006959915161\\
16.0830001831055 14.7006959915161\\
16.132999420166 14.7006959915161\\
16.1829986572266 14.7006959915161\\
16.2329998016357 14.7006959915161\\
16.2829990386963 14.7006959915161\\
16.3330001831055 14.7006959915161\\
16.382999420166 14.7006959915161\\
16.4329986572266 14.7006959915161\\
16.4829998016357 14.7006959915161\\
16.5329990386963 14.7006959915161\\
16.5830001831055 14.7006959915161\\
16.632999420166 14.7006959915161\\
16.6829986572266 14.7006959915161\\
16.7329998016357 14.7006959915161\\
16.7829990386963 14.7006959915161\\
16.8330001831055 14.7006959915161\\
16.882999420166 14.7006959915161\\
16.9329986572266 14.7006959915161\\
16.9829998016357 14.7006959915161\\
17.0329990386963 14.7006959915161\\
17.0830001831055 14.7006959915161\\
17.132999420166 14.7006959915161\\
17.1829986572266 14.7006959915161\\
17.2329998016357 14.7006959915161\\
17.2829990386963 14.7006959915161\\
17.3330001831055 14.7006959915161\\
17.382999420166 14.7006959915161\\
17.4329986572266 14.7006959915161\\
17.4829998016357 14.7006959915161\\
17.5329990386963 14.7006959915161\\
17.5830001831055 14.7006959915161\\
17.632999420166 14.7006959915161\\
17.6829986572266 14.7006959915161\\
17.7329998016357 14.7006969451904\\
17.7829990386963 14.7006969451904\\
17.8330001831055 14.7006978988647\\
17.882999420166 14.7006988525391\\
17.9329986572266 14.7007007598877\\
17.9829998016357 14.7007083892822\\
18.0329990386963 14.7007217407227\\
18.0830001831055 14.7007513046265\\
18.132999420166 14.7008190155029\\
18.1829986572266 14.7009763717651\\
18.2329998016357 14.7013454437256\\
18.2829990386963 14.7021923065186\\
18.3330001831055 14.7040529251099\\
18.382999420166 14.7077770233154\\
18.4329986572266 14.7141399383545\\
18.4829998016357 14.7227468490601\\
18.5329990386963 14.736795425415\\
18.5830001831055 14.7316160202026\\
18.632999420166 14.7320680618286\\
18.6829986572266 14.7325525283813\\
18.7329998016357 14.7327566146851\\
18.7829990386963 14.7328624725342\\
18.8330001831055 14.7327880859375\\
18.882999420166 14.7330493927002\\
18.9329986572266 14.7325439453125\\
18.9829998016357 14.7327919006348\\
19.0329990386963 14.7332077026367\\
19.0830001831055 14.733585357666\\
19.132999420166 14.7336692810059\\
19.1829986572266 14.7340040206909\\
19.2329998016357 14.7342357635498\\
19.2829990386963 14.7345161437988\\
19.3330001831055 14.7347736358643\\
19.382999420166 14.7350425720215\\
19.4329986572266 14.7353057861328\\
19.4829998016357 14.7355718612671\\
19.5329990386963 14.7358360290527\\
19.5830001831055 14.736102104187\\
19.632999420166 14.736367225647\\
19.6829986572266 14.7366323471069\\
19.7329998016357 14.7368974685669\\
19.7829990386963 14.7371625900269\\
19.8330001831055 14.7374277114868\\
19.882999420166 14.7376937866211\\
19.9329986572266 14.7379589080811\\
19.9829998016357 14.7382249832153\\
20.0329990386963 14.7384901046753\\
20.0830001831055 14.7387561798096\\
20.132999420166 14.7390213012695\\
20.1829986572266 14.7392864227295\\
20.2329998016357 14.7395524978638\\
20.2829990386963 14.739818572998\\
20.3330001831055 14.7400846481323\\
20.382999420166 14.7403497695923\\
20.4329986572266 14.7406158447266\\
20.4829998016357 14.7408819198608\\
20.5329990386963 14.7411479949951\\
20.5830001831055 14.7414150238037\\
20.632999420166 14.741681098938\\
20.6829986572266 14.7419471740723\\
20.7329998016357 14.7422132492065\\
20.7829990386963 14.7424802780151\\
20.8330001831055 14.7427463531494\\
20.882999420166 14.743013381958\\
20.9329986572266 14.7432804107666\\
20.9829998016357 14.7435464859009\\
21.0329990386963 14.7438135147095\\
21.0830001831055 14.74427318573\\
21.132999420166 14.7446737289429\\
21.1829986572266 14.745005607605\\
21.2329998016357 14.7453308105469\\
21.2829990386963 14.7456521987915\\
21.3330001831055 14.7459745407104\\
21.382999420166 14.7462959289551\\
21.4329986572266 14.7466163635254\\
21.4829998016357 14.7469367980957\\
21.5329990386963 14.747257232666\\
21.5830001831055 14.7475776672363\\
21.632999420166 14.7478981018066\\
21.6829986572266 14.7482175827026\\
21.7329998016357 14.7485370635986\\
21.7829990386963 14.7488565444946\\
21.8330001831055 14.7491750717163\\
21.882999420166 14.7494945526123\\
21.9329986572266 14.749813079834\\
21.9829998016357 14.7501316070557\\
22.0329990386963 14.7504501342773\\
22.0830001831055 14.750768661499\\
22.132999420166 14.7510871887207\\
22.1829986572266 14.7514057159424\\
22.2329998016357 14.7517242431641\\
22.2829990386963 14.7520427703857\\
22.3330001831055 14.7523612976074\\
22.382999420166 14.7526798248291\\
22.4329986572266 14.7529983520508\\
22.4829998016357 14.7533178329468\\
22.5329990386963 14.7536287307739\\
22.5830001831055 14.7539052963257\\
22.632999420166 14.7539625167847\\
22.6829986572266 14.7542324066162\\
22.7329998016357 14.7544956207275\\
22.7829990386963 14.7547655105591\\
22.8330001831055 14.7550325393677\\
22.882999420166 14.7552995681763\\
22.9329986572266 14.7555656433105\\
22.9829998016357 14.7558307647705\\
23.0329990386963 14.7560949325562\\
23.0830001831055 14.7563581466675\\
23.132999420166 14.7566204071045\\
23.1829986572266 14.7568817138672\\
23.2329998016357 14.7571420669556\\
23.2829990386963 14.7574014663696\\
23.3330001831055 14.7576589584351\\
23.382999420166 14.7579154968262\\
23.4329986572266 14.7581701278687\\
23.4829998016357 14.7584238052368\\
23.5329990386963 14.7586755752563\\
23.5830001831055 14.7589254379272\\
23.632999420166 14.7591924667358\\
23.6829986572266 14.7595157623291\\
23.7329998016357 14.7598123550415\\
23.7829990386963 14.7600879669189\\
23.8330001831055 14.7605056762695\\
23.882999420166 14.7608366012573\\
23.9329986572266 14.7611351013184\\
23.9829998016357 14.7614488601685\\
24.0329990386963 14.7617654800415\\
24.0830001831055 14.7620906829834\\
24.132999420166 14.7624225616455\\
24.1829986572266 14.7627649307251\\
24.2329998016357 14.7631196975708\\
24.2829990386963 14.7634897232056\\
24.3330001831055 14.763879776001\\
24.382999420166 14.7642955780029\\
24.4329986572266 14.7650690078735\\
24.4829998016357 14.7707233428955\\
24.5329990386963 14.7723875045776\\
24.5830001831055 14.7741479873657\\
24.632999420166 14.7759227752686\\
24.6829986572266 14.773232460022\\
24.7329998016357 14.7683525085449\\
24.7829990386963 14.7686109542847\\
24.8330001831055 14.7688226699829\\
24.882999420166 14.7691144943237\\
24.9329986572266 14.7694578170776\\
24.9829998016357 14.769850730896\\
25.0329990386963 14.7702875137329\\
25.0830001831055 14.7707557678223\\
25.132999420166 14.7712450027466\\
25.1829986572266 14.7717475891113\\
25.2329998016357 14.7722597122192\\
25.2829990386963 14.7727794647217\\
25.3330001831055 14.7733039855957\\
25.382999420166 14.7738332748413\\
25.4329986572266 14.7743673324585\\
25.4829998016357 14.7749042510986\\
25.5329990386963 14.7754459381104\\
25.5830001831055 14.7759885787964\\
25.632999420166 14.7765321731567\\
25.6829986572266 14.77707862854\\
25.7329998016357 14.777626991272\\
25.7829990386963 14.7781782150269\\
25.8330001831055 14.7787294387817\\
25.882999420166 14.7792835235596\\
25.9329986572266 14.7798385620117\\
25.9829998016357 14.7803945541382\\
26.0329990386963 14.7809505462646\\
26.0830001831055 14.7815074920654\\
26.132999420166 14.7820644378662\\
26.1829986572266 14.782621383667\\
26.2329998016357 14.7831783294678\\
26.2829990386963 14.7837352752686\\
26.3330001831055 14.7842922210693\\
26.382999420166 14.7848491668701\\
26.4329986572266 14.7854061126709\\
26.4829998016357 14.785964012146\\
26.5329990386963 14.7865209579468\\
26.5830001831055 14.7870779037476\\
26.632999420166 14.7876358032227\\
26.6829986572266 14.7881927490234\\
26.7329998016357 14.7887496948242\\
26.7829990386963 14.7893075942993\\
26.8330001831055 14.7898645401001\\
26.882999420166 14.7904224395752\\
26.9329986572266 14.790979385376\\
26.9829998016357 14.7915372848511\\
27.0329990386963 14.7920942306519\\
27.0830001831055 14.792652130127\\
27.132999420166 14.7932090759277\\
27.1829986572266 14.7937669754028\\
27.2329998016357 14.7943248748779\\
27.2829990386963 14.794882774353\\
27.3330001831055 14.7954397201538\\
27.382999420166 14.7959976196289\\
27.4329986572266 14.796555519104\\
27.4829998016357 14.7971134185791\\
27.5329990386963 14.7976713180542\\
27.5830001831055 14.7982292175293\\
27.632999420166 14.7987871170044\\
27.6829986572266 14.7993450164795\\
27.7329998016357 14.7999029159546\\
27.7829990386963 14.8004608154297\\
27.8330001831055 14.8010196685791\\
27.882999420166 14.8015775680542\\
27.9329986572266 14.8021354675293\\
27.9829998016357 14.8026943206787\\
28.0329990386963 14.8032522201538\\
28.0830001831055 14.8038101196289\\
28.132999420166 14.8043689727783\\
28.1829986572266 14.8049411773682\\
28.2329998016357 14.8054819107056\\
28.2829990386963 14.8060426712036\\
28.3330001831055 14.8066034317017\\
28.382999420166 14.8071641921997\\
28.4329986572266 14.8077259063721\\
28.4829998016357 14.8082866668701\\
28.5329990386963 14.8088483810425\\
28.5830001831055 14.8094091415405\\
28.632999420166 14.8099708557129\\
28.6829986572266 14.8105316162109\\
28.7329998016357 14.8110933303833\\
28.7829990386963 14.8116550445557\\
28.8330001831055 14.8122158050537\\
28.882999420166 14.8127775192261\\
28.9329986572266 14.8133392333984\\
28.9829998016357 14.8138999938965\\
29.0329990386963 14.8144617080688\\
29.0830001831055 14.8150234222412\\
29.132999420166 14.8155851364136\\
29.1829986572266 14.8161458969116\\
29.2329998016357 14.816707611084\\
29.2829990386963 14.8172693252563\\
29.3330001831055 14.8178310394287\\
29.382999420166 14.8183927536011\\
29.4329986572266 14.8189544677734\\
29.4829998016357 14.8195152282715\\
29.5329990386963 14.8200769424438\\
29.5830001831055 14.8206386566162\\
29.632999420166 14.8212003707886\\
29.6829986572266 14.8217620849609\\
29.7329998016357 14.8223237991333\\
29.7829990386963 14.8228855133057\\
29.8330001831055 14.823447227478\\
29.882999420166 14.8240089416504\\
29.9329986572266 14.8245706558228\\
29.9829998016357 14.8251333236694\\
30.0329990386963 14.8256950378418\\
30.0830001831055 14.8262577056885\\
30.132999420166 14.8268203735352\\
30.1829986572266 14.8273830413818\\
30.2329998016357 14.8279514312744\\
30.2829990386963 14.8285255432129\\
30.3330001831055 14.8291139602661\\
30.382999420166 14.8297395706177\\
30.4329986572266 14.8304510116577\\
30.4829998016357 14.8313598632813\\
30.5329990386963 14.8327074050903\\
30.5830001831055 14.8348760604858\\
30.632999420166 14.8382768630981\\
30.6829986572266 14.8426961898804\\
30.7329998016357 14.8467597961426\\
30.7829990386963 14.8509330749512\\
30.8330001831055 14.8402805328369\\
30.882999420166 14.8396711349487\\
30.9329986572266 14.8393115997314\\
30.9829998016357 14.8390712738037\\
31.0329990386963 14.8386363983154\\
31.0830001831055 14.8393602371216\\
31.132999420166 14.8388528823853\\
31.1829986572266 14.8393907546997\\
31.2329998016357 14.8397521972656\\
31.2829990386963 14.8403911590576\\
31.3330001831055 14.8409175872803\\
31.382999420166 14.8414974212646\\
31.4329986572266 14.8420515060425\\
31.4829998016357 14.8426179885864\\
31.5329990386963 14.8431787490845\\
31.5830001831055 14.8437423706055\\
31.632999420166 14.8443040847778\\
31.6829986572266 14.8448667526245\\
31.7329998016357 14.8454294204712\\
31.7829990386963 14.8459920883179\\
31.8330001831055 14.8465547561646\\
31.882999420166 14.8471174240112\\
31.9329986572266 14.8476800918579\\
31.9829998016357 14.8482427597046\\
32.0330009460449 14.8488054275513\\
32.0830001831055 14.8493680953979\\
32.132999420166 14.8499307632446\\
32.1829986572266 14.8504934310913\\
32.2329978942871 14.851056098938\\
32.2830009460449 14.8516187667847\\
32.3330001831055 14.8521814346313\\
32.382999420166 14.852744102478\\
32.4329986572266 14.8533067703247\\
32.4829978942871 14.8538703918457\\
32.5330009460449 14.8544330596924\\
32.5830001831055 14.8549957275391\\
32.632999420166 14.8555583953857\\
32.6829986572266 14.8561220169067\\
32.7329978942871 14.8566846847534\\
32.7830009460449 14.8572473526001\\
32.8330001831055 14.8578100204468\\
32.882999420166 14.8583736419678\\
32.9329986572266 14.8589363098145\\
32.9829978942871 14.8594989776611\\
33.0330009460449 14.8600625991821\\
33.0830001831055 14.8606252670288\\
33.132999420166 14.8611879348755\\
33.1829986572266 14.8617515563965\\
33.2329978942871 14.8623142242432\\
33.2830009460449 14.8628978729248\\
33.3330001831055 14.8637247085571\\
33.382999420166 14.8642950057983\\
33.4329986572266 14.8648681640625\\
33.4829978942871 14.8654308319092\\
33.5330009460449 14.8659944534302\\
33.5830001831055 14.8665580749512\\
33.632999420166 14.8671207427979\\
33.6829986572266 14.8676843643188\\
33.7329978942871 14.8682479858398\\
33.7830009460449 14.8688116073608\\
33.8330001831055 14.8693752288818\\
33.882999420166 14.8699398040771\\
33.9329986572266 14.8705034255981\\
33.9829978942871 14.8710670471191\\
34.0330009460449 14.8716316223145\\
34.0830001831055 14.8721952438354\\
34.132999420166 14.8727598190308\\
34.1829986572266 14.8733243942261\\
34.2329978942871 14.8738889694214\\
34.2830009460449 14.8744535446167\\
34.3330001831055 14.875018119812\\
34.382999420166 14.8755826950073\\
34.4329986572266 14.876148223877\\
34.4829978942871 14.8767127990723\\
34.5330009460449 14.8772783279419\\
34.5830001831055 14.8778438568115\\
34.632999420166 14.8784093856812\\
34.6829986572266 14.8789749145508\\
34.7329978942871 14.8795413970947\\
34.7830009460449 14.880087852478\\
34.8330001831055 14.8805704116821\\
34.882999420166 14.8809146881104\\
34.9329986572266 14.881498336792\\
34.9829978942871 14.8820486068726\\
35.0330009460449 14.8826103210449\\
35.0830001831055 14.8831691741943\\
35.132999420166 14.8837280273438\\
35.1829986572266 14.8842868804932\\
35.2329978942871 14.8848457336426\\
35.2830009460449 14.8854036331177\\
35.3330001831055 14.8859615325928\\
35.382999420166 14.8865194320679\\
35.4329986572266 14.887077331543\\
35.4829978942871 14.8876352310181\\
35.5330009460449 14.8881921768188\\
35.5830001831055 14.8887491226196\\
35.632999420166 14.8893051147461\\
35.6829986572266 14.8898620605469\\
35.7329978942871 14.8904180526733\\
35.7830009460449 14.8909730911255\\
35.8330001831055 14.8915281295776\\
35.882999420166 14.8921184539795\\
35.9329986572266 14.8927307128906\\
35.9829978942871 14.8933115005493\\
36.0330009460449 14.8938789367676\\
36.0830001831055 14.8945960998535\\
36.132999420166 14.8951826095581\\
36.1829986572266 14.8957376480103\\
36.2329978942871 14.8963165283203\\
36.2830009460449 14.896897315979\\
36.3330001831055 14.8974828720093\\
36.382999420166 14.8980741500854\\
36.4329986572266 14.8986721038818\\
36.4829978942871 14.8992786407471\\
36.5330009460449 14.8998966217041\\
36.5830001831055 14.9005279541016\\
36.632999420166 14.9011783599854\\
36.6829986572266 14.9033393859863\\
36.7329978942871 14.90807056427\\
36.7830009460449 14.9093885421753\\
36.8330001831055 14.9108400344849\\
36.882999420166 14.9124784469604\\
36.9329986572266 14.9075269699097\\
36.9829978942871 14.9048509597778\\
37.0330009460449 14.9052963256836\\
37.0830001831055 14.9056463241577\\
37.132999420166 14.9060688018799\\
37.1829986572266 14.9065093994141\\
37.2329978942871 14.9069881439209\\
37.2830009460449 14.9074954986572\\
37.3330001831055 14.9080219268799\\
37.382999420166 14.9085597991943\\
37.4329986572266 14.909104347229\\
37.4829978942871 14.9096527099609\\
37.5330009460449 14.9102039337158\\
37.5830001831055 14.910756111145\\
37.632999420166 14.9113101959229\\
37.6829986572266 14.911865234375\\
37.7329978942871 14.9124212265015\\
37.7830009460449 14.9129800796509\\
37.8330001831055 14.9135370254517\\
37.882999420166 14.9140939712524\\
37.9329986572266 14.9146509170532\\
37.9829978942871 14.9152097702026\\
38.0330009460449 14.9157695770264\\
38.0830001831055 14.9163293838501\\
38.132999420166 14.9168901443481\\
38.1829986572266 14.9174509048462\\
38.2329978942871 14.9180135726929\\
38.2830009460449 14.9185762405396\\
38.3330001831055 14.9191398620605\\
38.382999420166 14.9197034835815\\
38.4329986572266 14.9202661514282\\
38.4829978942871 14.9208297729492\\
38.5330009460449 14.9213933944702\\
38.5830001831055 14.9219570159912\\
38.632999420166 14.9225206375122\\
38.6829986572266 14.9230833053589\\
38.7329978942871 14.9236469268799\\
38.7830009460449 14.9242105484009\\
38.8330001831055 14.9247741699219\\
38.882999420166 14.9253377914429\\
38.9329986572266 14.9259014129639\\
38.9829978942871 14.9264650344849\\
39.0330009460449 14.9270277023315\\
39.0830001831055 14.9275913238525\\
39.132999420166 14.9281549453735\\
39.1829986572266 14.9287185668945\\
39.2329978942871 14.9292821884155\\
39.2830009460449 14.9298458099365\\
39.3330001831055 14.9304094314575\\
39.382999420166 14.9309720993042\\
39.4329986572266 14.9315357208252\\
39.4829978942871 14.9320993423462\\
39.5330009460449 14.9326629638672\\
39.5830001831055 14.9332265853882\\
39.632999420166 14.9337902069092\\
39.6829986572266 14.9343528747559\\
39.7329978942871 14.9349164962769\\
39.7830009460449 14.9354801177979\\
39.8330001831055 14.9360437393188\\
39.882999420166 14.9366073608398\\
39.9329986572266 14.9371709823608\\
39.9829978942871 14.9377336502075\\
40.0330009460449 14.9382972717285\\
40.0830001831055 14.9388608932495\\
40.132999420166 14.9394245147705\\
40.1829986572266 14.9399881362915\\
40.2329978942871 14.9405517578125\\
40.2830009460449 14.9411144256592\\
40.3330001831055 14.9416780471802\\
40.382999420166 14.9422416687012\\
40.4329986572266 14.9428052902222\\
40.4829978942871 14.9433689117432\\
40.5330009460449 14.9439325332642\\
40.5830001831055 14.9444961547852\\
40.632999420166 14.9450588226318\\
40.6829986572266 14.9456224441528\\
40.7329978942871 14.9461860656738\\
40.7830009460449 14.9467639923096\\
40.8330001831055 14.9473075866699\\
40.882999420166 14.9478693008423\\
40.9329986572266 14.9484329223633\\
40.9829978942871 14.9489965438843\\
41.0330009460449 14.9495601654053\\
41.0830001831055 14.9501237869263\\
41.132999420166 14.9506874084473\\
41.1829986572266 14.9512510299683\\
41.2329978942871 14.9518146514893\\
41.2830009460449 14.9523782730103\\
41.3330001831055 14.9529418945313\\
41.382999420166 14.9535055160522\\
41.4329986572266 14.9540691375732\\
41.4829978942871 14.9546327590942\\
41.5330009460449 14.9551963806152\\
41.5830001831055 14.9557600021362\\
41.632999420166 14.9563236236572\\
41.6829986572266 14.9568872451782\\
41.7329978942871 14.9574508666992\\
41.7830009460449 14.9580144882202\\
41.8330001831055 14.9585781097412\\
41.882999420166 14.9591417312622\\
41.9329986572266 14.9597053527832\\
41.9829978942871 14.9602689743042\\
42.0330009460449 14.9608325958252\\
42.0830001831055 14.9613962173462\\
42.132999420166 14.9619598388672\\
42.1829986572266 14.9625234603882\\
42.2329978942871 14.9630870819092\\
42.2830009460449 14.9636507034302\\
42.3330001831055 14.9642152786255\\
42.382999420166 14.9647798538208\\
42.4329986572266 14.9653453826904\\
42.4829978942871 14.9659156799316\\
42.5330009460449 14.9664926528931\\
42.5830001831055 14.9670848846436\\
42.632999420166 14.9677181243896\\
42.6829986572266 14.9684448242188\\
42.7329978942871 14.9693870544434\\
42.7830009460449 14.970799446106\\
42.8330001831055 14.9730787277222\\
42.882999420166 14.9766092300415\\
42.9329986572266 14.9810733795166\\
42.9829978942871 14.9849967956543\\
43.0330009460449 14.988151550293\\
43.0830001831055 14.9780673980713\\
43.132999420166 14.9775657653809\\
43.1829986572266 14.9772109985352\\
43.2329978942871 14.9769897460938\\
43.2830009460449 14.9766273498535\\
43.3330001831055 14.977240562439\\
43.382999420166 14.9767799377441\\
43.4329986572266 14.9774084091187\\
43.4829978942871 14.9777393341064\\
43.5330009460449 14.9783945083618\\
43.5830001831055 14.9789152145386\\
43.632999420166 14.9794998168945\\
43.6829986572266 14.9800539016724\\
43.7329978942871 14.9806222915649\\
43.7830009460449 14.9811840057373\\
43.8330001831055 14.9817485809326\\
43.882999420166 14.9823122024536\\
43.9329986572266 14.9828767776489\\
43.9829978942871 14.9834403991699\\
44.0330009460449 14.9840040206909\\
44.0830001831055 14.9845676422119\\
44.132999420166 14.9851322174072\\
44.1829986572266 14.9856958389282\\
44.2329978942871 14.9862604141235\\
44.2830009460449 14.9868240356445\\
44.3330001831055 14.9873876571655\\
44.382999420166 14.9879522323608\\
44.4329986572266 14.9885158538818\\
44.4829978942871 14.9890804290771\\
44.5330009460449 14.9896440505981\\
44.5830001831055 14.9902076721191\\
44.632999420166 14.9907722473145\\
44.6829986572266 14.9913358688354\\
44.7329978942871 14.9919004440308\\
44.7830009460449 14.9924640655518\\
44.8330001831055 14.9930286407471\\
44.882999420166 14.9935922622681\\
44.9329986572266 14.9941568374634\\
44.9829978942871 14.9947214126587\\
45.0330009460449 14.9952850341797\\
45.0830001831055 14.995849609375\\
45.132999420166 14.9964141845703\\
45.1829986572266 14.9969778060913\\
45.2329978942871 14.9975423812866\\
45.2830009460449 14.9981069564819\\
45.3330001831055 14.9986705780029\\
45.382999420166 14.9992351531982\\
45.4329986572266 14.9997997283936\\
45.4829978942871 15.0003643035889\\
45.5330009460449 15.0009698867798\\
45.5830001831055 15.0017766952515\\
45.632999420166 15.002347946167\\
45.6829986572266 15.0029211044312\\
45.7329978942871 15.0034847259521\\
45.7830009460449 15.0040502548218\\
45.8330001831055 15.0046148300171\\
45.882999420166 15.0051794052124\\
45.9329986572266 15.0057439804077\\
45.9829978942871 15.0063095092773\\
46.0330009460449 15.0068740844727\\
46.0830001831055 15.0074396133423\\
46.132999420166 15.0080041885376\\
46.1829986572266 15.0085697174072\\
46.2329978942871 15.0091352462769\\
46.2830009460449 15.0097007751465\\
46.3330001831055 15.0102663040161\\
46.382999420166 15.0108318328857\\
46.4329986572266 15.0113973617554\\
46.4829978942871 15.011962890625\\
46.5330009460449 15.0125284194946\\
46.5830001831055 15.0130949020386\\
46.632999420166 15.0136604309082\\
46.6829986572266 15.0142269134521\\
46.7329978942871 15.0147933959961\\
46.7830009460449 15.01535987854\\
46.8330001831055 15.015926361084\\
46.882999420166 15.0164928436279\\
46.9329986572266 15.0170602798462\\
46.9829978942871 15.0176267623901\\
47.0330009460449 15.0181751251221\\
47.0830001831055 15.0186462402344\\
47.132999420166 15.0190086364746\\
47.1829986572266 15.0195903778076\\
47.2329978942871 15.0201425552368\\
47.2830009460449 15.0207052230835\\
47.3330001831055 15.0212650299072\\
47.382999420166 15.0218257904053\\
47.4329986572266 15.022385597229\\
47.4829978942871 15.0229454040527\\
47.5330009460449 15.0235052108765\\
47.5830001831055 15.0240640640259\\
47.632999420166 15.0246229171753\\
47.6829986572266 15.0251817703247\\
47.7329978942871 15.0257406234741\\
47.7830009460449 15.0262994766235\\
47.8330001831055 15.0268573760986\\
47.882999420166 15.0274152755737\\
47.9329986572266 15.0279722213745\\
47.9829978942871 15.0285301208496\\
48.0330009460449 15.0290870666504\\
48.0830001831055 15.0296430587769\\
48.132999420166 15.0302352905273\\
48.1829986572266 15.0308437347412\\
48.2329978942871 15.0314235687256\\
48.2830009460449 15.0319919586182\\
48.3330001831055 15.0327033996582\\
48.382999420166 15.0332880020142\\
48.4329986572266 15.0338430404663\\
48.4829978942871 15.0344219207764\\
48.5330009460449 15.0350036621094\\
48.5830001831055 15.035590171814\\
48.632999420166 15.0361824035645\\
48.6829986572266 15.0367813110352\\
48.7329978942871 15.0373878479004\\
48.7830009460449 15.0380058288574\\
48.8330001831055 15.0386390686035\\
48.882999420166 15.0392904281616\\
48.9329986572266 15.0416345596313\\
48.9829978942871 15.0461940765381\\
49.0330009460449 15.0475215911865\\
49.0830001831055 15.048975944519\\
49.132999420166 15.0506229400635\\
49.1829986572266 15.0452537536621\\
49.2329978942871 15.0429649353027\\
49.2830009460449 15.0434093475342\\
49.3330001831055 15.0437631607056\\
49.382999420166 15.0441865921021\\
49.4329986572266 15.0446300506592\\
49.4829978942871 15.0451107025146\\
49.5330009460449 15.0456190109253\\
49.5830001831055 15.0461473464966\\
49.632999420166 15.0466861724854\\
49.6829986572266 15.0472316741943\\
49.7329978942871 15.0477800369263\\
49.7830009460449 15.0483312606812\\
49.8330001831055 15.0488843917847\\
49.882999420166 15.0494394302368\\
49.9329986572266 15.049994468689\\
49.9829978942871 15.0505514144897\\
50.0330009460449 15.0511102676392\\
50.0830001831055 15.0516681671143\\
50.132999420166 15.0522260665894\\
50.1829986572266 15.0527839660645\\
50.2329978942871 15.0533428192139\\
50.2830009460449 15.0539026260376\\
50.3330001831055 15.0544633865356\\
50.382999420166 15.0550241470337\\
50.4329986572266 15.0555858612061\\
50.4829978942871 15.0561485290527\\
50.5330009460449 15.0567111968994\\
50.5830001831055 15.0572748184204\\
50.632999420166 15.0578384399414\\
50.6829986572266 15.0584030151367\\
50.7329978942871 15.0589666366577\\
50.7830009460449 15.0595302581787\\
50.8330001831055 15.0600938796997\\
50.882999420166 15.060658454895\\
50.9329986572266 15.061222076416\\
50.9829978942871 15.061785697937\\
51.0330009460449 15.0623502731323\\
51.0830001831055 15.0629138946533\\
51.132999420166 15.0634775161743\\
51.1829986572266 15.0640420913696\\
51.2329978942871 15.0646057128906\\
51.2830009460449 15.0651693344116\\
51.3330001831055 15.0657339096069\\
51.382999420166 15.0662975311279\\
51.4329986572266 15.0668611526489\\
51.4829978942871 15.0674247741699\\
51.5330009460449 15.0679893493652\\
51.5830001831055 15.0685529708862\\
51.632999420166 15.0691165924072\\
51.6829986572266 15.0696811676025\\
51.7329978942871 15.0702447891235\\
51.7830009460449 15.0708084106445\\
51.8330001831055 15.0713720321655\\
51.882999420166 15.0719366073608\\
51.9329986572266 15.0725002288818\\
51.9829978942871 15.0730638504028\\
52.0330009460449 15.0736274719238\\
52.0830001831055 15.0741920471191\\
52.132999420166 15.0747556686401\\
52.1829986572266 15.0753192901611\\
52.2329978942871 15.0758829116821\\
52.2830009460449 15.0764474868774\\
52.3330001831055 15.0770111083984\\
52.382999420166 15.0775747299194\\
52.4329986572266 15.0781383514404\\
52.4829978942871 15.0787029266357\\
52.5330009460449 15.0792665481567\\
52.5830001831055 15.0798301696777\\
52.632999420166 15.080394744873\\
52.6829986572266 15.080958366394\\
52.7329978942871 15.081521987915\\
52.7830009460449 15.082085609436\\
52.8330001831055 15.0826501846313\\
52.882999420166 15.0832138061523\\
52.9329986572266 15.0837774276733\\
52.9829978942871 15.0843410491943\\
53.0330009460449 15.0849056243896\\
53.0830001831055 15.0854692459106\\
53.132999420166 15.0860328674316\\
53.1829986572266 15.0865964889526\\
53.2329978942871 15.0871744155884\\
53.2830009460449 15.087718963623\\
53.3330001831055 15.0882816314697\\
53.382999420166 15.0888452529907\\
53.4329986572266 15.0894088745117\\
53.4829978942871 15.089973449707\\
53.5330009460449 15.090537071228\\
53.5830001831055 15.091100692749\\
53.632999420166 15.0916652679443\\
53.6829986572266 15.0922288894653\\
53.7329978942871 15.0927925109863\\
53.7830009460449 15.0933570861816\\
53.8330001831055 15.0939207077026\\
53.882999420166 15.0944843292236\\
53.9329986572266 15.0950489044189\\
53.9829978942871 15.0956125259399\\
54.0330009460449 15.0961761474609\\
54.0830001831055 15.0967407226563\\
54.132999420166 15.0973043441772\\
54.1829986572266 15.0978679656982\\
54.2329978942871 15.0984325408936\\
54.2830009460449 15.0989961624146\\
54.3330001831055 15.0995597839355\\
54.382999420166 15.1001243591309\\
54.4329986572266 15.1006879806519\\
54.4829978942871 15.1012525558472\\
54.5330009460449 15.1018171310425\\
54.5830001831055 15.1023817062378\\
54.632999420166 15.1029462814331\\
54.6829986572266 15.1035118103027\\
54.7329978942871 15.1040830612183\\
54.7830009460449 15.1046590805054\\
54.8330001831055 15.1052522659302\\
54.882999420166 15.1058855056763\\
54.9329986572266 15.1066122055054\\
54.9829978942871 15.107551574707\\
55.0330009460449 15.1089582443237\\
55.0830001831055 15.1112308502197\\
55.132999420166 15.1147518157959\\
55.1829986572266 15.1192111968994\\
55.2329978942871 15.1231470108032\\
55.2830009460449 15.1263866424561\\
55.3330001831055 15.1162586212158\\
55.382999420166 15.1157426834106\\
55.4329986572266 15.1153869628906\\
55.4829978942871 15.1151685714722\\
55.5330009460449 15.114803314209\\
55.5830001831055 15.1154193878174\\
55.632999420166 15.1149578094482\\
55.6829986572266 15.1155824661255\\
55.7329978942871 15.1159133911133\\
55.7830009460449 15.1165685653687\\
55.8330001831055 15.1170902252197\\
55.882999420166 15.1176748275757\\
55.9329986572266 15.1182289123535\\
55.9829978942871 15.1187973022461\\
56.0330009460449 15.1193599700928\\
56.0830001831055 15.1199245452881\\
56.132999420166 15.1204881668091\\
56.1829986572266 15.1210527420044\\
56.2329978942871 15.1216173171997\\
56.2830009460449 15.1221809387207\\
56.3330001831055 15.122745513916\\
56.382999420166 15.1233100891113\\
56.4329986572266 15.1238737106323\\
56.4829978942871 15.1244382858276\\
56.5330009460449 15.1250028610229\\
56.5830001831055 15.1255664825439\\
56.632999420166 15.1261310577393\\
56.6829986572266 15.1266956329346\\
56.7329978942871 15.1272592544556\\
56.7830009460449 15.1278238296509\\
56.8330001831055 15.1283884048462\\
56.882999420166 15.1289520263672\\
56.9329986572266 15.1295166015625\\
56.9829978942871 15.1300811767578\\
57.0330009460449 15.1306457519531\\
57.0830001831055 15.1312103271484\\
57.132999420166 15.1317749023438\\
57.1829986572266 15.1323394775391\\
57.2329978942871 15.1329030990601\\
57.2830009460449 15.1334676742554\\
57.3330001831055 15.1340322494507\\
57.382999420166 15.134596824646\\
57.4329986572266 15.1351613998413\\
57.4829978942871 15.1357259750366\\
57.5330009460449 15.1362905502319\\
57.5830001831055 15.1368551254272\\
57.632999420166 15.1374197006226\\
57.6829986572266 15.1379842758179\\
57.7329978942871 15.1385488510132\\
57.7830009460449 15.1391496658325\\
57.8330001831055 15.1399621963501\\
57.882999420166 15.1405324935913\\
57.9329986572266 15.1411066055298\\
57.9829978942871 15.1416702270508\\
58.0330009460449 15.1422357559204\\
58.0830001831055 15.1428003311157\\
58.132999420166 15.1433658599854\\
58.1829986572266 15.1439304351807\\
58.2329978942871 15.1444959640503\\
58.2830009460449 15.1450614929199\\
58.3330001831055 15.1456270217896\\
58.382999420166 15.1461925506592\\
58.4329986572266 15.1467571258545\\
58.4829978942871 15.1473236083984\\
58.5330009460449 15.1478891372681\\
58.5830001831055 15.1484546661377\\
58.632999420166 15.1490201950073\\
58.6829986572266 15.1495866775513\\
58.7329978942871 15.1501522064209\\
58.7830009460449 15.1507186889648\\
58.8330001831055 15.1512842178345\\
58.882999420166 15.1518507003784\\
58.9329986572266 15.1524171829224\\
58.9829978942871 15.1529836654663\\
59.0330009460449 15.1535511016846\\
59.0830001831055 15.1541175842285\\
59.132999420166 15.1546840667725\\
59.1829986572266 15.1552515029907\\
59.2329978942871 15.155818939209\\
59.2830009460449 15.1563692092896\\
59.3330001831055 15.1568479537964\\
59.382999420166 15.1572017669678\\
59.4329986572266 15.1577854156494\\
59.4829978942871 15.1583375930786\\
59.5330009460449 15.1589012145996\\
59.5830001831055 15.1594610214233\\
59.632999420166 15.1600217819214\\
59.6829986572266 15.1605815887451\\
59.7329978942871 15.1611423492432\\
59.7830009460449 15.1617021560669\\
59.8330001831055 15.1622619628906\\
59.882999420166 15.1628217697144\\
59.9329986572266 15.1633806228638\\
59.9664993286133 15.1638851165771\\
60 15.1643180847168\\
60.0368499755859 15.1646995544434\\
60.0773849487305 15.165111541748\\
60.1219749450684 15.165563583374\\
60.1710243225098 15.1660614013672\\
60.2210235595703 15.166597366333\\
60.2710227966309 15.1671552658081\\
60.3210220336914 15.1677122116089\\
60.371021270752 15.1682863235474\\
60.4210243225098 15.1689004898071\\
60.4710235595703 15.1694860458374\\
60.5210227966309 15.17005443573\\
60.5710220336914 15.1707353591919\\
60.621021270752 15.1713314056396\\
60.6710243225098 15.1718969345093\\
60.7210235595703 15.1724758148193\\
60.7710227966309 15.173056602478\\
60.8210220336914 15.1736421585083\\
60.871021270752 15.1742324829102\\
60.9210243225098 15.1748294830322\\
60.9710235595703 15.1754341125488\\
61.0210227966309 15.1760482788086\\
61.0710220336914 15.1766767501831\\
61.121021270752 15.1773223876953\\
61.1710243225098 15.1781845092773\\
61.2210235595703 15.1841049194336\\
61.2710227966309 15.185357093811\\
61.3210220336914 15.1867847442627\\
61.371021270752 15.18834400177\\
61.4210243225098 15.1857347488403\\
61.4710235595703 15.1810884475708\\
61.5210227966309 15.181529045105\\
61.5710220336914 15.1818866729736\\
61.621021270752 15.1822891235352\\
61.6710243225098 15.1827259063721\\
61.7210235595703 15.1831970214844\\
61.7710227966309 15.1836986541748\\
61.8210220336914 15.1842222213745\\
61.871021270752 15.1847591400146\\
61.9210243225098 15.185302734375\\
61.9710235595703 15.1858510971069\\
62.0210227966309 15.1864023208618\\
62.0710220336914 15.1869554519653\\
62.121021270752 15.1875095367432\\
62.1710243225098 15.1880645751953\\
62.2210235595703 15.1886215209961\\
62.2710227966309 15.1891794204712\\
62.3210220336914 15.1897392272949\\
62.371021270752 15.1902961730957\\
62.4210243225098 15.1908540725708\\
62.4710235595703 15.1914129257202\\
62.5210227966309 15.1919727325439\\
62.5710220336914 15.192533493042\\
62.621021270752 15.1930952072144\\
62.6710243225098 15.1936559677124\\
62.7210235595703 15.1942186355591\\
62.7710227966309 15.1947813034058\\
62.8210220336914 15.1953449249268\\
62.871021270752 15.1959095001221\\
62.9210243225098 15.1964731216431\\
62.9710235595703 15.1970376968384\\
63.0210227966309 15.1976013183594\\
63.0710220336914 15.1981649398804\\
63.121021270752 15.1987295150757\\
63.1710243225098 15.1992931365967\\
63.2210235595703 15.199857711792\\
63.2710227966309 15.200421333313\\
63.3210220336914 15.2009859085083\\
63.371021270752 15.2015495300293\\
63.4210243225098 15.2021141052246\\
63.4710235595703 15.2026777267456\\
63.5210227966309 15.2032423019409\\
63.5710220336914 15.2038059234619\\
63.621021270752 15.2043704986572\\
63.6710243225098 15.2049341201782\\
63.7210235595703 15.2054986953735\\
63.7710227966309 15.2060623168945\\
63.8210220336914 15.2066268920898\\
63.871021270752 15.2071905136108\\
63.9210243225098 15.2077550888062\\
63.9710235595703 15.2083187103271\\
64.0210189819336 15.2088823318481\\
64.0710220336914 15.2094469070435\\
64.1210250854492 15.2100105285645\\
64.1710205078125 15.2105751037598\\
64.2210235595703 15.2111387252808\\
64.2710189819336 15.2117033004761\\
64.3210220336914 15.2122669219971\\
64.3710250854492 15.2128305435181\\
64.4210205078125 15.2133951187134\\
64.4710235595703 15.2139587402344\\
64.5210189819336 15.2145233154297\\
64.5710220336914 15.2150869369507\\
64.6210250854492 15.215651512146\\
64.6710205078125 15.216215133667\\
64.7210235595703 15.216778755188\\
64.7710189819336 15.2173433303833\\
64.8210220336914 15.2179069519043\\
64.8710250854492 15.2184715270996\\
64.9210205078125 15.2190351486206\\
64.9710235595703 15.2195997238159\\
65.0210189819336 15.2201633453369\\
65.0710220336914 15.2207269668579\\
65.1210250854492 15.2212915420532\\
65.1710205078125 15.2218551635742\\
65.2210235595703 15.2224197387695\\
65.2710189819336 15.2229833602905\\
65.3210220336914 15.2235469818115\\
65.3710250854492 15.2241115570068\\
65.4210205078125 15.2246751785278\\
65.4710235595703 15.2252397537231\\
65.5210189819336 15.2258033752441\\
65.5710220336914 15.2263679504395\\
65.6210250854492 15.2269315719604\\
65.6710205078125 15.2275085449219\\
65.7210235595703 15.2280540466309\\
65.7710189819336 15.2286167144775\\
65.8210220336914 15.2291803359985\\
65.8710250854492 15.2297449111938\\
65.9210205078125 15.2303085327148\\
65.9710235595703 15.2308731079102\\
66.0210189819336 15.2314367294312\\
66.0710220336914 15.2320013046265\\
66.1210250854492 15.2325649261475\\
66.1710205078125 15.2331295013428\\
66.2210235595703 15.2336931228638\\
66.2710189819336 15.2342576980591\\
66.3210220336914 15.2348213195801\\
66.3710250854492 15.2353858947754\\
66.4210205078125 15.2359495162964\\
66.4710235595703 15.2365140914917\\
66.5210189819336 15.2370777130127\\
66.5710220336914 15.237642288208\\
66.6210250854492 15.238205909729\\
66.6710205078125 15.2387704849243\\
66.7210235595703 15.2393341064453\\
66.7710189819336 15.2398986816406\\
66.8210220336914 15.2404632568359\\
66.8710250854492 15.2410287857056\\
66.9210205078125 15.2415933609009\\
66.9710235595703 15.2421636581421\\
67.0210189819336 15.2427368164063\\
67.0710220336914 15.2433242797852\\
67.1210250854492 15.2439413070679\\
67.1710205078125 15.2446298599243\\
67.2210235595703 15.2454862594604\\
67.2710189819336 15.2467136383057\\
67.3210220336914 15.2486724853516\\
67.3710250854492 15.2517976760864\\
67.4210205078125 15.2560768127441\\
67.4710235595703 15.2603721618652\\
67.5210189819336 15.2666931152344\\
67.5710220336914 15.2549810409546\\
67.6210250854492 15.2541036605835\\
67.6710205078125 15.2537355422974\\
67.7210235595703 15.2534761428833\\
67.7710189819336 15.252890586853\\
67.8210220336914 15.2538366317749\\
67.8710250854492 15.2532920837402\\
67.9210205078125 15.2536020278931\\
67.9710235595703 15.2540302276611\\
68.0210189819336 15.2546396255493\\
68.0710220336914 15.2551832199097\\
68.1210250854492 15.2557573318481\\
68.1710205078125 15.2563171386719\\
68.2210235595703 15.2568826675415\\
68.2710189819336 15.2574462890625\\
68.3210220336914 15.2580108642578\\
68.3710250854492 15.2585754394531\\
68.4210205078125 15.2591390609741\\
68.4710235595703 15.2597036361694\\
68.5210189819336 15.2602682113647\\
68.5710220336914 15.2608327865601\\
68.6210250854492 15.2613964080811\\
68.6710205078125 15.2619609832764\\
68.7210235595703 15.2625255584717\\
68.7710189819336 15.263090133667\\
68.8210220336914 15.2636547088623\\
68.8710250854492 15.2642183303833\\
68.9210205078125 15.2647829055786\\
68.9710235595703 15.2653474807739\\
69.0210189819336 15.2659120559692\\
69.0710220336914 15.2664766311646\\
69.1210250854492 15.2670402526855\\
69.1710205078125 15.2676048278809\\
69.2210235595703 15.2681694030762\\
69.2710189819336 15.2687339782715\\
69.3210220336914 15.2692985534668\\
69.3710250854492 15.2698631286621\\
69.4210205078125 15.2704277038574\\
69.4710235595703 15.2709922790527\\
69.5210189819336 15.271556854248\\
69.5710220336914 15.2721214294434\\
69.6210250854492 15.2726860046387\\
69.6710205078125 15.273250579834\\
69.7210235595703 15.2738161087036\\
69.7710189819336 15.2743806838989\\
69.8210220336914 15.2749452590942\\
69.8710250854492 15.2755098342896\\
69.9210205078125 15.2760744094849\\
69.9710235595703 15.2766389846802\\
70.0210189819336 15.2772035598755\\
70.0710220336914 15.2779998779297\\
70.1210250854492 15.2786207199097\\
70.1710205078125 15.2791948318481\\
70.2210235595703 15.2797603607178\\
70.2710189819336 15.2803258895874\\
70.3210220336914 15.2808904647827\\
70.3710250854492 15.2814559936523\\
70.4210205078125 15.2820205688477\\
70.4710235595703 15.2825860977173\\
70.5210189819336 15.2831516265869\\
70.5710220336914 15.2837171554565\\
70.6210250854492 15.2842826843262\\
70.6710205078125 15.2848482131958\\
70.7210235595703 15.2854137420654\\
70.7710189819336 15.2859792709351\\
70.8210220336914 15.286545753479\\
70.8710250854492 15.2871112823486\\
70.9210205078125 15.2876768112183\\
70.9710235595703 15.2882432937622\\
71.0210189819336 15.2888097763062\\
71.0710220336914 15.2893762588501\\
71.1210250854492 15.2899417877197\\
71.1710205078125 15.290509223938\\
71.2210235595703 15.2910757064819\\
71.2710189819336 15.2916421890259\\
71.3210220336914 15.2922086715698\\
71.3710250854492 15.2927761077881\\
71.4210205078125 15.2933435440063\\
71.4710235595703 15.2939109802246\\
71.5210189819336 15.294472694397\\
71.5710220336914 15.2950096130371\\
71.6210250854492 15.2953109741211\\
71.6710205078125 15.2958812713623\\
71.7210235595703 15.2964334487915\\
71.7710189819336 15.2969970703125\\
71.8210220336914 15.2975568771362\\
71.8710250854492 15.2981176376343\\
71.9210205078125 15.2986783981323\\
71.9710235595703 15.2992391586304\\
72.0210189819336 15.2997989654541\\
72.0710220336914 15.3003587722778\\
72.1210250854492 15.3009185791016\\
72.1710205078125 15.3014783859253\\
72.2210235595703 15.302038192749\\
72.2710189819336 15.3025970458984\\
72.3210220336914 15.3031558990479\\
72.3710250854492 15.3037147521973\\
72.4210205078125 15.3042736053467\\
72.4710235595703 15.3048315048218\\
72.5210189819336 15.3053894042969\\
72.5710220336914 15.3059463500977\\
72.6210250854492 15.3065156936646\\
72.6710205078125 15.3071308135986\\
72.7210235595703 15.3077173233032\\
72.7710189819336 15.3082857131958\\
72.8210220336914 15.3089561462402\\
72.8710250854492 15.3095541000366\\
72.9210205078125 15.3101224899292\\
72.9710235595703 15.3107013702393\\
73.0210189819336 15.3112821578979\\
73.0710220336914 15.3118667602539\\
73.1210250854492 15.3124570846558\\
73.1710205078125 15.3130531311035\\
73.2210235595703 15.3136568069458\\
73.2710189819336 15.3142709732056\\
73.3210220336914 15.3148975372314\\
73.3710250854492 15.315541267395\\
73.4210205078125 15.3162088394165\\
73.4710235595703 15.3220834732056\\
73.5210189819336 15.3235168457031\\
73.5710220336914 15.3249397277832\\
73.6210250854492 15.3264799118042\\
73.6710205078125 15.3245944976807\\
73.7210235595703 15.3193454742432\\
73.7710189819336 15.3197622299194\\
73.8210220336914 15.3201322555542\\
73.8710250854492 15.3205299377441\\
73.9210205078125 15.3209657669067\\
73.9710235595703 15.3214340209961\\
74.0210189819336 15.3219337463379\\
74.0710220336914 15.3224563598633\\
74.1210250854492 15.3229923248291\\
74.1710205078125 15.3235359191895\\
74.2210235595703 15.3240842819214\\
74.2710189819336 15.3246355056763\\
74.3210220336914 15.3251876831055\\
74.3710250854492 15.3257417678833\\
74.4210205078125 15.3262977600098\\
74.4710235595703 15.3268547058105\\
74.5210189819336 15.3274126052856\\
74.5710220336914 15.3279724121094\\
74.6210250854492 15.3285293579102\\
74.6710205078125 15.3290872573853\\
74.7210235595703 15.3296461105347\\
74.7710189819336 15.3302059173584\\
74.8210220336914 15.3307666778564\\
74.8710250854492 15.3313274383545\\
74.9210205078125 15.3318891525269\\
74.9605102539063 15.3324527740479\\
75 15.332896232605\\
};
\addlegendentry{6 "};

\addplot [
color=black,
dashed
]
table[row sep=crcr]{
0 14.700608253479\\
9.99999971718069e-10 14.700608253479\\
2.49999998480632e-09 14.700608253479\\
4.74999994892755e-09 14.700608253479\\
8.1250002281763e-09 14.700608253479\\
1.31874999809156e-08 14.700608253479\\
2.0781250498203e-08 14.700608253479\\
3.21718758300449e-08 14.700608253479\\
4.92578138278077e-08 14.700608253479\\
7.48867208244519e-08 14.700608253479\\
1.13330081319418e-07 14.700608253479\\
1.70995122061868e-07 14.700608253479\\
2.57492672517401e-07 14.700608253479\\
3.8723902662241e-07 14.700608253479\\
5.81858500936505e-07 14.700608253479\\
8.73787769251066e-07 14.700608253479\\
1.31168167172291e-06 14.700608253479\\
1.9685226106958e-06 14.700608253479\\
2.95378367809462e-06 14.700608253479\\
4.43167573394021e-06 14.700608253479\\
6.64851359033491e-06 14.7006092071533\\
9.97376992017962e-06 14.7006092071533\\
1.49616553244414e-05 14.7006092071533\\
2.24434825213393e-05 14.7006101608276\\
3.36662233166862e-05 14.700611114502\\
5.05003372381907e-05 14.7006120681763\\
7.57515081204474e-05 14.7006130218506\\
0.000113628258986864 14.7006149291992\\
0.000170443381648511 14.7006158828735\\
0.000255666091106832 14.7006177902222\\
0.000383500126190484 14.7006187438965\\
0.000575251178815961 14.7006187438965\\
0.000862877757754177 14.7006168365479\\
0.00129431765526533 14.7006139755249\\
0.00194147753063589 14.7006101608276\\
0.00291221728548408 14.7006063461304\\
0.00436832662671804 14.7006044387817\\
0.00597004732117057 14.7006063461304\\
0.00773194013163447 14.700608253479\\
0.009670021943748 14.7006092071533\\
0.0118019115179777 14.7006092071533\\
0.0141469910740852 14.7006092071533\\
0.0167265776544809 14.700608253479\\
0.0195641238242388 14.700608253479\\
0.0226854234933853 14.700608253479\\
0.0261188540607691 14.700608253479\\
0.0298956278711557 14.700608253479\\
0.0340500771999359 14.700608253479\\
0.0386199727654457 14.700608253479\\
0.0436468608677387 14.700608253479\\
0.0491764321923256 14.700608253479\\
0.0552589632570744 14.700608253479\\
0.0619497485458851 14.700608253479\\
0.0693096145987511 14.700608253479\\
0.0774054601788521 14.700608253479\\
0.0863108932971954 14.7006072998047\\
0.0961068719625473 14.7006092071533\\
0.106882445514202 14.7006092071533\\
0.118735581636429 14.7006092071533\\
0.13177402317524 14.7006092071533\\
0.146116316318512 14.7006092071533\\
0.161892831325531 14.7006092071533\\
0.179247006773949 14.7006092071533\\
0.198336601257324 14.7006092071533\\
0.21933513879776 14.7006092071533\\
0.242433547973633 14.700608253479\\
0.267841786146164 14.7006092071533\\
0.29579085111618 14.7006092071533\\
0.32653483748436 14.7006092071533\\
0.360353201627731 14.7006092071533\\
0.397553414106369 14.7006092071533\\
0.438473641872406 14.7006092071533\\
0.483485877513886 14.7006092071533\\
0.532999336719513 14.7006092071533\\
0.582999348640442 14.7006092071533\\
0.632999360561371 14.7006092071533\\
0.6829993724823 14.7006092071533\\
0.732999384403229 14.7006092071533\\
0.782999336719513 14.7006092071533\\
0.832999348640442 14.7006092071533\\
0.882999360561371 14.7006092071533\\
0.9329993724823 14.7006092071533\\
0.982999384403229 14.7006092071533\\
1.03299939632416 14.700608253479\\
1.08299934864044 14.700608253479\\
1.13299930095673 14.700608253479\\
1.1829993724823 14.700608253479\\
1.23299932479858 14.700608253479\\
1.28299939632416 14.700608253479\\
1.33299934864044 14.700608253479\\
1.38299930095673 14.700608253479\\
1.4329993724823 14.700608253479\\
1.48299932479858 14.700608253479\\
1.53299939632416 14.700608253479\\
1.58299934864044 14.700608253479\\
1.63299930095673 14.700608253479\\
1.6829993724823 14.700608253479\\
1.73299932479858 14.700608253479\\
1.78299939632416 14.700608253479\\
1.83299934864044 14.700608253479\\
1.88299930095673 14.700608253479\\
1.9329993724823 14.700608253479\\
1.98299932479858 14.700608253479\\
2.03299927711487 14.700608253479\\
2.08299946784973 14.700608253479\\
2.13299942016602 14.700608253479\\
2.1829993724823 14.700608253479\\
2.23299932479858 14.700608253479\\
2.28299927711487 14.700608253479\\
2.33299946784973 14.700608253479\\
2.38299942016602 14.700608253479\\
2.4329993724823 14.700608253479\\
2.48299932479858 14.700608253479\\
2.53299927711487 14.700608253479\\
2.58299946784973 14.700608253479\\
2.63299942016602 14.700608253479\\
2.6829993724823 14.700608253479\\
2.73299932479858 14.700608253479\\
2.78299927711487 14.700608253479\\
2.83299946784973 14.700608253479\\
2.88299942016602 14.700608253479\\
2.9329993724823 14.700608253479\\
2.98299932479858 14.700608253479\\
3.03299927711487 14.700608253479\\
3.08299946784973 14.700608253479\\
3.13299942016602 14.700608253479\\
3.1829993724823 14.700608253479\\
3.23299932479858 14.700608253479\\
3.28299927711487 14.700608253479\\
3.33299946784973 14.700608253479\\
3.38299942016602 14.700608253479\\
3.4329993724823 14.700608253479\\
3.48299932479858 14.700608253479\\
3.53299927711487 14.700608253479\\
3.58299946784973 14.700608253479\\
3.63299942016602 14.700608253479\\
3.6829993724823 14.700608253479\\
3.73299932479858 14.700608253479\\
3.78299927711487 14.700608253479\\
3.83299946784973 14.700608253479\\
3.88299942016602 14.700608253479\\
3.9329993724823 14.700608253479\\
3.98299932479858 14.700608253479\\
4.03299951553345 14.700608253479\\
4.08299922943115 14.700608253479\\
4.13299942016602 14.700608253479\\
4.18299913406372 14.700608253479\\
4.23299932479858 14.700608253479\\
4.28299951553345 14.700608253479\\
4.33299922943115 14.700608253479\\
4.38299942016602 14.700608253479\\
4.43299913406372 14.700608253479\\
4.48299932479858 14.700608253479\\
4.53299951553345 14.700608253479\\
4.58299922943115 14.700608253479\\
4.63299942016602 14.700608253479\\
4.68299913406372 14.700608253479\\
4.73299932479858 14.700608253479\\
4.78299951553345 14.700608253479\\
4.83299922943115 14.700608253479\\
4.88299942016602 14.700608253479\\
4.93299913406372 14.700608253479\\
4.98299932479858 14.700608253479\\
5.03299951553345 14.700608253479\\
5.08299922943115 14.700608253479\\
5.13299942016602 14.700608253479\\
5.18299913406372 14.700608253479\\
5.23299932479858 14.700608253479\\
5.28299951553345 14.700608253479\\
5.33299922943115 14.700608253479\\
5.38299942016602 14.700608253479\\
5.43299913406372 14.700608253479\\
5.48299932479858 14.700608253479\\
5.53299951553345 14.700608253479\\
5.58299922943115 14.700608253479\\
5.63299942016602 14.700608253479\\
5.68299913406372 14.700608253479\\
5.73299932479858 14.700608253479\\
5.78299951553345 14.700608253479\\
5.83299922943115 14.700608253479\\
5.88299942016602 14.700608253479\\
5.93299913406372 14.700608253479\\
5.98299932479858 14.700608253479\\
6.03299951553345 14.700608253479\\
6.08299922943115 14.7006072998047\\
6.13299942016602 14.7006063461304\\
6.18299913406372 14.7006053924561\\
6.23299932479858 14.7006044387817\\
6.28299951553345 14.7005987167358\\
6.33299922943115 14.7006139755249\\
6.38299942016602 14.7006092071533\\
6.43299913406372 14.7006092071533\\
6.48299932479858 14.7006092071533\\
6.53299951553345 14.7006092071533\\
6.58299922943115 14.700611114502\\
6.63299942016602 14.7006101608276\\
6.68299913406372 14.7006092071533\\
6.73299932479858 14.700608253479\\
6.78299951553345 14.700608253479\\
6.83299922943115 14.700608253479\\
6.88299942016602 14.700608253479\\
6.93299913406372 14.700608253479\\
6.98299932479858 14.700608253479\\
7.03299951553345 14.700608253479\\
7.08299922943115 14.700608253479\\
7.13299942016602 14.700608253479\\
7.18299913406372 14.700608253479\\
7.23299932479858 14.700608253479\\
7.28299951553345 14.700608253479\\
7.33299922943115 14.700608253479\\
7.38299942016602 14.700608253479\\
7.43299913406372 14.700608253479\\
7.48299932479858 14.700608253479\\
7.53299951553345 14.700608253479\\
7.58299922943115 14.700608253479\\
7.63299942016602 14.700608253479\\
7.68299913406372 14.700608253479\\
7.73299932479858 14.700608253479\\
7.78299951553345 14.700608253479\\
7.83299922943115 14.700608253479\\
7.88299942016602 14.700608253479\\
7.93299913406372 14.700608253479\\
7.98299932479858 14.700608253479\\
8.03299903869629 14.700608253479\\
8.08299922943115 14.700608253479\\
8.13299942016602 14.700608253479\\
8.18299961090088 14.700608253479\\
8.23299980163574 14.700608253479\\
8.28299903869629 14.700608253479\\
8.33299922943115 14.700608253479\\
8.38299942016602 14.700608253479\\
8.43299961090088 14.700608253479\\
8.48299980163574 14.700608253479\\
8.53299903869629 14.700608253479\\
8.58299922943115 14.700608253479\\
8.63299942016602 14.700608253479\\
8.68299961090088 14.700608253479\\
8.73299980163574 14.700608253479\\
8.78299903869629 14.700608253479\\
8.83299922943115 14.700608253479\\
8.88299942016602 14.700608253479\\
8.93299961090088 14.700608253479\\
8.98299980163574 14.700608253479\\
9.03299903869629 14.700608253479\\
9.08299922943115 14.700608253479\\
9.13299942016602 14.700608253479\\
9.18299961090088 14.700608253479\\
9.23299980163574 14.700608253479\\
9.28299903869629 14.700608253479\\
9.33299922943115 14.700608253479\\
9.38299942016602 14.700608253479\\
9.43299961090088 14.700608253479\\
9.48299980163574 14.700608253479\\
9.53299903869629 14.700608253479\\
9.58299922943115 14.700608253479\\
9.63299942016602 14.700608253479\\
9.68299961090088 14.700608253479\\
9.73299980163574 14.700608253479\\
9.78299903869629 14.700608253479\\
9.83299922943115 14.700608253479\\
9.88299942016602 14.700608253479\\
9.93299961090088 14.700608253479\\
9.98299980163574 14.700608253479\\
10.0329990386963 14.700608253479\\
10.0829992294312 14.700608253479\\
10.132999420166 14.700608253479\\
10.1829996109009 14.700608253479\\
10.2329998016357 14.700608253479\\
10.2829990386963 14.700608253479\\
10.3329992294312 14.700608253479\\
10.382999420166 14.700608253479\\
10.4329996109009 14.700608253479\\
10.4829998016357 14.700608253479\\
10.5329990386963 14.700608253479\\
10.5829992294312 14.700608253479\\
10.632999420166 14.700608253479\\
10.6829996109009 14.700608253479\\
10.7329998016357 14.700608253479\\
10.7829990386963 14.700608253479\\
10.8329992294312 14.700608253479\\
10.882999420166 14.700608253479\\
10.9329996109009 14.700608253479\\
10.9829998016357 14.700608253479\\
11.0329990386963 14.700608253479\\
11.0829992294312 14.700608253479\\
11.132999420166 14.700608253479\\
11.1829996109009 14.700608253479\\
11.2329998016357 14.700608253479\\
11.2829990386963 14.700608253479\\
11.3329992294312 14.700608253479\\
11.382999420166 14.700608253479\\
11.4329996109009 14.700608253479\\
11.4829998016357 14.700608253479\\
11.5329990386963 14.700608253479\\
11.5829992294312 14.700608253479\\
11.632999420166 14.700608253479\\
11.6829996109009 14.700608253479\\
11.7329998016357 14.700608253479\\
11.7829990386963 14.700608253479\\
11.8329992294312 14.700608253479\\
11.882999420166 14.700608253479\\
11.9329996109009 14.700608253479\\
11.9829998016357 14.700608253479\\
12.0329990386963 14.700608253479\\
12.0829992294312 14.700608253479\\
12.132999420166 14.700608253479\\
12.1829996109009 14.700608253479\\
12.2329998016357 14.700608253479\\
12.2829990386963 14.700608253479\\
12.3329992294312 14.700608253479\\
12.382999420166 14.7006092071533\\
12.4329996109009 14.7006092071533\\
12.4829998016357 14.7006092071533\\
12.5329990386963 14.7006072998047\\
12.5829992294312 14.700608253479\\
12.632999420166 14.700608253479\\
12.6829996109009 14.7006092071533\\
12.7329998016357 14.700608253479\\
12.7829990386963 14.700608253479\\
12.8329992294312 14.700608253479\\
12.882999420166 14.700608253479\\
12.9329996109009 14.700608253479\\
12.9829998016357 14.700608253479\\
13.0329990386963 14.700608253479\\
13.0829992294312 14.700608253479\\
13.132999420166 14.700608253479\\
13.1829996109009 14.700608253479\\
13.2329998016357 14.700608253479\\
13.2829990386963 14.700608253479\\
13.3329992294312 14.700608253479\\
13.382999420166 14.700608253479\\
13.4329996109009 14.700608253479\\
13.4829998016357 14.700608253479\\
13.5329990386963 14.700608253479\\
13.5829992294312 14.700608253479\\
13.632999420166 14.700608253479\\
13.6829996109009 14.700608253479\\
13.7329998016357 14.700608253479\\
13.7829990386963 14.700608253479\\
13.8329992294312 14.700608253479\\
13.882999420166 14.700608253479\\
13.9329996109009 14.700608253479\\
13.9829998016357 14.700608253479\\
14.0329990386963 14.700608253479\\
14.0829992294312 14.700608253479\\
14.132999420166 14.700608253479\\
14.1829996109009 14.700608253479\\
14.2329998016357 14.700608253479\\
14.2829990386963 14.700608253479\\
14.3329992294312 14.700608253479\\
14.382999420166 14.700608253479\\
14.4329996109009 14.700608253479\\
14.4829998016357 14.700608253479\\
14.5329990386963 14.700608253479\\
14.5829992294312 14.700608253479\\
14.632999420166 14.700608253479\\
14.6829996109009 14.700608253479\\
14.7329998016357 14.700608253479\\
14.7829990386963 14.700608253479\\
14.8329992294312 14.700608253479\\
14.882999420166 14.700608253479\\
14.9329996109009 14.700608253479\\
14.9829998016357 14.700608253479\\
15.0329990386963 14.700608253479\\
15.0829992294312 14.700608253479\\
15.132999420166 14.700608253479\\
15.1829996109009 14.700608253479\\
15.2329998016357 14.700608253479\\
15.2829990386963 14.700608253479\\
15.3329992294312 14.700608253479\\
15.382999420166 14.700608253479\\
15.4329996109009 14.700608253479\\
15.4829998016357 14.700608253479\\
15.5329990386963 14.700608253479\\
15.5829992294312 14.700608253479\\
15.632999420166 14.700608253479\\
15.6829996109009 14.700608253479\\
15.7329998016357 14.700608253479\\
15.7829990386963 14.700608253479\\
15.8329992294312 14.700608253479\\
15.882999420166 14.700608253479\\
15.9329996109009 14.700608253479\\
15.9829998016357 14.700608253479\\
16.0329990386963 14.700608253479\\
16.0830001831055 14.700608253479\\
16.132999420166 14.700608253479\\
16.1829986572266 14.700608253479\\
16.2329998016357 14.700608253479\\
16.2829990386963 14.700608253479\\
16.3330001831055 14.700608253479\\
16.382999420166 14.700608253479\\
16.4329986572266 14.700608253479\\
16.4829998016357 14.700608253479\\
16.5329990386963 14.700608253479\\
16.5830001831055 14.700608253479\\
16.632999420166 14.700608253479\\
16.6829986572266 14.700608253479\\
16.7329998016357 14.700608253479\\
16.7829990386963 14.700608253479\\
16.8330001831055 14.700608253479\\
16.882999420166 14.700608253479\\
16.9329986572266 14.700608253479\\
16.9829998016357 14.700608253479\\
17.0329990386963 14.700608253479\\
17.0830001831055 14.700608253479\\
17.132999420166 14.700608253479\\
17.1829986572266 14.700608253479\\
17.2329998016357 14.700608253479\\
17.2829990386963 14.700608253479\\
17.3330001831055 14.700608253479\\
17.382999420166 14.700608253479\\
17.4329986572266 14.700608253479\\
17.4829998016357 14.700608253479\\
17.5329990386963 14.700608253479\\
17.5830001831055 14.700608253479\\
17.632999420166 14.700608253479\\
17.6829986572266 14.700608253479\\
17.7329998016357 14.700608253479\\
17.7829990386963 14.700608253479\\
17.8330001831055 14.700608253479\\
17.882999420166 14.700608253479\\
17.9329986572266 14.700608253479\\
17.9829998016357 14.700608253479\\
18.0329990386963 14.700608253479\\
18.0830001831055 14.700608253479\\
18.132999420166 14.700608253479\\
18.1829986572266 14.700608253479\\
18.2329998016357 14.700608253479\\
18.2829990386963 14.700608253479\\
18.3330001831055 14.7006072998047\\
18.382999420166 14.7006063461304\\
18.4329986572266 14.7006053924561\\
18.4829998016357 14.7006044387817\\
18.5329990386963 14.7006006240845\\
18.5830001831055 14.7006139755249\\
18.632999420166 14.700608253479\\
18.6829986572266 14.7006092071533\\
18.7329998016357 14.7006092071533\\
18.7829990386963 14.7006092071533\\
18.8330001831055 14.700611114502\\
18.882999420166 14.7006092071533\\
18.9329986572266 14.7006092071533\\
18.9829998016357 14.700608253479\\
19.0329990386963 14.700608253479\\
19.0830001831055 14.700608253479\\
19.132999420166 14.700608253479\\
19.1829986572266 14.700608253479\\
19.2329998016357 14.700608253479\\
19.2829990386963 14.700608253479\\
19.3330001831055 14.700608253479\\
19.382999420166 14.700608253479\\
19.4329986572266 14.700608253479\\
19.4829998016357 14.700608253479\\
19.5329990386963 14.700608253479\\
19.5830001831055 14.700608253479\\
19.632999420166 14.700608253479\\
19.6829986572266 14.700608253479\\
19.7329998016357 14.700608253479\\
19.7829990386963 14.700608253479\\
19.8330001831055 14.700608253479\\
19.882999420166 14.700608253479\\
19.9329986572266 14.700608253479\\
19.9829998016357 14.700608253479\\
20.0329990386963 14.700608253479\\
20.0830001831055 14.700608253479\\
20.132999420166 14.700608253479\\
20.1829986572266 14.700608253479\\
20.2329998016357 14.700608253479\\
20.2829990386963 14.700608253479\\
20.3330001831055 14.700608253479\\
20.382999420166 14.700608253479\\
20.4329986572266 14.700608253479\\
20.4829998016357 14.700608253479\\
20.5329990386963 14.700608253479\\
20.5830001831055 14.700608253479\\
20.632999420166 14.700608253479\\
20.6829986572266 14.700608253479\\
20.7329998016357 14.700608253479\\
20.7829990386963 14.700608253479\\
20.8330001831055 14.700608253479\\
20.882999420166 14.700608253479\\
20.9329986572266 14.700608253479\\
20.9829998016357 14.700608253479\\
21.0329990386963 14.700608253479\\
21.0830001831055 14.700608253479\\
21.132999420166 14.700608253479\\
21.1829986572266 14.700608253479\\
21.2329998016357 14.700608253479\\
21.2829990386963 14.700608253479\\
21.3330001831055 14.700608253479\\
21.382999420166 14.700608253479\\
21.4329986572266 14.700608253479\\
21.4829998016357 14.700608253479\\
21.5329990386963 14.700608253479\\
21.5830001831055 14.700608253479\\
21.632999420166 14.700608253479\\
21.6829986572266 14.700608253479\\
21.7329998016357 14.700608253479\\
21.7829990386963 14.700608253479\\
21.8330001831055 14.700608253479\\
21.882999420166 14.700608253479\\
21.9329986572266 14.700608253479\\
21.9829998016357 14.700608253479\\
22.0329990386963 14.700608253479\\
22.0830001831055 14.700608253479\\
22.132999420166 14.700608253479\\
22.1829986572266 14.700608253479\\
22.2329998016357 14.700608253479\\
22.2829990386963 14.700608253479\\
22.3330001831055 14.700608253479\\
22.382999420166 14.700608253479\\
22.4329986572266 14.700608253479\\
22.4829998016357 14.700608253479\\
22.5329990386963 14.700608253479\\
22.5830001831055 14.700608253479\\
22.632999420166 14.700608253479\\
22.6829986572266 14.700608253479\\
22.7329998016357 14.700608253479\\
22.7829990386963 14.700608253479\\
22.8330001831055 14.700608253479\\
22.882999420166 14.700608253479\\
22.9329986572266 14.700608253479\\
22.9829998016357 14.700608253479\\
23.0329990386963 14.700608253479\\
23.0830001831055 14.700608253479\\
23.132999420166 14.700608253479\\
23.1829986572266 14.700608253479\\
23.2329998016357 14.700608253479\\
23.2829990386963 14.700608253479\\
23.3330001831055 14.700608253479\\
23.382999420166 14.700608253479\\
23.4329986572266 14.700608253479\\
23.4829998016357 14.700608253479\\
23.5329990386963 14.700608253479\\
23.5830001831055 14.700608253479\\
23.632999420166 14.700608253479\\
23.6829986572266 14.700608253479\\
23.7329998016357 14.700608253479\\
23.7829990386963 14.700608253479\\
23.8330001831055 14.700608253479\\
23.882999420166 14.700608253479\\
23.9329986572266 14.700608253479\\
23.9829998016357 14.700608253479\\
24.0329990386963 14.700608253479\\
24.0830001831055 14.700608253479\\
24.132999420166 14.700608253479\\
24.1829986572266 14.700608253479\\
24.2329998016357 14.700608253479\\
24.2829990386963 14.700608253479\\
24.3330001831055 14.700608253479\\
24.382999420166 14.700608253479\\
24.4329986572266 14.700608253479\\
24.4829998016357 14.7006072998047\\
24.5329990386963 14.700608253479\\
24.5830001831055 14.7006072998047\\
24.632999420166 14.7006072998047\\
24.6829986572266 14.700608253479\\
24.7329998016357 14.700608253479\\
24.7829990386963 14.7006072998047\\
24.8330001831055 14.7006072998047\\
24.882999420166 14.7006092071533\\
24.9329986572266 14.700608253479\\
24.9829998016357 14.7006092071533\\
25.0329990386963 14.7006092071533\\
25.0830001831055 14.7006092071533\\
25.132999420166 14.7006092071533\\
25.1829986572266 14.7006092071533\\
25.2329998016357 14.7006092071533\\
25.2829990386963 14.7006092071533\\
25.3330001831055 14.7006092071533\\
25.382999420166 14.7006092071533\\
25.4329986572266 14.7006092071533\\
25.4829998016357 14.7006092071533\\
25.5329990386963 14.7006092071533\\
25.5830001831055 14.7006092071533\\
25.632999420166 14.7006092071533\\
25.6829986572266 14.7006092071533\\
25.7329998016357 14.7006092071533\\
25.7829990386963 14.7006092071533\\
25.8330001831055 14.7006092071533\\
25.882999420166 14.7006092071533\\
25.9329986572266 14.7006092071533\\
25.9829998016357 14.7006092071533\\
26.0329990386963 14.7006092071533\\
26.0830001831055 14.7006092071533\\
26.132999420166 14.7006092071533\\
26.1829986572266 14.7006092071533\\
26.2329998016357 14.7006092071533\\
26.2829990386963 14.7006092071533\\
26.3330001831055 14.7006092071533\\
26.382999420166 14.7006092071533\\
26.4329986572266 14.7006092071533\\
26.4829998016357 14.7006092071533\\
26.5329990386963 14.7006092071533\\
26.5830001831055 14.7006092071533\\
26.632999420166 14.7006092071533\\
26.6829986572266 14.7006092071533\\
26.7329998016357 14.7006092071533\\
26.7829990386963 14.7006092071533\\
26.8330001831055 14.7006092071533\\
26.882999420166 14.7006092071533\\
26.9329986572266 14.7006092071533\\
26.9829998016357 14.7006092071533\\
27.0329990386963 14.7006092071533\\
27.0830001831055 14.7006092071533\\
27.132999420166 14.7006092071533\\
27.1829986572266 14.7006092071533\\
27.2329998016357 14.7006092071533\\
27.2829990386963 14.7006092071533\\
27.3330001831055 14.7006092071533\\
27.382999420166 14.7006092071533\\
27.4329986572266 14.7006092071533\\
27.4829998016357 14.7006092071533\\
27.5329990386963 14.7006092071533\\
27.5830001831055 14.7006092071533\\
27.632999420166 14.7006092071533\\
27.6829986572266 14.7006092071533\\
27.7329998016357 14.7006092071533\\
27.7829990386963 14.7006092071533\\
27.8330001831055 14.7006092071533\\
27.882999420166 14.7006092071533\\
27.9329986572266 14.7006092071533\\
27.9829998016357 14.7006092071533\\
28.0329990386963 14.7006092071533\\
28.0830001831055 14.7006092071533\\
28.132999420166 14.7006092071533\\
28.1829986572266 14.7006092071533\\
28.2329998016357 14.7006092071533\\
28.2829990386963 14.7006092071533\\
28.3330001831055 14.7006092071533\\
28.382999420166 14.7006092071533\\
28.4329986572266 14.7006092071533\\
28.4829998016357 14.7006092071533\\
28.5329990386963 14.7006092071533\\
28.5830001831055 14.7006092071533\\
28.632999420166 14.7006092071533\\
28.6829986572266 14.7006092071533\\
28.7329998016357 14.7006092071533\\
28.7829990386963 14.7006092071533\\
28.8330001831055 14.7006092071533\\
28.882999420166 14.7006092071533\\
28.9329986572266 14.7006092071533\\
28.9829998016357 14.7006092071533\\
29.0329990386963 14.7006092071533\\
29.0830001831055 14.7006092071533\\
29.132999420166 14.7006092071533\\
29.1829986572266 14.7006092071533\\
29.2329998016357 14.7006092071533\\
29.2829990386963 14.7006092071533\\
29.3330001831055 14.7006092071533\\
29.382999420166 14.7006092071533\\
29.4329986572266 14.7006092071533\\
29.4829998016357 14.7006092071533\\
29.5329990386963 14.7006092071533\\
29.5830001831055 14.7006092071533\\
29.632999420166 14.7006092071533\\
29.6829986572266 14.7006092071533\\
29.7329998016357 14.7006092071533\\
29.7829990386963 14.7006092071533\\
29.8330001831055 14.7006092071533\\
29.882999420166 14.7006092071533\\
29.9329986572266 14.7006092071533\\
29.9829998016357 14.7006101608276\\
30.0329990386963 14.7006101608276\\
30.0830001831055 14.700611114502\\
30.132999420166 14.7006130218506\\
30.1829986572266 14.7006158828735\\
30.2329998016357 14.7006254196167\\
30.2829990386963 14.700644493103\\
30.3330001831055 14.7006864547729\\
30.382999420166 14.7007827758789\\
30.4329986572266 14.7010097503662\\
30.4829998016357 14.7015361785889\\
30.5329990386963 14.7027359008789\\
30.5830001831055 14.7052803039551\\
30.632999420166 14.7100868225098\\
30.6829986572266 14.717583656311\\
30.7329998016357 14.7265748977661\\
30.7829990386963 14.7365665435791\\
30.8330001831055 14.7316799163818\\
30.882999420166 14.7322463989258\\
30.9329986572266 14.7325878143311\\
30.9829998016357 14.7327337265015\\
31.0329990386963 14.7324562072754\\
31.0830001831055 14.7331066131592\\
31.132999420166 14.7325487136841\\
31.1829986572266 14.7328224182129\\
31.2329998016357 14.7329158782959\\
31.2829990386963 14.733247756958\\
31.3330001831055 14.7334814071655\\
31.382999420166 14.7337617874146\\
31.4329986572266 14.7340202331543\\
31.4829998016357 14.7342891693115\\
31.5329990386963 14.7345533370972\\
31.5830001831055 14.7348194122314\\
31.632999420166 14.7350845336914\\
31.6829986572266 14.7353496551514\\
31.7329998016357 14.7356147766113\\
31.7829990386963 14.7358808517456\\
31.8330001831055 14.7361459732056\\
31.882999420166 14.7364120483398\\
31.9329986572266 14.7366771697998\\
31.9829998016357 14.7369422912598\\
32.0330009460449 14.737208366394\\
32.0830001831055 14.7374744415283\\
32.132999420166 14.7377395629883\\
32.1829986572266 14.7380056381226\\
32.2329978942871 14.7382707595825\\
32.2830009460449 14.7385368347168\\
32.3330001831055 14.7388029098511\\
32.382999420166 14.739068031311\\
32.4329986572266 14.7393341064453\\
32.4829978942871 14.7396001815796\\
32.5330009460449 14.7398662567139\\
32.5830001831055 14.7401323318481\\
32.632999420166 14.7403984069824\\
32.6829986572266 14.7406644821167\\
32.7329978942871 14.740930557251\\
32.7830009460449 14.7411966323853\\
32.8330001831055 14.7414627075195\\
32.882999420166 14.7417297363281\\
32.9329986572266 14.7419958114624\\
32.9829978942871 14.7422618865967\\
33.0330009460449 14.7425289154053\\
33.0830001831055 14.7427949905396\\
33.132999420166 14.7430620193481\\
33.1829986572266 14.7433290481567\\
33.2329978942871 14.7435960769653\\
33.2830009460449 14.7438831329346\\
33.3330001831055 14.7444171905518\\
33.382999420166 14.7447443008423\\
33.4329986572266 14.7450742721558\\
33.4829978942871 14.7453947067261\\
33.5330009460449 14.7457160949707\\
33.5830001831055 14.746036529541\\
33.632999420166 14.7463569641113\\
33.6829986572266 14.7466764450073\\
33.7329978942871 14.7469959259033\\
33.7830009460449 14.7473154067993\\
33.8330001831055 14.7476348876953\\
33.882999420166 14.747953414917\\
33.9329986572266 14.7482719421387\\
33.9829978942871 14.7485904693604\\
34.0330009460449 14.748908996582\\
34.0830001831055 14.7492275238037\\
34.132999420166 14.7495450973511\\
34.1829986572266 14.7498636245728\\
34.2329978942871 14.7501811981201\\
34.2830009460449 14.7504987716675\\
34.3330001831055 14.7508163452148\\
34.382999420166 14.7511339187622\\
34.4329986572266 14.7514514923096\\
34.4829978942871 14.7517690658569\\
34.5330009460449 14.7520866394043\\
34.5830001831055 14.7524042129517\\
34.632999420166 14.752721786499\\
34.6829986572266 14.7530403137207\\
34.7329978942871 14.7533578872681\\
34.7830009460449 14.7536573410034\\
34.8330001831055 14.7538890838623\\
34.882999420166 14.7539710998535\\
34.9329986572266 14.7542629241943\\
34.9829978942871 14.7545232772827\\
35.0330009460449 14.7547941207886\\
35.0830001831055 14.7550611495972\\
35.132999420166 14.7553291320801\\
35.1829986572266 14.7555952072144\\
35.2329978942871 14.7558612823486\\
35.2830009460449 14.7561254501343\\
35.3330001831055 14.7563896179199\\
35.382999420166 14.7566528320313\\
35.4329986572266 14.7569150924683\\
35.4829978942871 14.757176399231\\
35.5330009460449 14.757435798645\\
35.5830001831055 14.7576951980591\\
35.632999420166 14.7579526901245\\
35.6829986572266 14.7582092285156\\
35.7329978942871 14.7584648132324\\
35.7830009460449 14.7587184906006\\
35.8330001831055 14.7589702606201\\
35.882999420166 14.759256362915\\
35.9329986572266 14.7595682144165\\
35.9829978942871 14.7598562240601\\
36.0330009460449 14.7601327896118\\
36.0830001831055 14.7605600357056\\
36.132999420166 14.7608766555786\\
36.1829986572266 14.7611665725708\\
36.2329978942871 14.7614774703979\\
36.2830009460449 14.7617931365967\\
36.3330001831055 14.7621154785156\\
36.382999420166 14.7624473571777\\
36.4329986572266 14.762788772583\\
36.4829978942871 14.7631425857544\\
36.5330009460449 14.7635126113892\\
36.5830001831055 14.7639036178589\\
36.632999420166 14.7643222808838\\
36.6829986572266 14.766261100769\\
36.7329978942871 14.7709617614746\\
36.7830009460449 14.7726602554321\\
36.8330001831055 14.7744035720825\\
36.882999420166 14.7761945724487\\
36.9329986572266 14.771183013916\\
36.9829978942871 14.7682933807373\\
37.0330009460449 14.7685632705688\\
37.0830001831055 14.7687816619873\\
37.132999420166 14.7690944671631\\
37.1829986572266 14.7694463729858\\
37.2329978942871 14.769850730896\\
37.2830009460449 14.7702941894531\\
37.3330001831055 14.7707662582397\\
37.382999420166 14.771258354187\\
37.4329986572266 14.7717618942261\\
37.4829978942871 14.7722749710083\\
37.5330009460449 14.7727947235107\\
37.5830001831055 14.7733192443848\\
37.632999420166 14.7738485336304\\
37.6829986572266 14.7743816375732\\
37.7329978942871 14.7749185562134\\
37.7830009460449 14.7754602432251\\
37.8330001831055 14.7760019302368\\
37.882999420166 14.7765455245972\\
37.9329986572266 14.7770910263062\\
37.9829978942871 14.7776393890381\\
38.0330009460449 14.7781896591187\\
38.0830001831055 14.7787418365479\\
38.132999420166 14.7792940139771\\
38.1829986572266 14.7798480987549\\
38.2329978942871 14.780403137207\\
38.2830009460449 14.7809591293335\\
38.3330001831055 14.78151512146\\
38.382999420166 14.7820720672607\\
38.4329986572266 14.7826290130615\\
38.4829978942871 14.783185005188\\
38.5330009460449 14.7837419509888\\
38.5830001831055 14.7842988967896\\
38.632999420166 14.7848558425903\\
38.6829986572266 14.7854127883911\\
38.7329978942871 14.7859687805176\\
38.7830009460449 14.7865257263184\\
38.8330001831055 14.7870826721191\\
38.882999420166 14.7876396179199\\
38.9329986572266 14.7881965637207\\
38.9829978942871 14.7887535095215\\
39.0330009460449 14.7893104553223\\
39.0830001831055 14.789867401123\\
39.132999420166 14.7904253005981\\
39.1829986572266 14.7909822463989\\
39.2329978942871 14.7915391921997\\
39.2830009460449 14.7920961380005\\
39.3330001831055 14.7926530838013\\
39.382999420166 14.7932100296021\\
39.4329986572266 14.7937679290771\\
39.4829978942871 14.7943248748779\\
39.5330009460449 14.7948818206787\\
39.5830001831055 14.7954397201538\\
39.632999420166 14.7959966659546\\
39.6829986572266 14.7965536117554\\
39.7329978942871 14.7971115112305\\
39.7830009460449 14.7976684570313\\
39.8330001831055 14.7982263565063\\
39.882999420166 14.7987842559814\\
39.9329986572266 14.7993412017822\\
39.9829978942871 14.7998991012573\\
40.0330009460449 14.8004570007324\\
40.0830001831055 14.8010139465332\\
40.132999420166 14.8015718460083\\
40.1829986572266 14.8021297454834\\
40.2329978942871 14.8026876449585\\
40.2830009460449 14.8032455444336\\
40.3330001831055 14.8038034439087\\
40.382999420166 14.8043613433838\\
40.4329986572266 14.8049192428589\\
40.4829978942871 14.805477142334\\
40.5330009460449 14.8060350418091\\
40.5830001831055 14.8065929412842\\
40.632999420166 14.8071508407593\\
40.6829986572266 14.8077087402344\\
40.7329978942871 14.8082675933838\\
40.7830009460449 14.8088388442993\\
40.8330001831055 14.809380531311\\
40.882999420166 14.8099403381348\\
40.9329986572266 14.8105010986328\\
40.9829978942871 14.8110618591309\\
41.0330009460449 14.8116226196289\\
41.0830001831055 14.812183380127\\
41.132999420166 14.812744140625\\
41.1829986572266 14.8133058547974\\
41.2329978942871 14.8138666152954\\
41.2830009460449 14.8144273757935\\
41.3330001831055 14.8149890899658\\
41.382999420166 14.8155498504639\\
41.4329986572266 14.8161106109619\\
41.4829978942871 14.8166723251343\\
41.5330009460449 14.8172330856323\\
41.5830001831055 14.8177938461304\\
41.632999420166 14.8183555603027\\
41.6829986572266 14.8189163208008\\
41.7329978942871 14.8194780349731\\
41.7830009460449 14.8200387954712\\
41.8330001831055 14.8206005096436\\
41.882999420166 14.8211612701416\\
41.9329986572266 14.821722984314\\
41.9829978942871 14.822283744812\\
42.0330009460449 14.8228454589844\\
42.0830001831055 14.8234062194824\\
42.132999420166 14.8239679336548\\
42.1829986572266 14.8245286941528\\
42.2329978942871 14.8250904083252\\
42.2830009460449 14.8256521224976\\
42.3330001831055 14.8262147903442\\
42.382999420166 14.8267774581909\\
42.4329986572266 14.8273401260376\\
42.4829978942871 14.8279085159302\\
42.5330009460449 14.8284826278687\\
42.5830001831055 14.8290729522705\\
42.632999420166 14.829704284668\\
42.6829986572266 14.8304290771484\\
42.7329978942871 14.8313684463501\\
42.7830009460449 14.8327779769897\\
42.8330001831055 14.8350563049316\\
42.882999420166 14.8385848999023\\
42.9329986572266 14.8430452346802\\
42.9829978942871 14.8469676971436\\
43.0330009460449 14.8501205444336\\
43.0830001831055 14.840033531189\\
43.132999420166 14.8395309448242\\
43.1829986572266 14.8391733169556\\
43.2329978942871 14.8389511108398\\
43.2830009460449 14.838586807251\\
43.3330001831055 14.8391971588135\\
43.382999420166 14.83873462677\\
43.4329986572266 14.8393611907959\\
43.4829978942871 14.8396902084351\\
43.5330009460449 14.8403444290161\\
43.5830001831055 14.8408632278442\\
43.632999420166 14.8414459228516\\
43.6829986572266 14.8419981002808\\
43.7329978942871 14.8425645828247\\
43.7830009460449 14.8431243896484\\
43.8330001831055 14.8436880111694\\
43.882999420166 14.8442487716675\\
43.9329986572266 14.8448114395142\\
43.9829978942871 14.8453731536865\\
44.0330009460449 14.8459358215332\\
44.0830001831055 14.8464975357056\\
44.132999420166 14.8470602035522\\
44.1829986572266 14.8476228713989\\
44.2329978942871 14.8481845855713\\
44.2830009460449 14.848747253418\\
44.3330001831055 14.8493089675903\\
44.382999420166 14.849871635437\\
44.4329986572266 14.8504343032837\\
44.4829978942871 14.8509960174561\\
44.5330009460449 14.8515586853027\\
44.5830001831055 14.8521213531494\\
44.632999420166 14.8526830673218\\
44.6829986572266 14.8532457351685\\
44.7329978942871 14.8538084030151\\
44.7830009460449 14.8543710708618\\
44.8330001831055 14.8549337387085\\
44.882999420166 14.8554954528809\\
44.9329986572266 14.8560581207275\\
44.9829978942871 14.8566207885742\\
45.0330009460449 14.8571834564209\\
45.0830001831055 14.8577461242676\\
45.132999420166 14.8583087921143\\
45.1829986572266 14.8588714599609\\
45.2329978942871 14.8594331741333\\
45.2830009460449 14.85999584198\\
45.3330001831055 14.8605585098267\\
45.382999420166 14.8611211776733\\
45.4329986572266 14.86168384552\\
45.4829978942871 14.8622465133667\\
45.5330009460449 14.862850189209\\
45.5830001831055 14.8636560440063\\
45.632999420166 14.8642253875732\\
45.6829986572266 14.8647975921631\\
45.7329978942871 14.8653593063354\\
45.7830009460449 14.8659229278564\\
45.8330001831055 14.8664855957031\\
45.882999420166 14.8670482635498\\
45.9329986572266 14.8676118850708\\
45.9829978942871 14.8681745529175\\
46.0330009460449 14.8687381744385\\
46.0830001831055 14.8693017959595\\
46.132999420166 14.8698644638062\\
46.1829986572266 14.8704280853271\\
46.2329978942871 14.8709917068481\\
46.2830009460449 14.8715562820435\\
46.3330001831055 14.8721199035645\\
46.382999420166 14.8726835250854\\
46.4329986572266 14.8732471466064\\
46.4829978942871 14.8738117218018\\
46.5330009460449 14.8743762969971\\
46.5830001831055 14.8749399185181\\
46.632999420166 14.8755044937134\\
46.6829986572266 14.8760690689087\\
46.7329978942871 14.876633644104\\
46.7830009460449 14.8771991729736\\
46.8330001831055 14.8777637481689\\
46.882999420166 14.8783292770386\\
46.9329986572266 14.8788948059082\\
46.9829978942871 14.8794603347778\\
47.0330009460449 14.8800067901611\\
47.0830001831055 14.8804759979248\\
47.132999420166 14.8808364868164\\
47.1829986572266 14.8814172744751\\
47.2329978942871 14.88196849823\\
47.2830009460449 14.882529258728\\
47.3330001831055 14.8830881118774\\
47.382999420166 14.8836469650269\\
47.4329986572266 14.884204864502\\
47.4829978942871 14.8847637176514\\
47.5330009460449 14.8853216171265\\
47.5830001831055 14.8858795166016\\
47.632999420166 14.8864374160767\\
47.6829986572266 14.8869953155518\\
47.7329978942871 14.8875522613525\\
47.7830009460449 14.8881101608276\\
47.8330001831055 14.8886671066284\\
47.882999420166 14.8892230987549\\
47.9329986572266 14.8897790908813\\
47.9829978942871 14.8903350830078\\
48.0330009460449 14.8908910751343\\
48.0830001831055 14.8914461135864\\
48.132999420166 14.8920373916626\\
48.1829986572266 14.8926439285278\\
48.2329978942871 14.8932228088379\\
48.2830009460449 14.8937902450562\\
48.3330001831055 14.8945007324219\\
48.382999420166 14.8950834274292\\
48.4329986572266 14.8956384658813\\
48.4829978942871 14.8962163925171\\
48.5330009460449 14.8967962265015\\
48.5830001831055 14.8973817825317\\
48.632999420166 14.8979730606079\\
48.6829986572266 14.8985710144043\\
48.7329978942871 14.8991765975952\\
48.7830009460449 14.8997936248779\\
48.8330001831055 14.9004249572754\\
48.882999420166 14.9010753631592\\
48.9329986572266 14.9034194946289\\
48.9829978942871 14.907977104187\\
49.0330009460449 14.9093046188354\\
49.0830001831055 14.9107570648193\\
49.132999420166 14.9124031066895\\
49.1829986572266 14.9070329666138\\
49.2329978942871 14.9047431945801\\
49.2830009460449 14.9051876068115\\
49.3330001831055 14.9055404663086\\
49.382999420166 14.9059619903564\\
49.4329986572266 14.9064044952393\\
49.4829978942871 14.9068841934204\\
49.5330009460449 14.9073925018311\\
49.5830001831055 14.9079189300537\\
49.632999420166 14.9084577560425\\
49.6829986572266 14.9090023040771\\
49.7329978942871 14.9095497131348\\
49.7830009460449 14.9100999832153\\
49.8330001831055 14.9106521606445\\
49.882999420166 14.9112062454224\\
49.9329986572266 14.9117603302002\\
49.9829978942871 14.9123163223267\\
50.0330009460449 14.9128751754761\\
50.0830001831055 14.9134321212769\\
50.132999420166 14.9139890670776\\
50.1829986572266 14.9145460128784\\
50.2329978942871 14.9151039123535\\
50.2830009460449 14.9156627655029\\
50.3330001831055 14.9162225723267\\
50.382999420166 14.9167833328247\\
50.4329986572266 14.9173440933228\\
50.4829978942871 14.9179058074951\\
50.5330009460449 14.9184684753418\\
50.5830001831055 14.9190311431885\\
50.632999420166 14.9195938110352\\
50.6829986572266 14.9201574325562\\
50.7329978942871 14.9207201004028\\
50.7830009460449 14.9212837219238\\
50.8330001831055 14.9218463897705\\
50.882999420166 14.9224100112915\\
50.9329986572266 14.9229726791382\\
50.9829978942871 14.9235363006592\\
51.0330009460449 14.9240989685059\\
51.0830001831055 14.9246625900269\\
51.132999420166 14.9252252578735\\
51.1829986572266 14.9257888793945\\
51.2329978942871 14.9263515472412\\
51.2830009460449 14.9269151687622\\
51.3330001831055 14.9274778366089\\
51.382999420166 14.9280414581299\\
51.4329986572266 14.9286041259766\\
51.4829978942871 14.9291677474976\\
51.5330009460449 14.9297313690186\\
51.5830001831055 14.9302940368652\\
51.632999420166 14.9308576583862\\
51.6829986572266 14.9314203262329\\
51.7329978942871 14.9319839477539\\
51.7830009460449 14.9325466156006\\
51.8330001831055 14.9331102371216\\
51.882999420166 14.9336729049683\\
51.9329986572266 14.9342365264893\\
51.9829978942871 14.9347991943359\\
52.0330009460449 14.9353628158569\\
52.0830001831055 14.9359254837036\\
52.132999420166 14.9364891052246\\
52.1829986572266 14.9370517730713\\
52.2329978942871 14.9376153945923\\
52.2830009460449 14.938178062439\\
52.3330001831055 14.93874168396\\
52.382999420166 14.9393043518066\\
52.4329986572266 14.9398679733276\\
52.4829978942871 14.9404315948486\\
52.5330009460449 14.9409942626953\\
52.5830001831055 14.9415578842163\\
52.632999420166 14.942120552063\\
52.6829986572266 14.942684173584\\
52.7329978942871 14.9432468414307\\
52.7830009460449 14.9438104629517\\
52.8330001831055 14.9443731307983\\
52.882999420166 14.9449367523193\\
52.9329986572266 14.9455003738403\\
52.9829978942871 14.946063041687\\
53.0330009460449 14.946626663208\\
53.0830001831055 14.9471893310547\\
53.132999420166 14.9477529525757\\
53.1829986572266 14.9483156204224\\
53.2329978942871 14.9488925933838\\
53.2830009460449 14.9494361877441\\
53.3330001831055 14.9499988555908\\
53.382999420166 14.9505615234375\\
53.4329986572266 14.9511251449585\\
53.4829978942871 14.9516878128052\\
53.5330009460449 14.9522514343262\\
53.5830001831055 14.9528150558472\\
53.632999420166 14.9533777236938\\
53.6829986572266 14.9539413452148\\
53.7329978942871 14.9545049667358\\
53.7830009460449 14.9550676345825\\
53.8330001831055 14.9556312561035\\
53.882999420166 14.9561948776245\\
53.9329986572266 14.9567584991455\\
53.9829978942871 14.9573211669922\\
54.0330009460449 14.9578847885132\\
54.0830001831055 14.9584484100342\\
54.132999420166 14.9590110778809\\
54.1829986572266 14.9595746994019\\
54.2329978942871 14.9601383209229\\
54.2830009460449 14.9607009887695\\
54.3330001831055 14.9612646102905\\
54.382999420166 14.9618282318115\\
54.4329986572266 14.9623918533325\\
54.4829978942871 14.9629554748535\\
54.5330009460449 14.9635190963745\\
54.5830001831055 14.9640827178955\\
54.632999420166 14.9646472930908\\
54.6829986572266 14.9652118682861\\
54.7329978942871 14.9657821655273\\
54.7830009460449 14.9663581848145\\
54.8330001831055 14.9669513702393\\
54.882999420166 14.967583656311\\
54.9329986572266 14.9683094024658\\
54.9829978942871 14.9692487716675\\
55.0330009460449 14.9706544876099\\
55.0830001831055 14.9729261398315\\
55.132999420166 14.9764461517334\\
55.1829986572266 14.9809064865112\\
55.2329978942871 14.9848403930664\\
55.2830009460449 14.9880790710449\\
55.3330001831055 14.9779510498047\\
55.382999420166 14.9774351119995\\
55.4329986572266 14.9770784378052\\
55.4829978942871 14.9768590927124\\
55.5330009460449 14.9764928817749\\
55.5830001831055 14.9771089553833\\
55.632999420166 14.9766473770142\\
55.6829986572266 14.9772710800171\\
55.7329978942871 14.9776010513306\\
55.7830009460449 14.9782562255859\\
55.8330001831055 14.9787769317627\\
55.882999420166 14.9793605804443\\
55.9329986572266 14.9799146652222\\
55.9829978942871 14.9804830551147\\
56.0330009460449 14.9810447692871\\
56.0830001831055 14.9816093444824\\
56.132999420166 14.9821720123291\\
56.1829986572266 14.9827356338501\\
56.2329978942871 14.9832992553711\\
56.2830009460449 14.9838628768921\\
56.3330001831055 14.9844264984131\\
56.382999420166 14.9849910736084\\
56.4329986572266 14.9855546951294\\
56.4829978942871 14.9861183166504\\
56.5330009460449 14.9866819381714\\
56.5830001831055 14.9872455596924\\
56.632999420166 14.9878091812134\\
56.6829986572266 14.9883728027344\\
56.7329978942871 14.9889373779297\\
56.7830009460449 14.9895009994507\\
56.8330001831055 14.9900646209717\\
56.882999420166 14.9906282424927\\
56.9329986572266 14.9911918640137\\
56.9829978942871 14.9917554855347\\
57.0330009460449 14.99232006073\\
57.0830001831055 14.992883682251\\
57.132999420166 14.993447303772\\
57.1829986572266 14.9940118789673\\
57.2329978942871 14.9945755004883\\
57.2830009460449 14.9951391220093\\
57.3330001831055 14.9957036972046\\
57.382999420166 14.9962673187256\\
57.4329986572266 14.9968318939209\\
57.4829978942871 14.9973955154419\\
57.5330009460449 14.9979600906372\\
57.5830001831055 14.9985237121582\\
57.632999420166 14.9990882873535\\
57.6829986572266 14.9996519088745\\
57.7329978942871 15.0002164840698\\
57.7830009460449 15.0008153915405\\
57.8330001831055 15.0016279220581\\
57.882999420166 15.0021982192993\\
57.9329986572266 15.0027713775635\\
57.9829978942871 15.0033349990845\\
58.0330009460449 15.0038995742798\\
58.0830001831055 15.0044641494751\\
58.132999420166 15.0050287246704\\
58.1829986572266 15.0055932998657\\
58.2329978942871 15.006157875061\\
58.2830009460449 15.0067224502563\\
58.3330001831055 15.007287979126\\
58.382999420166 15.0078525543213\\
58.4329986572266 15.0084171295166\\
58.4829978942871 15.0089826583862\\
58.5330009460449 15.0095481872559\\
58.5830001831055 15.0101127624512\\
58.632999420166 15.0106782913208\\
58.6829986572266 15.0112438201904\\
58.7329978942871 15.0118093490601\\
58.7830009460449 15.0123748779297\\
58.8330001831055 15.0129404067993\\
58.882999420166 15.0135068893433\\
58.9329986572266 15.0140724182129\\
58.9829978942871 15.0146389007568\\
59.0330009460449 15.0152053833008\\
59.0830001831055 15.0157709121704\\
59.132999420166 15.0163373947144\\
59.1829986572266 15.0169048309326\\
59.2329978942871 15.0174713134766\\
59.2830009460449 15.0180215835571\\
59.3330001831055 15.0184993743896\\
59.382999420166 15.018853187561\\
59.4329986572266 15.0194358825684\\
59.4829978942871 15.0199880599976\\
59.5330009460449 15.0205507278442\\
59.5830001831055 15.021110534668\\
59.632999420166 15.0216703414917\\
59.6829986572266 15.0222301483154\\
59.7329978942871 15.0227899551392\\
59.7830009460449 15.0233497619629\\
59.8330001831055 15.0239086151123\\
59.882999420166 15.0244674682617\\
59.9329986572266 15.0250263214111\\
59.9664993286133 15.0255308151245\\
60 15.0259637832642\\
60.0368499755859 15.0263442993164\\
60.0773849487305 15.0267562866211\\
60.1219749450684 15.0272083282471\\
60.1710243225098 15.0277051925659\\
60.2210235595703 15.0282411575317\\
60.2710227966309 15.0287981033325\\
60.3210220336914 15.0293550491333\\
60.371021270752 15.0299282073975\\
60.4210243225098 15.0305423736572\\
60.4710235595703 15.0311279296875\\
60.5210227966309 15.0316953659058\\
60.5710220336914 15.0323762893677\\
60.621021270752 15.0329713821411\\
60.6710243225098 15.0335369110107\\
60.7210235595703 15.0341148376465\\
60.7710227966309 15.0346956253052\\
60.8210220336914 15.0352802276611\\
60.871021270752 15.035870552063\\
60.9210243225098 15.0364665985107\\
60.9710235595703 15.0370712280273\\
61.0210227966309 15.0376853942871\\
61.0710220336914 15.0383129119873\\
61.121021270752 15.0389585494995\\
61.1710243225098 15.0398197174072\\
61.2210235595703 15.0457401275635\\
61.2710227966309 15.0469923019409\\
61.3210220336914 15.0484189987183\\
61.371021270752 15.0499782562256\\
61.4210243225098 15.0473680496216\\
61.4710235595703 15.0427207946777\\
61.5210227966309 15.0431613922119\\
61.5710220336914 15.0435190200806\\
61.621021270752 15.0439214706421\\
61.6710243225098 15.0443572998047\\
61.7210235595703 15.0448274612427\\
61.7710227966309 15.0453290939331\\
61.8210220336914 15.0458526611328\\
61.871021270752 15.0463886260986\\
61.9210243225098 15.046932220459\\
61.9710235595703 15.0474805831909\\
62.0210227966309 15.0480308532715\\
62.0710220336914 15.0485830307007\\
62.121021270752 15.0491371154785\\
62.1710243225098 15.0496921539307\\
62.2210235595703 15.0502490997314\\
62.2710227966309 15.0508069992065\\
62.3210220336914 15.051365852356\\
62.371021270752 15.0519227981567\\
62.4210243225098 15.0524797439575\\
62.4710235595703 15.0530385971069\\
62.5210227966309 15.0535974502563\\
62.5710220336914 15.0541582107544\\
62.621021270752 15.0547189712524\\
62.6710243225098 15.0552797317505\\
62.7210235595703 15.0558414459229\\
62.7710227966309 15.0564050674438\\
62.8210220336914 15.0569677352905\\
62.871021270752 15.0575313568115\\
62.9210243225098 15.0580949783325\\
62.9710235595703 15.0586585998535\\
63.0210227966309 15.0592222213745\\
63.0710220336914 15.0597858428955\\
63.121021270752 15.0603504180908\\
63.1710243225098 15.0609140396118\\
63.2210235595703 15.0614776611328\\
63.2710227966309 15.0620412826538\\
63.3210220336914 15.0626049041748\\
63.371021270752 15.0631685256958\\
63.4210243225098 15.0637321472168\\
63.4710235595703 15.0642957687378\\
63.5210227966309 15.0648593902588\\
63.5710220336914 15.0654239654541\\
63.621021270752 15.0659875869751\\
63.6710243225098 15.0665512084961\\
63.7210235595703 15.0671148300171\\
63.7710227966309 15.0676784515381\\
63.8210220336914 15.0682420730591\\
63.871021270752 15.0688056945801\\
63.9210243225098 15.0693693161011\\
63.9710235595703 15.0699329376221\\
64.0210189819336 15.0704965591431\\
64.0710220336914 15.0710601806641\\
64.1210250854492 15.0716238021851\\
64.1710205078125 15.0721883773804\\
64.2210235595703 15.0727519989014\\
64.2710189819336 15.0733156204224\\
64.3210220336914 15.0738792419434\\
64.3710250854492 15.0744428634644\\
64.4210205078125 15.0750064849854\\
64.4710235595703 15.0755701065063\\
64.5210189819336 15.0761337280273\\
64.5710220336914 15.0766973495483\\
64.6210250854492 15.0772609710693\\
64.6710205078125 15.0778245925903\\
64.7210235595703 15.0783882141113\\
64.7710189819336 15.0789518356323\\
64.8210220336914 15.0795154571533\\
64.8710250854492 15.0800800323486\\
64.9210205078125 15.0806436538696\\
64.9710235595703 15.0812072753906\\
65.0210189819336 15.0817708969116\\
65.0710220336914 15.0823345184326\\
65.1210250854492 15.0828981399536\\
65.1710205078125 15.0834617614746\\
65.2210235595703 15.0840253829956\\
65.2710189819336 15.0845890045166\\
65.3210220336914 15.0851526260376\\
65.3710250854492 15.0857162475586\\
65.4210205078125 15.0862798690796\\
65.4710235595703 15.0868434906006\\
65.5210189819336 15.0874080657959\\
65.5710220336914 15.0879716873169\\
65.6210250854492 15.0885353088379\\
65.6710205078125 15.0891122817993\\
65.7210235595703 15.089656829834\\
65.7710189819336 15.0902194976807\\
65.8210220336914 15.0907831192017\\
65.8710250854492 15.0913467407227\\
65.9210205078125 15.0919103622437\\
65.9710235595703 15.0924739837646\\
66.0210189819336 15.0930376052856\\
66.0710220336914 15.093602180481\\
66.1210250854492 15.094165802002\\
66.1710205078125 15.0947294235229\\
66.2210235595703 15.0952930450439\\
66.2710189819336 15.0958566665649\\
66.3210220336914 15.0964202880859\\
66.3710250854492 15.0969848632813\\
66.4210205078125 15.0975484848022\\
66.4710235595703 15.0981121063232\\
66.5210189819336 15.0986757278442\\
66.5710220336914 15.0992393493652\\
66.6210250854492 15.0998029708862\\
66.6710205078125 15.1003675460815\\
66.7210235595703 15.1009311676025\\
66.7710189819336 15.1014957427979\\
66.8210220336914 15.1020593643188\\
66.8710250854492 15.1026248931885\\
66.9210205078125 15.1031894683838\\
66.9710235595703 15.1037588119507\\
67.0210189819336 15.1043319702148\\
67.0710220336914 15.1049184799194\\
67.1210250854492 15.1055355072021\\
67.1710205078125 15.1062250137329\\
67.2210235595703 15.1070804595947\\
67.2710189819336 15.1083078384399\\
67.3210220336914 15.1102666854858\\
67.3710250854492 15.1133909225464\\
67.4210205078125 15.1176700592041\\
67.4710235595703 15.1219644546509\\
67.5210189819336 15.12828540802\\
67.5710220336914 15.1165723800659\\
67.6210250854492 15.1156949996948\\
67.6710205078125 15.1153268814087\\
67.7210235595703 15.1150665283203\\
67.7710189819336 15.1144819259644\\
67.8210220336914 15.1154270172119\\
67.8710250854492 15.1148824691772\\
67.9210205078125 15.1151924133301\\
67.9710235595703 15.1156206130981\\
68.0210189819336 15.116229057312\\
68.0710220336914 15.1167726516724\\
68.1210250854492 15.1173458099365\\
68.1710205078125 15.1179056167603\\
68.2210235595703 15.1184711456299\\
68.2710189819336 15.1190347671509\\
68.3210220336914 15.1195993423462\\
68.3710250854492 15.1201629638672\\
68.4210205078125 15.1207265853882\\
68.4710235595703 15.1212911605835\\
68.5210189819336 15.1218547821045\\
68.5710220336914 15.1224193572998\\
68.6210250854492 15.1229829788208\\
68.6710205078125 15.1235475540161\\
68.7210235595703 15.1241111755371\\
68.7710189819336 15.1246757507324\\
68.8210220336914 15.1252393722534\\
68.8710250854492 15.1258039474487\\
68.9210205078125 15.1263675689697\\
68.9710235595703 15.126932144165\\
69.0210189819336 15.1274967193604\\
69.0710220336914 15.1280603408813\\
69.1210250854492 15.1286239624023\\
69.1710205078125 15.1291885375977\\
69.2210235595703 15.129753112793\\
69.2710189819336 15.130316734314\\
69.3210220336914 15.1308813095093\\
69.3710250854492 15.1314458847046\\
69.4210205078125 15.1320104598999\\
69.4710235595703 15.1325740814209\\
69.5210189819336 15.1331386566162\\
69.5710220336914 15.1337032318115\\
69.6210250854492 15.1342678070068\\
69.6710205078125 15.1348323822021\\
69.7210235595703 15.1353960037231\\
69.7710189819336 15.1359605789185\\
69.8210220336914 15.1365251541138\\
69.8710250854492 15.1370897293091\\
69.9210205078125 15.1376543045044\\
69.9710235595703 15.1382188796997\\
70.0210189819336 15.138783454895\\
70.0710220336914 15.1395788192749\\
70.1210250854492 15.1401987075806\\
70.1710205078125 15.140772819519\\
70.2210235595703 15.1413383483887\\
70.2710189819336 15.1419038772583\\
70.3210220336914 15.1424684524536\\
70.3710250854492 15.1430330276489\\
70.4210205078125 15.1435985565186\\
70.4710235595703 15.1441631317139\\
70.5210189819336 15.1447286605835\\
70.5710220336914 15.1452941894531\\
70.6210250854492 15.1458587646484\\
70.6710205078125 15.1464242935181\\
70.7210235595703 15.1469898223877\\
70.7710189819336 15.1475553512573\\
70.8210220336914 15.148120880127\\
70.8710250854492 15.1486864089966\\
70.9210205078125 15.1492519378662\\
70.9710235595703 15.1498184204102\\
71.0210189819336 15.1503839492798\\
71.0710220336914 15.1509504318237\\
71.1210250854492 15.1515169143677\\
71.1710205078125 15.1520824432373\\
71.2210235595703 15.1526489257813\\
71.2710189819336 15.1532154083252\\
71.3210220336914 15.1537828445435\\
71.3710250854492 15.1543493270874\\
71.4210205078125 15.1549167633057\\
71.4710235595703 15.1554832458496\\
71.5210189819336 15.156044960022\\
71.5710220336914 15.1565818786621\\
71.6210250854492 15.1568832397461\\
71.6710205078125 15.157452583313\\
71.7210235595703 15.1580047607422\\
71.7710189819336 15.1585683822632\\
71.8210220336914 15.1591281890869\\
71.8710250854492 15.159688949585\\
71.9210205078125 15.1602487564087\\
71.9710235595703 15.1608095169067\\
72.0210189819336 15.1613693237305\\
72.0710220336914 15.1619291305542\\
72.1210250854492 15.1624889373779\\
72.1710205078125 15.1630477905273\\
72.2210235595703 15.1636075973511\\
72.2710189819336 15.1641664505005\\
72.3210220336914 15.1647253036499\\
72.3710250854492 15.165283203125\\
72.4210205078125 15.1658420562744\\
72.4710235595703 15.1663999557495\\
72.5210189819336 15.1669569015503\\
72.5710220336914 15.1675148010254\\
72.6210250854492 15.1680841445923\\
72.6710205078125 15.1686983108521\\
72.7210235595703 15.1692848205566\\
72.7710189819336 15.1698532104492\\
72.8210220336914 15.1705236434937\\
72.8710250854492 15.1711206436157\\
72.9210205078125 15.1716890335083\\
72.9710235595703 15.1722679138184\\
73.0210189819336 15.1728477478027\\
73.0710220336914 15.173433303833\\
73.1210250854492 15.1740226745605\\
73.1710205078125 15.1746187210083\\
73.2210235595703 15.1752223968506\\
73.2710189819336 15.175835609436\\
73.3210220336914 15.1764621734619\\
73.3710250854492 15.1771059036255\\
73.4210205078125 15.1777725219727\\
73.4710235595703 15.183648109436\\
73.5210189819336 15.1850805282593\\
73.5710220336914 15.1865034103394\\
73.6210250854492 15.1880435943604\\
73.6710205078125 15.1861581802368\\
73.7210235595703 15.1809091567993\\
73.7710189819336 15.1813259124756\\
73.8210220336914 15.181694984436\\
73.8710250854492 15.182092666626\\
73.9210205078125 15.1825284957886\\
73.9710235595703 15.1829967498779\\
74.0210189819336 15.1834964752197\\
74.0710220336914 15.1840181350708\\
74.1210250854492 15.1845541000366\\
74.1710205078125 15.185097694397\\
74.2210235595703 15.1856460571289\\
74.2710189819336 15.1861963272095\\
74.3210220336914 15.186749458313\\
74.3710250854492 15.1873035430908\\
74.4210205078125 15.187858581543\\
74.4710235595703 15.1884145736694\\
74.5210189819336 15.1889724731445\\
74.5710220336914 15.1895322799683\\
74.6210250854492 15.1900901794434\\
74.6710205078125 15.1906471252441\\
74.7210235595703 15.1912059783936\\
74.7710189819336 15.1917657852173\\
74.8210220336914 15.192325592041\\
74.8710250854492 15.1928873062134\\
74.9210205078125 15.1934480667114\\
74.9605102539063 15.1940116882324\\
75 15.1944551467896\\
};
\addlegendentry{10 "};

\addplot [
color=black,
dotted
]
table[row sep=crcr]{
0 14.7005214691162\\
9.99999971718069e-10 14.7005214691162\\
2.49999998480632e-09 14.7005214691162\\
4.74999994892755e-09 14.7005214691162\\
8.1250002281763e-09 14.7005214691162\\
1.31874999809156e-08 14.7005214691162\\
2.0781250498203e-08 14.7005214691162\\
3.21718758300449e-08 14.7005214691162\\
4.92578138278077e-08 14.7005214691162\\
7.48867208244519e-08 14.7005214691162\\
1.13330081319418e-07 14.7005214691162\\
1.70995122061868e-07 14.7005214691162\\
2.57492672517401e-07 14.7005214691162\\
3.8723902662241e-07 14.7005214691162\\
5.81858500936505e-07 14.7005214691162\\
8.73787769251066e-07 14.7005214691162\\
1.31168167172291e-06 14.7005214691162\\
1.9685226106958e-06 14.7005214691162\\
2.95378367809462e-06 14.7005214691162\\
4.43167573394021e-06 14.7005214691162\\
6.64851359033491e-06 14.7005214691162\\
9.97376992017962e-06 14.7005214691162\\
1.49616553244414e-05 14.7005224227905\\
2.24434825213393e-05 14.7005224227905\\
3.36662233166862e-05 14.7005233764648\\
5.05003372381907e-05 14.7005233764648\\
7.57515081204474e-05 14.7005243301392\\
0.000113628258986864 14.7005252838135\\
0.000170443381648511 14.7005262374878\\
0.000255666091106832 14.7005271911621\\
0.000383500126190484 14.7005281448364\\
0.000575251178815961 14.7005290985107\\
0.000862877757754177 14.7005290985107\\
0.00129431765526533 14.7005281448364\\
0.00194147753063589 14.7005243301392\\
0.00291221728548408 14.7005195617676\\
0.00436832662671804 14.7005176544189\\
0.00597004732117057 14.7005195617676\\
0.00773194013163447 14.7005214691162\\
0.009670021943748 14.7005224227905\\
0.0118019115179777 14.7005224227905\\
0.0141469910740852 14.7005214691162\\
0.0167265776544809 14.7005214691162\\
0.0195641238242388 14.7005214691162\\
0.0226854234933853 14.7005214691162\\
0.0261188540607691 14.7005214691162\\
0.0298956278711557 14.7005214691162\\
0.0340500771999359 14.7005214691162\\
0.0386199727654457 14.7005214691162\\
0.0436468608677387 14.7005214691162\\
0.0491764321923256 14.7005214691162\\
0.0552589632570744 14.7005214691162\\
0.0619497485458851 14.7005214691162\\
0.0693096145987511 14.7005214691162\\
0.0774054601788521 14.7005214691162\\
0.0863108932971954 14.7005205154419\\
0.0961068719625473 14.7005224227905\\
0.106882445514202 14.7005224227905\\
0.118735581636429 14.7005214691162\\
0.13177402317524 14.7005214691162\\
0.146116316318512 14.7005214691162\\
0.161892831325531 14.7005214691162\\
0.179247006773949 14.7005214691162\\
0.198336601257324 14.7005214691162\\
0.21933513879776 14.7005214691162\\
0.242433547973633 14.7005214691162\\
0.267841786146164 14.7005214691162\\
0.29579085111618 14.7005214691162\\
0.32653483748436 14.7005214691162\\
0.360353201627731 14.7005214691162\\
0.397553414106369 14.7005214691162\\
0.438473641872406 14.7005214691162\\
0.483485877513886 14.7005214691162\\
0.532999336719513 14.7005214691162\\
0.582999348640442 14.7005214691162\\
0.632999360561371 14.7005214691162\\
0.6829993724823 14.7005214691162\\
0.732999384403229 14.7005214691162\\
0.782999336719513 14.7005214691162\\
0.832999348640442 14.7005214691162\\
0.882999360561371 14.7005214691162\\
0.9329993724823 14.7005214691162\\
0.982999384403229 14.7005214691162\\
1.03299939632416 14.7005214691162\\
1.08299934864044 14.7005214691162\\
1.13299930095673 14.7005214691162\\
1.1829993724823 14.7005214691162\\
1.23299932479858 14.7005214691162\\
1.28299939632416 14.7005214691162\\
1.33299934864044 14.7005214691162\\
1.38299930095673 14.7005214691162\\
1.4329993724823 14.7005214691162\\
1.48299932479858 14.7005214691162\\
1.53299939632416 14.7005214691162\\
1.58299934864044 14.7005214691162\\
1.63299930095673 14.7005214691162\\
1.6829993724823 14.7005214691162\\
1.73299932479858 14.7005214691162\\
1.78299939632416 14.7005214691162\\
1.83299934864044 14.7005214691162\\
1.88299930095673 14.7005214691162\\
1.9329993724823 14.7005214691162\\
1.98299932479858 14.7005214691162\\
2.03299927711487 14.7005214691162\\
2.08299946784973 14.7005214691162\\
2.13299942016602 14.7005214691162\\
2.1829993724823 14.7005214691162\\
2.23299932479858 14.7005214691162\\
2.28299927711487 14.7005214691162\\
2.33299946784973 14.7005214691162\\
2.38299942016602 14.7005214691162\\
2.4329993724823 14.7005214691162\\
2.48299932479858 14.7005214691162\\
2.53299927711487 14.7005214691162\\
2.58299946784973 14.7005214691162\\
2.63299942016602 14.7005214691162\\
2.6829993724823 14.7005214691162\\
2.73299932479858 14.7005214691162\\
2.78299927711487 14.7005214691162\\
2.83299946784973 14.7005214691162\\
2.88299942016602 14.7005214691162\\
2.9329993724823 14.7005214691162\\
2.98299932479858 14.7005214691162\\
3.03299927711487 14.7005214691162\\
3.08299946784973 14.7005214691162\\
3.13299942016602 14.7005214691162\\
3.1829993724823 14.7005214691162\\
3.23299932479858 14.7005214691162\\
3.28299927711487 14.7005214691162\\
3.33299946784973 14.7005214691162\\
3.38299942016602 14.7005214691162\\
3.4329993724823 14.7005214691162\\
3.48299932479858 14.7005214691162\\
3.53299927711487 14.7005214691162\\
3.58299946784973 14.7005214691162\\
3.63299942016602 14.7005214691162\\
3.6829993724823 14.7005214691162\\
3.73299932479858 14.7005214691162\\
3.78299927711487 14.7005214691162\\
3.83299946784973 14.7005214691162\\
3.88299942016602 14.7005214691162\\
3.9329993724823 14.7005214691162\\
3.98299932479858 14.7005214691162\\
4.03299951553345 14.7005214691162\\
4.08299922943115 14.7005214691162\\
4.13299942016602 14.7005214691162\\
4.18299913406372 14.7005214691162\\
4.23299932479858 14.7005214691162\\
4.28299951553345 14.7005214691162\\
4.33299922943115 14.7005214691162\\
4.38299942016602 14.7005214691162\\
4.43299913406372 14.7005214691162\\
4.48299932479858 14.7005214691162\\
4.53299951553345 14.7005214691162\\
4.58299922943115 14.7005214691162\\
4.63299942016602 14.7005214691162\\
4.68299913406372 14.7005214691162\\
4.73299932479858 14.7005214691162\\
4.78299951553345 14.7005214691162\\
4.83299922943115 14.7005214691162\\
4.88299942016602 14.7005214691162\\
4.93299913406372 14.7005214691162\\
4.98299932479858 14.7005214691162\\
5.03299951553345 14.7005214691162\\
5.08299922943115 14.7005214691162\\
5.13299942016602 14.7005214691162\\
5.18299913406372 14.7005214691162\\
5.23299932479858 14.7005214691162\\
5.28299951553345 14.7005214691162\\
5.33299922943115 14.7005214691162\\
5.38299942016602 14.7005214691162\\
5.43299913406372 14.7005214691162\\
5.48299932479858 14.7005214691162\\
5.53299951553345 14.7005214691162\\
5.58299922943115 14.7005214691162\\
5.63299942016602 14.7005214691162\\
5.68299913406372 14.7005214691162\\
5.73299932479858 14.7005214691162\\
5.78299951553345 14.7005214691162\\
5.83299922943115 14.7005214691162\\
5.88299942016602 14.7005214691162\\
5.93299913406372 14.7005214691162\\
5.98299932479858 14.7005214691162\\
6.03299951553345 14.7005214691162\\
6.08299922943115 14.7005205154419\\
6.13299942016602 14.7005195617676\\
6.18299913406372 14.7005186080933\\
6.23299932479858 14.7005176544189\\
6.28299951553345 14.7005138397217\\
6.33299922943115 14.7005262374878\\
6.38299942016602 14.7005224227905\\
6.43299913406372 14.7005214691162\\
6.48299932479858 14.7005224227905\\
6.53299951553345 14.7005224227905\\
6.58299922943115 14.7005243301392\\
6.63299942016602 14.7005233764648\\
6.68299913406372 14.7005224227905\\
6.73299932479858 14.7005214691162\\
6.78299951553345 14.7005214691162\\
6.83299922943115 14.7005214691162\\
6.88299942016602 14.7005214691162\\
6.93299913406372 14.7005214691162\\
6.98299932479858 14.7005214691162\\
7.03299951553345 14.7005214691162\\
7.08299922943115 14.7005214691162\\
7.13299942016602 14.7005214691162\\
7.18299913406372 14.7005214691162\\
7.23299932479858 14.7005214691162\\
7.28299951553345 14.7005214691162\\
7.33299922943115 14.7005214691162\\
7.38299942016602 14.7005214691162\\
7.43299913406372 14.7005214691162\\
7.48299932479858 14.7005214691162\\
7.53299951553345 14.7005214691162\\
7.58299922943115 14.7005214691162\\
7.63299942016602 14.7005214691162\\
7.68299913406372 14.7005214691162\\
7.73299932479858 14.7005214691162\\
7.78299951553345 14.7005214691162\\
7.83299922943115 14.7005214691162\\
7.88299942016602 14.7005214691162\\
7.93299913406372 14.7005214691162\\
7.98299932479858 14.7005214691162\\
8.03299903869629 14.7005214691162\\
8.08299922943115 14.7005214691162\\
8.13299942016602 14.7005214691162\\
8.18299961090088 14.7005214691162\\
8.23299980163574 14.7005214691162\\
8.28299903869629 14.7005214691162\\
8.33299922943115 14.7005214691162\\
8.38299942016602 14.7005214691162\\
8.43299961090088 14.7005214691162\\
8.48299980163574 14.7005214691162\\
8.53299903869629 14.7005214691162\\
8.58299922943115 14.7005214691162\\
8.63299942016602 14.7005214691162\\
8.68299961090088 14.7005214691162\\
8.73299980163574 14.7005214691162\\
8.78299903869629 14.7005214691162\\
8.83299922943115 14.7005214691162\\
8.88299942016602 14.7005214691162\\
8.93299961090088 14.7005214691162\\
8.98299980163574 14.7005214691162\\
9.03299903869629 14.7005214691162\\
9.08299922943115 14.7005214691162\\
9.13299942016602 14.7005214691162\\
9.18299961090088 14.7005214691162\\
9.23299980163574 14.7005214691162\\
9.28299903869629 14.7005214691162\\
9.33299922943115 14.7005214691162\\
9.38299942016602 14.7005214691162\\
9.43299961090088 14.7005214691162\\
9.48299980163574 14.7005214691162\\
9.53299903869629 14.7005214691162\\
9.58299922943115 14.7005214691162\\
9.63299942016602 14.7005214691162\\
9.68299961090088 14.7005214691162\\
9.73299980163574 14.7005214691162\\
9.78299903869629 14.7005214691162\\
9.83299922943115 14.7005214691162\\
9.88299942016602 14.7005214691162\\
9.93299961090088 14.7005214691162\\
9.98299980163574 14.7005214691162\\
10.0329990386963 14.7005214691162\\
10.0829992294312 14.7005214691162\\
10.132999420166 14.7005214691162\\
10.1829996109009 14.7005214691162\\
10.2329998016357 14.7005214691162\\
10.2829990386963 14.7005214691162\\
10.3329992294312 14.7005214691162\\
10.382999420166 14.7005214691162\\
10.4329996109009 14.7005214691162\\
10.4829998016357 14.7005214691162\\
10.5329990386963 14.7005214691162\\
10.5829992294312 14.7005214691162\\
10.632999420166 14.7005214691162\\
10.6829996109009 14.7005214691162\\
10.7329998016357 14.7005214691162\\
10.7829990386963 14.7005214691162\\
10.8329992294312 14.7005214691162\\
10.882999420166 14.7005214691162\\
10.9329996109009 14.7005214691162\\
10.9829998016357 14.7005214691162\\
11.0329990386963 14.7005214691162\\
11.0829992294312 14.7005214691162\\
11.132999420166 14.7005214691162\\
11.1829996109009 14.7005214691162\\
11.2329998016357 14.7005214691162\\
11.2829990386963 14.7005214691162\\
11.3329992294312 14.7005214691162\\
11.382999420166 14.7005214691162\\
11.4329996109009 14.7005214691162\\
11.4829998016357 14.7005214691162\\
11.5329990386963 14.7005214691162\\
11.5829992294312 14.7005214691162\\
11.632999420166 14.7005214691162\\
11.6829996109009 14.7005214691162\\
11.7329998016357 14.7005214691162\\
11.7829990386963 14.7005214691162\\
11.8329992294312 14.7005214691162\\
11.882999420166 14.7005214691162\\
11.9329996109009 14.7005214691162\\
11.9829998016357 14.7005214691162\\
12.0329990386963 14.7005214691162\\
12.0829992294312 14.7005214691162\\
12.132999420166 14.7005214691162\\
12.1829996109009 14.7005214691162\\
12.2329998016357 14.7005214691162\\
12.2829990386963 14.7005214691162\\
12.3329992294312 14.7005214691162\\
12.382999420166 14.7005224227905\\
12.4329996109009 14.7005224227905\\
12.4829998016357 14.7005214691162\\
12.5329990386963 14.7005205154419\\
12.5829992294312 14.7005214691162\\
12.632999420166 14.7005214691162\\
12.6829996109009 14.7005224227905\\
12.7329998016357 14.7005214691162\\
12.7829990386963 14.7005214691162\\
12.8329992294312 14.7005214691162\\
12.882999420166 14.7005214691162\\
12.9329996109009 14.7005214691162\\
12.9829998016357 14.7005214691162\\
13.0329990386963 14.7005214691162\\
13.0829992294312 14.7005214691162\\
13.132999420166 14.7005214691162\\
13.1829996109009 14.7005214691162\\
13.2329998016357 14.7005214691162\\
13.2829990386963 14.7005214691162\\
13.3329992294312 14.7005214691162\\
13.382999420166 14.7005214691162\\
13.4329996109009 14.7005214691162\\
13.4829998016357 14.7005214691162\\
13.5329990386963 14.7005214691162\\
13.5829992294312 14.7005214691162\\
13.632999420166 14.7005214691162\\
13.6829996109009 14.7005214691162\\
13.7329998016357 14.7005214691162\\
13.7829990386963 14.7005214691162\\
13.8329992294312 14.7005214691162\\
13.882999420166 14.7005214691162\\
13.9329996109009 14.7005214691162\\
13.9829998016357 14.7005214691162\\
14.0329990386963 14.7005214691162\\
14.0829992294312 14.7005214691162\\
14.132999420166 14.7005214691162\\
14.1829996109009 14.7005214691162\\
14.2329998016357 14.7005214691162\\
14.2829990386963 14.7005214691162\\
14.3329992294312 14.7005214691162\\
14.382999420166 14.7005214691162\\
14.4329996109009 14.7005214691162\\
14.4829998016357 14.7005214691162\\
14.5329990386963 14.7005214691162\\
14.5829992294312 14.7005214691162\\
14.632999420166 14.7005214691162\\
14.6829996109009 14.7005214691162\\
14.7329998016357 14.7005214691162\\
14.7829990386963 14.7005214691162\\
14.8329992294312 14.7005214691162\\
14.882999420166 14.7005214691162\\
14.9329996109009 14.7005214691162\\
14.9829998016357 14.7005214691162\\
15.0329990386963 14.7005214691162\\
15.0829992294312 14.7005214691162\\
15.132999420166 14.7005214691162\\
15.1829996109009 14.7005214691162\\
15.2329998016357 14.7005214691162\\
15.2829990386963 14.7005214691162\\
15.3329992294312 14.7005214691162\\
15.382999420166 14.7005214691162\\
15.4329996109009 14.7005214691162\\
15.4829998016357 14.7005214691162\\
15.5329990386963 14.7005214691162\\
15.5829992294312 14.7005214691162\\
15.632999420166 14.7005214691162\\
15.6829996109009 14.7005214691162\\
15.7329998016357 14.7005214691162\\
15.7829990386963 14.7005214691162\\
15.8329992294312 14.7005214691162\\
15.882999420166 14.7005214691162\\
15.9329996109009 14.7005214691162\\
15.9829998016357 14.7005214691162\\
16.0329990386963 14.7005214691162\\
16.0830001831055 14.7005214691162\\
16.132999420166 14.7005214691162\\
16.1829986572266 14.7005214691162\\
16.2329998016357 14.7005214691162\\
16.2829990386963 14.7005214691162\\
16.3330001831055 14.7005214691162\\
16.382999420166 14.7005214691162\\
16.4329986572266 14.7005214691162\\
16.4829998016357 14.7005214691162\\
16.5329990386963 14.7005214691162\\
16.5830001831055 14.7005214691162\\
16.632999420166 14.7005214691162\\
16.6829986572266 14.7005214691162\\
16.7329998016357 14.7005214691162\\
16.7829990386963 14.7005214691162\\
16.8330001831055 14.7005214691162\\
16.882999420166 14.7005214691162\\
16.9329986572266 14.7005214691162\\
16.9829998016357 14.7005214691162\\
17.0329990386963 14.7005214691162\\
17.0830001831055 14.7005214691162\\
17.132999420166 14.7005214691162\\
17.1829986572266 14.7005214691162\\
17.2329998016357 14.7005214691162\\
17.2829990386963 14.7005214691162\\
17.3330001831055 14.7005214691162\\
17.382999420166 14.7005214691162\\
17.4329986572266 14.7005214691162\\
17.4829998016357 14.7005214691162\\
17.5329990386963 14.7005214691162\\
17.5830001831055 14.7005214691162\\
17.632999420166 14.7005214691162\\
17.6829986572266 14.7005214691162\\
17.7329998016357 14.7005214691162\\
17.7829990386963 14.7005214691162\\
17.8330001831055 14.7005214691162\\
17.882999420166 14.7005214691162\\
17.9329986572266 14.7005214691162\\
17.9829998016357 14.7005214691162\\
18.0329990386963 14.7005214691162\\
18.0830001831055 14.7005214691162\\
18.132999420166 14.7005214691162\\
18.1829986572266 14.7005214691162\\
18.2329998016357 14.7005214691162\\
18.2829990386963 14.7005214691162\\
18.3330001831055 14.7005205154419\\
18.382999420166 14.7005195617676\\
18.4329986572266 14.7005186080933\\
18.4829998016357 14.7005176544189\\
18.5329990386963 14.700514793396\\
18.5830001831055 14.7005262374878\\
18.632999420166 14.7005214691162\\
18.6829986572266 14.7005224227905\\
18.7329998016357 14.7005224227905\\
18.7829990386963 14.7005224227905\\
18.8330001831055 14.7005233764648\\
18.882999420166 14.7005224227905\\
18.9329986572266 14.7005224227905\\
18.9829998016357 14.7005214691162\\
19.0329990386963 14.7005214691162\\
19.0830001831055 14.7005214691162\\
19.132999420166 14.7005214691162\\
19.1829986572266 14.7005214691162\\
19.2329998016357 14.7005214691162\\
19.2829990386963 14.7005214691162\\
19.3330001831055 14.7005214691162\\
19.382999420166 14.7005214691162\\
19.4329986572266 14.7005214691162\\
19.4829998016357 14.7005214691162\\
19.5329990386963 14.7005214691162\\
19.5830001831055 14.7005214691162\\
19.632999420166 14.7005214691162\\
19.6829986572266 14.7005214691162\\
19.7329998016357 14.7005214691162\\
19.7829990386963 14.7005214691162\\
19.8330001831055 14.7005214691162\\
19.882999420166 14.7005214691162\\
19.9329986572266 14.7005214691162\\
19.9829998016357 14.7005214691162\\
20.0329990386963 14.7005214691162\\
20.0830001831055 14.7005214691162\\
20.132999420166 14.7005214691162\\
20.1829986572266 14.7005214691162\\
20.2329998016357 14.7005214691162\\
20.2829990386963 14.7005214691162\\
20.3330001831055 14.7005214691162\\
20.382999420166 14.7005214691162\\
20.4329986572266 14.7005214691162\\
20.4829998016357 14.7005214691162\\
20.5329990386963 14.7005214691162\\
20.5830001831055 14.7005214691162\\
20.632999420166 14.7005214691162\\
20.6829986572266 14.7005214691162\\
20.7329998016357 14.7005214691162\\
20.7829990386963 14.7005214691162\\
20.8330001831055 14.7005214691162\\
20.882999420166 14.7005214691162\\
20.9329986572266 14.7005214691162\\
20.9829998016357 14.7005214691162\\
21.0329990386963 14.7005214691162\\
21.0830001831055 14.7005214691162\\
21.132999420166 14.7005214691162\\
21.1829986572266 14.7005214691162\\
21.2329998016357 14.7005214691162\\
21.2829990386963 14.7005214691162\\
21.3330001831055 14.7005214691162\\
21.382999420166 14.7005214691162\\
21.4329986572266 14.7005214691162\\
21.4829998016357 14.7005214691162\\
21.5329990386963 14.7005214691162\\
21.5830001831055 14.7005214691162\\
21.632999420166 14.7005214691162\\
21.6829986572266 14.7005214691162\\
21.7329998016357 14.7005214691162\\
21.7829990386963 14.7005214691162\\
21.8330001831055 14.7005214691162\\
21.882999420166 14.7005214691162\\
21.9329986572266 14.7005214691162\\
21.9829998016357 14.7005214691162\\
22.0329990386963 14.7005214691162\\
22.0830001831055 14.7005214691162\\
22.132999420166 14.7005214691162\\
22.1829986572266 14.7005214691162\\
22.2329998016357 14.7005214691162\\
22.2829990386963 14.7005214691162\\
22.3330001831055 14.7005214691162\\
22.382999420166 14.7005214691162\\
22.4329986572266 14.7005214691162\\
22.4829998016357 14.7005214691162\\
22.5329990386963 14.7005214691162\\
22.5830001831055 14.7005214691162\\
22.632999420166 14.7005214691162\\
22.6829986572266 14.7005214691162\\
22.7329998016357 14.7005214691162\\
22.7829990386963 14.7005214691162\\
22.8330001831055 14.7005214691162\\
22.882999420166 14.7005214691162\\
22.9329986572266 14.7005214691162\\
22.9829998016357 14.7005214691162\\
23.0329990386963 14.7005214691162\\
23.0830001831055 14.7005214691162\\
23.132999420166 14.7005214691162\\
23.1829986572266 14.7005214691162\\
23.2329998016357 14.7005214691162\\
23.2829990386963 14.7005214691162\\
23.3330001831055 14.7005214691162\\
23.382999420166 14.7005214691162\\
23.4329986572266 14.7005214691162\\
23.4829998016357 14.7005214691162\\
23.5329990386963 14.7005214691162\\
23.5830001831055 14.7005214691162\\
23.632999420166 14.7005214691162\\
23.6829986572266 14.7005214691162\\
23.7329998016357 14.7005214691162\\
23.7829990386963 14.7005214691162\\
23.8330001831055 14.7005214691162\\
23.882999420166 14.7005214691162\\
23.9329986572266 14.7005214691162\\
23.9829998016357 14.7005214691162\\
24.0329990386963 14.7005214691162\\
24.0830001831055 14.7005214691162\\
24.132999420166 14.7005214691162\\
24.1829986572266 14.7005214691162\\
24.2329998016357 14.7005214691162\\
24.2829990386963 14.7005214691162\\
24.3330001831055 14.7005214691162\\
24.382999420166 14.7005214691162\\
24.4329986572266 14.7005214691162\\
24.4829998016357 14.7005214691162\\
24.5329990386963 14.7005214691162\\
24.5830001831055 14.7005214691162\\
24.632999420166 14.7005214691162\\
24.6829986572266 14.7005224227905\\
24.7329998016357 14.7005224227905\\
24.7829990386963 14.7005214691162\\
24.8330001831055 14.7005214691162\\
24.882999420166 14.7005224227905\\
24.9329986572266 14.7005214691162\\
24.9829998016357 14.7005214691162\\
25.0329990386963 14.7005214691162\\
25.0830001831055 14.7005214691162\\
25.132999420166 14.7005214691162\\
25.1829986572266 14.7005214691162\\
25.2329998016357 14.7005214691162\\
25.2829990386963 14.7005214691162\\
25.3330001831055 14.7005214691162\\
25.382999420166 14.7005214691162\\
25.4329986572266 14.7005214691162\\
25.4829998016357 14.7005214691162\\
25.5329990386963 14.7005214691162\\
25.5830001831055 14.7005214691162\\
25.632999420166 14.7005214691162\\
25.6829986572266 14.7005214691162\\
25.7329998016357 14.7005214691162\\
25.7829990386963 14.7005214691162\\
25.8330001831055 14.7005214691162\\
25.882999420166 14.7005214691162\\
25.9329986572266 14.7005214691162\\
25.9829998016357 14.7005214691162\\
26.0329990386963 14.7005214691162\\
26.0830001831055 14.7005214691162\\
26.132999420166 14.7005214691162\\
26.1829986572266 14.7005214691162\\
26.2329998016357 14.7005214691162\\
26.2829990386963 14.7005214691162\\
26.3330001831055 14.7005214691162\\
26.382999420166 14.7005214691162\\
26.4329986572266 14.7005214691162\\
26.4829998016357 14.7005214691162\\
26.5329990386963 14.7005214691162\\
26.5830001831055 14.7005214691162\\
26.632999420166 14.7005214691162\\
26.6829986572266 14.7005214691162\\
26.7329998016357 14.7005214691162\\
26.7829990386963 14.7005214691162\\
26.8330001831055 14.7005214691162\\
26.882999420166 14.7005214691162\\
26.9329986572266 14.7005214691162\\
26.9829998016357 14.7005214691162\\
27.0329990386963 14.7005214691162\\
27.0830001831055 14.7005214691162\\
27.132999420166 14.7005214691162\\
27.1829986572266 14.7005214691162\\
27.2329998016357 14.7005214691162\\
27.2829990386963 14.7005214691162\\
27.3330001831055 14.7005214691162\\
27.382999420166 14.7005214691162\\
27.4329986572266 14.7005214691162\\
27.4829998016357 14.7005214691162\\
27.5329990386963 14.7005214691162\\
27.5830001831055 14.7005214691162\\
27.632999420166 14.7005214691162\\
27.6829986572266 14.7005214691162\\
27.7329998016357 14.7005214691162\\
27.7829990386963 14.7005214691162\\
27.8330001831055 14.7005214691162\\
27.882999420166 14.7005214691162\\
27.9329986572266 14.7005214691162\\
27.9829998016357 14.7005214691162\\
28.0329990386963 14.7005214691162\\
28.0830001831055 14.7005214691162\\
28.132999420166 14.7005214691162\\
28.1829986572266 14.7005214691162\\
28.2329998016357 14.7005214691162\\
28.2829990386963 14.7005214691162\\
28.3330001831055 14.7005214691162\\
28.382999420166 14.7005214691162\\
28.4329986572266 14.7005214691162\\
28.4829998016357 14.7005214691162\\
28.5329990386963 14.7005214691162\\
28.5830001831055 14.7005214691162\\
28.632999420166 14.7005214691162\\
28.6829986572266 14.7005214691162\\
28.7329998016357 14.7005214691162\\
28.7829990386963 14.7005214691162\\
28.8330001831055 14.7005214691162\\
28.882999420166 14.7005214691162\\
28.9329986572266 14.7005214691162\\
28.9829998016357 14.7005214691162\\
29.0329990386963 14.7005214691162\\
29.0830001831055 14.7005214691162\\
29.132999420166 14.7005214691162\\
29.1829986572266 14.7005214691162\\
29.2329998016357 14.7005214691162\\
29.2829990386963 14.7005214691162\\
29.3330001831055 14.7005214691162\\
29.382999420166 14.7005214691162\\
29.4329986572266 14.7005214691162\\
29.4829998016357 14.7005214691162\\
29.5329990386963 14.7005214691162\\
29.5830001831055 14.7005214691162\\
29.632999420166 14.7005214691162\\
29.6829986572266 14.7005214691162\\
29.7329998016357 14.7005214691162\\
29.7829990386963 14.7005214691162\\
29.8330001831055 14.7005214691162\\
29.882999420166 14.7005214691162\\
29.9329986572266 14.7005214691162\\
29.9829998016357 14.7005214691162\\
30.0329990386963 14.7005214691162\\
30.0830001831055 14.7005214691162\\
30.132999420166 14.7005214691162\\
30.1829986572266 14.7005214691162\\
30.2329998016357 14.7005214691162\\
30.2829990386963 14.7005214691162\\
30.3330001831055 14.7005214691162\\
30.382999420166 14.7005214691162\\
30.4329986572266 14.7005205154419\\
30.4829998016357 14.7005205154419\\
30.5329990386963 14.7005214691162\\
30.5830001831055 14.7005205154419\\
30.632999420166 14.7005195617676\\
30.6829986572266 14.7005176544189\\
30.7329998016357 14.7005176544189\\
30.7829990386963 14.7005176544189\\
30.8330001831055 14.7005262374878\\
30.882999420166 14.7005214691162\\
30.9329986572266 14.7005224227905\\
30.9829998016357 14.7005224227905\\
31.0329990386963 14.7005233764648\\
31.0830001831055 14.7005224227905\\
31.132999420166 14.7005224227905\\
31.1829986572266 14.7005214691162\\
31.2329998016357 14.7005214691162\\
31.2829990386963 14.7005214691162\\
31.3330001831055 14.7005214691162\\
31.382999420166 14.7005214691162\\
31.4329986572266 14.7005214691162\\
31.4829998016357 14.7005214691162\\
31.5329990386963 14.7005214691162\\
31.5830001831055 14.7005214691162\\
31.632999420166 14.7005214691162\\
31.6829986572266 14.7005214691162\\
31.7329998016357 14.7005214691162\\
31.7829990386963 14.7005214691162\\
31.8330001831055 14.7005214691162\\
31.882999420166 14.7005214691162\\
31.9329986572266 14.7005214691162\\
31.9829998016357 14.7005214691162\\
32.0330009460449 14.7005214691162\\
32.0830001831055 14.7005214691162\\
32.132999420166 14.7005214691162\\
32.1829986572266 14.7005214691162\\
32.2329978942871 14.7005214691162\\
32.2830009460449 14.7005214691162\\
32.3330001831055 14.7005214691162\\
32.382999420166 14.7005214691162\\
32.4329986572266 14.7005214691162\\
32.4829978942871 14.7005214691162\\
32.5330009460449 14.7005214691162\\
32.5830001831055 14.7005214691162\\
32.632999420166 14.7005214691162\\
32.6829986572266 14.7005214691162\\
32.7329978942871 14.7005214691162\\
32.7830009460449 14.7005214691162\\
32.8330001831055 14.7005214691162\\
32.882999420166 14.7005214691162\\
32.9329986572266 14.7005214691162\\
32.9829978942871 14.7005214691162\\
33.0330009460449 14.7005214691162\\
33.0830001831055 14.7005214691162\\
33.132999420166 14.7005214691162\\
33.1829986572266 14.7005214691162\\
33.2329978942871 14.7005214691162\\
33.2830009460449 14.7005214691162\\
33.3330001831055 14.7005214691162\\
33.382999420166 14.7005214691162\\
33.4329986572266 14.7005214691162\\
33.4829978942871 14.7005214691162\\
33.5330009460449 14.7005214691162\\
33.5830001831055 14.7005214691162\\
33.632999420166 14.7005214691162\\
33.6829986572266 14.7005214691162\\
33.7329978942871 14.7005214691162\\
33.7830009460449 14.7005214691162\\
33.8330001831055 14.7005214691162\\
33.882999420166 14.7005214691162\\
33.9329986572266 14.7005214691162\\
33.9829978942871 14.7005214691162\\
34.0330009460449 14.7005214691162\\
34.0830001831055 14.7005214691162\\
34.132999420166 14.7005214691162\\
34.1829986572266 14.7005214691162\\
34.2329978942871 14.7005214691162\\
34.2830009460449 14.7005214691162\\
34.3330001831055 14.7005214691162\\
34.382999420166 14.7005214691162\\
34.4329986572266 14.7005214691162\\
34.4829978942871 14.7005214691162\\
34.5330009460449 14.7005214691162\\
34.5830001831055 14.7005214691162\\
34.632999420166 14.7005214691162\\
34.6829986572266 14.7005214691162\\
34.7329978942871 14.7005214691162\\
34.7830009460449 14.7005214691162\\
34.8330001831055 14.7005214691162\\
34.882999420166 14.7005214691162\\
34.9329986572266 14.7005214691162\\
34.9829978942871 14.7005214691162\\
35.0330009460449 14.7005214691162\\
35.0830001831055 14.7005214691162\\
35.132999420166 14.7005214691162\\
35.1829986572266 14.7005214691162\\
35.2329978942871 14.7005214691162\\
35.2830009460449 14.7005214691162\\
35.3330001831055 14.7005214691162\\
35.382999420166 14.7005214691162\\
35.4329986572266 14.7005214691162\\
35.4829978942871 14.7005214691162\\
35.5330009460449 14.7005214691162\\
35.5830001831055 14.7005214691162\\
35.632999420166 14.7005214691162\\
35.6829986572266 14.7005214691162\\
35.7329978942871 14.7005214691162\\
35.7830009460449 14.7005214691162\\
35.8330001831055 14.7005214691162\\
35.882999420166 14.7005214691162\\
35.9329986572266 14.7005214691162\\
35.9829978942871 14.7005214691162\\
36.0330009460449 14.7005214691162\\
36.0830001831055 14.7005214691162\\
36.132999420166 14.7005214691162\\
36.1829986572266 14.7005214691162\\
36.2329978942871 14.7005214691162\\
36.2830009460449 14.7005214691162\\
36.3330001831055 14.7005214691162\\
36.382999420166 14.7005214691162\\
36.4329986572266 14.7005214691162\\
36.4829978942871 14.7005214691162\\
36.5330009460449 14.7005214691162\\
36.5830001831055 14.7005214691162\\
36.632999420166 14.7005214691162\\
36.6829986572266 14.7005214691162\\
36.7329978942871 14.7005205154419\\
36.7830009460449 14.7005205154419\\
36.8330001831055 14.7005205154419\\
36.882999420166 14.7005205154419\\
36.9329986572266 14.7005214691162\\
36.9829978942871 14.7005205154419\\
37.0330009460449 14.7005205154419\\
37.0830001831055 14.7005205154419\\
37.132999420166 14.7005224227905\\
37.1829986572266 14.7005214691162\\
37.2329978942871 14.7005214691162\\
37.2830009460449 14.7005224227905\\
37.3330001831055 14.7005224227905\\
37.382999420166 14.7005224227905\\
37.4329986572266 14.7005224227905\\
37.4829978942871 14.7005224227905\\
37.5330009460449 14.7005224227905\\
37.5830001831055 14.7005224227905\\
37.632999420166 14.7005224227905\\
37.6829986572266 14.7005224227905\\
37.7329978942871 14.7005224227905\\
37.7830009460449 14.7005224227905\\
37.8330001831055 14.7005224227905\\
37.882999420166 14.7005224227905\\
37.9329986572266 14.7005224227905\\
37.9829978942871 14.7005224227905\\
38.0330009460449 14.7005224227905\\
38.0830001831055 14.7005224227905\\
38.132999420166 14.7005224227905\\
38.1829986572266 14.7005224227905\\
38.2329978942871 14.7005224227905\\
38.2830009460449 14.7005224227905\\
38.3330001831055 14.7005224227905\\
38.382999420166 14.7005224227905\\
38.4329986572266 14.7005224227905\\
38.4829978942871 14.7005224227905\\
38.5330009460449 14.7005224227905\\
38.5830001831055 14.7005224227905\\
38.632999420166 14.7005224227905\\
38.6829986572266 14.7005224227905\\
38.7329978942871 14.7005224227905\\
38.7830009460449 14.7005224227905\\
38.8330001831055 14.7005224227905\\
38.882999420166 14.7005224227905\\
38.9329986572266 14.7005224227905\\
38.9829978942871 14.7005224227905\\
39.0330009460449 14.7005224227905\\
39.0830001831055 14.7005224227905\\
39.132999420166 14.7005224227905\\
39.1829986572266 14.7005224227905\\
39.2329978942871 14.7005224227905\\
39.2830009460449 14.7005224227905\\
39.3330001831055 14.7005224227905\\
39.382999420166 14.7005224227905\\
39.4329986572266 14.7005224227905\\
39.4829978942871 14.7005224227905\\
39.5330009460449 14.7005224227905\\
39.5830001831055 14.7005224227905\\
39.632999420166 14.7005224227905\\
39.6829986572266 14.7005224227905\\
39.7329978942871 14.7005224227905\\
39.7830009460449 14.7005224227905\\
39.8330001831055 14.7005224227905\\
39.882999420166 14.7005224227905\\
39.9329986572266 14.7005224227905\\
39.9829978942871 14.7005224227905\\
40.0330009460449 14.7005224227905\\
40.0830001831055 14.7005224227905\\
40.132999420166 14.7005224227905\\
40.1829986572266 14.7005224227905\\
40.2329978942871 14.7005224227905\\
40.2830009460449 14.7005224227905\\
40.3330001831055 14.7005224227905\\
40.382999420166 14.7005224227905\\
40.4329986572266 14.7005224227905\\
40.4829978942871 14.7005224227905\\
40.5330009460449 14.7005224227905\\
40.5830001831055 14.7005224227905\\
40.632999420166 14.7005224227905\\
40.6829986572266 14.7005224227905\\
40.7329978942871 14.7005224227905\\
40.7830009460449 14.7005224227905\\
40.8330001831055 14.7005224227905\\
40.882999420166 14.7005224227905\\
40.9329986572266 14.7005224227905\\
40.9829978942871 14.7005224227905\\
41.0330009460449 14.7005224227905\\
41.0830001831055 14.7005224227905\\
41.132999420166 14.7005224227905\\
41.1829986572266 14.7005224227905\\
41.2329978942871 14.7005224227905\\
41.2830009460449 14.7005224227905\\
41.3330001831055 14.7005224227905\\
41.382999420166 14.7005224227905\\
41.4329986572266 14.7005224227905\\
41.4829978942871 14.7005224227905\\
41.5330009460449 14.7005224227905\\
41.5830001831055 14.7005224227905\\
41.632999420166 14.7005224227905\\
41.6829986572266 14.7005224227905\\
41.7329978942871 14.7005224227905\\
41.7830009460449 14.7005224227905\\
41.8330001831055 14.7005224227905\\
41.882999420166 14.7005224227905\\
41.9329986572266 14.7005224227905\\
41.9829978942871 14.7005224227905\\
42.0330009460449 14.7005224227905\\
42.0830001831055 14.7005224227905\\
42.132999420166 14.7005224227905\\
42.1829986572266 14.7005224227905\\
42.2329978942871 14.7005233764648\\
42.2830009460449 14.7005233764648\\
42.3330001831055 14.7005243301392\\
42.382999420166 14.7005271911621\\
42.4329986572266 14.7005300521851\\
42.4829978942871 14.7005405426025\\
42.5330009460449 14.7005605697632\\
42.5830001831055 14.7006063461304\\
42.632999420166 14.7007122039795\\
42.6829986572266 14.7009592056274\\
42.7329978942871 14.7015323638916\\
42.7830009460449 14.7028322219849\\
42.8330001831055 14.7055625915527\\
42.882999420166 14.710638999939\\
42.9329986572266 14.7183713912964\\
42.9829978942871 14.7273731231689\\
43.0330009460449 14.7363405227661\\
43.0830001831055 14.7316150665283\\
43.132999420166 14.7322130203247\\
43.1829986572266 14.7325210571289\\
43.2329978942871 14.7326583862305\\
43.2830009460449 14.7324333190918\\
43.3330001831055 14.7329702377319\\
43.382999420166 14.7324342727661\\
43.4329986572266 14.7327842712402\\
43.4829978942871 14.7328500747681\\
43.5330009460449 14.7331953048706\\
43.5830001831055 14.7334232330322\\
43.632999420166 14.7337064743042\\
43.6829986572266 14.7339639663696\\
43.7329978942871 14.7342329025269\\
43.7830009460449 14.7344970703125\\
43.8330001831055 14.7347631454468\\
43.882999420166 14.7350282669067\\
43.9329986572266 14.735294342041\\
43.9829978942871 14.735559463501\\
44.0330009460449 14.7358255386353\\
44.0830001831055 14.7360906600952\\
44.132999420166 14.7363567352295\\
44.1829986572266 14.7366218566895\\
44.2329978942871 14.7368879318237\\
44.2830009460449 14.7371530532837\\
44.3330001831055 14.737419128418\\
44.382999420166 14.7376852035522\\
44.4329986572266 14.7379512786865\\
44.4829978942871 14.7382164001465\\
44.5330009460449 14.7384824752808\\
44.5830001831055 14.738748550415\\
44.632999420166 14.7390146255493\\
44.6829986572266 14.7392807006836\\
44.7329978942871 14.7395467758179\\
44.7830009460449 14.7398128509521\\
44.8330001831055 14.7400789260864\\
44.882999420166 14.7403450012207\\
44.9329986572266 14.740611076355\\
44.9829978942871 14.7408771514893\\
45.0330009460449 14.7411441802979\\
45.0830001831055 14.7414102554321\\
45.132999420166 14.7416763305664\\
45.1829986572266 14.741943359375\\
45.2329978942871 14.7422094345093\\
45.2830009460449 14.7424764633179\\
45.3330001831055 14.7427434921265\\
45.382999420166 14.7430105209351\\
45.4329986572266 14.7432765960693\\
45.4829978942871 14.7435436248779\\
45.5330009460449 14.7438526153564\\
45.5830001831055 14.7443704605103\\
45.632999420166 14.7446956634521\\
45.6829986572266 14.7450246810913\\
45.7329978942871 14.7453451156616\\
45.7830009460449 14.7456655502319\\
45.8330001831055 14.7459859848022\\
45.882999420166 14.7463064193726\\
45.9329986572266 14.7466259002686\\
45.9829978942871 14.7469453811646\\
46.0330009460449 14.7472648620605\\
46.0830001831055 14.7475833892822\\
46.132999420166 14.7479019165039\\
46.1829986572266 14.7482204437256\\
46.2329978942871 14.7485389709473\\
46.2830009460449 14.7488574981689\\
46.3330001831055 14.7491750717163\\
46.382999420166 14.7494926452637\\
46.4329986572266 14.749810218811\\
46.4829978942871 14.7501277923584\\
46.5330009460449 14.7504453659058\\
46.5830001831055 14.7507629394531\\
46.632999420166 14.7510805130005\\
46.6829986572266 14.7513980865479\\
46.7329978942871 14.7517156600952\\
46.7830009460449 14.7520332336426\\
46.8330001831055 14.7523508071899\\
46.882999420166 14.7526683807373\\
46.9329986572266 14.7529859542847\\
46.9829978942871 14.753303527832\\
47.0330009460449 14.7536029815674\\
47.0830001831055 14.7538213729858\\
47.132999420166 14.7539186477661\\
47.1829986572266 14.7542085647583\\
47.2329978942871 14.7544708251953\\
47.2830009460449 14.7547416687012\\
47.3330001831055 14.7550096511841\\
47.382999420166 14.755277633667\\
47.4329986572266 14.7555446624756\\
47.4829978942871 14.7558107376099\\
47.5330009460449 14.7560758590698\\
47.5830001831055 14.7563409805298\\
47.632999420166 14.7566051483154\\
47.6829986572266 14.7568674087524\\
47.7329978942871 14.7571296691895\\
47.7830009460449 14.7573909759521\\
47.8330001831055 14.7576513290405\\
47.882999420166 14.7579097747803\\
47.9329986572266 14.75816822052\\
47.9829978942871 14.7584247589111\\
48.0330009460449 14.7586793899536\\
48.0830001831055 14.7589340209961\\
48.132999420166 14.759222984314\\
48.1829986572266 14.7595310211182\\
48.2329978942871 14.7598171234131\\
48.2830009460449 14.7600936889648\\
48.3330001831055 14.7605152130127\\
48.382999420166 14.7608289718628\\
48.4329986572266 14.7611169815063\\
48.4829978942871 14.7614269256592\\
48.5330009460449 14.7617406845093\\
48.5830001831055 14.7620630264282\\
48.632999420166 14.7623920440674\\
48.6829986572266 14.7627325057983\\
48.7329978942871 14.7630853652954\\
48.7830009460449 14.7634544372559\\
48.8330001831055 14.7638444900513\\
48.882999420166 14.7642621994019\\
48.9329986572266 14.7663831710815\\
48.9829978942871 14.7709302902222\\
49.0330009460449 14.772629737854\\
49.0830001831055 14.7743663787842\\
49.132999420166 14.7761554718018\\
49.1829986572266 14.770712852478\\
49.2329978942871 14.7682104110718\\
49.2830009460449 14.7684812545776\\
49.3330001831055 14.768702507019\\
49.382999420166 14.7690172195435\\
49.4329986572266 14.7693710327148\\
49.4829978942871 14.7697763442993\\
49.5330009460449 14.7702217102051\\
49.5830001831055 14.770694732666\\
49.632999420166 14.7711868286133\\
49.6829986572266 14.7716903686523\\
49.7329978942871 14.7722034454346\\
49.7830009460449 14.772723197937\\
49.8330001831055 14.773247718811\\
49.882999420166 14.7737770080566\\
49.9329986572266 14.7743101119995\\
49.9829978942871 14.7748470306396\\
50.0330009460449 14.7753877639771\\
50.0830001831055 14.7759304046631\\
50.132999420166 14.7764730453491\\
50.1829986572266 14.7770185470581\\
50.2329978942871 14.77756690979\\
50.2830009460449 14.7781162261963\\
50.3330001831055 14.7786684036255\\
50.382999420166 14.7792205810547\\
50.4329986572266 14.7797746658325\\
50.4829978942871 14.7803297042847\\
50.5330009460449 14.7808847427368\\
50.5830001831055 14.7814407348633\\
50.632999420166 14.7819976806641\\
50.6829986572266 14.7825536727905\\
50.7329978942871 14.7831106185913\\
50.7830009460449 14.7836675643921\\
50.8330001831055 14.7842245101929\\
50.882999420166 14.7847805023193\\
50.9329986572266 14.7853374481201\\
50.9829978942871 14.7858943939209\\
51.0330009460449 14.7864513397217\\
51.0830001831055 14.7870082855225\\
51.132999420166 14.7875652313232\\
51.1829986572266 14.788122177124\\
51.2329978942871 14.7886791229248\\
51.2830009460449 14.7892351150513\\
51.3330001831055 14.7897920608521\\
51.382999420166 14.7903490066528\\
51.4329986572266 14.7909059524536\\
51.4829978942871 14.7914638519287\\
51.5330009460449 14.7920207977295\\
51.5830001831055 14.7925777435303\\
51.632999420166 14.7931346893311\\
51.6829986572266 14.7936916351318\\
51.7329978942871 14.7942485809326\\
51.7830009460449 14.7948055267334\\
51.8330001831055 14.7953634262085\\
51.882999420166 14.7959203720093\\
51.9329986572266 14.7964773178101\\
51.9829978942871 14.7970352172852\\
52.0330009460449 14.7975921630859\\
52.0830001831055 14.7981491088867\\
52.132999420166 14.7987070083618\\
52.1829986572266 14.7992639541626\\
52.2329978942871 14.7998218536377\\
52.2830009460449 14.8003797531128\\
52.3330001831055 14.8009366989136\\
52.382999420166 14.8014945983887\\
52.4329986572266 14.8020515441895\\
52.4829978942871 14.8026094436646\\
52.5330009460449 14.8031673431396\\
52.5830001831055 14.8037252426147\\
52.632999420166 14.8042821884155\\
52.6829986572266 14.8048400878906\\
52.7329978942871 14.8053979873657\\
52.7830009460449 14.8059558868408\\
52.8330001831055 14.8065137863159\\
52.882999420166 14.807071685791\\
52.9329986572266 14.8076295852661\\
52.9829978942871 14.8081874847412\\
53.0330009460449 14.8087463378906\\
53.0830001831055 14.8093042373657\\
53.132999420166 14.8098621368408\\
53.1829986572266 14.8104200363159\\
53.2329978942871 14.8109922409058\\
53.2830009460449 14.8115329742432\\
53.3330001831055 14.8120927810669\\
53.382999420166 14.8126535415649\\
53.4329986572266 14.813214302063\\
53.4829978942871 14.813775062561\\
53.5330009460449 14.8143358230591\\
53.5830001831055 14.8148965835571\\
53.632999420166 14.8154582977295\\
53.6829986572266 14.8160190582275\\
53.7329978942871 14.8165798187256\\
53.7830009460449 14.8171405792236\\
53.8330001831055 14.8177013397217\\
53.882999420166 14.818263053894\\
53.9329986572266 14.8188238143921\\
53.9829978942871 14.8193845748901\\
54.0330009460449 14.8199453353882\\
54.0830001831055 14.8205070495605\\
54.132999420166 14.8210678100586\\
54.1829986572266 14.8216285705566\\
54.2329978942871 14.822190284729\\
54.2830009460449 14.8227510452271\\
54.3330001831055 14.8233127593994\\
54.382999420166 14.8238735198975\\
54.4329986572266 14.8244342803955\\
54.4829978942871 14.8249959945679\\
54.5330009460449 14.8255577087402\\
54.5830001831055 14.8261194229126\\
54.632999420166 14.8266820907593\\
54.6829986572266 14.827244758606\\
54.7329978942871 14.8278131484985\\
54.7830009460449 14.8283863067627\\
54.8330001831055 14.8289775848389\\
54.882999420166 14.8296070098877\\
54.9329986572266 14.8303308486938\\
54.9829978942871 14.8312683105469\\
55.0330009460449 14.8326721191406\\
55.0830001831055 14.8349418640137\\
55.132999420166 14.8384599685669\\
55.1829986572266 14.8429174423218\\
55.2329978942871 14.8468503952026\\
55.2830009460449 14.8500862121582\\
55.3330001831055 14.8399543762207\\
55.382999420166 14.8394393920898\\
55.4329986572266 14.8390808105469\\
55.4829978942871 14.8388595581055\\
55.5330009460449 14.8384914398193\\
55.5830001831055 14.8391056060791\\
55.632999420166 14.8386421203613\\
55.6829986572266 14.8392639160156\\
55.7329978942871 14.8395919799805\\
55.7830009460449 14.8402462005615\\
55.8330001831055 14.8407649993896\\
55.882999420166 14.8413467407227\\
55.9329986572266 14.8418989181519\\
55.9829978942871 14.8424654006958\\
56.0330009460449 14.8430252075195\\
56.0830001831055 14.8435878753662\\
56.132999420166 14.8441495895386\\
56.1829986572266 14.8447113037109\\
56.2329978942871 14.8452730178833\\
56.2830009460449 14.84583568573\\
56.3330001831055 14.8463973999023\\
56.382999420166 14.8469591140747\\
56.4329986572266 14.8475217819214\\
56.4829978942871 14.8480834960938\\
56.5330009460449 14.8486461639404\\
56.5830001831055 14.8492078781128\\
56.632999420166 14.8497705459595\\
56.6829986572266 14.8503322601318\\
56.7329978942871 14.8508949279785\\
56.7830009460449 14.8514566421509\\
56.8330001831055 14.8520183563232\\
56.882999420166 14.8525810241699\\
56.9329986572266 14.8531427383423\\
56.9829978942871 14.853705406189\\
57.0330009460449 14.8542680740356\\
57.0830001831055 14.854829788208\\
57.132999420166 14.8553924560547\\
57.1829986572266 14.8559551239014\\
57.2329978942871 14.8565168380737\\
57.2830009460449 14.8570795059204\\
57.3330001831055 14.8576421737671\\
57.382999420166 14.8582038879395\\
57.4329986572266 14.8587665557861\\
57.4829978942871 14.8593292236328\\
57.5330009460449 14.8598909378052\\
57.5830001831055 14.8604536056519\\
57.632999420166 14.8610162734985\\
57.6829986572266 14.8615789413452\\
57.7329978942871 14.8621406555176\\
57.7830009460449 14.8627395629883\\
57.8330001831055 14.8635501861572\\
57.882999420166 14.8641176223755\\
57.9329986572266 14.8646898269653\\
57.9829978942871 14.8652515411377\\
58.0330009460449 14.8658142089844\\
58.0830001831055 14.8663778305054\\
58.132999420166 14.8669404983521\\
58.1829986572266 14.8675031661987\\
58.2329978942871 14.8680658340454\\
58.2830009460449 14.8686294555664\\
58.3330001831055 14.8691930770874\\
58.382999420166 14.8697557449341\\
58.4329986572266 14.8703193664551\\
58.4829978942871 14.8708829879761\\
58.5330009460449 14.8714466094971\\
58.5830001831055 14.8720102310181\\
58.632999420166 14.8725738525391\\
58.6829986572266 14.8731374740601\\
58.7329978942871 14.8737010955811\\
58.7830009460449 14.8742656707764\\
58.8330001831055 14.8748292922974\\
58.882999420166 14.8753938674927\\
58.9329986572266 14.875958442688\\
58.9829978942871 14.8765230178833\\
59.0330009460449 14.8770875930786\\
59.0830001831055 14.8776521682739\\
59.132999420166 14.8782176971436\\
59.1829986572266 14.8787822723389\\
59.2329978942871 14.8793478012085\\
59.2830009460449 14.8798961639404\\
59.3330001831055 14.8803730010986\\
59.382999420166 14.8807249069214\\
59.4329986572266 14.8813066482544\\
59.4829978942871 14.881856918335\\
59.5330009460449 14.8824186325073\\
59.5830001831055 14.8829765319824\\
59.632999420166 14.8835353851318\\
59.6829986572266 14.8840942382813\\
59.7329978942871 14.8846521377563\\
59.7830009460449 14.8852100372314\\
59.8330001831055 14.8857679367065\\
59.882999420166 14.8863258361816\\
59.9329986572266 14.8868837356567\\
59.9664993286133 14.8873863220215\\
60 14.8878183364868\\
60.0368499755859 14.8881988525391\\
60.0773849487305 14.8886089324951\\
60.1219749450684 14.8890600204468\\
60.1710243225098 14.8895568847656\\
60.2210235595703 14.8900909423828\\
60.2710227966309 14.8906469345093\\
60.3210220336914 14.8912019729614\\
60.371021270752 14.8917751312256\\
60.4210243225098 14.8923873901367\\
60.4710235595703 14.8929719924927\\
60.5210227966309 14.8935384750366\\
60.5710220336914 14.8942174911499\\
60.621021270752 14.894811630249\\
60.6710243225098 14.8953762054443\\
60.7210235595703 14.8959531784058\\
60.7710227966309 14.8965330123901\\
60.8210220336914 14.8971166610718\\
60.871021270752 14.8977060317993\\
60.9210243225098 14.8983011245728\\
60.9710235595703 14.8989038467407\\
61.0210227966309 14.8995170593262\\
61.0710220336914 14.9001436233521\\
61.121021270752 14.9007883071899\\
61.1710243225098 14.9016494750977\\
61.2210235595703 14.9075689315796\\
61.2710227966309 14.9088191986084\\
61.3210220336914 14.9102449417114\\
61.371021270752 14.9118032455444\\
61.4210243225098 14.9091911315918\\
61.4710235595703 14.9045438766479\\
61.5210227966309 14.9049844741821\\
61.5710220336914 14.9053401947021\\
61.621021270752 14.905740737915\\
61.6710243225098 14.9061765670776\\
61.7210235595703 14.9066457748413\\
61.7710227966309 14.9071464538574\\
61.8210220336914 14.9076690673828\\
61.871021270752 14.9082040786743\\
61.9210243225098 14.9087476730347\\
61.9710235595703 14.909294128418\\
62.0210227966309 14.9098443984985\\
62.0710220336914 14.9103956222534\\
62.121021270752 14.9109487533569\\
62.1710243225098 14.9115028381348\\
62.2210235595703 14.9120588302612\\
62.2710227966309 14.912615776062\\
62.3210220336914 14.9131736755371\\
62.371021270752 14.9137306213379\\
62.4210243225098 14.9142866134644\\
62.4710235595703 14.9148445129395\\
62.5210227966309 14.9154033660889\\
62.5710220336914 14.9159622192383\\
62.621021270752 14.9165229797363\\
62.6710243225098 14.9170827865601\\
62.7210235595703 14.9176445007324\\
62.7710227966309 14.9182062149048\\
62.8210220336914 14.9187688827515\\
62.871021270752 14.9193315505981\\
62.9210243225098 14.9198942184448\\
62.9710235595703 14.9204568862915\\
63.0210227966309 14.9210205078125\\
63.0710220336914 14.9215831756592\\
63.121021270752 14.9221458435059\\
63.1710243225098 14.9227094650269\\
63.2210235595703 14.9232721328735\\
63.2710227966309 14.9238348007202\\
63.3210220336914 14.9243984222412\\
63.371021270752 14.9249610900879\\
63.4210243225098 14.9255237579346\\
63.4710235595703 14.9260873794556\\
63.5210227966309 14.9266500473022\\
63.5710220336914 14.9272127151489\\
63.621021270752 14.9277763366699\\
63.6710243225098 14.9283390045166\\
63.7210235595703 14.9289016723633\\
63.7710227966309 14.9294652938843\\
63.8210220336914 14.930027961731\\
63.871021270752 14.9305906295776\\
63.9210243225098 14.9311542510986\\
63.9710235595703 14.9317169189453\\
64.0210189819336 14.932279586792\\
64.0710220336914 14.932843208313\\
64.1210250854492 14.9334058761597\\
64.1710205078125 14.9339685440063\\
64.2210235595703 14.9345321655273\\
64.2710189819336 14.935094833374\\
64.3210220336914 14.9356575012207\\
64.3710250854492 14.9362211227417\\
64.4210205078125 14.9367837905884\\
64.4710235595703 14.9373464584351\\
64.5210189819336 14.9379100799561\\
64.5710220336914 14.9384727478027\\
64.6210250854492 14.9390354156494\\
64.6710205078125 14.9395990371704\\
64.7210235595703 14.9401617050171\\
64.7710189819336 14.9407243728638\\
64.8210220336914 14.9412879943848\\
64.8710250854492 14.9418506622314\\
64.9210205078125 14.9424133300781\\
64.9710235595703 14.9429769515991\\
65.0210189819336 14.9435396194458\\
65.0710220336914 14.9441032409668\\
65.1210250854492 14.9446659088135\\
65.1710205078125 14.9452285766602\\
65.2210235595703 14.9457921981812\\
65.2710189819336 14.9463548660278\\
65.3210220336914 14.9469175338745\\
65.3710250854492 14.9474811553955\\
65.4210205078125 14.9480438232422\\
65.4710235595703 14.9486064910889\\
65.5210189819336 14.9491701126099\\
65.5710220336914 14.9497327804565\\
65.6210250854492 14.9502964019775\\
65.6710205078125 14.9508724212646\\
65.7210235595703 14.951416015625\\
65.7710189819336 14.9519786834717\\
65.8210220336914 14.9525413513184\\
65.8710250854492 14.953104019165\\
65.9210205078125 14.953667640686\\
65.9710235595703 14.9542303085327\\
66.0210189819336 14.9547939300537\\
66.0710220336914 14.9553565979004\\
66.1210250854492 14.9559202194214\\
66.1710205078125 14.9564828872681\\
66.2210235595703 14.9570465087891\\
66.2710189819336 14.9576091766357\\
66.3210220336914 14.9581727981567\\
66.3710250854492 14.9587354660034\\
66.4210205078125 14.9592990875244\\
66.4710235595703 14.9598617553711\\
66.5210189819336 14.9604253768921\\
66.5710220336914 14.9609880447388\\
66.6210250854492 14.9615516662598\\
66.6710205078125 14.9621143341064\\
66.7210235595703 14.9626779556274\\
66.7710189819336 14.9632415771484\\
66.8210220336914 14.9638051986694\\
66.8710250854492 14.9643697738647\\
66.9210205078125 14.9649333953857\\
66.9710235595703 14.9655027389526\\
67.0210189819336 14.9660749435425\\
67.0710220336914 14.9666614532471\\
67.1210250854492 14.9672775268555\\
67.1710205078125 14.9679660797119\\
67.2210235595703 14.9688205718994\\
67.2710189819336 14.9700479507446\\
67.3210220336914 14.9720058441162\\
67.3710250854492 14.9751300811768\\
67.4210205078125 14.9794082641602\\
67.4710235595703 14.9837017059326\\
67.5210189819336 14.9900217056274\\
67.5710220336914 14.978307723999\\
67.6210250854492 14.9774312973022\\
67.6710205078125 14.9770612716675\\
67.7210235595703 14.9768018722534\\
67.7710189819336 14.9762153625488\\
67.8210220336914 14.9771604537964\\
67.8710250854492 14.9766149520874\\
67.9210205078125 14.9769239425659\\
67.9710235595703 14.977352142334\\
68.0210189819336 14.9779596328735\\
68.0710220336914 14.9785032272339\\
68.1210250854492 14.9790754318237\\
68.1710205078125 14.9796342849731\\
68.2210235595703 14.9801998138428\\
68.2710189819336 14.9807624816895\\
68.3210220336914 14.9813261032104\\
68.3710250854492 14.9818897247314\\
68.4210205078125 14.9824533462524\\
68.4710235595703 14.9830169677734\\
68.5210189819336 14.9835796356201\\
68.5710220336914 14.9841432571411\\
68.6210250854492 14.9847068786621\\
68.6710205078125 14.9852705001831\\
68.7210235595703 14.9858341217041\\
68.7710189819336 14.9863977432251\\
68.8210220336914 14.9869613647461\\
68.8710250854492 14.9875249862671\\
68.9210205078125 14.9880886077881\\
68.9710235595703 14.9886522293091\\
69.0210189819336 14.9892158508301\\
69.0710220336914 14.9897794723511\\
69.1210250854492 14.9903430938721\\
69.1710205078125 14.9909067153931\\
69.2210235595703 14.9914703369141\\
69.2710189819336 14.9920339584351\\
69.3210220336914 14.9925975799561\\
69.3710250854492 14.9931612014771\\
69.4210205078125 14.993724822998\\
69.4710235595703 14.994288444519\\
69.5210189819336 14.9948530197144\\
69.5710220336914 14.9954166412354\\
69.6210250854492 14.9959802627563\\
69.6710205078125 14.9965438842773\\
69.7210235595703 14.9971084594727\\
69.7710189819336 14.9976720809937\\
69.8210220336914 14.9982357025146\\
69.8710250854492 14.99880027771\\
69.9210205078125 14.999363899231\\
69.9710235595703 14.999927520752\\
70.0210189819336 15.0004920959473\\
70.0710220336914 15.0012865066528\\
70.1210250854492 15.0019063949585\\
70.1710205078125 15.0024795532227\\
70.2210235595703 15.0030450820923\\
70.2710189819336 15.0036096572876\\
70.3210220336914 15.0041732788086\\
70.3710250854492 15.0047378540039\\
70.4210205078125 15.0053024291992\\
70.4710235595703 15.0058670043945\\
70.5210189819336 15.0064315795898\\
70.5710220336914 15.0069961547852\\
70.6210250854492 15.0075607299805\\
70.6710205078125 15.0081262588501\\
70.7210235595703 15.0086908340454\\
70.7710189819336 15.0092554092407\\
70.8210220336914 15.0098209381104\\
70.8710250854492 15.01038646698\\
70.9210205078125 15.0109510421753\\
70.9710235595703 15.0115165710449\\
71.0210189819336 15.0120820999146\\
71.0710220336914 15.0126476287842\\
71.1210250854492 15.0132131576538\\
71.1710205078125 15.0137786865234\\
71.2210235595703 15.0143451690674\\
71.2710189819336 15.014910697937\\
71.3210220336914 15.015477180481\\
71.3710250854492 15.0160436630249\\
71.4210205078125 15.0166101455688\\
71.4710235595703 15.0171766281128\\
71.5210189819336 15.0177373886108\\
71.5710220336914 15.018274307251\\
71.6210250854492 15.0185747146606\\
71.6710205078125 15.0191440582275\\
71.7210235595703 15.0196962356567\\
71.7710189819336 15.0202579498291\\
71.8210220336914 15.0208177566528\\
71.8710250854492 15.0213785171509\\
71.9210205078125 15.0219383239746\\
71.9710235595703 15.022497177124\\
72.0210189819336 15.0230569839478\\
72.0710220336914 15.0236167907715\\
72.1210250854492 15.0241756439209\\
72.1710205078125 15.0247344970703\\
72.2210235595703 15.0252933502197\\
72.2710189819336 15.0258512496948\\
72.3210220336914 15.0264101028442\\
72.3710250854492 15.0269680023193\\
72.4210205078125 15.0275259017944\\
72.4710235595703 15.0280828475952\\
72.5210189819336 15.0286407470703\\
72.5710220336914 15.0291976928711\\
72.6210250854492 15.0297660827637\\
72.6710205078125 15.0303792953491\\
72.7210235595703 15.0309658050537\\
72.7710189819336 15.031533241272\\
72.8210220336914 15.0322036743164\\
72.8710250854492 15.0328006744385\\
72.9210205078125 15.0333690643311\\
72.9710235595703 15.0339469909668\\
73.0210189819336 15.0345268249512\\
73.0710220336914 15.0351104736328\\
73.1210250854492 15.0356998443604\\
73.1710205078125 15.0362958908081\\
73.2210235595703 15.0368986129761\\
73.2710189819336 15.0375118255615\\
73.3210220336914 15.0381374359131\\
73.3710250854492 15.0387811660767\\
73.4210205078125 15.0394477844238\\
73.4710235595703 15.0453224182129\\
73.5210189819336 15.0467548370361\\
73.5710220336914 15.0481767654419\\
73.6210250854492 15.0497159957886\\
73.6710205078125 15.047830581665\\
73.7210235595703 15.0425806045532\\
73.7710189819336 15.0429973602295\\
73.8210220336914 15.0433654785156\\
73.8710250854492 15.0437641143799\\
73.9210205078125 15.0441980361938\\
73.9710235595703 15.0446662902832\\
74.0210189819336 15.0451650619507\\
74.0710220336914 15.0456867218018\\
74.1210250854492 15.0462226867676\\
74.1710205078125 15.0467662811279\\
74.2210235595703 15.0473136901855\\
74.2710189819336 15.0478639602661\\
74.3210220336914 15.048415184021\\
74.3710250854492 15.0489692687988\\
74.4210205078125 15.049524307251\\
74.4710235595703 15.0500802993774\\
74.5210189819336 15.0506372451782\\
74.5710220336914 15.051197052002\\
74.6210250854492 15.0517539978027\\
74.6710205078125 15.0523109436035\\
74.7210235595703 15.0528688430786\\
74.7710189819336 15.053427696228\\
74.8210220336914 15.0539884567261\\
74.8710250854492 15.0545492172241\\
74.9210205078125 15.0551090240479\\
74.9605102539063 15.0556726455688\\
75 15.0561151504517\\
};
\addlegendentry{14 "};

\addplot [
color=green,
solid
]
table[row sep=crcr]{
0 14.7004346847534\\
9.99999971718069e-10 14.7004346847534\\
2.49999998480632e-09 14.7004346847534\\
4.74999994892755e-09 14.7004346847534\\
8.1250002281763e-09 14.7004346847534\\
1.31874999809156e-08 14.7004346847534\\
2.0781250498203e-08 14.7004346847534\\
3.21718758300449e-08 14.7004346847534\\
4.92578138278077e-08 14.7004346847534\\
7.48867208244519e-08 14.7004346847534\\
1.13330081319418e-07 14.7004346847534\\
1.70995122061868e-07 14.7004346847534\\
2.57492672517401e-07 14.7004346847534\\
3.8723902662241e-07 14.7004346847534\\
5.81858500936505e-07 14.7004346847534\\
8.73787769251066e-07 14.7004346847534\\
1.31168167172291e-06 14.7004346847534\\
1.9685226106958e-06 14.7004346847534\\
2.95378367809462e-06 14.7004346847534\\
4.43167573394021e-06 14.7004346847534\\
6.64851359033491e-06 14.7004346847534\\
9.97376992017962e-06 14.7004346847534\\
1.49616553244414e-05 14.7004346847534\\
2.24434825213393e-05 14.7004346847534\\
3.36662233166862e-05 14.7004346847534\\
5.05003372381907e-05 14.7004356384277\\
7.57515081204474e-05 14.7004356384277\\
0.000113628258986864 14.7004356384277\\
0.000170443381648511 14.7004365921021\\
0.000255666091106832 14.7004375457764\\
0.000383500126190484 14.7004384994507\\
0.000575251178815961 14.7004404067993\\
0.000862877757754177 14.7004413604736\\
0.00129431765526533 14.7004413604736\\
0.00194147753063589 14.7004375457764\\
0.00291221728548408 14.7004327774048\\
0.00436832662671804 14.7004318237305\\
0.00597004732117057 14.7004327774048\\
0.00773194013163447 14.7004346847534\\
0.009670021943748 14.7004356384277\\
0.0118019115179777 14.7004356384277\\
0.0141469910740852 14.7004346847534\\
0.0167265776544809 14.7004346847534\\
0.0195641238242388 14.7004346847534\\
0.0226854234933853 14.7004346847534\\
0.0261188540607691 14.7004346847534\\
0.0298956278711557 14.7004346847534\\
0.0340500771999359 14.7004346847534\\
0.0386199727654457 14.7004346847534\\
0.0436468608677387 14.7004346847534\\
0.0491764321923256 14.7004346847534\\
0.0552589632570744 14.7004346847534\\
0.0619497485458851 14.7004346847534\\
0.0693096145987511 14.7004346847534\\
0.0774054601788521 14.7004346847534\\
0.0863108932971954 14.7004337310791\\
0.0961068719625473 14.7004346847534\\
0.106882445514202 14.7004346847534\\
0.118735581636429 14.7004346847534\\
0.13177402317524 14.7004346847534\\
0.146116316318512 14.7004346847534\\
0.161892831325531 14.7004346847534\\
0.179247006773949 14.7004346847534\\
0.198336601257324 14.7004346847534\\
0.21933513879776 14.7004346847534\\
0.242433547973633 14.7004346847534\\
0.267841786146164 14.7004346847534\\
0.29579085111618 14.7004346847534\\
0.32653483748436 14.7004346847534\\
0.360353201627731 14.7004346847534\\
0.397553414106369 14.7004346847534\\
0.438473641872406 14.7004346847534\\
0.483485877513886 14.7004346847534\\
0.532999336719513 14.7004346847534\\
0.582999348640442 14.7004346847534\\
0.632999360561371 14.7004346847534\\
0.6829993724823 14.7004346847534\\
0.732999384403229 14.7004346847534\\
0.782999336719513 14.7004346847534\\
0.832999348640442 14.7004346847534\\
0.882999360561371 14.7004346847534\\
0.9329993724823 14.7004346847534\\
0.982999384403229 14.7004346847534\\
1.03299939632416 14.7004346847534\\
1.08299934864044 14.7004346847534\\
1.13299930095673 14.7004346847534\\
1.1829993724823 14.7004346847534\\
1.23299932479858 14.7004346847534\\
1.28299939632416 14.7004346847534\\
1.33299934864044 14.7004346847534\\
1.38299930095673 14.7004346847534\\
1.4329993724823 14.7004346847534\\
1.48299932479858 14.7004346847534\\
1.53299939632416 14.7004346847534\\
1.58299934864044 14.7004346847534\\
1.63299930095673 14.7004346847534\\
1.6829993724823 14.7004346847534\\
1.73299932479858 14.7004346847534\\
1.78299939632416 14.7004346847534\\
1.83299934864044 14.7004346847534\\
1.88299930095673 14.7004346847534\\
1.9329993724823 14.7004346847534\\
1.98299932479858 14.7004346847534\\
2.03299927711487 14.7004346847534\\
2.08299946784973 14.7004346847534\\
2.13299942016602 14.7004346847534\\
2.1829993724823 14.7004346847534\\
2.23299932479858 14.7004346847534\\
2.28299927711487 14.7004346847534\\
2.33299946784973 14.7004346847534\\
2.38299942016602 14.7004346847534\\
2.4329993724823 14.7004346847534\\
2.48299932479858 14.7004346847534\\
2.53299927711487 14.7004346847534\\
2.58299946784973 14.7004346847534\\
2.63299942016602 14.7004346847534\\
2.6829993724823 14.7004346847534\\
2.73299932479858 14.7004346847534\\
2.78299927711487 14.7004346847534\\
2.83299946784973 14.7004346847534\\
2.88299942016602 14.7004346847534\\
2.9329993724823 14.7004346847534\\
2.98299932479858 14.7004346847534\\
3.03299927711487 14.7004346847534\\
3.08299946784973 14.7004346847534\\
3.13299942016602 14.7004346847534\\
3.1829993724823 14.7004346847534\\
3.23299932479858 14.7004346847534\\
3.28299927711487 14.7004346847534\\
3.33299946784973 14.7004346847534\\
3.38299942016602 14.7004346847534\\
3.4329993724823 14.7004346847534\\
3.48299932479858 14.7004346847534\\
3.53299927711487 14.7004346847534\\
3.58299946784973 14.7004346847534\\
3.63299942016602 14.7004346847534\\
3.6829993724823 14.7004346847534\\
3.73299932479858 14.7004346847534\\
3.78299927711487 14.7004346847534\\
3.83299946784973 14.7004346847534\\
3.88299942016602 14.7004346847534\\
3.9329993724823 14.7004346847534\\
3.98299932479858 14.7004346847534\\
4.03299951553345 14.7004346847534\\
4.08299922943115 14.7004346847534\\
4.13299942016602 14.7004346847534\\
4.18299913406372 14.7004346847534\\
4.23299932479858 14.7004346847534\\
4.28299951553345 14.7004346847534\\
4.33299922943115 14.7004346847534\\
4.38299942016602 14.7004346847534\\
4.43299913406372 14.7004346847534\\
4.48299932479858 14.7004346847534\\
4.53299951553345 14.7004346847534\\
4.58299922943115 14.7004346847534\\
4.63299942016602 14.7004346847534\\
4.68299913406372 14.7004346847534\\
4.73299932479858 14.7004346847534\\
4.78299951553345 14.7004346847534\\
4.83299922943115 14.7004346847534\\
4.88299942016602 14.7004346847534\\
4.93299913406372 14.7004346847534\\
4.98299932479858 14.7004346847534\\
5.03299951553345 14.7004346847534\\
5.08299922943115 14.7004346847534\\
5.13299942016602 14.7004346847534\\
5.18299913406372 14.7004346847534\\
5.23299932479858 14.7004346847534\\
5.28299951553345 14.7004346847534\\
5.33299922943115 14.7004346847534\\
5.38299942016602 14.7004346847534\\
5.43299913406372 14.7004346847534\\
5.48299932479858 14.7004346847534\\
5.53299951553345 14.7004346847534\\
5.58299922943115 14.7004346847534\\
5.63299942016602 14.7004346847534\\
5.68299913406372 14.7004346847534\\
5.73299932479858 14.7004346847534\\
5.78299951553345 14.7004346847534\\
5.83299922943115 14.7004346847534\\
5.88299942016602 14.7004346847534\\
5.93299913406372 14.7004346847534\\
5.98299932479858 14.7004346847534\\
6.03299951553345 14.7004346847534\\
6.08299922943115 14.7004337310791\\
6.13299942016602 14.7004337310791\\
6.18299913406372 14.7004327774048\\
6.23299932479858 14.7004318237305\\
6.28299951553345 14.7004280090332\\
6.33299922943115 14.7004384994507\\
6.38299942016602 14.7004346847534\\
6.43299913406372 14.7004346847534\\
6.48299932479858 14.7004346847534\\
6.53299951553345 14.7004346847534\\
6.58299922943115 14.7004365921021\\
6.63299942016602 14.7004356384277\\
6.68299913406372 14.7004346847534\\
6.73299932479858 14.7004346847534\\
6.78299951553345 14.7004346847534\\
6.83299922943115 14.7004346847534\\
6.88299942016602 14.7004346847534\\
6.93299913406372 14.7004346847534\\
6.98299932479858 14.7004346847534\\
7.03299951553345 14.7004346847534\\
7.08299922943115 14.7004346847534\\
7.13299942016602 14.7004346847534\\
7.18299913406372 14.7004346847534\\
7.23299932479858 14.7004346847534\\
7.28299951553345 14.7004346847534\\
7.33299922943115 14.7004346847534\\
7.38299942016602 14.7004346847534\\
7.43299913406372 14.7004346847534\\
7.48299932479858 14.7004346847534\\
7.53299951553345 14.7004346847534\\
7.58299922943115 14.7004346847534\\
7.63299942016602 14.7004346847534\\
7.68299913406372 14.7004346847534\\
7.73299932479858 14.7004346847534\\
7.78299951553345 14.7004346847534\\
7.83299922943115 14.7004346847534\\
7.88299942016602 14.7004346847534\\
7.93299913406372 14.7004346847534\\
7.98299932479858 14.7004346847534\\
8.03299903869629 14.7004346847534\\
8.08299922943115 14.7004346847534\\
8.13299942016602 14.7004346847534\\
8.18299961090088 14.7004346847534\\
8.23299980163574 14.7004346847534\\
8.28299903869629 14.7004346847534\\
8.33299922943115 14.7004346847534\\
8.38299942016602 14.7004346847534\\
8.43299961090088 14.7004346847534\\
8.48299980163574 14.7004346847534\\
8.53299903869629 14.7004346847534\\
8.58299922943115 14.7004346847534\\
8.63299942016602 14.7004346847534\\
8.68299961090088 14.7004346847534\\
8.73299980163574 14.7004346847534\\
8.78299903869629 14.7004346847534\\
8.83299922943115 14.7004346847534\\
8.88299942016602 14.7004346847534\\
8.93299961090088 14.7004346847534\\
8.98299980163574 14.7004346847534\\
9.03299903869629 14.7004346847534\\
9.08299922943115 14.7004346847534\\
9.13299942016602 14.7004346847534\\
9.18299961090088 14.7004346847534\\
9.23299980163574 14.7004346847534\\
9.28299903869629 14.7004346847534\\
9.33299922943115 14.7004346847534\\
9.38299942016602 14.7004346847534\\
9.43299961090088 14.7004346847534\\
9.48299980163574 14.7004346847534\\
9.53299903869629 14.7004346847534\\
9.58299922943115 14.7004346847534\\
9.63299942016602 14.7004346847534\\
9.68299961090088 14.7004346847534\\
9.73299980163574 14.7004346847534\\
9.78299903869629 14.7004346847534\\
9.83299922943115 14.7004346847534\\
9.88299942016602 14.7004346847534\\
9.93299961090088 14.7004346847534\\
9.98299980163574 14.7004346847534\\
10.0329990386963 14.7004346847534\\
10.0829992294312 14.7004346847534\\
10.132999420166 14.7004346847534\\
10.1829996109009 14.7004346847534\\
10.2329998016357 14.7004346847534\\
10.2829990386963 14.7004346847534\\
10.3329992294312 14.7004346847534\\
10.382999420166 14.7004346847534\\
10.4329996109009 14.7004346847534\\
10.4829998016357 14.7004346847534\\
10.5329990386963 14.7004346847534\\
10.5829992294312 14.7004346847534\\
10.632999420166 14.7004346847534\\
10.6829996109009 14.7004346847534\\
10.7329998016357 14.7004346847534\\
10.7829990386963 14.7004346847534\\
10.8329992294312 14.7004346847534\\
10.882999420166 14.7004346847534\\
10.9329996109009 14.7004346847534\\
10.9829998016357 14.7004346847534\\
11.0329990386963 14.7004346847534\\
11.0829992294312 14.7004346847534\\
11.132999420166 14.7004346847534\\
11.1829996109009 14.7004346847534\\
11.2329998016357 14.7004346847534\\
11.2829990386963 14.7004346847534\\
11.3329992294312 14.7004346847534\\
11.382999420166 14.7004346847534\\
11.4329996109009 14.7004346847534\\
11.4829998016357 14.7004346847534\\
11.5329990386963 14.7004346847534\\
11.5829992294312 14.7004346847534\\
11.632999420166 14.7004346847534\\
11.6829996109009 14.7004346847534\\
11.7329998016357 14.7004346847534\\
11.7829990386963 14.7004346847534\\
11.8329992294312 14.7004346847534\\
11.882999420166 14.7004346847534\\
11.9329996109009 14.7004346847534\\
11.9829998016357 14.7004346847534\\
12.0329990386963 14.7004346847534\\
12.0829992294312 14.7004346847534\\
12.132999420166 14.7004346847534\\
12.1829996109009 14.7004346847534\\
12.2329998016357 14.7004346847534\\
12.2829990386963 14.7004346847534\\
12.3329992294312 14.7004346847534\\
12.382999420166 14.7004346847534\\
12.4329996109009 14.7004356384277\\
12.4829998016357 14.7004346847534\\
12.5329990386963 14.7004337310791\\
12.5829992294312 14.7004346847534\\
12.632999420166 14.7004346847534\\
12.6829996109009 14.7004356384277\\
12.7329998016357 14.7004346847534\\
12.7829990386963 14.7004346847534\\
12.8329992294312 14.7004346847534\\
12.882999420166 14.7004346847534\\
12.9329996109009 14.7004346847534\\
12.9829998016357 14.7004346847534\\
13.0329990386963 14.7004346847534\\
13.0829992294312 14.7004346847534\\
13.132999420166 14.7004346847534\\
13.1829996109009 14.7004346847534\\
13.2329998016357 14.7004346847534\\
13.2829990386963 14.7004346847534\\
13.3329992294312 14.7004346847534\\
13.382999420166 14.7004346847534\\
13.4329996109009 14.7004346847534\\
13.4829998016357 14.7004346847534\\
13.5329990386963 14.7004346847534\\
13.5829992294312 14.7004346847534\\
13.632999420166 14.7004346847534\\
13.6829996109009 14.7004346847534\\
13.7329998016357 14.7004346847534\\
13.7829990386963 14.7004346847534\\
13.8329992294312 14.7004346847534\\
13.882999420166 14.7004346847534\\
13.9329996109009 14.7004346847534\\
13.9829998016357 14.7004346847534\\
14.0329990386963 14.7004346847534\\
14.0829992294312 14.7004346847534\\
14.132999420166 14.7004346847534\\
14.1829996109009 14.7004346847534\\
14.2329998016357 14.7004346847534\\
14.2829990386963 14.7004346847534\\
14.3329992294312 14.7004346847534\\
14.382999420166 14.7004346847534\\
14.4329996109009 14.7004346847534\\
14.4829998016357 14.7004346847534\\
14.5329990386963 14.7004346847534\\
14.5829992294312 14.7004346847534\\
14.632999420166 14.7004346847534\\
14.6829996109009 14.7004346847534\\
14.7329998016357 14.7004346847534\\
14.7829990386963 14.7004346847534\\
14.8329992294312 14.7004346847534\\
14.882999420166 14.7004346847534\\
14.9329996109009 14.7004346847534\\
14.9829998016357 14.7004346847534\\
15.0329990386963 14.7004346847534\\
15.0829992294312 14.7004346847534\\
15.132999420166 14.7004346847534\\
15.1829996109009 14.7004346847534\\
15.2329998016357 14.7004346847534\\
15.2829990386963 14.7004346847534\\
15.3329992294312 14.7004346847534\\
15.382999420166 14.7004346847534\\
15.4329996109009 14.7004346847534\\
15.4829998016357 14.7004346847534\\
15.5329990386963 14.7004346847534\\
15.5829992294312 14.7004346847534\\
15.632999420166 14.7004346847534\\
15.6829996109009 14.7004346847534\\
15.7329998016357 14.7004346847534\\
15.7829990386963 14.7004346847534\\
15.8329992294312 14.7004346847534\\
15.882999420166 14.7004346847534\\
15.9329996109009 14.7004346847534\\
15.9829998016357 14.7004346847534\\
16.0329990386963 14.7004346847534\\
16.0830001831055 14.7004346847534\\
16.132999420166 14.7004346847534\\
16.1829986572266 14.7004346847534\\
16.2329998016357 14.7004346847534\\
16.2829990386963 14.7004346847534\\
16.3330001831055 14.7004346847534\\
16.382999420166 14.7004346847534\\
16.4329986572266 14.7004346847534\\
16.4829998016357 14.7004346847534\\
16.5329990386963 14.7004346847534\\
16.5830001831055 14.7004346847534\\
16.632999420166 14.7004346847534\\
16.6829986572266 14.7004346847534\\
16.7329998016357 14.7004346847534\\
16.7829990386963 14.7004346847534\\
16.8330001831055 14.7004346847534\\
16.882999420166 14.7004346847534\\
16.9329986572266 14.7004346847534\\
16.9829998016357 14.7004346847534\\
17.0329990386963 14.7004346847534\\
17.0830001831055 14.7004346847534\\
17.132999420166 14.7004346847534\\
17.1829986572266 14.7004346847534\\
17.2329998016357 14.7004346847534\\
17.2829990386963 14.7004346847534\\
17.3330001831055 14.7004346847534\\
17.382999420166 14.7004346847534\\
17.4329986572266 14.7004346847534\\
17.4829998016357 14.7004346847534\\
17.5329990386963 14.7004346847534\\
17.5830001831055 14.7004346847534\\
17.632999420166 14.7004346847534\\
17.6829986572266 14.7004346847534\\
17.7329998016357 14.7004346847534\\
17.7829990386963 14.7004346847534\\
17.8330001831055 14.7004346847534\\
17.882999420166 14.7004346847534\\
17.9329986572266 14.7004346847534\\
17.9829998016357 14.7004346847534\\
18.0329990386963 14.7004346847534\\
18.0830001831055 14.7004346847534\\
18.132999420166 14.7004346847534\\
18.1829986572266 14.7004346847534\\
18.2329998016357 14.7004346847534\\
18.2829990386963 14.7004337310791\\
18.3330001831055 14.7004337310791\\
18.382999420166 14.7004327774048\\
18.4329986572266 14.7004318237305\\
18.4829998016357 14.7004318237305\\
18.5329990386963 14.7004289627075\\
18.5830001831055 14.7004384994507\\
18.632999420166 14.7004346847534\\
18.6829986572266 14.7004346847534\\
18.7329998016357 14.7004346847534\\
18.7829990386963 14.7004356384277\\
18.8330001831055 14.7004365921021\\
18.882999420166 14.7004356384277\\
18.9329986572266 14.7004346847534\\
18.9829998016357 14.7004346847534\\
19.0329990386963 14.7004346847534\\
19.0830001831055 14.7004346847534\\
19.132999420166 14.7004346847534\\
19.1829986572266 14.7004346847534\\
19.2329998016357 14.7004346847534\\
19.2829990386963 14.7004346847534\\
19.3330001831055 14.7004346847534\\
19.382999420166 14.7004346847534\\
19.4329986572266 14.7004346847534\\
19.4829998016357 14.7004346847534\\
19.5329990386963 14.7004346847534\\
19.5830001831055 14.7004346847534\\
19.632999420166 14.7004346847534\\
19.6829986572266 14.7004346847534\\
19.7329998016357 14.7004346847534\\
19.7829990386963 14.7004346847534\\
19.8330001831055 14.7004346847534\\
19.882999420166 14.7004346847534\\
19.9329986572266 14.7004346847534\\
19.9829998016357 14.7004346847534\\
20.0329990386963 14.7004346847534\\
20.0830001831055 14.7004346847534\\
20.132999420166 14.7004346847534\\
20.1829986572266 14.7004346847534\\
20.2329998016357 14.7004346847534\\
20.2829990386963 14.7004346847534\\
20.3330001831055 14.7004346847534\\
20.382999420166 14.7004346847534\\
20.4329986572266 14.7004346847534\\
20.4829998016357 14.7004346847534\\
20.5329990386963 14.7004346847534\\
20.5830001831055 14.7004346847534\\
20.632999420166 14.7004346847534\\
20.6829986572266 14.7004346847534\\
20.7329998016357 14.7004346847534\\
20.7829990386963 14.7004346847534\\
20.8330001831055 14.7004346847534\\
20.882999420166 14.7004346847534\\
20.9329986572266 14.7004346847534\\
20.9829998016357 14.7004346847534\\
21.0329990386963 14.7004346847534\\
21.0830001831055 14.7004346847534\\
21.132999420166 14.7004346847534\\
21.1829986572266 14.7004346847534\\
21.2329998016357 14.7004346847534\\
21.2829990386963 14.7004346847534\\
21.3330001831055 14.7004346847534\\
21.382999420166 14.7004346847534\\
21.4329986572266 14.7004346847534\\
21.4829998016357 14.7004346847534\\
21.5329990386963 14.7004346847534\\
21.5830001831055 14.7004346847534\\
21.632999420166 14.7004346847534\\
21.6829986572266 14.7004346847534\\
21.7329998016357 14.7004346847534\\
21.7829990386963 14.7004346847534\\
21.8330001831055 14.7004346847534\\
21.882999420166 14.7004346847534\\
21.9329986572266 14.7004346847534\\
21.9829998016357 14.7004346847534\\
22.0329990386963 14.7004346847534\\
22.0830001831055 14.7004346847534\\
22.132999420166 14.7004346847534\\
22.1829986572266 14.7004346847534\\
22.2329998016357 14.7004346847534\\
22.2829990386963 14.7004346847534\\
22.3330001831055 14.7004346847534\\
22.382999420166 14.7004346847534\\
22.4329986572266 14.7004346847534\\
22.4829998016357 14.7004346847534\\
22.5329990386963 14.7004346847534\\
22.5830001831055 14.7004346847534\\
22.632999420166 14.7004346847534\\
22.6829986572266 14.7004346847534\\
22.7329998016357 14.7004346847534\\
22.7829990386963 14.7004346847534\\
22.8330001831055 14.7004346847534\\
22.882999420166 14.7004346847534\\
22.9329986572266 14.7004346847534\\
22.9829998016357 14.7004346847534\\
23.0329990386963 14.7004346847534\\
23.0830001831055 14.7004346847534\\
23.132999420166 14.7004346847534\\
23.1829986572266 14.7004346847534\\
23.2329998016357 14.7004346847534\\
23.2829990386963 14.7004346847534\\
23.3330001831055 14.7004346847534\\
23.382999420166 14.7004346847534\\
23.4329986572266 14.7004346847534\\
23.4829998016357 14.7004346847534\\
23.5329990386963 14.7004346847534\\
23.5830001831055 14.7004346847534\\
23.632999420166 14.7004346847534\\
23.6829986572266 14.7004346847534\\
23.7329998016357 14.7004346847534\\
23.7829990386963 14.7004346847534\\
23.8330001831055 14.7004346847534\\
23.882999420166 14.7004346847534\\
23.9329986572266 14.7004346847534\\
23.9829998016357 14.7004346847534\\
24.0329990386963 14.7004346847534\\
24.0830001831055 14.7004346847534\\
24.132999420166 14.7004346847534\\
24.1829986572266 14.7004346847534\\
24.2329998016357 14.7004346847534\\
24.2829990386963 14.7004346847534\\
24.3330001831055 14.7004346847534\\
24.382999420166 14.7004346847534\\
24.4329986572266 14.7004346847534\\
24.4829998016357 14.7004346847534\\
24.5329990386963 14.7004346847534\\
24.5830001831055 14.7004346847534\\
24.632999420166 14.7004346847534\\
24.6829986572266 14.7004346847534\\
24.7329998016357 14.7004346847534\\
24.7829990386963 14.7004346847534\\
24.8330001831055 14.7004346847534\\
24.882999420166 14.7004346847534\\
24.9329986572266 14.7004346847534\\
24.9829998016357 14.7004346847534\\
25.0329990386963 14.7004346847534\\
25.0830001831055 14.7004346847534\\
25.132999420166 14.7004346847534\\
25.1829986572266 14.7004346847534\\
25.2329998016357 14.7004346847534\\
25.2829990386963 14.7004346847534\\
25.3330001831055 14.7004346847534\\
25.382999420166 14.7004346847534\\
25.4329986572266 14.7004346847534\\
25.4829998016357 14.7004346847534\\
25.5329990386963 14.7004346847534\\
25.5830001831055 14.7004346847534\\
25.632999420166 14.7004346847534\\
25.6829986572266 14.7004346847534\\
25.7329998016357 14.7004346847534\\
25.7829990386963 14.7004346847534\\
25.8330001831055 14.7004346847534\\
25.882999420166 14.7004346847534\\
25.9329986572266 14.7004346847534\\
25.9829998016357 14.7004346847534\\
26.0329990386963 14.7004346847534\\
26.0830001831055 14.7004346847534\\
26.132999420166 14.7004346847534\\
26.1829986572266 14.7004346847534\\
26.2329998016357 14.7004346847534\\
26.2829990386963 14.7004346847534\\
26.3330001831055 14.7004346847534\\
26.382999420166 14.7004346847534\\
26.4329986572266 14.7004346847534\\
26.4829998016357 14.7004346847534\\
26.5329990386963 14.7004346847534\\
26.5830001831055 14.7004346847534\\
26.632999420166 14.7004346847534\\
26.6829986572266 14.7004346847534\\
26.7329998016357 14.7004346847534\\
26.7829990386963 14.7004346847534\\
26.8330001831055 14.7004346847534\\
26.882999420166 14.7004346847534\\
26.9329986572266 14.7004346847534\\
26.9829998016357 14.7004346847534\\
27.0329990386963 14.7004346847534\\
27.0830001831055 14.7004346847534\\
27.132999420166 14.7004346847534\\
27.1829986572266 14.7004346847534\\
27.2329998016357 14.7004346847534\\
27.2829990386963 14.7004346847534\\
27.3330001831055 14.7004346847534\\
27.382999420166 14.7004346847534\\
27.4329986572266 14.7004346847534\\
27.4829998016357 14.7004346847534\\
27.5329990386963 14.7004346847534\\
27.5830001831055 14.7004346847534\\
27.632999420166 14.7004346847534\\
27.6829986572266 14.7004346847534\\
27.7329998016357 14.7004346847534\\
27.7829990386963 14.7004346847534\\
27.8330001831055 14.7004346847534\\
27.882999420166 14.7004346847534\\
27.9329986572266 14.7004346847534\\
27.9829998016357 14.7004346847534\\
28.0329990386963 14.7004346847534\\
28.0830001831055 14.7004346847534\\
28.132999420166 14.7004346847534\\
28.1829986572266 14.7004346847534\\
28.2329998016357 14.7004346847534\\
28.2829990386963 14.7004346847534\\
28.3330001831055 14.7004346847534\\
28.382999420166 14.7004346847534\\
28.4329986572266 14.7004346847534\\
28.4829998016357 14.7004346847534\\
28.5329990386963 14.7004346847534\\
28.5830001831055 14.7004346847534\\
28.632999420166 14.7004346847534\\
28.6829986572266 14.7004346847534\\
28.7329998016357 14.7004346847534\\
28.7829990386963 14.7004346847534\\
28.8330001831055 14.7004346847534\\
28.882999420166 14.7004346847534\\
28.9329986572266 14.7004346847534\\
28.9829998016357 14.7004346847534\\
29.0329990386963 14.7004346847534\\
29.0830001831055 14.7004346847534\\
29.132999420166 14.7004346847534\\
29.1829986572266 14.7004346847534\\
29.2329998016357 14.7004346847534\\
29.2829990386963 14.7004346847534\\
29.3330001831055 14.7004346847534\\
29.382999420166 14.7004346847534\\
29.4329986572266 14.7004346847534\\
29.4829998016357 14.7004346847534\\
29.5329990386963 14.7004346847534\\
29.5830001831055 14.7004346847534\\
29.632999420166 14.7004346847534\\
29.6829986572266 14.7004346847534\\
29.7329998016357 14.7004346847534\\
29.7829990386963 14.7004346847534\\
29.8330001831055 14.7004346847534\\
29.882999420166 14.7004346847534\\
29.9329986572266 14.7004346847534\\
29.9829998016357 14.7004346847534\\
30.0329990386963 14.7004346847534\\
30.0830001831055 14.7004346847534\\
30.132999420166 14.7004346847534\\
30.1829986572266 14.7004346847534\\
30.2329998016357 14.7004346847534\\
30.2829990386963 14.7004346847534\\
30.3330001831055 14.7004346847534\\
30.382999420166 14.7004346847534\\
30.4329986572266 14.7004337310791\\
30.4829998016357 14.7004337310791\\
30.5329990386963 14.7004346847534\\
30.5830001831055 14.7004337310791\\
30.632999420166 14.7004327774048\\
30.6829986572266 14.7004318237305\\
30.7329998016357 14.7004318237305\\
30.7829990386963 14.7004308700562\\
30.8330001831055 14.7004384994507\\
30.882999420166 14.7004346847534\\
30.9329986572266 14.7004346847534\\
30.9829998016357 14.7004346847534\\
31.0329990386963 14.7004365921021\\
31.0830001831055 14.7004356384277\\
31.132999420166 14.7004356384277\\
31.1829986572266 14.7004346847534\\
31.2329998016357 14.7004346847534\\
31.2829990386963 14.7004346847534\\
31.3330001831055 14.7004346847534\\
31.382999420166 14.7004346847534\\
31.4329986572266 14.7004346847534\\
31.4829998016357 14.7004346847534\\
31.5329990386963 14.7004346847534\\
31.5830001831055 14.7004346847534\\
31.632999420166 14.7004346847534\\
31.6829986572266 14.7004346847534\\
31.7329998016357 14.7004346847534\\
31.7829990386963 14.7004346847534\\
31.8330001831055 14.7004346847534\\
31.882999420166 14.7004346847534\\
31.9329986572266 14.7004346847534\\
31.9829998016357 14.7004346847534\\
32.0330009460449 14.7004346847534\\
32.0830001831055 14.7004346847534\\
32.132999420166 14.7004346847534\\
32.1829986572266 14.7004346847534\\
32.2329978942871 14.7004346847534\\
32.2830009460449 14.7004346847534\\
32.3330001831055 14.7004346847534\\
32.382999420166 14.7004346847534\\
32.4329986572266 14.7004346847534\\
32.4829978942871 14.7004346847534\\
32.5330009460449 14.7004346847534\\
32.5830001831055 14.7004346847534\\
32.632999420166 14.7004346847534\\
32.6829986572266 14.7004346847534\\
32.7329978942871 14.7004346847534\\
32.7830009460449 14.7004346847534\\
32.8330001831055 14.7004346847534\\
32.882999420166 14.7004346847534\\
32.9329986572266 14.7004346847534\\
32.9829978942871 14.7004346847534\\
33.0330009460449 14.7004346847534\\
33.0830001831055 14.7004346847534\\
33.132999420166 14.7004346847534\\
33.1829986572266 14.7004346847534\\
33.2329978942871 14.7004346847534\\
33.2830009460449 14.7004346847534\\
33.3330001831055 14.7004346847534\\
33.382999420166 14.7004346847534\\
33.4329986572266 14.7004346847534\\
33.4829978942871 14.7004346847534\\
33.5330009460449 14.7004346847534\\
33.5830001831055 14.7004346847534\\
33.632999420166 14.7004346847534\\
33.6829986572266 14.7004346847534\\
33.7329978942871 14.7004346847534\\
33.7830009460449 14.7004346847534\\
33.8330001831055 14.7004346847534\\
33.882999420166 14.7004346847534\\
33.9329986572266 14.7004346847534\\
33.9829978942871 14.7004346847534\\
34.0330009460449 14.7004346847534\\
34.0830001831055 14.7004346847534\\
34.132999420166 14.7004346847534\\
34.1829986572266 14.7004346847534\\
34.2329978942871 14.7004346847534\\
34.2830009460449 14.7004346847534\\
34.3330001831055 14.7004346847534\\
34.382999420166 14.7004346847534\\
34.4329986572266 14.7004346847534\\
34.4829978942871 14.7004346847534\\
34.5330009460449 14.7004346847534\\
34.5830001831055 14.7004346847534\\
34.632999420166 14.7004346847534\\
34.6829986572266 14.7004346847534\\
34.7329978942871 14.7004346847534\\
34.7830009460449 14.7004346847534\\
34.8330001831055 14.7004346847534\\
34.882999420166 14.7004346847534\\
34.9329986572266 14.7004346847534\\
34.9829978942871 14.7004346847534\\
35.0330009460449 14.7004346847534\\
35.0830001831055 14.7004346847534\\
35.132999420166 14.7004346847534\\
35.1829986572266 14.7004346847534\\
35.2329978942871 14.7004346847534\\
35.2830009460449 14.7004346847534\\
35.3330001831055 14.7004346847534\\
35.382999420166 14.7004346847534\\
35.4329986572266 14.7004346847534\\
35.4829978942871 14.7004346847534\\
35.5330009460449 14.7004346847534\\
35.5830001831055 14.7004346847534\\
35.632999420166 14.7004346847534\\
35.6829986572266 14.7004346847534\\
35.7329978942871 14.7004346847534\\
35.7830009460449 14.7004346847534\\
35.8330001831055 14.7004346847534\\
35.882999420166 14.7004346847534\\
35.9329986572266 14.7004346847534\\
35.9829978942871 14.7004346847534\\
36.0330009460449 14.7004346847534\\
36.0830001831055 14.7004346847534\\
36.132999420166 14.7004346847534\\
36.1829986572266 14.7004346847534\\
36.2329978942871 14.7004346847534\\
36.2830009460449 14.7004346847534\\
36.3330001831055 14.7004346847534\\
36.382999420166 14.7004346847534\\
36.4329986572266 14.7004346847534\\
36.4829978942871 14.7004346847534\\
36.5330009460449 14.7004346847534\\
36.5830001831055 14.7004346847534\\
36.632999420166 14.7004346847534\\
36.6829986572266 14.7004346847534\\
36.7329978942871 14.7004346847534\\
36.7830009460449 14.7004346847534\\
36.8330001831055 14.7004346847534\\
36.882999420166 14.7004346847534\\
36.9329986572266 14.7004356384277\\
36.9829978942871 14.7004346847534\\
37.0330009460449 14.7004337310791\\
37.0830001831055 14.7004346847534\\
37.132999420166 14.7004346847534\\
37.1829986572266 14.7004346847534\\
37.2329978942871 14.7004346847534\\
37.2830009460449 14.7004346847534\\
37.3330001831055 14.7004346847534\\
37.382999420166 14.7004346847534\\
37.4329986572266 14.7004346847534\\
37.4829978942871 14.7004346847534\\
37.5330009460449 14.7004346847534\\
37.5830001831055 14.7004346847534\\
37.632999420166 14.7004346847534\\
37.6829986572266 14.7004346847534\\
37.7329978942871 14.7004346847534\\
37.7830009460449 14.7004346847534\\
37.8330001831055 14.7004346847534\\
37.882999420166 14.7004346847534\\
37.9329986572266 14.7004346847534\\
37.9829978942871 14.7004346847534\\
38.0330009460449 14.7004346847534\\
38.0830001831055 14.7004346847534\\
38.132999420166 14.7004346847534\\
38.1829986572266 14.7004346847534\\
38.2329978942871 14.7004346847534\\
38.2830009460449 14.7004346847534\\
38.3330001831055 14.7004346847534\\
38.382999420166 14.7004346847534\\
38.4329986572266 14.7004346847534\\
38.4829978942871 14.7004346847534\\
38.5330009460449 14.7004346847534\\
38.5830001831055 14.7004346847534\\
38.632999420166 14.7004346847534\\
38.6829986572266 14.7004346847534\\
38.7329978942871 14.7004346847534\\
38.7830009460449 14.7004346847534\\
38.8330001831055 14.7004346847534\\
38.882999420166 14.7004346847534\\
38.9329986572266 14.7004346847534\\
38.9829978942871 14.7004346847534\\
39.0330009460449 14.7004346847534\\
39.0830001831055 14.7004346847534\\
39.132999420166 14.7004346847534\\
39.1829986572266 14.7004346847534\\
39.2329978942871 14.7004346847534\\
39.2830009460449 14.7004346847534\\
39.3330001831055 14.7004346847534\\
39.382999420166 14.7004346847534\\
39.4329986572266 14.7004346847534\\
39.4829978942871 14.7004346847534\\
39.5330009460449 14.7004346847534\\
39.5830001831055 14.7004346847534\\
39.632999420166 14.7004346847534\\
39.6829986572266 14.7004346847534\\
39.7329978942871 14.7004346847534\\
39.7830009460449 14.7004346847534\\
39.8330001831055 14.7004346847534\\
39.882999420166 14.7004346847534\\
39.9329986572266 14.7004346847534\\
39.9829978942871 14.7004346847534\\
40.0330009460449 14.7004346847534\\
40.0830001831055 14.7004346847534\\
40.132999420166 14.7004346847534\\
40.1829986572266 14.7004346847534\\
40.2329978942871 14.7004346847534\\
40.2830009460449 14.7004346847534\\
40.3330001831055 14.7004346847534\\
40.382999420166 14.7004346847534\\
40.4329986572266 14.7004346847534\\
40.4829978942871 14.7004346847534\\
40.5330009460449 14.7004346847534\\
40.5830001831055 14.7004346847534\\
40.632999420166 14.7004346847534\\
40.6829986572266 14.7004346847534\\
40.7329978942871 14.7004346847534\\
40.7830009460449 14.7004346847534\\
40.8330001831055 14.7004346847534\\
40.882999420166 14.7004346847534\\
40.9329986572266 14.7004346847534\\
40.9829978942871 14.7004346847534\\
41.0330009460449 14.7004346847534\\
41.0830001831055 14.7004346847534\\
41.132999420166 14.7004346847534\\
41.1829986572266 14.7004346847534\\
41.2329978942871 14.7004346847534\\
41.2830009460449 14.7004346847534\\
41.3330001831055 14.7004346847534\\
41.382999420166 14.7004346847534\\
41.4329986572266 14.7004346847534\\
41.4829978942871 14.7004346847534\\
41.5330009460449 14.7004346847534\\
41.5830001831055 14.7004346847534\\
41.632999420166 14.7004346847534\\
41.6829986572266 14.7004346847534\\
41.7329978942871 14.7004346847534\\
41.7830009460449 14.7004346847534\\
41.8330001831055 14.7004346847534\\
41.882999420166 14.7004346847534\\
41.9329986572266 14.7004346847534\\
41.9829978942871 14.7004346847534\\
42.0330009460449 14.7004346847534\\
42.0830001831055 14.7004346847534\\
42.132999420166 14.7004346847534\\
42.1829986572266 14.7004346847534\\
42.2329978942871 14.7004346847534\\
42.2830009460449 14.7004346847534\\
42.3330001831055 14.7004346847534\\
42.382999420166 14.7004346847534\\
42.4329986572266 14.7004346847534\\
42.4829978942871 14.7004346847534\\
42.5330009460449 14.7004346847534\\
42.5830001831055 14.7004346847534\\
42.632999420166 14.7004346847534\\
42.6829986572266 14.7004337310791\\
42.7329978942871 14.7004337310791\\
42.7830009460449 14.7004346847534\\
42.8330001831055 14.7004337310791\\
42.882999420166 14.7004327774048\\
42.9329986572266 14.7004318237305\\
42.9829978942871 14.7004318237305\\
43.0330009460449 14.7004318237305\\
43.0830001831055 14.7004375457764\\
43.132999420166 14.7004346847534\\
43.1829986572266 14.7004346847534\\
43.2329978942871 14.7004346847534\\
43.2830009460449 14.7004365921021\\
43.3330001831055 14.7004356384277\\
43.382999420166 14.7004356384277\\
43.4329986572266 14.7004346847534\\
43.4829978942871 14.7004346847534\\
43.5330009460449 14.7004346847534\\
43.5830001831055 14.7004346847534\\
43.632999420166 14.7004346847534\\
43.6829986572266 14.7004346847534\\
43.7329978942871 14.7004346847534\\
43.7830009460449 14.7004346847534\\
43.8330001831055 14.7004346847534\\
43.882999420166 14.7004346847534\\
43.9329986572266 14.7004346847534\\
43.9829978942871 14.7004346847534\\
44.0330009460449 14.7004346847534\\
44.0830001831055 14.7004346847534\\
44.132999420166 14.7004346847534\\
44.1829986572266 14.7004346847534\\
44.2329978942871 14.7004346847534\\
44.2830009460449 14.7004346847534\\
44.3330001831055 14.7004346847534\\
44.382999420166 14.7004346847534\\
44.4329986572266 14.7004346847534\\
44.4829978942871 14.7004346847534\\
44.5330009460449 14.7004346847534\\
44.5830001831055 14.7004346847534\\
44.632999420166 14.7004346847534\\
44.6829986572266 14.7004346847534\\
44.7329978942871 14.7004346847534\\
44.7830009460449 14.7004346847534\\
44.8330001831055 14.7004346847534\\
44.882999420166 14.7004346847534\\
44.9329986572266 14.7004346847534\\
44.9829978942871 14.7004346847534\\
45.0330009460449 14.7004346847534\\
45.0830001831055 14.7004346847534\\
45.132999420166 14.7004346847534\\
45.1829986572266 14.7004346847534\\
45.2329978942871 14.7004346847534\\
45.2830009460449 14.7004346847534\\
45.3330001831055 14.7004346847534\\
45.382999420166 14.7004346847534\\
45.4329986572266 14.7004346847534\\
45.4829978942871 14.7004346847534\\
45.5330009460449 14.7004346847534\\
45.5830001831055 14.7004346847534\\
45.632999420166 14.7004346847534\\
45.6829986572266 14.7004346847534\\
45.7329978942871 14.7004346847534\\
45.7830009460449 14.7004346847534\\
45.8330001831055 14.7004346847534\\
45.882999420166 14.7004346847534\\
45.9329986572266 14.7004346847534\\
45.9829978942871 14.7004346847534\\
46.0330009460449 14.7004346847534\\
46.0830001831055 14.7004346847534\\
46.132999420166 14.7004346847534\\
46.1829986572266 14.7004346847534\\
46.2329978942871 14.7004346847534\\
46.2830009460449 14.7004346847534\\
46.3330001831055 14.7004346847534\\
46.382999420166 14.7004346847534\\
46.4329986572266 14.7004346847534\\
46.4829978942871 14.7004346847534\\
46.5330009460449 14.7004346847534\\
46.5830001831055 14.7004346847534\\
46.632999420166 14.7004346847534\\
46.6829986572266 14.7004346847534\\
46.7329978942871 14.7004346847534\\
46.7830009460449 14.7004346847534\\
46.8330001831055 14.7004346847534\\
46.882999420166 14.7004346847534\\
46.9329986572266 14.7004346847534\\
46.9829978942871 14.7004346847534\\
47.0330009460449 14.7004346847534\\
47.0830001831055 14.7004346847534\\
47.132999420166 14.7004346847534\\
47.1829986572266 14.7004346847534\\
47.2329978942871 14.7004346847534\\
47.2830009460449 14.7004346847534\\
47.3330001831055 14.7004346847534\\
47.382999420166 14.7004346847534\\
47.4329986572266 14.7004346847534\\
47.4829978942871 14.7004346847534\\
47.5330009460449 14.7004346847534\\
47.5830001831055 14.7004346847534\\
47.632999420166 14.7004346847534\\
47.6829986572266 14.7004346847534\\
47.7329978942871 14.7004346847534\\
47.7830009460449 14.7004346847534\\
47.8330001831055 14.7004346847534\\
47.882999420166 14.7004346847534\\
47.9329986572266 14.7004346847534\\
47.9829978942871 14.7004346847534\\
48.0330009460449 14.7004346847534\\
48.0830001831055 14.7004346847534\\
48.132999420166 14.7004346847534\\
48.1829986572266 14.7004346847534\\
48.2329978942871 14.7004346847534\\
48.2830009460449 14.7004346847534\\
48.3330001831055 14.7004346847534\\
48.382999420166 14.7004346847534\\
48.4329986572266 14.7004346847534\\
48.4829978942871 14.7004346847534\\
48.5330009460449 14.7004346847534\\
48.5830001831055 14.7004346847534\\
48.632999420166 14.7004346847534\\
48.6829986572266 14.7004346847534\\
48.7329978942871 14.7004346847534\\
48.7830009460449 14.7004346847534\\
48.8330001831055 14.7004346847534\\
48.882999420166 14.7004346847534\\
48.9329986572266 14.7004337310791\\
48.9829978942871 14.7004337310791\\
49.0330009460449 14.7004337310791\\
49.0830001831055 14.7004337310791\\
49.132999420166 14.7004337310791\\
49.1829986572266 14.7004337310791\\
49.2329978942871 14.7004337310791\\
49.2830009460449 14.7004337310791\\
49.3330001831055 14.7004337310791\\
49.382999420166 14.7004346847534\\
49.4329986572266 14.7004346847534\\
49.4829978942871 14.7004346847534\\
49.5330009460449 14.7004346847534\\
49.5830001831055 14.7004356384277\\
49.632999420166 14.7004356384277\\
49.6829986572266 14.7004356384277\\
49.7329978942871 14.7004356384277\\
49.7830009460449 14.7004356384277\\
49.8330001831055 14.7004356384277\\
49.882999420166 14.7004356384277\\
49.9329986572266 14.7004356384277\\
49.9829978942871 14.7004356384277\\
50.0330009460449 14.7004356384277\\
50.0830001831055 14.7004356384277\\
50.132999420166 14.7004356384277\\
50.1829986572266 14.7004356384277\\
50.2329978942871 14.7004356384277\\
50.2830009460449 14.7004356384277\\
50.3330001831055 14.7004356384277\\
50.382999420166 14.7004356384277\\
50.4329986572266 14.7004356384277\\
50.4829978942871 14.7004356384277\\
50.5330009460449 14.7004356384277\\
50.5830001831055 14.7004356384277\\
50.632999420166 14.7004356384277\\
50.6829986572266 14.7004356384277\\
50.7329978942871 14.7004356384277\\
50.7830009460449 14.7004356384277\\
50.8330001831055 14.7004356384277\\
50.882999420166 14.7004356384277\\
50.9329986572266 14.7004356384277\\
50.9829978942871 14.7004356384277\\
51.0330009460449 14.7004356384277\\
51.0830001831055 14.7004356384277\\
51.132999420166 14.7004356384277\\
51.1829986572266 14.7004356384277\\
51.2329978942871 14.7004356384277\\
51.2830009460449 14.7004356384277\\
51.3330001831055 14.7004356384277\\
51.382999420166 14.7004356384277\\
51.4329986572266 14.7004356384277\\
51.4829978942871 14.7004356384277\\
51.5330009460449 14.7004356384277\\
51.5830001831055 14.7004356384277\\
51.632999420166 14.7004356384277\\
51.6829986572266 14.7004356384277\\
51.7329978942871 14.7004356384277\\
51.7830009460449 14.7004356384277\\
51.8330001831055 14.7004356384277\\
51.882999420166 14.7004356384277\\
51.9329986572266 14.7004356384277\\
51.9829978942871 14.7004356384277\\
52.0330009460449 14.7004356384277\\
52.0830001831055 14.7004356384277\\
52.132999420166 14.7004356384277\\
52.1829986572266 14.7004356384277\\
52.2329978942871 14.7004356384277\\
52.2830009460449 14.7004356384277\\
52.3330001831055 14.7004356384277\\
52.382999420166 14.7004356384277\\
52.4329986572266 14.7004356384277\\
52.4829978942871 14.7004356384277\\
52.5330009460449 14.7004356384277\\
52.5830001831055 14.7004356384277\\
52.632999420166 14.7004356384277\\
52.6829986572266 14.7004356384277\\
52.7329978942871 14.7004356384277\\
52.7830009460449 14.7004356384277\\
52.8330001831055 14.7004356384277\\
52.882999420166 14.7004356384277\\
52.9329986572266 14.7004356384277\\
52.9829978942871 14.7004356384277\\
53.0330009460449 14.7004356384277\\
53.0830001831055 14.7004356384277\\
53.132999420166 14.7004356384277\\
53.1829986572266 14.7004356384277\\
53.2329978942871 14.7004356384277\\
53.2830009460449 14.7004356384277\\
53.3330001831055 14.7004356384277\\
53.382999420166 14.7004356384277\\
53.4329986572266 14.7004356384277\\
53.4829978942871 14.7004356384277\\
53.5330009460449 14.7004356384277\\
53.5830001831055 14.7004356384277\\
53.632999420166 14.7004356384277\\
53.6829986572266 14.7004356384277\\
53.7329978942871 14.7004356384277\\
53.7830009460449 14.7004356384277\\
53.8330001831055 14.7004356384277\\
53.882999420166 14.7004356384277\\
53.9329986572266 14.7004356384277\\
53.9829978942871 14.7004356384277\\
54.0330009460449 14.7004356384277\\
54.0830001831055 14.7004356384277\\
54.132999420166 14.7004356384277\\
54.1829986572266 14.7004356384277\\
54.2329978942871 14.7004356384277\\
54.2830009460449 14.7004356384277\\
54.3330001831055 14.7004356384277\\
54.382999420166 14.7004356384277\\
54.4329986572266 14.7004356384277\\
54.4829978942871 14.7004356384277\\
54.5330009460449 14.7004365921021\\
54.5830001831055 14.7004375457764\\
54.632999420166 14.700439453125\\
54.6829986572266 14.7004432678223\\
54.7329978942871 14.7004537582397\\
54.7830009460449 14.7004737854004\\
54.8330001831055 14.7005186080933\\
54.882999420166 14.7006235122681\\
54.9329986572266 14.7008695602417\\
54.9829978942871 14.7014398574829\\
55.0330009460449 14.7027311325073\\
55.0830001831055 14.7054481506348\\
55.132999420166 14.710503578186\\
55.1829986572266 14.7182159423828\\
55.2329978942871 14.7272167205811\\
55.2830009460449 14.7362689971924\\
55.3330001831055 14.7315301895142\\
55.382999420166 14.732123374939\\
55.4329986572266 14.7324323654175\\
55.4829978942871 14.732572555542\\
55.5330009460449 14.732346534729\\
55.5830001831055 14.7328872680664\\
55.632999420166 14.7323513031006\\
55.6829986572266 14.7326984405518\\
55.7329978942871 14.7327632904053\\
55.7830009460449 14.7331085205078\\
55.8330001831055 14.7333374023438\\
55.882999420166 14.7336206436157\\
55.9329986572266 14.7338771820068\\
55.9829978942871 14.7341470718384\\
56.0330009460449 14.734411239624\\
56.0830001831055 14.7346773147583\\
56.132999420166 14.7349424362183\\
56.1829986572266 14.7352085113525\\
56.2329978942871 14.7354736328125\\
56.2830009460449 14.7357397079468\\
56.3330001831055 14.7360057830811\\
56.382999420166 14.736270904541\\
56.4329986572266 14.7365369796753\\
56.4829978942871 14.7368030548096\\
56.5330009460449 14.7370681762695\\
56.5830001831055 14.7373342514038\\
56.632999420166 14.7376003265381\\
56.6829986572266 14.7378664016724\\
56.7329978942871 14.7381324768066\\
56.7830009460449 14.7383985519409\\
56.8330001831055 14.7386646270752\\
56.882999420166 14.7389307022095\\
56.9329986572266 14.7391967773438\\
56.9829978942871 14.739462852478\\
57.0330009460449 14.7397289276123\\
57.0830001831055 14.7399950027466\\
57.132999420166 14.7402610778809\\
57.1829986572266 14.7405281066895\\
57.2329978942871 14.7407941818237\\
57.2830009460449 14.741060256958\\
57.3330001831055 14.7413272857666\\
57.382999420166 14.7415943145752\\
57.4329986572266 14.7418603897095\\
57.4829978942871 14.7421274185181\\
57.5330009460449 14.7423944473267\\
57.5830001831055 14.7426614761353\\
57.632999420166 14.7429285049438\\
57.6829986572266 14.7431955337524\\
57.7329978942871 14.743462562561\\
57.7830009460449 14.7437648773193\\
57.8330001831055 14.7442874908447\\
57.882999420166 14.7446117401123\\
57.9329986572266 14.7449407577515\\
57.9829978942871 14.7452611923218\\
58.0330009460449 14.7455816268921\\
58.0830001831055 14.7459020614624\\
58.132999420166 14.7462224960327\\
58.1829986572266 14.7465419769287\\
58.2329978942871 14.7468614578247\\
58.2830009460449 14.7471799850464\\
58.3330001831055 14.7474994659424\\
58.382999420166 14.7478179931641\\
58.4329986572266 14.7481365203857\\
58.4829978942871 14.7484540939331\\
58.5330009460449 14.7487726211548\\
58.5830001831055 14.7490901947021\\
58.632999420166 14.7494087219238\\
58.6829986572266 14.7497262954712\\
58.7329978942871 14.7500438690186\\
58.7830009460449 14.7503614425659\\
58.8330001831055 14.750678062439\\
58.882999420166 14.7509956359863\\
58.9329986572266 14.7513132095337\\
58.9829978942871 14.7516307830811\\
59.0330009460449 14.7519483566284\\
59.0830001831055 14.7522659301758\\
59.132999420166 14.7525835037231\\
59.1829986572266 14.7529010772705\\
59.2329978942871 14.7532186508179\\
59.2830009460449 14.7535200119019\\
59.3330001831055 14.7537460327148\\
59.382999420166 14.7538366317749\\
59.4329986572266 14.7541275024414\\
59.4829978942871 14.7543897628784\\
59.5330009460449 14.7546615600586\\
59.5830001831055 14.7549295425415\\
59.632999420166 14.7551975250244\\
59.6829986572266 14.7554655075073\\
59.7329978942871 14.7557315826416\\
59.7830009460449 14.7559976577759\\
59.8330001831055 14.7562627792358\\
59.882999420166 14.7565279006958\\
59.9329986572266 14.7567911148071\\
59.9664993286133 14.7569971084595\\
60 14.7572288513184\\
60.0368499755859 14.7574110031128\\
60.0773849487305 14.757604598999\\
60.1219749450684 14.7578163146973\\
60.1710243225098 14.7580490112305\\
60.2210235595703 14.7582921981812\\
60.2710227966309 14.7585487365723\\
60.3210220336914 14.7588043212891\\
60.371021270752 14.7590751647949\\
60.4210243225098 14.7593879699707\\
60.4710235595703 14.7596797943115\\
60.5210227966309 14.7599563598633\\
60.5710220336914 14.7603464126587\\
60.621021270752 14.7606678009033\\
60.6710243225098 14.7609624862671\\
60.7210235595703 14.7612705230713\\
60.7710227966309 14.7615833282471\\
60.8210220336914 14.7619018554688\\
60.871021270752 14.762228012085\\
60.9210243225098 14.7625646591187\\
60.9710235595703 14.7629127502441\\
61.0210227966309 14.7632761001587\\
61.0710220336914 14.7636585235596\\
61.121021270752 14.7640657424927\\
61.1710243225098 14.7646999359131\\
61.2210235595703 14.7704296112061\\
61.2710227966309 14.7720718383789\\
61.3210220336914 14.7738084793091\\
61.371021270752 14.7755517959595\\
61.4210243225098 14.7729339599609\\
61.4710235595703 14.7680797576904\\
61.5210227966309 14.7683343887329\\
61.5710220336914 14.7685499191284\\
61.621021270752 14.768837928772\\
61.6710243225098 14.7691802978516\\
61.7210235595703 14.7695722579956\\
61.7710227966309 14.7700071334839\\
61.8210220336914 14.7704734802246\\
61.871021270752 14.7709617614746\\
61.9210243225098 14.7714624404907\\
61.9710235595703 14.7719736099243\\
62.0210227966309 14.7724914550781\\
62.0710220336914 14.7730150222778\\
62.121021270752 14.7735433578491\\
62.1710243225098 14.7740755081177\\
62.2210235595703 14.7746105194092\\
62.2710227966309 14.7751502990723\\
62.3210220336914 14.7756929397583\\
62.371021270752 14.77623462677\\
62.4210243225098 14.776780128479\\
62.4710235595703 14.7773265838623\\
62.5210227966309 14.7778759002686\\
62.5710220336914 14.7784271240234\\
62.621021270752 14.7789793014526\\
62.6710243225098 14.7795324325562\\
62.7210235595703 14.7800874710083\\
62.7710227966309 14.7806425094604\\
62.8210220336914 14.7811985015869\\
62.871021270752 14.7817554473877\\
62.9210243225098 14.7823114395142\\
62.9710235595703 14.7828683853149\\
63.0210227966309 14.7834243774414\\
63.0710220336914 14.7839813232422\\
63.121021270752 14.784538269043\\
63.1710243225098 14.7850952148438\\
63.2210235595703 14.7856512069702\\
63.2710227966309 14.786208152771\\
63.3210220336914 14.7867650985718\\
63.371021270752 14.7873220443726\\
63.4210243225098 14.7878789901733\\
63.4710235595703 14.7884359359741\\
63.5210227966309 14.7889928817749\\
63.5710220336914 14.7895498275757\\
63.621021270752 14.7901067733765\\
63.6710243225098 14.7906637191772\\
63.7210235595703 14.791220664978\\
63.7710227966309 14.7917776107788\\
63.8210220336914 14.7923345565796\\
63.871021270752 14.7928915023804\\
63.9210243225098 14.7934484481812\\
63.9710235595703 14.7940053939819\\
64.0210189819336 14.7945623397827\\
64.0710220336914 14.7951202392578\\
64.1210250854492 14.7956771850586\\
64.1710205078125 14.7962341308594\\
64.2210235595703 14.7967910766602\\
64.2710189819336 14.7973489761353\\
64.3210220336914 14.797905921936\\
64.3710250854492 14.7984628677368\\
64.4210205078125 14.7990207672119\\
64.4710235595703 14.7995777130127\\
64.5210189819336 14.8001356124878\\
64.5710220336914 14.8006925582886\\
64.6210250854492 14.8012504577637\\
64.6710205078125 14.8018074035645\\
64.7210235595703 14.8023653030396\\
64.7710189819336 14.8029232025146\\
64.8210220336914 14.8034801483154\\
64.8710250854492 14.8040380477905\\
64.9210205078125 14.8045959472656\\
64.9710235595703 14.8051538467407\\
65.0210189819336 14.8057117462158\\
65.0710220336914 14.8062686920166\\
65.1210250854492 14.8068265914917\\
65.1710205078125 14.8073844909668\\
65.2210235595703 14.8079423904419\\
65.2710189819336 14.808500289917\\
65.3210220336914 14.8090581893921\\
65.3710250854492 14.8096170425415\\
65.4210205078125 14.8101749420166\\
65.4710235595703 14.8107328414917\\
65.5210189819336 14.8112907409668\\
65.5710220336914 14.8118486404419\\
65.6210250854492 14.8124074935913\\
65.6710205078125 14.8129787445068\\
65.7210235595703 14.8135204315186\\
65.7710189819336 14.8140802383423\\
65.8210220336914 14.8146409988403\\
65.8710250854492 14.8152017593384\\
65.9210205078125 14.8157625198364\\
65.9710235595703 14.8163232803345\\
66.0210189819336 14.8168840408325\\
66.0710220336914 14.8174448013306\\
66.1210250854492 14.8180055618286\\
66.1710205078125 14.818567276001\\
66.2210235595703 14.819128036499\\
66.2710189819336 14.8196887969971\\
66.3210220336914 14.8202495574951\\
66.3710250854492 14.8208112716675\\
66.4210205078125 14.8213720321655\\
66.4710235595703 14.8219327926636\\
66.5210189819336 14.8224935531616\\
66.5710220336914 14.823055267334\\
66.6210250854492 14.823616027832\\
66.6710205078125 14.8241767883301\\
66.7210235595703 14.8247385025024\\
66.7710189819336 14.8253002166748\\
66.8210220336914 14.8258609771729\\
66.8710250854492 14.8264236450195\\
66.9210205078125 14.8269853591919\\
66.9710235595703 14.8275527954102\\
67.0210189819336 14.8281230926514\\
67.0710220336914 14.8287076950073\\
67.1210250854492 14.8293218612671\\
67.1710205078125 14.8300075531006\\
67.2210235595703 14.8308610916138\\
67.2710189819336 14.832085609436\\
67.3210220336914 14.834041595459\\
67.3710250854492 14.8371639251709\\
67.4210205078125 14.8414402008057\\
67.4710235595703 14.8457326889038\\
67.5210189819336 14.85205078125\\
67.5710220336914 14.84033203125\\
67.6210250854492 14.8394575119019\\
67.6710205078125 14.8390846252441\\
67.7210235595703 14.8388223648071\\
67.7710189819336 14.8382349014282\\
67.8210220336914 14.8391790390015\\
67.8710250854492 14.8386316299438\\
67.9210205078125 14.8389387130737\\
67.9710235595703 14.8393650054932\\
68.0210189819336 14.8399705886841\\
68.0710220336914 14.8405122756958\\
68.1210250854492 14.8410835266113\\
68.1710205078125 14.8416404724121\\
68.2210235595703 14.8422050476074\\
68.2710189819336 14.8427658081055\\
68.3210220336914 14.8433275222778\\
68.3710250854492 14.8438892364502\\
68.4210205078125 14.8444509506226\\
68.4710235595703 14.8450126647949\\
68.5210189819336 14.8455743789673\\
68.5710220336914 14.846137046814\\
68.6210250854492 14.8466987609863\\
68.6710205078125 14.8472604751587\\
68.7210235595703 14.8478221893311\\
68.7710189819336 14.8483848571777\\
68.8210220336914 14.8489465713501\\
68.8710250854492 14.8495082855225\\
68.9210205078125 14.8500709533691\\
68.9710235595703 14.8506326675415\\
69.0210189819336 14.8511953353882\\
69.0710220336914 14.8517570495605\\
69.1210250854492 14.8523187637329\\
69.1710205078125 14.8528814315796\\
69.2210235595703 14.853443145752\\
69.2710189819336 14.8540058135986\\
69.3210220336914 14.854567527771\\
69.3710250854492 14.8551301956177\\
69.4210205078125 14.85569190979\\
69.4710235595703 14.8562545776367\\
69.5210189819336 14.8568162918091\\
69.5710220336914 14.8573789596558\\
69.6210250854492 14.8579416275024\\
69.6710205078125 14.8585033416748\\
69.7210235595703 14.8590660095215\\
69.7710189819336 14.8596277236938\\
69.8210220336914 14.8601903915405\\
69.8710250854492 14.8607530593872\\
69.9210205078125 14.8613147735596\\
69.9710235595703 14.8618774414063\\
70.0210189819336 14.8624401092529\\
70.0710220336914 14.8632335662842\\
70.1210250854492 14.8638515472412\\
70.1710205078125 14.8644227981567\\
70.2210235595703 14.8649864196777\\
70.2710189819336 14.8655490875244\\
70.3210220336914 14.8661127090454\\
70.3710250854492 14.8666753768921\\
70.4210205078125 14.8672380447388\\
70.4710235595703 14.8678007125854\\
70.5210189819336 14.8683633804321\\
70.5710220336914 14.8689270019531\\
70.6210250854492 14.8694896697998\\
70.6710205078125 14.8700532913208\\
70.7210235595703 14.8706169128418\\
70.7710189819336 14.8711805343628\\
70.8210220336914 14.8717432022095\\
70.8710250854492 14.8723068237305\\
70.9210205078125 14.8728704452515\\
70.9710235595703 14.8734350204468\\
71.0210189819336 14.8739986419678\\
71.0710220336914 14.8745622634888\\
71.1210250854492 14.8751268386841\\
71.1710205078125 14.8756914138794\\
71.2210235595703 14.8762550354004\\
71.2710189819336 14.8768196105957\\
71.3210220336914 14.8773851394653\\
71.3710250854492 14.8779497146606\\
71.4210205078125 14.878514289856\\
71.4710235595703 14.8790798187256\\
71.5210189819336 14.879638671875\\
71.5710220336914 14.8801746368408\\
71.6210250854492 14.8804731369019\\
71.6710205078125 14.8810415267944\\
71.7210235595703 14.881591796875\\
71.7710189819336 14.882152557373\\
71.8210220336914 14.8827114105225\\
71.8710250854492 14.8832702636719\\
71.9210205078125 14.883828163147\\
71.9710235595703 14.8843870162964\\
72.0210189819336 14.8849449157715\\
72.0710220336914 14.8855028152466\\
72.1210250854492 14.8860607147217\\
72.1710205078125 14.8866186141968\\
72.2210235595703 14.8871755599976\\
72.2710189819336 14.8877334594727\\
72.3210220336914 14.8882904052734\\
72.3710250854492 14.8888473510742\\
72.4210205078125 14.8894033432007\\
72.4710235595703 14.8899602890015\\
72.5210189819336 14.8905162811279\\
72.5710220336914 14.8910713195801\\
72.6210250854492 14.8916397094727\\
72.6710205078125 14.8922519683838\\
72.7210235595703 14.8928365707397\\
72.7710189819336 14.8934030532837\\
72.8210220336914 14.8940725326538\\
72.8710250854492 14.8946685791016\\
72.9210205078125 14.8952350616455\\
72.9710235595703 14.8958120346069\\
73.0210189819336 14.896390914917\\
73.0710220336914 14.8969745635986\\
73.1210250854492 14.8975629806519\\
73.1710205078125 14.898157119751\\
73.2210235595703 14.8987588882446\\
73.2710189819336 14.8993711471558\\
73.3210220336914 14.8999967575073\\
73.3710250854492 14.9006385803223\\
73.4210205078125 14.9013042449951\\
73.4710235595703 14.9071779251099\\
73.5210189819336 14.9086084365845\\
73.5710220336914 14.9100303649902\\
73.6210250854492 14.9115686416626\\
73.6710205078125 14.9096813201904\\
73.7210235595703 14.9044313430786\\
73.7710189819336 14.9048480987549\\
73.8210220336914 14.9052152633667\\
73.8710250854492 14.9056119918823\\
73.9210205078125 14.9060459136963\\
73.9710235595703 14.906512260437\\
74.0210189819336 14.9070110321045\\
74.0710220336914 14.9075317382813\\
74.1210250854492 14.9080667495728\\
74.1710205078125 14.9086093902588\\
74.2210235595703 14.9091558456421\\
74.2710189819336 14.9097051620483\\
74.3210220336914 14.9102563858032\\
74.3710250854492 14.9108095169067\\
74.4210205078125 14.9113636016846\\
74.4710235595703 14.9119186401367\\
74.5210189819336 14.9124755859375\\
74.5710220336914 14.9130334854126\\
74.6210250854492 14.9135904312134\\
74.6710205078125 14.9141464233398\\
74.7210235595703 14.9147033691406\\
74.7710189819336 14.91526222229\\
74.8210220336914 14.9158210754395\\
74.8710250854492 14.9163818359375\\
74.9210205078125 14.9169416427612\\
74.9605102539063 14.9175033569336\\
75 14.9179458618164\\
};
\addlegendentry{18 "};

\addplot [
color=green,
dash pattern=on 1pt off 3pt on 3pt off 3pt
]
table[row sep=crcr]{
0 14.7003479003906\\
9.99999971718069e-10 14.7003479003906\\
2.49999998480632e-09 14.7003479003906\\
4.74999994892755e-09 14.7003479003906\\
8.1250002281763e-09 14.7003479003906\\
1.31874999809156e-08 14.7003479003906\\
2.0781250498203e-08 14.7003479003906\\
3.21718758300449e-08 14.7003479003906\\
4.92578138278077e-08 14.7003479003906\\
7.48867208244519e-08 14.7003479003906\\
1.13330081319418e-07 14.7003479003906\\
1.70995122061868e-07 14.7003479003906\\
2.57492672517401e-07 14.7003479003906\\
3.8723902662241e-07 14.7003479003906\\
5.81858500936505e-07 14.7003479003906\\
8.73787769251066e-07 14.7003479003906\\
1.31168167172291e-06 14.7003479003906\\
1.9685226106958e-06 14.7003479003906\\
2.95378367809462e-06 14.7003479003906\\
4.43167573394021e-06 14.7003479003906\\
6.64851359033491e-06 14.7003479003906\\
9.97376992017962e-06 14.7003479003906\\
1.49616553244414e-05 14.7003479003906\\
2.24434825213393e-05 14.7003479003906\\
3.36662233166862e-05 14.7003479003906\\
5.05003372381907e-05 14.7003479003906\\
7.57515081204474e-05 14.7003479003906\\
0.000113628258986864 14.7003479003906\\
0.000170443381648511 14.7003479003906\\
0.000255666091106832 14.7003488540649\\
0.000383500126190484 14.7003498077393\\
0.000575251178815961 14.7003507614136\\
0.000862877757754177 14.7003536224365\\
0.00129431765526533 14.7003536224365\\
0.00194147753063589 14.7003507614136\\
0.00291221728548408 14.7003469467163\\
0.00436832662671804 14.7003450393677\\
0.00597004732117057 14.700345993042\\
0.00773194013163447 14.7003479003906\\
0.009670021943748 14.7003479003906\\
0.0118019115179777 14.7003479003906\\
0.0141469910740852 14.7003479003906\\
0.0167265776544809 14.7003479003906\\
0.0195641238242388 14.7003479003906\\
0.0226854234933853 14.7003479003906\\
0.0261188540607691 14.7003479003906\\
0.0298956278711557 14.7003479003906\\
0.0340500771999359 14.7003479003906\\
0.0386199727654457 14.7003479003906\\
0.0436468608677387 14.7003479003906\\
0.0491764321923256 14.7003479003906\\
0.0552589632570744 14.7003479003906\\
0.0619497485458851 14.7003479003906\\
0.0693096145987511 14.7003479003906\\
0.0774054601788521 14.7003479003906\\
0.0863108932971954 14.7003469467163\\
0.0961068719625473 14.7003479003906\\
0.106882445514202 14.7003479003906\\
0.118735581636429 14.7003479003906\\
0.13177402317524 14.7003479003906\\
0.146116316318512 14.7003479003906\\
0.161892831325531 14.7003479003906\\
0.179247006773949 14.7003479003906\\
0.198336601257324 14.7003479003906\\
0.21933513879776 14.7003479003906\\
0.242433547973633 14.7003479003906\\
0.267841786146164 14.7003479003906\\
0.29579085111618 14.7003479003906\\
0.32653483748436 14.7003479003906\\
0.360353201627731 14.7003479003906\\
0.397553414106369 14.7003479003906\\
0.438473641872406 14.7003479003906\\
0.483485877513886 14.7003479003906\\
0.532999336719513 14.7003479003906\\
0.582999348640442 14.7003479003906\\
0.632999360561371 14.7003479003906\\
0.6829993724823 14.7003479003906\\
0.732999384403229 14.7003479003906\\
0.782999336719513 14.7003479003906\\
0.832999348640442 14.7003479003906\\
0.882999360561371 14.7003479003906\\
0.9329993724823 14.7003479003906\\
0.982999384403229 14.7003479003906\\
1.03299939632416 14.7003479003906\\
1.08299934864044 14.7003479003906\\
1.13299930095673 14.7003479003906\\
1.1829993724823 14.7003479003906\\
1.23299932479858 14.7003479003906\\
1.28299939632416 14.7003479003906\\
1.33299934864044 14.7003479003906\\
1.38299930095673 14.7003479003906\\
1.4329993724823 14.7003479003906\\
1.48299932479858 14.7003479003906\\
1.53299939632416 14.7003479003906\\
1.58299934864044 14.7003479003906\\
1.63299930095673 14.7003479003906\\
1.6829993724823 14.7003479003906\\
1.73299932479858 14.7003479003906\\
1.78299939632416 14.7003479003906\\
1.83299934864044 14.7003479003906\\
1.88299930095673 14.7003479003906\\
1.9329993724823 14.7003479003906\\
1.98299932479858 14.7003479003906\\
2.03299927711487 14.7003479003906\\
2.08299946784973 14.7003479003906\\
2.13299942016602 14.7003479003906\\
2.1829993724823 14.7003479003906\\
2.23299932479858 14.7003479003906\\
2.28299927711487 14.7003479003906\\
2.33299946784973 14.7003479003906\\
2.38299942016602 14.7003479003906\\
2.4329993724823 14.7003479003906\\
2.48299932479858 14.7003479003906\\
2.53299927711487 14.7003479003906\\
2.58299946784973 14.7003479003906\\
2.63299942016602 14.7003479003906\\
2.6829993724823 14.7003479003906\\
2.73299932479858 14.7003479003906\\
2.78299927711487 14.7003479003906\\
2.83299946784973 14.7003479003906\\
2.88299942016602 14.7003479003906\\
2.9329993724823 14.7003479003906\\
2.98299932479858 14.7003479003906\\
3.03299927711487 14.7003479003906\\
3.08299946784973 14.7003479003906\\
3.13299942016602 14.7003479003906\\
3.1829993724823 14.7003479003906\\
3.23299932479858 14.7003479003906\\
3.28299927711487 14.7003479003906\\
3.33299946784973 14.7003479003906\\
3.38299942016602 14.7003479003906\\
3.4329993724823 14.7003479003906\\
3.48299932479858 14.7003479003906\\
3.53299927711487 14.7003479003906\\
3.58299946784973 14.7003479003906\\
3.63299942016602 14.7003479003906\\
3.6829993724823 14.7003479003906\\
3.73299932479858 14.7003479003906\\
3.78299927711487 14.7003479003906\\
3.83299946784973 14.7003479003906\\
3.88299942016602 14.7003479003906\\
3.9329993724823 14.7003479003906\\
3.98299932479858 14.7003479003906\\
4.03299951553345 14.7003479003906\\
4.08299922943115 14.7003479003906\\
4.13299942016602 14.7003479003906\\
4.18299913406372 14.7003479003906\\
4.23299932479858 14.7003479003906\\
4.28299951553345 14.7003479003906\\
4.33299922943115 14.7003479003906\\
4.38299942016602 14.7003479003906\\
4.43299913406372 14.7003479003906\\
4.48299932479858 14.7003479003906\\
4.53299951553345 14.7003479003906\\
4.58299922943115 14.7003479003906\\
4.63299942016602 14.7003479003906\\
4.68299913406372 14.7003479003906\\
4.73299932479858 14.7003479003906\\
4.78299951553345 14.7003479003906\\
4.83299922943115 14.7003479003906\\
4.88299942016602 14.7003479003906\\
4.93299913406372 14.7003479003906\\
4.98299932479858 14.7003479003906\\
5.03299951553345 14.7003479003906\\
5.08299922943115 14.7003479003906\\
5.13299942016602 14.7003479003906\\
5.18299913406372 14.7003479003906\\
5.23299932479858 14.7003479003906\\
5.28299951553345 14.7003479003906\\
5.33299922943115 14.7003479003906\\
5.38299942016602 14.7003479003906\\
5.43299913406372 14.7003479003906\\
5.48299932479858 14.7003479003906\\
5.53299951553345 14.7003479003906\\
5.58299922943115 14.7003479003906\\
5.63299942016602 14.7003479003906\\
5.68299913406372 14.7003479003906\\
5.73299932479858 14.7003479003906\\
5.78299951553345 14.7003479003906\\
5.83299922943115 14.7003479003906\\
5.88299942016602 14.7003479003906\\
5.93299913406372 14.7003479003906\\
5.98299932479858 14.7003479003906\\
6.03299951553345 14.7003469467163\\
6.08299922943115 14.7003469467163\\
6.13299942016602 14.7003469467163\\
6.18299913406372 14.700345993042\\
6.23299932479858 14.7003450393677\\
6.28299951553345 14.7003421783447\\
6.33299922943115 14.7003507614136\\
6.38299942016602 14.7003479003906\\
6.43299913406372 14.7003479003906\\
6.48299932479858 14.7003479003906\\
6.53299951553345 14.7003479003906\\
6.58299922943115 14.7003498077393\\
6.63299942016602 14.7003488540649\\
6.68299913406372 14.7003479003906\\
6.73299932479858 14.7003479003906\\
6.78299951553345 14.7003479003906\\
6.83299922943115 14.7003479003906\\
6.88299942016602 14.7003479003906\\
6.93299913406372 14.7003479003906\\
6.98299932479858 14.7003479003906\\
7.03299951553345 14.7003479003906\\
7.08299922943115 14.7003479003906\\
7.13299942016602 14.7003479003906\\
7.18299913406372 14.7003479003906\\
7.23299932479858 14.7003479003906\\
7.28299951553345 14.7003479003906\\
7.33299922943115 14.7003479003906\\
7.38299942016602 14.7003479003906\\
7.43299913406372 14.7003479003906\\
7.48299932479858 14.7003479003906\\
7.53299951553345 14.7003479003906\\
7.58299922943115 14.7003479003906\\
7.63299942016602 14.7003479003906\\
7.68299913406372 14.7003479003906\\
7.73299932479858 14.7003479003906\\
7.78299951553345 14.7003479003906\\
7.83299922943115 14.7003479003906\\
7.88299942016602 14.7003479003906\\
7.93299913406372 14.7003479003906\\
7.98299932479858 14.7003479003906\\
8.03299903869629 14.7003479003906\\
8.08299922943115 14.7003479003906\\
8.13299942016602 14.7003479003906\\
8.18299961090088 14.7003479003906\\
8.23299980163574 14.7003479003906\\
8.28299903869629 14.7003479003906\\
8.33299922943115 14.7003479003906\\
8.38299942016602 14.7003479003906\\
8.43299961090088 14.7003479003906\\
8.48299980163574 14.7003479003906\\
8.53299903869629 14.7003479003906\\
8.58299922943115 14.7003479003906\\
8.63299942016602 14.7003479003906\\
8.68299961090088 14.7003479003906\\
8.73299980163574 14.7003479003906\\
8.78299903869629 14.7003479003906\\
8.83299922943115 14.7003479003906\\
8.88299942016602 14.7003479003906\\
8.93299961090088 14.7003479003906\\
8.98299980163574 14.7003479003906\\
9.03299903869629 14.7003479003906\\
9.08299922943115 14.7003479003906\\
9.13299942016602 14.7003479003906\\
9.18299961090088 14.7003479003906\\
9.23299980163574 14.7003479003906\\
9.28299903869629 14.7003479003906\\
9.33299922943115 14.7003479003906\\
9.38299942016602 14.7003479003906\\
9.43299961090088 14.7003479003906\\
9.48299980163574 14.7003479003906\\
9.53299903869629 14.7003479003906\\
9.58299922943115 14.7003479003906\\
9.63299942016602 14.7003479003906\\
9.68299961090088 14.7003479003906\\
9.73299980163574 14.7003479003906\\
9.78299903869629 14.7003479003906\\
9.83299922943115 14.7003479003906\\
9.88299942016602 14.7003479003906\\
9.93299961090088 14.7003479003906\\
9.98299980163574 14.7003479003906\\
10.0329990386963 14.7003479003906\\
10.0829992294312 14.7003479003906\\
10.132999420166 14.7003479003906\\
10.1829996109009 14.7003479003906\\
10.2329998016357 14.7003479003906\\
10.2829990386963 14.7003479003906\\
10.3329992294312 14.7003479003906\\
10.382999420166 14.7003479003906\\
10.4329996109009 14.7003479003906\\
10.4829998016357 14.7003479003906\\
10.5329990386963 14.7003479003906\\
10.5829992294312 14.7003479003906\\
10.632999420166 14.7003479003906\\
10.6829996109009 14.7003479003906\\
10.7329998016357 14.7003479003906\\
10.7829990386963 14.7003479003906\\
10.8329992294312 14.7003479003906\\
10.882999420166 14.7003479003906\\
10.9329996109009 14.7003479003906\\
10.9829998016357 14.7003479003906\\
11.0329990386963 14.7003479003906\\
11.0829992294312 14.7003479003906\\
11.132999420166 14.7003479003906\\
11.1829996109009 14.7003479003906\\
11.2329998016357 14.7003479003906\\
11.2829990386963 14.7003479003906\\
11.3329992294312 14.7003479003906\\
11.382999420166 14.7003479003906\\
11.4329996109009 14.7003479003906\\
11.4829998016357 14.7003479003906\\
11.5329990386963 14.7003479003906\\
11.5829992294312 14.7003479003906\\
11.632999420166 14.7003479003906\\
11.6829996109009 14.7003479003906\\
11.7329998016357 14.7003479003906\\
11.7829990386963 14.7003479003906\\
11.8329992294312 14.7003479003906\\
11.882999420166 14.7003479003906\\
11.9329996109009 14.7003479003906\\
11.9829998016357 14.7003479003906\\
12.0329990386963 14.7003479003906\\
12.0829992294312 14.7003479003906\\
12.132999420166 14.7003479003906\\
12.1829996109009 14.7003479003906\\
12.2329998016357 14.7003479003906\\
12.2829990386963 14.7003479003906\\
12.3329992294312 14.7003479003906\\
12.382999420166 14.7003479003906\\
12.4329996109009 14.7003479003906\\
12.4829998016357 14.7003479003906\\
12.5329990386963 14.7003469467163\\
12.5829992294312 14.7003479003906\\
12.632999420166 14.7003479003906\\
12.6829996109009 14.7003479003906\\
12.7329998016357 14.7003479003906\\
12.7829990386963 14.7003479003906\\
12.8329992294312 14.7003479003906\\
12.882999420166 14.7003479003906\\
12.9329996109009 14.7003479003906\\
12.9829998016357 14.7003479003906\\
13.0329990386963 14.7003479003906\\
13.0829992294312 14.7003479003906\\
13.132999420166 14.7003479003906\\
13.1829996109009 14.7003479003906\\
13.2329998016357 14.7003479003906\\
13.2829990386963 14.7003479003906\\
13.3329992294312 14.7003479003906\\
13.382999420166 14.7003479003906\\
13.4329996109009 14.7003479003906\\
13.4829998016357 14.7003479003906\\
13.5329990386963 14.7003479003906\\
13.5829992294312 14.7003479003906\\
13.632999420166 14.7003479003906\\
13.6829996109009 14.7003479003906\\
13.7329998016357 14.7003479003906\\
13.7829990386963 14.7003479003906\\
13.8329992294312 14.7003479003906\\
13.882999420166 14.7003479003906\\
13.9329996109009 14.7003479003906\\
13.9829998016357 14.7003479003906\\
14.0329990386963 14.7003479003906\\
14.0829992294312 14.7003479003906\\
14.132999420166 14.7003479003906\\
14.1829996109009 14.7003479003906\\
14.2329998016357 14.7003479003906\\
14.2829990386963 14.7003479003906\\
14.3329992294312 14.7003479003906\\
14.382999420166 14.7003479003906\\
14.4329996109009 14.7003479003906\\
14.4829998016357 14.7003479003906\\
14.5329990386963 14.7003479003906\\
14.5829992294312 14.7003479003906\\
14.632999420166 14.7003479003906\\
14.6829996109009 14.7003479003906\\
14.7329998016357 14.7003479003906\\
14.7829990386963 14.7003479003906\\
14.8329992294312 14.7003479003906\\
14.882999420166 14.7003479003906\\
14.9329996109009 14.7003479003906\\
14.9829998016357 14.7003479003906\\
15.0329990386963 14.7003479003906\\
15.0829992294312 14.7003479003906\\
15.132999420166 14.7003479003906\\
15.1829996109009 14.7003479003906\\
15.2329998016357 14.7003479003906\\
15.2829990386963 14.7003479003906\\
15.3329992294312 14.7003479003906\\
15.382999420166 14.7003479003906\\
15.4329996109009 14.7003479003906\\
15.4829998016357 14.7003479003906\\
15.5329990386963 14.7003479003906\\
15.5829992294312 14.7003479003906\\
15.632999420166 14.7003479003906\\
15.6829996109009 14.7003479003906\\
15.7329998016357 14.7003479003906\\
15.7829990386963 14.7003479003906\\
15.8329992294312 14.7003479003906\\
15.882999420166 14.7003479003906\\
15.9329996109009 14.7003479003906\\
15.9829998016357 14.7003479003906\\
16.0329990386963 14.7003479003906\\
16.0830001831055 14.7003479003906\\
16.132999420166 14.7003479003906\\
16.1829986572266 14.7003479003906\\
16.2329998016357 14.7003479003906\\
16.2829990386963 14.7003479003906\\
16.3330001831055 14.7003479003906\\
16.382999420166 14.7003479003906\\
16.4329986572266 14.7003479003906\\
16.4829998016357 14.7003479003906\\
16.5329990386963 14.7003479003906\\
16.5830001831055 14.7003479003906\\
16.632999420166 14.7003479003906\\
16.6829986572266 14.7003479003906\\
16.7329998016357 14.7003479003906\\
16.7829990386963 14.7003479003906\\
16.8330001831055 14.7003479003906\\
16.882999420166 14.7003479003906\\
16.9329986572266 14.7003479003906\\
16.9829998016357 14.7003479003906\\
17.0329990386963 14.7003479003906\\
17.0830001831055 14.7003479003906\\
17.132999420166 14.7003479003906\\
17.1829986572266 14.7003479003906\\
17.2329998016357 14.7003479003906\\
17.2829990386963 14.7003479003906\\
17.3330001831055 14.7003479003906\\
17.382999420166 14.7003479003906\\
17.4329986572266 14.7003479003906\\
17.4829998016357 14.7003479003906\\
17.5329990386963 14.7003479003906\\
17.5830001831055 14.7003479003906\\
17.632999420166 14.7003479003906\\
17.6829986572266 14.7003479003906\\
17.7329998016357 14.7003479003906\\
17.7829990386963 14.7003479003906\\
17.8330001831055 14.7003479003906\\
17.882999420166 14.7003479003906\\
17.9329986572266 14.7003479003906\\
17.9829998016357 14.7003479003906\\
18.0329990386963 14.7003479003906\\
18.0830001831055 14.7003479003906\\
18.132999420166 14.7003479003906\\
18.1829986572266 14.7003469467163\\
18.2329998016357 14.7003479003906\\
18.2829990386963 14.7003469467163\\
18.3330001831055 14.7003469467163\\
18.382999420166 14.700345993042\\
18.4329986572266 14.700345993042\\
18.4829998016357 14.7003450393677\\
18.5329990386963 14.700343132019\\
18.5830001831055 14.7003507614136\\
18.632999420166 14.7003479003906\\
18.6829986572266 14.7003479003906\\
18.7329998016357 14.7003479003906\\
18.7829990386963 14.7003479003906\\
18.8330001831055 14.7003488540649\\
18.882999420166 14.7003479003906\\
18.9329986572266 14.7003479003906\\
18.9829998016357 14.7003479003906\\
19.0329990386963 14.7003479003906\\
19.0830001831055 14.7003479003906\\
19.132999420166 14.7003479003906\\
19.1829986572266 14.7003479003906\\
19.2329998016357 14.7003479003906\\
19.2829990386963 14.7003479003906\\
19.3330001831055 14.7003479003906\\
19.382999420166 14.7003479003906\\
19.4329986572266 14.7003479003906\\
19.4829998016357 14.7003479003906\\
19.5329990386963 14.7003479003906\\
19.5830001831055 14.7003479003906\\
19.632999420166 14.7003479003906\\
19.6829986572266 14.7003479003906\\
19.7329998016357 14.7003479003906\\
19.7829990386963 14.7003479003906\\
19.8330001831055 14.7003479003906\\
19.882999420166 14.7003479003906\\
19.9329986572266 14.7003479003906\\
19.9829998016357 14.7003479003906\\
20.0329990386963 14.7003479003906\\
20.0830001831055 14.7003479003906\\
20.132999420166 14.7003479003906\\
20.1829986572266 14.7003479003906\\
20.2329998016357 14.7003479003906\\
20.2829990386963 14.7003479003906\\
20.3330001831055 14.7003479003906\\
20.382999420166 14.7003479003906\\
20.4329986572266 14.7003479003906\\
20.4829998016357 14.7003479003906\\
20.5329990386963 14.7003479003906\\
20.5830001831055 14.7003479003906\\
20.632999420166 14.7003479003906\\
20.6829986572266 14.7003479003906\\
20.7329998016357 14.7003479003906\\
20.7829990386963 14.7003479003906\\
20.8330001831055 14.7003479003906\\
20.882999420166 14.7003479003906\\
20.9329986572266 14.7003479003906\\
20.9829998016357 14.7003479003906\\
21.0329990386963 14.7003479003906\\
21.0830001831055 14.7003479003906\\
21.132999420166 14.7003479003906\\
21.1829986572266 14.7003479003906\\
21.2329998016357 14.7003479003906\\
21.2829990386963 14.7003479003906\\
21.3330001831055 14.7003479003906\\
21.382999420166 14.7003479003906\\
21.4329986572266 14.7003479003906\\
21.4829998016357 14.7003479003906\\
21.5329990386963 14.7003479003906\\
21.5830001831055 14.7003479003906\\
21.632999420166 14.7003479003906\\
21.6829986572266 14.7003479003906\\
21.7329998016357 14.7003479003906\\
21.7829990386963 14.7003479003906\\
21.8330001831055 14.7003479003906\\
21.882999420166 14.7003479003906\\
21.9329986572266 14.7003479003906\\
21.9829998016357 14.7003479003906\\
22.0329990386963 14.7003479003906\\
22.0830001831055 14.7003479003906\\
22.132999420166 14.7003479003906\\
22.1829986572266 14.7003479003906\\
22.2329998016357 14.7003479003906\\
22.2829990386963 14.7003479003906\\
22.3330001831055 14.7003479003906\\
22.382999420166 14.7003479003906\\
22.4329986572266 14.7003479003906\\
22.4829998016357 14.7003479003906\\
22.5329990386963 14.7003479003906\\
22.5830001831055 14.7003479003906\\
22.632999420166 14.7003479003906\\
22.6829986572266 14.7003479003906\\
22.7329998016357 14.7003479003906\\
22.7829990386963 14.7003479003906\\
22.8330001831055 14.7003479003906\\
22.882999420166 14.7003479003906\\
22.9329986572266 14.7003479003906\\
22.9829998016357 14.7003479003906\\
23.0329990386963 14.7003479003906\\
23.0830001831055 14.7003479003906\\
23.132999420166 14.7003479003906\\
23.1829986572266 14.7003479003906\\
23.2329998016357 14.7003479003906\\
23.2829990386963 14.7003479003906\\
23.3330001831055 14.7003479003906\\
23.382999420166 14.7003479003906\\
23.4329986572266 14.7003479003906\\
23.4829998016357 14.7003479003906\\
23.5329990386963 14.7003479003906\\
23.5830001831055 14.7003479003906\\
23.632999420166 14.7003479003906\\
23.6829986572266 14.7003479003906\\
23.7329998016357 14.7003479003906\\
23.7829990386963 14.7003479003906\\
23.8330001831055 14.7003479003906\\
23.882999420166 14.7003479003906\\
23.9329986572266 14.7003479003906\\
23.9829998016357 14.7003479003906\\
24.0329990386963 14.7003479003906\\
24.0830001831055 14.7003479003906\\
24.132999420166 14.7003479003906\\
24.1829986572266 14.7003479003906\\
24.2329998016357 14.7003479003906\\
24.2829990386963 14.7003479003906\\
24.3330001831055 14.7003479003906\\
24.382999420166 14.7003479003906\\
24.4329986572266 14.7003479003906\\
24.4829998016357 14.7003479003906\\
24.5329990386963 14.7003479003906\\
24.5830001831055 14.7003479003906\\
24.632999420166 14.7003479003906\\
24.6829986572266 14.7003479003906\\
24.7329998016357 14.7003479003906\\
24.7829990386963 14.7003469467163\\
24.8330001831055 14.7003479003906\\
24.882999420166 14.7003479003906\\
24.9329986572266 14.7003479003906\\
24.9829998016357 14.7003479003906\\
25.0329990386963 14.7003479003906\\
25.0830001831055 14.7003479003906\\
25.132999420166 14.7003479003906\\
25.1829986572266 14.7003479003906\\
25.2329998016357 14.7003479003906\\
25.2829990386963 14.7003479003906\\
25.3330001831055 14.7003479003906\\
25.382999420166 14.7003479003906\\
25.4329986572266 14.7003479003906\\
25.4829998016357 14.7003479003906\\
25.5329990386963 14.7003479003906\\
25.5830001831055 14.7003479003906\\
25.632999420166 14.7003479003906\\
25.6829986572266 14.7003479003906\\
25.7329998016357 14.7003479003906\\
25.7829990386963 14.7003479003906\\
25.8330001831055 14.7003479003906\\
25.882999420166 14.7003479003906\\
25.9329986572266 14.7003479003906\\
25.9829998016357 14.7003479003906\\
26.0329990386963 14.7003479003906\\
26.0830001831055 14.7003479003906\\
26.132999420166 14.7003479003906\\
26.1829986572266 14.7003479003906\\
26.2329998016357 14.7003479003906\\
26.2829990386963 14.7003479003906\\
26.3330001831055 14.7003479003906\\
26.382999420166 14.7003479003906\\
26.4329986572266 14.7003479003906\\
26.4829998016357 14.7003479003906\\
26.5329990386963 14.7003479003906\\
26.5830001831055 14.7003479003906\\
26.632999420166 14.7003479003906\\
26.6829986572266 14.7003479003906\\
26.7329998016357 14.7003479003906\\
26.7829990386963 14.7003479003906\\
26.8330001831055 14.7003479003906\\
26.882999420166 14.7003479003906\\
26.9329986572266 14.7003479003906\\
26.9829998016357 14.7003479003906\\
27.0329990386963 14.7003479003906\\
27.0830001831055 14.7003479003906\\
27.132999420166 14.7003479003906\\
27.1829986572266 14.7003479003906\\
27.2329998016357 14.7003479003906\\
27.2829990386963 14.7003479003906\\
27.3330001831055 14.7003479003906\\
27.382999420166 14.7003479003906\\
27.4329986572266 14.7003479003906\\
27.4829998016357 14.7003479003906\\
27.5329990386963 14.7003479003906\\
27.5830001831055 14.7003479003906\\
27.632999420166 14.7003479003906\\
27.6829986572266 14.7003479003906\\
27.7329998016357 14.7003479003906\\
27.7829990386963 14.7003479003906\\
27.8330001831055 14.7003479003906\\
27.882999420166 14.7003479003906\\
27.9329986572266 14.7003479003906\\
27.9829998016357 14.7003479003906\\
28.0329990386963 14.7003479003906\\
28.0830001831055 14.7003479003906\\
28.132999420166 14.7003479003906\\
28.1829986572266 14.7003479003906\\
28.2329998016357 14.7003479003906\\
28.2829990386963 14.7003479003906\\
28.3330001831055 14.7003479003906\\
28.382999420166 14.7003479003906\\
28.4329986572266 14.7003479003906\\
28.4829998016357 14.7003479003906\\
28.5329990386963 14.7003479003906\\
28.5830001831055 14.7003479003906\\
28.632999420166 14.7003479003906\\
28.6829986572266 14.7003479003906\\
28.7329998016357 14.7003479003906\\
28.7829990386963 14.7003479003906\\
28.8330001831055 14.7003479003906\\
28.882999420166 14.7003479003906\\
28.9329986572266 14.7003479003906\\
28.9829998016357 14.7003479003906\\
29.0329990386963 14.7003479003906\\
29.0830001831055 14.7003479003906\\
29.132999420166 14.7003479003906\\
29.1829986572266 14.7003479003906\\
29.2329998016357 14.7003479003906\\
29.2829990386963 14.7003479003906\\
29.3330001831055 14.7003479003906\\
29.382999420166 14.7003479003906\\
29.4329986572266 14.7003479003906\\
29.4829998016357 14.7003479003906\\
29.5329990386963 14.7003479003906\\
29.5830001831055 14.7003479003906\\
29.632999420166 14.7003479003906\\
29.6829986572266 14.7003479003906\\
29.7329998016357 14.7003479003906\\
29.7829990386963 14.7003479003906\\
29.8330001831055 14.7003479003906\\
29.882999420166 14.7003479003906\\
29.9329986572266 14.7003479003906\\
29.9829998016357 14.7003479003906\\
30.0329990386963 14.7003479003906\\
30.0830001831055 14.7003479003906\\
30.132999420166 14.7003479003906\\
30.1829986572266 14.7003479003906\\
30.2329998016357 14.7003479003906\\
30.2829990386963 14.7003479003906\\
30.3330001831055 14.7003479003906\\
30.382999420166 14.7003479003906\\
30.4329986572266 14.7003469467163\\
30.4829998016357 14.7003469467163\\
30.5329990386963 14.7003479003906\\
30.5830001831055 14.7003469467163\\
30.632999420166 14.700345993042\\
30.6829986572266 14.7003450393677\\
30.7329998016357 14.7003450393677\\
30.7829990386963 14.7003450393677\\
30.8330001831055 14.7003507614136\\
30.882999420166 14.7003479003906\\
30.9329986572266 14.7003479003906\\
30.9829998016357 14.7003479003906\\
31.0329990386963 14.7003488540649\\
31.0830001831055 14.7003479003906\\
31.132999420166 14.7003488540649\\
31.1829986572266 14.7003479003906\\
31.2329998016357 14.7003479003906\\
31.2829990386963 14.7003479003906\\
31.3330001831055 14.7003479003906\\
31.382999420166 14.7003479003906\\
31.4329986572266 14.7003479003906\\
31.4829998016357 14.7003479003906\\
31.5329990386963 14.7003479003906\\
31.5830001831055 14.7003479003906\\
31.632999420166 14.7003479003906\\
31.6829986572266 14.7003479003906\\
31.7329998016357 14.7003479003906\\
31.7829990386963 14.7003479003906\\
31.8330001831055 14.7003479003906\\
31.882999420166 14.7003479003906\\
31.9329986572266 14.7003479003906\\
31.9829998016357 14.7003479003906\\
32.0330009460449 14.7003479003906\\
32.0830001831055 14.7003479003906\\
32.132999420166 14.7003479003906\\
32.1829986572266 14.7003479003906\\
32.2329978942871 14.7003479003906\\
32.2830009460449 14.7003479003906\\
32.3330001831055 14.7003479003906\\
32.382999420166 14.7003479003906\\
32.4329986572266 14.7003479003906\\
32.4829978942871 14.7003479003906\\
32.5330009460449 14.7003479003906\\
32.5830001831055 14.7003479003906\\
32.632999420166 14.7003479003906\\
32.6829986572266 14.7003479003906\\
32.7329978942871 14.7003479003906\\
32.7830009460449 14.7003479003906\\
32.8330001831055 14.7003479003906\\
32.882999420166 14.7003479003906\\
32.9329986572266 14.7003479003906\\
32.9829978942871 14.7003479003906\\
33.0330009460449 14.7003479003906\\
33.0830001831055 14.7003479003906\\
33.132999420166 14.7003479003906\\
33.1829986572266 14.7003479003906\\
33.2329978942871 14.7003479003906\\
33.2830009460449 14.7003479003906\\
33.3330001831055 14.7003479003906\\
33.382999420166 14.7003479003906\\
33.4329986572266 14.7003479003906\\
33.4829978942871 14.7003479003906\\
33.5330009460449 14.7003479003906\\
33.5830001831055 14.7003479003906\\
33.632999420166 14.7003479003906\\
33.6829986572266 14.7003479003906\\
33.7329978942871 14.7003479003906\\
33.7830009460449 14.7003479003906\\
33.8330001831055 14.7003479003906\\
33.882999420166 14.7003479003906\\
33.9329986572266 14.7003479003906\\
33.9829978942871 14.7003479003906\\
34.0330009460449 14.7003479003906\\
34.0830001831055 14.7003479003906\\
34.132999420166 14.7003479003906\\
34.1829986572266 14.7003479003906\\
34.2329978942871 14.7003479003906\\
34.2830009460449 14.7003479003906\\
34.3330001831055 14.7003479003906\\
34.382999420166 14.7003479003906\\
34.4329986572266 14.7003479003906\\
34.4829978942871 14.7003479003906\\
34.5330009460449 14.7003479003906\\
34.5830001831055 14.7003479003906\\
34.632999420166 14.7003479003906\\
34.6829986572266 14.7003479003906\\
34.7329978942871 14.7003479003906\\
34.7830009460449 14.7003479003906\\
34.8330001831055 14.7003479003906\\
34.882999420166 14.7003479003906\\
34.9329986572266 14.7003479003906\\
34.9829978942871 14.7003479003906\\
35.0330009460449 14.7003479003906\\
35.0830001831055 14.7003479003906\\
35.132999420166 14.7003479003906\\
35.1829986572266 14.7003479003906\\
35.2329978942871 14.7003479003906\\
35.2830009460449 14.7003479003906\\
35.3330001831055 14.7003479003906\\
35.382999420166 14.7003479003906\\
35.4329986572266 14.7003479003906\\
35.4829978942871 14.7003479003906\\
35.5330009460449 14.7003479003906\\
35.5830001831055 14.7003479003906\\
35.632999420166 14.7003479003906\\
35.6829986572266 14.7003479003906\\
35.7329978942871 14.7003479003906\\
35.7830009460449 14.7003479003906\\
35.8330001831055 14.7003479003906\\
35.882999420166 14.7003479003906\\
35.9329986572266 14.7003479003906\\
35.9829978942871 14.7003479003906\\
36.0330009460449 14.7003479003906\\
36.0830001831055 14.7003479003906\\
36.132999420166 14.7003479003906\\
36.1829986572266 14.7003479003906\\
36.2329978942871 14.7003479003906\\
36.2830009460449 14.7003479003906\\
36.3330001831055 14.7003479003906\\
36.382999420166 14.7003479003906\\
36.4329986572266 14.7003479003906\\
36.4829978942871 14.7003479003906\\
36.5330009460449 14.7003479003906\\
36.5830001831055 14.7003479003906\\
36.632999420166 14.7003479003906\\
36.6829986572266 14.7003479003906\\
36.7329978942871 14.7003479003906\\
36.7830009460449 14.7003479003906\\
36.8330001831055 14.7003479003906\\
36.882999420166 14.7003479003906\\
36.9329986572266 14.7003479003906\\
36.9829978942871 14.7003479003906\\
37.0330009460449 14.7003469467163\\
37.0830001831055 14.7003479003906\\
37.132999420166 14.7003479003906\\
37.1829986572266 14.7003479003906\\
37.2329978942871 14.7003479003906\\
37.2830009460449 14.7003479003906\\
37.3330001831055 14.7003479003906\\
37.382999420166 14.7003479003906\\
37.4329986572266 14.7003479003906\\
37.4829978942871 14.7003479003906\\
37.5330009460449 14.7003479003906\\
37.5830001831055 14.7003479003906\\
37.632999420166 14.7003479003906\\
37.6829986572266 14.7003479003906\\
37.7329978942871 14.7003479003906\\
37.7830009460449 14.7003479003906\\
37.8330001831055 14.7003479003906\\
37.882999420166 14.7003479003906\\
37.9329986572266 14.7003479003906\\
37.9829978942871 14.7003479003906\\
38.0330009460449 14.7003479003906\\
38.0830001831055 14.7003479003906\\
38.132999420166 14.7003479003906\\
38.1829986572266 14.7003479003906\\
38.2329978942871 14.7003479003906\\
38.2830009460449 14.7003479003906\\
38.3330001831055 14.7003479003906\\
38.382999420166 14.7003479003906\\
38.4329986572266 14.7003479003906\\
38.4829978942871 14.7003479003906\\
38.5330009460449 14.7003479003906\\
38.5830001831055 14.7003479003906\\
38.632999420166 14.7003479003906\\
38.6829986572266 14.7003479003906\\
38.7329978942871 14.7003479003906\\
38.7830009460449 14.7003479003906\\
38.8330001831055 14.7003479003906\\
38.882999420166 14.7003479003906\\
38.9329986572266 14.7003479003906\\
38.9829978942871 14.7003479003906\\
39.0330009460449 14.7003479003906\\
39.0830001831055 14.7003479003906\\
39.132999420166 14.7003479003906\\
39.1829986572266 14.7003479003906\\
39.2329978942871 14.7003479003906\\
39.2830009460449 14.7003479003906\\
39.3330001831055 14.7003479003906\\
39.382999420166 14.7003479003906\\
39.4329986572266 14.7003479003906\\
39.4829978942871 14.7003479003906\\
39.5330009460449 14.7003479003906\\
39.5830001831055 14.7003479003906\\
39.632999420166 14.7003479003906\\
39.6829986572266 14.7003479003906\\
39.7329978942871 14.7003479003906\\
39.7830009460449 14.7003479003906\\
39.8330001831055 14.7003479003906\\
39.882999420166 14.7003479003906\\
39.9329986572266 14.7003479003906\\
39.9829978942871 14.7003479003906\\
40.0330009460449 14.7003479003906\\
40.0830001831055 14.7003479003906\\
40.132999420166 14.7003479003906\\
40.1829986572266 14.7003479003906\\
40.2329978942871 14.7003479003906\\
40.2830009460449 14.7003479003906\\
40.3330001831055 14.7003479003906\\
40.382999420166 14.7003479003906\\
40.4329986572266 14.7003479003906\\
40.4829978942871 14.7003479003906\\
40.5330009460449 14.7003479003906\\
40.5830001831055 14.7003479003906\\
40.632999420166 14.7003479003906\\
40.6829986572266 14.7003479003906\\
40.7329978942871 14.7003479003906\\
40.7830009460449 14.7003479003906\\
40.8330001831055 14.7003479003906\\
40.882999420166 14.7003479003906\\
40.9329986572266 14.7003479003906\\
40.9829978942871 14.7003479003906\\
41.0330009460449 14.7003479003906\\
41.0830001831055 14.7003479003906\\
41.132999420166 14.7003479003906\\
41.1829986572266 14.7003479003906\\
41.2329978942871 14.7003479003906\\
41.2830009460449 14.7003479003906\\
41.3330001831055 14.7003479003906\\
41.382999420166 14.7003479003906\\
41.4329986572266 14.7003479003906\\
41.4829978942871 14.7003479003906\\
41.5330009460449 14.7003479003906\\
41.5830001831055 14.7003479003906\\
41.632999420166 14.7003479003906\\
41.6829986572266 14.7003479003906\\
41.7329978942871 14.7003479003906\\
41.7830009460449 14.7003479003906\\
41.8330001831055 14.7003479003906\\
41.882999420166 14.7003479003906\\
41.9329986572266 14.7003479003906\\
41.9829978942871 14.7003479003906\\
42.0330009460449 14.7003479003906\\
42.0830001831055 14.7003479003906\\
42.132999420166 14.7003479003906\\
42.1829986572266 14.7003479003906\\
42.2329978942871 14.7003479003906\\
42.2830009460449 14.7003479003906\\
42.3330001831055 14.7003479003906\\
42.382999420166 14.7003479003906\\
42.4329986572266 14.7003479003906\\
42.4829978942871 14.7003479003906\\
42.5330009460449 14.7003479003906\\
42.5830001831055 14.7003479003906\\
42.632999420166 14.7003479003906\\
42.6829986572266 14.7003469467163\\
42.7329978942871 14.7003469467163\\
42.7830009460449 14.7003479003906\\
42.8330001831055 14.7003469467163\\
42.882999420166 14.700345993042\\
42.9329986572266 14.7003450393677\\
42.9829978942871 14.7003450393677\\
43.0330009460449 14.7003450393677\\
43.0830001831055 14.7003507614136\\
43.132999420166 14.7003479003906\\
43.1829986572266 14.7003479003906\\
43.2329978942871 14.7003479003906\\
43.2830009460449 14.7003488540649\\
43.3330001831055 14.7003479003906\\
43.382999420166 14.7003488540649\\
43.4329986572266 14.7003479003906\\
43.4829978942871 14.7003479003906\\
43.5330009460449 14.7003479003906\\
43.5830001831055 14.7003479003906\\
43.632999420166 14.7003479003906\\
43.6829986572266 14.7003479003906\\
43.7329978942871 14.7003479003906\\
43.7830009460449 14.7003479003906\\
43.8330001831055 14.7003479003906\\
43.882999420166 14.7003479003906\\
43.9329986572266 14.7003479003906\\
43.9829978942871 14.7003479003906\\
44.0330009460449 14.7003479003906\\
44.0830001831055 14.7003479003906\\
44.132999420166 14.7003479003906\\
44.1829986572266 14.7003479003906\\
44.2329978942871 14.7003479003906\\
44.2830009460449 14.7003479003906\\
44.3330001831055 14.7003479003906\\
44.382999420166 14.7003479003906\\
44.4329986572266 14.7003479003906\\
44.4829978942871 14.7003479003906\\
44.5330009460449 14.7003479003906\\
44.5830001831055 14.7003479003906\\
44.632999420166 14.7003479003906\\
44.6829986572266 14.7003479003906\\
44.7329978942871 14.7003479003906\\
44.7830009460449 14.7003479003906\\
44.8330001831055 14.7003479003906\\
44.882999420166 14.7003479003906\\
44.9329986572266 14.7003479003906\\
44.9829978942871 14.7003479003906\\
45.0330009460449 14.7003479003906\\
45.0830001831055 14.7003479003906\\
45.132999420166 14.7003479003906\\
45.1829986572266 14.7003479003906\\
45.2329978942871 14.7003479003906\\
45.2830009460449 14.7003479003906\\
45.3330001831055 14.7003479003906\\
45.382999420166 14.7003479003906\\
45.4329986572266 14.7003479003906\\
45.4829978942871 14.7003479003906\\
45.5330009460449 14.7003479003906\\
45.5830001831055 14.7003479003906\\
45.632999420166 14.7003479003906\\
45.6829986572266 14.7003479003906\\
45.7329978942871 14.7003479003906\\
45.7830009460449 14.7003479003906\\
45.8330001831055 14.7003479003906\\
45.882999420166 14.7003479003906\\
45.9329986572266 14.7003479003906\\
45.9829978942871 14.7003479003906\\
46.0330009460449 14.7003479003906\\
46.0830001831055 14.7003479003906\\
46.132999420166 14.7003479003906\\
46.1829986572266 14.7003479003906\\
46.2329978942871 14.7003479003906\\
46.2830009460449 14.7003479003906\\
46.3330001831055 14.7003479003906\\
46.382999420166 14.7003479003906\\
46.4329986572266 14.7003479003906\\
46.4829978942871 14.7003479003906\\
46.5330009460449 14.7003479003906\\
46.5830001831055 14.7003479003906\\
46.632999420166 14.7003479003906\\
46.6829986572266 14.7003479003906\\
46.7329978942871 14.7003479003906\\
46.7830009460449 14.7003479003906\\
46.8330001831055 14.7003479003906\\
46.882999420166 14.7003479003906\\
46.9329986572266 14.7003479003906\\
46.9829978942871 14.7003479003906\\
47.0330009460449 14.7003479003906\\
47.0830001831055 14.7003479003906\\
47.132999420166 14.7003479003906\\
47.1829986572266 14.7003479003906\\
47.2329978942871 14.7003479003906\\
47.2830009460449 14.7003479003906\\
47.3330001831055 14.7003479003906\\
47.382999420166 14.7003479003906\\
47.4329986572266 14.7003479003906\\
47.4829978942871 14.7003479003906\\
47.5330009460449 14.7003479003906\\
47.5830001831055 14.7003479003906\\
47.632999420166 14.7003479003906\\
47.6829986572266 14.7003479003906\\
47.7329978942871 14.7003479003906\\
47.7830009460449 14.7003479003906\\
47.8330001831055 14.7003479003906\\
47.882999420166 14.7003479003906\\
47.9329986572266 14.7003479003906\\
47.9829978942871 14.7003479003906\\
48.0330009460449 14.7003479003906\\
48.0830001831055 14.7003479003906\\
48.132999420166 14.7003479003906\\
48.1829986572266 14.7003479003906\\
48.2329978942871 14.7003479003906\\
48.2830009460449 14.7003479003906\\
48.3330001831055 14.7003479003906\\
48.382999420166 14.7003479003906\\
48.4329986572266 14.7003479003906\\
48.4829978942871 14.7003479003906\\
48.5330009460449 14.7003479003906\\
48.5830001831055 14.7003479003906\\
48.632999420166 14.7003479003906\\
48.6829986572266 14.7003479003906\\
48.7329978942871 14.7003479003906\\
48.7830009460449 14.7003479003906\\
48.8330001831055 14.7003479003906\\
48.882999420166 14.7003479003906\\
48.9329986572266 14.7003479003906\\
48.9829978942871 14.7003479003906\\
49.0330009460449 14.7003479003906\\
49.0830001831055 14.7003479003906\\
49.132999420166 14.7003479003906\\
49.1829986572266 14.7003479003906\\
49.2329978942871 14.7003479003906\\
49.2830009460449 14.7003469467163\\
49.3330001831055 14.7003479003906\\
49.382999420166 14.7003479003906\\
49.4329986572266 14.7003479003906\\
49.4829978942871 14.7003479003906\\
49.5330009460449 14.7003479003906\\
49.5830001831055 14.7003479003906\\
49.632999420166 14.7003479003906\\
49.6829986572266 14.7003479003906\\
49.7329978942871 14.7003479003906\\
49.7830009460449 14.7003479003906\\
49.8330001831055 14.7003479003906\\
49.882999420166 14.7003479003906\\
49.9329986572266 14.7003479003906\\
49.9829978942871 14.7003479003906\\
50.0330009460449 14.7003479003906\\
50.0830001831055 14.7003479003906\\
50.132999420166 14.7003479003906\\
50.1829986572266 14.7003479003906\\
50.2329978942871 14.7003479003906\\
50.2830009460449 14.7003479003906\\
50.3330001831055 14.7003479003906\\
50.382999420166 14.7003479003906\\
50.4329986572266 14.7003479003906\\
50.4829978942871 14.7003479003906\\
50.5330009460449 14.7003479003906\\
50.5830001831055 14.7003479003906\\
50.632999420166 14.7003479003906\\
50.6829986572266 14.7003479003906\\
50.7329978942871 14.7003479003906\\
50.7830009460449 14.7003479003906\\
50.8330001831055 14.7003479003906\\
50.882999420166 14.7003479003906\\
50.9329986572266 14.7003479003906\\
50.9829978942871 14.7003479003906\\
51.0330009460449 14.7003479003906\\
51.0830001831055 14.7003479003906\\
51.132999420166 14.7003479003906\\
51.1829986572266 14.7003479003906\\
51.2329978942871 14.7003479003906\\
51.2830009460449 14.7003479003906\\
51.3330001831055 14.7003479003906\\
51.382999420166 14.7003479003906\\
51.4329986572266 14.7003479003906\\
51.4829978942871 14.7003479003906\\
51.5330009460449 14.7003479003906\\
51.5830001831055 14.7003479003906\\
51.632999420166 14.7003479003906\\
51.6829986572266 14.7003479003906\\
51.7329978942871 14.7003479003906\\
51.7830009460449 14.7003479003906\\
51.8330001831055 14.7003479003906\\
51.882999420166 14.7003479003906\\
51.9329986572266 14.7003479003906\\
51.9829978942871 14.7003479003906\\
52.0330009460449 14.7003479003906\\
52.0830001831055 14.7003479003906\\
52.132999420166 14.7003479003906\\
52.1829986572266 14.7003479003906\\
52.2329978942871 14.7003479003906\\
52.2830009460449 14.7003479003906\\
52.3330001831055 14.7003479003906\\
52.382999420166 14.7003479003906\\
52.4329986572266 14.7003479003906\\
52.4829978942871 14.7003479003906\\
52.5330009460449 14.7003479003906\\
52.5830001831055 14.7003479003906\\
52.632999420166 14.7003479003906\\
52.6829986572266 14.7003479003906\\
52.7329978942871 14.7003479003906\\
52.7830009460449 14.7003479003906\\
52.8330001831055 14.7003479003906\\
52.882999420166 14.7003479003906\\
52.9329986572266 14.7003479003906\\
52.9829978942871 14.7003479003906\\
53.0330009460449 14.7003479003906\\
53.0830001831055 14.7003479003906\\
53.132999420166 14.7003479003906\\
53.1829986572266 14.7003479003906\\
53.2329978942871 14.7003479003906\\
53.2830009460449 14.7003479003906\\
53.3330001831055 14.7003479003906\\
53.382999420166 14.7003479003906\\
53.4329986572266 14.7003479003906\\
53.4829978942871 14.7003479003906\\
53.5330009460449 14.7003479003906\\
53.5830001831055 14.7003479003906\\
53.632999420166 14.7003479003906\\
53.6829986572266 14.7003479003906\\
53.7329978942871 14.7003479003906\\
53.7830009460449 14.7003479003906\\
53.8330001831055 14.7003479003906\\
53.882999420166 14.7003479003906\\
53.9329986572266 14.7003479003906\\
53.9829978942871 14.7003479003906\\
54.0330009460449 14.7003479003906\\
54.0830001831055 14.7003479003906\\
54.132999420166 14.7003479003906\\
54.1829986572266 14.7003479003906\\
54.2329978942871 14.7003479003906\\
54.2830009460449 14.7003479003906\\
54.3330001831055 14.7003479003906\\
54.382999420166 14.7003479003906\\
54.4329986572266 14.7003479003906\\
54.4829978942871 14.7003479003906\\
54.5330009460449 14.7003479003906\\
54.5830001831055 14.7003479003906\\
54.632999420166 14.7003479003906\\
54.6829986572266 14.7003479003906\\
54.7329978942871 14.7003479003906\\
54.7830009460449 14.7003479003906\\
54.8330001831055 14.7003479003906\\
54.882999420166 14.7003479003906\\
54.9329986572266 14.7003469467163\\
54.9829978942871 14.7003469467163\\
55.0330009460449 14.7003479003906\\
55.0830001831055 14.7003469467163\\
55.132999420166 14.700345993042\\
55.1829986572266 14.7003450393677\\
55.2329978942871 14.7003450393677\\
55.2830009460449 14.7003450393677\\
55.3330001831055 14.7003507614136\\
55.382999420166 14.7003479003906\\
55.4329986572266 14.7003479003906\\
55.4829978942871 14.7003479003906\\
55.5330009460449 14.7003488540649\\
55.5830001831055 14.7003479003906\\
55.632999420166 14.7003488540649\\
55.6829986572266 14.7003479003906\\
55.7329978942871 14.7003479003906\\
55.7830009460449 14.7003479003906\\
55.8330001831055 14.7003479003906\\
55.882999420166 14.7003479003906\\
55.9329986572266 14.7003479003906\\
55.9829978942871 14.7003479003906\\
56.0330009460449 14.7003479003906\\
56.0830001831055 14.7003479003906\\
56.132999420166 14.7003479003906\\
56.1829986572266 14.7003479003906\\
56.2329978942871 14.7003479003906\\
56.2830009460449 14.7003479003906\\
56.3330001831055 14.7003479003906\\
56.382999420166 14.7003479003906\\
56.4329986572266 14.7003479003906\\
56.4829978942871 14.7003479003906\\
56.5330009460449 14.7003479003906\\
56.5830001831055 14.7003479003906\\
56.632999420166 14.7003479003906\\
56.6829986572266 14.7003479003906\\
56.7329978942871 14.7003479003906\\
56.7830009460449 14.7003479003906\\
56.8330001831055 14.7003479003906\\
56.882999420166 14.7003479003906\\
56.9329986572266 14.7003479003906\\
56.9829978942871 14.7003479003906\\
57.0330009460449 14.7003479003906\\
57.0830001831055 14.7003479003906\\
57.132999420166 14.7003479003906\\
57.1829986572266 14.7003479003906\\
57.2329978942871 14.7003479003906\\
57.2830009460449 14.7003479003906\\
57.3330001831055 14.7003479003906\\
57.382999420166 14.7003479003906\\
57.4329986572266 14.7003479003906\\
57.4829978942871 14.7003479003906\\
57.5330009460449 14.7003479003906\\
57.5830001831055 14.7003479003906\\
57.632999420166 14.7003479003906\\
57.6829986572266 14.7003479003906\\
57.7329978942871 14.7003479003906\\
57.7830009460449 14.7003479003906\\
57.8330001831055 14.7003479003906\\
57.882999420166 14.7003479003906\\
57.9329986572266 14.7003479003906\\
57.9829978942871 14.7003479003906\\
58.0330009460449 14.7003479003906\\
58.0830001831055 14.7003479003906\\
58.132999420166 14.7003479003906\\
58.1829986572266 14.7003479003906\\
58.2329978942871 14.7003479003906\\
58.2830009460449 14.7003479003906\\
58.3330001831055 14.7003479003906\\
58.382999420166 14.7003479003906\\
58.4329986572266 14.7003479003906\\
58.4829978942871 14.7003479003906\\
58.5330009460449 14.7003479003906\\
58.5830001831055 14.7003479003906\\
58.632999420166 14.7003479003906\\
58.6829986572266 14.7003479003906\\
58.7329978942871 14.7003479003906\\
58.7830009460449 14.7003479003906\\
58.8330001831055 14.7003479003906\\
58.882999420166 14.7003479003906\\
58.9329986572266 14.7003479003906\\
58.9829978942871 14.7003479003906\\
59.0330009460449 14.7003479003906\\
59.0830001831055 14.7003479003906\\
59.132999420166 14.7003479003906\\
59.1829986572266 14.7003479003906\\
59.2329978942871 14.7003479003906\\
59.2830009460449 14.7003479003906\\
59.3330001831055 14.7003479003906\\
59.382999420166 14.7003479003906\\
59.4329986572266 14.7003479003906\\
59.4829978942871 14.7003479003906\\
59.5330009460449 14.7003479003906\\
59.5830001831055 14.7003479003906\\
59.632999420166 14.7003479003906\\
59.6829986572266 14.7003479003906\\
59.7329978942871 14.7003479003906\\
59.7830009460449 14.7003479003906\\
59.8330001831055 14.7003479003906\\
59.882999420166 14.7003479003906\\
59.9329986572266 14.7003479003906\\
59.9664993286133 14.7003479003906\\
60 14.7003479003906\\
60.0368499755859 14.7003479003906\\
60.0773849487305 14.7003479003906\\
60.1219749450684 14.7003479003906\\
60.1710243225098 14.7003479003906\\
60.2210235595703 14.7003479003906\\
60.2710227966309 14.7003479003906\\
60.3210220336914 14.7003479003906\\
60.371021270752 14.7003479003906\\
60.4210243225098 14.7003479003906\\
60.4710235595703 14.7003479003906\\
60.5210227966309 14.7003479003906\\
60.5710220336914 14.7003479003906\\
60.621021270752 14.7003479003906\\
60.6710243225098 14.7003479003906\\
60.7210235595703 14.7003479003906\\
60.7710227966309 14.7003479003906\\
60.8210220336914 14.7003479003906\\
60.871021270752 14.7003479003906\\
60.9210243225098 14.7003479003906\\
60.9710235595703 14.7003479003906\\
61.0210227966309 14.7003479003906\\
61.0710220336914 14.7003469467163\\
61.121021270752 14.7003469467163\\
61.1710243225098 14.7003469467163\\
61.2210235595703 14.7003469467163\\
61.2710227966309 14.7003469467163\\
61.3210220336914 14.7003469467163\\
61.371021270752 14.7003469467163\\
61.4210243225098 14.7003469467163\\
61.4710235595703 14.7003469467163\\
61.5210227966309 14.7003469467163\\
61.5710220336914 14.7003469467163\\
61.621021270752 14.7003479003906\\
61.6710243225098 14.7003479003906\\
61.7210235595703 14.7003479003906\\
61.7710227966309 14.7003479003906\\
61.8210220336914 14.7003479003906\\
61.871021270752 14.7003488540649\\
61.9210243225098 14.7003488540649\\
61.9710235595703 14.7003488540649\\
62.0210227966309 14.7003488540649\\
62.0710220336914 14.7003488540649\\
62.121021270752 14.7003488540649\\
62.1710243225098 14.7003488540649\\
62.2210235595703 14.7003488540649\\
62.2710227966309 14.7003488540649\\
62.3210220336914 14.7003488540649\\
62.371021270752 14.7003488540649\\
62.4210243225098 14.7003488540649\\
62.4710235595703 14.7003488540649\\
62.5210227966309 14.7003488540649\\
62.5710220336914 14.7003488540649\\
62.621021270752 14.7003488540649\\
62.6710243225098 14.7003488540649\\
62.7210235595703 14.7003488540649\\
62.7710227966309 14.7003488540649\\
62.8210220336914 14.7003488540649\\
62.871021270752 14.7003488540649\\
62.9210243225098 14.7003488540649\\
62.9710235595703 14.7003488540649\\
63.0210227966309 14.7003488540649\\
63.0710220336914 14.7003488540649\\
63.121021270752 14.7003488540649\\
63.1710243225098 14.7003488540649\\
63.2210235595703 14.7003488540649\\
63.2710227966309 14.7003488540649\\
63.3210220336914 14.7003488540649\\
63.371021270752 14.7003488540649\\
63.4210243225098 14.7003488540649\\
63.4710235595703 14.7003488540649\\
63.5210227966309 14.7003488540649\\
63.5710220336914 14.7003488540649\\
63.621021270752 14.7003488540649\\
63.6710243225098 14.7003488540649\\
63.7210235595703 14.7003488540649\\
63.7710227966309 14.7003488540649\\
63.8210220336914 14.7003488540649\\
63.871021270752 14.7003488540649\\
63.9210243225098 14.7003488540649\\
63.9710235595703 14.7003488540649\\
64.0210189819336 14.7003488540649\\
64.0710220336914 14.7003488540649\\
64.1210250854492 14.7003488540649\\
64.1710205078125 14.7003488540649\\
64.2210235595703 14.7003488540649\\
64.2710189819336 14.7003488540649\\
64.3210220336914 14.7003488540649\\
64.3710250854492 14.7003488540649\\
64.4210205078125 14.7003488540649\\
64.4710235595703 14.7003488540649\\
64.5210189819336 14.7003488540649\\
64.5710220336914 14.7003488540649\\
64.6210250854492 14.7003488540649\\
64.6710205078125 14.7003488540649\\
64.7210235595703 14.7003488540649\\
64.7710189819336 14.7003488540649\\
64.8210220336914 14.7003488540649\\
64.8710250854492 14.7003488540649\\
64.9210205078125 14.7003488540649\\
64.9710235595703 14.7003488540649\\
65.0210189819336 14.7003488540649\\
65.0710220336914 14.7003488540649\\
65.1210250854492 14.7003488540649\\
65.1710205078125 14.7003488540649\\
65.2210235595703 14.7003488540649\\
65.2710189819336 14.7003488540649\\
65.3210220336914 14.7003488540649\\
65.3710250854492 14.7003488540649\\
65.4210205078125 14.7003488540649\\
65.4710235595703 14.7003488540649\\
65.5210189819336 14.7003488540649\\
65.5710220336914 14.7003488540649\\
65.6210250854492 14.7003488540649\\
65.6710205078125 14.7003488540649\\
65.7210235595703 14.7003488540649\\
65.7710189819336 14.7003488540649\\
65.8210220336914 14.7003488540649\\
65.8710250854492 14.7003488540649\\
65.9210205078125 14.7003488540649\\
65.9710235595703 14.7003488540649\\
66.0210189819336 14.7003488540649\\
66.0710220336914 14.7003488540649\\
66.1210250854492 14.7003488540649\\
66.1710205078125 14.7003488540649\\
66.2210235595703 14.7003488540649\\
66.2710189819336 14.7003488540649\\
66.3210220336914 14.7003488540649\\
66.3710250854492 14.7003488540649\\
66.4210205078125 14.7003488540649\\
66.4710235595703 14.7003488540649\\
66.5210189819336 14.7003488540649\\
66.5710220336914 14.7003488540649\\
66.6210250854492 14.7003488540649\\
66.6710205078125 14.7003488540649\\
66.7210235595703 14.7003488540649\\
66.7710189819336 14.7003498077393\\
66.8210220336914 14.7003507614136\\
66.8710250854492 14.7003526687622\\
66.9210205078125 14.7003545761108\\
66.9710235595703 14.7003631591797\\
67.0210189819336 14.7003784179688\\
67.0710220336914 14.7004137039185\\
67.1210250854492 14.7004947662354\\
67.1710205078125 14.7006845474243\\
67.2210235595703 14.7011270523071\\
67.2710189819336 14.7021389007568\\
67.3210220336914 14.7043256759644\\
67.3710250854492 14.7085828781128\\
67.4210205078125 14.7155389785767\\
67.4710235595703 14.7244033813477\\
67.5210189819336 14.7364835739136\\
67.5710220336914 14.7313833236694\\
67.6210250854492 14.7318706512451\\
67.6710205078125 14.7322731018066\\
67.7210235595703 14.7324552536011\\
67.7710189819336 14.7320680618286\\
67.8210220336914 14.7329444885254\\
67.8710250854492 14.7323942184448\\
67.9210205078125 14.7324752807617\\
67.9710235595703 14.7326250076294\\
68.0210189819336 14.7329301834106\\
68.0710220336914 14.7331771850586\\
68.1210250854492 14.7334518432617\\
68.1710205078125 14.7337131500244\\
68.2210235595703 14.7339811325073\\
68.2710189819336 14.7342462539673\\
68.3210220336914 14.7345123291016\\
68.3710250854492 14.7347774505615\\
68.4210205078125 14.7350435256958\\
68.4710235595703 14.7353086471558\\
68.5210189819336 14.73557472229\\
68.5710220336914 14.7358407974243\\
68.6210250854492 14.7361068725586\\
68.6710205078125 14.7363719940186\\
68.7210235595703 14.7366380691528\\
68.7710189819336 14.7369041442871\\
68.8210220336914 14.7371702194214\\
68.8710250854492 14.7374362945557\\
68.9210205078125 14.7377023696899\\
68.9710235595703 14.7379684448242\\
69.0210189819336 14.7382345199585\\
69.0710220336914 14.7385005950928\\
69.1210250854492 14.7387666702271\\
69.1710205078125 14.7390327453613\\
69.2210235595703 14.7392988204956\\
69.2710189819336 14.7395648956299\\
69.3210220336914 14.7398319244385\\
69.3710250854492 14.7400979995728\\
69.4210205078125 14.740364074707\\
69.4710235595703 14.7406311035156\\
69.5210189819336 14.7408971786499\\
69.5710220336914 14.7411642074585\\
69.6210250854492 14.7414312362671\\
69.6710205078125 14.7416973114014\\
69.7210235595703 14.74196434021\\
69.7710189819336 14.7422313690186\\
69.8210220336914 14.7424983978271\\
69.8710250854492 14.7427654266357\\
69.9210205078125 14.7430324554443\\
69.9710235595703 14.7433004379272\\
70.0210189819336 14.7435674667358\\
70.0710220336914 14.7440662384033\\
70.1210250854492 14.7444314956665\\
70.1710205078125 14.7447595596313\\
70.2210235595703 14.7450819015503\\
70.2710189819336 14.7454023361206\\
70.3210220336914 14.7457227706909\\
70.3710250854492 14.7460432052612\\
70.4210205078125 14.7463626861572\\
70.4710235595703 14.7466821670532\\
70.5210189819336 14.7470006942749\\
70.5710220336914 14.7473201751709\\
70.6210250854492 14.7476387023926\\
70.6710205078125 14.7479572296143\\
70.7210235595703 14.7482757568359\\
70.7710189819336 14.7485933303833\\
70.8210220336914 14.748911857605\\
70.8710250854492 14.7492294311523\\
70.9210205078125 14.7495470046997\\
70.9710235595703 14.7498645782471\\
71.0210189819336 14.7501821517944\\
71.0710220336914 14.7504997253418\\
71.1210250854492 14.7508172988892\\
71.1710205078125 14.7511348724365\\
71.2210235595703 14.7514524459839\\
71.2710189819336 14.7517690658569\\
71.3210220336914 14.7520866394043\\
71.3710250854492 14.7524042129517\\
71.4210205078125 14.752721786499\\
71.4710235595703 14.7530403137207\\
71.5210189819336 14.7533521652222\\
71.5710220336914 14.7536392211914\\
71.6210250854492 14.7536859512329\\
71.6710205078125 14.7539653778076\\
71.7210235595703 14.7542276382446\\
71.7710189819336 14.7544994354248\\
71.8210220336914 14.754768371582\\
71.8710250854492 14.7550373077393\\
71.9210205078125 14.7553052902222\\
71.9710235595703 14.7555723190308\\
72.0210189819336 14.7558393478394\\
72.0710220336914 14.7561054229736\\
72.1210250854492 14.7563705444336\\
72.1710205078125 14.7566347122192\\
72.2210235595703 14.7568988800049\\
72.2710189819336 14.7571611404419\\
72.3210220336914 14.7574234008789\\
72.3710250854492 14.7576837539673\\
72.4210205078125 14.7579441070557\\
72.4710235595703 14.7582025527954\\
72.5210189819336 14.7584600448608\\
72.5710220336914 14.758716583252\\
72.6210250854492 14.7589836120605\\
72.6710205078125 14.7592973709106\\
72.7210235595703 14.7595891952515\\
72.7710189819336 14.7598667144775\\
72.8210220336914 14.7602462768555\\
72.8710250854492 14.7605676651001\\
72.9210205078125 14.7608652114868\\
72.9710235595703 14.7611722946167\\
73.0210189819336 14.7614831924438\\
73.0710220336914 14.7618007659912\\
73.1210250854492 14.7621259689331\\
73.1710205078125 14.7624607086182\\
73.2210235595703 14.762806892395\\
73.2710189819336 14.7631683349609\\
73.3210220336914 14.7635478973389\\
73.3710250854492 14.763952255249\\
73.4210205078125 14.7643899917603\\
73.4710235595703 14.7700481414795\\
73.5210189819336 14.7718496322632\\
73.5710220336914 14.7735862731934\\
73.6210250854492 14.7753210067749\\
73.6710205078125 14.7734422683716\\
73.7210235595703 14.767991065979\\
73.7710189819336 14.7682199478149\\
73.8210220336914 14.768443107605\\
73.8710250854492 14.768726348877\\
73.9210205078125 14.7690658569336\\
73.9710235595703 14.7694540023804\\
74.0210189819336 14.7698860168457\\
74.0710220336914 14.7703504562378\\
74.1210250854492 14.7708368301392\\
74.1710205078125 14.771336555481\\
74.2210235595703 14.7718467712402\\
74.2710189819336 14.772364616394\\
74.3210220336914 14.7728872299194\\
74.3710250854492 14.7734146118164\\
74.4210205078125 14.773946762085\\
74.4710235595703 14.7744817733765\\
74.5210189819336 14.7750205993652\\
74.5710220336914 14.7755632400513\\
74.6210250854492 14.7761058807373\\
74.6710205078125 14.776650428772\\
74.7210235595703 14.7771968841553\\
74.7710189819336 14.7777452468872\\
74.8210220336914 14.7782964706421\\
74.8710250854492 14.7788486480713\\
74.9210205078125 14.7794017791748\\
74.9605102539063 14.7799577713013\\
75 14.7803945541382\\
};
\addlegendentry{22 "};

\end{axis}
\end{tikzpicture}%
\caption{Filling problem with \dtmax{} = \expneg{5.0}{2}{[s]} using the linear solver.}
\label{fig:vmpLinFill5em2}
\end{figure}

The solution from the simulation with \dtmax{} = \expneg{5.0}{2}{[s]}, \fig{fig:vmpLinFill5em2}, is more in line with the analytic solution, \fig{fig:vmpAnalyticSol}.
The slight spike visible in \fig{fig:vmpLinFill5em2} are characteristic of \cobra{} software.
They cannot be removed by either resolution of nonlinearities or reduction in timestep size.
\tab{tab:vmpLinRunTime} shows the run time data for the linear cases that were run.

\begin{table}[h!tb]
\centering
\singlespace
\pgfplotstabletypeset[sci zerofill,sci E, col sep=comma,
	columns/0/.style={ column name= $[ \text{s} ]$, precision=1},
	columns/1/.style={ column name= $[ - ]$, int detect},
	columns/2/.style={ column name= $[ \text{s} ]$, precision=3},
	columns/3/.style={ column name= $[ \text{s} ]$, precision=3},
	every head row/.style={
		before row={
			\toprule
			\dtmax{} & $N_{t}$ & $T_{\text{CPU}}$& $\frac{T_{\text{CPU}}}{N_{t}}$\\
		},
		after row=\midrule
	},
	every last row/.style={
after row=\bottomrule}]{tables/vmpLinRunTimeData.tex}
\caption{Linear solver's data for the fill problem.}
\label{tab:vmpLinRunTime}
\end{table}

With the nonlinear solver active, the same three timestep sizes were run.
The pressures from the \dtmax{} = \expneg{1.0}{1}{[s]} case is shown in \fig{fig:nlnFill1em1}.
The measure of error, $e_{p}$, for this simulation was \expneg{3.5}{2}{[psia]}.
This error measure was half that of the linear case at the same \dtmax{}.

\begin{figure}[h!tb]
\centering
% This file was created by matlab2tikz v0.4.3.
% Copyright (c) 2008--2013, Nico Schlömer <nico.schloemer@gmail.com>
% All rights reserved.
% 
\tikzsetnextfilename{plots/vmpNLN1em1_eps}
\begin{tikzpicture}

\begin{axis}[%
width=0.6\textwidth,
height=0.381611494576701\textwidth,
scale only axis,
xmin=0,
xmax=75,
xlabel={Time [s]},
ymin=14.7,
ymax=15.8,
ylabel={Pressure [psia]},
legend style={at={(0.03,0.97)},anchor=north west,draw=black,fill=white,legend cell align=left}
]
\addplot [
color=black,
solid
]
table[row sep=crcr]{
0 14.7007818222046\\
9.99999971718069e-10 14.7007818222046\\
2.49999998480632e-09 14.7007818222046\\
4.74999994892755e-09 14.7007818222046\\
8.1250002281763e-09 14.7007818222046\\
1.31874999809156e-08 14.7007818222046\\
2.0781250498203e-08 14.7007818222046\\
3.21718758300449e-08 14.7007818222046\\
4.92578138278077e-08 14.7007818222046\\
7.48867208244519e-08 14.7007818222046\\
1.13330081319418e-07 14.7007818222046\\
1.70995122061868e-07 14.7007818222046\\
2.57492672517401e-07 14.7007818222046\\
3.8723902662241e-07 14.7007818222046\\
5.81858500936505e-07 14.7007818222046\\
8.73787769251066e-07 14.7007818222046\\
1.31168167172291e-06 14.7007818222046\\
1.9685226106958e-06 14.7007827758789\\
2.95378367809462e-06 14.7007827758789\\
4.43167573394021e-06 14.7007827758789\\
6.64851359033491e-06 14.7007827758789\\
9.97376992017962e-06 14.7007837295532\\
1.49616553244414e-05 14.7007846832275\\
2.24434825213393e-05 14.7007856369019\\
3.36662233166862e-05 14.7007865905762\\
5.05003372381907e-05 14.7007884979248\\
7.57515081204474e-05 14.7007904052734\\
0.000113628258986864 14.7007932662964\\
0.000170443381648511 14.700795173645\\
0.000255666091106832 14.7007970809937\\
0.000383500126190484 14.7007970809937\\
0.000575251178815961 14.7007942199707\\
0.000862877757754177 14.7007904052734\\
0.00129431765526533 14.7007865905762\\
0.00194147753063589 14.7007837295532\\
0.00291221728548408 14.700779914856\\
0.00436832662671804 14.7007780075073\\
0.00597004732117057 14.700779914856\\
0.00773194013163447 14.7007827758789\\
0.009670021943748 14.7007837295532\\
0.0118019115179777 14.7007827758789\\
0.0141469910740852 14.7007827758789\\
0.0167265776544809 14.7007827758789\\
0.0195641238242388 14.7007827758789\\
0.0226854234933853 14.7007827758789\\
0.0261188540607691 14.7007827758789\\
0.0298956278711557 14.7007827758789\\
0.0340500771999359 14.7007827758789\\
0.0386199727654457 14.7007827758789\\
0.0436468608677387 14.7007827758789\\
0.0491764321923256 14.7007827758789\\
0.0552589632570744 14.7007827758789\\
0.0619497485458851 14.7007827758789\\
0.0693096145987511 14.7007827758789\\
0.0774054601788521 14.700777053833\\
0.0863108932971954 14.7007865905762\\
0.0961068719625473 14.7007827758789\\
0.106882445514202 14.7007827758789\\
0.118735581636429 14.7007827758789\\
0.13177402317524 14.7007827758789\\
0.146116316318512 14.7007827758789\\
0.161892831325531 14.7007827758789\\
0.179247006773949 14.7007827758789\\
0.198336601257324 14.7007827758789\\
0.21933513879776 14.7007827758789\\
0.242433547973633 14.7007827758789\\
0.267841786146164 14.7007827758789\\
0.29579085111618 14.7007827758789\\
0.32653483748436 14.7007827758789\\
0.360353201627731 14.7007827758789\\
0.397553414106369 14.7007837295532\\
0.438473641872406 14.7007837295532\\
0.483485877513886 14.7007837295532\\
0.532999336719513 14.7007837295532\\
0.587464153766632 14.7007837295532\\
0.647375464439392 14.7007837295532\\
0.71327793598175 14.7007837295532\\
0.7857705950737 14.7007837295532\\
0.865512549877167 14.7007837295532\\
0.953228712081909 14.7007837295532\\
1.04971647262573 14.7007837295532\\
1.14971649646759 14.7007827758789\\
1.24971640110016 14.7007827758789\\
1.34971642494202 14.7007827758789\\
1.44971644878387 14.7007827758789\\
1.54971647262573 14.7007827758789\\
1.64971649646759 14.7007827758789\\
1.74971640110016 14.7007827758789\\
1.84971642494202 14.7007827758789\\
1.94971644878387 14.7007827758789\\
2.04971647262573 14.7007827758789\\
2.1497163772583 14.7007827758789\\
2.24971652030945 14.7007827758789\\
2.34971642494202 14.7007827758789\\
2.44971656799316 14.7007827758789\\
2.54971647262573 14.7007827758789\\
2.6497163772583 14.7007827758789\\
2.74971652030945 14.7007827758789\\
2.84971642494202 14.7007827758789\\
2.94971656799316 14.7007827758789\\
3.04971647262573 14.7007827758789\\
3.1497163772583 14.7007827758789\\
3.24971652030945 14.7007827758789\\
3.34971642494202 14.7007827758789\\
3.44971656799316 14.7007827758789\\
3.54971647262573 14.7007827758789\\
3.6497163772583 14.7007827758789\\
3.74971652030945 14.7007827758789\\
3.84971642494202 14.7007827758789\\
3.94971656799316 14.7007827758789\\
4.04971647262573 14.7007827758789\\
4.1497163772583 14.7007827758789\\
4.24971628189087 14.7007827758789\\
4.3497166633606 14.7007827758789\\
4.44971656799316 14.7007827758789\\
4.54971647262573 14.7007827758789\\
4.6497163772583 14.7007827758789\\
4.74971628189087 14.7007827758789\\
4.8497166633606 14.7007827758789\\
4.94971656799316 14.7007827758789\\
5.04971647262573 14.7007837295532\\
5.1497163772583 14.7007837295532\\
5.24971628189087 14.7007837295532\\
5.3497166633606 14.7007837295532\\
5.44971656799316 14.7007837295532\\
5.54971647262573 14.7007837295532\\
5.6497163772583 14.7007846832275\\
5.74971628189087 14.7007894515991\\
5.8497166633606 14.7008085250854\\
5.94971656799316 14.7009077072144\\
6.04971647262573 14.7014436721802\\
6.1497163772583 14.7040977478027\\
6.24971628189087 14.7128324508667\\
6.3497166633606 14.7271690368652\\
6.44971656799316 14.7334566116333\\
6.54971647262573 14.7314443588257\\
6.6497163772583 14.7334966659546\\
6.74971628189087 14.7322854995728\\
6.8497166633606 14.7336616516113\\
6.94971656799316 14.7335815429688\\
7.04971647262573 14.7345237731934\\
7.1497163772583 14.7347774505615\\
7.24971628189087 14.7354955673218\\
7.3497166633606 14.7358999252319\\
7.44971656799316 14.7365169525146\\
7.54971647262573 14.7369899749756\\
7.6497163772583 14.7375602722168\\
7.74971628189087 14.7380647659302\\
7.8497166633606 14.7386140823364\\
7.94971656799316 14.7391328811646\\
8.04971599578857 14.739673614502\\
8.1497163772583 14.7401990890503\\
8.24971675872803 14.7407350540161\\
8.34971618652344 14.7412633895874\\
8.44971656799316 14.7417974472046\\
8.54971599578857 14.7423276901245\\
8.6497163772583 14.7428607940674\\
8.74971675872803 14.7433929443359\\
8.84971618652344 14.7439479827881\\
8.94971656799316 14.7448120117188\\
9.04971599578857 14.7454252243042\\
9.1497163772583 14.7460870742798\\
9.24971675872803 14.7467250823975\\
9.34971618652344 14.7473707199097\\
9.44971656799316 14.748010635376\\
9.54971599578857 14.7486515045166\\
9.6497163772583 14.7492904663086\\
9.74971675872803 14.7499284744263\\
9.84971618652344 14.7505655288696\\
9.94971656799316 14.751202583313\\
10.0497159957886 14.751838684082\\
10.1497163772583 14.7524747848511\\
10.249716758728 14.7531108856201\\
10.3497161865234 14.7537298202515\\
10.4497165679932 14.7540988922119\\
10.5497159957886 14.7545909881592\\
10.6497163772583 14.7551307678223\\
10.749716758728 14.755651473999\\
10.8497161865234 14.7561731338501\\
10.9497165679932 14.756688117981\\
11.0497159957886 14.7571992874146\\
11.1497163772583 14.7577028274536\\
11.249716758728 14.7581987380981\\
11.3497161865234 14.7586870193481\\
11.4497165679932 14.7591991424561\\
11.5497159957886 14.7598543167114\\
11.6497163772583 14.7605504989624\\
11.749716758728 14.7611818313599\\
11.8497161865234 14.7618141174316\\
11.9497165679932 14.7624731063843\\
12.0497159957886 14.7631711959839\\
12.1497163772583 14.7639303207397\\
12.249716758728 14.7665214538574\\
12.3471450805664 14.7732496261597\\
12.4250917434692 14.7781705856323\\
12.4879579544067 14.768609046936\\
12.557110786438 14.7686433792114\\
12.6331787109375 14.7690782546997\\
12.7168531417847 14.7695474624634\\
12.808895111084 14.7702302932739\\
12.9088954925537 14.7711057662964\\
13.0088949203491 14.7721309661865\\
13.1088953018188 14.7731733322144\\
13.2088947296143 14.7742366790771\\
13.308895111084 14.7753143310547\\
13.4088954925537 14.7763986587524\\
13.5088949203491 14.777494430542\\
13.6088953018188 14.7785978317261\\
13.7088947296143 14.7797069549561\\
13.808895111084 14.780816078186\\
13.9088954925537 14.781925201416\\
14.0088949203491 14.783034324646\\
14.1088953018188 14.784143447876\\
14.2088947296143 14.7852535247803\\
14.308895111084 14.7863636016846\\
14.4088954925537 14.7874736785889\\
14.5088949203491 14.7885837554932\\
14.6088953018188 14.7896947860718\\
14.7088947296143 14.7908058166504\\
14.808895111084 14.7919178009033\\
14.9088954925537 14.7930297851563\\
15.0088949203491 14.7941417694092\\
15.1088953018188 14.7952537536621\\
15.2088947296143 14.7963666915894\\
15.308895111084 14.7974805831909\\
15.4088954925537 14.7986001968384\\
15.5088949203491 14.7997035980225\\
15.6088953018188 14.8008260726929\\
15.7088947296143 14.8019437789917\\
15.808895111084 14.8030633926392\\
15.9088954925537 14.8041830062866\\
16.0088958740234 14.8053026199341\\
16.1088943481445 14.8064222335815\\
16.2088947296143 14.8075428009033\\
16.308895111084 14.8086624145508\\
16.4088954925537 14.8097829818726\\
16.5088958740234 14.8109035491943\\
16.6088943481445 14.8120241165161\\
16.7088947296143 14.8131446838379\\
16.808895111084 14.814266204834\\
16.9088954925537 14.8153867721558\\
17.0088958740234 14.8165082931519\\
17.1088943481445 14.8176298141479\\
17.2088947296143 14.818751335144\\
17.308895111084 14.8198738098145\\
17.4088954925537 14.8209953308105\\
17.5088958740234 14.822117805481\\
17.6088943481445 14.8232402801514\\
17.7088947296143 14.8243627548218\\
17.808895111084 14.8254852294922\\
17.9088954925537 14.8266096115112\\
18.0088958740234 14.8277368545532\\
18.1088943481445 14.8288841247559\\
18.2088947296143 14.830132484436\\
18.308895111084 14.8319272994995\\
18.4088954925537 14.8359346389771\\
18.5088958740234 14.8431234359741\\
18.6088943481445 14.8468885421753\\
18.7088947296143 14.8412246704102\\
18.808895111084 14.8385257720947\\
18.9088954925537 14.8396024703979\\
19.0088958740234 14.8392372131348\\
19.1088943481445 14.8409767150879\\
19.2088947296143 14.8416633605957\\
19.308895111084 14.8430795669556\\
19.4088954925537 14.8440103530884\\
19.5088958740234 14.8452653884888\\
19.6088943481445 14.8463039398193\\
19.7088947296143 14.8474884033203\\
19.808895111084 14.848575592041\\
19.9088954925537 14.8497276306152\\
20.0088958740234 14.8508358001709\\
20.1088943481445 14.8519744873047\\
20.2088947296143 14.8530921936035\\
20.308895111084 14.8542242050171\\
20.4088954925537 14.8553476333618\\
20.5088958740234 14.8564767837524\\
20.6088943481445 14.8576021194458\\
20.7088947296143 14.8587303161621\\
20.808895111084 14.8598566055298\\
20.9088954925537 14.8609848022461\\
21.0088958740234 14.8621110916138\\
21.1088943481445 14.8633079528809\\
21.2088947296143 14.8647031784058\\
21.308895111084 14.8657855987549\\
21.4088954925537 14.8669328689575\\
21.5088958740234 14.8680543899536\\
21.6088943481445 14.8691864013672\\
21.7088947296143 14.8703145980835\\
21.808895111084 14.8714447021484\\
21.9088954925537 14.8725738525391\\
22.0088958740234 14.8737049102783\\
22.1088943481445 14.8748359680176\\
22.2088947296143 14.8759670257568\\
22.308895111084 14.8770990371704\\
22.4088954925537 14.878231048584\\
22.5088958740234 14.8793640136719\\
22.6088943481445 14.8804807662964\\
22.7088947296143 14.8813285827637\\
22.808895111084 14.8824396133423\\
22.9088954925537 14.8835592269897\\
23.0088958740234 14.8846778869629\\
23.1088943481445 14.8857955932617\\
23.2088947296143 14.8869123458862\\
23.308895111084 14.8880281448364\\
23.4088954925537 14.889142036438\\
23.5088958740234 14.8902549743652\\
23.6088943481445 14.8913650512695\\
23.7088947296143 14.8925113677979\\
23.808895111084 14.8937644958496\\
23.9088954925537 14.894998550415\\
24.0088958740234 14.8961601257324\\
24.1088943481445 14.8973197937012\\
24.2088947296143 14.8984975814819\\
24.308895111084 14.8997011184692\\
24.4088954925537 14.9009475708008\\
24.5088958740234 14.9045486450195\\
24.6048831939697 14.910834312439\\
24.681676864624 14.9136018753052\\
24.7464084625244 14.9051694869995\\
24.8176136016846 14.9055252075195\\
24.895938873291 14.9060869216919\\
24.9820976257324 14.9067459106445\\
25.0768718719482 14.9075870513916\\
25.176872253418 14.9086112976074\\
25.2768707275391 14.9097194671631\\
25.3768711090088 14.9108257293701\\
25.4768714904785 14.911937713623\\
25.5768718719482 14.9130535125732\\
25.676872253418 14.9141712188721\\
25.7768707275391 14.9152879714966\\
25.8768711090088 14.9164094924927\\
25.9768714904785 14.9175319671631\\
26.0768718719482 14.9186611175537\\
26.176872253418 14.9197883605957\\
26.2768707275391 14.920916557312\\
26.3768711090088 14.9220447540283\\
26.4768714904785 14.9231729507446\\
26.5768718719482 14.9243001937866\\
26.676872253418 14.9254283905029\\
26.7768707275391 14.9265565872192\\
26.8768711090088 14.9276838302612\\
26.9768714904785 14.9288120269775\\
27.0768718719482 14.9299402236938\\
27.176872253418 14.9310674667358\\
27.2768707275391 14.9321956634521\\
27.3768711090088 14.9333238601685\\
27.4768714904785 14.9344511032104\\
27.5768718719482 14.9355792999268\\
27.676872253418 14.9367065429688\\
27.7768707275391 14.9378347396851\\
27.8768711090088 14.9389629364014\\
27.9768714904785 14.9400901794434\\
28.0768718719482 14.9412183761597\\
28.176872253418 14.9423456192017\\
28.2768707275391 14.9434804916382\\
28.3768711090088 14.9445905685425\\
28.4768714904785 14.9457235336304\\
28.5768718719482 14.946849822998\\
28.676872253418 14.9479780197144\\
28.7768707275391 14.9491062164307\\
28.8768711090088 14.950234413147\\
28.9768714904785 14.9513626098633\\
29.0768718719482 14.9524898529053\\
29.176872253418 14.9536180496216\\
29.2768707275391 14.9547462463379\\
29.3768711090088 14.9558744430542\\
29.4768714904785 14.9570026397705\\
29.5768718719482 14.9581298828125\\
29.676872253418 14.9592580795288\\
29.7768707275391 14.9603862762451\\
29.8768711090088 14.9615144729614\\
29.9768714904785 14.9626426696777\\
30.0768718719482 14.963770866394\\
30.176872253418 14.964900970459\\
30.2768707275391 14.9660387039185\\
30.3768711090088 14.9672107696533\\
30.4768714904785 14.9685831069946\\
30.5768718719482 14.9709510803223\\
30.676872253418 14.9764404296875\\
30.7768707275391 14.9835319519043\\
30.8768711090088 14.9819793701172\\
30.9768714904785 14.9787912368774\\
31.0768718719482 14.9764232635498\\
31.176872253418 14.9774360656738\\
31.2768707275391 14.9776411056519\\
31.3768711090088 14.9792537689209\\
31.4768714904785 14.9800567626953\\
31.5768718719482 14.9814014434814\\
31.676872253418 14.982385635376\\
31.7768707275391 14.9836101531982\\
31.8768711090088 14.9846744537354\\
31.9768714904785 14.9858465194702\\
32.0768699645996 14.986946105957\\
32.176872253418 14.988094329834\\
32.2768707275391 14.9892101287842\\
32.3768730163574 14.9903469085693\\
32.4768714904785 14.9914693832397\\
32.5768699645996 14.9926023483276\\
32.676872253418 14.9937286376953\\
32.7768707275391 14.9948596954346\\
32.8768730163574 14.9959878921509\\
32.9768714904785 14.9971179962158\\
33.0768699645996 14.9982461929321\\
33.176872253418 14.9993753433228\\
33.2768707275391 15.0005044937134\\
33.3768730163574 15.0018339157104\\
33.4768714904785 15.0030736923218\\
33.5768699645996 15.0041904449463\\
33.676872253418 15.0053262710571\\
33.7768707275391 15.0064544677734\\
33.8768730163574 15.0075855255127\\
33.9768714904785 15.0087156295776\\
34.0768699645996 15.0098466873169\\
34.176872253418 15.0109786987305\\
34.2768707275391 15.0121097564697\\
34.3768730163574 15.0132417678833\\
34.4768714904785 15.0143747329712\\
34.5768699645996 15.0155076980591\\
34.676872253418 15.0166416168213\\
34.7768707275391 15.0177755355835\\
34.8768730163574 15.0188503265381\\
34.9768714904785 15.0196676254272\\
35.0768699645996 15.0208778381348\\
35.176872253418 15.0219659805298\\
35.2768707275391 15.0230979919434\\
35.3768730163574 15.0242118835449\\
35.4768714904785 15.0253314971924\\
35.5768699645996 15.0264482498169\\
35.676872253418 15.0275640487671\\
35.7768707275391 15.0286779403687\\
35.8768730163574 15.0297899246216\\
35.9768714904785 15.0309648513794\\
36.0768699645996 15.0322227478027\\
36.176872253418 15.0334167480469\\
36.2768707275391 15.034571647644\\
36.3768730163574 15.0357360839844\\
36.4768714904785 15.036919593811\\
36.5768699645996 15.0381326675415\\
36.676872253418 15.0393943786621\\
36.7768707275391 15.0450391769409\\
36.8673248291016 15.049633026123\\
36.9396858215332 15.0495872497559\\
37.0192794799805 15.043267250061\\
37.1068305969238 15.0439195632935\\
37.2031402587891 15.0445756912231\\
37.3031425476074 15.0454921722412\\
37.4031410217285 15.046555519104\\
37.5031433105469 15.0476732254028\\
37.603141784668 15.0487785339355\\
37.7031402587891 15.0498905181885\\
37.8031425476074 15.0510053634644\\
37.9031410217285 15.0521259307861\\
38.0031433105469 15.053240776062\\
38.103141784668 15.0543622970581\\
38.2031402587891 15.0554847717285\\
38.3031425476074 15.0566120147705\\
38.4031410217285 15.0577402114868\\
38.5031433105469 15.0588684082031\\
38.603141784668 15.0599966049194\\
38.7031402587891 15.0611248016357\\
38.8031425476074 15.0622529983521\\
38.9031410217285 15.0633811950684\\
39.0031433105469 15.064510345459\\
39.103141784668 15.0656385421753\\
39.2031402587891 15.0667667388916\\
39.3031425476074 15.0678939819336\\
39.4031410217285 15.0690221786499\\
39.5031433105469 15.0701503753662\\
39.603141784668 15.0712785720825\\
39.7031402587891 15.0724067687988\\
39.8031425476074 15.0735349655151\\
39.9031410217285 15.0746631622314\\
40.0031433105469 15.0757913589478\\
40.103141784668 15.0769195556641\\
40.2031402587891 15.0780477523804\\
40.3031425476074 15.0791759490967\\
40.4031410217285 15.080304145813\\
40.5031433105469 15.0814323425293\\
40.603141784668 15.0825595855713\\
40.7031402587891 15.0836877822876\\
40.8031425476074 15.0848159790039\\
40.9031410217285 15.0859508514404\\
41.0031433105469 15.0870609283447\\
41.103141784668 15.0881948471069\\
41.2031402587891 15.0893211364746\\
41.3031425476074 15.0904502868652\\
41.4031410217285 15.0915775299072\\
41.5031433105469 15.0927066802979\\
41.603141784668 15.0938348770142\\
41.7031402587891 15.0949630737305\\
41.8031425476074 15.0960912704468\\
41.9031410217285 15.0972194671631\\
42.0031433105469 15.0983476638794\\
42.103141784668 15.0994758605957\\
42.2031402587891 15.100604057312\\
42.3031425476074 15.1017332077026\\
42.4031410217285 15.1028623580933\\
42.5031433105469 15.1039972305298\\
42.603141784668 15.1051588058472\\
42.7031402587891 15.1064653396606\\
42.8031425476074 15.1085252761841\\
42.9031410217285 15.1132879257202\\
43.0031433105469 15.1205997467041\\
43.103141784668 15.1216335296631\\
43.2031402587891 15.1170101165771\\
43.3031425476074 15.1146373748779\\
43.4031410217285 15.1156311035156\\
43.5031433105469 15.1155347824097\\
43.603141784668 15.1172714233398\\
43.7031402587891 15.1179876327515\\
43.8031425476074 15.1193904876709\\
43.9031410217285 15.1203365325928\\
44.0031433105469 15.1215858459473\\
44.103141784668 15.122633934021\\
44.2031402587891 15.1238164901733\\
44.3031425476074 15.1249094009399\\
44.4031410217285 15.1260623931885\\
44.5031433105469 15.1271753311157\\
44.603141784668 15.1283149719238\\
44.7031402587891 15.1294364929199\\
44.8031425476074 15.1305704116821\\
44.9031410217285 15.1316957473755\\
45.0031433105469 15.1328277587891\\
45.103141784668 15.1339550018311\\
45.2031402587891 15.1350860595703\\
45.3031425476074 15.1362142562866\\
45.4031410217285 15.1373443603516\\
45.5031433105469 15.1384735107422\\
45.603141784668 15.1397323608398\\
45.7031402587891 15.1410551071167\\
45.8031425476074 15.1421518325806\\
45.9031410217285 15.1432952880859\\
46.0031433105469 15.1444206237793\\
46.103141784668 15.1455535888672\\
46.2031402587891 15.1466836929321\\
46.3031425476074 15.1478147506714\\
46.4031410217285 15.148946762085\\
46.5031433105469 15.1500787734985\\
46.603141784668 15.1512107849121\\
46.7031402587891 15.15234375\\
46.8031425476074 15.1534767150879\\
46.9031410217285 15.1546106338501\\
47.0031433105469 15.1557445526123\\
47.103141784668 15.1568603515625\\
47.2031402587891 15.1576738357544\\
47.3031425476074 15.1588430404663\\
47.4031410217285 15.1599454879761\\
47.5031433105469 15.1610727310181\\
47.603141784668 15.1621894836426\\
47.7031402587891 15.1633100509644\\
47.8031425476074 15.1644268035889\\
47.9031410217285 15.1655435562134\\
48.0031433105469 15.1666593551636\\
48.103141784668 15.1677732467651\\
48.2031402587891 15.1689252853394\\
48.3031425476074 15.1701631546021\\
48.4031410217285 15.1713724136353\\
48.5031433105469 15.1725301742554\\
48.603141784668 15.1736907958984\\
48.7031402587891 15.1748695373535\\
48.8031425476074 15.1760740280151\\
48.9031410217285 15.1773223876953\\
49.0031433105469 15.1814069747925\\
49.0980262756348 15.1872444152832\\
49.1739387512207 15.1892910003662\\
49.2413482666016 15.1814727783203\\
49.315502166748 15.1819105148315\\
49.3970718383789 15.1824760437012\\
49.4867973327637 15.1831884384155\\
49.5854949951172 15.1840810775757\\
49.6854934692383 15.1851644515991\\
49.7854957580566 15.1862697601318\\
49.8854942321777 15.1873788833618\\
49.9854927062988 15.1884908676147\\
50.0854949951172 15.1896085739136\\
50.1854934692383 15.1907272338867\\
50.2854957580566 15.1918439865112\\
50.3854942321777 15.1929664611816\\
50.4854927062988 15.1940898895264\\
50.5854949951172 15.1952180862427\\
50.6854934692383 15.196346282959\\
50.7854957580566 15.1974744796753\\
50.8854942321777 15.1986026763916\\
50.9854927062988 15.1997308731079\\
51.0854949951172 15.2008600234985\\
51.1854934692383 15.2019882202148\\
51.2854957580566 15.2031164169312\\
51.3854942321777 15.2042446136475\\
51.4854927062988 15.2053728103638\\
51.5854949951172 15.2065010070801\\
51.6854934692383 15.2076292037964\\
51.7854957580566 15.2087574005127\\
51.8854942321777 15.209885597229\\
51.9854927062988 15.2110137939453\\
52.0854949951172 15.2121429443359\\
52.1854934692383 15.2132711410522\\
52.2854957580566 15.2143993377686\\
52.3854942321777 15.2155275344849\\
52.4854927062988 15.2166557312012\\
52.5854949951172 15.2177839279175\\
52.6854934692383 15.2189121246338\\
52.7854957580566 15.2200403213501\\
52.8854942321777 15.2211685180664\\
52.9854927062988 15.2222967147827\\
53.0854949951172 15.223424911499\\
53.1854934692383 15.2245531082153\\
53.2854957580566 15.2256879806519\\
53.3854942321777 15.2267980575562\\
53.4854927062988 15.227931022644\\
53.5854949951172 15.229058265686\\
53.6854934692383 15.2301864624023\\
53.7854957580566 15.2313146591187\\
53.8854942321777 15.2324438095093\\
53.9854927062988 15.2335720062256\\
54.0854949951172 15.2347002029419\\
54.1854934692383 15.2358283996582\\
54.2854957580566 15.2369565963745\\
54.3854942321777 15.2380847930908\\
54.4854927062988 15.2392139434814\\
54.5854949951172 15.2403430938721\\
54.6854934692383 15.241473197937\\
54.7854957580566 15.2426128387451\\
54.8854942321777 15.2437982559204\\
54.9854927062988 15.2452363967896\\
55.0854949951172 15.2478904724121\\
55.1854934692383 15.2539110183716\\
55.2854957580566 15.2606420516968\\
55.3854942321777 15.2597846984863\\
55.4854927062988 15.2512769699097\\
55.5854949951172 15.2546529769897\\
55.6854934692383 15.253002166748\\
55.7854957580566 15.2549076080322\\
55.8854942321777 15.2553720474243\\
55.9854927062988 15.2569379806519\\
56.0854949951172 15.2577753067017\\
56.1854934692383 15.2590970993042\\
56.2854957580566 15.2600975036621\\
56.3854942321777 15.2613115310669\\
56.4854927062988 15.2623834609985\\
56.5854949951172 15.2635507583618\\
56.6854934692383 15.2646541595459\\
56.7854957580566 15.2658004760742\\
56.8854942321777 15.2669172286987\\
56.9854927062988 15.2680540084839\\
57.0854949951172 15.2691783905029\\
57.1854934692383 15.2703104019165\\
57.2854957580566 15.2714376449585\\
57.3854942321777 15.2725687026978\\
57.4854927062988 15.2736968994141\\
57.5854949951172 15.274827003479\\
57.6854934692383 15.2759561538696\\
57.7854957580566 15.2770862579346\\
57.8854942321777 15.2784557342529\\
57.9854927062988 15.2796449661255\\
58.0854949951172 15.2807722091675\\
58.1854934692383 15.2819042205811\\
58.2854957580566 15.283034324646\\
58.3854942321777 15.2841653823853\\
58.4854927062988 15.2852964401245\\
58.5854949951172 15.2864274978638\\
58.6854934692383 15.2875595092773\\
58.7854957580566 15.2886915206909\\
58.8854942321777 15.2898244857788\\
58.9854927062988 15.2909574508667\\
59.0854949951172 15.2920904159546\\
59.1854934692383 15.2932243347168\\
59.2854957580566 15.2943592071533\\
59.3854942321777 15.2953977584839\\
59.4854927062988 15.2962646484375\\
59.5854949951172 15.2974615097046\\
59.6854934692383 15.2985553741455\\
59.7854957580566 15.2996864318848\\
59.8854942321777 15.300802230835\\
59.9427452087402 15.3018217086792\\
60 15.3025913238525\\
60.0629768371582 15.3032102584839\\
60.1322555541992 15.303918838501\\
60.2084579467773 15.3046903610229\\
60.2922821044922 15.3055410385132\\
60.3844909667969 15.3064756393433\\
60.484489440918 15.3075647354126\\
60.5844879150391 15.3087959289551\\
60.6844902038574 15.3099842071533\\
60.7844886779785 15.3111343383789\\
60.8844909667969 15.3123006820679\\
60.984489440918 15.3134851455688\\
61.0844879150391 15.3147001266479\\
61.1844902038574 15.3159656524658\\
61.2844886779785 15.3222026824951\\
61.3734016418457 15.326319694519\\
61.444522857666 15.3252925872803\\
61.5227546691895 15.3197784423828\\
61.6088104248047 15.3204212188721\\
61.7034721374512 15.321081161499\\
61.8034706115723 15.3219861984253\\
61.9034729003906 15.3230514526367\\
62.0034713745117 15.3241662979126\\
62.1034736633301 15.3252716064453\\
62.2034721374512 15.3263835906982\\
62.3034706115723 15.3274984359741\\
62.4034729003906 15.3286180496216\\
62.5034713745117 15.3297338485718\\
62.6034736633301 15.3308553695679\\
62.7034721374512 15.3319787979126\\
62.8034706115723 15.333104133606\\
62.9034729003906 15.3342332839966\\
63.0034713745117 15.3353614807129\\
63.1034736633301 15.3364896774292\\
63.2034721374512 15.3376178741455\\
63.3034706115723 15.3387470245361\\
63.4034729003906 15.3398752212524\\
63.5034713745117 15.3410034179688\\
63.6034736633301 15.3421316146851\\
63.7034721374512 15.3432598114014\\
63.8034706115723 15.3443880081177\\
63.9034729003906 15.3455171585083\\
64.0034713745117 15.3466453552246\\
64.1034698486328 15.3477735519409\\
64.2034759521484 15.3489017486572\\
64.3034744262695 15.3500299453735\\
64.4034729003906 15.3511581420898\\
64.5034713745117 15.3522863388062\\
64.6034698486328 15.3534145355225\\
64.7034759521484 15.3545427322388\\
64.8034744262695 15.3556709289551\\
64.9034729003906 15.3567991256714\\
65.0034713745117 15.3579273223877\\
65.1034698486328 15.359055519104\\
65.2034759521484 15.3601837158203\\
65.3034744262695 15.3613119125366\\
65.4034729003906 15.3624401092529\\
65.5034713745117 15.3635683059692\\
65.6034698486328 15.3646965026855\\
65.7034759521484 15.3658313751221\\
65.8034744262695 15.3669424057007\\
65.9034729003906 15.3680763244629\\
66.0034713745117 15.3692026138306\\
66.1034698486328 15.3703317642212\\
66.2034759521484 15.3714599609375\\
66.3034744262695 15.3725881576538\\
66.4034729003906 15.3737163543701\\
66.5034713745117 15.3748445510864\\
66.6034698486328 15.3759727478027\\
66.7034759521484 15.377100944519\\
66.8034744262695 15.3782300949097\\
66.9034729003906 15.3793601989746\\
67.0034713745117 15.3804950714111\\
67.1034698486328 15.3816547393799\\
67.2034759521484 15.3829555511475\\
67.3034744262695 15.3849878311157\\
67.4034729003906 15.389669418335\\
67.5034713745117 15.3969898223877\\
67.6034698486328 15.3983030319214\\
67.7034759521484 15.3935499191284\\
67.8034744262695 15.3911628723145\\
67.9034729003906 15.3921585083008\\
68.0034713745117 15.3920278549194\\
68.1034698486328 15.3937759399414\\
68.2034759521484 15.3944845199585\\
68.3034744262695 15.3958921432495\\
68.4034729003906 15.3968353271484\\
68.5034713745117 15.3980865478516\\
68.6034698486328 15.399133682251\\
68.7034759521484 15.4003171920776\\
68.8034744262695 15.4014101028442\\
68.9034729003906 15.4025630950928\\
69.0034713745117 15.40367603302\\
69.1034698486328 15.4048156738281\\
69.2034759521484 15.4059371948242\\
69.3034744262695 15.4070711135864\\
69.4034729003906 15.4081974029541\\
69.5034713745117 15.4093284606934\\
69.6034698486328 15.4104566574097\\
69.7034759521484 15.4115867614746\\
69.8034744262695 15.4127159118652\\
69.9034729003906 15.4138460159302\\
70.0034713745117 15.4149751663208\\
70.1034698486328 15.416223526001\\
70.2034759521484 15.4175586700439\\
70.3034744262695 15.4186515808105\\
70.4034729003906 15.4197959899902\\
70.5034713745117 15.4209213256836\\
70.6034698486328 15.4220533370972\\
70.7034759521484 15.4231843948364\\
70.8034744262695 15.4243154525757\\
70.9034729003906 15.4254474639893\\
71.0034713745117 15.4265794754028\\
71.1034698486328 15.4277124404907\\
71.2034759521484 15.4288444519043\\
71.3034744262695 15.4299783706665\\
71.4034729003906 15.4311122894287\\
71.5034713745117 15.4322462081909\\
71.6034698486328 15.4333648681641\\
71.7034759521484 15.4341907501221\\
71.8034744262695 15.4353456497192\\
71.9034729003906 15.4364528656006\\
72.0034713745117 15.4375791549683\\
72.1034698486328 15.4386968612671\\
72.2034759521484 15.4398174285889\\
72.3034744262695 15.440936088562\\
72.4034729003906 15.4420537948608\\
72.5034713745117 15.4431705474854\\
72.6034698486328 15.4442853927612\\
72.7034759521484 15.4454326629639\\
72.8034744262695 15.4466590881348\\
72.9034729003906 15.4478664398193\\
73.0034713745117 15.4490242004395\\
73.1034698486328 15.4501829147339\\
73.2034759521484 15.4513597488403\\
73.3034744262695 15.4525623321533\\
73.4034729003906 15.4538068771362\\
73.5034713745117 15.4575824737549\\
73.5992889404297 15.4636659622192\\
73.6759490966797 15.4660387039185\\
73.7424468994141 15.4580221176147\\
73.8155975341797 15.4584217071533\\
73.8960571289063 15.4589920043945\\
73.9845733642578 15.4596843719482\\
74.0819320678711 15.460560798645\\
74.1819305419922 15.4616231918335\\
74.2819290161133 15.4627304077148\\
74.3819351196289 15.4638385772705\\
74.48193359375 15.4649505615234\\
74.5819320678711 15.4660673141479\\
74.6819305419922 15.4671869277954\\
74.7819290161133 15.4683036804199\\
74.8819351196289 15.469425201416\\
74.9409637451172 15.4705533981323\\
75 15.4712133407593\\
};
\addlegendentry{2"};

\addplot [
color=black,
dash pattern=on 1pt off 3pt on 3pt off 3pt
]
table[row sep=crcr]{
0 14.7006950378418\\
9.99999971718069e-10 14.7006950378418\\
2.49999998480632e-09 14.7006950378418\\
4.74999994892755e-09 14.7006950378418\\
8.1250002281763e-09 14.7006950378418\\
1.31874999809156e-08 14.7006950378418\\
2.0781250498203e-08 14.7006950378418\\
3.21718758300449e-08 14.7006950378418\\
4.92578138278077e-08 14.7006950378418\\
7.48867208244519e-08 14.7006950378418\\
1.13330081319418e-07 14.7006950378418\\
1.70995122061868e-07 14.7006950378418\\
2.57492672517401e-07 14.7006950378418\\
3.8723902662241e-07 14.7006950378418\\
5.81858500936505e-07 14.7006950378418\\
8.73787769251066e-07 14.7006950378418\\
1.31168167172291e-06 14.7006950378418\\
1.9685226106958e-06 14.7006950378418\\
2.95378367809462e-06 14.7006959915161\\
4.43167573394021e-06 14.7006959915161\\
6.64851359033491e-06 14.7006959915161\\
9.97376992017962e-06 14.7006959915161\\
1.49616553244414e-05 14.7006969451904\\
2.24434825213393e-05 14.7006978988647\\
3.36662233166862e-05 14.7006988525391\\
5.05003372381907e-05 14.7006998062134\\
7.57515081204474e-05 14.700701713562\\
0.000113628258986864 14.7007036209106\\
0.000170443381648511 14.7007055282593\\
0.000255666091106832 14.7007074356079\\
0.000383500126190484 14.7007083892822\\
0.000575251178815961 14.7007074356079\\
0.000862877757754177 14.7007036209106\\
0.00129431765526533 14.7007007598877\\
0.00194147753063589 14.7006969451904\\
0.00291221728548408 14.7006931304932\\
0.00436832662671804 14.7006912231445\\
0.00597004732117057 14.7006931304932\\
0.00773194013163447 14.7006950378418\\
0.009670021943748 14.7006959915161\\
0.0118019115179777 14.7006959915161\\
0.0141469910740852 14.7006959915161\\
0.0167265776544809 14.7006959915161\\
0.0195641238242388 14.7006959915161\\
0.0226854234933853 14.7006959915161\\
0.0261188540607691 14.7006959915161\\
0.0298956278711557 14.7006959915161\\
0.0340500771999359 14.7006950378418\\
0.0386199727654457 14.7006950378418\\
0.0436468608677387 14.7006950378418\\
0.0491764321923256 14.7006950378418\\
0.0552589632570744 14.7006950378418\\
0.0619497485458851 14.7006959915161\\
0.0693096145987511 14.7006959915161\\
0.0774054601788521 14.7006912231445\\
0.0863108932971954 14.7006988525391\\
0.0961068719625473 14.7006959915161\\
0.106882445514202 14.7006959915161\\
0.118735581636429 14.7006959915161\\
0.13177402317524 14.7006959915161\\
0.146116316318512 14.7006959915161\\
0.161892831325531 14.7006959915161\\
0.179247006773949 14.7006959915161\\
0.198336601257324 14.7006959915161\\
0.21933513879776 14.7006959915161\\
0.242433547973633 14.7006950378418\\
0.267841786146164 14.7006959915161\\
0.29579085111618 14.7006959915161\\
0.32653483748436 14.7006959915161\\
0.360353201627731 14.7006959915161\\
0.397553414106369 14.7006959915161\\
0.438473641872406 14.7006959915161\\
0.483485877513886 14.7006959915161\\
0.532999336719513 14.7006959915161\\
0.587464153766632 14.7006959915161\\
0.647375464439392 14.7006959915161\\
0.71327793598175 14.7006959915161\\
0.7857705950737 14.7006959915161\\
0.865512549877167 14.7006959915161\\
0.953228712081909 14.7006959915161\\
1.04971647262573 14.7006950378418\\
1.14971649646759 14.7006950378418\\
1.24971640110016 14.7006950378418\\
1.34971642494202 14.7006950378418\\
1.44971644878387 14.7006950378418\\
1.54971647262573 14.7006950378418\\
1.64971649646759 14.7006950378418\\
1.74971640110016 14.7006950378418\\
1.84971642494202 14.7006950378418\\
1.94971644878387 14.7006950378418\\
2.04971647262573 14.7006950378418\\
2.1497163772583 14.7006950378418\\
2.24971652030945 14.7006950378418\\
2.34971642494202 14.7006950378418\\
2.44971656799316 14.7006950378418\\
2.54971647262573 14.7006950378418\\
2.6497163772583 14.7006950378418\\
2.74971652030945 14.7006950378418\\
2.84971642494202 14.7006950378418\\
2.94971656799316 14.7006950378418\\
3.04971647262573 14.7006950378418\\
3.1497163772583 14.7006950378418\\
3.24971652030945 14.7006950378418\\
3.34971642494202 14.7006950378418\\
3.44971656799316 14.7006950378418\\
3.54971647262573 14.7006950378418\\
3.6497163772583 14.7006950378418\\
3.74971652030945 14.7006950378418\\
3.84971642494202 14.7006950378418\\
3.94971656799316 14.7006950378418\\
4.04971647262573 14.7006950378418\\
4.1497163772583 14.7006950378418\\
4.24971628189087 14.7006950378418\\
4.3497166633606 14.7006950378418\\
4.44971656799316 14.7006950378418\\
4.54971647262573 14.7006950378418\\
4.6497163772583 14.7006950378418\\
4.74971628189087 14.7006950378418\\
4.8497166633606 14.7006950378418\\
4.94971656799316 14.7006950378418\\
5.04971647262573 14.7006950378418\\
5.1497163772583 14.7006950378418\\
5.24971628189087 14.7006950378418\\
5.3497166633606 14.7006950378418\\
5.44971656799316 14.7006950378418\\
5.54971647262573 14.7006950378418\\
5.6497163772583 14.7006950378418\\
5.74971628189087 14.7006950378418\\
5.8497166633606 14.7006950378418\\
5.94971656799316 14.7006950378418\\
6.04971647262573 14.7006950378418\\
6.1497163772583 14.7006940841675\\
6.24971628189087 14.7006921768188\\
6.3497166633606 14.7006912231445\\
6.44971656799316 14.7006940841675\\
6.54971647262573 14.7006969451904\\
6.6497163772583 14.7006959915161\\
6.74971628189087 14.7006978988647\\
6.8497166633606 14.7006940841675\\
6.94971656799316 14.7006959915161\\
7.04971647262573 14.7006950378418\\
7.1497163772583 14.7006959915161\\
7.24971628189087 14.7006950378418\\
7.3497166633606 14.7006950378418\\
7.44971656799316 14.7006950378418\\
7.54971647262573 14.7006950378418\\
7.6497163772583 14.7006950378418\\
7.74971628189087 14.7006950378418\\
7.8497166633606 14.7006950378418\\
7.94971656799316 14.7006950378418\\
8.04971599578857 14.7006950378418\\
8.1497163772583 14.7006950378418\\
8.24971675872803 14.7006950378418\\
8.34971618652344 14.7006950378418\\
8.44971656799316 14.7006950378418\\
8.54971599578857 14.7006950378418\\
8.6497163772583 14.7006950378418\\
8.74971675872803 14.7006950378418\\
8.84971618652344 14.7006950378418\\
8.94971656799316 14.7006950378418\\
9.04971599578857 14.7006950378418\\
9.1497163772583 14.7006950378418\\
9.24971675872803 14.7006950378418\\
9.34971618652344 14.7006950378418\\
9.44971656799316 14.7006950378418\\
9.54971599578857 14.7006950378418\\
9.6497163772583 14.7006950378418\\
9.74971675872803 14.7006950378418\\
9.84971618652344 14.7006950378418\\
9.94971656799316 14.7006950378418\\
10.0497159957886 14.7006950378418\\
10.1497163772583 14.7006950378418\\
10.249716758728 14.7006950378418\\
10.3497161865234 14.7006950378418\\
10.4497165679932 14.7006950378418\\
10.5497159957886 14.7006950378418\\
10.6497163772583 14.7006950378418\\
10.749716758728 14.7006950378418\\
10.8497161865234 14.7006950378418\\
10.9497165679932 14.7006950378418\\
11.0497159957886 14.7006950378418\\
11.1497163772583 14.7006950378418\\
11.249716758728 14.7006950378418\\
11.3497161865234 14.7006950378418\\
11.4497165679932 14.7006950378418\\
11.5497159957886 14.7006950378418\\
11.6497163772583 14.7006959915161\\
11.749716758728 14.7006959915161\\
11.8497161865234 14.7006950378418\\
11.9497165679932 14.7006950378418\\
12.0497159957886 14.7006950378418\\
12.1497163772583 14.7006950378418\\
12.249716758728 14.7006950378418\\
12.3471450805664 14.7006950378418\\
12.4250917434692 14.7006950378418\\
12.4879579544067 14.7006959915161\\
12.557110786438 14.7006931304932\\
12.6331787109375 14.7006959915161\\
12.7168531417847 14.7006959915161\\
12.808895111084 14.7006959915161\\
12.9088954925537 14.7006959915161\\
13.0088949203491 14.7006959915161\\
13.1088953018188 14.7006969451904\\
13.2088947296143 14.7006969451904\\
13.308895111084 14.7006969451904\\
13.4088954925537 14.7006969451904\\
13.5088949203491 14.7006969451904\\
13.6088953018188 14.7006969451904\\
13.7088947296143 14.7006959915161\\
13.808895111084 14.7006959915161\\
13.9088954925537 14.7006959915161\\
14.0088949203491 14.7006959915161\\
14.1088953018188 14.7006959915161\\
14.2088947296143 14.7006959915161\\
14.308895111084 14.7006959915161\\
14.4088954925537 14.7006959915161\\
14.5088949203491 14.7006959915161\\
14.6088953018188 14.7006959915161\\
14.7088947296143 14.7006959915161\\
14.808895111084 14.7006959915161\\
14.9088954925537 14.7006959915161\\
15.0088949203491 14.7006959915161\\
15.1088953018188 14.7006959915161\\
15.2088947296143 14.7006959915161\\
15.308895111084 14.7006959915161\\
15.4088954925537 14.7006969451904\\
15.5088949203491 14.7006959915161\\
15.6088953018188 14.7006959915161\\
15.7088947296143 14.7006959915161\\
15.808895111084 14.7006959915161\\
15.9088954925537 14.7006959915161\\
16.0088958740234 14.7006959915161\\
16.1088943481445 14.7006959915161\\
16.2088947296143 14.7006959915161\\
16.308895111084 14.7006959915161\\
16.4088954925537 14.7006959915161\\
16.5088958740234 14.7006959915161\\
16.6088943481445 14.7006959915161\\
16.7088947296143 14.7006959915161\\
16.808895111084 14.7006959915161\\
16.9088954925537 14.7006959915161\\
17.0088958740234 14.7006959915161\\
17.1088943481445 14.7006959915161\\
17.2088947296143 14.7006959915161\\
17.308895111084 14.7006959915161\\
17.4088954925537 14.7006959915161\\
17.5088958740234 14.7006959915161\\
17.6088943481445 14.7006959915161\\
17.7088947296143 14.7006959915161\\
17.808895111084 14.7006969451904\\
17.9088954925537 14.7006988525391\\
18.0088958740234 14.7007055282593\\
18.1088943481445 14.700737953186\\
18.2088947296143 14.7009048461914\\
18.308895111084 14.7017889022827\\
18.4088954925537 14.705792427063\\
18.5088958740234 14.7166261672974\\
18.6088943481445 14.7299156188965\\
18.7088947296143 14.7327671051025\\
18.808895111084 14.7320785522461\\
18.9088954925537 14.7330684661865\\
19.0088958740234 14.7325553894043\\
19.1088943481445 14.7335662841797\\
19.2088947296143 14.7337579727173\\
19.308895111084 14.7345132827759\\
19.4088954925537 14.7348937988281\\
19.5088958740234 14.7355241775513\\
19.6088943481445 14.7359867095947\\
19.7088947296143 14.7365627288818\\
19.808895111084 14.7370634078979\\
19.9088954925537 14.7376136779785\\
20.0088958740234 14.7381315231323\\
20.1088943481445 14.7386722564697\\
20.2088947296143 14.7391967773438\\
20.308895111084 14.7397327423096\\
20.4088954925537 14.7402610778809\\
20.5088958740234 14.740795135498\\
20.6088943481445 14.7413263320923\\
20.7088947296143 14.7418594360352\\
20.808895111084 14.7423915863037\\
20.9088954925537 14.7429256439209\\
21.0088958740234 14.7434587478638\\
21.1088943481445 14.7440614700317\\
21.2088947296143 14.7448883056641\\
21.308895111084 14.7454996109009\\
21.4088954925537 14.7461566925049\\
21.5088958740234 14.7467947006226\\
21.6088943481445 14.7474365234375\\
21.7088947296143 14.7480764389038\\
21.808895111084 14.7487154006958\\
21.9088954925537 14.7493534088135\\
22.0088958740234 14.7499904632568\\
22.1088943481445 14.7506275177002\\
22.2088947296143 14.7512645721436\\
22.308895111084 14.7519016265869\\
22.4088954925537 14.752537727356\\
22.5088958740234 14.7531747817993\\
22.6088943481445 14.753794670105\\
22.7088947296143 14.7541408538818\\
22.808895111084 14.7546730041504\\
22.9088954925537 14.7552080154419\\
23.0088958740234 14.7557401657104\\
23.1088943481445 14.7562694549561\\
23.2088947296143 14.7567949295044\\
23.308895111084 14.7573156356812\\
23.4088954925537 14.7578315734863\\
23.5088958740234 14.7583417892456\\
23.6088943481445 14.758846282959\\
23.7088947296143 14.7593812942505\\
23.808895111084 14.760027885437\\
23.9088954925537 14.7606897354126\\
24.0088958740234 14.7613086700439\\
24.1088943481445 14.7619352340698\\
24.2088947296143 14.7625885009766\\
24.308895111084 14.7632789611816\\
24.4088954925537 14.7640314102173\\
24.5088958740234 14.7671670913696\\
24.6048831939697 14.7735557556152\\
24.681676864624 14.7769861221313\\
24.7464084625244 14.7684936523438\\
24.8176136016846 14.7686386108398\\
24.895938873291 14.7690258026123\\
24.9820976257324 14.7695426940918\\
25.0768718719482 14.7702665328979\\
25.176872253418 14.7711944580078\\
25.2768707275391 14.7722263336182\\
25.3768711090088 14.7732744216919\\
25.4768714904785 14.7743406295776\\
25.5768718719482 14.7754201889038\\
25.676872253418 14.7765102386475\\
25.7768707275391 14.7776050567627\\
25.8768711090088 14.7787103652954\\
25.9768714904785 14.7798185348511\\
26.0768718719482 14.780933380127\\
26.176872253418 14.7820472717285\\
26.2768707275391 14.7831611633301\\
26.3768711090088 14.7842750549316\\
26.4768714904785 14.7853899002075\\
26.5768718719482 14.7865047454834\\
26.676872253418 14.787618637085\\
26.7768707275391 14.7887334823608\\
26.8768711090088 14.7898483276367\\
26.9768714904785 14.7909631729126\\
27.0768718719482 14.7920780181885\\
27.176872253418 14.7931938171387\\
27.2768707275391 14.7943086624146\\
27.3768711090088 14.7954244613647\\
27.4768714904785 14.7965402603149\\
27.5768718719482 14.7976560592651\\
27.676872253418 14.7987718582153\\
27.7768707275391 14.7998876571655\\
27.8768711090088 14.80100440979\\
27.9768714904785 14.8021211624146\\
28.0768718719482 14.8032369613647\\
28.176872253418 14.8043537139893\\
28.2768707275391 14.8054780960083\\
28.3768711090088 14.8065824508667\\
28.4768714904785 14.8077096939087\\
28.5768718719482 14.8088302612305\\
28.676872253418 14.8099536895752\\
28.7768707275391 14.8110761642456\\
28.8768711090088 14.8121995925903\\
28.9768714904785 14.8133220672607\\
29.0768718719482 14.8144454956055\\
29.176872253418 14.8155679702759\\
29.2768707275391 14.8166913986206\\
29.3768711090088 14.8178148269653\\
29.4768714904785 14.8189373016357\\
29.5768718719482 14.8200607299805\\
29.676872253418 14.8211841583252\\
29.7768707275391 14.8223075866699\\
29.8768711090088 14.8234310150146\\
29.9768714904785 14.8245544433594\\
30.0768718719482 14.8256788253784\\
30.176872253418 14.8268051147461\\
30.2768707275391 14.8279371261597\\
30.3768711090088 14.8291063308716\\
30.4768714904785 14.8304738998413\\
30.5768718719482 14.8328371047974\\
30.676872253418 14.8383226394653\\
30.7768707275391 14.8454103469849\\
30.8768711090088 14.8438529968262\\
30.9768714904785 14.8406610488892\\
31.0768718719482 14.8382902145386\\
31.176872253418 14.8392992019653\\
31.2768707275391 14.8395013809204\\
31.3768711090088 14.8411102294922\\
31.4768714904785 14.8419094085693\\
31.5768718719482 14.8432512283325\\
31.676872253418 14.8442316055298\\
31.7768707275391 14.8454532623291\\
31.8768711090088 14.8465137481689\\
31.9768714904785 14.8476819992065\\
32.0768699645996 14.8487787246704\\
32.176872253418 14.8499231338501\\
32.2768707275391 14.8510360717773\\
32.3768730163574 14.8521709442139\\
32.4768714904785 14.8532905578613\\
32.5768699645996 14.854419708252\\
32.676872253418 14.8555431365967\\
32.7768707275391 14.8566703796387\\
32.8768730163574 14.857795715332\\
32.9768714904785 14.8589220046997\\
33.0768699645996 14.8600473403931\\
33.176872253418 14.8611736297607\\
33.2768707275391 14.8622999191284\\
33.3768730163574 14.8636255264282\\
33.4768714904785 14.8648624420166\\
33.5768699645996 14.8659763336182\\
33.676872253418 14.8671083450317\\
33.7768707275391 14.8682336807251\\
33.8768730163574 14.8693618774414\\
33.9768714904785 14.8704891204834\\
34.0768699645996 14.8716173171997\\
34.176872253418 14.872745513916\\
34.2768707275391 14.8738746643066\\
34.3768730163574 14.8750038146973\\
34.4768714904785 14.8761329650879\\
34.5768699645996 14.8772640228271\\
34.676872253418 14.8783941268921\\
34.7768707275391 14.8795261383057\\
34.8768730163574 14.880597114563\\
34.9768714904785 14.8814125061035\\
35.0768699645996 14.8826198577881\\
35.176872253418 14.8837060928345\\
35.2768707275391 14.8848352432251\\
35.3768730163574 14.885947227478\\
35.4768714904785 14.8870639801025\\
35.5768699645996 14.8881778717041\\
35.676872253418 14.8892917633057\\
35.7768707275391 14.8904037475586\\
35.8768730163574 14.8915138244629\\
35.9768714904785 14.8926858901978\\
36.0768699645996 14.8939428329468\\
36.176872253418 14.895133972168\\
36.2768707275391 14.8962869644165\\
36.3768730163574 14.8974504470825\\
36.4768714904785 14.8986320495605\\
36.5768699645996 14.8998432159424\\
36.676872253418 14.9011030197144\\
36.7768707275391 14.9067459106445\\
36.8673248291016 14.911337852478\\
36.9396858215332 14.9112911224365\\
37.0192794799805 14.9049701690674\\
37.1068305969238 14.9056205749512\\
37.2031402587891 14.9062757492065\\
37.3031425476074 14.9071912765503\\
37.4031410217285 14.9082527160645\\
37.5031433105469 14.909369468689\\
37.603141784668 14.910472869873\\
37.7031402587891 14.9115839004517\\
37.8031425476074 14.9126968383789\\
37.9031410217285 14.9138154983521\\
38.0031433105469 14.9149293899536\\
38.103141784668 14.9160499572754\\
38.2031402587891 14.9171714782715\\
38.3031425476074 14.9182968139648\\
38.4031410217285 14.9194240570068\\
38.5031433105469 14.9205503463745\\
38.603141784668 14.9216775894165\\
38.7031402587891 14.9228048324585\\
38.8031425476074 14.9239311218262\\
38.9031410217285 14.9250583648682\\
39.0031433105469 14.9261856079102\\
39.103141784668 14.9273128509521\\
39.2031402587891 14.9284391403198\\
39.3031425476074 14.9295663833618\\
39.4031410217285 14.9306936264038\\
39.5031433105469 14.9318199157715\\
39.603141784668 14.9329471588135\\
39.7031402587891 14.9340744018555\\
39.8031425476074 14.9352006912231\\
39.9031410217285 14.9363279342651\\
40.0031433105469 14.9374551773071\\
40.103141784668 14.9385814666748\\
40.2031402587891 14.9397087097168\\
40.3031425476074 14.9408359527588\\
40.4031410217285 14.9419622421265\\
40.5031433105469 14.9430894851685\\
40.603141784668 14.9442167282104\\
40.7031402587891 14.9453430175781\\
40.8031425476074 14.9464702606201\\
40.9031410217285 14.9476041793823\\
41.0031433105469 14.9487133026123\\
41.103141784668 14.9498453140259\\
41.2031402587891 14.9509706497192\\
41.3031425476074 14.9520988464355\\
41.4031410217285 14.9532260894775\\
41.5031433105469 14.9543533325195\\
41.603141784668 14.9554805755615\\
41.7031402587891 14.9566068649292\\
41.8031425476074 14.9577341079712\\
41.9031410217285 14.9588613510132\\
42.0031433105469 14.9599885940552\\
42.103141784668 14.9611158370972\\
42.2031402587891 14.9622440338135\\
42.3031425476074 14.9633712768555\\
42.4031410217285 14.9645004272461\\
42.5031433105469 14.9656343460083\\
42.603141784668 14.9667940139771\\
42.7031402587891 14.9680995941162\\
42.8031425476074 14.9701595306396\\
42.9031410217285 14.9749202728271\\
43.0031433105469 14.982232093811\\
43.103141784668 14.9832639694214\\
43.2031402587891 14.9786396026611\\
43.3031425476074 14.9762668609619\\
43.4031410217285 14.9772596359253\\
43.5031433105469 14.977162361145\\
43.603141784668 14.9788980484009\\
43.7031402587891 14.9796133041382\\
43.8031425476074 14.981014251709\\
43.9031410217285 14.9819593429565\\
44.0031433105469 14.983208656311\\
44.103141784668 14.9842557907104\\
44.2031402587891 14.9854373931885\\
44.3031425476074 14.9865293502808\\
44.4031410217285 14.987681388855\\
44.5031433105469 14.9887933731079\\
44.603141784668 14.989933013916\\
44.7031402587891 14.9910535812378\\
44.8031425476074 14.9921865463257\\
44.9031410217285 14.9933109283447\\
45.0031433105469 14.994441986084\\
45.103141784668 14.995569229126\\
45.2031402587891 14.9966983795166\\
45.3031425476074 14.9978265762329\\
45.4031410217285 14.9989557266235\\
45.5031433105469 15.0000839233398\\
45.603141784668 15.0013418197632\\
45.7031402587891 15.0026636123657\\
45.8031425476074 15.0037603378296\\
45.9031410217285 15.0049028396606\\
46.0031433105469 15.0060272216797\\
46.103141784668 15.0071592330933\\
46.2031402587891 15.0082883834839\\
46.3031425476074 15.0094194412231\\
46.4031410217285 15.0105495452881\\
46.5031433105469 15.0116815567017\\
46.603141784668 15.0128126144409\\
46.7031402587891 15.0139446258545\\
46.8031425476074 15.0150775909424\\
46.9031410217285 15.0162105560303\\
47.0031433105469 15.0173435211182\\
47.103141784668 15.018458366394\\
47.2031402587891 15.0192718505859\\
47.3031425476074 15.0204401016235\\
47.4031410217285 15.021541595459\\
47.5031433105469 15.022668838501\\
47.603141784668 15.0237846374512\\
47.7031402587891 15.0249042510986\\
47.8031425476074 15.0260210037231\\
47.9031410217285 15.0271368026733\\
48.0031433105469 15.0282516479492\\
48.103141784668 15.0293645858765\\
48.2031402587891 15.0305166244507\\
48.3031425476074 15.0317535400391\\
48.4031410217285 15.0329627990723\\
48.5031433105469 15.0341196060181\\
48.603141784668 15.0352802276611\\
48.7031402587891 15.0364580154419\\
48.8031425476074 15.0376615524292\\
48.9031410217285 15.0389089584351\\
49.0031433105469 15.0429925918579\\
49.0980262756348 15.0488300323486\\
49.1739387512207 15.0508756637573\\
49.2413482666016 15.0430574417114\\
49.315502166748 15.0434942245483\\
49.3970718383789 15.044059753418\\
49.4867973327637 15.044771194458\\
49.5854949951172 15.0456638336182\\
49.6854934692383 15.0467462539673\\
49.7854957580566 15.0478515625\\
49.8854942321777 15.04896068573\\
49.9854927062988 15.0500717163086\\
50.0854949951172 15.0511884689331\\
50.1854934692383 15.0523061752319\\
50.2854957580566 15.0534229278564\\
50.3854942321777 15.0545444488525\\
50.4854927062988 15.0556678771973\\
50.5854949951172 15.0567951202393\\
50.6854934692383 15.0579233169556\\
50.7854957580566 15.0590505599976\\
50.8854942321777 15.0601787567139\\
50.9854927062988 15.0613059997559\\
51.0854949951172 15.0624341964722\\
51.1854934692383 15.0635623931885\\
51.2854957580566 15.0646896362305\\
51.3854942321777 15.0658178329468\\
51.4854927062988 15.0669450759888\\
51.5854949951172 15.0680732727051\\
51.6854934692383 15.0692014694214\\
51.7854957580566 15.0703287124634\\
51.8854942321777 15.0714569091797\\
51.9854927062988 15.0725841522217\\
52.0854949951172 15.073712348938\\
52.1854934692383 15.07483959198\\
52.2854957580566 15.0759677886963\\
52.3854942321777 15.0770950317383\\
52.4854927062988 15.0782232284546\\
52.5854949951172 15.0793504714966\\
52.6854934692383 15.0804786682129\\
52.7854957580566 15.0816059112549\\
52.8854942321777 15.0827341079712\\
52.9854927062988 15.0838613510132\\
53.0854949951172 15.0849895477295\\
53.1854934692383 15.0861167907715\\
53.2854957580566 15.087251663208\\
53.3854942321777 15.0883617401123\\
53.4854927062988 15.0894947052002\\
53.5854949951172 15.0906209945679\\
53.6854934692383 15.0917491912842\\
53.7854957580566 15.0928764343262\\
53.8854942321777 15.0940046310425\\
53.9854927062988 15.0951328277588\\
54.0854949951172 15.0962610244751\\
54.1854934692383 15.0973882675171\\
54.2854957580566 15.0985164642334\\
54.3854942321777 15.0996446609497\\
54.4854927062988 15.100772857666\\
54.5854949951172 15.1019010543823\\
54.6854934692383 15.1030311584473\\
54.7854957580566 15.1041707992554\\
54.8854942321777 15.1053552627563\\
54.9854927062988 15.1067934036255\\
55.0854949951172 15.109447479248\\
55.1854934692383 15.1154670715332\\
55.2854957580566 15.1221971511841\\
55.3854942321777 15.1213397979736\\
55.4854927062988 15.112832069397\\
55.5854949951172 15.1162080764771\\
55.6854934692383 15.114556312561\\
55.7854957580566 15.1164617538452\\
55.8854942321777 15.116925239563\\
55.9854927062988 15.1184911727905\\
56.0854949951172 15.1193284988403\\
56.1854934692383 15.1206493377686\\
56.2854957580566 15.1216497421265\\
56.3854942321777 15.1228637695313\\
56.4854927062988 15.1239347457886\\
56.5854949951172 15.1251010894775\\
56.6854934692383 15.1262044906616\\
56.7854957580566 15.1273508071899\\
56.8854942321777 15.1284675598145\\
56.9854927062988 15.1296033859253\\
57.0854949951172 15.1307277679443\\
57.1854934692383 15.1318597793579\\
57.2854957580566 15.1329860687256\\
57.3854942321777 15.1341171264648\\
57.4854927062988 15.1352453231812\\
57.5854949951172 15.1363754272461\\
57.6854934692383 15.1375036239624\\
57.7854957580566 15.1386337280273\\
57.8854942321777 15.1400032043457\\
57.9854927062988 15.1411914825439\\
58.0854949951172 15.1423187255859\\
58.1854934692383 15.1434497833252\\
58.2854957580566 15.1445798873901\\
58.3854942321777 15.1457109451294\\
58.4854927062988 15.1468410491943\\
58.5854949951172 15.1479721069336\\
58.6854934692383 15.1491041183472\\
58.7854957580566 15.1502361297607\\
58.8854942321777 15.1513681411743\\
58.9854927062988 15.1525011062622\\
59.0854949951172 15.1536340713501\\
59.1854934692383 15.1547679901123\\
59.2854957580566 15.1559019088745\\
59.3854942321777 15.1569404602051\\
59.4854927062988 15.1578063964844\\
59.5854949951172 15.1590032577515\\
59.6854934692383 15.1600971221924\\
59.7854957580566 15.1612272262573\\
59.8854942321777 15.1623430252075\\
59.9427452087402 15.1633625030518\\
60 15.1641311645508\\
60.0629768371582 15.1647500991821\\
60.1322555541992 15.1654586791992\\
60.2084579467773 15.1662302017212\\
60.2922821044922 15.1670808792114\\
60.3844909667969 15.1680145263672\\
60.484489440918 15.1691036224365\\
60.5844879150391 15.170334815979\\
60.6844902038574 15.1715230941772\\
60.7844886779785 15.1726722717285\\
60.8844909667969 15.1738395690918\\
60.984489440918 15.1750230789185\\
61.0844879150391 15.1762380599976\\
61.1844902038574 15.1775035858154\\
61.2844886779785 15.1837396621704\\
61.3734016418457 15.1878566741943\\
61.444522857666 15.1868286132813\\
61.5227546691895 15.1813154220581\\
61.6088104248047 15.181957244873\\
61.7034721374512 15.1826171875\\
61.8034706115723 15.1835222244263\\
61.9034729003906 15.1845874786377\\
62.0034713745117 15.1857013702393\\
62.1034736633301 15.186806678772\\
62.2034721374512 15.1879186630249\\
62.3034706115723 15.1890325546265\\
62.4034729003906 15.1901531219482\\
62.5034713745117 15.1912689208984\\
62.6034736633301 15.1923894882202\\
62.7034721374512 15.1935129165649\\
62.8034706115723 15.194637298584\\
62.9034729003906 15.1957664489746\\
63.0034713745117 15.1968946456909\\
63.1034736633301 15.1980228424072\\
63.2034721374512 15.1991510391235\\
63.3034706115723 15.2002792358398\\
63.4034729003906 15.2014074325562\\
63.5034713745117 15.2025356292725\\
63.6034736633301 15.2036638259888\\
63.7034721374512 15.2047920227051\\
63.8034706115723 15.2059202194214\\
63.9034729003906 15.2070484161377\\
64.0034713745117 15.208176612854\\
64.1034698486328 15.209303855896\\
64.2034759521484 15.2104320526123\\
64.3034744262695 15.2115602493286\\
64.4034729003906 15.2126884460449\\
64.5034713745117 15.2138166427612\\
64.6034698486328 15.2149448394775\\
64.7034759521484 15.2160730361938\\
64.8034744262695 15.2172002792358\\
64.9034729003906 15.2183284759521\\
65.0034713745117 15.2194566726685\\
65.1034698486328 15.2205848693848\\
65.2034759521484 15.2217130661011\\
65.3034744262695 15.2228412628174\\
65.4034729003906 15.2239685058594\\
65.5034713745117 15.2250967025757\\
65.6034698486328 15.226224899292\\
65.7034759521484 15.2273597717285\\
65.8034744262695 15.2284708023071\\
65.9034729003906 15.229603767395\\
66.0034713745117 15.2307300567627\\
66.1034698486328 15.2318592071533\\
66.2034759521484 15.2329864501953\\
66.3034744262695 15.2341146469116\\
66.4034729003906 15.2352428436279\\
66.5034713745117 15.2363710403442\\
66.6034698486328 15.2374992370605\\
66.7034759521484 15.2386274337769\\
66.8034744262695 15.2397565841675\\
66.9034729003906 15.2408857345581\\
67.0034713745117 15.2420206069946\\
67.1034698486328 15.2431812286377\\
67.2034759521484 15.244481086731\\
67.3034744262695 15.2465133666992\\
67.4034729003906 15.2511949539185\\
67.5034713745117 15.2585144042969\\
67.6034698486328 15.2598285675049\\
67.7034759521484 15.2550745010376\\
67.8034744262695 15.2526874542236\\
67.9034729003906 15.2536821365356\\
68.0034713745117 15.2535524368286\\
68.1034698486328 15.2553005218506\\
68.2034759521484 15.2560081481934\\
68.3034744262695 15.2574157714844\\
68.4034729003906 15.2583589553833\\
68.5034713745117 15.2596101760864\\
68.6034698486328 15.2606573104858\\
68.7034759521484 15.2618408203125\\
68.8034744262695 15.2629327774048\\
68.9034729003906 15.2640857696533\\
69.0034713745117 15.2651987075806\\
69.1034698486328 15.2663383483887\\
69.2034759521484 15.2674598693848\\
69.3034744262695 15.268593788147\\
69.4034729003906 15.2697191238403\\
69.5034713745117 15.2708511352539\\
69.6034698486328 15.2719783782959\\
69.7034759521484 15.2731084823608\\
69.8034744262695 15.2742376327515\\
69.9034729003906 15.2753677368164\\
70.0034713745117 15.276496887207\\
70.1034698486328 15.2777452468872\\
70.2034759521484 15.2790794372559\\
70.3034744262695 15.2801723480225\\
70.4034729003906 15.2813167572021\\
70.5034713745117 15.2824420928955\\
70.6034698486328 15.2835741043091\\
70.7034759521484 15.2847051620483\\
70.8034744262695 15.2858362197876\\
70.9034729003906 15.2869682312012\\
71.0034713745117 15.2881002426147\\
71.1034698486328 15.2892322540283\\
71.2034759521484 15.2903652191162\\
71.3034744262695 15.2914981842041\\
71.4034729003906 15.2926321029663\\
71.5034713745117 15.2937660217285\\
71.6034698486328 15.2948846817017\\
71.7034759521484 15.2957105636597\\
71.8034744262695 15.2968645095825\\
71.9034729003906 15.2979717254639\\
72.0034713745117 15.2990980148315\\
72.1034698486328 15.3002166748047\\
72.2034759521484 15.3013362884521\\
72.3034744262695 15.3024549484253\\
72.4034729003906 15.3035726547241\\
72.5034713745117 15.3046884536743\\
72.6034698486328 15.3058042526245\\
72.7034759521484 15.3069515228271\\
72.8034744262695 15.308177947998\\
72.9034729003906 15.3093843460083\\
73.0034713745117 15.3105421066284\\
73.1034698486328 15.3117008209229\\
73.2034759521484 15.3128776550293\\
73.3034744262695 15.3140802383423\\
73.4034729003906 15.3153247833252\\
73.5034713745117 15.3191003799438\\
73.5992889404297 15.3251829147339\\
73.6759490966797 15.3275556564331\\
73.7424468994141 15.3195400238037\\
73.8155975341797 15.319938659668\\
73.8960571289063 15.3205089569092\\
73.9845733642578 15.3212013244629\\
74.0819320678711 15.3220777511597\\
74.1819305419922 15.3231401443481\\
74.2819290161133 15.3242473602295\\
74.3819351196289 15.3253555297852\\
74.48193359375 15.3264675140381\\
74.5819320678711 15.3275842666626\\
74.6819305419922 15.3287038803101\\
74.7819290161133 15.3298196792603\\
74.8819351196289 15.3309421539307\\
74.9409637451172 15.3320693969727\\
75 15.3327293395996\\
};
\addlegendentry{6 "};

\addplot [
color=black,
dashed
]
table[row sep=crcr]{
0 14.700608253479\\
9.99999971718069e-10 14.700608253479\\
2.49999998480632e-09 14.700608253479\\
4.74999994892755e-09 14.700608253479\\
8.1250002281763e-09 14.700608253479\\
1.31874999809156e-08 14.700608253479\\
2.0781250498203e-08 14.700608253479\\
3.21718758300449e-08 14.700608253479\\
4.92578138278077e-08 14.700608253479\\
7.48867208244519e-08 14.700608253479\\
1.13330081319418e-07 14.700608253479\\
1.70995122061868e-07 14.700608253479\\
2.57492672517401e-07 14.700608253479\\
3.8723902662241e-07 14.700608253479\\
5.81858500936505e-07 14.700608253479\\
8.73787769251066e-07 14.700608253479\\
1.31168167172291e-06 14.700608253479\\
1.9685226106958e-06 14.700608253479\\
2.95378367809462e-06 14.700608253479\\
4.43167573394021e-06 14.700608253479\\
6.64851359033491e-06 14.7006092071533\\
9.97376992017962e-06 14.7006092071533\\
1.49616553244414e-05 14.7006092071533\\
2.24434825213393e-05 14.7006101608276\\
3.36662233166862e-05 14.700611114502\\
5.05003372381907e-05 14.7006120681763\\
7.57515081204474e-05 14.7006130218506\\
0.000113628258986864 14.7006149291992\\
0.000170443381648511 14.7006158828735\\
0.000255666091106832 14.7006177902222\\
0.000383500126190484 14.7006187438965\\
0.000575251178815961 14.7006187438965\\
0.000862877757754177 14.7006168365479\\
0.00129431765526533 14.7006139755249\\
0.00194147753063589 14.7006101608276\\
0.00291221728548408 14.7006063461304\\
0.00436832662671804 14.7006044387817\\
0.00597004732117057 14.7006063461304\\
0.00773194013163447 14.700608253479\\
0.009670021943748 14.7006092071533\\
0.0118019115179777 14.7006092071533\\
0.0141469910740852 14.7006092071533\\
0.0167265776544809 14.700608253479\\
0.0195641238242388 14.700608253479\\
0.0226854234933853 14.700608253479\\
0.0261188540607691 14.700608253479\\
0.0298956278711557 14.700608253479\\
0.0340500771999359 14.700608253479\\
0.0386199727654457 14.700608253479\\
0.0436468608677387 14.700608253479\\
0.0491764321923256 14.700608253479\\
0.0552589632570744 14.700608253479\\
0.0619497485458851 14.700608253479\\
0.0693096145987511 14.700608253479\\
0.0774054601788521 14.7006044387817\\
0.0863108932971954 14.7006120681763\\
0.0961068719625473 14.7006092071533\\
0.106882445514202 14.700608253479\\
0.118735581636429 14.700608253479\\
0.13177402317524 14.700608253479\\
0.146116316318512 14.7006092071533\\
0.161892831325531 14.7006092071533\\
0.179247006773949 14.7006092071533\\
0.198336601257324 14.7006092071533\\
0.21933513879776 14.7006092071533\\
0.242433547973633 14.700608253479\\
0.267841786146164 14.7006092071533\\
0.29579085111618 14.7006092071533\\
0.32653483748436 14.7006092071533\\
0.360353201627731 14.7006092071533\\
0.397553414106369 14.7006092071533\\
0.438473641872406 14.7006092071533\\
0.483485877513886 14.7006092071533\\
0.532999336719513 14.7006092071533\\
0.587464153766632 14.7006092071533\\
0.647375464439392 14.7006092071533\\
0.71327793598175 14.7006092071533\\
0.7857705950737 14.7006092071533\\
0.865512549877167 14.7006092071533\\
0.953228712081909 14.7006092071533\\
1.04971647262573 14.700608253479\\
1.14971649646759 14.700608253479\\
1.24971640110016 14.700608253479\\
1.34971642494202 14.700608253479\\
1.44971644878387 14.700608253479\\
1.54971647262573 14.700608253479\\
1.64971649646759 14.700608253479\\
1.74971640110016 14.700608253479\\
1.84971642494202 14.700608253479\\
1.94971644878387 14.700608253479\\
2.04971647262573 14.700608253479\\
2.1497163772583 14.700608253479\\
2.24971652030945 14.700608253479\\
2.34971642494202 14.700608253479\\
2.44971656799316 14.700608253479\\
2.54971647262573 14.700608253479\\
2.6497163772583 14.700608253479\\
2.74971652030945 14.700608253479\\
2.84971642494202 14.700608253479\\
2.94971656799316 14.700608253479\\
3.04971647262573 14.700608253479\\
3.1497163772583 14.700608253479\\
3.24971652030945 14.700608253479\\
3.34971642494202 14.700608253479\\
3.44971656799316 14.700608253479\\
3.54971647262573 14.700608253479\\
3.6497163772583 14.700608253479\\
3.74971652030945 14.700608253479\\
3.84971642494202 14.700608253479\\
3.94971656799316 14.700608253479\\
4.04971647262573 14.700608253479\\
4.1497163772583 14.700608253479\\
4.24971628189087 14.700608253479\\
4.3497166633606 14.700608253479\\
4.44971656799316 14.700608253479\\
4.54971647262573 14.700608253479\\
4.6497163772583 14.700608253479\\
4.74971628189087 14.700608253479\\
4.8497166633606 14.700608253479\\
4.94971656799316 14.700608253479\\
5.04971647262573 14.700608253479\\
5.1497163772583 14.700608253479\\
5.24971628189087 14.700608253479\\
5.3497166633606 14.700608253479\\
5.44971656799316 14.700608253479\\
5.54971647262573 14.700608253479\\
5.6497163772583 14.700608253479\\
5.74971628189087 14.700608253479\\
5.8497166633606 14.700608253479\\
5.94971656799316 14.700608253479\\
6.04971647262573 14.700608253479\\
6.1497163772583 14.7006072998047\\
6.24971628189087 14.7006063461304\\
6.3497166633606 14.7006044387817\\
6.44971656799316 14.7006072998047\\
6.54971647262573 14.7006101608276\\
6.6497163772583 14.7006092071533\\
6.74971628189087 14.700611114502\\
6.8497166633606 14.7006072998047\\
6.94971656799316 14.7006092071533\\
7.04971647262573 14.700608253479\\
7.1497163772583 14.700608253479\\
7.24971628189087 14.700608253479\\
7.3497166633606 14.700608253479\\
7.44971656799316 14.700608253479\\
7.54971647262573 14.700608253479\\
7.6497163772583 14.700608253479\\
7.74971628189087 14.700608253479\\
7.8497166633606 14.700608253479\\
7.94971656799316 14.700608253479\\
8.04971599578857 14.700608253479\\
8.1497163772583 14.700608253479\\
8.24971675872803 14.700608253479\\
8.34971618652344 14.700608253479\\
8.44971656799316 14.700608253479\\
8.54971599578857 14.700608253479\\
8.6497163772583 14.700608253479\\
8.74971675872803 14.700608253479\\
8.84971618652344 14.700608253479\\
8.94971656799316 14.700608253479\\
9.04971599578857 14.700608253479\\
9.1497163772583 14.700608253479\\
9.24971675872803 14.700608253479\\
9.34971618652344 14.700608253479\\
9.44971656799316 14.700608253479\\
9.54971599578857 14.700608253479\\
9.6497163772583 14.700608253479\\
9.74971675872803 14.700608253479\\
9.84971618652344 14.700608253479\\
9.94971656799316 14.700608253479\\
10.0497159957886 14.700608253479\\
10.1497163772583 14.700608253479\\
10.249716758728 14.700608253479\\
10.3497161865234 14.700608253479\\
10.4497165679932 14.700608253479\\
10.5497159957886 14.700608253479\\
10.6497163772583 14.700608253479\\
10.749716758728 14.700608253479\\
10.8497161865234 14.700608253479\\
10.9497165679932 14.700608253479\\
11.0497159957886 14.700608253479\\
11.1497163772583 14.700608253479\\
11.249716758728 14.700608253479\\
11.3497161865234 14.700608253479\\
11.4497165679932 14.700608253479\\
11.5497159957886 14.700608253479\\
11.6497163772583 14.700608253479\\
11.749716758728 14.700608253479\\
11.8497161865234 14.700608253479\\
11.9497165679932 14.700608253479\\
12.0497159957886 14.700608253479\\
12.1497163772583 14.700608253479\\
12.249716758728 14.700608253479\\
12.3471450805664 14.700608253479\\
12.4250917434692 14.700608253479\\
12.4879579544067 14.7006101608276\\
12.557110786438 14.7006072998047\\
12.6331787109375 14.7006092071533\\
12.7168531417847 14.700608253479\\
12.808895111084 14.700608253479\\
12.9088954925537 14.700608253479\\
13.0088949203491 14.700608253479\\
13.1088953018188 14.700608253479\\
13.2088947296143 14.700608253479\\
13.308895111084 14.700608253479\\
13.4088954925537 14.700608253479\\
13.5088949203491 14.700608253479\\
13.6088953018188 14.700608253479\\
13.7088947296143 14.700608253479\\
13.808895111084 14.700608253479\\
13.9088954925537 14.700608253479\\
14.0088949203491 14.700608253479\\
14.1088953018188 14.700608253479\\
14.2088947296143 14.700608253479\\
14.308895111084 14.700608253479\\
14.4088954925537 14.700608253479\\
14.5088949203491 14.700608253479\\
14.6088953018188 14.700608253479\\
14.7088947296143 14.700608253479\\
14.808895111084 14.700608253479\\
14.9088954925537 14.700608253479\\
15.0088949203491 14.700608253479\\
15.1088953018188 14.700608253479\\
15.2088947296143 14.700608253479\\
15.308895111084 14.700608253479\\
15.4088954925537 14.700608253479\\
15.5088949203491 14.700608253479\\
15.6088953018188 14.700608253479\\
15.7088947296143 14.700608253479\\
15.808895111084 14.700608253479\\
15.9088954925537 14.700608253479\\
16.0088958740234 14.700608253479\\
16.1088943481445 14.700608253479\\
16.2088947296143 14.700608253479\\
16.308895111084 14.700608253479\\
16.4088954925537 14.700608253479\\
16.5088958740234 14.700608253479\\
16.6088943481445 14.700608253479\\
16.7088947296143 14.700608253479\\
16.808895111084 14.700608253479\\
16.9088954925537 14.700608253479\\
17.0088958740234 14.700608253479\\
17.1088943481445 14.700608253479\\
17.2088947296143 14.700608253479\\
17.308895111084 14.700608253479\\
17.4088954925537 14.700608253479\\
17.5088958740234 14.700608253479\\
17.6088943481445 14.700608253479\\
17.7088947296143 14.700608253479\\
17.808895111084 14.700608253479\\
17.9088954925537 14.700608253479\\
18.0088958740234 14.700608253479\\
18.1088943481445 14.700608253479\\
18.2088947296143 14.700608253479\\
18.308895111084 14.7006072998047\\
18.4088954925537 14.700608253479\\
18.5088958740234 14.7006053924561\\
18.6088943481445 14.7006053924561\\
18.7088947296143 14.700608253479\\
18.808895111084 14.7006092071533\\
18.9088954925537 14.7006101608276\\
19.0088958740234 14.7006092071533\\
19.1088943481445 14.700608253479\\
19.2088947296143 14.700608253479\\
19.308895111084 14.700608253479\\
19.4088954925537 14.700608253479\\
19.5088958740234 14.700608253479\\
19.6088943481445 14.700608253479\\
19.7088947296143 14.700608253479\\
19.808895111084 14.700608253479\\
19.9088954925537 14.700608253479\\
20.0088958740234 14.700608253479\\
20.1088943481445 14.700608253479\\
20.2088947296143 14.700608253479\\
20.308895111084 14.700608253479\\
20.4088954925537 14.700608253479\\
20.5088958740234 14.700608253479\\
20.6088943481445 14.700608253479\\
20.7088947296143 14.700608253479\\
20.808895111084 14.700608253479\\
20.9088954925537 14.700608253479\\
21.0088958740234 14.700608253479\\
21.1088943481445 14.700608253479\\
21.2088947296143 14.700608253479\\
21.308895111084 14.700608253479\\
21.4088954925537 14.700608253479\\
21.5088958740234 14.700608253479\\
21.6088943481445 14.700608253479\\
21.7088947296143 14.700608253479\\
21.808895111084 14.700608253479\\
21.9088954925537 14.700608253479\\
22.0088958740234 14.700608253479\\
22.1088943481445 14.700608253479\\
22.2088947296143 14.700608253479\\
22.308895111084 14.700608253479\\
22.4088954925537 14.700608253479\\
22.5088958740234 14.700608253479\\
22.6088943481445 14.700608253479\\
22.7088947296143 14.700608253479\\
22.808895111084 14.700608253479\\
22.9088954925537 14.700608253479\\
23.0088958740234 14.700608253479\\
23.1088943481445 14.700608253479\\
23.2088947296143 14.700608253479\\
23.308895111084 14.700608253479\\
23.4088954925537 14.700608253479\\
23.5088958740234 14.700608253479\\
23.6088943481445 14.700608253479\\
23.7088947296143 14.700608253479\\
23.808895111084 14.700608253479\\
23.9088954925537 14.700608253479\\
24.0088958740234 14.700608253479\\
24.1088943481445 14.700608253479\\
24.2088947296143 14.700608253479\\
24.308895111084 14.700608253479\\
24.4088954925537 14.700608253479\\
24.5088958740234 14.700608253479\\
24.6048831939697 14.700608253479\\
24.681676864624 14.7006072998047\\
24.7464084625244 14.700608253479\\
24.8176136016846 14.700608253479\\
24.895938873291 14.700608253479\\
24.9820976257324 14.7006092071533\\
25.0768718719482 14.7006092071533\\
25.176872253418 14.7006092071533\\
25.2768707275391 14.7006092071533\\
25.3768711090088 14.7006092071533\\
25.4768714904785 14.7006092071533\\
25.5768718719482 14.7006092071533\\
25.676872253418 14.7006092071533\\
25.7768707275391 14.7006092071533\\
25.8768711090088 14.7006092071533\\
25.9768714904785 14.7006092071533\\
26.0768718719482 14.7006092071533\\
26.176872253418 14.7006092071533\\
26.2768707275391 14.7006092071533\\
26.3768711090088 14.7006092071533\\
26.4768714904785 14.7006092071533\\
26.5768718719482 14.7006092071533\\
26.676872253418 14.7006092071533\\
26.7768707275391 14.7006092071533\\
26.8768711090088 14.7006092071533\\
26.9768714904785 14.7006092071533\\
27.0768718719482 14.7006092071533\\
27.176872253418 14.7006092071533\\
27.2768707275391 14.7006092071533\\
27.3768711090088 14.7006092071533\\
27.4768714904785 14.7006092071533\\
27.5768718719482 14.7006092071533\\
27.676872253418 14.7006092071533\\
27.7768707275391 14.7006092071533\\
27.8768711090088 14.7006092071533\\
27.9768714904785 14.7006092071533\\
28.0768718719482 14.7006092071533\\
28.176872253418 14.7006092071533\\
28.2768707275391 14.7006092071533\\
28.3768711090088 14.7006092071533\\
28.4768714904785 14.7006092071533\\
28.5768718719482 14.7006092071533\\
28.676872253418 14.7006092071533\\
28.7768707275391 14.7006092071533\\
28.8768711090088 14.7006092071533\\
28.9768714904785 14.7006092071533\\
29.0768718719482 14.7006092071533\\
29.176872253418 14.7006092071533\\
29.2768707275391 14.7006092071533\\
29.3768711090088 14.7006092071533\\
29.4768714904785 14.7006092071533\\
29.5768718719482 14.7006092071533\\
29.676872253418 14.7006092071533\\
29.7768707275391 14.7006092071533\\
29.8768711090088 14.7006092071533\\
29.9768714904785 14.7006092071533\\
30.0768718719482 14.7006101608276\\
30.176872253418 14.7006130218506\\
30.2768707275391 14.700626373291\\
30.3768711090088 14.7006864547729\\
30.4768714904785 14.7010078430176\\
30.5768718719482 14.702657699585\\
30.676872253418 14.7090797424316\\
30.7768707275391 14.7222843170166\\
30.8768711090088 14.7311239242554\\
30.9768714904785 14.7330932617188\\
31.0768718719482 14.7321681976318\\
31.176872253418 14.732889175415\\
31.2768707275391 14.732741355896\\
31.3768711090088 14.7336473464966\\
31.4768714904785 14.7339267730713\\
31.5768718719482 14.7346258163452\\
31.676872253418 14.7350444793701\\
31.7768707275391 14.735650062561\\
31.8768711090088 14.7361316680908\\
31.9768714904785 14.7366962432861\\
32.0768699645996 14.7372055053711\\
32.176872253418 14.7377510070801\\
32.2768707275391 14.7382726669312\\
32.3768730163574 14.7388114929199\\
32.4768714904785 14.7393388748169\\
32.5768699645996 14.7398738861084\\
32.676872253418 14.7404041290283\\
32.7768707275391 14.7409372329712\\
32.8768730163574 14.7414693832397\\
32.9768714904785 14.7420024871826\\
33.0768699645996 14.7425346374512\\
33.176872253418 14.7430686950684\\
33.2768707275391 14.7436017990112\\
33.3768730163574 14.7443351745605\\
33.4768714904785 14.7450523376465\\
33.5768699645996 14.7456855773926\\
33.676872253418 14.7463312149048\\
33.7768707275391 14.7469692230225\\
33.8768730163574 14.7476081848145\\
33.9768714904785 14.7482461929321\\
34.0768699645996 14.7488822937012\\
34.176872253418 14.7495183944702\\
34.2768707275391 14.7501544952393\\
34.3768730163574 14.750789642334\\
34.4768714904785 14.7514247894287\\
34.5768699645996 14.7520589828491\\
34.676872253418 14.7526941299438\\
34.7768707275391 14.7533292770386\\
34.8768730163574 14.7539043426514\\
34.9768714904785 14.7542066574097\\
35.0768699645996 14.7548160552979\\
35.176872253418 14.7553253173828\\
35.2768707275391 14.7558679580688\\
35.3768730163574 14.7563934326172\\
35.4768714904785 14.7569208145142\\
35.5768699645996 14.7574415206909\\
35.676872253418 14.7579593658447\\
35.7768707275391 14.7584714889526\\
35.8768730163574 14.7589778900146\\
35.9768714904785 14.75954246521\\
36.0768699645996 14.7602033615112\\
36.176872253418 14.760835647583\\
36.2768707275391 14.7614469528198\\
36.3768730163574 14.7620754241943\\
36.4768714904785 14.7627305984497\\
36.5768699645996 14.7634296417236\\
36.676872253418 14.764199256897\\
36.7768707275391 14.7693824768066\\
36.8673248291016 14.774468421936\\
36.9396858215332 14.7748498916626\\
37.0192794799805 14.7683591842651\\
37.1068305969238 14.7687578201294\\
37.2031402587891 14.7692346572876\\
37.3031425476074 14.7700033187866\\
37.4031410217285 14.770959854126\\
37.5031433105469 14.7719964981079\\
37.603141784668 14.7730398178101\\
37.7031402587891 14.7741031646729\\
37.8031425476074 14.7751789093018\\
37.9031410217285 14.7762670516968\\
38.0031433105469 14.7773590087891\\
38.103141784668 14.7784605026245\\
38.2031402587891 14.7795677185059\\
38.3031425476074 14.7806787490845\\
38.4031410217285 14.781792640686\\
38.5031433105469 14.7829065322876\\
38.603141784668 14.7840204238892\\
38.7031402587891 14.7851333618164\\
38.8031425476074 14.786247253418\\
38.9031410217285 14.7873611450195\\
39.0031433105469 14.7884759902954\\
39.103141784668 14.789589881897\\
39.2031402587891 14.7907037734985\\
39.3031425476074 14.7918176651001\\
39.4031410217285 14.792932510376\\
39.5031433105469 14.7940473556519\\
39.603141784668 14.7951612472534\\
39.7031402587891 14.7962760925293\\
39.8031425476074 14.7973909378052\\
39.9031410217285 14.7985067367554\\
40.0031433105469 14.7996215820313\\
40.103141784668 14.8007373809814\\
40.2031402587891 14.8018522262573\\
40.3031425476074 14.8029680252075\\
40.4031410217285 14.8040838241577\\
40.5031433105469 14.8051996231079\\
40.603141784668 14.8063163757324\\
40.7031402587891 14.8074321746826\\
40.8031425476074 14.8085489273071\\
40.9031410217285 14.8096723556519\\
41.0031433105469 14.8107757568359\\
41.103141784668 14.8119029998779\\
41.2031402587891 14.8130235671997\\
41.3031425476074 14.8141460418701\\
41.4031410217285 14.8152685165405\\
41.5031433105469 14.8163900375366\\
41.603141784668 14.817512512207\\
41.7031402587891 14.8186349868774\\
41.8031425476074 14.8197574615479\\
41.9031410217285 14.8208799362183\\
42.0031433105469 14.8220024108887\\
42.103141784668 14.8231258392334\\
42.2031402587891 14.8242483139038\\
42.3031425476074 14.8253717422485\\
42.4031410217285 14.8264961242676\\
42.5031433105469 14.8276252746582\\
42.603141784668 14.8287811279297\\
42.7031402587891 14.8300819396973\\
42.8031425476074 14.8321371078491\\
42.9031410217285 14.8368940353394\\
43.0031433105469 14.8442010879517\\
43.103141784668 14.8452301025391\\
43.2031402587891 14.8406019210815\\
43.3031425476074 14.8382244110107\\
43.4031410217285 14.8392143249512\\
43.5031433105469 14.8391122817993\\
43.603141784668 14.8408451080322\\
43.7031402587891 14.8415565490723\\
43.8031425476074 14.8429546356201\\
43.9031410217285 14.8438959121704\\
44.0031433105469 14.845142364502\\
44.103141784668 14.8461856842041\\
44.2031402587891 14.8473634719849\\
44.3031425476074 14.8484525680542\\
44.4031410217285 14.8496017456055\\
44.5031433105469 14.8507099151611\\
44.603141784668 14.851845741272\\
44.7031402587891 14.8529624938965\\
44.8031425476074 14.8540925979614\\
44.9031410217285 14.8552150726318\\
45.0031433105469 14.8563423156738\\
45.103141784668 14.8574657440186\\
45.2031402587891 14.8585920333862\\
45.3031425476074 14.8597164154053\\
45.4031410217285 14.8608417510986\\
45.5031433105469 14.861967086792\\
45.603141784668 14.8632221221924\\
45.7031402587891 14.8645401000977\\
45.8031425476074 14.8656330108643\\
45.9031410217285 14.8667726516724\\
46.0031433105469 14.8678941726685\\
46.103141784668 14.8690223693848\\
46.2031402587891 14.8701486587524\\
46.3031425476074 14.8712759017944\\
46.4031410217285 14.8724031448364\\
46.5031433105469 14.8735313415527\\
46.603141784668 14.874659538269\\
46.7031402587891 14.8757886886597\\
46.8031425476074 14.8769178390503\\
46.9031410217285 14.8780479431152\\
47.0031433105469 14.8791780471802\\
47.103141784668 14.8802900314331\\
47.2031402587891 14.8810997009277\\
47.3031425476074 14.8822660446167\\
47.4031410217285 14.8833646774292\\
47.5031433105469 14.8844890594482\\
47.603141784668 14.8856029510498\\
47.7031402587891 14.88671875\\
47.8031425476074 14.8878335952759\\
47.9031410217285 14.8889465332031\\
48.0031433105469 14.8900594711304\\
48.103141784668 14.891170501709\\
48.2031402587891 14.8923196792603\\
48.3031425476074 14.8935546875\\
48.4031410217285 14.8947610855103\\
48.5031433105469 14.8959159851074\\
48.603141784668 14.8970737457275\\
48.7031402587891 14.8982496261597\\
48.8031425476074 14.8994512557983\\
48.9031410217285 14.9006967544556\\
49.0031433105469 14.9047794342041\\
49.0980262756348 14.9106140136719\\
49.1739387512207 14.9126577377319\\
49.2413482666016 14.9048385620117\\
49.315502166748 14.9052743911743\\
49.3970718383789 14.9058389663696\\
49.4867973327637 14.906548500061\\
49.5854949951172 14.9074401855469\\
49.6854934692383 14.9085206985474\\
49.7854957580566 14.9096240997314\\
49.8854942321777 14.9107313156128\\
49.9854927062988 14.9118404388428\\
50.0854949951172 14.912956237793\\
50.1854934692383 14.9140720367432\\
50.2854957580566 14.9151878356934\\
50.3854942321777 14.9163074493408\\
50.4854927062988 14.9174289703369\\
50.5854949951172 14.9185543060303\\
50.6854934692383 14.9196815490723\\
50.7854957580566 14.9208078384399\\
50.8854942321777 14.9219341278076\\
50.9854927062988 14.9230604171753\\
51.0854949951172 14.924186706543\\
51.1854934692383 14.9253129959106\\
51.2854957580566 14.9264392852783\\
51.3854942321777 14.927565574646\\
51.4854927062988 14.9286918640137\\
51.5854949951172 14.9298181533813\\
51.6854934692383 14.930944442749\\
51.7854957580566 14.9320707321167\\
51.8854942321777 14.9331970214844\\
51.9854927062988 14.9343233108521\\
52.0854949951172 14.9354496002197\\
52.1854934692383 14.9365758895874\\
52.2854957580566 14.9377021789551\\
52.3854942321777 14.9388284683228\\
52.4854927062988 14.9399547576904\\
52.5854949951172 14.9410820007324\\
52.6854934692383 14.9422082901001\\
52.7854957580566 14.9433345794678\\
52.8854942321777 14.9444608688354\\
52.9854927062988 14.9455871582031\\
53.0854949951172 14.9467134475708\\
53.1854934692383 14.9478397369385\\
53.2854957580566 14.9489727020264\\
53.3854942321777 14.9500818252563\\
53.4854927062988 14.9512138366699\\
53.5854949951172 14.952338218689\\
53.6854934692383 14.953465461731\\
53.7854957580566 14.9545917510986\\
53.8854942321777 14.9557189941406\\
53.9854927062988 14.9568452835083\\
54.0854949951172 14.957971572876\\
54.1854934692383 14.959098815918\\
54.2854957580566 14.9602251052856\\
54.3854942321777 14.9613513946533\\
54.4854927062988 14.9624786376953\\
54.5854949951172 14.9636058807373\\
54.6854934692383 14.9647350311279\\
54.7854957580566 14.9658727645874\\
54.8854942321777 14.9670562744141\\
54.9854927062988 14.9684934616089\\
55.0854949951172 14.9711465835571\\
55.1854934692383 14.977165222168\\
55.2854957580566 14.9838943481445\\
55.3854942321777 14.9830350875854\\
55.4854927062988 14.9745264053345\\
55.5854949951172 14.9779005050659\\
55.6854934692383 14.9762477874756\\
55.7854957580566 14.9781522750854\\
55.8854942321777 14.9786148071289\\
55.9854927062988 14.9801797866821\\
56.0854949951172 14.9810152053833\\
56.1854934692383 14.9823360443115\\
56.2854957580566 14.9833345413208\\
56.3854942321777 14.9845476150513\\
56.4854927062988 14.9856176376343\\
56.5854949951172 14.9867830276489\\
56.6854934692383 14.9878854751587\\
56.7854957580566 14.9890298843384\\
56.8854942321777 14.9901466369629\\
56.9854927062988 14.9912815093994\\
57.0854949951172 14.9924039840698\\
57.1854934692383 14.9935350418091\\
57.2854957580566 14.9946613311768\\
57.3854942321777 14.9957904815674\\
57.4854927062988 14.9969177246094\\
57.5854949951172 14.9980459213257\\
57.6854934692383 14.999174118042\\
57.7854957580566 15.0003023147583\\
57.8854942321777 15.0016708374023\\
57.9854927062988 15.0028591156006\\
58.0854949951172 15.0039844512939\\
58.1854934692383 15.0051155090332\\
58.2854957580566 15.0062437057495\\
58.3854942321777 15.0073738098145\\
58.4854927062988 15.0085039138794\\
58.5854949951172 15.0096340179443\\
58.6854934692383 15.0107641220093\\
58.7854957580566 15.0118951797485\\
58.8854942321777 15.0130262374878\\
58.9854927062988 15.0141582489014\\
59.0854949951172 15.0152902603149\\
59.1854934692383 15.0164232254028\\
59.2854957580566 15.017557144165\\
59.3854942321777 15.018593788147\\
59.4854927062988 15.0194597244263\\
59.5854949951172 15.020655632019\\
59.6854934692383 15.0217485427856\\
59.7854957580566 15.0228776931763\\
59.8854942321777 15.0239925384521\\
59.9427452087402 15.0250110626221\\
60 15.0257797241211\\
60.0629768371582 15.0263977050781\\
60.1322555541992 15.0271062850952\\
60.2084579467773 15.0278768539429\\
60.2922821044922 15.0287265777588\\
60.3844909667969 15.0296592712402\\
60.484489440918 15.0307474136353\\
60.5844879150391 15.0319776535034\\
60.6844902038574 15.0331649780273\\
60.7844886779785 15.0343141555786\\
60.8844909667969 15.0354795455933\\
60.984489440918 15.0366621017456\\
61.0844879150391 15.0378761291504\\
61.1844902038574 15.0391416549683\\
61.2844886779785 15.0453767776489\\
61.3734016418457 15.0494928359985\\
61.444522857666 15.0484647750854\\
61.5227546691895 15.042950630188\\
61.6088104248047 15.0435914993286\\
61.7034721374512 15.0442504882813\\
61.8034706115723 15.0451545715332\\
61.9034729003906 15.0462188720703\\
62.0034713745117 15.0473327636719\\
62.1034736633301 15.0484371185303\\
62.2034721374512 15.0495481491089\\
62.3034706115723 15.0506620407104\\
62.4034729003906 15.0517816543579\\
62.5034713745117 15.0528964996338\\
62.6034736633301 15.0540161132813\\
62.7034721374512 15.0551385879517\\
62.8034706115723 15.0562629699707\\
62.9034729003906 15.057391166687\\
63.0034713745117 15.058518409729\\
63.1034736633301 15.059645652771\\
63.2034721374512 15.060772895813\\
63.3034706115723 15.0619010925293\\
63.4034729003906 15.0630283355713\\
63.5034713745117 15.0641555786133\\
63.6034736633301 15.0652828216553\\
63.7034721374512 15.0664100646973\\
63.8034706115723 15.0675382614136\\
63.9034729003906 15.0686655044556\\
64.0034713745117 15.0697927474976\\
64.1034698486328 15.0709199905396\\
64.2034759521484 15.0720472335815\\
64.3034744262695 15.0731744766235\\
64.4034729003906 15.0743017196655\\
64.5034713745117 15.0754299163818\\
64.6034698486328 15.0765571594238\\
64.7034759521484 15.0776844024658\\
64.8034744262695 15.0788116455078\\
64.9034729003906 15.0799388885498\\
65.0034713745117 15.0810661315918\\
65.1034698486328 15.0821933746338\\
65.2034759521484 15.0833215713501\\
65.3034744262695 15.0844488143921\\
65.4034729003906 15.0855760574341\\
65.5034713745117 15.0867033004761\\
65.6034698486328 15.0878305435181\\
65.7034759521484 15.0889644622803\\
65.8034744262695 15.0900745391846\\
65.9034729003906 15.0912075042725\\
66.0034713745117 15.0923328399658\\
66.1034698486328 15.0934610366821\\
66.2034759521484 15.0945882797241\\
66.3034744262695 15.0957164764404\\
66.4034729003906 15.0968437194824\\
66.5034713745117 15.0979709625244\\
66.6034698486328 15.0990991592407\\
66.7034759521484 15.1002264022827\\
66.8034744262695 15.101354598999\\
66.9034729003906 15.1024837493896\\
67.0034713745117 15.1036176681519\\
67.1034698486328 15.1047773361206\\
67.2034759521484 15.1060771942139\\
67.3034744262695 15.1081085205078\\
67.4034729003906 15.1127891540527\\
67.5034713745117 15.1201086044312\\
67.6034698486328 15.1214218139648\\
67.7034759521484 15.1166677474976\\
67.8034744262695 15.1142797470093\\
67.9034729003906 15.1152744293213\\
68.0034713745117 15.1151437759399\\
68.1034698486328 15.1168909072876\\
68.2034759521484 15.1175985336304\\
68.3034744262695 15.1190052032471\\
68.4034729003906 15.1199474334717\\
68.5034713745117 15.1211996078491\\
68.6034698486328 15.1222448348999\\
68.7034759521484 15.1234283447266\\
68.8034744262695 15.1245203018188\\
68.9034729003906 15.1256723403931\\
69.0034713745117 15.126784324646\\
69.1034698486328 15.1279239654541\\
69.2034759521484 15.1290445327759\\
69.3034744262695 15.1301784515381\\
69.4034729003906 15.1313037872314\\
69.5034713745117 15.1324338912964\\
69.6034698486328 15.1335611343384\\
69.7034759521484 15.1346912384033\\
69.8034744262695 15.1358194351196\\
69.9034729003906 15.1369485855103\\
70.0034713745117 15.1380777359009\\
70.1034698486328 15.1393251419067\\
70.2034759521484 15.1406593322754\\
70.3034744262695 15.141752243042\\
70.4034729003906 15.1428956985474\\
70.5034713745117 15.1440210342407\\
70.6034698486328 15.1451530456543\\
70.7034759521484 15.1462821960449\\
70.8034744262695 15.1474132537842\\
70.9034729003906 15.1485443115234\\
71.0034713745117 15.149676322937\\
71.1034698486328 15.1508073806763\\
71.2034759521484 15.1519403457642\\
71.3034744262695 15.1530733108521\\
71.4034729003906 15.1542062759399\\
71.5034713745117 15.1553401947021\\
71.6034698486328 15.156457901001\\
71.7034759521484 15.1572828292847\\
71.8034744262695 15.1584367752075\\
71.9034729003906 15.1595439910889\\
72.0034713745117 15.1606693267822\\
72.1034698486328 15.1617879867554\\
72.2034759521484 15.1629076004028\\
72.3034744262695 15.1640253067017\\
72.4034729003906 15.1651420593262\\
72.5034713745117 15.1662578582764\\
72.6034698486328 15.1673727035522\\
72.7034759521484 15.1685199737549\\
72.8034744262695 15.1697454452515\\
72.9034729003906 15.170952796936\\
73.0034713745117 15.1721096038818\\
73.1034698486328 15.1732683181763\\
73.2034759521484 15.1744441986084\\
73.3034744262695 15.1756458282471\\
73.4034729003906 15.17689037323\\
73.5034713745117 15.1806659698486\\
73.5992889404297 15.1867485046387\\
73.6759490966797 15.1891202926636\\
73.7424468994141 15.1811037063599\\
73.8155975341797 15.1815032958984\\
73.8960571289063 15.1820735931396\\
73.9845733642578 15.182765007019\\
74.0819320678711 15.1836404800415\\
74.1819305419922 15.18470287323\\
74.2819290161133 15.1858100891113\\
74.3819351196289 15.1869173049927\\
74.48193359375 15.1880292892456\\
74.5819320678711 15.1891450881958\\
74.6819305419922 15.1902647018433\\
74.7819290161133 15.1913805007935\\
74.8819351196289 15.1925020217896\\
74.9409637451172 15.1936292648315\\
75 15.1942892074585\\
};
\addlegendentry{10 "};

\addplot [
color=black,
dotted
]
table[row sep=crcr]{
0 14.7005214691162\\
9.99999971718069e-10 14.7005214691162\\
2.49999998480632e-09 14.7005214691162\\
4.74999994892755e-09 14.7005214691162\\
8.1250002281763e-09 14.7005214691162\\
1.31874999809156e-08 14.7005214691162\\
2.0781250498203e-08 14.7005214691162\\
3.21718758300449e-08 14.7005214691162\\
4.92578138278077e-08 14.7005214691162\\
7.48867208244519e-08 14.7005214691162\\
1.13330081319418e-07 14.7005214691162\\
1.70995122061868e-07 14.7005214691162\\
2.57492672517401e-07 14.7005214691162\\
3.8723902662241e-07 14.7005214691162\\
5.81858500936505e-07 14.7005214691162\\
8.73787769251066e-07 14.7005214691162\\
1.31168167172291e-06 14.7005214691162\\
1.9685226106958e-06 14.7005214691162\\
2.95378367809462e-06 14.7005214691162\\
4.43167573394021e-06 14.7005214691162\\
6.64851359033491e-06 14.7005214691162\\
9.97376992017962e-06 14.7005214691162\\
1.49616553244414e-05 14.7005224227905\\
2.24434825213393e-05 14.7005224227905\\
3.36662233166862e-05 14.7005233764648\\
5.05003372381907e-05 14.7005233764648\\
7.57515081204474e-05 14.7005243301392\\
0.000113628258986864 14.7005252838135\\
0.000170443381648511 14.7005262374878\\
0.000255666091106832 14.7005271911621\\
0.000383500126190484 14.7005281448364\\
0.000575251178815961 14.7005290985107\\
0.000862877757754177 14.7005290985107\\
0.00129431765526533 14.7005281448364\\
0.00194147753063589 14.7005243301392\\
0.00291221728548408 14.7005195617676\\
0.00436832662671804 14.7005176544189\\
0.00597004732117057 14.7005195617676\\
0.00773194013163447 14.7005214691162\\
0.009670021943748 14.7005224227905\\
0.0118019115179777 14.7005224227905\\
0.0141469910740852 14.7005214691162\\
0.0167265776544809 14.7005214691162\\
0.0195641238242388 14.7005214691162\\
0.0226854234933853 14.7005214691162\\
0.0261188540607691 14.7005214691162\\
0.0298956278711557 14.7005214691162\\
0.0340500771999359 14.7005214691162\\
0.0386199727654457 14.7005214691162\\
0.0436468608677387 14.7005214691162\\
0.0491764321923256 14.7005214691162\\
0.0552589632570744 14.7005214691162\\
0.0619497485458851 14.7005214691162\\
0.0693096145987511 14.7005214691162\\
0.0774054601788521 14.7005186080933\\
0.0863108932971954 14.7005243301392\\
0.0961068719625473 14.7005214691162\\
0.106882445514202 14.7005214691162\\
0.118735581636429 14.7005214691162\\
0.13177402317524 14.7005214691162\\
0.146116316318512 14.7005214691162\\
0.161892831325531 14.7005214691162\\
0.179247006773949 14.7005214691162\\
0.198336601257324 14.7005214691162\\
0.21933513879776 14.7005214691162\\
0.242433547973633 14.7005214691162\\
0.267841786146164 14.7005214691162\\
0.29579085111618 14.7005214691162\\
0.32653483748436 14.7005214691162\\
0.360353201627731 14.7005214691162\\
0.397553414106369 14.7005214691162\\
0.438473641872406 14.7005214691162\\
0.483485877513886 14.7005214691162\\
0.532999336719513 14.7005214691162\\
0.587464153766632 14.7005214691162\\
0.647375464439392 14.7005214691162\\
0.71327793598175 14.7005214691162\\
0.7857705950737 14.7005214691162\\
0.865512549877167 14.7005214691162\\
0.953228712081909 14.7005214691162\\
1.04971647262573 14.7005214691162\\
1.14971649646759 14.7005214691162\\
1.24971640110016 14.7005214691162\\
1.34971642494202 14.7005214691162\\
1.44971644878387 14.7005214691162\\
1.54971647262573 14.7005214691162\\
1.64971649646759 14.7005214691162\\
1.74971640110016 14.7005214691162\\
1.84971642494202 14.7005214691162\\
1.94971644878387 14.7005214691162\\
2.04971647262573 14.7005214691162\\
2.1497163772583 14.7005214691162\\
2.24971652030945 14.7005214691162\\
2.34971642494202 14.7005214691162\\
2.44971656799316 14.7005214691162\\
2.54971647262573 14.7005214691162\\
2.6497163772583 14.7005214691162\\
2.74971652030945 14.7005214691162\\
2.84971642494202 14.7005214691162\\
2.94971656799316 14.7005214691162\\
3.04971647262573 14.7005214691162\\
3.1497163772583 14.7005214691162\\
3.24971652030945 14.7005214691162\\
3.34971642494202 14.7005214691162\\
3.44971656799316 14.7005214691162\\
3.54971647262573 14.7005214691162\\
3.6497163772583 14.7005214691162\\
3.74971652030945 14.7005214691162\\
3.84971642494202 14.7005214691162\\
3.94971656799316 14.7005214691162\\
4.04971647262573 14.7005214691162\\
4.1497163772583 14.7005214691162\\
4.24971628189087 14.7005214691162\\
4.3497166633606 14.7005214691162\\
4.44971656799316 14.7005214691162\\
4.54971647262573 14.7005214691162\\
4.6497163772583 14.7005214691162\\
4.74971628189087 14.7005214691162\\
4.8497166633606 14.7005214691162\\
4.94971656799316 14.7005214691162\\
5.04971647262573 14.7005214691162\\
5.1497163772583 14.7005214691162\\
5.24971628189087 14.7005214691162\\
5.3497166633606 14.7005214691162\\
5.44971656799316 14.7005214691162\\
5.54971647262573 14.7005214691162\\
5.6497163772583 14.7005214691162\\
5.74971628189087 14.7005214691162\\
5.8497166633606 14.7005214691162\\
5.94971656799316 14.7005214691162\\
6.04971647262573 14.7005214691162\\
6.1497163772583 14.7005205154419\\
6.24971628189087 14.7005195617676\\
6.3497166633606 14.7005186080933\\
6.44971656799316 14.7005205154419\\
6.54971647262573 14.7005224227905\\
6.6497163772583 14.7005224227905\\
6.74971628189087 14.7005233764648\\
6.8497166633606 14.7005205154419\\
6.94971656799316 14.7005214691162\\
7.04971647262573 14.7005214691162\\
7.1497163772583 14.7005214691162\\
7.24971628189087 14.7005214691162\\
7.3497166633606 14.7005214691162\\
7.44971656799316 14.7005214691162\\
7.54971647262573 14.7005214691162\\
7.6497163772583 14.7005214691162\\
7.74971628189087 14.7005214691162\\
7.8497166633606 14.7005214691162\\
7.94971656799316 14.7005214691162\\
8.04971599578857 14.7005214691162\\
8.1497163772583 14.7005214691162\\
8.24971675872803 14.7005214691162\\
8.34971618652344 14.7005214691162\\
8.44971656799316 14.7005214691162\\
8.54971599578857 14.7005214691162\\
8.6497163772583 14.7005214691162\\
8.74971675872803 14.7005214691162\\
8.84971618652344 14.7005214691162\\
8.94971656799316 14.7005214691162\\
9.04971599578857 14.7005214691162\\
9.1497163772583 14.7005214691162\\
9.24971675872803 14.7005214691162\\
9.34971618652344 14.7005214691162\\
9.44971656799316 14.7005214691162\\
9.54971599578857 14.7005214691162\\
9.6497163772583 14.7005214691162\\
9.74971675872803 14.7005214691162\\
9.84971618652344 14.7005214691162\\
9.94971656799316 14.7005214691162\\
10.0497159957886 14.7005214691162\\
10.1497163772583 14.7005214691162\\
10.249716758728 14.7005214691162\\
10.3497161865234 14.7005214691162\\
10.4497165679932 14.7005214691162\\
10.5497159957886 14.7005214691162\\
10.6497163772583 14.7005214691162\\
10.749716758728 14.7005214691162\\
10.8497161865234 14.7005214691162\\
10.9497165679932 14.7005214691162\\
11.0497159957886 14.7005214691162\\
11.1497163772583 14.7005214691162\\
11.249716758728 14.7005214691162\\
11.3497161865234 14.7005214691162\\
11.4497165679932 14.7005214691162\\
11.5497159957886 14.7005214691162\\
11.6497163772583 14.7005214691162\\
11.749716758728 14.7005214691162\\
11.8497161865234 14.7005214691162\\
11.9497165679932 14.7005214691162\\
12.0497159957886 14.7005214691162\\
12.1497163772583 14.7005214691162\\
12.249716758728 14.7005214691162\\
12.3471450805664 14.7005214691162\\
12.4250917434692 14.7005214691162\\
12.4879579544067 14.7005224227905\\
12.557110786438 14.7005205154419\\
12.6331787109375 14.7005224227905\\
12.7168531417847 14.7005214691162\\
12.808895111084 14.7005214691162\\
12.9088954925537 14.7005214691162\\
13.0088949203491 14.7005214691162\\
13.1088953018188 14.7005214691162\\
13.2088947296143 14.7005214691162\\
13.308895111084 14.7005214691162\\
13.4088954925537 14.7005214691162\\
13.5088949203491 14.7005214691162\\
13.6088953018188 14.7005214691162\\
13.7088947296143 14.7005214691162\\
13.808895111084 14.7005214691162\\
13.9088954925537 14.7005214691162\\
14.0088949203491 14.7005214691162\\
14.1088953018188 14.7005214691162\\
14.2088947296143 14.7005214691162\\
14.308895111084 14.7005214691162\\
14.4088954925537 14.7005214691162\\
14.5088949203491 14.7005214691162\\
14.6088953018188 14.7005214691162\\
14.7088947296143 14.7005214691162\\
14.808895111084 14.7005214691162\\
14.9088954925537 14.7005214691162\\
15.0088949203491 14.7005214691162\\
15.1088953018188 14.7005214691162\\
15.2088947296143 14.7005214691162\\
15.308895111084 14.7005214691162\\
15.4088954925537 14.7005214691162\\
15.5088949203491 14.7005214691162\\
15.6088953018188 14.7005214691162\\
15.7088947296143 14.7005214691162\\
15.808895111084 14.7005214691162\\
15.9088954925537 14.7005214691162\\
16.0088958740234 14.7005214691162\\
16.1088943481445 14.7005214691162\\
16.2088947296143 14.7005214691162\\
16.308895111084 14.7005214691162\\
16.4088954925537 14.7005214691162\\
16.5088958740234 14.7005214691162\\
16.6088943481445 14.7005214691162\\
16.7088947296143 14.7005214691162\\
16.808895111084 14.7005214691162\\
16.9088954925537 14.7005214691162\\
17.0088958740234 14.7005214691162\\
17.1088943481445 14.7005214691162\\
17.2088947296143 14.7005214691162\\
17.308895111084 14.7005214691162\\
17.4088954925537 14.7005214691162\\
17.5088958740234 14.7005214691162\\
17.6088943481445 14.7005214691162\\
17.7088947296143 14.7005214691162\\
17.808895111084 14.7005214691162\\
17.9088954925537 14.7005214691162\\
18.0088958740234 14.7005214691162\\
18.1088943481445 14.7005214691162\\
18.2088947296143 14.7005214691162\\
18.308895111084 14.7005205154419\\
18.4088954925537 14.7005214691162\\
18.5088958740234 14.7005195617676\\
18.6088943481445 14.7005186080933\\
18.7088947296143 14.7005214691162\\
18.808895111084 14.7005224227905\\
18.9088954925537 14.7005224227905\\
19.0088958740234 14.7005224227905\\
19.1088943481445 14.7005214691162\\
19.2088947296143 14.7005214691162\\
19.308895111084 14.7005214691162\\
19.4088954925537 14.7005214691162\\
19.5088958740234 14.7005214691162\\
19.6088943481445 14.7005214691162\\
19.7088947296143 14.7005214691162\\
19.808895111084 14.7005214691162\\
19.9088954925537 14.7005214691162\\
20.0088958740234 14.7005214691162\\
20.1088943481445 14.7005214691162\\
20.2088947296143 14.7005214691162\\
20.308895111084 14.7005214691162\\
20.4088954925537 14.7005214691162\\
20.5088958740234 14.7005214691162\\
20.6088943481445 14.7005214691162\\
20.7088947296143 14.7005214691162\\
20.808895111084 14.7005214691162\\
20.9088954925537 14.7005214691162\\
21.0088958740234 14.7005214691162\\
21.1088943481445 14.7005214691162\\
21.2088947296143 14.7005214691162\\
21.308895111084 14.7005214691162\\
21.4088954925537 14.7005214691162\\
21.5088958740234 14.7005214691162\\
21.6088943481445 14.7005214691162\\
21.7088947296143 14.7005214691162\\
21.808895111084 14.7005214691162\\
21.9088954925537 14.7005214691162\\
22.0088958740234 14.7005214691162\\
22.1088943481445 14.7005214691162\\
22.2088947296143 14.7005214691162\\
22.308895111084 14.7005214691162\\
22.4088954925537 14.7005214691162\\
22.5088958740234 14.7005214691162\\
22.6088943481445 14.7005214691162\\
22.7088947296143 14.7005214691162\\
22.808895111084 14.7005214691162\\
22.9088954925537 14.7005214691162\\
23.0088958740234 14.7005214691162\\
23.1088943481445 14.7005214691162\\
23.2088947296143 14.7005214691162\\
23.308895111084 14.7005214691162\\
23.4088954925537 14.7005214691162\\
23.5088958740234 14.7005214691162\\
23.6088943481445 14.7005214691162\\
23.7088947296143 14.7005214691162\\
23.808895111084 14.7005214691162\\
23.9088954925537 14.7005214691162\\
24.0088958740234 14.7005214691162\\
24.1088943481445 14.7005214691162\\
24.2088947296143 14.7005214691162\\
24.308895111084 14.7005214691162\\
24.4088954925537 14.7005214691162\\
24.5088958740234 14.7005214691162\\
24.6048831939697 14.7005214691162\\
24.681676864624 14.7005214691162\\
24.7464084625244 14.7005224227905\\
24.8176136016846 14.7005214691162\\
24.895938873291 14.7005214691162\\
24.9820976257324 14.7005214691162\\
25.0768718719482 14.7005214691162\\
25.176872253418 14.7005214691162\\
25.2768707275391 14.7005214691162\\
25.3768711090088 14.7005214691162\\
25.4768714904785 14.7005214691162\\
25.5768718719482 14.7005214691162\\
25.676872253418 14.7005214691162\\
25.7768707275391 14.7005214691162\\
25.8768711090088 14.7005214691162\\
25.9768714904785 14.7005214691162\\
26.0768718719482 14.7005214691162\\
26.176872253418 14.7005214691162\\
26.2768707275391 14.7005214691162\\
26.3768711090088 14.7005214691162\\
26.4768714904785 14.7005214691162\\
26.5768718719482 14.7005214691162\\
26.676872253418 14.7005214691162\\
26.7768707275391 14.7005214691162\\
26.8768711090088 14.7005214691162\\
26.9768714904785 14.7005214691162\\
27.0768718719482 14.7005214691162\\
27.176872253418 14.7005214691162\\
27.2768707275391 14.7005214691162\\
27.3768711090088 14.7005214691162\\
27.4768714904785 14.7005214691162\\
27.5768718719482 14.7005214691162\\
27.676872253418 14.7005214691162\\
27.7768707275391 14.7005214691162\\
27.8768711090088 14.7005214691162\\
27.9768714904785 14.7005214691162\\
28.0768718719482 14.7005214691162\\
28.176872253418 14.7005214691162\\
28.2768707275391 14.7005214691162\\
28.3768711090088 14.7005214691162\\
28.4768714904785 14.7005214691162\\
28.5768718719482 14.7005214691162\\
28.676872253418 14.7005214691162\\
28.7768707275391 14.7005214691162\\
28.8768711090088 14.7005214691162\\
28.9768714904785 14.7005214691162\\
29.0768718719482 14.7005214691162\\
29.176872253418 14.7005214691162\\
29.2768707275391 14.7005214691162\\
29.3768711090088 14.7005214691162\\
29.4768714904785 14.7005214691162\\
29.5768718719482 14.7005214691162\\
29.676872253418 14.7005214691162\\
29.7768707275391 14.7005214691162\\
29.8768711090088 14.7005214691162\\
29.9768714904785 14.7005214691162\\
30.0768718719482 14.7005214691162\\
30.176872253418 14.7005214691162\\
30.2768707275391 14.7005214691162\\
30.3768711090088 14.7005214691162\\
30.4768714904785 14.7005214691162\\
30.5768718719482 14.7005195617676\\
30.676872253418 14.7005214691162\\
30.7768707275391 14.7005186080933\\
30.8768711090088 14.7005205154419\\
30.9768714904785 14.7005214691162\\
31.0768718719482 14.7005233764648\\
31.176872253418 14.7005224227905\\
31.2768707275391 14.7005214691162\\
31.3768711090088 14.7005214691162\\
31.4768714904785 14.7005214691162\\
31.5768718719482 14.7005214691162\\
31.676872253418 14.7005214691162\\
31.7768707275391 14.7005214691162\\
31.8768711090088 14.7005214691162\\
31.9768714904785 14.7005214691162\\
32.0768699645996 14.7005214691162\\
32.176872253418 14.7005214691162\\
32.2768707275391 14.7005214691162\\
32.3768730163574 14.7005214691162\\
32.4768714904785 14.7005214691162\\
32.5768699645996 14.7005214691162\\
32.676872253418 14.7005214691162\\
32.7768707275391 14.7005214691162\\
32.8768730163574 14.7005214691162\\
32.9768714904785 14.7005214691162\\
33.0768699645996 14.7005214691162\\
33.176872253418 14.7005214691162\\
33.2768707275391 14.7005214691162\\
33.3768730163574 14.7005214691162\\
33.4768714904785 14.7005214691162\\
33.5768699645996 14.7005214691162\\
33.676872253418 14.7005214691162\\
33.7768707275391 14.7005214691162\\
33.8768730163574 14.7005214691162\\
33.9768714904785 14.7005214691162\\
34.0768699645996 14.7005214691162\\
34.176872253418 14.7005214691162\\
34.2768707275391 14.7005214691162\\
34.3768730163574 14.7005214691162\\
34.4768714904785 14.7005214691162\\
34.5768699645996 14.7005214691162\\
34.676872253418 14.7005214691162\\
34.7768707275391 14.7005214691162\\
34.8768730163574 14.7005214691162\\
34.9768714904785 14.7005214691162\\
35.0768699645996 14.7005214691162\\
35.176872253418 14.7005214691162\\
35.2768707275391 14.7005214691162\\
35.3768730163574 14.7005214691162\\
35.4768714904785 14.7005214691162\\
35.5768699645996 14.7005214691162\\
35.676872253418 14.7005214691162\\
35.7768707275391 14.7005214691162\\
35.8768730163574 14.7005214691162\\
35.9768714904785 14.7005214691162\\
36.0768699645996 14.7005214691162\\
36.176872253418 14.7005214691162\\
36.2768707275391 14.7005214691162\\
36.3768730163574 14.7005214691162\\
36.4768714904785 14.7005214691162\\
36.5768699645996 14.7005214691162\\
36.676872253418 14.7005214691162\\
36.7768707275391 14.7005214691162\\
36.8673248291016 14.7005205154419\\
36.9396858215332 14.7005205154419\\
37.0192794799805 14.7005205154419\\
37.1068305969238 14.7005214691162\\
37.2031402587891 14.7005224227905\\
37.3031425476074 14.7005224227905\\
37.4031410217285 14.7005224227905\\
37.5031433105469 14.7005224227905\\
37.603141784668 14.7005224227905\\
37.7031402587891 14.7005224227905\\
37.8031425476074 14.7005224227905\\
37.9031410217285 14.7005224227905\\
38.0031433105469 14.7005224227905\\
38.103141784668 14.7005224227905\\
38.2031402587891 14.7005224227905\\
38.3031425476074 14.7005224227905\\
38.4031410217285 14.7005224227905\\
38.5031433105469 14.7005224227905\\
38.603141784668 14.7005224227905\\
38.7031402587891 14.7005224227905\\
38.8031425476074 14.7005224227905\\
38.9031410217285 14.7005224227905\\
39.0031433105469 14.7005224227905\\
39.103141784668 14.7005224227905\\
39.2031402587891 14.7005224227905\\
39.3031425476074 14.7005224227905\\
39.4031410217285 14.7005224227905\\
39.5031433105469 14.7005224227905\\
39.603141784668 14.7005224227905\\
39.7031402587891 14.7005224227905\\
39.8031425476074 14.7005224227905\\
39.9031410217285 14.7005224227905\\
40.0031433105469 14.7005224227905\\
40.103141784668 14.7005224227905\\
40.2031402587891 14.7005224227905\\
40.3031425476074 14.7005224227905\\
40.4031410217285 14.7005224227905\\
40.5031433105469 14.7005224227905\\
40.603141784668 14.7005224227905\\
40.7031402587891 14.7005224227905\\
40.8031425476074 14.7005224227905\\
40.9031410217285 14.7005224227905\\
41.0031433105469 14.7005224227905\\
41.103141784668 14.7005224227905\\
41.2031402587891 14.7005224227905\\
41.3031425476074 14.7005224227905\\
41.4031410217285 14.7005224227905\\
41.5031433105469 14.7005224227905\\
41.603141784668 14.7005224227905\\
41.7031402587891 14.7005224227905\\
41.8031425476074 14.7005224227905\\
41.9031410217285 14.7005224227905\\
42.0031433105469 14.7005224227905\\
42.103141784668 14.7005224227905\\
42.2031402587891 14.7005224227905\\
42.3031425476074 14.7005233764648\\
42.4031410217285 14.7005252838135\\
42.5031433105469 14.7005348205566\\
42.603141784668 14.7005796432495\\
42.7031402587891 14.7008152008057\\
42.8031425476074 14.7020473480225\\
42.9031410217285 14.7072324752808\\
43.0031433105469 14.7194356918335\\
43.103141784668 14.7306337356567\\
43.2031402587891 14.73277759552\\
43.3031425476074 14.732048034668\\
43.4031410217285 14.7328615188599\\
43.5031433105469 14.7325019836426\\
43.603141784668 14.7335052490234\\
43.7031402587891 14.733717918396\\
43.8031425476074 14.7344608306885\\
43.9031410217285 14.7348508834839\\
44.0031433105469 14.7354764938354\\
44.103141784668 14.7359447479248\\
44.2031402587891 14.736517906189\\
44.3031425476074 14.737021446228\\
44.4031410217285 14.7375726699829\\
44.5031433105469 14.738091468811\\
44.603141784668 14.7386322021484\\
44.7031402587891 14.7391576766968\\
44.8031425476074 14.7396936416626\\
44.9031410217285 14.7402238845825\\
45.0031433105469 14.7407579421997\\
45.103141784668 14.7412900924683\\
45.2031402587891 14.7418231964111\\
45.3031425476074 14.742356300354\\
45.4031410217285 14.7428903579712\\
45.5031433105469 14.7434234619141\\
45.603141784668 14.7440872192383\\
45.7031402587891 14.7448625564575\\
45.8031425476074 14.7454805374146\\
45.9031410217285 14.7461309432983\\
46.0031433105469 14.7467670440674\\
46.103141784668 14.7474069595337\\
46.2031402587891 14.7480440139771\\
46.3031425476074 14.7486801147461\\
46.4031410217285 14.7493162155151\\
46.5031433105469 14.7499513626099\\
46.603141784668 14.7505865097046\\
46.7031402587891 14.7512216567993\\
46.8031425476074 14.7518558502197\\
46.9031410217285 14.7524909973145\\
47.0031433105469 14.7531251907349\\
47.103141784668 14.7537412643433\\
47.2031402587891 14.7540502548218\\
47.3031425476074 14.7546281814575\\
47.4031410217285 14.7551498413086\\
47.5031433105469 14.7556896209717\\
47.603141784668 14.7562189102173\\
47.7031402587891 14.7567481994629\\
47.8031425476074 14.7572727203369\\
47.9031410217285 14.7577934265137\\
48.0031433105469 14.7583093643188\\
48.103141784668 14.7588205337524\\
48.2031402587891 14.7593660354614\\
48.3031425476074 14.7600030899048\\
48.4031410217285 14.7606420516968\\
48.5031433105469 14.761251449585\\
48.603141784668 14.7618713378906\\
48.7031402587891 14.7625169754028\\
48.8031425476074 14.763201713562\\
48.9031410217285 14.7639484405518\\
49.0031433105469 14.7675609588623\\
49.0980262756348 14.7735862731934\\
49.1739387512207 14.7762155532837\\
49.2413482666016 14.7682962417603\\
49.315502166748 14.7685136795044\\
49.3970718383789 14.7689046859741\\
49.4867973327637 14.7694711685181\\
49.5854949951172 14.7702465057373\\
49.6854934692383 14.7712316513062\\
49.7854957580566 14.7722635269165\\
49.8854942321777 14.7733144760132\\
49.9854927062988 14.7743806838989\\
50.0854949951172 14.7754611968994\\
50.1854934692383 14.7765502929688\\
50.2854957580566 14.7776441574097\\
50.3854942321777 14.7787475585938\\
50.4854927062988 14.7798557281494\\
50.5854949951172 14.7809677124023\\
50.6854934692383 14.7820816040039\\
50.7854957580566 14.7831945419312\\
50.8854942321777 14.7843084335327\\
50.9854927062988 14.7854223251343\\
51.0854949951172 14.7865352630615\\
51.1854934692383 14.7876491546631\\
51.2854957580566 14.7887630462646\\
51.3854942321777 14.7898769378662\\
51.4854927062988 14.7909917831421\\
51.5854949951172 14.7921056747437\\
51.6854934692383 14.7932195663452\\
51.7854957580566 14.7943344116211\\
51.8854942321777 14.7954483032227\\
51.9854927062988 14.7965631484985\\
52.0854949951172 14.7976779937744\\
52.1854934692383 14.7987928390503\\
52.2854957580566 14.7999076843262\\
52.3854942321777 14.8010225296021\\
52.4854927062988 14.8021383285522\\
52.5854949951172 14.8032531738281\\
52.6854934692383 14.8043689727783\\
52.7854957580566 14.8054847717285\\
52.8854942321777 14.8066005706787\\
52.9854927062988 14.8077163696289\\
53.0854949951172 14.8088321685791\\
53.1854934692383 14.8099489212036\\
53.2854957580566 14.8110723495483\\
53.3854942321777 14.8121757507324\\
53.4854927062988 14.8133029937744\\
53.5854949951172 14.8144226074219\\
53.6854934692383 14.8155450820923\\
53.7854957580566 14.8166666030884\\
53.8854942321777 14.8177890777588\\
53.9854927062988 14.8189115524292\\
54.0854949951172 14.8200330734253\\
54.1854934692383 14.8211555480957\\
54.2854957580566 14.8222780227661\\
54.3854942321777 14.8233995437622\\
54.4854927062988 14.8245220184326\\
54.5854949951172 14.8256454467773\\
54.6854934692383 14.8267707824707\\
54.7854957580566 14.8279037475586\\
54.8854942321777 14.829083442688\\
54.9854927062988 14.8305158615112\\
55.0854949951172 14.8331642150879\\
55.1854934692383 14.8391799926758\\
55.2854957580566 14.8459043502808\\
55.3854942321777 14.8450412750244\\
55.4854927062988 14.8365287780762\\
55.5854949951172 14.839900970459\\
55.6854934692383 14.8382434844971\\
55.7854957580566 14.840145111084\\
55.8854942321777 14.8406038284302\\
55.9854927062988 14.8421649932861\\
56.0854949951172 14.8429975509644\\
56.1854934692383 14.8443145751953\\
56.2854957580566 14.8453102111816\\
56.3854942321777 14.8465194702148\\
56.4854927062988 14.8475866317749\\
56.5854949951172 14.8487491607666\\
56.6854934692383 14.8498477935791\\
56.7854957580566 14.8509893417358\\
56.8854942321777 14.8521022796631\\
56.9854927062988 14.8532342910767\\
57.0854949951172 14.8543539047241\\
57.1854934692383 14.8554821014404\\
57.2854957580566 14.8566045761108\\
57.3854942321777 14.8577308654785\\
57.4854927062988 14.8588542938232\\
57.5854949951172 14.8599805831909\\
57.6854934692383 14.86110496521\\
57.7854957580566 14.862229347229\\
57.8854942321777 14.8635950088501\\
57.9854927062988 14.8647794723511\\
58.0854949951172 14.8659019470215\\
58.1854934692383 14.8670291900635\\
58.2854957580566 14.8681545257568\\
58.3854942321777 14.8692817687988\\
58.4854927062988 14.8704080581665\\
58.5854949951172 14.8715353012085\\
58.6854934692383 14.8726625442505\\
58.7854957580566 14.8737897872925\\
58.8854942321777 14.8749179840088\\
58.9854927062988 14.8760461807251\\
59.0854949951172 14.87717628479\\
59.1854934692383 14.8783054351807\\
59.2854957580566 14.8794364929199\\
59.3854942321777 14.8804702758789\\
59.4854927062988 14.8813333511353\\
59.5854949951172 14.8825263977051\\
59.6854934692383 14.8836164474487\\
59.7854957580566 14.8847427368164\\
59.8854942321777 14.8858547210693\\
59.9427452087402 14.8868703842163\\
60 14.887638092041\\
60.0629768371582 14.8882551193237\\
60.1322555541992 14.8889617919922\\
60.2084579467773 14.8897304534912\\
60.2922821044922 14.8905792236328\\
60.3844909667969 14.8915100097656\\
60.484489440918 14.892596244812\\
60.5844879150391 14.8938236236572\\
60.6844902038574 14.8950090408325\\
60.7844886779785 14.8961553573608\\
60.8844909667969 14.8973197937012\\
60.984489440918 14.8985004425049\\
61.0844879150391 14.899712562561\\
61.1844902038574 14.900975227356\\
61.2844886779785 14.907208442688\\
61.3734016418457 14.911322593689\\
61.444522857666 14.9102926254272\\
61.5227546691895 14.9047775268555\\
61.6088104248047 14.9054174423218\\
61.7034721374512 14.9060745239258\\
61.8034706115723 14.9069776535034\\
61.9034729003906 14.9080400466919\\
62.0034713745117 14.9091520309448\\
62.1034736633301 14.9102554321289\\
62.2034721374512 14.9113645553589\\
62.3034706115723 14.9124765396118\\
62.4034729003906 14.9135942459106\\
62.5034713745117 14.9147071838379\\
62.6034736633301 14.915825843811\\
62.7034721374512 14.9169473648071\\
62.8034706115723 14.9180698394775\\
62.9034729003906 14.9191961288452\\
63.0034713745117 14.9203224182129\\
63.1034736633301 14.9214477539063\\
63.2034721374512 14.9225740432739\\
63.3034706115723 14.9237003326416\\
63.4034729003906 14.924825668335\\
63.5034713745117 14.9259519577026\\
63.6034736633301 14.9270782470703\\
63.7034721374512 14.9282035827637\\
63.8034706115723 14.9293298721313\\
63.9034729003906 14.930456161499\\
64.0034713745117 14.9315814971924\\
64.1034698486328 14.9327077865601\\
64.2034759521484 14.9338331222534\\
64.3034744262695 14.9349594116211\\
64.4034729003906 14.9360857009888\\
64.5034713745117 14.9372110366821\\
64.6034698486328 14.9383373260498\\
64.7034759521484 14.9394636154175\\
64.8034744262695 14.9405889511108\\
64.9034729003906 14.9417152404785\\
65.0034713745117 14.9428415298462\\
65.1034698486328 14.9439668655396\\
65.2034759521484 14.9450931549072\\
65.3034744262695 14.9462194442749\\
65.4034729003906 14.9473447799683\\
65.5034713745117 14.9484710693359\\
65.6034698486328 14.9495973587036\\
65.7034759521484 14.9507303237915\\
65.8034744262695 14.9518384933472\\
65.9034729003906 14.9529695510864\\
66.0034713745117 14.9540939331055\\
66.1034698486328 14.9552211761475\\
66.2034759521484 14.9563474655151\\
66.3034744262695 14.9574737548828\\
66.4034729003906 14.9586000442505\\
66.5034713745117 14.9597263336182\\
66.6034698486328 14.9608526229858\\
66.7034759521484 14.9619789123535\\
66.8034744262695 14.9631052017212\\
66.9034729003906 14.9642333984375\\
67.0034713745117 14.9653663635254\\
67.1034698486328 14.9665241241455\\
67.2034759521484 14.9678230285645\\
67.3034744262695 14.9698534011841\\
67.4034729003906 14.9745330810547\\
67.5034713745117 14.9818506240845\\
67.6034698486328 14.9831628799438\\
67.7034759521484 14.9784069061279\\
67.8034744262695 14.9760179519653\\
67.9034729003906 14.977011680603\\
68.0034713745117 14.9768800735474\\
68.1034698486328 14.9786262512207\\
68.2034759521484 14.9793319702148\\
68.3034744262695 14.9807376861572\\
68.4034729003906 14.9816789627075\\
68.5034713745117 14.9829292297363\\
68.6034698486328 14.9839744567871\\
68.7034759521484 14.9851560592651\\
68.8034744262695 14.9862461090088\\
68.9034729003906 14.987398147583\\
69.0034713745117 14.9885091781616\\
69.1034698486328 14.9896469116211\\
69.2034759521484 14.9907665252686\\
69.3034744262695 14.9918985366821\\
69.4034729003906 14.9930229187012\\
69.5034713745117 14.9941530227661\\
69.6034698486328 14.9952793121338\\
69.7034759521484 14.9964075088501\\
69.8034744262695 14.9975347518921\\
69.9034729003906 14.9986629486084\\
70.0034713745117 14.9997901916504\\
70.1034698486328 15.0010375976563\\
70.2034759521484 15.0023698806763\\
70.3034744262695 15.0034618377686\\
70.4034729003906 15.0046043395996\\
70.5034713745117 15.0057277679443\\
70.6034698486328 15.0068588256836\\
70.7034759521484 15.0079879760742\\
70.8034744262695 15.0091180801392\\
70.9034729003906 15.0102481842041\\
71.0034713745117 15.011378288269\\
71.1034698486328 15.0125093460083\\
71.2034759521484 15.0136404037476\\
71.3034744262695 15.0147724151611\\
71.4034729003906 15.0159044265747\\
71.5034713745117 15.0170373916626\\
71.6034698486328 15.0181541442871\\
71.7034759521484 15.0189781188965\\
71.8034744262695 15.020131111145\\
71.9034729003906 15.0212373733521\\
72.0034713745117 15.0223617553711\\
72.1034698486328 15.0234785079956\\
72.2034759521484 15.0245971679688\\
72.3034744262695 15.0257139205933\\
72.4034729003906 15.0268306732178\\
72.5034713745117 15.0279455184937\\
72.6034698486328 15.0290594100952\\
72.7034759521484 15.0302047729492\\
72.8034744262695 15.0314302444458\\
72.9034729003906 15.0326356887817\\
73.0034713745117 15.0337915420532\\
73.1034698486328 15.0349493026733\\
73.2034759521484 15.0361251831055\\
73.3034744262695 15.0373258590698\\
73.4034729003906 15.0385684967041\\
73.5034713745117 15.0423431396484\\
73.5992889404297 15.0484247207642\\
73.6759490966797 15.0507965087891\\
73.7424468994141 15.042778968811\\
73.8155975341797 15.0431776046753\\
73.8960571289063 15.0437469482422\\
73.9845733642578 15.0444383621216\\
74.0819320678711 15.0453128814697\\
74.1819305419922 15.0463743209839\\
74.2819290161133 15.0474805831909\\
74.3819351196289 15.0485868453979\\
74.48193359375 15.0496978759766\\
74.5819320678711 15.0508127212524\\
74.6819305419922 15.0519313812256\\
74.7819290161133 15.0530462265015\\
74.8819351196289 15.0541667938232\\
74.9409637451172 15.0552930831909\\
75 15.0559530258179\\
};
\addlegendentry{14 "};

\addplot [
color=green,
solid
]
table[row sep=crcr]{
0 14.7004346847534\\
9.99999971718069e-10 14.7004346847534\\
2.49999998480632e-09 14.7004346847534\\
4.74999994892755e-09 14.7004346847534\\
8.1250002281763e-09 14.7004346847534\\
1.31874999809156e-08 14.7004346847534\\
2.0781250498203e-08 14.7004346847534\\
3.21718758300449e-08 14.7004346847534\\
4.92578138278077e-08 14.7004346847534\\
7.48867208244519e-08 14.7004346847534\\
1.13330081319418e-07 14.7004346847534\\
1.70995122061868e-07 14.7004346847534\\
2.57492672517401e-07 14.7004346847534\\
3.8723902662241e-07 14.7004346847534\\
5.81858500936505e-07 14.7004346847534\\
8.73787769251066e-07 14.7004346847534\\
1.31168167172291e-06 14.7004346847534\\
1.9685226106958e-06 14.7004346847534\\
2.95378367809462e-06 14.7004346847534\\
4.43167573394021e-06 14.7004346847534\\
6.64851359033491e-06 14.7004346847534\\
9.97376992017962e-06 14.7004346847534\\
1.49616553244414e-05 14.7004346847534\\
2.24434825213393e-05 14.7004346847534\\
3.36662233166862e-05 14.7004346847534\\
5.05003372381907e-05 14.7004356384277\\
7.57515081204474e-05 14.7004356384277\\
0.000113628258986864 14.7004356384277\\
0.000170443381648511 14.7004365921021\\
0.000255666091106832 14.7004375457764\\
0.000383500126190484 14.7004384994507\\
0.000575251178815961 14.7004404067993\\
0.000862877757754177 14.7004413604736\\
0.00129431765526533 14.7004413604736\\
0.00194147753063589 14.7004375457764\\
0.00291221728548408 14.7004327774048\\
0.00436832662671804 14.7004318237305\\
0.00597004732117057 14.7004327774048\\
0.00773194013163447 14.7004346847534\\
0.009670021943748 14.7004356384277\\
0.0118019115179777 14.7004356384277\\
0.0141469910740852 14.7004346847534\\
0.0167265776544809 14.7004346847534\\
0.0195641238242388 14.7004346847534\\
0.0226854234933853 14.7004346847534\\
0.0261188540607691 14.7004346847534\\
0.0298956278711557 14.7004346847534\\
0.0340500771999359 14.7004346847534\\
0.0386199727654457 14.7004346847534\\
0.0436468608677387 14.7004346847534\\
0.0491764321923256 14.7004346847534\\
0.0552589632570744 14.7004346847534\\
0.0619497485458851 14.7004346847534\\
0.0693096145987511 14.7004346847534\\
0.0774054601788521 14.7004318237305\\
0.0863108932971954 14.7004365921021\\
0.0961068719625473 14.7004346847534\\
0.106882445514202 14.7004346847534\\
0.118735581636429 14.7004346847534\\
0.13177402317524 14.7004346847534\\
0.146116316318512 14.7004346847534\\
0.161892831325531 14.7004346847534\\
0.179247006773949 14.7004346847534\\
0.198336601257324 14.7004346847534\\
0.21933513879776 14.7004346847534\\
0.242433547973633 14.7004346847534\\
0.267841786146164 14.7004346847534\\
0.29579085111618 14.7004346847534\\
0.32653483748436 14.7004346847534\\
0.360353201627731 14.7004346847534\\
0.397553414106369 14.7004346847534\\
0.438473641872406 14.7004346847534\\
0.483485877513886 14.7004346847534\\
0.532999336719513 14.7004346847534\\
0.587464153766632 14.7004346847534\\
0.647375464439392 14.7004346847534\\
0.71327793598175 14.7004346847534\\
0.7857705950737 14.7004346847534\\
0.865512549877167 14.7004346847534\\
0.953228712081909 14.7004346847534\\
1.04971647262573 14.7004346847534\\
1.14971649646759 14.7004346847534\\
1.24971640110016 14.7004346847534\\
1.34971642494202 14.7004346847534\\
1.44971644878387 14.7004346847534\\
1.54971647262573 14.7004346847534\\
1.64971649646759 14.7004346847534\\
1.74971640110016 14.7004346847534\\
1.84971642494202 14.7004346847534\\
1.94971644878387 14.7004346847534\\
2.04971647262573 14.7004346847534\\
2.1497163772583 14.7004346847534\\
2.24971652030945 14.7004346847534\\
2.34971642494202 14.7004346847534\\
2.44971656799316 14.7004346847534\\
2.54971647262573 14.7004346847534\\
2.6497163772583 14.7004346847534\\
2.74971652030945 14.7004346847534\\
2.84971642494202 14.7004346847534\\
2.94971656799316 14.7004346847534\\
3.04971647262573 14.7004346847534\\
3.1497163772583 14.7004346847534\\
3.24971652030945 14.7004346847534\\
3.34971642494202 14.7004346847534\\
3.44971656799316 14.7004346847534\\
3.54971647262573 14.7004346847534\\
3.6497163772583 14.7004346847534\\
3.74971652030945 14.7004346847534\\
3.84971642494202 14.7004346847534\\
3.94971656799316 14.7004346847534\\
4.04971647262573 14.7004346847534\\
4.1497163772583 14.7004346847534\\
4.24971628189087 14.7004346847534\\
4.3497166633606 14.7004346847534\\
4.44971656799316 14.7004346847534\\
4.54971647262573 14.7004346847534\\
4.6497163772583 14.7004346847534\\
4.74971628189087 14.7004346847534\\
4.8497166633606 14.7004346847534\\
4.94971656799316 14.7004346847534\\
5.04971647262573 14.7004346847534\\
5.1497163772583 14.7004346847534\\
5.24971628189087 14.7004346847534\\
5.3497166633606 14.7004346847534\\
5.44971656799316 14.7004346847534\\
5.54971647262573 14.7004346847534\\
5.6497163772583 14.7004346847534\\
5.74971628189087 14.7004346847534\\
5.8497166633606 14.7004346847534\\
5.94971656799316 14.7004346847534\\
6.04971647262573 14.7004346847534\\
6.1497163772583 14.7004337310791\\
6.24971628189087 14.7004327774048\\
6.3497166633606 14.7004318237305\\
6.44971656799316 14.7004337310791\\
6.54971647262573 14.7004356384277\\
6.6497163772583 14.7004356384277\\
6.74971628189087 14.7004365921021\\
6.8497166633606 14.7004337310791\\
6.94971656799316 14.7004346847534\\
7.04971647262573 14.7004346847534\\
7.1497163772583 14.7004346847534\\
7.24971628189087 14.7004346847534\\
7.3497166633606 14.7004346847534\\
7.44971656799316 14.7004346847534\\
7.54971647262573 14.7004346847534\\
7.6497163772583 14.7004346847534\\
7.74971628189087 14.7004346847534\\
7.8497166633606 14.7004346847534\\
7.94971656799316 14.7004346847534\\
8.04971599578857 14.7004346847534\\
8.1497163772583 14.7004346847534\\
8.24971675872803 14.7004346847534\\
8.34971618652344 14.7004346847534\\
8.44971656799316 14.7004346847534\\
8.54971599578857 14.7004346847534\\
8.6497163772583 14.7004346847534\\
8.74971675872803 14.7004346847534\\
8.84971618652344 14.7004346847534\\
8.94971656799316 14.7004346847534\\
9.04971599578857 14.7004346847534\\
9.1497163772583 14.7004346847534\\
9.24971675872803 14.7004346847534\\
9.34971618652344 14.7004346847534\\
9.44971656799316 14.7004346847534\\
9.54971599578857 14.7004346847534\\
9.6497163772583 14.7004346847534\\
9.74971675872803 14.7004346847534\\
9.84971618652344 14.7004346847534\\
9.94971656799316 14.7004346847534\\
10.0497159957886 14.7004346847534\\
10.1497163772583 14.7004346847534\\
10.249716758728 14.7004346847534\\
10.3497161865234 14.7004346847534\\
10.4497165679932 14.7004346847534\\
10.5497159957886 14.7004346847534\\
10.6497163772583 14.7004346847534\\
10.749716758728 14.7004346847534\\
10.8497161865234 14.7004346847534\\
10.9497165679932 14.7004346847534\\
11.0497159957886 14.7004346847534\\
11.1497163772583 14.7004346847534\\
11.249716758728 14.7004346847534\\
11.3497161865234 14.7004346847534\\
11.4497165679932 14.7004346847534\\
11.5497159957886 14.7004346847534\\
11.6497163772583 14.7004346847534\\
11.749716758728 14.7004346847534\\
11.8497161865234 14.7004346847534\\
11.9497165679932 14.7004346847534\\
12.0497159957886 14.7004346847534\\
12.1497163772583 14.7004346847534\\
12.249716758728 14.7004346847534\\
12.3471450805664 14.7004346847534\\
12.4250917434692 14.7004346847534\\
12.4879579544067 14.7004356384277\\
12.557110786438 14.7004337310791\\
12.6331787109375 14.7004346847534\\
12.7168531417847 14.7004346847534\\
12.808895111084 14.7004346847534\\
12.9088954925537 14.7004346847534\\
13.0088949203491 14.7004346847534\\
13.1088953018188 14.7004346847534\\
13.2088947296143 14.7004346847534\\
13.308895111084 14.7004346847534\\
13.4088954925537 14.7004346847534\\
13.5088949203491 14.7004346847534\\
13.6088953018188 14.7004346847534\\
13.7088947296143 14.7004346847534\\
13.808895111084 14.7004346847534\\
13.9088954925537 14.7004346847534\\
14.0088949203491 14.7004346847534\\
14.1088953018188 14.7004346847534\\
14.2088947296143 14.7004346847534\\
14.308895111084 14.7004346847534\\
14.4088954925537 14.7004346847534\\
14.5088949203491 14.7004346847534\\
14.6088953018188 14.7004346847534\\
14.7088947296143 14.7004346847534\\
14.808895111084 14.7004346847534\\
14.9088954925537 14.7004346847534\\
15.0088949203491 14.7004346847534\\
15.1088953018188 14.7004346847534\\
15.2088947296143 14.7004346847534\\
15.308895111084 14.7004346847534\\
15.4088954925537 14.7004346847534\\
15.5088949203491 14.7004346847534\\
15.6088953018188 14.7004346847534\\
15.7088947296143 14.7004346847534\\
15.808895111084 14.7004346847534\\
15.9088954925537 14.7004346847534\\
16.0088958740234 14.7004346847534\\
16.1088943481445 14.7004346847534\\
16.2088947296143 14.7004346847534\\
16.308895111084 14.7004346847534\\
16.4088954925537 14.7004346847534\\
16.5088958740234 14.7004346847534\\
16.6088943481445 14.7004346847534\\
16.7088947296143 14.7004346847534\\
16.808895111084 14.7004346847534\\
16.9088954925537 14.7004346847534\\
17.0088958740234 14.7004346847534\\
17.1088943481445 14.7004346847534\\
17.2088947296143 14.7004346847534\\
17.308895111084 14.7004346847534\\
17.4088954925537 14.7004346847534\\
17.5088958740234 14.7004346847534\\
17.6088943481445 14.7004346847534\\
17.7088947296143 14.7004346847534\\
17.808895111084 14.7004346847534\\
17.9088954925537 14.7004346847534\\
18.0088958740234 14.7004346847534\\
18.1088943481445 14.7004346847534\\
18.2088947296143 14.7004346847534\\
18.308895111084 14.7004337310791\\
18.4088954925537 14.7004346847534\\
18.5088958740234 14.7004327774048\\
18.6088943481445 14.7004327774048\\
18.7088947296143 14.7004346847534\\
18.808895111084 14.7004356384277\\
18.9088954925537 14.7004356384277\\
19.0088958740234 14.7004356384277\\
19.1088943481445 14.7004346847534\\
19.2088947296143 14.7004346847534\\
19.308895111084 14.7004346847534\\
19.4088954925537 14.7004346847534\\
19.5088958740234 14.7004346847534\\
19.6088943481445 14.7004346847534\\
19.7088947296143 14.7004346847534\\
19.808895111084 14.7004346847534\\
19.9088954925537 14.7004346847534\\
20.0088958740234 14.7004346847534\\
20.1088943481445 14.7004346847534\\
20.2088947296143 14.7004346847534\\
20.308895111084 14.7004346847534\\
20.4088954925537 14.7004346847534\\
20.5088958740234 14.7004346847534\\
20.6088943481445 14.7004346847534\\
20.7088947296143 14.7004346847534\\
20.808895111084 14.7004346847534\\
20.9088954925537 14.7004346847534\\
21.0088958740234 14.7004346847534\\
21.1088943481445 14.7004346847534\\
21.2088947296143 14.7004346847534\\
21.308895111084 14.7004346847534\\
21.4088954925537 14.7004346847534\\
21.5088958740234 14.7004346847534\\
21.6088943481445 14.7004346847534\\
21.7088947296143 14.7004346847534\\
21.808895111084 14.7004346847534\\
21.9088954925537 14.7004346847534\\
22.0088958740234 14.7004346847534\\
22.1088943481445 14.7004346847534\\
22.2088947296143 14.7004346847534\\
22.308895111084 14.7004346847534\\
22.4088954925537 14.7004346847534\\
22.5088958740234 14.7004346847534\\
22.6088943481445 14.7004346847534\\
22.7088947296143 14.7004346847534\\
22.808895111084 14.7004346847534\\
22.9088954925537 14.7004346847534\\
23.0088958740234 14.7004346847534\\
23.1088943481445 14.7004346847534\\
23.2088947296143 14.7004346847534\\
23.308895111084 14.7004346847534\\
23.4088954925537 14.7004346847534\\
23.5088958740234 14.7004346847534\\
23.6088943481445 14.7004346847534\\
23.7088947296143 14.7004346847534\\
23.808895111084 14.7004346847534\\
23.9088954925537 14.7004346847534\\
24.0088958740234 14.7004346847534\\
24.1088943481445 14.7004346847534\\
24.2088947296143 14.7004346847534\\
24.308895111084 14.7004346847534\\
24.4088954925537 14.7004346847534\\
24.5088958740234 14.7004346847534\\
24.6048831939697 14.7004346847534\\
24.681676864624 14.7004346847534\\
24.7464084625244 14.7004346847534\\
24.8176136016846 14.7004346847534\\
24.895938873291 14.7004346847534\\
24.9820976257324 14.7004346847534\\
25.0768718719482 14.7004346847534\\
25.176872253418 14.7004346847534\\
25.2768707275391 14.7004346847534\\
25.3768711090088 14.7004346847534\\
25.4768714904785 14.7004346847534\\
25.5768718719482 14.7004346847534\\
25.676872253418 14.7004346847534\\
25.7768707275391 14.7004346847534\\
25.8768711090088 14.7004346847534\\
25.9768714904785 14.7004346847534\\
26.0768718719482 14.7004346847534\\
26.176872253418 14.7004346847534\\
26.2768707275391 14.7004346847534\\
26.3768711090088 14.7004346847534\\
26.4768714904785 14.7004346847534\\
26.5768718719482 14.7004346847534\\
26.676872253418 14.7004346847534\\
26.7768707275391 14.7004346847534\\
26.8768711090088 14.7004346847534\\
26.9768714904785 14.7004346847534\\
27.0768718719482 14.7004346847534\\
27.176872253418 14.7004346847534\\
27.2768707275391 14.7004346847534\\
27.3768711090088 14.7004346847534\\
27.4768714904785 14.7004346847534\\
27.5768718719482 14.7004346847534\\
27.676872253418 14.7004346847534\\
27.7768707275391 14.7004346847534\\
27.8768711090088 14.7004346847534\\
27.9768714904785 14.7004346847534\\
28.0768718719482 14.7004346847534\\
28.176872253418 14.7004346847534\\
28.2768707275391 14.7004346847534\\
28.3768711090088 14.7004346847534\\
28.4768714904785 14.7004346847534\\
28.5768718719482 14.7004346847534\\
28.676872253418 14.7004346847534\\
28.7768707275391 14.7004346847534\\
28.8768711090088 14.7004346847534\\
28.9768714904785 14.7004346847534\\
29.0768718719482 14.7004346847534\\
29.176872253418 14.7004346847534\\
29.2768707275391 14.7004346847534\\
29.3768711090088 14.7004346847534\\
29.4768714904785 14.7004346847534\\
29.5768718719482 14.7004346847534\\
29.676872253418 14.7004346847534\\
29.7768707275391 14.7004346847534\\
29.8768711090088 14.7004346847534\\
29.9768714904785 14.7004346847534\\
30.0768718719482 14.7004346847534\\
30.176872253418 14.7004346847534\\
30.2768707275391 14.7004346847534\\
30.3768711090088 14.7004346847534\\
30.4768714904785 14.7004346847534\\
30.5768718719482 14.7004337310791\\
30.676872253418 14.7004346847534\\
30.7768707275391 14.7004318237305\\
30.8768711090088 14.7004337310791\\
30.9768714904785 14.7004346847534\\
31.0768718719482 14.7004356384277\\
31.176872253418 14.7004346847534\\
31.2768707275391 14.7004346847534\\
31.3768711090088 14.7004346847534\\
31.4768714904785 14.7004346847534\\
31.5768718719482 14.7004346847534\\
31.676872253418 14.7004346847534\\
31.7768707275391 14.7004346847534\\
31.8768711090088 14.7004346847534\\
31.9768714904785 14.7004346847534\\
32.0768699645996 14.7004346847534\\
32.176872253418 14.7004346847534\\
32.2768707275391 14.7004346847534\\
32.3768730163574 14.7004346847534\\
32.4768714904785 14.7004346847534\\
32.5768699645996 14.7004346847534\\
32.676872253418 14.7004346847534\\
32.7768707275391 14.7004346847534\\
32.8768730163574 14.7004346847534\\
32.9768714904785 14.7004346847534\\
33.0768699645996 14.7004346847534\\
33.176872253418 14.7004346847534\\
33.2768707275391 14.7004346847534\\
33.3768730163574 14.7004346847534\\
33.4768714904785 14.7004346847534\\
33.5768699645996 14.7004346847534\\
33.676872253418 14.7004346847534\\
33.7768707275391 14.7004346847534\\
33.8768730163574 14.7004346847534\\
33.9768714904785 14.7004346847534\\
34.0768699645996 14.7004346847534\\
34.176872253418 14.7004346847534\\
34.2768707275391 14.7004346847534\\
34.3768730163574 14.7004346847534\\
34.4768714904785 14.7004346847534\\
34.5768699645996 14.7004346847534\\
34.676872253418 14.7004346847534\\
34.7768707275391 14.7004346847534\\
34.8768730163574 14.7004346847534\\
34.9768714904785 14.7004346847534\\
35.0768699645996 14.7004346847534\\
35.176872253418 14.7004346847534\\
35.2768707275391 14.7004346847534\\
35.3768730163574 14.7004346847534\\
35.4768714904785 14.7004346847534\\
35.5768699645996 14.7004346847534\\
35.676872253418 14.7004346847534\\
35.7768707275391 14.7004346847534\\
35.8768730163574 14.7004346847534\\
35.9768714904785 14.7004346847534\\
36.0768699645996 14.7004346847534\\
36.176872253418 14.7004346847534\\
36.2768707275391 14.7004346847534\\
36.3768730163574 14.7004346847534\\
36.4768714904785 14.7004346847534\\
36.5768699645996 14.7004346847534\\
36.676872253418 14.7004346847534\\
36.7768707275391 14.7004346847534\\
36.8673248291016 14.7004346847534\\
36.9396858215332 14.7004346847534\\
37.0192794799805 14.7004346847534\\
37.1068305969238 14.7004346847534\\
37.2031402587891 14.7004346847534\\
37.3031425476074 14.7004346847534\\
37.4031410217285 14.7004346847534\\
37.5031433105469 14.7004346847534\\
37.603141784668 14.7004346847534\\
37.7031402587891 14.7004346847534\\
37.8031425476074 14.7004346847534\\
37.9031410217285 14.7004346847534\\
38.0031433105469 14.7004346847534\\
38.103141784668 14.7004346847534\\
38.2031402587891 14.7004346847534\\
38.3031425476074 14.7004346847534\\
38.4031410217285 14.7004346847534\\
38.5031433105469 14.7004346847534\\
38.603141784668 14.7004346847534\\
38.7031402587891 14.7004346847534\\
38.8031425476074 14.7004346847534\\
38.9031410217285 14.7004346847534\\
39.0031433105469 14.7004346847534\\
39.103141784668 14.7004346847534\\
39.2031402587891 14.7004346847534\\
39.3031425476074 14.7004346847534\\
39.4031410217285 14.7004346847534\\
39.5031433105469 14.7004346847534\\
39.603141784668 14.7004346847534\\
39.7031402587891 14.7004346847534\\
39.8031425476074 14.7004346847534\\
39.9031410217285 14.7004346847534\\
40.0031433105469 14.7004346847534\\
40.103141784668 14.7004346847534\\
40.2031402587891 14.7004346847534\\
40.3031425476074 14.7004346847534\\
40.4031410217285 14.7004346847534\\
40.5031433105469 14.7004346847534\\
40.603141784668 14.7004346847534\\
40.7031402587891 14.7004346847534\\
40.8031425476074 14.7004346847534\\
40.9031410217285 14.7004346847534\\
41.0031433105469 14.7004346847534\\
41.103141784668 14.7004346847534\\
41.2031402587891 14.7004346847534\\
41.3031425476074 14.7004346847534\\
41.4031410217285 14.7004346847534\\
41.5031433105469 14.7004346847534\\
41.603141784668 14.7004346847534\\
41.7031402587891 14.7004346847534\\
41.8031425476074 14.7004346847534\\
41.9031410217285 14.7004346847534\\
42.0031433105469 14.7004346847534\\
42.103141784668 14.7004346847534\\
42.2031402587891 14.7004346847534\\
42.3031425476074 14.7004346847534\\
42.4031410217285 14.7004346847534\\
42.5031433105469 14.7004346847534\\
42.603141784668 14.7004346847534\\
42.7031402587891 14.7004346847534\\
42.8031425476074 14.7004337310791\\
42.9031410217285 14.7004337310791\\
43.0031433105469 14.7004327774048\\
43.103141784668 14.7004327774048\\
43.2031402587891 14.7004346847534\\
43.3031425476074 14.7004356384277\\
43.4031410217285 14.7004356384277\\
43.5031433105469 14.7004346847534\\
43.603141784668 14.7004346847534\\
43.7031402587891 14.7004346847534\\
43.8031425476074 14.7004346847534\\
43.9031410217285 14.7004346847534\\
44.0031433105469 14.7004346847534\\
44.103141784668 14.7004346847534\\
44.2031402587891 14.7004346847534\\
44.3031425476074 14.7004346847534\\
44.4031410217285 14.7004346847534\\
44.5031433105469 14.7004346847534\\
44.603141784668 14.7004346847534\\
44.7031402587891 14.7004346847534\\
44.8031425476074 14.7004346847534\\
44.9031410217285 14.7004346847534\\
45.0031433105469 14.7004346847534\\
45.103141784668 14.7004346847534\\
45.2031402587891 14.7004346847534\\
45.3031425476074 14.7004346847534\\
45.4031410217285 14.7004346847534\\
45.5031433105469 14.7004346847534\\
45.603141784668 14.7004346847534\\
45.7031402587891 14.7004346847534\\
45.8031425476074 14.7004346847534\\
45.9031410217285 14.7004346847534\\
46.0031433105469 14.7004346847534\\
46.103141784668 14.7004346847534\\
46.2031402587891 14.7004346847534\\
46.3031425476074 14.7004346847534\\
46.4031410217285 14.7004346847534\\
46.5031433105469 14.7004346847534\\
46.603141784668 14.7004346847534\\
46.7031402587891 14.7004346847534\\
46.8031425476074 14.7004346847534\\
46.9031410217285 14.7004346847534\\
47.0031433105469 14.7004346847534\\
47.103141784668 14.7004346847534\\
47.2031402587891 14.7004346847534\\
47.3031425476074 14.7004346847534\\
47.4031410217285 14.7004346847534\\
47.5031433105469 14.7004346847534\\
47.603141784668 14.7004346847534\\
47.7031402587891 14.7004346847534\\
47.8031425476074 14.7004346847534\\
47.9031410217285 14.7004346847534\\
48.0031433105469 14.7004346847534\\
48.103141784668 14.7004346847534\\
48.2031402587891 14.7004346847534\\
48.3031425476074 14.7004346847534\\
48.4031410217285 14.7004346847534\\
48.5031433105469 14.7004346847534\\
48.603141784668 14.7004346847534\\
48.7031402587891 14.7004346847534\\
48.8031425476074 14.7004346847534\\
48.9031410217285 14.7004346847534\\
49.0031433105469 14.7004337310791\\
49.0980262756348 14.7004337310791\\
49.1739387512207 14.7004337310791\\
49.2413482666016 14.7004337310791\\
49.315502166748 14.7004346847534\\
49.3970718383789 14.7004346847534\\
49.4867973327637 14.7004356384277\\
49.5854949951172 14.7004356384277\\
49.6854934692383 14.7004356384277\\
49.7854957580566 14.7004356384277\\
49.8854942321777 14.7004356384277\\
49.9854927062988 14.7004356384277\\
50.0854949951172 14.7004356384277\\
50.1854934692383 14.7004356384277\\
50.2854957580566 14.7004356384277\\
50.3854942321777 14.7004356384277\\
50.4854927062988 14.7004356384277\\
50.5854949951172 14.7004356384277\\
50.6854934692383 14.7004356384277\\
50.7854957580566 14.7004356384277\\
50.8854942321777 14.7004356384277\\
50.9854927062988 14.7004356384277\\
51.0854949951172 14.7004356384277\\
51.1854934692383 14.7004356384277\\
51.2854957580566 14.7004356384277\\
51.3854942321777 14.7004356384277\\
51.4854927062988 14.7004356384277\\
51.5854949951172 14.7004356384277\\
51.6854934692383 14.7004356384277\\
51.7854957580566 14.7004356384277\\
51.8854942321777 14.7004356384277\\
51.9854927062988 14.7004356384277\\
52.0854949951172 14.7004356384277\\
52.1854934692383 14.7004356384277\\
52.2854957580566 14.7004356384277\\
52.3854942321777 14.7004356384277\\
52.4854927062988 14.7004356384277\\
52.5854949951172 14.7004356384277\\
52.6854934692383 14.7004356384277\\
52.7854957580566 14.7004356384277\\
52.8854942321777 14.7004356384277\\
52.9854927062988 14.7004356384277\\
53.0854949951172 14.7004356384277\\
53.1854934692383 14.7004356384277\\
53.2854957580566 14.7004356384277\\
53.3854942321777 14.7004356384277\\
53.4854927062988 14.7004356384277\\
53.5854949951172 14.7004356384277\\
53.6854934692383 14.7004356384277\\
53.7854957580566 14.7004356384277\\
53.8854942321777 14.7004356384277\\
53.9854927062988 14.7004356384277\\
54.0854949951172 14.7004356384277\\
54.1854934692383 14.7004356384277\\
54.2854957580566 14.7004356384277\\
54.3854942321777 14.7004356384277\\
54.4854927062988 14.7004356384277\\
54.5854949951172 14.7004365921021\\
54.6854934692383 14.7004404067993\\
54.7854957580566 14.7004556655884\\
54.8854942321777 14.700532913208\\
54.9854927062988 14.7009401321411\\
55.0854949951172 14.7029867172241\\
55.1854934692383 14.7104139328003\\
55.2854957580566 14.7241764068604\\
55.3854942321777 14.7336578369141\\
55.4854927062988 14.7304201126099\\
55.5854949951172 14.7332744598389\\
55.6854934692383 14.7320175170898\\
55.7854957580566 14.7331829071045\\
55.8854942321777 14.7332029342651\\
55.9854927062988 14.7340717315674\\
56.0854949951172 14.7343788146973\\
56.1854934692383 14.7350587844849\\
56.2854957580566 14.7354907989502\\
56.3854942321777 14.7360887527466\\
56.4854927062988 14.7365760803223\\
56.5854949951172 14.7371368408203\\
56.6854934692383 14.7376489639282\\
56.7854957580566 14.7381944656372\\
56.8854942321777 14.7387170791626\\
56.9854927062988 14.7392559051514\\
57.0854949951172 14.7397842407227\\
57.1854934692383 14.7403192520142\\
57.2854957580566 14.7408504486084\\
57.3854942321777 14.7413845062256\\
57.4854927062988 14.7419166564941\\
57.5854949951172 14.7424507141113\\
57.6854934692383 14.7429838180542\\
57.7854957580566 14.7435188293457\\
57.8854942321777 14.7442932128906\\
57.9854927062988 14.7449750900269\\
58.0854949951172 14.7456140518188\\
58.1854934692383 14.7462558746338\\
58.2854957580566 14.7468938827515\\
58.3854942321777 14.7475328445435\\
58.4854927062988 14.7481689453125\\
58.5854949951172 14.7488050460815\\
58.6854934692383 14.7494411468506\\
58.7854957580566 14.7500762939453\\
58.8854942321777 14.7507104873657\\
58.9854927062988 14.7513456344604\\
59.0854949951172 14.7519798278809\\
59.1854934692383 14.7526140213013\\
59.2854957580566 14.753249168396\\
59.3854942321777 14.7537879943848\\
59.4854927062988 14.75412940979\\
59.5854949951172 14.7547292709351\\
59.6854934692383 14.7552442550659\\
59.7854957580566 14.7557859420776\\
59.8854942321777 14.7563152313232\\
59.9427452087402 14.756739616394\\
60 14.7571563720703\\
60.0629768371582 14.7574405670166\\
60.1322555541992 14.7577753067017\\
60.2084579467773 14.7581338882446\\
60.2922821044922 14.7585296630859\\
60.3844909667969 14.758960723877\\
60.484489440918 14.7594938278198\\
60.5844879150391 14.7601346969604\\
60.6844902038574 14.760760307312\\
60.7844886779785 14.7613649368286\\
60.8844909667969 14.7619905471802\\
60.984489440918 14.7626438140869\\
61.0844879150391 14.7633419036865\\
61.1844902038574 14.7641124725342\\
61.2844886779785 14.7698860168457\\
61.3734016418457 14.774582862854\\
61.444522857666 14.7739067077637\\
61.5227546691895 14.76819896698\\
61.6088104248047 14.7685995101929\\
61.7034721374512 14.769082069397\\
61.8034706115723 14.7698421478271\\
61.9034729003906 14.7707986831665\\
62.0034713745117 14.7718315124512\\
62.1034736633301 14.772873878479\\
62.2034721374512 14.7739362716675\\
62.3034706115723 14.7750110626221\\
62.4034729003906 14.7760992050171\\
62.5034713745117 14.7771892547607\\
62.6034736633301 14.7782907485962\\
62.7034721374512 14.7793970108032\\
62.8034706115723 14.7805061340332\\
62.9034729003906 14.7816209793091\\
63.0034713745117 14.7827339172363\\
63.1034736633301 14.7838468551636\\
63.2034721374512 14.7849607467651\\
63.3034706115723 14.7860746383667\\
63.4034729003906 14.7871885299683\\
63.5034713745117 14.7883024215698\\
63.6034736633301 14.7894163131714\\
63.7034721374512 14.7905302047729\\
63.8034706115723 14.7916440963745\\
63.9034729003906 14.7927579879761\\
64.0034713745117 14.7938718795776\\
64.1034698486328 14.7949867248535\\
64.2034759521484 14.7961015701294\\
64.3034744262695 14.797215461731\\
64.4034729003906 14.7983303070068\\
64.5034713745117 14.7994451522827\\
64.6034698486328 14.8005599975586\\
64.7034759521484 14.8016748428345\\
64.8034744262695 14.8027906417847\\
64.9034729003906 14.8039054870605\\
65.0034713745117 14.8050212860107\\
65.1034698486328 14.8061370849609\\
65.2034759521484 14.8072528839111\\
65.3034744262695 14.8083686828613\\
65.4034729003906 14.8094844818115\\
65.5034713745117 14.8106002807617\\
65.6034698486328 14.8117170333862\\
65.7034759521484 14.812840461731\\
65.8034744262695 14.813943862915\\
65.9034729003906 14.815071105957\\
66.0034713745117 14.8161907196045\\
66.1034698486328 14.8173131942749\\
66.2034759521484 14.818434715271\\
66.3034744262695 14.8195571899414\\
66.4034729003906 14.8206787109375\\
66.5034713745117 14.8218011856079\\
66.6034698486328 14.822922706604\\
66.7034759521484 14.8240451812744\\
66.8034744262695 14.8251676559448\\
66.9034729003906 14.8262910842896\\
67.0034713745117 14.8274202346802\\
67.1034698486328 14.828574180603\\
67.2034759521484 14.8298692703247\\
67.3034744262695 14.8318948745728\\
67.4034729003906 14.8365707397461\\
67.5034713745117 14.8438854217529\\
67.6034698486328 14.8451929092407\\
67.7034759521484 14.8404340744019\\
67.8034744262695 14.838041305542\\
67.9034729003906 14.8390321731567\\
68.0034713745117 14.8388967514038\\
68.1034698486328 14.8406400680542\\
68.2034759521484 14.8413429260254\\
68.3034744262695 14.8427448272705\\
68.4034729003906 14.8436832427979\\
68.5034713745117 14.8449296951294\\
68.6034698486328 14.8459720611572\\
68.7034759521484 14.847149848938\\
68.8034744262695 14.848237991333\\
68.9034729003906 14.84938621521\\
69.0034713745117 14.8504943847656\\
69.1034698486328 14.8516292572021\\
69.2034759521484 14.8527460098267\\
69.3034744262695 14.8538751602173\\
69.4034729003906 14.8549966812134\\
69.5034713745117 14.8561229705811\\
69.6034698486328 14.8572463989258\\
69.7034759521484 14.8583717346191\\
69.8034744262695 14.8594961166382\\
69.9034729003906 14.8606204986572\\
70.0034713745117 14.8617448806763\\
70.1034698486328 14.8629884719849\\
70.2034759521484 14.8643188476563\\
70.3034744262695 14.8654069900513\\
70.4034729003906 14.8665466308594\\
70.5034713745117 14.8676671981812\\
70.6034698486328 14.8687944412231\\
70.7034759521484 14.8699207305908\\
70.8034744262695 14.8710470199585\\
70.9034729003906 14.8721742630005\\
71.0034713745117 14.8733015060425\\
71.1034698486328 14.8744297027588\\
71.2034759521484 14.8755578994751\\
71.3034744262695 14.8766860961914\\
71.4034729003906 14.877815246582\\
71.5034713745117 14.878945350647\\
71.6034698486328 14.8800601959229\\
71.7034759521484 14.8808813095093\\
71.8034744262695 14.8820314407349\\
71.9034729003906 14.8831338882446\\
72.0034713745117 14.884256362915\\
72.1034698486328 14.8853702545166\\
72.2034759521484 14.8864870071411\\
72.3034744262695 14.8876008987427\\
72.4034729003906 14.8887147903442\\
72.5034713745117 14.8898277282715\\
72.6034698486328 14.8909387588501\\
72.7034759521484 14.8920831680298\\
72.8034744262695 14.8933057785034\\
72.9034729003906 14.8945093154907\\
73.0034713745117 14.8956632614136\\
73.1034698486328 14.8968191146851\\
73.2034759521484 14.8979921340942\\
73.3034744262695 14.89919090271\\
73.4034729003906 14.9004325866699\\
73.5034713745117 14.9042053222656\\
73.5992889404297 14.9102849960327\\
73.6759490966797 14.912654876709\\
73.7424468994141 14.9046354293823\\
73.8155975341797 14.9050340652466\\
73.8960571289063 14.9056015014648\\
73.9845733642578 14.9062919616699\\
74.0819320678711 14.9071655273438\\
74.1819305419922 14.9082250595093\\
74.2819290161133 14.9093294143677\\
74.3819351196289 14.9104347229004\\
74.48193359375 14.9115438461304\\
74.5819320678711 14.9126577377319\\
74.6819305419922 14.9137744903564\\
74.7819290161133 14.914888381958\\
74.8819351196289 14.9160070419312\\
74.9409637451172 14.9171323776245\\
75 14.9177913665771\\
};
\addlegendentry{18 "};

\addplot [
color=green,
dash pattern=on 1pt off 3pt on 3pt off 3pt
]
table[row sep=crcr]{
0 14.7003479003906\\
9.99999971718069e-10 14.7003479003906\\
2.49999998480632e-09 14.7003479003906\\
4.74999994892755e-09 14.7003479003906\\
8.1250002281763e-09 14.7003479003906\\
1.31874999809156e-08 14.7003479003906\\
2.0781250498203e-08 14.7003479003906\\
3.21718758300449e-08 14.7003479003906\\
4.92578138278077e-08 14.7003479003906\\
7.48867208244519e-08 14.7003479003906\\
1.13330081319418e-07 14.7003479003906\\
1.70995122061868e-07 14.7003479003906\\
2.57492672517401e-07 14.7003479003906\\
3.8723902662241e-07 14.7003479003906\\
5.81858500936505e-07 14.7003479003906\\
8.73787769251066e-07 14.7003479003906\\
1.31168167172291e-06 14.7003479003906\\
1.9685226106958e-06 14.7003479003906\\
2.95378367809462e-06 14.7003479003906\\
4.43167573394021e-06 14.7003479003906\\
6.64851359033491e-06 14.7003479003906\\
9.97376992017962e-06 14.7003479003906\\
1.49616553244414e-05 14.7003479003906\\
2.24434825213393e-05 14.7003479003906\\
3.36662233166862e-05 14.7003479003906\\
5.05003372381907e-05 14.7003479003906\\
7.57515081204474e-05 14.7003479003906\\
0.000113628258986864 14.7003479003906\\
0.000170443381648511 14.7003479003906\\
0.000255666091106832 14.7003488540649\\
0.000383500126190484 14.7003498077393\\
0.000575251178815961 14.7003507614136\\
0.000862877757754177 14.7003536224365\\
0.00129431765526533 14.7003536224365\\
0.00194147753063589 14.7003507614136\\
0.00291221728548408 14.7003469467163\\
0.00436832662671804 14.7003450393677\\
0.00597004732117057 14.700345993042\\
0.00773194013163447 14.7003479003906\\
0.009670021943748 14.7003479003906\\
0.0118019115179777 14.7003479003906\\
0.0141469910740852 14.7003479003906\\
0.0167265776544809 14.7003479003906\\
0.0195641238242388 14.7003479003906\\
0.0226854234933853 14.7003479003906\\
0.0261188540607691 14.7003479003906\\
0.0298956278711557 14.7003479003906\\
0.0340500771999359 14.7003479003906\\
0.0386199727654457 14.7003479003906\\
0.0436468608677387 14.7003479003906\\
0.0491764321923256 14.7003479003906\\
0.0552589632570744 14.7003479003906\\
0.0619497485458851 14.7003479003906\\
0.0693096145987511 14.7003479003906\\
0.0774054601788521 14.7003450393677\\
0.0863108932971954 14.7003498077393\\
0.0961068719625473 14.7003479003906\\
0.106882445514202 14.7003479003906\\
0.118735581636429 14.7003479003906\\
0.13177402317524 14.7003479003906\\
0.146116316318512 14.7003479003906\\
0.161892831325531 14.7003479003906\\
0.179247006773949 14.7003479003906\\
0.198336601257324 14.7003479003906\\
0.21933513879776 14.7003479003906\\
0.242433547973633 14.7003479003906\\
0.267841786146164 14.7003479003906\\
0.29579085111618 14.7003479003906\\
0.32653483748436 14.7003479003906\\
0.360353201627731 14.7003479003906\\
0.397553414106369 14.7003479003906\\
0.438473641872406 14.7003479003906\\
0.483485877513886 14.7003479003906\\
0.532999336719513 14.7003479003906\\
0.587464153766632 14.7003479003906\\
0.647375464439392 14.7003479003906\\
0.71327793598175 14.7003479003906\\
0.7857705950737 14.7003479003906\\
0.865512549877167 14.7003479003906\\
0.953228712081909 14.7003479003906\\
1.04971647262573 14.7003479003906\\
1.14971649646759 14.7003479003906\\
1.24971640110016 14.7003479003906\\
1.34971642494202 14.7003479003906\\
1.44971644878387 14.7003479003906\\
1.54971647262573 14.7003479003906\\
1.64971649646759 14.7003479003906\\
1.74971640110016 14.7003479003906\\
1.84971642494202 14.7003479003906\\
1.94971644878387 14.7003479003906\\
2.04971647262573 14.7003479003906\\
2.1497163772583 14.7003479003906\\
2.24971652030945 14.7003479003906\\
2.34971642494202 14.7003479003906\\
2.44971656799316 14.7003479003906\\
2.54971647262573 14.7003479003906\\
2.6497163772583 14.7003479003906\\
2.74971652030945 14.7003479003906\\
2.84971642494202 14.7003479003906\\
2.94971656799316 14.7003479003906\\
3.04971647262573 14.7003479003906\\
3.1497163772583 14.7003479003906\\
3.24971652030945 14.7003479003906\\
3.34971642494202 14.7003479003906\\
3.44971656799316 14.7003479003906\\
3.54971647262573 14.7003479003906\\
3.6497163772583 14.7003479003906\\
3.74971652030945 14.7003479003906\\
3.84971642494202 14.7003479003906\\
3.94971656799316 14.7003479003906\\
4.04971647262573 14.7003479003906\\
4.1497163772583 14.7003479003906\\
4.24971628189087 14.7003479003906\\
4.3497166633606 14.7003479003906\\
4.44971656799316 14.7003479003906\\
4.54971647262573 14.7003479003906\\
4.6497163772583 14.7003479003906\\
4.74971628189087 14.7003479003906\\
4.8497166633606 14.7003479003906\\
4.94971656799316 14.7003479003906\\
5.04971647262573 14.7003479003906\\
5.1497163772583 14.7003479003906\\
5.24971628189087 14.7003479003906\\
5.3497166633606 14.7003479003906\\
5.44971656799316 14.7003479003906\\
5.54971647262573 14.7003479003906\\
5.6497163772583 14.7003479003906\\
5.74971628189087 14.7003479003906\\
5.8497166633606 14.7003479003906\\
5.94971656799316 14.7003479003906\\
6.04971647262573 14.7003479003906\\
6.1497163772583 14.7003469467163\\
6.24971628189087 14.700345993042\\
6.3497166633606 14.700345993042\\
6.44971656799316 14.7003469467163\\
6.54971647262573 14.7003488540649\\
6.6497163772583 14.7003479003906\\
6.74971628189087 14.7003488540649\\
6.8497166633606 14.7003469467163\\
6.94971656799316 14.7003479003906\\
7.04971647262573 14.7003469467163\\
7.1497163772583 14.7003479003906\\
7.24971628189087 14.7003479003906\\
7.3497166633606 14.7003479003906\\
7.44971656799316 14.7003479003906\\
7.54971647262573 14.7003479003906\\
7.6497163772583 14.7003479003906\\
7.74971628189087 14.7003479003906\\
7.8497166633606 14.7003479003906\\
7.94971656799316 14.7003479003906\\
8.04971599578857 14.7003479003906\\
8.1497163772583 14.7003479003906\\
8.24971675872803 14.7003479003906\\
8.34971618652344 14.7003479003906\\
8.44971656799316 14.7003479003906\\
8.54971599578857 14.7003479003906\\
8.6497163772583 14.7003479003906\\
8.74971675872803 14.7003479003906\\
8.84971618652344 14.7003479003906\\
8.94971656799316 14.7003479003906\\
9.04971599578857 14.7003479003906\\
9.1497163772583 14.7003479003906\\
9.24971675872803 14.7003479003906\\
9.34971618652344 14.7003479003906\\
9.44971656799316 14.7003479003906\\
9.54971599578857 14.7003479003906\\
9.6497163772583 14.7003479003906\\
9.74971675872803 14.7003479003906\\
9.84971618652344 14.7003479003906\\
9.94971656799316 14.7003479003906\\
10.0497159957886 14.7003479003906\\
10.1497163772583 14.7003479003906\\
10.249716758728 14.7003479003906\\
10.3497161865234 14.7003479003906\\
10.4497165679932 14.7003479003906\\
10.5497159957886 14.7003479003906\\
10.6497163772583 14.7003479003906\\
10.749716758728 14.7003479003906\\
10.8497161865234 14.7003479003906\\
10.9497165679932 14.7003479003906\\
11.0497159957886 14.7003479003906\\
11.1497163772583 14.7003479003906\\
11.249716758728 14.7003479003906\\
11.3497161865234 14.7003479003906\\
11.4497165679932 14.7003479003906\\
11.5497159957886 14.7003479003906\\
11.6497163772583 14.7003479003906\\
11.749716758728 14.7003479003906\\
11.8497161865234 14.7003479003906\\
11.9497165679932 14.7003479003906\\
12.0497159957886 14.7003479003906\\
12.1497163772583 14.7003479003906\\
12.249716758728 14.7003479003906\\
12.3471450805664 14.7003479003906\\
12.4250917434692 14.7003479003906\\
12.4879579544067 14.7003488540649\\
12.557110786438 14.7003469467163\\
12.6331787109375 14.7003479003906\\
12.7168531417847 14.7003479003906\\
12.808895111084 14.7003479003906\\
12.9088954925537 14.7003479003906\\
13.0088949203491 14.7003479003906\\
13.1088953018188 14.7003479003906\\
13.2088947296143 14.7003479003906\\
13.308895111084 14.7003479003906\\
13.4088954925537 14.7003479003906\\
13.5088949203491 14.7003479003906\\
13.6088953018188 14.7003479003906\\
13.7088947296143 14.7003479003906\\
13.808895111084 14.7003479003906\\
13.9088954925537 14.7003479003906\\
14.0088949203491 14.7003479003906\\
14.1088953018188 14.7003479003906\\
14.2088947296143 14.7003479003906\\
14.308895111084 14.7003479003906\\
14.4088954925537 14.7003479003906\\
14.5088949203491 14.7003479003906\\
14.6088953018188 14.7003479003906\\
14.7088947296143 14.7003479003906\\
14.808895111084 14.7003479003906\\
14.9088954925537 14.7003479003906\\
15.0088949203491 14.7003479003906\\
15.1088953018188 14.7003479003906\\
15.2088947296143 14.7003479003906\\
15.308895111084 14.7003479003906\\
15.4088954925537 14.7003479003906\\
15.5088949203491 14.7003479003906\\
15.6088953018188 14.7003479003906\\
15.7088947296143 14.7003479003906\\
15.808895111084 14.7003479003906\\
15.9088954925537 14.7003479003906\\
16.0088958740234 14.7003479003906\\
16.1088943481445 14.7003479003906\\
16.2088947296143 14.7003479003906\\
16.308895111084 14.7003479003906\\
16.4088954925537 14.7003479003906\\
16.5088958740234 14.7003479003906\\
16.6088943481445 14.7003479003906\\
16.7088947296143 14.7003479003906\\
16.808895111084 14.7003479003906\\
16.9088954925537 14.7003479003906\\
17.0088958740234 14.7003479003906\\
17.1088943481445 14.7003479003906\\
17.2088947296143 14.7003479003906\\
17.308895111084 14.7003479003906\\
17.4088954925537 14.7003479003906\\
17.5088958740234 14.7003479003906\\
17.6088943481445 14.7003479003906\\
17.7088947296143 14.7003479003906\\
17.808895111084 14.7003479003906\\
17.9088954925537 14.7003479003906\\
18.0088958740234 14.7003479003906\\
18.1088943481445 14.7003479003906\\
18.2088947296143 14.7003479003906\\
18.308895111084 14.7003469467163\\
18.4088954925537 14.7003479003906\\
18.5088958740234 14.700345993042\\
18.6088943481445 14.700345993042\\
18.7088947296143 14.7003479003906\\
18.808895111084 14.7003479003906\\
18.9088954925537 14.7003488540649\\
19.0088958740234 14.7003479003906\\
19.1088943481445 14.7003479003906\\
19.2088947296143 14.7003479003906\\
19.308895111084 14.7003479003906\\
19.4088954925537 14.7003479003906\\
19.5088958740234 14.7003479003906\\
19.6088943481445 14.7003479003906\\
19.7088947296143 14.7003479003906\\
19.808895111084 14.7003479003906\\
19.9088954925537 14.7003479003906\\
20.0088958740234 14.7003479003906\\
20.1088943481445 14.7003479003906\\
20.2088947296143 14.7003479003906\\
20.308895111084 14.7003479003906\\
20.4088954925537 14.7003479003906\\
20.5088958740234 14.7003479003906\\
20.6088943481445 14.7003479003906\\
20.7088947296143 14.7003479003906\\
20.808895111084 14.7003479003906\\
20.9088954925537 14.7003479003906\\
21.0088958740234 14.7003479003906\\
21.1088943481445 14.7003479003906\\
21.2088947296143 14.7003479003906\\
21.308895111084 14.7003479003906\\
21.4088954925537 14.7003479003906\\
21.5088958740234 14.7003479003906\\
21.6088943481445 14.7003479003906\\
21.7088947296143 14.7003479003906\\
21.808895111084 14.7003479003906\\
21.9088954925537 14.7003479003906\\
22.0088958740234 14.7003479003906\\
22.1088943481445 14.7003479003906\\
22.2088947296143 14.7003479003906\\
22.308895111084 14.7003479003906\\
22.4088954925537 14.7003479003906\\
22.5088958740234 14.7003479003906\\
22.6088943481445 14.7003479003906\\
22.7088947296143 14.7003479003906\\
22.808895111084 14.7003479003906\\
22.9088954925537 14.7003479003906\\
23.0088958740234 14.7003479003906\\
23.1088943481445 14.7003479003906\\
23.2088947296143 14.7003479003906\\
23.308895111084 14.7003479003906\\
23.4088954925537 14.7003479003906\\
23.5088958740234 14.7003479003906\\
23.6088943481445 14.7003479003906\\
23.7088947296143 14.7003479003906\\
23.808895111084 14.7003479003906\\
23.9088954925537 14.7003479003906\\
24.0088958740234 14.7003479003906\\
24.1088943481445 14.7003479003906\\
24.2088947296143 14.7003479003906\\
24.308895111084 14.7003479003906\\
24.4088954925537 14.7003479003906\\
24.5088958740234 14.7003479003906\\
24.6048831939697 14.7003479003906\\
24.681676864624 14.7003479003906\\
24.7464084625244 14.7003479003906\\
24.8176136016846 14.7003479003906\\
24.895938873291 14.7003479003906\\
24.9820976257324 14.7003479003906\\
25.0768718719482 14.7003479003906\\
25.176872253418 14.7003479003906\\
25.2768707275391 14.7003479003906\\
25.3768711090088 14.7003479003906\\
25.4768714904785 14.7003479003906\\
25.5768718719482 14.7003479003906\\
25.676872253418 14.7003479003906\\
25.7768707275391 14.7003479003906\\
25.8768711090088 14.7003479003906\\
25.9768714904785 14.7003479003906\\
26.0768718719482 14.7003479003906\\
26.176872253418 14.7003479003906\\
26.2768707275391 14.7003479003906\\
26.3768711090088 14.7003479003906\\
26.4768714904785 14.7003479003906\\
26.5768718719482 14.7003479003906\\
26.676872253418 14.7003479003906\\
26.7768707275391 14.7003479003906\\
26.8768711090088 14.7003479003906\\
26.9768714904785 14.7003479003906\\
27.0768718719482 14.7003479003906\\
27.176872253418 14.7003479003906\\
27.2768707275391 14.7003479003906\\
27.3768711090088 14.7003479003906\\
27.4768714904785 14.7003479003906\\
27.5768718719482 14.7003479003906\\
27.676872253418 14.7003479003906\\
27.7768707275391 14.7003479003906\\
27.8768711090088 14.7003479003906\\
27.9768714904785 14.7003479003906\\
28.0768718719482 14.7003479003906\\
28.176872253418 14.7003479003906\\
28.2768707275391 14.7003479003906\\
28.3768711090088 14.7003479003906\\
28.4768714904785 14.7003479003906\\
28.5768718719482 14.7003479003906\\
28.676872253418 14.7003479003906\\
28.7768707275391 14.7003479003906\\
28.8768711090088 14.7003479003906\\
28.9768714904785 14.7003479003906\\
29.0768718719482 14.7003479003906\\
29.176872253418 14.7003479003906\\
29.2768707275391 14.7003479003906\\
29.3768711090088 14.7003479003906\\
29.4768714904785 14.7003479003906\\
29.5768718719482 14.7003479003906\\
29.676872253418 14.7003479003906\\
29.7768707275391 14.7003479003906\\
29.8768711090088 14.7003479003906\\
29.9768714904785 14.7003479003906\\
30.0768718719482 14.7003479003906\\
30.176872253418 14.7003479003906\\
30.2768707275391 14.7003479003906\\
30.3768711090088 14.7003479003906\\
30.4768714904785 14.7003469467163\\
30.5768718719482 14.7003469467163\\
30.676872253418 14.7003469467163\\
30.7768707275391 14.700345993042\\
30.8768711090088 14.7003469467163\\
30.9768714904785 14.7003479003906\\
31.0768718719482 14.7003488540649\\
31.176872253418 14.7003479003906\\
31.2768707275391 14.7003479003906\\
31.3768711090088 14.7003479003906\\
31.4768714904785 14.7003479003906\\
31.5768718719482 14.7003479003906\\
31.676872253418 14.7003479003906\\
31.7768707275391 14.7003479003906\\
31.8768711090088 14.7003479003906\\
31.9768714904785 14.7003479003906\\
32.0768699645996 14.7003479003906\\
32.176872253418 14.7003479003906\\
32.2768707275391 14.7003479003906\\
32.3768730163574 14.7003479003906\\
32.4768714904785 14.7003479003906\\
32.5768699645996 14.7003479003906\\
32.676872253418 14.7003479003906\\
32.7768707275391 14.7003479003906\\
32.8768730163574 14.7003479003906\\
32.9768714904785 14.7003479003906\\
33.0768699645996 14.7003479003906\\
33.176872253418 14.7003479003906\\
33.2768707275391 14.7003479003906\\
33.3768730163574 14.7003479003906\\
33.4768714904785 14.7003479003906\\
33.5768699645996 14.7003479003906\\
33.676872253418 14.7003479003906\\
33.7768707275391 14.7003479003906\\
33.8768730163574 14.7003479003906\\
33.9768714904785 14.7003479003906\\
34.0768699645996 14.7003479003906\\
34.176872253418 14.7003479003906\\
34.2768707275391 14.7003479003906\\
34.3768730163574 14.7003479003906\\
34.4768714904785 14.7003479003906\\
34.5768699645996 14.7003479003906\\
34.676872253418 14.7003479003906\\
34.7768707275391 14.7003479003906\\
34.8768730163574 14.7003479003906\\
34.9768714904785 14.7003479003906\\
35.0768699645996 14.7003479003906\\
35.176872253418 14.7003479003906\\
35.2768707275391 14.7003479003906\\
35.3768730163574 14.7003479003906\\
35.4768714904785 14.7003479003906\\
35.5768699645996 14.7003479003906\\
35.676872253418 14.7003479003906\\
35.7768707275391 14.7003479003906\\
35.8768730163574 14.7003479003906\\
35.9768714904785 14.7003479003906\\
36.0768699645996 14.7003479003906\\
36.176872253418 14.7003479003906\\
36.2768707275391 14.7003479003906\\
36.3768730163574 14.7003479003906\\
36.4768714904785 14.7003479003906\\
36.5768699645996 14.7003479003906\\
36.676872253418 14.7003479003906\\
36.7768707275391 14.7003479003906\\
36.8673248291016 14.7003479003906\\
36.9396858215332 14.7003479003906\\
37.0192794799805 14.7003479003906\\
37.1068305969238 14.7003479003906\\
37.2031402587891 14.7003479003906\\
37.3031425476074 14.7003479003906\\
37.4031410217285 14.7003479003906\\
37.5031433105469 14.7003479003906\\
37.603141784668 14.7003479003906\\
37.7031402587891 14.7003479003906\\
37.8031425476074 14.7003479003906\\
37.9031410217285 14.7003479003906\\
38.0031433105469 14.7003479003906\\
38.103141784668 14.7003479003906\\
38.2031402587891 14.7003479003906\\
38.3031425476074 14.7003479003906\\
38.4031410217285 14.7003479003906\\
38.5031433105469 14.7003479003906\\
38.603141784668 14.7003479003906\\
38.7031402587891 14.7003479003906\\
38.8031425476074 14.7003479003906\\
38.9031410217285 14.7003479003906\\
39.0031433105469 14.7003479003906\\
39.103141784668 14.7003479003906\\
39.2031402587891 14.7003479003906\\
39.3031425476074 14.7003479003906\\
39.4031410217285 14.7003479003906\\
39.5031433105469 14.7003479003906\\
39.603141784668 14.7003479003906\\
39.7031402587891 14.7003479003906\\
39.8031425476074 14.7003479003906\\
39.9031410217285 14.7003479003906\\
40.0031433105469 14.7003479003906\\
40.103141784668 14.7003479003906\\
40.2031402587891 14.7003479003906\\
40.3031425476074 14.7003479003906\\
40.4031410217285 14.7003479003906\\
40.5031433105469 14.7003479003906\\
40.603141784668 14.7003479003906\\
40.7031402587891 14.7003479003906\\
40.8031425476074 14.7003479003906\\
40.9031410217285 14.7003479003906\\
41.0031433105469 14.7003479003906\\
41.103141784668 14.7003479003906\\
41.2031402587891 14.7003479003906\\
41.3031425476074 14.7003479003906\\
41.4031410217285 14.7003479003906\\
41.5031433105469 14.7003479003906\\
41.603141784668 14.7003479003906\\
41.7031402587891 14.7003479003906\\
41.8031425476074 14.7003479003906\\
41.9031410217285 14.7003479003906\\
42.0031433105469 14.7003479003906\\
42.103141784668 14.7003479003906\\
42.2031402587891 14.7003479003906\\
42.3031425476074 14.7003479003906\\
42.4031410217285 14.7003479003906\\
42.5031433105469 14.7003479003906\\
42.603141784668 14.7003479003906\\
42.7031402587891 14.7003479003906\\
42.8031425476074 14.7003469467163\\
42.9031410217285 14.7003469467163\\
43.0031433105469 14.700345993042\\
43.103141784668 14.700345993042\\
43.2031402587891 14.7003479003906\\
43.3031425476074 14.7003479003906\\
43.4031410217285 14.7003488540649\\
43.5031433105469 14.7003479003906\\
43.603141784668 14.7003479003906\\
43.7031402587891 14.7003479003906\\
43.8031425476074 14.7003479003906\\
43.9031410217285 14.7003479003906\\
44.0031433105469 14.7003479003906\\
44.103141784668 14.7003479003906\\
44.2031402587891 14.7003479003906\\
44.3031425476074 14.7003479003906\\
44.4031410217285 14.7003479003906\\
44.5031433105469 14.7003479003906\\
44.603141784668 14.7003479003906\\
44.7031402587891 14.7003479003906\\
44.8031425476074 14.7003479003906\\
44.9031410217285 14.7003479003906\\
45.0031433105469 14.7003479003906\\
45.103141784668 14.7003479003906\\
45.2031402587891 14.7003479003906\\
45.3031425476074 14.7003479003906\\
45.4031410217285 14.7003479003906\\
45.5031433105469 14.7003479003906\\
45.603141784668 14.7003479003906\\
45.7031402587891 14.7003479003906\\
45.8031425476074 14.7003479003906\\
45.9031410217285 14.7003479003906\\
46.0031433105469 14.7003479003906\\
46.103141784668 14.7003479003906\\
46.2031402587891 14.7003479003906\\
46.3031425476074 14.7003479003906\\
46.4031410217285 14.7003479003906\\
46.5031433105469 14.7003479003906\\
46.603141784668 14.7003479003906\\
46.7031402587891 14.7003479003906\\
46.8031425476074 14.7003479003906\\
46.9031410217285 14.7003479003906\\
47.0031433105469 14.7003479003906\\
47.103141784668 14.7003479003906\\
47.2031402587891 14.7003479003906\\
47.3031425476074 14.7003479003906\\
47.4031410217285 14.7003479003906\\
47.5031433105469 14.7003479003906\\
47.603141784668 14.7003479003906\\
47.7031402587891 14.7003479003906\\
47.8031425476074 14.7003479003906\\
47.9031410217285 14.7003479003906\\
48.0031433105469 14.7003479003906\\
48.103141784668 14.7003479003906\\
48.2031402587891 14.7003479003906\\
48.3031425476074 14.7003479003906\\
48.4031410217285 14.7003479003906\\
48.5031433105469 14.7003479003906\\
48.603141784668 14.7003479003906\\
48.7031402587891 14.7003479003906\\
48.8031425476074 14.7003479003906\\
48.9031410217285 14.7003479003906\\
49.0031433105469 14.7003479003906\\
49.0980262756348 14.7003479003906\\
49.1739387512207 14.7003479003906\\
49.2413482666016 14.7003479003906\\
49.315502166748 14.7003479003906\\
49.3970718383789 14.7003479003906\\
49.4867973327637 14.7003479003906\\
49.5854949951172 14.7003479003906\\
49.6854934692383 14.7003479003906\\
49.7854957580566 14.7003479003906\\
49.8854942321777 14.7003479003906\\
49.9854927062988 14.7003479003906\\
50.0854949951172 14.7003479003906\\
50.1854934692383 14.7003479003906\\
50.2854957580566 14.7003479003906\\
50.3854942321777 14.7003479003906\\
50.4854927062988 14.7003479003906\\
50.5854949951172 14.7003479003906\\
50.6854934692383 14.7003479003906\\
50.7854957580566 14.7003479003906\\
50.8854942321777 14.7003479003906\\
50.9854927062988 14.7003479003906\\
51.0854949951172 14.7003479003906\\
51.1854934692383 14.7003479003906\\
51.2854957580566 14.7003479003906\\
51.3854942321777 14.7003479003906\\
51.4854927062988 14.7003479003906\\
51.5854949951172 14.7003479003906\\
51.6854934692383 14.7003479003906\\
51.7854957580566 14.7003479003906\\
51.8854942321777 14.7003479003906\\
51.9854927062988 14.7003479003906\\
52.0854949951172 14.7003479003906\\
52.1854934692383 14.7003479003906\\
52.2854957580566 14.7003479003906\\
52.3854942321777 14.7003479003906\\
52.4854927062988 14.7003479003906\\
52.5854949951172 14.7003479003906\\
52.6854934692383 14.7003479003906\\
52.7854957580566 14.7003479003906\\
52.8854942321777 14.7003479003906\\
52.9854927062988 14.7003479003906\\
53.0854949951172 14.7003479003906\\
53.1854934692383 14.7003479003906\\
53.2854957580566 14.7003479003906\\
53.3854942321777 14.7003479003906\\
53.4854927062988 14.7003479003906\\
53.5854949951172 14.7003479003906\\
53.6854934692383 14.7003479003906\\
53.7854957580566 14.7003479003906\\
53.8854942321777 14.7003479003906\\
53.9854927062988 14.7003479003906\\
54.0854949951172 14.7003479003906\\
54.1854934692383 14.7003479003906\\
54.2854957580566 14.7003479003906\\
54.3854942321777 14.7003479003906\\
54.4854927062988 14.7003479003906\\
54.5854949951172 14.7003479003906\\
54.6854934692383 14.7003479003906\\
54.7854957580566 14.7003479003906\\
54.8854942321777 14.7003479003906\\
54.9854927062988 14.7003469467163\\
55.0854949951172 14.700345993042\\
55.1854934692383 14.7003469467163\\
55.2854957580566 14.700345993042\\
55.3854942321777 14.7003469467163\\
55.4854927062988 14.7003488540649\\
55.5854949951172 14.7003479003906\\
55.6854934692383 14.7003488540649\\
55.7854957580566 14.7003469467163\\
55.8854942321777 14.7003479003906\\
55.9854927062988 14.7003479003906\\
56.0854949951172 14.7003479003906\\
56.1854934692383 14.7003479003906\\
56.2854957580566 14.7003479003906\\
56.3854942321777 14.7003479003906\\
56.4854927062988 14.7003479003906\\
56.5854949951172 14.7003479003906\\
56.6854934692383 14.7003479003906\\
56.7854957580566 14.7003479003906\\
56.8854942321777 14.7003479003906\\
56.9854927062988 14.7003479003906\\
57.0854949951172 14.7003479003906\\
57.1854934692383 14.7003479003906\\
57.2854957580566 14.7003479003906\\
57.3854942321777 14.7003479003906\\
57.4854927062988 14.7003479003906\\
57.5854949951172 14.7003479003906\\
57.6854934692383 14.7003479003906\\
57.7854957580566 14.7003479003906\\
57.8854942321777 14.7003479003906\\
57.9854927062988 14.7003479003906\\
58.0854949951172 14.7003479003906\\
58.1854934692383 14.7003479003906\\
58.2854957580566 14.7003479003906\\
58.3854942321777 14.7003479003906\\
58.4854927062988 14.7003479003906\\
58.5854949951172 14.7003479003906\\
58.6854934692383 14.7003479003906\\
58.7854957580566 14.7003479003906\\
58.8854942321777 14.7003479003906\\
58.9854927062988 14.7003479003906\\
59.0854949951172 14.7003479003906\\
59.1854934692383 14.7003479003906\\
59.2854957580566 14.7003479003906\\
59.3854942321777 14.7003479003906\\
59.4854927062988 14.7003479003906\\
59.5854949951172 14.7003479003906\\
59.6854934692383 14.7003479003906\\
59.7854957580566 14.7003479003906\\
59.8854942321777 14.7003479003906\\
59.9427452087402 14.7003479003906\\
60 14.7003479003906\\
60.0629768371582 14.7003479003906\\
60.1322555541992 14.7003479003906\\
60.2084579467773 14.7003479003906\\
60.2922821044922 14.7003479003906\\
60.3844909667969 14.7003479003906\\
60.484489440918 14.7003479003906\\
60.5844879150391 14.7003479003906\\
60.6844902038574 14.7003479003906\\
60.7844886779785 14.7003479003906\\
60.8844909667969 14.7003479003906\\
60.984489440918 14.7003479003906\\
61.0844879150391 14.7003479003906\\
61.1844902038574 14.7003469467163\\
61.2844886779785 14.7003469467163\\
61.3734016418457 14.7003469467163\\
61.444522857666 14.7003469467163\\
61.5227546691895 14.7003469467163\\
61.6088104248047 14.7003479003906\\
61.7034721374512 14.7003479003906\\
61.8034706115723 14.7003488540649\\
61.9034729003906 14.7003488540649\\
62.0034713745117 14.7003488540649\\
62.1034736633301 14.7003488540649\\
62.2034721374512 14.7003488540649\\
62.3034706115723 14.7003488540649\\
62.4034729003906 14.7003488540649\\
62.5034713745117 14.7003488540649\\
62.6034736633301 14.7003488540649\\
62.7034721374512 14.7003488540649\\
62.8034706115723 14.7003488540649\\
62.9034729003906 14.7003488540649\\
63.0034713745117 14.7003488540649\\
63.1034736633301 14.7003488540649\\
63.2034721374512 14.7003488540649\\
63.3034706115723 14.7003488540649\\
63.4034729003906 14.7003488540649\\
63.5034713745117 14.7003488540649\\
63.6034736633301 14.7003488540649\\
63.7034721374512 14.7003488540649\\
63.8034706115723 14.7003488540649\\
63.9034729003906 14.7003488540649\\
64.0034713745117 14.7003488540649\\
64.1034698486328 14.7003488540649\\
64.2034759521484 14.7003488540649\\
64.3034744262695 14.7003488540649\\
64.4034729003906 14.7003488540649\\
64.5034713745117 14.7003488540649\\
64.6034698486328 14.7003488540649\\
64.7034759521484 14.7003488540649\\
64.8034744262695 14.7003488540649\\
64.9034729003906 14.7003488540649\\
65.0034713745117 14.7003488540649\\
65.1034698486328 14.7003488540649\\
65.2034759521484 14.7003488540649\\
65.3034744262695 14.7003488540649\\
65.4034729003906 14.7003488540649\\
65.5034713745117 14.7003488540649\\
65.6034698486328 14.7003488540649\\
65.7034759521484 14.7003488540649\\
65.8034744262695 14.7003488540649\\
65.9034729003906 14.7003488540649\\
66.0034713745117 14.7003488540649\\
66.1034698486328 14.7003488540649\\
66.2034759521484 14.7003488540649\\
66.3034744262695 14.7003488540649\\
66.4034729003906 14.7003488540649\\
66.5034713745117 14.7003488540649\\
66.6034698486328 14.7003488540649\\
66.7034759521484 14.7003488540649\\
66.8034744262695 14.7003498077393\\
66.9034729003906 14.7003517150879\\
67.0034713745117 14.7003612518311\\
67.1034698486328 14.7004041671753\\
67.2034759521484 14.7006320953369\\
67.3034744262695 14.7018260955811\\
67.4034729003906 14.706883430481\\
67.5034713745117 14.7189617156982\\
67.6034698486328 14.7303943634033\\
67.7034759521484 14.7325849533081\\
67.8034744262695 14.7318706512451\\
67.9034729003906 14.7326974868774\\
68.0034713745117 14.7323160171509\\
68.1034698486328 14.7333288192749\\
68.2034759521484 14.7335357666016\\
68.3034744262695 14.7342824935913\\
68.4034729003906 14.7346706390381\\
68.5034713745117 14.7352981567383\\
68.6034698486328 14.7357664108276\\
68.7034759521484 14.7363405227661\\
68.8034744262695 14.7368440628052\\
68.9034729003906 14.7373943328857\\
69.0034713745117 14.7379140853882\\
69.1034698486328 14.7384548187256\\
69.2034759521484 14.7389812469482\\
69.3034744262695 14.7395181655884\\
69.4034729003906 14.7400484085083\\
69.5034713745117 14.7405824661255\\
69.6034698486328 14.741114616394\\
69.7034759521484 14.7416486740112\\
69.8034744262695 14.7421817779541\\
69.9034729003906 14.7427167892456\\
70.0034713745117 14.7432508468628\\
70.1034698486328 14.7439041137695\\
70.2034759521484 14.7446870803833\\
70.3034744262695 14.7453022003174\\
70.4034729003906 14.7459535598755\\
70.5034713745117 14.7465887069702\\
70.6034698486328 14.7472286224365\\
70.7034759521484 14.7478656768799\\
70.8034744262695 14.7485017776489\\
70.9034729003906 14.749137878418\\
71.0034713745117 14.7497730255127\\
71.1034698486328 14.7504072189331\\
71.2034759521484 14.7510423660278\\
71.3034744262695 14.7516765594482\\
71.4034729003906 14.7523107528687\\
71.5034713745117 14.7529458999634\\
71.6034698486328 14.7535648345947\\
71.7034759521484 14.7538862228394\\
71.8034744262695 14.7544536590576\\
71.9034729003906 14.7549810409546\\
72.0034713745117 14.7555208206177\\
72.1034698486328 14.7560529708862\\
72.2034759521484 14.7565832138062\\
72.3034744262695 14.7571105957031\\
72.4034729003906 14.7576341629028\\
72.5034713745117 14.758152961731\\
72.6034698486328 14.7586679458618\\
72.7034759521484 14.7592115402222\\
72.8034744262695 14.7598390579224\\
72.9034729003906 14.7604732513428\\
73.0034713745117 14.761079788208\\
73.1034698486328 14.7616958618164\\
73.2034759521484 14.762336730957\\
73.3034744262695 14.7630157470703\\
73.4034729003906 14.7637548446655\\
73.5034713745117 14.767053604126\\
73.5992889404297 14.7732543945313\\
73.6759490966797 14.7762327194214\\
73.7424468994141 14.7681264877319\\
73.8155975341797 14.7683124542236\\
73.8960571289063 14.7687044143677\\
73.9845733642578 14.769250869751\\
74.0819320678711 14.7700080871582\\
74.1819305419922 14.7709722518921\\
74.2819290161133 14.7720031738281\\
74.3819351196289 14.7730512619019\\
74.48193359375 14.7741165161133\\
74.5819320678711 14.7751951217651\\
74.6819305419922 14.7762832641602\\
74.7819290161133 14.7773761749268\\
74.8819351196289 14.7784786224365\\
74.9409637451172 14.7795896530151\\
75 14.7802410125732\\
};
\addlegendentry{22 "};

\end{axis}
\end{tikzpicture}%
\caption{Filling problem with \dtmax{} = \expneg{1.0}{1}{[s]} with the nonlinear solver active.}
\label{fig:nlnFill1em1}
\end{figure}

The solution obtained by the nonlinear solver at this large \dtmax{} is comparable to those obtained by the smaller timestep sizes, \dtmax{} = \expneg{5.0}{2}{[s]} and \expneg{1.5}{3}{[s]}, with the linear solver.
Examining the actual timestep sizes shows that the nonlinear simulation with \dtmax{} = \expneg{5.0}{2}{[s]} actually run at \dtmax{} for the majority of timesteps.

\begin{figure}[h!tb]
\centering
% This file was created by matlab2tikz v0.4.3.
% Copyright (c) 2008--2013, Nico Schlömer <nico.schloemer@gmail.com>
% All rights reserved.
% 
\tikzsetnextfilename{plots/vmpDeltaTNln1em1_eps}
\begin{tikzpicture}

\begin{axis}[%
width=0.8\textwidth,
height=0.491294629700995\textwidth,
scale only axis,
xmin=0,
xmax=75,
xlabel={$\text{\$\dt}{}\text{\$}$},
ymode=log,
ymin=0.001,
ymax=10,
yminorticks=true,
ylabel={Time [s]},
legend style={at={(0.97,0.03)},anchor=south east,draw=black,fill=white,legend cell align=left}
]
\addplot [
color=blue,
solid
]
table[row sep=crcr]{
0 9.99999971718069e-10\\
9.99999971718069e-10 9.99999971718069e-10\\
2.49999998480632e-09 1.50000001308825e-09\\
4.74999994892755e-09 2.24999996412123e-09\\
8.1250002281763e-09 3.37500005720415e-09\\
1.31874999809156e-08 5.06250019682852e-09\\
2.0781250498203e-08 7.59374962910897e-09\\
3.21718758300449e-08 1.13906253318419e-08\\
4.92578138278077e-08 1.70859379977628e-08\\
7.48867208244519e-08 2.56289069966442e-08\\
1.13330081319418e-07 3.84433604949663e-08\\
1.70995122061868e-07 5.76650407424495e-08\\
2.57492672517401e-07 8.64975575609606e-08\\
3.8723902662241e-07 1.29746339894155e-07\\
5.81858500936505e-07 1.94619502735804e-07\\
8.73787769251066e-07 2.91929268314561e-07\\
1.31168167172291e-06 4.37893902471842e-07\\
1.9685226106958e-06 6.56840825286054e-07\\
2.95378367809462e-06 9.85261294772499e-07\\
4.43167573394021e-06 1.47789182847191e-06\\
6.64851359033491e-06 2.2168378563947e-06\\
9.97376992017962e-06 3.32525678459206e-06\\
1.49616553244414e-05 4.98788494951441e-06\\
2.24434825213393e-05 7.48182765164529e-06\\
3.36662233166862e-05 1.12227417048416e-05\\
5.05003372381907e-05 1.68341121025151e-05\\
7.57515081204474e-05 2.52511690632673e-05\\
0.000113628258986864 3.78767508664168e-05\\
0.000170443381648511 5.68151299376041e-05\\
0.000255666091106832 8.52226949064061e-05\\
0.000383500126190484 0.000127834035083652\\
0.000575251178815961 0.000191751052625477\\
0.000862877757754177 0.000287626578938216\\
0.00129431765526533 0.000431439897511154\\
0.00194147753063589 0.000647159817162901\\
0.00291221728548408 0.000970739754848182\\
0.00436832662671804 0.00145610957406461\\
0.00597004732117057 0.0016017205780372\\
0.00773194013163447 0.00176189257763326\\
0.009670021943748 0.00193808192852885\\
0.0118019115179777 0.002131890039891\\
0.0141469910740852 0.00234507909044623\\
0.0167265776544809 0.00257958704605699\\
0.0195641238242388 0.00283754570409656\\
0.0226854234933853 0.00312130036763847\\
0.0261188540607691 0.00343343033455312\\
0.0298956278711557 0.00377677334472537\\
0.0340500771999359 0.00415445072576404\\
0.0386199727654457 0.0045698955655098\\
0.0436468608677387 0.00502688530832529\\
0.0491764321923256 0.0055295736528933\\
0.0552589632570744 0.00608253106474876\\
0.0619497485458851 0.00669078435748816\\
0.0693096145987511 0.00735986279323697\\
0.0774054601788521 0.00809584930539131\\
0.0863108932971954 0.00890543404966593\\
0.0961068719625473 0.00979597773402929\\
0.106882445514202 0.0107755754143\\
0.118735581636429 0.0118531323969364\\
0.13177402317524 0.0130384461954236\\
0.146116316318512 0.0143422903493047\\
0.161892831325531 0.0157765187323093\\
0.179247006773949 0.0173541717231274\\
0.198336601257324 0.0190895888954401\\
0.21933513879776 0.0209985468536615\\
0.242433547973633 0.0230984017252922\\
0.267841786146164 0.0254082418978214\\
0.29579085111618 0.0279490668326616\\
0.32653483748436 0.0307439733296633\\
0.360353201627731 0.0338183715939522\\
0.397553414106369 0.0372002087533474\\
0.438473641872406 0.040920227766037\\
0.483485877513886 0.0450122505426407\\
0.532999336719513 0.0495134778320789\\
0.587464153766632 0.0544648244976997\\
0.647375464439392 0.0599113069474697\\
0.71327793598175 0.0659024342894554\\
0.7857705950737 0.0724926814436913\\
0.865512549877167 0.0797419473528862\\
0.953228712081909 0.0877161398530006\\
1.04971647262573 0.0964877605438232\\
1.14971649646759 0.100000001490116\\
1.24971640110016 0.100000001490116\\
1.34971642494202 0.100000001490116\\
1.44971644878387 0.100000001490116\\
1.54971647262573 0.100000001490116\\
1.64971649646759 0.100000001490116\\
1.74971640110016 0.100000001490116\\
1.84971642494202 0.100000001490116\\
1.94971644878387 0.100000001490116\\
2.04971647262573 0.100000001490116\\
2.1497163772583 0.100000001490116\\
2.24971652030945 0.100000001490116\\
2.34971642494202 0.100000001490116\\
2.44971656799316 0.100000001490116\\
2.54971647262573 0.100000001490116\\
2.6497163772583 0.100000001490116\\
2.74971652030945 0.100000001490116\\
2.84971642494202 0.100000001490116\\
2.94971656799316 0.100000001490116\\
3.04971647262573 0.100000001490116\\
3.1497163772583 0.100000001490116\\
3.24971652030945 0.100000001490116\\
3.34971642494202 0.100000001490116\\
3.44971656799316 0.100000001490116\\
3.54971647262573 0.100000001490116\\
3.6497163772583 0.100000001490116\\
3.74971652030945 0.100000001490116\\
3.84971642494202 0.100000001490116\\
3.94971656799316 0.100000001490116\\
4.04971647262573 0.100000001490116\\
4.1497163772583 0.100000001490116\\
4.24971628189087 0.100000001490116\\
4.3497166633606 0.100000001490116\\
4.44971656799316 0.100000001490116\\
4.54971647262573 0.100000001490116\\
4.6497163772583 0.100000001490116\\
4.74971628189087 0.100000001490116\\
4.8497166633606 0.100000001490116\\
4.94971656799316 0.100000001490116\\
5.04971647262573 0.100000001490116\\
5.1497163772583 0.100000001490116\\
5.24971628189087 0.100000001490116\\
5.3497166633606 0.100000001490116\\
5.44971656799316 0.100000001490116\\
5.54971647262573 0.100000001490116\\
5.6497163772583 0.100000001490116\\
5.74971628189087 0.100000001490116\\
5.8497166633606 0.100000001490116\\
5.94971656799316 0.100000001490116\\
6.04971647262573 0.100000001490116\\
6.1497163772583 0.100000001490116\\
6.24971628189087 0.100000001490116\\
6.3497166633606 0.100000001490116\\
6.44971656799316 0.100000001490116\\
6.54971647262573 0.100000001490116\\
6.6497163772583 0.100000001490116\\
6.74971628189087 0.100000001490116\\
6.8497166633606 0.100000001490116\\
6.94971656799316 0.100000001490116\\
7.04971647262573 0.100000001490116\\
7.1497163772583 0.100000001490116\\
7.24971628189087 0.100000001490116\\
7.3497166633606 0.100000001490116\\
7.44971656799316 0.100000001490116\\
7.54971647262573 0.100000001490116\\
7.6497163772583 0.100000001490116\\
7.74971628189087 0.100000001490116\\
7.8497166633606 0.100000001490116\\
7.94971656799316 0.100000001490116\\
8.04971599578857 0.100000001490116\\
8.1497163772583 0.100000001490116\\
8.24971675872803 0.100000001490116\\
8.34971618652344 0.100000001490116\\
8.44971656799316 0.100000001490116\\
8.54971599578857 0.100000001490116\\
8.6497163772583 0.100000001490116\\
8.74971675872803 0.100000001490116\\
8.84971618652344 0.100000001490116\\
8.94971656799316 0.100000001490116\\
9.04971599578857 0.100000001490116\\
9.1497163772583 0.100000001490116\\
9.24971675872803 0.100000001490116\\
9.34971618652344 0.100000001490116\\
9.44971656799316 0.100000001490116\\
9.54971599578857 0.100000001490116\\
9.6497163772583 0.100000001490116\\
9.74971675872803 0.100000001490116\\
9.84971618652344 0.100000001490116\\
9.94971656799316 0.100000001490116\\
10.0497159957886 0.100000001490116\\
10.1497163772583 0.100000001490116\\
10.249716758728 0.100000001490116\\
10.3497161865234 0.100000001490116\\
10.4497165679932 0.100000001490116\\
10.5497159957886 0.100000001490116\\
10.6497163772583 0.100000001490116\\
10.749716758728 0.100000001490116\\
10.8497161865234 0.100000001490116\\
10.9497165679932 0.100000001490116\\
11.0497159957886 0.100000001490116\\
11.1497163772583 0.100000001490116\\
11.249716758728 0.100000001490116\\
11.3497161865234 0.100000001490116\\
11.4497165679932 0.100000001490116\\
11.5497159957886 0.100000001490116\\
11.6497163772583 0.100000001490116\\
11.749716758728 0.100000001490116\\
11.8497161865234 0.100000001490116\\
11.9497165679932 0.100000001490116\\
12.0497159957886 0.100000001490116\\
12.1497163772583 0.100000001490116\\
12.249716758728 0.100000001490116\\
12.3471450805664 0.097429059445858\\
12.4250917434692 0.0779464840888977\\
12.4879579544067 0.0628659725189209\\
12.557110786438 0.0691525712609291\\
12.6331787109375 0.076067827641964\\
12.7168531417847 0.0836746096611023\\
12.808895111084 0.0920420661568642\\
12.9088954925537 0.100000001490116\\
13.0088949203491 0.100000001490116\\
13.1088953018188 0.100000001490116\\
13.2088947296143 0.100000001490116\\
13.308895111084 0.100000001490116\\
13.4088954925537 0.100000001490116\\
13.5088949203491 0.100000001490116\\
13.6088953018188 0.100000001490116\\
13.7088947296143 0.100000001490116\\
13.808895111084 0.100000001490116\\
13.9088954925537 0.100000001490116\\
14.0088949203491 0.100000001490116\\
14.1088953018188 0.100000001490116\\
14.2088947296143 0.100000001490116\\
14.308895111084 0.100000001490116\\
14.4088954925537 0.100000001490116\\
14.5088949203491 0.100000001490116\\
14.6088953018188 0.100000001490116\\
14.7088947296143 0.100000001490116\\
14.808895111084 0.100000001490116\\
14.9088954925537 0.100000001490116\\
15.0088949203491 0.100000001490116\\
15.1088953018188 0.100000001490116\\
15.2088947296143 0.100000001490116\\
15.308895111084 0.100000001490116\\
15.4088954925537 0.100000001490116\\
15.5088949203491 0.100000001490116\\
15.6088953018188 0.100000001490116\\
15.7088947296143 0.100000001490116\\
15.808895111084 0.100000001490116\\
15.9088954925537 0.100000001490116\\
16.0088958740234 0.100000001490116\\
16.1088943481445 0.100000001490116\\
16.2088947296143 0.100000001490116\\
16.308895111084 0.100000001490116\\
16.4088954925537 0.100000001490116\\
16.5088958740234 0.100000001490116\\
16.6088943481445 0.100000001490116\\
16.7088947296143 0.100000001490116\\
16.808895111084 0.100000001490116\\
16.9088954925537 0.100000001490116\\
17.0088958740234 0.100000001490116\\
17.1088943481445 0.100000001490116\\
17.2088947296143 0.100000001490116\\
17.308895111084 0.100000001490116\\
17.4088954925537 0.100000001490116\\
17.5088958740234 0.100000001490116\\
17.6088943481445 0.100000001490116\\
17.7088947296143 0.100000001490116\\
17.808895111084 0.100000001490116\\
17.9088954925537 0.100000001490116\\
18.0088958740234 0.100000001490116\\
18.1088943481445 0.100000001490116\\
18.2088947296143 0.100000001490116\\
18.308895111084 0.100000001490116\\
18.4088954925537 0.100000001490116\\
18.5088958740234 0.100000001490116\\
18.6088943481445 0.100000001490116\\
18.7088947296143 0.100000001490116\\
18.808895111084 0.100000001490116\\
18.9088954925537 0.100000001490116\\
19.0088958740234 0.100000001490116\\
19.1088943481445 0.100000001490116\\
19.2088947296143 0.100000001490116\\
19.308895111084 0.100000001490116\\
19.4088954925537 0.100000001490116\\
19.5088958740234 0.100000001490116\\
19.6088943481445 0.100000001490116\\
19.7088947296143 0.100000001490116\\
19.808895111084 0.100000001490116\\
19.9088954925537 0.100000001490116\\
20.0088958740234 0.100000001490116\\
20.1088943481445 0.100000001490116\\
20.2088947296143 0.100000001490116\\
20.308895111084 0.100000001490116\\
20.4088954925537 0.100000001490116\\
20.5088958740234 0.100000001490116\\
20.6088943481445 0.100000001490116\\
20.7088947296143 0.100000001490116\\
20.808895111084 0.100000001490116\\
20.9088954925537 0.100000001490116\\
21.0088958740234 0.100000001490116\\
21.1088943481445 0.100000001490116\\
21.2088947296143 0.100000001490116\\
21.308895111084 0.100000001490116\\
21.4088954925537 0.100000001490116\\
21.5088958740234 0.100000001490116\\
21.6088943481445 0.100000001490116\\
21.7088947296143 0.100000001490116\\
21.808895111084 0.100000001490116\\
21.9088954925537 0.100000001490116\\
22.0088958740234 0.100000001490116\\
22.1088943481445 0.100000001490116\\
22.2088947296143 0.100000001490116\\
22.308895111084 0.100000001490116\\
22.4088954925537 0.100000001490116\\
22.5088958740234 0.100000001490116\\
22.6088943481445 0.100000001490116\\
22.7088947296143 0.100000001490116\\
22.808895111084 0.100000001490116\\
22.9088954925537 0.100000001490116\\
23.0088958740234 0.100000001490116\\
23.1088943481445 0.100000001490116\\
23.2088947296143 0.100000001490116\\
23.308895111084 0.100000001490116\\
23.4088954925537 0.100000001490116\\
23.5088958740234 0.100000001490116\\
23.6088943481445 0.100000001490116\\
23.7088947296143 0.100000001490116\\
23.808895111084 0.100000001490116\\
23.9088954925537 0.100000001490116\\
24.0088958740234 0.100000001490116\\
24.1088943481445 0.100000001490116\\
24.2088947296143 0.100000001490116\\
24.308895111084 0.100000001490116\\
24.4088954925537 0.100000001490116\\
24.5088958740234 0.100000001490116\\
24.6048831939697 0.0959879383444786\\
24.681676864624 0.0767935216426849\\
24.7464084625244 0.0647319331765175\\
24.8176136016846 0.0712051242589951\\
24.895938873291 0.0783256366848946\\
24.9820976257324 0.0861582010984421\\
25.0768718719482 0.0947740226984024\\
25.176872253418 0.100000001490116\\
25.2768707275391 0.100000001490116\\
25.3768711090088 0.100000001490116\\
25.4768714904785 0.100000001490116\\
25.5768718719482 0.100000001490116\\
25.676872253418 0.100000001490116\\
25.7768707275391 0.100000001490116\\
25.8768711090088 0.100000001490116\\
25.9768714904785 0.100000001490116\\
26.0768718719482 0.100000001490116\\
26.176872253418 0.100000001490116\\
26.2768707275391 0.100000001490116\\
26.3768711090088 0.100000001490116\\
26.4768714904785 0.100000001490116\\
26.5768718719482 0.100000001490116\\
26.676872253418 0.100000001490116\\
26.7768707275391 0.100000001490116\\
26.8768711090088 0.100000001490116\\
26.9768714904785 0.100000001490116\\
27.0768718719482 0.100000001490116\\
27.176872253418 0.100000001490116\\
27.2768707275391 0.100000001490116\\
27.3768711090088 0.100000001490116\\
27.4768714904785 0.100000001490116\\
27.5768718719482 0.100000001490116\\
27.676872253418 0.100000001490116\\
27.7768707275391 0.100000001490116\\
27.8768711090088 0.100000001490116\\
27.9768714904785 0.100000001490116\\
28.0768718719482 0.100000001490116\\
28.176872253418 0.100000001490116\\
28.2768707275391 0.100000001490116\\
28.3768711090088 0.100000001490116\\
28.4768714904785 0.100000001490116\\
28.5768718719482 0.100000001490116\\
28.676872253418 0.100000001490116\\
28.7768707275391 0.100000001490116\\
28.8768711090088 0.100000001490116\\
28.9768714904785 0.100000001490116\\
29.0768718719482 0.100000001490116\\
29.176872253418 0.100000001490116\\
29.2768707275391 0.100000001490116\\
29.3768711090088 0.100000001490116\\
29.4768714904785 0.100000001490116\\
29.5768718719482 0.100000001490116\\
29.676872253418 0.100000001490116\\
29.7768707275391 0.100000001490116\\
29.8768711090088 0.100000001490116\\
29.9768714904785 0.100000001490116\\
30.0768718719482 0.100000001490116\\
30.176872253418 0.100000001490116\\
30.2768707275391 0.100000001490116\\
30.3768711090088 0.100000001490116\\
30.4768714904785 0.100000001490116\\
30.5768718719482 0.100000001490116\\
30.676872253418 0.100000001490116\\
30.7768707275391 0.100000001490116\\
30.8768711090088 0.100000001490116\\
30.9768714904785 0.100000001490116\\
31.0768718719482 0.100000001490116\\
31.176872253418 0.100000001490116\\
31.2768707275391 0.100000001490116\\
31.3768711090088 0.100000001490116\\
31.4768714904785 0.100000001490116\\
31.5768718719482 0.100000001490116\\
31.676872253418 0.100000001490116\\
31.7768707275391 0.100000001490116\\
31.8768711090088 0.100000001490116\\
31.9768714904785 0.100000001490116\\
32.0768699645996 0.100000001490116\\
32.176872253418 0.100000001490116\\
32.2768707275391 0.100000001490116\\
32.3768730163574 0.100000001490116\\
32.4768714904785 0.100000001490116\\
32.5768699645996 0.100000001490116\\
32.676872253418 0.100000001490116\\
32.7768707275391 0.100000001490116\\
32.8768730163574 0.100000001490116\\
32.9768714904785 0.100000001490116\\
33.0768699645996 0.100000001490116\\
33.176872253418 0.100000001490116\\
33.2768707275391 0.100000001490116\\
33.3768730163574 0.100000001490116\\
33.4768714904785 0.100000001490116\\
33.5768699645996 0.100000001490116\\
33.676872253418 0.100000001490116\\
33.7768707275391 0.100000001490116\\
33.8768730163574 0.100000001490116\\
33.9768714904785 0.100000001490116\\
34.0768699645996 0.100000001490116\\
34.176872253418 0.100000001490116\\
34.2768707275391 0.100000001490116\\
34.3768730163574 0.100000001490116\\
34.4768714904785 0.100000001490116\\
34.5768699645996 0.100000001490116\\
34.676872253418 0.100000001490116\\
34.7768707275391 0.100000001490116\\
34.8768730163574 0.100000001490116\\
34.9768714904785 0.100000001490116\\
35.0768699645996 0.100000001490116\\
35.176872253418 0.100000001490116\\
35.2768707275391 0.100000001490116\\
35.3768730163574 0.100000001490116\\
35.4768714904785 0.100000001490116\\
35.5768699645996 0.100000001490116\\
35.676872253418 0.100000001490116\\
35.7768707275391 0.100000001490116\\
35.8768730163574 0.100000001490116\\
35.9768714904785 0.100000001490116\\
36.0768699645996 0.100000001490116\\
36.176872253418 0.100000001490116\\
36.2768707275391 0.100000001490116\\
36.3768730163574 0.100000001490116\\
36.4768714904785 0.100000001490116\\
36.5768699645996 0.100000001490116\\
36.676872253418 0.100000001490116\\
36.7768707275391 0.100000001490116\\
36.8673248291016 0.0904538929462433\\
36.9396858215332 0.0723586231470108\\
37.0192794799805 0.0795944854617119\\
37.1068305969238 0.087553933262825\\
37.2031402587891 0.0963093265891075\\
37.3031425476074 0.100000001490116\\
37.4031410217285 0.100000001490116\\
37.5031433105469 0.100000001490116\\
37.603141784668 0.100000001490116\\
37.7031402587891 0.100000001490116\\
37.8031425476074 0.100000001490116\\
37.9031410217285 0.100000001490116\\
38.0031433105469 0.100000001490116\\
38.103141784668 0.100000001490116\\
38.2031402587891 0.100000001490116\\
38.3031425476074 0.100000001490116\\
38.4031410217285 0.100000001490116\\
38.5031433105469 0.100000001490116\\
38.603141784668 0.100000001490116\\
38.7031402587891 0.100000001490116\\
38.8031425476074 0.100000001490116\\
38.9031410217285 0.100000001490116\\
39.0031433105469 0.100000001490116\\
39.103141784668 0.100000001490116\\
39.2031402587891 0.100000001490116\\
39.3031425476074 0.100000001490116\\
39.4031410217285 0.100000001490116\\
39.5031433105469 0.100000001490116\\
39.603141784668 0.100000001490116\\
39.7031402587891 0.100000001490116\\
39.8031425476074 0.100000001490116\\
39.9031410217285 0.100000001490116\\
40.0031433105469 0.100000001490116\\
40.103141784668 0.100000001490116\\
40.2031402587891 0.100000001490116\\
40.3031425476074 0.100000001490116\\
40.4031410217285 0.100000001490116\\
40.5031433105469 0.100000001490116\\
40.603141784668 0.100000001490116\\
40.7031402587891 0.100000001490116\\
40.8031425476074 0.100000001490116\\
40.9031410217285 0.100000001490116\\
41.0031433105469 0.100000001490116\\
41.103141784668 0.100000001490116\\
41.2031402587891 0.100000001490116\\
41.3031425476074 0.100000001490116\\
41.4031410217285 0.100000001490116\\
41.5031433105469 0.100000001490116\\
41.603141784668 0.100000001490116\\
41.7031402587891 0.100000001490116\\
41.8031425476074 0.100000001490116\\
41.9031410217285 0.100000001490116\\
42.0031433105469 0.100000001490116\\
42.103141784668 0.100000001490116\\
42.2031402587891 0.100000001490116\\
42.3031425476074 0.100000001490116\\
42.4031410217285 0.100000001490116\\
42.5031433105469 0.100000001490116\\
42.603141784668 0.100000001490116\\
42.7031402587891 0.100000001490116\\
42.8031425476074 0.100000001490116\\
42.9031410217285 0.100000001490116\\
43.0031433105469 0.100000001490116\\
43.103141784668 0.100000001490116\\
43.2031402587891 0.100000001490116\\
43.3031425476074 0.100000001490116\\
43.4031410217285 0.100000001490116\\
43.5031433105469 0.100000001490116\\
43.603141784668 0.100000001490116\\
43.7031402587891 0.100000001490116\\
43.8031425476074 0.100000001490116\\
43.9031410217285 0.100000001490116\\
44.0031433105469 0.100000001490116\\
44.103141784668 0.100000001490116\\
44.2031402587891 0.100000001490116\\
44.3031425476074 0.100000001490116\\
44.4031410217285 0.100000001490116\\
44.5031433105469 0.100000001490116\\
44.603141784668 0.100000001490116\\
44.7031402587891 0.100000001490116\\
44.8031425476074 0.100000001490116\\
44.9031410217285 0.100000001490116\\
45.0031433105469 0.100000001490116\\
45.103141784668 0.100000001490116\\
45.2031402587891 0.100000001490116\\
45.3031425476074 0.100000001490116\\
45.4031410217285 0.100000001490116\\
45.5031433105469 0.100000001490116\\
45.603141784668 0.100000001490116\\
45.7031402587891 0.100000001490116\\
45.8031425476074 0.100000001490116\\
45.9031410217285 0.100000001490116\\
46.0031433105469 0.100000001490116\\
46.103141784668 0.100000001490116\\
46.2031402587891 0.100000001490116\\
46.3031425476074 0.100000001490116\\
46.4031410217285 0.100000001490116\\
46.5031433105469 0.100000001490116\\
46.603141784668 0.100000001490116\\
46.7031402587891 0.100000001490116\\
46.8031425476074 0.100000001490116\\
46.9031410217285 0.100000001490116\\
47.0031433105469 0.100000001490116\\
47.103141784668 0.100000001490116\\
47.2031402587891 0.100000001490116\\
47.3031425476074 0.100000001490116\\
47.4031410217285 0.100000001490116\\
47.5031433105469 0.100000001490116\\
47.603141784668 0.100000001490116\\
47.7031402587891 0.100000001490116\\
47.8031425476074 0.100000001490116\\
47.9031410217285 0.100000001490116\\
48.0031433105469 0.100000001490116\\
48.103141784668 0.100000001490116\\
48.2031402587891 0.100000001490116\\
48.3031425476074 0.100000001490116\\
48.4031410217285 0.100000001490116\\
48.5031433105469 0.100000001490116\\
48.603141784668 0.100000001490116\\
48.7031402587891 0.100000001490116\\
48.8031425476074 0.100000001490116\\
48.9031410217285 0.100000001490116\\
49.0031433105469 0.100000001490116\\
49.0980262756348 0.0948861092329025\\
49.1739387512207 0.0759102776646614\\
49.2413482666016 0.0674118101596832\\
49.315502166748 0.0741529911756516\\
49.3970718383789 0.0815682858228683\\
49.4867973327637 0.0897251218557358\\
49.5854949951172 0.0986976325511932\\
49.6854934692383 0.100000001490116\\
49.7854957580566 0.100000001490116\\
49.8854942321777 0.100000001490116\\
49.9854927062988 0.100000001490116\\
50.0854949951172 0.100000001490116\\
50.1854934692383 0.100000001490116\\
50.2854957580566 0.100000001490116\\
50.3854942321777 0.100000001490116\\
50.4854927062988 0.100000001490116\\
50.5854949951172 0.100000001490116\\
50.6854934692383 0.100000001490116\\
50.7854957580566 0.100000001490116\\
50.8854942321777 0.100000001490116\\
50.9854927062988 0.100000001490116\\
51.0854949951172 0.100000001490116\\
51.1854934692383 0.100000001490116\\
51.2854957580566 0.100000001490116\\
51.3854942321777 0.100000001490116\\
51.4854927062988 0.100000001490116\\
51.5854949951172 0.100000001490116\\
51.6854934692383 0.100000001490116\\
51.7854957580566 0.100000001490116\\
51.8854942321777 0.100000001490116\\
51.9854927062988 0.100000001490116\\
52.0854949951172 0.100000001490116\\
52.1854934692383 0.100000001490116\\
52.2854957580566 0.100000001490116\\
52.3854942321777 0.100000001490116\\
52.4854927062988 0.100000001490116\\
52.5854949951172 0.100000001490116\\
52.6854934692383 0.100000001490116\\
52.7854957580566 0.100000001490116\\
52.8854942321777 0.100000001490116\\
52.9854927062988 0.100000001490116\\
53.0854949951172 0.100000001490116\\
53.1854934692383 0.100000001490116\\
53.2854957580566 0.100000001490116\\
53.3854942321777 0.100000001490116\\
53.4854927062988 0.100000001490116\\
53.5854949951172 0.100000001490116\\
53.6854934692383 0.100000001490116\\
53.7854957580566 0.100000001490116\\
53.8854942321777 0.100000001490116\\
53.9854927062988 0.100000001490116\\
54.0854949951172 0.100000001490116\\
54.1854934692383 0.100000001490116\\
54.2854957580566 0.100000001490116\\
54.3854942321777 0.100000001490116\\
54.4854927062988 0.100000001490116\\
54.5854949951172 0.100000001490116\\
54.6854934692383 0.100000001490116\\
54.7854957580566 0.100000001490116\\
54.8854942321777 0.100000001490116\\
54.9854927062988 0.100000001490116\\
55.0854949951172 0.100000001490116\\
55.1854934692383 0.100000001490116\\
55.2854957580566 0.100000001490116\\
55.3854942321777 0.100000001490116\\
55.4854927062988 0.100000001490116\\
55.5854949951172 0.100000001490116\\
55.6854934692383 0.100000001490116\\
55.7854957580566 0.100000001490116\\
55.8854942321777 0.100000001490116\\
55.9854927062988 0.100000001490116\\
56.0854949951172 0.100000001490116\\
56.1854934692383 0.100000001490116\\
56.2854957580566 0.100000001490116\\
56.3854942321777 0.100000001490116\\
56.4854927062988 0.100000001490116\\
56.5854949951172 0.100000001490116\\
56.6854934692383 0.100000001490116\\
56.7854957580566 0.100000001490116\\
56.8854942321777 0.100000001490116\\
56.9854927062988 0.100000001490116\\
57.0854949951172 0.100000001490116\\
57.1854934692383 0.100000001490116\\
57.2854957580566 0.100000001490116\\
57.3854942321777 0.100000001490116\\
57.4854927062988 0.100000001490116\\
57.5854949951172 0.100000001490116\\
57.6854934692383 0.100000001490116\\
57.7854957580566 0.100000001490116\\
57.8854942321777 0.100000001490116\\
57.9854927062988 0.100000001490116\\
58.0854949951172 0.100000001490116\\
58.1854934692383 0.100000001490116\\
58.2854957580566 0.100000001490116\\
58.3854942321777 0.100000001490116\\
58.4854927062988 0.100000001490116\\
58.5854949951172 0.100000001490116\\
58.6854934692383 0.100000001490116\\
58.7854957580566 0.100000001490116\\
58.8854942321777 0.100000001490116\\
58.9854927062988 0.100000001490116\\
59.0854949951172 0.100000001490116\\
59.1854934692383 0.100000001490116\\
59.2854957580566 0.100000001490116\\
59.3854942321777 0.100000001490116\\
59.4854927062988 0.100000001490116\\
59.5854949951172 0.100000001490116\\
59.6854934692383 0.100000001490116\\
59.7854957580566 0.100000001490116\\
59.8854942321777 0.100000001490116\\
59.9427452087402 0.0572530552744865\\
60 0.0572530552744865\\
60.0629768371582 0.0629783570766449\\
60.1322555541992 0.0692761912941933\\
60.2084579467773 0.076203815639019\\
60.2922821044922 0.0838241949677467\\
60.3844909667969 0.0922066122293472\\
60.484489440918 0.100000001490116\\
60.5844879150391 0.100000001490116\\
60.6844902038574 0.100000001490116\\
60.7844886779785 0.100000001490116\\
60.8844909667969 0.100000001490116\\
60.984489440918 0.100000001490116\\
61.0844879150391 0.100000001490116\\
61.1844902038574 0.100000001490116\\
61.2844886779785 0.100000001490116\\
61.3734016418457 0.0889127254486084\\
61.444522857666 0.0711205452680588\\
61.5227546691895 0.0782325938344002\\
61.6088104248047 0.0860558524727821\\
61.7034721374512 0.0946614444255829\\
61.8034706115723 0.100000001490116\\
61.9034729003906 0.100000001490116\\
62.0034713745117 0.100000001490116\\
62.1034736633301 0.100000001490116\\
62.2034721374512 0.100000001490116\\
62.3034706115723 0.100000001490116\\
62.4034729003906 0.100000001490116\\
62.5034713745117 0.100000001490116\\
62.6034736633301 0.100000001490116\\
62.7034721374512 0.100000001490116\\
62.8034706115723 0.100000001490116\\
62.9034729003906 0.100000001490116\\
63.0034713745117 0.100000001490116\\
63.1034736633301 0.100000001490116\\
63.2034721374512 0.100000001490116\\
63.3034706115723 0.100000001490116\\
63.4034729003906 0.100000001490116\\
63.5034713745117 0.100000001490116\\
63.6034736633301 0.100000001490116\\
63.7034721374512 0.100000001490116\\
63.8034706115723 0.100000001490116\\
63.9034729003906 0.100000001490116\\
64.0034713745117 0.100000001490116\\
64.1034698486328 0.100000001490116\\
64.2034759521484 0.100000001490116\\
64.3034744262695 0.100000001490116\\
64.4034729003906 0.100000001490116\\
64.5034713745117 0.100000001490116\\
64.6034698486328 0.100000001490116\\
64.7034759521484 0.100000001490116\\
64.8034744262695 0.100000001490116\\
64.9034729003906 0.100000001490116\\
65.0034713745117 0.100000001490116\\
65.1034698486328 0.100000001490116\\
65.2034759521484 0.100000001490116\\
65.3034744262695 0.100000001490116\\
65.4034729003906 0.100000001490116\\
65.5034713745117 0.100000001490116\\
65.6034698486328 0.100000001490116\\
65.7034759521484 0.100000001490116\\
65.8034744262695 0.100000001490116\\
65.9034729003906 0.100000001490116\\
66.0034713745117 0.100000001490116\\
66.1034698486328 0.100000001490116\\
66.2034759521484 0.100000001490116\\
66.3034744262695 0.100000001490116\\
66.4034729003906 0.100000001490116\\
66.5034713745117 0.100000001490116\\
66.6034698486328 0.100000001490116\\
66.7034759521484 0.100000001490116\\
66.8034744262695 0.100000001490116\\
66.9034729003906 0.100000001490116\\
67.0034713745117 0.100000001490116\\
67.1034698486328 0.100000001490116\\
67.2034759521484 0.100000001490116\\
67.3034744262695 0.100000001490116\\
67.4034729003906 0.100000001490116\\
67.5034713745117 0.100000001490116\\
67.6034698486328 0.100000001490116\\
67.7034759521484 0.100000001490116\\
67.8034744262695 0.100000001490116\\
67.9034729003906 0.100000001490116\\
68.0034713745117 0.100000001490116\\
68.1034698486328 0.100000001490116\\
68.2034759521484 0.100000001490116\\
68.3034744262695 0.100000001490116\\
68.4034729003906 0.100000001490116\\
68.5034713745117 0.100000001490116\\
68.6034698486328 0.100000001490116\\
68.7034759521484 0.100000001490116\\
68.8034744262695 0.100000001490116\\
68.9034729003906 0.100000001490116\\
69.0034713745117 0.100000001490116\\
69.1034698486328 0.100000001490116\\
69.2034759521484 0.100000001490116\\
69.3034744262695 0.100000001490116\\
69.4034729003906 0.100000001490116\\
69.5034713745117 0.100000001490116\\
69.6034698486328 0.100000001490116\\
69.7034759521484 0.100000001490116\\
69.8034744262695 0.100000001490116\\
69.9034729003906 0.100000001490116\\
70.0034713745117 0.100000001490116\\
70.1034698486328 0.100000001490116\\
70.2034759521484 0.100000001490116\\
70.3034744262695 0.100000001490116\\
70.4034729003906 0.100000001490116\\
70.5034713745117 0.100000001490116\\
70.6034698486328 0.100000001490116\\
70.7034759521484 0.100000001490116\\
70.8034744262695 0.100000001490116\\
70.9034729003906 0.100000001490116\\
71.0034713745117 0.100000001490116\\
71.1034698486328 0.100000001490116\\
71.2034759521484 0.100000001490116\\
71.3034744262695 0.100000001490116\\
71.4034729003906 0.100000001490116\\
71.5034713745117 0.100000001490116\\
71.6034698486328 0.100000001490116\\
71.7034759521484 0.100000001490116\\
71.8034744262695 0.100000001490116\\
71.9034729003906 0.100000001490116\\
72.0034713745117 0.100000001490116\\
72.1034698486328 0.100000001490116\\
72.2034759521484 0.100000001490116\\
72.3034744262695 0.100000001490116\\
72.4034729003906 0.100000001490116\\
72.5034713745117 0.100000001490116\\
72.6034698486328 0.100000001490116\\
72.7034759521484 0.100000001490116\\
72.8034744262695 0.100000001490116\\
72.9034729003906 0.100000001490116\\
73.0034713745117 0.100000001490116\\
73.1034698486328 0.100000001490116\\
73.2034759521484 0.100000001490116\\
73.3034744262695 0.100000001490116\\
73.4034729003906 0.100000001490116\\
73.5034713745117 0.100000001490116\\
73.5992889404297 0.0958154574036598\\
73.6759490966797 0.0766580924391747\\
73.7424468994141 0.0664994642138481\\
73.8155975341797 0.0731494128704071\\
73.8960571289063 0.0804643556475639\\
73.9845733642578 0.0885107889771461\\
74.0819320678711 0.0973618701100349\\
74.1819305419922 0.100000001490116\\
74.2819290161133 0.100000001490116\\
74.3819351196289 0.100000001490116\\
74.48193359375 0.100000001490116\\
74.5819320678711 0.100000001490116\\
74.6819305419922 0.100000001490116\\
74.7819290161133 0.100000001490116\\
74.8819351196289 0.100000001490116\\
74.9409637451172 0.0590341053903103\\
75 0.0590341053903103\\
};
\addlegendentry{$\text{\$\dt}{}\text{\$}$};

\addplot [
color=green!50!black,
solid
]
table[row sep=crcr]{
0 0\\
9.99999971718069e-10 0\\
2.49999998480632e-09 85000000\\
4.74999994892755e-09 70494680\\
8.1250002281763e-09 37102568\\
1.31874999809156e-08 21690818\\
2.0781250498203e-08 13364090\\
3.21718758300449e-08 8480746\\
4.92578138278077e-08 5478165\\
7.48867208244519e-08 3578038\\
1.13330081319418e-07 2353575.75\\
1.70995122061868e-07 1555275\\
2.57492672517401e-07 1030855.3125\\
3.8723902662241e-07 684636.625\\
5.81858500936505e-07 455314.28125\\
8.73787769251066e-07 303089.5625\\
1.31168167172291e-06 201873.5\\
1.9685226106958e-06 134529.890625\\
2.95378367809462e-06 89681.7890625\\
4.43167573394021e-06 59811.99609375\\
6.64851359033491e-06 39911.61328125\\
9.97376992017962e-06 26650.41796875\\
1.49616553244414e-05 17812.1484375\\
2.24434825213393e-05 11921.1318359375\\
3.36662233166862e-05 7994.34228515625\\
5.05003372381907e-05 5376.78271484375\\
7.57515081204474e-05 3631.95532226563\\
0.000113628258986864 2468.9462890625\\
0.000170443381648511 1693.86401367188\\
0.000255666091106832 1177.48413085938\\
0.000383500126190484 833.691528320313\\
0.000575251178815961 605.118591308594\\
0.000862877757754177 441.295166015625\\
0.00129431765526533 341.489929199219\\
0.00194147753063589 287.555236816406\\
0.00291221728548408 237.401443481445\\
0.00436832662671804 194.81623840332\\
0.00597004732117057 150.654479980469\\
0.00773194013163447 115.550582885742\\
0.009670021943748 86.7357635498047\\
0.0118019115179777 65.6419525146484\\
0.0141469910740852 50.6798439025879\\
0.0167265776544809 40.037727355957\\
0.0195641238242388 32.3421249389648\\
0.0226854234933853 26.6513023376465\\
0.0261188540607691 22.3416385650635\\
0.0298956278711557 19.0009498596191\\
0.0340500771999359 16.354040145874\\
0.0386199727654457 14.2144298553467\\
0.0436468608677387 12.453706741333\\
0.0491764321923256 10.9819717407227\\
0.0552589632570744 9.401442527771\\
0.0619497485458851 7.95069408416748\\
0.0693096145987511 6.92608118057251\\
0.0774054601788521 6.21828985214233\\
0.0863108932971954 5.7397608757019\\
0.0961068719625473 5.42976570129395\\
0.106882445514202 5.03537845611572\\
0.118735581636429 4.56377172470093\\
0.13177402317524 4.15827178955078\\
0.146116316318512 3.78844594955444\\
0.161892831325531 3.45185613632202\\
0.179247006773949 3.15235018730164\\
0.198336601257324 2.88588047027588\\
0.21933513879776 2.64901065826416\\
0.242433547973633 2.43878436088562\\
0.267841786146164 2.25263404846191\\
0.29579085111618 2.08860421180725\\
0.32653483748436 1.94486558437347\\
0.360353201627731 1.81956100463867\\
0.397553414106369 1.71087050437927\\
0.438473641872406 1.61704277992249\\
0.483485877513886 1.53640377521515\\
0.532999336719513 1.46735620498657\\
0.587464153766632 1.40838253498077\\
0.647375464439392 1.35805571079254\\
0.71327793598175 1.31505882740021\\
0.7857705950737 1.27820861339569\\
0.865512549877167 1.24647462368011\\
0.953228712081909 1.21898818016052\\
1.04971647262573 1.19503557682037\\
1.14971649646759 1.17404091358185\\
1.24971640110016 1.17016303539276\\
1.34971642494202 1.17581951618195\\
1.44971644878387 1.1801278591156\\
1.54971647262573 1.18373477458954\\
1.64971649646759 1.18672788143158\\
1.74971640110016 1.18926024436951\\
1.84971642494202 1.19142520427704\\
1.94971644878387 1.19329762458801\\
2.04971647262573 1.19493269920349\\
2.1497163772583 1.19637322425842\\
2.24971652030945 1.19765210151672\\
2.34971642494202 1.19879567623138\\
2.44971656799316 1.1998245716095\\
2.54971647262573 1.20075559616089\\
2.6497163772583 1.20160245895386\\
2.74971652030945 1.20237636566162\\
2.84971642494202 1.20308661460876\\
2.94971656799316 1.20374095439911\\
3.04971647262573 1.20434606075287\\
3.1497163772583 1.20490753650665\\
3.24971652030945 1.20543003082275\\
3.34971642494202 1.20591759681702\\
3.44971656799316 1.20637392997742\\
3.54971647262573 1.20680212974548\\
3.6497163772583 1.2072046995163\\
3.74971652030945 1.20758402347565\\
3.84971642494202 1.20794236660004\\
3.94971656799316 1.20828127861023\\
4.04971647262573 1.20860242843628\\
4.1497163772583 1.20890736579895\\
4.24971628189087 1.20919740200043\\
4.3497166633606 1.20947349071503\\
4.44971656799316 1.20973694324493\\
4.54971647262573 1.20998847484589\\
4.6497163772583 1.21022891998291\\
4.74971628189087 1.21045911312103\\
4.8497166633606 1.21067988872528\\
4.94971656799316 1.21089160442352\\
5.04971647262573 1.2110949754715\\
5.1497163772583 1.21129059791565\\
5.24971628189087 1.21147871017456\\
5.3497166633606 1.21166002750397\\
5.44971656799316 1.21183490753174\\
5.54971647262573 1.21200358867645\\
5.6497163772583 1.21217024326324\\
5.74971628189087 1.21235227584839\\
5.8497166633606 1.21259093284607\\
5.94971656799316 0.569896459579468\\
6.04971647262573 0.229990467429161\\
6.1497163772583 0.16082726418972\\
6.24971628189087 0.148520246148109\\
6.3497166633606 0.150714665651321\\
6.44971656799316 0.175922602415085\\
6.54971647262573 0.461817443370819\\
6.6497163772583 1.2215656042099\\
6.74971628189087 1.26876091957092\\
6.8497166633606 1.72197937965393\\
6.94971656799316 1.7610387802124\\
7.04971647262573 4.28866672515869\\
7.1497163772583 2.92551231384277\\
7.24971628189087 4.25510215759277\\
7.3497166633606 4.16603994369507\\
7.44971656799316 4.24072790145874\\
7.54971647262573 4.2058482170105\\
7.6497163772583 4.24410581588745\\
7.74971628189087 4.23302125930786\\
7.8497166633606 4.25484561920166\\
7.94971656799316 4.25422239303589\\
8.04971599578857 4.26852321624756\\
8.1497163772583 4.27247667312622\\
8.24971675872803 4.28324937820435\\
8.34971618652344 4.28917169570923\\
8.44971656799316 4.2982292175293\\
8.54971599578857 4.30496406555176\\
8.6497163772583 4.31314468383789\\
8.74971675872803 4.32018280029297\\
8.84971618652344 4.3278865814209\\
8.94971656799316 4.33501863479614\\
9.04971599578857 4.35094404220581\\
9.1497163772583 4.48558664321899\\
9.24971675872803 4.55673599243164\\
9.34971618652344 4.64484834671021\\
9.44971656799316 4.72746086120605\\
9.54971599578857 4.81404209136963\\
9.6497163772583 4.90116262435913\\
9.74971675872803 4.97972631454468\\
9.84971618652344 5.03302431106567\\
9.94971656799316 5.03480339050293\\
10.0497159957886 4.92250156402588\\
10.1497163772583 4.6663064956665\\
10.249716758728 4.42234563827515\\
10.3497161865234 4.18890905380249\\
10.4497165679932 4.03391981124878\\
10.5497159957886 3.93355441093445\\
10.6497163772583 3.82016158103943\\
10.749716758728 3.71599841117859\\
10.8497161865234 3.60080862045288\\
10.9497165679932 3.48215436935425\\
11.0497159957886 3.35734128952026\\
11.1497163772583 3.22732996940613\\
11.249716758728 3.09171867370605\\
11.3497161865234 2.95056438446045\\
11.4497165679932 2.80371618270874\\
11.5497159957886 2.5958080291748\\
11.6497163772583 2.1854031085968\\
11.749716758728 1.75571644306183\\
11.8497161865234 1.42471146583557\\
11.9497165679932 1.16299331188202\\
12.0497159957886 0.949612438678741\\
12.1497163772583 0.771006941795349\\
12.249716758728 0.618367552757263\\
12.3471450805664 0.490743786096573\\
12.4250917434692 0.382115751504898\\
12.4879579544067 0.284378916025162\\
12.557110786438 0.305059105157852\\
12.6331787109375 0.483226209878922\\
12.7168531417847 0.612945020198822\\
12.808895111084 0.736161291599274\\
12.9088954925537 0.929702579975128\\
13.0088949203491 1.15895223617554\\
13.1088953018188 1.16163110733032\\
13.2088947296143 1.1675169467926\\
13.308895111084 1.17316389083862\\
13.4088954925537 1.17800879478455\\
13.5088949203491 1.18200838565826\\
13.6088953018188 1.18549430370331\\
13.7088947296143 1.18857502937317\\
13.808895111084 1.19125854969025\\
13.9088954925537 1.19351530075073\\
14.0088949203491 1.19545936584473\\
14.1088953018188 1.19714152812958\\
14.2088947296143 1.1986129283905\\
14.308895111084 1.19990968704224\\
14.4088954925537 1.20106089115143\\
14.5088949203491 1.20208942890167\\
14.6088953018188 1.20301389694214\\
14.7088947296143 1.20384907722473\\
14.808895111084 1.20460712909698\\
14.9088954925537 1.20529842376709\\
15.0088949203491 1.20593118667603\\
15.1088953018188 1.20651257038116\\
15.2088947296143 1.20704853534698\\
15.308895111084 1.20754408836365\\
15.4088954925537 1.20800387859344\\
15.5088949203491 1.20843136310577\\
15.6088953018188 1.20867621898651\\
15.7088947296143 1.20911610126495\\
15.808895111084 1.20945584774017\\
15.9088954925537 1.20979905128479\\
16.0088958740234 1.21011316776276\\
16.1088943481445 1.21041190624237\\
16.2088947296143 1.21069264411926\\
16.308895111084 1.21095812320709\\
16.4088954925537 1.21120929718018\\
16.5088958740234 1.21144723892212\\
16.6088943481445 1.21167314052582\\
16.7088947296143 1.21188771724701\\
16.808895111084 1.21209192276001\\
16.9088954925537 1.21228647232056\\
17.0088958740234 1.2124719619751\\
17.1088943481445 1.21264898777008\\
17.2088947296143 1.21281826496124\\
17.308895111084 1.21298015117645\\
17.4088954925537 1.21313524246216\\
17.5088958740234 1.21328377723694\\
17.6088943481445 1.21342635154724\\
17.7088947296143 1.21356332302094\\
17.808895111084 1.21369481086731\\
17.9088954925537 1.21383059024811\\
18.0088958740234 1.21398985385895\\
18.1088943481445 1.21423876285553\\
18.2088947296143 0.404637038707733\\
18.308895111084 0.196825355291367\\
18.4088954925537 0.155300840735435\\
18.5088958740234 0.149005115032196\\
18.6088943481445 0.154574364423752\\
18.7088947296143 0.204286828637123\\
18.808895111084 0.927395105361938\\
18.9088954925537 2.83518004417419\\
19.0088958740234 2.91456174850464\\
19.1088943481445 3.21674060821533\\
19.2088947296143 2.69576859474182\\
19.308895111084 4.28575277328491\\
19.4088954925537 4.17252349853516\\
19.5088958740234 4.26479005813599\\
19.6088943481445 4.22131443023682\\
19.7088947296143 4.26735496520996\\
19.808895111084 4.25383710861206\\
19.9088954925537 4.27947902679443\\
20.0088958740234 4.2788200378418\\
20.1088943481445 4.29536819458008\\
20.2088947296143 4.30018854141235\\
20.308895111084 4.31260442733765\\
20.4088954925537 4.31967067718506\\
20.5088958740234 4.33007431030273\\
20.6088943481445 4.33806943893433\\
20.7088947296143 4.34756946563721\\
20.808895111084 4.35593223571777\\
20.9088954925537 4.36497068405151\\
21.0088958740234 4.37347030639648\\
21.1088943481445 4.38228178024292\\
21.2088947296143 4.39086627960205\\
21.308895111084 4.42577171325684\\
21.4088954925537 4.55558252334595\\
21.5088958740234 4.62860059738159\\
21.6088943481445 4.71782684326172\\
21.7088947296143 4.80273103713989\\
21.808895111084 4.64834070205688\\
21.9088954925537 4.45325899124146\\
22.0088958740234 4.26651334762573\\
22.1088943481445 4.08707618713379\\
22.2088947296143 3.91357016563416\\
22.308895111084 3.74512767791748\\
22.4088954925537 3.58099842071533\\
22.5088958740234 3.42061877250671\\
22.6088943481445 3.26353621482849\\
22.7088947296143 3.15284895896912\\
22.808895111084 3.04062795639038\\
22.9088954925537 2.94252467155457\\
23.0088958740234 2.84107232093811\\
23.1088943481445 2.7369441986084\\
23.2088947296143 2.63003325462341\\
23.308895111084 2.52039885520935\\
23.4088954925537 2.40800976753235\\
23.5088958740234 2.2929675579071\\
23.6088943481445 2.17515301704407\\
23.7088947296143 2.05447292327881\\
23.808895111084 1.89807093143463\\
23.9088954925537 1.65321934223175\\
24.0088958740234 1.3998726606369\\
24.1088943481445 1.18440771102905\\
24.2088947296143 0.999405980110168\\
24.308895111084 0.838077008724213\\
24.4088954925537 0.695396840572357\\
24.5088958740234 0.567694962024689\\
24.6048831939697 0.457048833370209\\
24.681676864624 0.358512878417969\\
24.7464084625244 0.294702798128128\\
24.8176136016846 0.31350639462471\\
24.895938873291 0.504592835903168\\
24.9820976257324 0.614853620529175\\
25.0768718719482 0.748729169368744\\
25.176872253418 0.925521790981293\\
25.2768707275391 1.1586719751358\\
25.3768711090088 1.16271007061005\\
25.4768714904785 1.1686829328537\\
25.5768718719482 1.17418205738068\\
25.676872253418 1.17888736724854\\
25.7768707275391 1.1828134059906\\
25.8768711090088 1.18609321117401\\
25.9768714904785 1.18898212909698\\
26.0768718719482 1.1914781332016\\
26.176872253418 1.19373285770416\\
26.2768707275391 1.19563567638397\\
26.3768711090088 1.19729816913605\\
26.4768714904785 1.19875347614288\\
26.5768718719482 1.20003747940063\\
26.676872253418 1.20117855072021\\
26.7768707275391 1.20219886302948\\
26.8768711090088 1.20311653614044\\
26.9768714904785 1.203946352005\\
27.0768718719482 1.20470011234283\\
27.176872253418 1.20538771152496\\
27.2768707275391 1.20601761341095\\
27.3768711090088 1.2065966129303\\
27.4768714904785 1.20713067054749\\
27.5768718719482 1.20762479305267\\
27.676872253418 1.20808339118958\\
27.7768707275391 1.20850992202759\\
27.8768711090088 1.20890772342682\\
27.9768714904785 1.20927953720093\\
28.0768718719482 1.20962798595428\\
28.176872253418 1.20995497703552\\
28.2768707275391 1.21026253700256\\
28.3768711090088 1.21055245399475\\
28.4768714904785 1.21066677570343\\
28.5768718719482 1.21099030971527\\
28.676872253418 1.21122121810913\\
28.7768707275391 1.21146643161774\\
28.8768711090088 1.21168994903564\\
28.9768714904785 1.21190559864044\\
29.0768718719482 1.21210968494415\\
29.176872253418 1.21230447292328\\
29.2768707275391 1.2124902009964\\
29.3768711090088 1.21266758441925\\
29.4768714904785 1.21283710002899\\
29.5768718719482 1.21299934387207\\
29.676872253418 1.21315467357636\\
29.7768707275391 1.213303565979\\
29.8768711090088 1.21344637870789\\
29.9768714904785 1.21358358860016\\
30.0768718719482 1.21371686458588\\
30.176872253418 1.21386086940765\\
30.2768707275391 1.21404349803925\\
30.3768711090088 0.824299871921539\\
30.4768714904785 0.284648150205612\\
30.5768718719482 0.172322496771812\\
30.676872253418 0.150880739092827\\
30.7768707275391 0.149873852729797\\
30.8768711090088 0.163130253553391\\
30.9768714904785 0.30712702870369\\
31.0768718719482 1.418536901474\\
31.176872253418 2.35184550285339\\
31.2768707275391 3.82349109649658\\
31.3768711090088 4.31609439849854\\
31.4768714904785 3.20206785202026\\
31.5768718719482 4.27649688720703\\
31.676872253418 4.19581699371338\\
31.7768707275391 4.26772546768188\\
31.8768711090088 4.23868036270142\\
31.9768714904785 4.27597045898438\\
32.0768699645996 4.26906776428223\\
32.176872253418 4.29107189178467\\
32.2768707275391 4.29360628128052\\
32.3768730163574 4.30877733230591\\
32.4768714904785 4.31528472900391\\
32.5768699645996 4.32731103897095\\
32.676872253418 4.33536243438721\\
32.7768707275391 4.3458776473999\\
32.8768730163574 4.35462045669556\\
32.9768714904785 4.36443471908569\\
33.0768699645996 4.3734245300293\\
33.176872253418 4.38289642333984\\
33.2768707275391 4.39198637008667\\
33.3768730163574 4.40130758285522\\
33.4768714904785 4.4105372428894\\
33.5768699645996 4.49715232849121\\
33.676872253418 4.59931993484497\\
33.7768707275391 4.68041276931763\\
33.8768730163574 4.59137678146362\\
33.9768714904785 4.41102981567383\\
34.0768699645996 4.23841619491577\\
34.176872253418 4.07290983200073\\
34.2768707275391 3.91290187835693\\
34.3768730163574 3.7575318813324\\
34.4768714904785 3.6059992313385\\
34.5768699645996 3.45772385597229\\
34.676872253418 3.31223678588867\\
34.7768707275391 3.16917586326599\\
34.8768730163574 3.02825689315796\\
34.9768714904785 2.94838762283325\\
35.0768699645996 2.81869053840637\\
35.176872253418 2.73253774642944\\
35.2768707275391 2.62850737571716\\
35.3768730163574 2.52772331237793\\
35.4768714904785 2.4227933883667\\
35.5768699645996 2.31631922721863\\
35.676872253418 2.20737051963806\\
35.7768707275391 2.09625101089478\\
35.8768730163574 1.98281788825989\\
35.9768714904785 1.8670734167099\\
36.0768699645996 1.70423758029938\\
36.176872253418 1.47838163375854\\
36.2768707275391 1.26714909076691\\
36.3768730163574 1.08275246620178\\
36.4768714904785 0.920351326465607\\
36.5768699645996 0.775507390499115\\
36.676872253418 0.644955396652222\\
36.7768707275391 0.526162087917328\\
36.8673248291016 0.424729198217392\\
36.9396858215332 0.332142859697342\\
37.0192794799805 0.333460032939911\\
37.1068305969238 0.325561821460724\\
37.2031402587891 0.568015158176422\\
37.3031425476074 0.676056802272797\\
37.4031410217285 0.896046340465546\\
37.5031433105469 1.10849940776825\\
37.603141784668 1.16196322441101\\
37.7031402587891 1.16790306568146\\
37.8031425476074 1.1735235452652\\
37.9031410217285 1.17831981182098\\
38.0031433105469 1.18238985538483\\
38.103141784668 1.1856894493103\\
38.2031402587891 1.18862688541412\\
38.3031425476074 1.19116282463074\\
38.4031410217285 1.19341146945953\\
38.5031433105469 1.19537472724915\\
38.603141784668 1.19706594944\\
38.7031402587891 1.19855034351349\\
38.8031425476074 1.19985854625702\\
38.9031410217285 1.20101976394653\\
39.0031433105469 1.20205724239349\\
39.103141784668 1.20298969745636\\
39.2031402587891 1.20383214950562\\
39.3031425476074 1.20459687709808\\
39.4031410217285 1.20529425144196\\
39.5031433105469 1.20593249797821\\
39.603141784668 1.2065190076828\\
39.7031402587891 1.20705962181091\\
39.8031425476074 1.20755970478058\\
39.9031410217285 1.20802342891693\\
40.0031433105469 1.20845484733582\\
40.103141784668 1.20885694026947\\
40.2031402587891 1.20923268795013\\
40.3031425476074 1.20958459377289\\
40.4031410217285 1.20991480350494\\
40.5031433105469 1.21022534370422\\
40.603141784668 1.21051788330078\\
40.7031402587891 1.21079385280609\\
40.8031425476074 1.21105480194092\\
40.9031410217285 1.21130168437958\\
41.0031433105469 1.21153581142426\\
41.103141784668 1.21161329746246\\
41.2031402587891 1.21188318729401\\
41.3031425476074 1.21207070350647\\
41.4031410217285 1.21227383613586\\
41.5031433105469 1.21245884895325\\
41.603141784668 1.2126384973526\\
41.7031402587891 1.21280908584595\\
41.8031425476074 1.2129727602005\\
41.9031410217285 1.21312928199768\\
42.0031433105469 1.21327936649323\\
42.103141784668 1.213423371315\\
42.2031402587891 1.21356153488159\\
42.3031425476074 1.21369481086731\\
42.4031410217285 1.21383559703827\\
42.5031433105469 1.21400666236877\\
42.603141784668 1.05263650417328\\
42.7031402587891 0.333202362060547\\
42.8031425476074 0.182096183300018\\
42.9031410217285 0.152573525905609\\
43.0031433105469 0.149262458086014\\
43.103141784668 0.158374831080437\\
43.2031402587891 0.24403940141201\\
43.3031425476074 1.12979364395142\\
43.4031410217285 3.0383894443512\\
43.5031433105469 3.59078884124756\\
43.603141784668 3.88610577583313\\
43.7031402587891 2.82533621788025\\
43.8031425476074 4.29283952713013\\
43.9031410217285 4.18750905990601\\
44.0031433105469 4.27593564987183\\
44.103141784668 4.23649740219116\\
44.2031402587891 4.2811107635498\\
44.3031425476074 4.26991081237793\\
44.4031410217285 4.29525089263916\\
44.5031433105469 4.29610013961792\\
44.603141784668 4.31288433074951\\
44.7031402587891 4.31882858276367\\
44.8031425476074 4.33173942565918\\
44.9031410217285 4.33971214294434\\
45.0031433105469 4.35077524185181\\
45.103141784668 4.35965776443481\\
45.2031402587891 4.36988592147827\\
45.3031425476074 4.37911224365234\\
45.4031410217285 4.38894557952881\\
45.5031433105469 4.39832210540771\\
45.603141784668 4.40799045562744\\
45.7031402587891 4.41751432418823\\
45.8031425476074 4.47744369506836\\
45.9031410217285 4.59556102752686\\
46.0031433105469 4.61039018630981\\
46.103141784668 4.43022632598877\\
46.2031402587891 4.26346588134766\\
46.3031425476074 4.10252618789673\\
46.4031410217285 3.94788432121277\\
46.5031433105469 3.79781651496887\\
46.603141784668 3.65169239044189\\
46.7031402587891 3.50878119468689\\
46.8031425476074 3.36859512329102\\
46.9031410217285 3.23072457313538\\
47.0031433105469 3.09485912322998\\
47.103141784668 2.96075367927551\\
47.2031402587891 2.86838793754578\\
47.3031425476074 2.74834084510803\\
47.4031410217285 2.65844511985779\\
47.5031433105469 2.55787873268127\\
47.603141784668 2.458327293396\\
47.7031402587891 2.35576319694519\\
47.8031425476074 2.25160646438599\\
47.9031410217285 2.14534831047058\\
48.0031433105469 2.03714823722839\\
48.103141784668 1.92692160606384\\
48.2031402587891 1.81465971469879\\
48.3031425476074 1.67341387271881\\
48.4031410217285 1.47451114654541\\
48.5031433105469 1.2729332447052\\
48.603141784668 1.09470844268799\\
48.7031402587891 0.936278700828552\\
48.8031425476074 0.793936729431152\\
48.9031410217285 0.664801061153412\\
49.0031433105469 0.546647846698761\\
49.0980262756348 0.443241447210312\\
49.1739387512207 0.348978370428085\\
49.2413482666016 0.303905159235001\\
49.315502166748 0.315572023391724\\
49.3970718383789 0.518777251243591\\
49.4867973327637 0.620157778263092\\
49.5854949951172 0.769343554973602\\
49.6854934692383 0.946607828140259\\
49.7854957580566 1.15899050235748\\
49.8854942321777 1.16400122642517\\
49.9854927062988 1.16996610164642\\
50.0854949951172 1.17527377605438\\
50.1854934692383 1.17982172966003\\
50.2854957580566 1.18359804153442\\
50.3854942321777 1.18675947189331\\
50.4854927062988 1.18954503536224\\
50.5854949951172 1.19195568561554\\
50.6854934692383 1.19410645961761\\
50.7854957580566 1.19597101211548\\
50.8854942321777 1.19758772850037\\
50.9854927062988 1.19900906085968\\
51.0854949951172 1.20026516914368\\
51.1854934692383 1.20138263702393\\
51.2854957580566 1.20238304138184\\
51.3854942321777 1.20328390598297\\
51.4854927062988 1.20409905910492\\
51.5854949951172 1.20484018325806\\
51.6854934692383 1.20551693439484\\
51.7854957580566 1.20613718032837\\
51.8854942321777 1.20670771598816\\
51.9854927062988 1.20723426342011\\
52.0854949951172 1.20772182941437\\
52.1854934692383 1.2081743478775\\
52.2854957580566 1.20859563350677\\
52.3854942321777 1.20898866653442\\
52.4854927062988 1.20935618877411\\
52.5854949951172 1.20970070362091\\
52.6854934692383 1.21002423763275\\
52.7854957580566 1.21032857894897\\
52.8854942321777 1.21061551570892\\
52.9854927062988 1.21088635921478\\
53.0854949951172 1.21114242076874\\
53.1854934692383 1.21138501167297\\
53.2854957580566 1.21161508560181\\
53.3854942321777 1.21183359622955\\
53.4854927062988 1.21190214157104\\
53.5854949951172 1.21215617656708\\
53.6854934692383 1.21233153343201\\
53.7854957580566 1.21252238750458\\
53.8854942321777 1.21269631385803\\
53.9854927062988 1.21286559104919\\
54.0854949951172 1.2130264043808\\
54.1854934692383 1.21318089962006\\
54.2854957580566 1.21332883834839\\
54.3854942321777 1.21347093582153\\
54.4854927062988 1.21360731124878\\
54.5854949951172 1.21374094486237\\
54.6854934692383 1.21388792991638\\
54.7854957580566 1.21408188343048\\
54.8854942321777 0.690327227115631\\
54.9854927062988 0.256180375814438\\
55.0854949951172 0.166727423667908\\
55.1854934692383 0.150058075785637\\
55.2854957580566 0.150691375136375\\
55.3854942321777 0.168208986520767\\
55.4854927062988 0.319592535495758\\
55.5854949951172 0.711706161499023\\
55.6854934692383 0.910080671310425\\
55.7854957580566 1.57912003993988\\
55.8854942321777 2.17107820510864\\
55.9854927062988 4.33444213867188\\
56.0854949951172 3.57229375839233\\
56.1854934692383 4.2912220954895\\
56.2854957580566 4.22108173370361\\
56.3854942321777 4.28601789474487\\
56.4854927062988 4.26176738739014\\
56.5854949951172 4.29599618911743\\
56.6854934692383 4.29133319854736\\
56.7854957580566 4.31207227706909\\
56.8854942321777 4.31573438644409\\
56.9854927062988 4.33047723770142\\
57.0854949951172 4.3376088142395\\
57.1854934692383 4.34955787658691\\
57.2854957580566 4.35813283920288\\
57.3854942321777 4.36886310577393\\
57.4854927062988 4.37804937362671\\
57.5854949951172 4.38820171356201\\
57.6854934692383 4.39765453338623\\
57.7854957580566 4.4075608253479\\
57.8854942321777 4.41716432571411\\
57.9854927062988 4.42709541320801\\
58.0854949951172 4.53120899200439\\
58.1854934692383 4.5854115486145\\
58.2854957580566 4.41473054885864\\
58.3854942321777 4.25136423110962\\
58.4854927062988 4.09569215774536\\
58.5854949951172 3.94553875923157\\
58.6854934692383 3.80004715919495\\
58.7854957580566 3.658282995224\\
58.8854942321777 3.51961207389832\\
58.9854927062988 3.38350868225098\\
59.0854949951172 3.24957346916199\\
59.1854934692383 3.11749076843262\\
59.2854957580566 2.98701643943787\\
59.3854942321777 2.85796093940735\\
59.4854927062988 2.77590370178223\\
59.5854949951172 2.65678095817566\\
59.6854934692383 2.56895279884338\\
59.7854957580566 2.467928647995\\
59.8854942321777 2.36904692649841\\
59.9427452087402 2.26697778701782\\
60 2.21854257583618\\
60.0629768371582 2.16835951805115\\
60.1322555541992 2.09641766548157\\
60.2084579467773 2.02029848098755\\
60.2922821044922 1.93489265441895\\
60.3844909667969 1.84019470214844\\
60.484489440918 1.73466420173645\\
60.5844879150391 1.57982110977173\\
60.6844902038574 1.37951159477234\\
60.7844886779785 1.19468307495117\\
60.8844909667969 1.02953791618347\\
60.984489440918 0.881291270256042\\
61.0844879150391 0.746779680252075\\
61.1844902038574 0.623743653297424\\
61.2844886779785 0.510333597660065\\
61.3734016418457 0.413161545991898\\
61.444522857666 0.323553651571274\\
61.5227546691895 0.346300065517426\\
61.6088104248047 0.344214648008347\\
61.7034721374512 0.567129671573639\\
61.8034706115723 0.67728328704834\\
61.9034729003906 0.884317457675934\\
62.0034713745117 1.0985780954361\\
62.1034736633301 1.1620854139328\\
62.2034721374512 1.16802680492401\\
62.3034706115723 1.17361605167389\\
62.4034729003906 1.17839860916138\\
62.5034713745117 1.18245685100555\\
62.6034736633301 1.18574905395508\\
62.7034721374512 1.18866527080536\\
62.8034706115723 1.19119012355804\\
62.9034729003906 1.19340312480927\\
63.0034713745117 1.19537734985352\\
63.1034736633301 1.19706666469574\\
63.2034721374512 1.19855010509491\\
63.3034706115723 1.19985914230347\\
63.4034729003906 1.20102059841156\\
63.5034713745117 1.20205855369568\\
63.6034736633301 1.20299124717712\\
63.7034721374512 1.20383393764496\\
63.8034706115723 1.2045990228653\\
63.9034729003906 1.20529651641846\\
64.0034713745117 1.20593500137329\\
64.1034698486328 1.20652174949646\\
64.2034759521484 1.20706260204315\\
64.3034744262695 1.2075629234314\\
64.4034729003906 1.20802688598633\\
64.5034713745117 1.2084584236145\\
64.6034698486328 1.20886063575745\\
64.7034759521484 1.20923662185669\\
64.8034744262695 1.20958864688873\\
64.9034729003906 1.20991909503937\\
65.0034713745117 1.21022975444794\\
65.1034698486328 1.21052241325378\\
65.2034759521484 1.21079850196838\\
65.3034744262695 1.2110595703125\\
65.4034729003906 1.21130669116974\\
65.5034713745117 1.21154081821442\\
65.6034698486328 1.21176314353943\\
65.7034759521484 1.2119745016098\\
65.8034744262695 1.21217560768127\\
65.9034729003906 1.21223425865173\\
66.0034713745117 1.2124707698822\\
66.1034698486328 1.21263229846954\\
66.2034759521484 1.2128096818924\\
66.3034744262695 1.21297109127045\\
66.4034729003906 1.21312856674194\\
66.5034713745117 1.2132785320282\\
66.6034698486328 1.21342265605927\\
66.7034759521484 1.21356093883514\\
66.8034744262695 1.21369421482086\\
66.9034729003906 1.21383476257324\\
67.0034713745117 1.21400475502014\\
67.1034698486328 1.08272671699524\\
67.2034759521484 0.339366883039474\\
67.3034744262695 0.183358520269394\\
67.4034729003906 0.152816817164421\\
67.5034713745117 0.149240598082542\\
67.6034698486328 0.158021062612534\\
67.7034759521484 0.239138424396515\\
67.8034744262695 1.10489594936371\\
67.9034729003906 3.08294558525085\\
68.0034713745117 3.56051683425903\\
68.1034698486328 3.71026849746704\\
68.2034759521484 2.80952620506287\\
68.3034744262695 4.29747104644775\\
68.4034729003906 4.18997192382813\\
68.5034713745117 4.28029823303223\\
68.6034698486328 4.2402024269104\\
68.7034759521484 4.28575038909912\\
68.8034744262695 4.27449893951416\\
68.9034729003906 4.30038356781006\\
69.0034713745117 4.30141735076904\\
69.1034698486328 4.31859636306763\\
69.2034759521484 4.32481575012207\\
69.3034744262695 4.33807039260864\\
69.4034729003906 4.34636116027832\\
69.5034713745117 4.35776138305664\\
69.6034698486328 4.36697816848755\\
69.7034759521484 4.3775486946106\\
69.8034744262695 4.38712549209595\\
69.9034729003906 4.39731597900391\\
70.0034713745117 4.40706205368042\\
70.1034698486328 4.41710901260376\\
70.2034759521484 4.4270224571228\\
70.3034744262695 4.48339891433716\\
70.4034729003906 4.55785274505615\\
70.5034713745117 4.39673566818237\\
70.6034698486328 4.23281192779541\\
70.7034759521484 4.0800986289978\\
70.8034744262695 3.93190884590149\\
70.9034729003906 3.78880858421326\\
71.0034713745117 3.64934754371643\\
71.1034698486328 3.51303124427795\\
71.2034759521484 3.37926125526428\\
71.3034744262695 3.24764275550842\\
71.4034729003906 3.11784243583679\\
71.5034713745117 2.98960947990417\\
71.6034698486328 2.86274576187134\\
71.7034759521484 2.7682056427002\\
71.8034744262695 2.65354633331299\\
71.9034729003906 2.56262278556824\\
72.0034713745117 2.46412110328674\\
72.1034698486328 2.36590909957886\\
72.2034759521484 2.26530838012695\\
72.3034744262695 2.16326212882996\\
72.4034729003906 2.05943131446838\\
72.5034713745117 1.9539178609848\\
72.6034698486328 1.84666204452515\\
72.7034759521484 1.73765349388123\\
72.8034744262695 1.60612034797668\\
72.9034729003906 1.42569875717163\\
73.0034713745117 1.23965847492218\\
73.1034698486328 1.07281708717346\\
73.2034759521484 0.922658205032349\\
73.3034744262695 0.786302387714386\\
73.4034729003906 0.66143798828125\\
73.5034713745117 0.546256899833679\\
73.5992889404297 0.444305121898651\\
73.6759490966797 0.351004213094711\\
73.7424468994141 0.298736333847046\\
73.8155975341797 0.314414352178574\\
73.8960571289063 0.509216666221619\\
73.9845733642578 0.614541888237\\
74.0819320678711 0.754121422767639\\
74.1819305419922 0.928532660007477\\
74.2819290161133 1.1588169336319\\
74.3819351196289 1.16353833675385\\
74.48193359375 1.16950809955597\\
74.5819320678711 1.17488503456116\\
74.6819305419922 1.17948806285858\\
74.7819290161133 1.18334412574768\\
74.8819351196289 1.18651950359344\\
74.9409637451172 1.18933439254761\\
75 1.23904156684875\\
};
\addlegendentry{$\text{\$\dtcrnt}{}\text{\$}$};

\end{axis}
\end{tikzpicture}%
\caption{Comparison of \dt{} to \dtcrnt{} for the nonlinear solver with \dtmax{} = \expneg{1.0}{1}{[s]}.}
\label{fig:vmpDeltaTNln1em1}
\end{figure}

The nonlinear solver actually took fewer total timesteps than the linear solver in this simulation.
This was possible because the calculated \dtcrnt{} limit is different for the nonlinear case.
The larger \dtcrnt{} is possible because the nonphysical oscillations in the linear solution's pressures are no longer present.
These deviations had induced large velocities, which had reduced \dtcrnt{} below \dtmax{} causing \cobra{} to reduce the timestep size repeatedly.
The nonlinear solver produces qualitatively and quantitatively similar solutions at more restrictive \dtmax{}.
At the smaller \dtmax{} both the linear and the nonlinear solver's solutions had similar error measures, approximately \expneg{3.5}{2}{[psia]}.

Reducing the nonlinear error at each timestep allows \cobra{} to run at larger \dt{} without a reduction in the accuracy of the solution.
However, the use of the nonlinear solver imposes additional computational costs.
These costs are evident in \tab{tab:vmpNlnRunTime}.
The run time per timestep of the nonlinear solver for this problem is on average 5.5 times that of the linear solver.

\begin{table}[h!tb]
\centering
\singlespace
\pgfplotstabletypeset[sci zerofill,sci E, col sep=comma,
	columns/0/.style={ column name= $[ \text{s} ]$, precision=1},
	columns/1/.style={ column name= $[ - ]$, int detect},
	columns/2/.style={ column name= $[ \text{s} ]$, precision=3},
	columns/3/.style={ column name= $[ \text{s} ]$, precision=3},
	every head row/.style={
		before row={
			\toprule
			\dtmax{} & $N_{t}$ & $T_{\text{CPU}}$& $\frac{T_{\text{CPU}}}{N_{t}}$\\
		},
		after row=\midrule
	},
	every last row/.style={
after row=\bottomrule}]{tables/vmpNlnRunTimeData.tex}
\caption{Nonlinear solver's data for the fill problem.}
\label{tab:vmpNlnRunTime}
\end{table}

These results make a case that using the nonlinear solver at large timestep sizes can produce more accurate solutions that are comparable to those from the linear solver at smaller timesteps.
The application of the domain decomposition algorithm provides more insight into impact of unresolved local nonlinearities.
The domain decomposition simulations were run with \dtmax{} = \expneg{1.0}{1}{[s]}.
The first simulation to be examined will be the one with the bottom channel in the nonlinear domain, \fig{fig:vmpDDBotChan}.

\begin{figure}[h!tb]
\centering
% This file was created by matlab2tikz v0.4.3.
% Copyright (c) 2008--2013, Nico Schlömer <nico.schloemer@gmail.com>
% All rights reserved.
% 
\tikzsetnextfilename{plots/vmpBotNLN1em1_eps}
%
% defining custom colors
\definecolor{mycolor1}{rgb}{0.800000011920929,0.800000011920929,0.800000011920929}%
%
\begin{tikzpicture}

\begin{axis}[%
width=0.8\textwidth,
height=0.630967741935484\textwidth,
scale only axis,
xmin=0,
xmax=75,
xlabel={Time [s]},
ymin=14.6,
ymax=15.8,
ylabel={Pressure [psia]},
legend style={at={(0.03,0.97)},anchor=north west,draw=black,fill=white,legend cell align=left}
]
\addplot [
color=black,
solid
]
table[row sep=crcr]{
0 14.7007818222046\\
9.99999971718069e-10 14.7007818222046\\
2.49999998480632e-09 14.7007818222046\\
4.74999994892755e-09 14.7007818222046\\
8.1250002281763e-09 14.7007818222046\\
1.31874999809156e-08 14.7007818222046\\
2.0781250498203e-08 14.7007818222046\\
3.21718758300449e-08 14.7007818222046\\
4.92578138278077e-08 14.7007818222046\\
7.48867208244519e-08 14.7007818222046\\
1.13330081319418e-07 14.7007818222046\\
1.70995122061868e-07 14.7007818222046\\
2.57492672517401e-07 14.7007818222046\\
3.8723902662241e-07 14.7007818222046\\
5.81858500936505e-07 14.7007818222046\\
8.73787769251066e-07 14.7007818222046\\
1.31168167172291e-06 14.7007818222046\\
1.9685226106958e-06 14.7007827758789\\
2.95378367809462e-06 14.7007827758789\\
4.43167573394021e-06 14.7007827758789\\
6.64851359033491e-06 14.7007827758789\\
9.97376992017962e-06 14.7007837295532\\
1.49616553244414e-05 14.7007846832275\\
2.24434825213393e-05 14.7007856369019\\
3.36662233166862e-05 14.7007865905762\\
5.05003372381907e-05 14.7007884979248\\
7.57515081204474e-05 14.7007904052734\\
0.000113628258986864 14.7007932662964\\
0.000170443381648511 14.700795173645\\
0.000255666091106832 14.7007970809937\\
0.000383500126190484 14.7007970809937\\
0.000575251178815961 14.7007942199707\\
0.000862877757754177 14.7007904052734\\
0.00129431765526533 14.7007865905762\\
0.00194147753063589 14.7007837295532\\
0.00291221728548408 14.700779914856\\
0.00436832662671804 14.7007780075073\\
0.00597004732117057 14.700779914856\\
0.00773194013163447 14.7007827758789\\
0.009670021943748 14.7007837295532\\
0.0118019115179777 14.7007827758789\\
0.0141469910740852 14.7007827758789\\
0.0167265776544809 14.7007827758789\\
0.0195641238242388 14.7007827758789\\
0.0226854234933853 14.7007827758789\\
0.0261188540607691 14.7007827758789\\
0.0298956278711557 14.7007827758789\\
0.0340500771999359 14.7007827758789\\
0.0386199727654457 14.7007827758789\\
0.0436468608677387 14.7007827758789\\
0.0491764321923256 14.7007827758789\\
0.0552589632570744 14.7007827758789\\
0.0619497485458851 14.7007827758789\\
0.0693096145987511 14.7007827758789\\
0.0774054601788521 14.700777053833\\
0.0863108932971954 14.7007865905762\\
0.0961068719625473 14.7007827758789\\
0.106882445514202 14.7007827758789\\
0.118735581636429 14.7007827758789\\
0.13177402317524 14.7007827758789\\
0.146116316318512 14.7007827758789\\
0.161892831325531 14.7007827758789\\
0.179247006773949 14.7007827758789\\
0.198336601257324 14.7007827758789\\
0.21933513879776 14.7007827758789\\
0.242433547973633 14.7007827758789\\
0.267841786146164 14.7007827758789\\
0.29579085111618 14.7007827758789\\
0.32653483748436 14.7007827758789\\
0.360353201627731 14.7007827758789\\
0.397553414106369 14.7007837295532\\
0.438473641872406 14.7007837295532\\
0.483485877513886 14.7007837295532\\
0.532999336719513 14.7007837295532\\
0.587464153766632 14.7007837295532\\
0.647375464439392 14.7007837295532\\
0.71327793598175 14.7007837295532\\
0.7857705950737 14.7007837295532\\
0.865512549877167 14.7007837295532\\
0.953228712081909 14.7007837295532\\
1.04971647262573 14.7007837295532\\
1.14971649646759 14.7007827758789\\
1.24971640110016 14.7007827758789\\
1.34971642494202 14.7007827758789\\
1.44971644878387 14.7007827758789\\
1.54971647262573 14.7007827758789\\
1.64971649646759 14.7007827758789\\
1.74971640110016 14.7007827758789\\
1.84971642494202 14.7007827758789\\
1.94971644878387 14.7007827758789\\
2.04971647262573 14.7007827758789\\
2.1497163772583 14.7007827758789\\
2.24971652030945 14.7007827758789\\
2.34971642494202 14.7007827758789\\
2.44971656799316 14.7007827758789\\
2.54971647262573 14.7007827758789\\
2.6497163772583 14.7007827758789\\
2.74971652030945 14.7007827758789\\
2.84971642494202 14.7007827758789\\
2.94971656799316 14.7007827758789\\
3.04971647262573 14.7007827758789\\
3.1497163772583 14.7007827758789\\
3.24971652030945 14.7007827758789\\
3.34971642494202 14.7007827758789\\
3.44971656799316 14.7007827758789\\
3.54971647262573 14.7007827758789\\
3.6497163772583 14.7007827758789\\
3.74971652030945 14.7007827758789\\
3.84971642494202 14.7007827758789\\
3.94971656799316 14.7007827758789\\
4.04971647262573 14.7007827758789\\
4.1497163772583 14.7007827758789\\
4.24971628189087 14.7007827758789\\
4.3497166633606 14.7007827758789\\
4.44971656799316 14.7007827758789\\
4.54971647262573 14.7007827758789\\
4.6497163772583 14.7007827758789\\
4.74971628189087 14.7007827758789\\
4.8497166633606 14.7007827758789\\
4.94971656799316 14.7007827758789\\
5.04971647262573 14.7007837295532\\
5.1497163772583 14.7007837295532\\
5.24971628189087 14.7007837295532\\
5.3497166633606 14.7007837295532\\
5.44971656799316 14.7007837295532\\
5.54971647262573 14.7007837295532\\
5.6497163772583 14.7007846832275\\
5.74971628189087 14.7007894515991\\
5.8497166633606 14.7008085250854\\
5.94971656799316 14.7009077072144\\
6.04971647262573 14.7014436721802\\
6.1497163772583 14.7040977478027\\
6.24971628189087 14.7128324508667\\
6.3497166633606 14.7271690368652\\
6.44971656799316 14.7334566116333\\
6.54971647262573 14.7314443588257\\
6.6497163772583 14.7334966659546\\
6.74971628189087 14.7322854995728\\
6.8497166633606 14.7336616516113\\
6.94971656799316 14.7335815429688\\
7.04971647262573 14.7345237731934\\
7.1497163772583 14.7347774505615\\
7.24971628189087 14.7354955673218\\
7.3497166633606 14.7358999252319\\
7.44971656799316 14.7365169525146\\
7.54971647262573 14.7369899749756\\
7.6497163772583 14.7375602722168\\
7.74971628189087 14.7380647659302\\
7.8497166633606 14.7386140823364\\
7.94971656799316 14.7391328811646\\
8.04971599578857 14.739673614502\\
8.1497163772583 14.7401990890503\\
8.24971675872803 14.7407350540161\\
8.34971618652344 14.7412633895874\\
8.44971656799316 14.7417974472046\\
8.54971599578857 14.7423276901245\\
8.6497163772583 14.7428607940674\\
8.74971675872803 14.7433929443359\\
8.84971618652344 14.7439479827881\\
8.94971656799316 14.7448120117188\\
9.04971599578857 14.7454252243042\\
9.1497163772583 14.7460870742798\\
9.24971675872803 14.7467250823975\\
9.34971618652344 14.7473707199097\\
9.44971656799316 14.748010635376\\
9.54971599578857 14.7486515045166\\
9.6497163772583 14.7492904663086\\
9.74971675872803 14.7499284744263\\
9.84971618652344 14.7505655288696\\
9.94971656799316 14.751202583313\\
10.0497159957886 14.751838684082\\
10.1497163772583 14.7524747848511\\
10.249716758728 14.7531108856201\\
10.3497161865234 14.7537298202515\\
10.4497165679932 14.7540988922119\\
10.5497159957886 14.7545909881592\\
10.6497163772583 14.7551307678223\\
10.749716758728 14.755651473999\\
10.8497161865234 14.7561731338501\\
10.9497165679932 14.756688117981\\
11.0497159957886 14.7571992874146\\
11.1497163772583 14.7577028274536\\
11.249716758728 14.7581987380981\\
11.3497161865234 14.7586870193481\\
11.4497165679932 14.7591991424561\\
11.5497159957886 14.7598543167114\\
11.6497163772583 14.7605504989624\\
11.749716758728 14.7611818313599\\
11.8497161865234 14.7618141174316\\
11.9497165679932 14.7624731063843\\
12.0497159957886 14.7631711959839\\
12.1497163772583 14.7639303207397\\
12.249716758728 14.7665214538574\\
12.3471450805664 14.7732496261597\\
12.4250917434692 14.7781705856323\\
12.4879579544067 14.768609046936\\
12.557110786438 14.7686433792114\\
12.6331787109375 14.7690782546997\\
12.7168531417847 14.7695474624634\\
12.808895111084 14.7702302932739\\
12.9088954925537 14.7711057662964\\
13.0088949203491 14.7721309661865\\
13.1088953018188 14.7731733322144\\
13.2088947296143 14.7742366790771\\
13.308895111084 14.7753143310547\\
13.4088954925537 14.7763986587524\\
13.5088949203491 14.777494430542\\
13.6088953018188 14.7785978317261\\
13.7088947296143 14.7797069549561\\
13.808895111084 14.780816078186\\
13.9088954925537 14.781925201416\\
14.0088949203491 14.783034324646\\
14.1088953018188 14.784143447876\\
14.2088947296143 14.7852535247803\\
14.308895111084 14.7863636016846\\
14.4088954925537 14.7874736785889\\
14.5088949203491 14.7885837554932\\
14.6088953018188 14.7896947860718\\
14.7088947296143 14.7908058166504\\
14.808895111084 14.7919178009033\\
14.9088954925537 14.7930297851563\\
15.0088949203491 14.7941417694092\\
15.1088953018188 14.7952537536621\\
15.2088947296143 14.7963666915894\\
15.308895111084 14.7974805831909\\
15.4088954925537 14.7986001968384\\
15.5088949203491 14.7997035980225\\
15.6088953018188 14.8008260726929\\
15.7088947296143 14.8019437789917\\
15.808895111084 14.8030633926392\\
15.9088954925537 14.8041830062866\\
16.0088958740234 14.8053026199341\\
16.1088943481445 14.8064222335815\\
16.2088947296143 14.8075428009033\\
16.308895111084 14.8086624145508\\
16.4088954925537 14.8097829818726\\
16.5088958740234 14.8109035491943\\
16.6088943481445 14.8120241165161\\
16.7088947296143 14.8131446838379\\
16.808895111084 14.814266204834\\
16.9088954925537 14.8153867721558\\
17.0088958740234 14.8165082931519\\
17.1088943481445 14.8176298141479\\
17.2088947296143 14.818751335144\\
17.308895111084 14.8198738098145\\
17.4088954925537 14.8209953308105\\
17.5088958740234 14.822117805481\\
17.6088943481445 14.8232402801514\\
17.7088947296143 14.8243627548218\\
17.808895111084 14.8254852294922\\
17.9088954925537 14.8266096115112\\
18.0088958740234 14.8277368545532\\
18.1088943481445 14.8288841247559\\
18.2088947296143 14.830132484436\\
18.308895111084 14.8319272994995\\
18.4088954925537 14.8359346389771\\
18.5088958740234 14.8431234359741\\
18.6088943481445 14.8468885421753\\
18.7088947296143 14.8412246704102\\
18.808895111084 14.8385257720947\\
18.9088954925537 14.8396024703979\\
19.0088958740234 14.8392372131348\\
19.1088943481445 14.8409767150879\\
19.2088947296143 14.8416633605957\\
19.308895111084 14.8430795669556\\
19.4088954925537 14.8440103530884\\
19.5088958740234 14.8452653884888\\
19.6088943481445 14.8463039398193\\
19.7088947296143 14.8474884033203\\
19.808895111084 14.848575592041\\
19.9088954925537 14.8497276306152\\
20.0088958740234 14.8508358001709\\
20.1088943481445 14.8519744873047\\
20.2088947296143 14.8530921936035\\
20.308895111084 14.8542242050171\\
20.4088954925537 14.8553476333618\\
20.5088958740234 14.8564767837524\\
20.6088943481445 14.8576021194458\\
20.7088947296143 14.8587303161621\\
20.808895111084 14.8598566055298\\
20.9088954925537 14.8609848022461\\
21.0088958740234 14.8621110916138\\
21.1088943481445 14.8633079528809\\
21.2088947296143 14.8647031784058\\
21.308895111084 14.8657855987549\\
21.4088954925537 14.8669328689575\\
21.5088958740234 14.8680543899536\\
21.6088943481445 14.8691864013672\\
21.7088947296143 14.8703145980835\\
21.808895111084 14.8714447021484\\
21.9088954925537 14.8725738525391\\
22.0088958740234 14.8737049102783\\
22.1088943481445 14.8748359680176\\
22.2088947296143 14.8759670257568\\
22.308895111084 14.8770990371704\\
22.4088954925537 14.878231048584\\
22.5088958740234 14.8793640136719\\
22.6088943481445 14.8804807662964\\
22.7088947296143 14.8813285827637\\
22.808895111084 14.8824396133423\\
22.9088954925537 14.8835592269897\\
23.0088958740234 14.8846778869629\\
23.1088943481445 14.8857955932617\\
23.2088947296143 14.8869123458862\\
23.308895111084 14.8880281448364\\
23.4088954925537 14.889142036438\\
23.5088958740234 14.8902549743652\\
23.6088943481445 14.8913650512695\\
23.7088947296143 14.8925113677979\\
23.808895111084 14.8937644958496\\
23.9088954925537 14.894998550415\\
24.0088958740234 14.8961601257324\\
24.1088943481445 14.8973197937012\\
24.2088947296143 14.8984975814819\\
24.308895111084 14.8997011184692\\
24.4088954925537 14.9009475708008\\
24.5088958740234 14.9045486450195\\
24.6048831939697 14.910834312439\\
24.681676864624 14.9136018753052\\
24.7464084625244 14.9051694869995\\
24.8176136016846 14.9055252075195\\
24.895938873291 14.9060869216919\\
24.9820976257324 14.9067459106445\\
25.0768718719482 14.9075870513916\\
25.176872253418 14.9086112976074\\
25.2768707275391 14.9097194671631\\
25.3768711090088 14.9108257293701\\
25.4768714904785 14.911937713623\\
25.5768718719482 14.9130535125732\\
25.676872253418 14.9141712188721\\
25.7768707275391 14.9152879714966\\
25.8768711090088 14.9164094924927\\
25.9768714904785 14.9175319671631\\
26.0768718719482 14.9186611175537\\
26.176872253418 14.9197883605957\\
26.2768707275391 14.920916557312\\
26.3768711090088 14.9220447540283\\
26.4768714904785 14.9231729507446\\
26.5768718719482 14.9243001937866\\
26.676872253418 14.9254283905029\\
26.7768707275391 14.9265565872192\\
26.8768711090088 14.9276838302612\\
26.9768714904785 14.9288120269775\\
27.0768718719482 14.9299402236938\\
27.176872253418 14.9310674667358\\
27.2768707275391 14.9321956634521\\
27.3768711090088 14.9333238601685\\
27.4768714904785 14.9344511032104\\
27.5768718719482 14.9355792999268\\
27.676872253418 14.9367065429688\\
27.7768707275391 14.9378347396851\\
27.8768711090088 14.9389629364014\\
27.9768714904785 14.9400901794434\\
28.0768718719482 14.9412183761597\\
28.176872253418 14.9423456192017\\
28.2768707275391 14.9434804916382\\
28.3768711090088 14.9445905685425\\
28.4768714904785 14.9457235336304\\
28.5768718719482 14.946849822998\\
28.676872253418 14.9479780197144\\
28.7768707275391 14.9491062164307\\
28.8768711090088 14.950234413147\\
28.9768714904785 14.9513626098633\\
29.0768718719482 14.9524898529053\\
29.176872253418 14.9536180496216\\
29.2768707275391 14.9547462463379\\
29.3768711090088 14.9558744430542\\
29.4768714904785 14.9570026397705\\
29.5768718719482 14.9581298828125\\
29.676872253418 14.9592580795288\\
29.7768707275391 14.9603862762451\\
29.8768711090088 14.9615144729614\\
29.9768714904785 14.9626426696777\\
30.0768718719482 14.963770866394\\
30.176872253418 14.964900970459\\
30.2768707275391 14.9660387039185\\
30.3768711090088 14.9672107696533\\
30.4768714904785 14.9685831069946\\
30.5768718719482 14.9709510803223\\
30.676872253418 14.9764404296875\\
30.7768707275391 14.9835319519043\\
30.8768711090088 14.9819793701172\\
30.9768714904785 14.9787912368774\\
31.0768718719482 14.9764232635498\\
31.176872253418 14.9774360656738\\
31.2768707275391 14.9776411056519\\
31.3768711090088 14.9792537689209\\
31.4768714904785 14.9800567626953\\
31.5768718719482 14.9814014434814\\
31.676872253418 14.982385635376\\
31.7768707275391 14.9836101531982\\
31.8768711090088 14.9846744537354\\
31.9768714904785 14.9858465194702\\
32.0768699645996 14.986946105957\\
32.176872253418 14.988094329834\\
32.2768707275391 14.9892101287842\\
32.3768730163574 14.9903469085693\\
32.4768714904785 14.9914693832397\\
32.5768699645996 14.9926023483276\\
32.676872253418 14.9937286376953\\
32.7768707275391 14.9948596954346\\
32.8768730163574 14.9959878921509\\
32.9768714904785 14.9971179962158\\
33.0768699645996 14.9982461929321\\
33.176872253418 14.9993753433228\\
33.2768707275391 15.0005044937134\\
33.3768730163574 15.0018339157104\\
33.4768714904785 15.0030736923218\\
33.5768699645996 15.0041904449463\\
33.676872253418 15.0053262710571\\
33.7768707275391 15.0064544677734\\
33.8768730163574 15.0075855255127\\
33.9768714904785 15.0087156295776\\
34.0768699645996 15.0098466873169\\
34.176872253418 15.0109786987305\\
34.2768707275391 15.0121097564697\\
34.3768730163574 15.0132417678833\\
34.4768714904785 15.0143747329712\\
34.5768699645996 15.0155076980591\\
34.676872253418 15.0166416168213\\
34.7768707275391 15.0177755355835\\
34.8768730163574 15.0188503265381\\
34.9768714904785 15.0196676254272\\
35.0768699645996 15.0208778381348\\
35.176872253418 15.0219659805298\\
35.2768707275391 15.0230979919434\\
35.3768730163574 15.0242118835449\\
35.4768714904785 15.0253314971924\\
35.5768699645996 15.0264482498169\\
35.676872253418 15.0275640487671\\
35.7768707275391 15.0286779403687\\
35.8768730163574 15.0297899246216\\
35.9768714904785 15.0309648513794\\
36.0768699645996 15.0322227478027\\
36.176872253418 15.0334167480469\\
36.2768707275391 15.034571647644\\
36.3768730163574 15.0357360839844\\
36.4768714904785 15.036919593811\\
36.5768699645996 15.0381326675415\\
36.676872253418 15.0393943786621\\
36.7768707275391 15.0450391769409\\
36.8673248291016 15.049633026123\\
36.9396858215332 15.0495882034302\\
37.0192794799805 15.043267250061\\
37.1068305969238 15.0439186096191\\
37.2031402587891 15.0445747375488\\
37.3031425476074 15.0454931259155\\
37.4031410217285 15.046555519104\\
37.5031433105469 15.0476732254028\\
37.603141784668 15.0487785339355\\
37.7031402587891 15.0498905181885\\
37.8031425476074 15.0510053634644\\
37.9031410217285 15.0521259307861\\
38.0031433105469 15.053240776062\\
38.103141784668 15.0543622970581\\
38.2031402587891 15.0554847717285\\
38.3031425476074 15.0566120147705\\
38.4031410217285 15.0577402114868\\
38.5031433105469 15.0588684082031\\
38.603141784668 15.0599966049194\\
38.7031402587891 15.0611248016357\\
38.8031425476074 15.0622529983521\\
38.9031410217285 15.0633811950684\\
39.0031433105469 15.064510345459\\
39.103141784668 15.0656385421753\\
39.2031402587891 15.0667657852173\\
39.3031425476074 15.0678939819336\\
39.4031410217285 15.0690221786499\\
39.5031433105469 15.0701503753662\\
39.603141784668 15.0712785720825\\
39.7031402587891 15.0724067687988\\
39.8031425476074 15.0735349655151\\
39.9031410217285 15.0746631622314\\
40.0031433105469 15.0757913589478\\
40.103141784668 15.0769195556641\\
40.2031402587891 15.0780477523804\\
40.3031425476074 15.0791759490967\\
40.4031410217285 15.080304145813\\
40.5031433105469 15.0814323425293\\
40.603141784668 15.0825595855713\\
40.7031402587891 15.0836877822876\\
40.8031425476074 15.0848159790039\\
40.9031410217285 15.0859508514404\\
41.0031433105469 15.0870609283447\\
41.103141784668 15.0881948471069\\
41.2031402587891 15.0893211364746\\
41.3031425476074 15.0904502868652\\
41.4031410217285 15.0915775299072\\
41.5031433105469 15.0927066802979\\
41.603141784668 15.0938348770142\\
41.7031402587891 15.0949630737305\\
41.8031425476074 15.0960912704468\\
41.9031410217285 15.0972194671631\\
42.0031433105469 15.0983476638794\\
42.103141784668 15.0994758605957\\
42.2031402587891 15.100604057312\\
42.3031425476074 15.1017332077026\\
42.4031410217285 15.1028623580933\\
42.5031433105469 15.1039972305298\\
42.603141784668 15.1051578521729\\
42.7031402587891 15.106463432312\\
42.8031425476074 15.1085205078125\\
42.9031410217285 15.1132192611694\\
43.0031433105469 15.1198301315308\\
43.103141784668 15.1210536956787\\
43.2031402587891 15.1182489395142\\
43.3031425476074 15.1148796081543\\
43.4031410217285 15.1154232025146\\
43.5031433105469 15.1156797409058\\
43.603141784668 15.1171436309814\\
43.7031402587891 15.1179189682007\\
43.8031425476074 15.1194105148315\\
43.9031410217285 15.1202983856201\\
44.0031433105469 15.1215867996216\\
44.103141784668 15.1226081848145\\
44.2031402587891 15.1238079071045\\
44.3031425476074 15.1248893737793\\
44.4031410217285 15.1260499954224\\
44.5031433105469 15.1271572113037\\
44.603141784668 15.1282997131348\\
44.7031402587891 15.1294183731079\\
44.8031425476074 15.1305532455444\\
44.9031410217285 15.1316776275635\\
45.0031433105469 15.1328096389771\\
45.103141784668 15.1339359283447\\
45.2031402587891 15.1350660324097\\
45.3031425476074 15.136194229126\\
45.4031410217285 15.1373233795166\\
45.5031433105469 15.1384515762329\\
45.603141784668 15.139723777771\\
45.7031402587891 15.1410293579102\\
45.8031425476074 15.1421155929565\\
45.9031410217285 15.1432571411133\\
46.0031433105469 15.1443815231323\\
46.103141784668 15.1455116271973\\
46.2031402587891 15.1466398239136\\
46.3031425476074 15.1477689743042\\
46.4031410217285 15.1488981246948\\
46.5031433105469 15.1500263214111\\
46.603141784668 15.1511554718018\\
46.7031402587891 15.1522836685181\\
46.8031425476074 15.1534128189087\\
46.9031410217285 15.154541015625\\
47.0031433105469 15.1556701660156\\
47.103141784668 15.1567935943604\\
47.2031402587891 15.1587295532227\\
47.3031425476074 15.1590728759766\\
47.4031410217285 15.1598958969116\\
47.5031433105469 15.1610918045044\\
47.603141784668 15.162181854248\\
47.7031402587891 15.163311958313\\
47.8031425476074 15.1644258499146\\
47.9031410217285 15.1655435562134\\
48.0031433105469 15.1666584014893\\
48.103141784668 15.1677722930908\\
48.2031402587891 15.1689252853394\\
48.3031425476074 15.1701030731201\\
48.4031410217285 15.1714162826538\\
48.5031433105469 15.1725482940674\\
48.603141784668 15.1736869812012\\
48.7031402587891 15.1748695373535\\
48.8031425476074 15.1760702133179\\
48.9031410217285 15.1773176193237\\
49.0031433105469 15.1813631057739\\
49.098030090332 15.186785697937\\
49.1739387512207 15.1910095214844\\
49.2346649169922 15.1802492141724\\
49.3014602661133 15.1998834609985\\
49.3450546264648 15.1551971435547\\
49.3930053710938 15.1706562042236\\
49.4215202331543 15.193733215332\\
49.4451713562012 15.3380928039551\\
49.47119140625 15.055549621582\\
49.4998092651367 15.1796255111694\\
49.515323638916 15.1865968704224\\
49.5276222229004 15.4837913513184\\
49.5411529541016 14.9221239089966\\
49.5560340881348 15.1116857528687\\
49.5575561523438 15.7338838577271\\
49.5592269897461 15.3602809906006\\
49.5610656738281 15.1738147735596\\
49.5630912780762 15.1993160247803\\
49.565315246582 15.2013034820557\\
49.5677642822266 15.1970310211182\\
49.5704574584961 15.1936655044556\\
49.5734214782715 15.1910648345947\\
49.5766792297363 15.1890506744385\\
49.5791168212891 15.1878967285156\\
49.5817947387695 15.1869592666626\\
49.5847434997559 15.1862087249756\\
49.5879859924316 15.1856250762939\\
49.591552734375 15.1851825714111\\
49.5954742431641 15.1848583221436\\
49.5997886657715 15.1846294403076\\
49.6045341491699 15.1844730377197\\
49.6097564697266 15.1843709945679\\
49.6155014038086 15.1843090057373\\
49.6218185424805 15.1842756271362\\
49.6287651062012 15.1842632293701\\
49.6364097595215 15.1842651367188\\
49.6448211669922 15.1842803955078\\
49.6540679931641 15.18430519104\\
49.6642456054688 15.1843414306641\\
49.6754379272461 15.1844673156738\\
49.6877479553223 15.184681892395\\
49.7012901306152 15.1848411560059\\
49.7161865234375 15.1849393844604\\
49.7325744628906 15.1849765777588\\
49.7505989074707 15.1849966049194\\
49.7704238891602 15.1850490570068\\
49.792236328125 15.1853446960449\\
49.8162269592285 15.1856250762939\\
49.8426170349121 15.1858415603638\\
49.8716468811035 15.1860265731812\\
49.9035758972168 15.1862478256226\\
49.9387016296387 15.1865177154541\\
49.9773406982422 15.1868839263916\\
50.0198440551758 15.1873455047607\\
50.066593170166 15.1879167556763\\
50.1180229187012 15.1886320114136\\
50.1745910644531 15.1894807815552\\
50.23681640625 15.1942272186279\\
50.3052673339844 15.1885976791382\\
50.3805618286133 15.1955699920654\\
50.4633865356445 15.1914644241333\\
50.5544929504395 15.1978197097778\\
50.6544914245605 15.1927089691162\\
50.7544898986816 15.1998929977417\\
50.8544921875 15.1946296691895\\
50.9544906616211 15.2021837234497\\
51.0544929504395 15.1967191696167\\
51.1544914245605 15.2043790817261\\
51.2544898986816 15.1989612579346\\
51.3544921875 15.2012796401978\\
51.3568458557129 15.3158397674561\\
51.359432220459 15.2306413650513\\
51.3622817993164 15.1889343261719\\
51.3654136657715 15.1953868865967\\
51.3688583374023 15.2051019668579\\
51.3726463317871 15.206883430481\\
51.3768157958984 15.2060050964355\\
51.3814010620117 15.2056035995483\\
51.3864479064941 15.205605506897\\
51.3919944763184 15.2056283950806\\
51.3980979919434 15.2056255340576\\
51.4048118591309 15.2056226730347\\
51.4121971130371 15.2056264877319\\
51.4203224182129 15.2056398391724\\
51.4292602539063 15.2056617736816\\
51.4390869140625 15.2056941986084\\
51.4499015808105 15.2057399749756\\
51.4617958068848 15.2058000564575\\
51.4748802185059 15.2058763504028\\
51.4892692565918 15.2059707641602\\
51.5051002502441 15.2060852050781\\
51.522518157959 15.2062206268311\\
51.5416717529297 15.2063808441162\\
51.562744140625 15.2065658569336\\
51.5859222412109 15.2067785263062\\
51.6114196777344 15.2070207595825\\
51.6394653320313 15.2072925567627\\
51.6703147888184 15.2075986862183\\
51.7042503356934 15.2079391479492\\
51.7415771484375 15.2083177566528\\
51.7826385498047 15.2087364196777\\
51.827808380127 15.2091979980469\\
51.8774948120117 15.2097072601318\\
51.9321479797363 15.2102670669556\\
51.9922637939453 15.2108840942383\\
52.0583953857422 15.2115631103516\\
52.1311378479004 15.212308883667\\
52.2111587524414 15.2131586074829\\
52.2991790771484 15.2166986465454\\
52.3959999084473 15.2122888565063\\
52.4959983825684 15.2162237167358\\
52.5960006713867 15.2233877182007\\
52.6160011291504 15.2175693511963\\
52.6228942871094 15.2351388931274\\
52.6304740905762 15.2396717071533\\
52.637622833252 15.2409086227417\\
52.6454811096191 15.2411556243896\\
52.6532897949219 15.240891456604\\
52.6618766784668 15.2379388809204\\
52.6713256835938 15.2349061965942\\
52.6817169189453 15.2345190048218\\
52.6931457519531 15.232684135437\\
52.7057189941406 15.2282514572144\\
52.7195510864258 15.2291049957275\\
52.7347640991211 15.2381162643433\\
52.7514991760254 15.251898765564\\
52.7699089050293 15.2411470413208\\
52.7901573181152 15.2278718948364\\
52.8124351501465 15.2203044891357\\
52.8369369506836 15.2122440338135\\
52.8638877868652 15.204083442688\\
52.8935356140137 15.1988039016724\\
52.9261474609375 15.1959705352783\\
52.9620208740234 15.1962118148804\\
53.0014839172363 15.2002544403076\\
53.0448913574219 15.2043762207031\\
53.0926399230957 15.2019996643066\\
53.1261940002441 15.2084064483643\\
53.1631088256836 15.2250728607178\\
53.1971549987793 15.2256288528442\\
53.2346038818359 15.2281303405762\\
53.2757987976074 15.2299757003784\\
53.3211135864258 15.2309226989746\\
53.3709602355957 15.2312850952148\\
53.4181785583496 15.2309350967407\\
53.4612884521484 15.230842590332\\
53.4995498657227 15.2311086654663\\
53.5327682495117 15.2360734939575\\
53.5693130493164 15.2341794967651\\
53.6095085144043 15.231861114502\\
53.6537246704102 15.230170249939\\
53.7023620605469 15.2305145263672\\
53.7558631896973 15.2309226989746\\
53.8147125244141 15.2312860488892\\
53.8794479370117 15.2318029403687\\
53.9506568908691 15.2324628829956\\
54.0289878845215 15.2331943511963\\
54.1151542663574 15.2340221405029\\
54.2099342346191 15.2361001968384\\
54.3099327087402 15.2347192764282\\
54.4099311828613 15.2384662628174\\
54.5099334716797 15.2368202209473\\
54.6099319458008 15.2386226654053\\
54.6148452758789 15.2837533950806\\
54.6202507019043 15.2247676849365\\
54.6261940002441 15.2574901580811\\
54.6327362060547 15.2257776260376\\
54.6399269104004 15.2558479309082\\
54.6478385925293 15.2280359268188\\
54.6565437316895 15.2537355422974\\
54.6661186218262 15.2300968170166\\
54.6766510009766 15.2391567230225\\
54.6824378967285 15.2456026077271\\
54.6888046264648 15.2419948577881\\
54.6958045959473 15.2416915893555\\
54.7035064697266 15.2417936325073\\
54.7119827270508 15.2418432235718\\
54.7213020324707 15.2418947219849\\
54.7315559387207 15.2419576644897\\
54.7428321838379 15.2420320510864\\
54.7552375793457 15.2421236038208\\
54.7688827514648 15.2422285079956\\
54.7838935852051 15.2423505783081\\
54.800407409668 15.2424917221069\\
54.8185691833496 15.242654800415\\
54.8385467529297 15.2428407669067\\
54.8605270385742 15.243052482605\\
54.8847007751465 15.2432889938354\\
54.9112930297852 15.2435665130615\\
54.940544128418 15.2438774108887\\
54.9727210998535 15.244236946106\\
55.008113861084 15.2446613311768\\
55.0470504760742 15.245189666748\\
55.0898742675781 15.2459030151367\\
55.1369857788086 15.2469835281372\\
55.1888084411621 15.2488088607788\\
55.2458114624023 15.2520313262939\\
55.3085136413574 15.2566051483154\\
55.3774871826172 15.2605056762695\\
55.4533576965332 15.2577028274536\\
55.5368156433105 15.2554473876953\\
55.6286201477051 15.2545604705811\\
55.7286186218262 15.2540254592896\\
55.8286209106445 15.2535276412964\\
55.9286193847656 15.254768371582\\
56.0286178588867 15.2557430267334\\
56.1286201477051 15.2571449279785\\
56.2286186218262 15.258092880249\\
56.3286209106445 15.2593393325806\\
56.4286193847656 15.2603893280029\\
56.5286178588867 15.2615690231323\\
56.6286201477051 15.2626628875732\\
56.7286186218262 15.2638139724731\\
56.8286209106445 15.2649602890015\\
56.9286193847656 15.266077041626\\
57.0286178588867 15.2671804428101\\
57.1286201477051 15.2683191299438\\
57.2286186218262 15.2694387435913\\
57.3286209106445 15.2705726623535\\
57.4286193847656 15.2716970443726\\
57.5286178588867 15.2728281021118\\
57.6286201477051 15.2739543914795\\
57.7286186218262 15.2750844955444\\
57.8286209106445 15.2762117385864\\
57.9286193847656 15.2773418426514\\
58.0286178588867 15.2787885665894\\
58.1286201477051 15.2798681259155\\
58.2286186218262 15.2810182571411\\
58.3286209106445 15.2821378707886\\
58.4286193847656 15.2832689285278\\
58.5286178588867 15.2843961715698\\
58.6286201477051 15.2855253219604\\
58.7286186218262 15.2866535186768\\
58.8286209106445 15.2877817153931\\
58.9286193847656 15.2889108657837\\
59.0286178588867 15.2900390625\\
59.1286201477051 15.2911701202393\\
59.2286186218262 15.2922983169556\\
59.3286209106445 15.2934274673462\\
59.4286193847656 15.2945556640625\\
59.5286178588867 15.2955083847046\\
59.6286201477051 15.2964887619019\\
59.7286186218262 15.2976522445679\\
59.8286209106445 15.2987480163574\\
59.9143104553223 15.3011493682861\\
60 15.3022813796997\\
60.0942611694336 15.3021421432495\\
60.1942596435547 15.3029870986938\\
60.2942581176758 15.3041391372681\\
60.3942604064941 15.3052577972412\\
60.4942588806152 15.3063888549805\\
60.5942611694336 15.3075199127197\\
60.6942596435547 15.3086557388306\\
60.7942581176758 15.3098049163818\\
60.8942604064941 15.3109512329102\\
60.9942588806152 15.312108039856\\
61.0942611694336 15.3132829666138\\
61.1942596435547 15.3144845962524\\
61.2942581176758 15.3157434463501\\
61.3942604064941 15.321587562561\\
61.4835205078125 15.3258171081543\\
61.5549278259277 15.3251113891602\\
61.6334800720215 15.3196935653687\\
61.7198829650879 15.3203210830688\\
61.8149299621582 15.3367099761963\\
61.8549957275391 15.2851409912109\\
61.8990707397461 15.3168239593506\\
61.9192581176758 15.3288249969482\\
61.9358787536621 15.3320579528809\\
61.9541664123535 15.3236618041992\\
61.9742813110352 15.3236713409424\\
61.9964065551758 15.3238325119019\\
62.0207443237305 15.3240013122559\\
62.0475158691406 15.3242092132568\\
62.0769653320313 15.3244562149048\\
62.1093559265137 15.324746131897\\
62.1449890136719 15.3250799179077\\
62.1841850280762 15.3254594802856\\
62.2273025512695 15.325888633728\\
62.274730682373 15.3263683319092\\
62.3269004821777 15.3269014358521\\
62.3842887878418 15.3274908065796\\
62.4474143981934 15.3281402587891\\
62.5168533325195 15.3288536071777\\
62.5932350158691 15.3296384811401\\
62.6772575378418 15.3396949768066\\
62.7696800231934 15.3228168487549\\
62.8696823120117 15.3325929641724\\
62.9696807861328 15.3367834091187\\
63.0696792602539 15.3411664962769\\
63.1196823120117 15.3907136917114\\
63.1572036743164 15.3497362136841\\
63.1601943969727 15.3309259414673\\
63.1634902954102 15.3392190933228\\
63.1671104431152 15.3568449020386\\
63.1710968017578 15.3588838577271\\
63.175479888916 15.3461332321167\\
63.1803016662598 15.3361902236938\\
63.185604095459 15.3247270584106\\
63.1914367675781 15.3136568069458\\
63.1978569030762 15.3068113327026\\
63.2049140930176 15.3020153045654\\
63.2126808166504 15.2971162796021\\
63.2212219238281 15.2908229827881\\
63.2306175231934 15.2818756103516\\
63.2409515380859 15.2716054916382\\
63.2523231506348 15.2633075714111\\
63.2648277282715 15.2594404220581\\
63.2785835266113 15.2605962753296\\
63.2937164306641 15.2666864395142\\
63.310359954834 15.2788066864014\\
63.328670501709 15.2962188720703\\
63.3488121032715 15.3215398788452\\
63.3709678649902 15.3141469955444\\
63.395336151123 15.3186225891113\\
63.4221420288086 15.3311576843262\\
63.4516296386719 15.3289785385132\\
63.4840698242188 15.3344812393188\\
63.4999580383301 15.3429727554321\\
63.5174407958984 15.347864151001\\
63.5366668701172 15.352068901062\\
63.5578193664551 15.3585720062256\\
63.5810852050781 15.3631057739258\\
63.6066780090332 15.359956741333\\
63.6348304748535 15.3532676696777\\
63.6657981872559 15.3498363494873\\
63.6998634338379 15.34876537323\\
63.7373313903809 15.348352432251\\
63.7785491943359 15.3478736877441\\
63.8238906860352 15.3477048873901\\
63.8737640380859 15.348180770874\\
63.9286231994629 15.3492069244385\\
63.9762001037598 15.3493194580078\\
64.0218276977539 15.3493127822876\\
64.0647964477539 15.3504915237427\\
64.112060546875 15.352147102356\\
64.1640548706055 15.3508062362671\\
64.2212448120117 15.3481941223145\\
64.2841567993164 15.3484973907471\\
64.3533554077148 15.3490533828735\\
64.4294738769531 15.3495655059814\\
64.5132064819336 15.3502960205078\\
64.6053161621094 15.3531799316406\\
64.7053146362305 15.3500699996948\\
64.8053131103516 15.3554763793945\\
64.9053115844727 15.3520078659058\\
65.0053176879883 15.354395866394\\
65.0079345703125 15.4639558792114\\
65.0108184814453 15.3271446228027\\
65.0139923095703 15.3169078826904\\
65.0165023803711 15.3648796081543\\
65.0192642211914 15.3793926239014\\
65.0223007202148 15.3610067367554\\
65.025634765625 15.3563375473022\\
65.0293121337891 15.3573131561279\\
65.0333557128906 15.3577079772949\\
65.0377960205078 15.357684135437\\
65.0426864624023 15.3576622009277\\
65.0480651855469 15.3576612472534\\
65.0539855957031 15.357666015625\\
65.0604934692383 15.3576745986938\\
65.0676498413086 15.3576898574829\\
65.0755233764648 15.3577127456665\\
65.0841903686523 15.3577442169189\\
65.0937194824219 15.3577861785889\\
65.1042022705078 15.3578386306763\\
65.1157302856445 15.3579053878784\\
65.1284103393555 15.3579864501953\\
65.1423645019531 15.3580846786499\\
65.1577072143555 15.3582010269165\\
65.1745910644531 15.3583374023438\\
65.1931610107422 15.3584957122803\\
65.2135848999023 15.358678817749\\
65.2360534667969 15.35888671875\\
65.2607727050781 15.3591232299805\\
65.287956237793 15.3593883514404\\
65.3178634643555 15.3596849441528\\
65.3507614135742 15.3600158691406\\
65.3869476318359 15.3603830337524\\
65.4267501831055 15.3607883453369\\
65.4705352783203 15.3612356185913\\
65.5186996459961 15.3617296218872\\
65.5716781616211 15.3622722625732\\
65.6299591064453 15.3628702163696\\
65.6940612792969 15.3635272979736\\
65.7645797729492 15.3642511367798\\
65.8421478271484 15.3650465011597\\
65.9274749755859 15.3659219741821\\
66.0213317871094 15.3685941696167\\
66.1213302612305 15.3660669326782\\
66.2213287353516 15.3691816329956\\
66.3213272094727 15.378080368042\\
66.3413314819336 15.3859300613403\\
66.3490600585938 15.4133529663086\\
66.3565673828125 15.4173974990845\\
66.3648223876953 15.4138975143433\\
66.3727111816406 15.4104156494141\\
66.3813858032227 15.4161310195923\\
66.3909301757813 15.4455499649048\\
66.4014358520508 15.4762687683105\\
66.4129867553711 15.4794082641602\\
66.4256896972656 15.4570236206055\\
66.4396667480469 15.4365787506104\\
66.4550399780273 15.4139404296875\\
66.4719543457031 15.3901805877686\\
66.4905548095703 15.3658885955811\\
66.5110244750977 15.3353652954102\\
66.5335311889648 15.2999982833862\\
66.5504302978516 15.2850399017334\\
66.5690307617188 15.2758960723877\\
66.5894775390625 15.2730531692505\\
66.6119766235352 15.2928838729858\\
66.6367263793945 15.3197946548462\\
66.663948059082 15.3735342025757\\
66.6938934326172 15.3803434371948\\
66.7268295288086 15.3830423355103\\
66.7630615234375 15.3843870162964\\
66.8029174804688 15.3834896087646\\
66.8467636108398 15.3816804885864\\
66.8949890136719 15.3803443908691\\
66.9480361938477 15.3936891555786\\
67.0063934326172 15.3768825531006\\
67.0705795288086 15.3887996673584\\
67.111457824707 15.372163772583\\
67.1564254760742 15.3757543563843\\
67.1973037719727 15.3756227493286\\
67.2330703735352 15.3765630722046\\
67.2724151611328 15.3781366348267\\
67.3156967163086 15.3802709579468\\
67.3633117675781 15.3817958831787\\
67.4156799316406 15.3833484649658\\
67.4732894897461 15.3856945037842\\
67.5366516113281 15.3882837295532\\
67.6063613891602 15.3914194107056\\
67.6830368041992 15.3942546844482\\
67.7673797607422 15.3919649124146\\
67.8601531982422 15.3919610977173\\
67.9601516723633 15.3914823532104\\
68.0601577758789 15.3912115097046\\
68.16015625 15.3908996582031\\
68.2601547241211 15.3916702270508\\
68.3601531982422 15.3928184509277\\
68.4601516723633 15.3940868377686\\
68.5601577758789 15.3951177597046\\
68.66015625 15.3963060379028\\
68.7601547241211 15.3973922729492\\
68.8601531982422 15.3985471725464\\
68.9601516723633 15.3996572494507\\
69.0601577758789 15.4007978439331\\
69.16015625 15.4019184112549\\
69.2601547241211 15.4030513763428\\
69.3601531982422 15.4041767120361\\
69.4601516723633 15.4053068161011\\
69.5601577758789 15.4064340591431\\
69.66015625 15.4075632095337\\
69.7601547241211 15.40869140625\\
69.8601531982422 15.4098196029663\\
69.9601516723633 15.4109477996826\\
70.0601577758789 15.4120769500732\\
70.16015625 15.4132051467896\\
70.2601547241211 15.4143447875977\\
70.3601531982422 15.4155149459839\\
70.4601516723633 15.4169282913208\\
70.5601577758789 15.4179801940918\\
70.66015625 15.4191312789917\\
70.7601547241211 15.4202537536621\\
70.8601531982422 15.4213848114014\\
70.9601516723633 15.4225130081177\\
71.0601577758789 15.4236421585083\\
71.16015625 15.4247703552246\\
71.2601547241211 15.4258995056152\\
71.3601531982422 15.4270277023315\\
71.4601516723633 15.4281568527222\\
71.5601577758789 15.4292850494385\\
71.66015625 15.4304141998291\\
71.7601547241211 15.4315423965454\\
71.8601531982422 15.4326705932617\\
71.9601516723633 15.4335880279541\\
72.0601577758789 15.4346160888672\\
72.16015625 15.4357604980469\\
72.2601547241211 15.436863899231\\
72.3601531982422 15.4379787445068\\
72.4601516723633 15.4390878677368\\
72.5601577758789 15.4401950836182\\
72.66015625 15.4412994384766\\
72.7601547241211 15.442400932312\\
72.8601531982422 15.4434976577759\\
72.9601516723633 15.4445896148682\\
73.0601577758789 15.4456777572632\\
73.16015625 15.4502382278442\\
73.2601547241211 15.4491577148438\\
73.3601531982422 15.4493551254272\\
73.4601516723633 15.4507055282593\\
73.5601577758789 15.4518146514893\\
73.66015625 15.4530668258667\\
73.7601547241211 15.4545545578003\\
73.8601531982422 15.4621381759644\\
73.943000793457 15.4655313491821\\
74.00927734375 15.4601888656616\\
74.0821685791016 15.4579839706421\\
74.162353515625 15.458553314209\\
74.2505569458008 15.4759168624878\\
74.2926559448242 15.4255390167236\\
74.3389587402344 15.4545516967773\\
74.3595275878906 15.4681148529053\\
74.3765106201172 15.4706335067749\\
74.3951873779297 15.461893081665\\
74.4157409667969 15.4618902206421\\
74.438346862793 15.4620590209961\\
74.4632110595703 15.4622325897217\\
74.4905624389648 15.4624462127686\\
74.5206527709961 15.462700843811\\
74.5537414550781 15.4629983901978\\
74.5901489257813 15.4633407592773\\
74.6301956176758 15.4637308120728\\
74.6742477416992 15.4641704559326\\
74.7227020263672 15.4646615982056\\
74.7760009765625 15.4652070999146\\
74.8346328735352 15.4658088684082\\
74.8991317749023 15.4664716720581\\
74.9495620727539 15.4672031402588\\
75 15.4677715301514\\
};
\addlegendentry{2"};

\addplot [
color=black,
dash pattern=on 1pt off 3pt on 3pt off 3pt
]
table[row sep=crcr]{
0 14.7006950378418\\
9.99999971718069e-10 14.7006950378418\\
2.49999998480632e-09 14.7006950378418\\
4.74999994892755e-09 14.7006950378418\\
8.1250002281763e-09 14.7006950378418\\
1.31874999809156e-08 14.7006950378418\\
2.0781250498203e-08 14.7006950378418\\
3.21718758300449e-08 14.7006950378418\\
4.92578138278077e-08 14.7006950378418\\
7.48867208244519e-08 14.7006950378418\\
1.13330081319418e-07 14.7006950378418\\
1.70995122061868e-07 14.7006950378418\\
2.57492672517401e-07 14.7006950378418\\
3.8723902662241e-07 14.7006950378418\\
5.81858500936505e-07 14.7006950378418\\
8.73787769251066e-07 14.7006950378418\\
1.31168167172291e-06 14.7006950378418\\
1.9685226106958e-06 14.7006950378418\\
2.95378367809462e-06 14.7006959915161\\
4.43167573394021e-06 14.7006959915161\\
6.64851359033491e-06 14.7006959915161\\
9.97376992017962e-06 14.7006959915161\\
1.49616553244414e-05 14.7006969451904\\
2.24434825213393e-05 14.7006978988647\\
3.36662233166862e-05 14.7006988525391\\
5.05003372381907e-05 14.7006998062134\\
7.57515081204474e-05 14.700701713562\\
0.000113628258986864 14.7007036209106\\
0.000170443381648511 14.7007055282593\\
0.000255666091106832 14.7007074356079\\
0.000383500126190484 14.7007083892822\\
0.000575251178815961 14.7007074356079\\
0.000862877757754177 14.7007036209106\\
0.00129431765526533 14.7007007598877\\
0.00194147753063589 14.7006969451904\\
0.00291221728548408 14.7006931304932\\
0.00436832662671804 14.7006912231445\\
0.00597004732117057 14.7006931304932\\
0.00773194013163447 14.7006950378418\\
0.009670021943748 14.7006959915161\\
0.0118019115179777 14.7006959915161\\
0.0141469910740852 14.7006959915161\\
0.0167265776544809 14.7006959915161\\
0.0195641238242388 14.7006959915161\\
0.0226854234933853 14.7006959915161\\
0.0261188540607691 14.7006959915161\\
0.0298956278711557 14.7006959915161\\
0.0340500771999359 14.7006959915161\\
0.0386199727654457 14.7006950378418\\
0.0436468608677387 14.7006950378418\\
0.0491764321923256 14.7006950378418\\
0.0552589632570744 14.7006950378418\\
0.0619497485458851 14.7006950378418\\
0.0693096145987511 14.7006959915161\\
0.0774054601788521 14.7006912231445\\
0.0863108932971954 14.7006988525391\\
0.0961068719625473 14.7006959915161\\
0.106882445514202 14.7006959915161\\
0.118735581636429 14.7006959915161\\
0.13177402317524 14.7006959915161\\
0.146116316318512 14.7006959915161\\
0.161892831325531 14.7006959915161\\
0.179247006773949 14.7006959915161\\
0.198336601257324 14.7006959915161\\
0.21933513879776 14.7006959915161\\
0.242433547973633 14.7006950378418\\
0.267841786146164 14.7006959915161\\
0.29579085111618 14.7006959915161\\
0.32653483748436 14.7006959915161\\
0.360353201627731 14.7006959915161\\
0.397553414106369 14.7006959915161\\
0.438473641872406 14.7006959915161\\
0.483485877513886 14.7006959915161\\
0.532999336719513 14.7006959915161\\
0.587464153766632 14.7006959915161\\
0.647375464439392 14.7006959915161\\
0.71327793598175 14.7006959915161\\
0.7857705950737 14.7006959915161\\
0.865512549877167 14.7006959915161\\
0.953228712081909 14.7006959915161\\
1.04971647262573 14.7006950378418\\
1.14971649646759 14.7006950378418\\
1.24971640110016 14.7006950378418\\
1.34971642494202 14.7006950378418\\
1.44971644878387 14.7006950378418\\
1.54971647262573 14.7006950378418\\
1.64971649646759 14.7006950378418\\
1.74971640110016 14.7006950378418\\
1.84971642494202 14.7006950378418\\
1.94971644878387 14.7006950378418\\
2.04971647262573 14.7006950378418\\
2.1497163772583 14.7006950378418\\
2.24971652030945 14.7006950378418\\
2.34971642494202 14.7006950378418\\
2.44971656799316 14.7006950378418\\
2.54971647262573 14.7006950378418\\
2.6497163772583 14.7006950378418\\
2.74971652030945 14.7006950378418\\
2.84971642494202 14.7006950378418\\
2.94971656799316 14.7006950378418\\
3.04971647262573 14.7006950378418\\
3.1497163772583 14.7006950378418\\
3.24971652030945 14.7006950378418\\
3.34971642494202 14.7006950378418\\
3.44971656799316 14.7006950378418\\
3.54971647262573 14.7006950378418\\
3.6497163772583 14.7006950378418\\
3.74971652030945 14.7006950378418\\
3.84971642494202 14.7006950378418\\
3.94971656799316 14.7006950378418\\
4.04971647262573 14.7006950378418\\
4.1497163772583 14.7006950378418\\
4.24971628189087 14.7006950378418\\
4.3497166633606 14.7006950378418\\
4.44971656799316 14.7006950378418\\
4.54971647262573 14.7006950378418\\
4.6497163772583 14.7006950378418\\
4.74971628189087 14.7006950378418\\
4.8497166633606 14.7006950378418\\
4.94971656799316 14.7006950378418\\
5.04971647262573 14.7006950378418\\
5.1497163772583 14.7006950378418\\
5.24971628189087 14.7006950378418\\
5.3497166633606 14.7006950378418\\
5.44971656799316 14.7006950378418\\
5.54971647262573 14.7006950378418\\
5.6497163772583 14.7006950378418\\
5.74971628189087 14.7006950378418\\
5.8497166633606 14.7006950378418\\
5.94971656799316 14.7006950378418\\
6.04971647262573 14.7006950378418\\
6.1497163772583 14.7006940841675\\
6.24971628189087 14.7006921768188\\
6.3497166633606 14.7006912231445\\
6.44971656799316 14.7006940841675\\
6.54971647262573 14.7006969451904\\
6.6497163772583 14.7006959915161\\
6.74971628189087 14.7006978988647\\
6.8497166633606 14.7006940841675\\
6.94971656799316 14.7006959915161\\
7.04971647262573 14.7006950378418\\
7.1497163772583 14.7006959915161\\
7.24971628189087 14.7006950378418\\
7.3497166633606 14.7006950378418\\
7.44971656799316 14.7006950378418\\
7.54971647262573 14.7006950378418\\
7.6497163772583 14.7006950378418\\
7.74971628189087 14.7006950378418\\
7.8497166633606 14.7006950378418\\
7.94971656799316 14.7006950378418\\
8.04971599578857 14.7006950378418\\
8.1497163772583 14.7006950378418\\
8.24971675872803 14.7006950378418\\
8.34971618652344 14.7006950378418\\
8.44971656799316 14.7006950378418\\
8.54971599578857 14.7006950378418\\
8.6497163772583 14.7006950378418\\
8.74971675872803 14.7006950378418\\
8.84971618652344 14.7006950378418\\
8.94971656799316 14.7006950378418\\
9.04971599578857 14.7006950378418\\
9.1497163772583 14.7006950378418\\
9.24971675872803 14.7006950378418\\
9.34971618652344 14.7006950378418\\
9.44971656799316 14.7006950378418\\
9.54971599578857 14.7006950378418\\
9.6497163772583 14.7006950378418\\
9.74971675872803 14.7006950378418\\
9.84971618652344 14.7006950378418\\
9.94971656799316 14.7006950378418\\
10.0497159957886 14.7006950378418\\
10.1497163772583 14.7006950378418\\
10.249716758728 14.7006950378418\\
10.3497161865234 14.7006950378418\\
10.4497165679932 14.7006950378418\\
10.5497159957886 14.7006950378418\\
10.6497163772583 14.7006950378418\\
10.749716758728 14.7006950378418\\
10.8497161865234 14.7006950378418\\
10.9497165679932 14.7006950378418\\
11.0497159957886 14.7006950378418\\
11.1497163772583 14.7006950378418\\
11.249716758728 14.7006950378418\\
11.3497161865234 14.7006950378418\\
11.4497165679932 14.7006950378418\\
11.5497159957886 14.7006950378418\\
11.6497163772583 14.7006959915161\\
11.749716758728 14.7006959915161\\
11.8497161865234 14.7006950378418\\
11.9497165679932 14.7006950378418\\
12.0497159957886 14.7006950378418\\
12.1497163772583 14.7006950378418\\
12.249716758728 14.7006950378418\\
12.3471450805664 14.7006950378418\\
12.4250917434692 14.7006950378418\\
12.4879579544067 14.7006959915161\\
12.557110786438 14.7006931304932\\
12.6331787109375 14.7006959915161\\
12.7168531417847 14.7006959915161\\
12.808895111084 14.7006959915161\\
12.9088954925537 14.7006959915161\\
13.0088949203491 14.7006959915161\\
13.1088953018188 14.7006969451904\\
13.2088947296143 14.7006969451904\\
13.308895111084 14.7006969451904\\
13.4088954925537 14.7006969451904\\
13.5088949203491 14.7006969451904\\
13.6088953018188 14.7006969451904\\
13.7088947296143 14.7006959915161\\
13.808895111084 14.7006959915161\\
13.9088954925537 14.7006959915161\\
14.0088949203491 14.7006959915161\\
14.1088953018188 14.7006959915161\\
14.2088947296143 14.7006959915161\\
14.308895111084 14.7006959915161\\
14.4088954925537 14.7006959915161\\
14.5088949203491 14.7006959915161\\
14.6088953018188 14.7006959915161\\
14.7088947296143 14.7006959915161\\
14.808895111084 14.7006959915161\\
14.9088954925537 14.7006959915161\\
15.0088949203491 14.7006959915161\\
15.1088953018188 14.7006959915161\\
15.2088947296143 14.7006959915161\\
15.308895111084 14.7006959915161\\
15.4088954925537 14.7006969451904\\
15.5088949203491 14.7006959915161\\
15.6088953018188 14.7006959915161\\
15.7088947296143 14.7006959915161\\
15.808895111084 14.7006959915161\\
15.9088954925537 14.7006959915161\\
16.0088958740234 14.7006959915161\\
16.1088943481445 14.7006959915161\\
16.2088947296143 14.7006959915161\\
16.308895111084 14.7006959915161\\
16.4088954925537 14.7006959915161\\
16.5088958740234 14.7006959915161\\
16.6088943481445 14.7006959915161\\
16.7088947296143 14.7006959915161\\
16.808895111084 14.7006959915161\\
16.9088954925537 14.7006959915161\\
17.0088958740234 14.7006959915161\\
17.1088943481445 14.7006959915161\\
17.2088947296143 14.7006959915161\\
17.308895111084 14.7006959915161\\
17.4088954925537 14.7006959915161\\
17.5088958740234 14.7006959915161\\
17.6088943481445 14.7006959915161\\
17.7088947296143 14.7006959915161\\
17.808895111084 14.7006969451904\\
17.9088954925537 14.7006988525391\\
18.0088958740234 14.7007055282593\\
18.1088943481445 14.700737953186\\
18.2088947296143 14.7009048461914\\
18.308895111084 14.7017889022827\\
18.4088954925537 14.705792427063\\
18.5088958740234 14.7166261672974\\
18.6088943481445 14.7299156188965\\
18.7088947296143 14.7327671051025\\
18.808895111084 14.7320785522461\\
18.9088954925537 14.7330684661865\\
19.0088958740234 14.7325553894043\\
19.1088943481445 14.7335662841797\\
19.2088947296143 14.7337579727173\\
19.308895111084 14.7345132827759\\
19.4088954925537 14.7348937988281\\
19.5088958740234 14.7355241775513\\
19.6088943481445 14.7359867095947\\
19.7088947296143 14.7365627288818\\
19.808895111084 14.7370634078979\\
19.9088954925537 14.7376136779785\\
20.0088958740234 14.7381315231323\\
20.1088943481445 14.7386722564697\\
20.2088947296143 14.7391967773438\\
20.308895111084 14.7397327423096\\
20.4088954925537 14.7402610778809\\
20.5088958740234 14.740795135498\\
20.6088943481445 14.7413263320923\\
20.7088947296143 14.7418594360352\\
20.808895111084 14.7423915863037\\
20.9088954925537 14.7429256439209\\
21.0088958740234 14.7434587478638\\
21.1088943481445 14.7440614700317\\
21.2088947296143 14.7448883056641\\
21.308895111084 14.7454996109009\\
21.4088954925537 14.7461566925049\\
21.5088958740234 14.7467947006226\\
21.6088943481445 14.7474365234375\\
21.7088947296143 14.7480764389038\\
21.808895111084 14.7487154006958\\
21.9088954925537 14.7493534088135\\
22.0088958740234 14.7499904632568\\
22.1088943481445 14.7506275177002\\
22.2088947296143 14.7512645721436\\
22.308895111084 14.7519016265869\\
22.4088954925537 14.752537727356\\
22.5088958740234 14.7531747817993\\
22.6088943481445 14.753794670105\\
22.7088947296143 14.7541408538818\\
22.808895111084 14.7546730041504\\
22.9088954925537 14.7552080154419\\
23.0088958740234 14.7557401657104\\
23.1088943481445 14.7562694549561\\
23.2088947296143 14.7567949295044\\
23.308895111084 14.7573156356812\\
23.4088954925537 14.7578315734863\\
23.5088958740234 14.7583417892456\\
23.6088943481445 14.758846282959\\
23.7088947296143 14.7593812942505\\
23.808895111084 14.760027885437\\
23.9088954925537 14.7606897354126\\
24.0088958740234 14.7613086700439\\
24.1088943481445 14.7619352340698\\
24.2088947296143 14.7625885009766\\
24.308895111084 14.7632789611816\\
24.4088954925537 14.7640314102173\\
24.5088958740234 14.7671670913696\\
24.6048831939697 14.7735557556152\\
24.681676864624 14.7769861221313\\
24.7464084625244 14.7684936523438\\
24.8176136016846 14.7686386108398\\
24.895938873291 14.7690258026123\\
24.9820976257324 14.7695426940918\\
25.0768718719482 14.7702665328979\\
25.176872253418 14.7711944580078\\
25.2768707275391 14.7722263336182\\
25.3768711090088 14.7732744216919\\
25.4768714904785 14.7743406295776\\
25.5768718719482 14.7754201889038\\
25.676872253418 14.7765102386475\\
25.7768707275391 14.7776050567627\\
25.8768711090088 14.7787103652954\\
25.9768714904785 14.7798185348511\\
26.0768718719482 14.780933380127\\
26.176872253418 14.7820472717285\\
26.2768707275391 14.7831611633301\\
26.3768711090088 14.7842750549316\\
26.4768714904785 14.7853899002075\\
26.5768718719482 14.7865047454834\\
26.676872253418 14.787618637085\\
26.7768707275391 14.7887334823608\\
26.8768711090088 14.7898483276367\\
26.9768714904785 14.7909631729126\\
27.0768718719482 14.7920780181885\\
27.176872253418 14.7931938171387\\
27.2768707275391 14.7943086624146\\
27.3768711090088 14.7954244613647\\
27.4768714904785 14.7965402603149\\
27.5768718719482 14.7976560592651\\
27.676872253418 14.7987718582153\\
27.7768707275391 14.7998876571655\\
27.8768711090088 14.80100440979\\
27.9768714904785 14.8021211624146\\
28.0768718719482 14.8032369613647\\
28.176872253418 14.8043537139893\\
28.2768707275391 14.8054780960083\\
28.3768711090088 14.8065824508667\\
28.4768714904785 14.8077096939087\\
28.5768718719482 14.8088302612305\\
28.676872253418 14.8099536895752\\
28.7768707275391 14.8110761642456\\
28.8768711090088 14.8121995925903\\
28.9768714904785 14.8133220672607\\
29.0768718719482 14.8144454956055\\
29.176872253418 14.8155679702759\\
29.2768707275391 14.8166913986206\\
29.3768711090088 14.8178148269653\\
29.4768714904785 14.8189373016357\\
29.5768718719482 14.8200607299805\\
29.676872253418 14.8211841583252\\
29.7768707275391 14.8223075866699\\
29.8768711090088 14.8234310150146\\
29.9768714904785 14.8245544433594\\
30.0768718719482 14.8256788253784\\
30.176872253418 14.8268051147461\\
30.2768707275391 14.8279371261597\\
30.3768711090088 14.8291063308716\\
30.4768714904785 14.8304738998413\\
30.5768718719482 14.8328371047974\\
30.676872253418 14.8383226394653\\
30.7768707275391 14.8454103469849\\
30.8768711090088 14.8438529968262\\
30.9768714904785 14.8406610488892\\
31.0768718719482 14.8382902145386\\
31.176872253418 14.8392992019653\\
31.2768707275391 14.8395013809204\\
31.3768711090088 14.8411102294922\\
31.4768714904785 14.8419094085693\\
31.5768718719482 14.8432512283325\\
31.676872253418 14.8442316055298\\
31.7768707275391 14.8454532623291\\
31.8768711090088 14.8465137481689\\
31.9768714904785 14.8476819992065\\
32.0768699645996 14.8487787246704\\
32.176872253418 14.8499231338501\\
32.2768707275391 14.8510360717773\\
32.3768730163574 14.8521709442139\\
32.4768714904785 14.8532905578613\\
32.5768699645996 14.854419708252\\
32.676872253418 14.8555431365967\\
32.7768707275391 14.8566703796387\\
32.8768730163574 14.857795715332\\
32.9768714904785 14.8589220046997\\
33.0768699645996 14.8600473403931\\
33.176872253418 14.8611736297607\\
33.2768707275391 14.8622999191284\\
33.3768730163574 14.8636255264282\\
33.4768714904785 14.8648624420166\\
33.5768699645996 14.8659763336182\\
33.676872253418 14.8671083450317\\
33.7768707275391 14.8682336807251\\
33.8768730163574 14.8693618774414\\
33.9768714904785 14.8704891204834\\
34.0768699645996 14.8716173171997\\
34.176872253418 14.872745513916\\
34.2768707275391 14.8738746643066\\
34.3768730163574 14.8750038146973\\
34.4768714904785 14.8761329650879\\
34.5768699645996 14.8772640228271\\
34.676872253418 14.8783941268921\\
34.7768707275391 14.8795261383057\\
34.8768730163574 14.880597114563\\
34.9768714904785 14.8814125061035\\
35.0768699645996 14.8826198577881\\
35.176872253418 14.8837060928345\\
35.2768707275391 14.8848352432251\\
35.3768730163574 14.885947227478\\
35.4768714904785 14.8870639801025\\
35.5768699645996 14.8881778717041\\
35.676872253418 14.8892917633057\\
35.7768707275391 14.8904037475586\\
35.8768730163574 14.8915138244629\\
35.9768714904785 14.8926858901978\\
36.0768699645996 14.8939428329468\\
36.176872253418 14.895133972168\\
36.2768707275391 14.8962869644165\\
36.3768730163574 14.8974504470825\\
36.4768714904785 14.8986320495605\\
36.5768699645996 14.8998432159424\\
36.676872253418 14.9011030197144\\
36.7768707275391 14.9067459106445\\
36.8673248291016 14.911337852478\\
36.9396858215332 14.9112911224365\\
37.0192794799805 14.9049701690674\\
37.1068305969238 14.9056196212769\\
37.2031402587891 14.9062757492065\\
37.3031425476074 14.9071912765503\\
37.4031410217285 14.9082527160645\\
37.5031433105469 14.909369468689\\
37.603141784668 14.910472869873\\
37.7031402587891 14.9115839004517\\
37.8031425476074 14.9126968383789\\
37.9031410217285 14.9138154983521\\
38.0031433105469 14.9149293899536\\
38.103141784668 14.9160499572754\\
38.2031402587891 14.9171714782715\\
38.3031425476074 14.9182968139648\\
38.4031410217285 14.9194240570068\\
38.5031433105469 14.9205503463745\\
38.603141784668 14.9216775894165\\
38.7031402587891 14.9228048324585\\
38.8031425476074 14.9239311218262\\
38.9031410217285 14.9250583648682\\
39.0031433105469 14.9261856079102\\
39.103141784668 14.9273128509521\\
39.2031402587891 14.9284391403198\\
39.3031425476074 14.9295663833618\\
39.4031410217285 14.9306936264038\\
39.5031433105469 14.9318199157715\\
39.603141784668 14.9329471588135\\
39.7031402587891 14.9340744018555\\
39.8031425476074 14.9352006912231\\
39.9031410217285 14.9363279342651\\
40.0031433105469 14.9374551773071\\
40.103141784668 14.9385814666748\\
40.2031402587891 14.9397087097168\\
40.3031425476074 14.9408359527588\\
40.4031410217285 14.9419622421265\\
40.5031433105469 14.9430894851685\\
40.603141784668 14.9442167282104\\
40.7031402587891 14.9453430175781\\
40.8031425476074 14.9464702606201\\
40.9031410217285 14.9476041793823\\
41.0031433105469 14.9487133026123\\
41.103141784668 14.9498453140259\\
41.2031402587891 14.9509706497192\\
41.3031425476074 14.9520988464355\\
41.4031410217285 14.9532260894775\\
41.5031433105469 14.9543533325195\\
41.603141784668 14.9554796218872\\
41.7031402587891 14.9566068649292\\
41.8031425476074 14.9577341079712\\
41.9031410217285 14.9588613510132\\
42.0031433105469 14.9599885940552\\
42.103141784668 14.9611158370972\\
42.2031402587891 14.9622430801392\\
42.3031425476074 14.9633712768555\\
42.4031410217285 14.9644994735718\\
42.5031433105469 14.9656343460083\\
42.603141784668 14.9667940139771\\
42.7031402587891 14.9680976867676\\
42.8031425476074 14.9701538085938\\
42.9031410217285 14.9748525619507\\
43.0031433105469 14.9814624786377\\
43.103141784668 14.982684135437\\
43.2031402587891 14.9798784255981\\
43.3031425476074 14.9765090942383\\
43.4031410217285 14.9770517349243\\
43.5031433105469 14.9773063659668\\
43.603141784668 14.9787702560425\\
43.7031402587891 14.9795446395874\\
43.8031425476074 14.9810352325439\\
43.9031410217285 14.9819221496582\\
44.0031433105469 14.9832096099854\\
44.103141784668 14.9842300415039\\
44.2031402587891 14.9854297637939\\
44.3031425476074 14.9865093231201\\
44.4031410217285 14.9876689910889\\
44.5031433105469 14.9887752532959\\
44.603141784668 14.9899168014526\\
44.7031402587891 14.9910354614258\\
44.8031425476074 14.992169380188\\
44.9031410217285 14.9932928085327\\
45.0031433105469 14.994423866272\\
45.103141784668 14.9955501556396\\
45.2031402587891 14.9966793060303\\
45.3031425476074 14.9978055953979\\
45.4031410217285 14.9989347457886\\
45.5031433105469 15.0000619888306\\
45.603141784668 15.0013332366943\\
45.7031402587891 15.0026388168335\\
45.8031425476074 15.0037231445313\\
45.9031410217285 15.004864692688\\
46.0031433105469 15.0059881210327\\
46.103141784668 15.0071172714233\\
46.2031402587891 15.0082454681396\\
46.3031425476074 15.009373664856\\
46.4031410217285 15.0105009078979\\
46.5031433105469 15.0116291046143\\
46.603141784668 15.0127573013306\\
46.7031402587891 15.0138854980469\\
46.8031425476074 15.0150127410889\\
46.9031410217285 15.0161409378052\\
47.0031433105469 15.0172691345215\\
47.103141784668 15.0183916091919\\
47.2031402587891 15.0203266143799\\
47.3031425476074 15.0206699371338\\
47.4031410217285 15.0214920043945\\
47.5031433105469 15.0226879119873\\
47.603141784668 15.0237770080566\\
47.7031402587891 15.0249061584473\\
47.8031425476074 15.0260190963745\\
47.9031410217285 15.0271368026733\\
48.0031433105469 15.0282506942749\\
48.103141784668 15.0293645858765\\
48.2031402587891 15.0305166244507\\
48.3031425476074 15.0316934585571\\
48.4031410217285 15.0330066680908\\
48.5031433105469 15.0341377258301\\
48.603141784668 15.0352754592896\\
48.7031402587891 15.0364570617676\\
48.8031425476074 15.0376577377319\\
48.9031410217285 15.0389041900635\\
49.0031433105469 15.0429487228394\\
49.098030090332 15.0483713150024\\
49.1739387512207 15.0525941848755\\
49.2346649169922 15.0418338775635\\
49.3014602661133 15.0614671707153\\
49.3450546264648 15.0167808532715\\
49.3930053710938 15.0322408676147\\
49.4215202331543 15.0553159713745\\
49.4451713562012 15.1996784210205\\
49.47119140625 14.9171228408813\\
49.4998092651367 15.0412168502808\\
49.515323638916 15.0481758117676\\
49.5276222229004 15.3453989028931\\
49.5411529541016 14.7836408615112\\
49.5560340881348 14.9733180999756\\
49.5575561523438 15.5988883972168\\
49.5592269897461 15.2169075012207\\
49.5610656738281 15.0362415313721\\
49.5630912780762 15.0616502761841\\
49.565315246582 15.0628156661987\\
49.5677642822266 15.0585985183716\\
49.5704574584961 15.0552501678467\\
49.5734214782715 15.0526494979858\\
49.5766792297363 15.0506343841553\\
49.5791168212891 15.0494794845581\\
49.5817947387695 15.0485420227051\\
49.5847434997559 15.0477914810181\\
49.5879859924316 15.0472068786621\\
49.591552734375 15.0467653274536\\
49.5954742431641 15.046441078186\\
49.5997886657715 15.0462112426758\\
49.6045341491699 15.0460548400879\\
49.6097564697266 15.045952796936\\
49.6155014038086 15.0458908081055\\
49.6218185424805 15.0458574295044\\
49.6287651062012 15.0458450317383\\
49.6364097595215 15.0458478927612\\
49.6448211669922 15.045862197876\\
49.6540679931641 15.0458869934082\\
49.6642456054688 15.0459232330322\\
49.6754379272461 15.046049118042\\
49.6877479553223 15.0462636947632\\
49.7012901306152 15.046422958374\\
49.7161865234375 15.0465211868286\\
49.7325744628906 15.046558380127\\
49.7505989074707 15.0465784072876\\
49.7704238891602 15.046630859375\\
49.792236328125 15.0469255447388\\
49.8162269592285 15.0472068786621\\
49.8426170349121 15.0474224090576\\
49.8716468811035 15.047607421875\\
49.9035758972168 15.0478286743164\\
49.9387016296387 15.0480985641479\\
49.9773406982422 15.0484647750854\\
50.0198440551758 15.0489253997803\\
50.066593170166 15.0494966506958\\
50.1180229187012 15.0502109527588\\
50.1745910644531 15.0510606765747\\
50.23681640625 15.0558061599731\\
50.3052673339844 15.0501766204834\\
50.3805618286133 15.0571479797363\\
50.4633865356445 15.0530424118042\\
50.5544929504395 15.0593967437744\\
50.6544914245605 15.0542850494385\\
50.7544898986816 15.061469078064\\
50.8544921875 15.0562057495117\\
50.9544906616211 15.0637588500977\\
51.0544929504395 15.0582933425903\\
51.1544914245605 15.0659532546997\\
51.2544898986816 15.0605354309082\\
51.3544921875 15.0628519058228\\
51.3568458557129 15.1776628494263\\
51.359432220459 15.0918312072754\\
51.3622817993164 15.0505838394165\\
51.3654136657715 15.0570240020752\\
51.3688583374023 15.0666770935059\\
51.3726463317871 15.0684490203857\\
51.3768157958984 15.0675754547119\\
51.3814010620117 15.0671768188477\\
51.3864479064941 15.067177772522\\
51.3919944763184 15.0672006607056\\
51.3980979919434 15.0671977996826\\
51.4048118591309 15.0671949386597\\
51.4121971130371 15.0671987533569\\
51.4203224182129 15.0672121047974\\
51.4292602539063 15.0672340393066\\
51.4390869140625 15.0672664642334\\
51.4499015808105 15.0673122406006\\
51.4617958068848 15.0673723220825\\
51.4748802185059 15.0674486160278\\
51.4892692565918 15.0675430297852\\
51.5051002502441 15.0676565170288\\
51.522518157959 15.0677928924561\\
51.5416717529297 15.0679521560669\\
51.562744140625 15.0681371688843\\
51.5859222412109 15.0683498382568\\
51.6114196777344 15.0685920715332\\
51.6394653320313 15.0688638687134\\
51.6703147888184 15.0691699981689\\
51.7042503356934 15.0695104598999\\
51.7415771484375 15.0698881149292\\
51.7826385498047 15.0703067779541\\
51.827808380127 15.0707683563232\\
51.8774948120117 15.0712776184082\\
51.9321479797363 15.0718374252319\\
51.9922637939453 15.0724534988403\\
52.0583953857422 15.0731325149536\\
52.1311378479004 15.073878288269\\
52.2111587524414 15.074728012085\\
52.2991790771484 15.0782670974731\\
52.3959999084473 15.0738563537598\\
52.4959983825684 15.0777912139893\\
52.5960006713867 15.0849552154541\\
52.6160011291504 15.0791358947754\\
52.6228942871094 15.0967102050781\\
52.6304740905762 15.1012353897095\\
52.637622833252 15.1024742126465\\
52.6454811096191 15.1027212142944\\
52.6532897949219 15.1024570465088\\
52.6618766784668 15.0995054244995\\
52.6713256835938 15.0964727401733\\
52.6817169189453 15.0960855484009\\
52.6931457519531 15.0942506790161\\
52.7057189941406 15.0898180007935\\
52.7195510864258 15.0906715393066\\
52.7347640991211 15.0996828079224\\
52.7514991760254 15.1134643554688\\
52.7699089050293 15.1027116775513\\
52.7901573181152 15.0894374847412\\
52.8124351501465 15.0818700790405\\
52.8369369506836 15.0738096237183\\
52.8638877868652 15.0656490325928\\
52.8935356140137 15.0603694915771\\
52.9261474609375 15.0575361251831\\
52.9620208740234 15.0577774047852\\
53.0014839172363 15.0618190765381\\
53.0448913574219 15.0659408569336\\
53.0926399230957 15.0635643005371\\
53.1261940002441 15.0699710845947\\
53.1631088256836 15.0866365432739\\
53.1971549987793 15.0871925354004\\
53.2346038818359 15.0896949768066\\
53.2757987976074 15.0915393829346\\
53.3211135864258 15.0924863815308\\
53.3709602355957 15.092848777771\\
53.4181785583496 15.0924978256226\\
53.4612884521484 15.0924053192139\\
53.4995498657227 15.0926713943481\\
53.5327682495117 15.0976362228394\\
53.5693130493164 15.0957412719727\\
53.6095085144043 15.0934238433838\\
53.6537246704102 15.0917320251465\\
53.7023620605469 15.0920763015747\\
53.7558631896973 15.0924835205078\\
53.8147125244141 15.0928478240967\\
53.8794479370117 15.0933637619019\\
53.9506568908691 15.0940237045288\\
54.0289878845215 15.0947551727295\\
54.1151542663574 15.0955820083618\\
54.2099342346191 15.0976600646973\\
54.3099327087402 15.0962791442871\\
54.4099311828613 15.100025177002\\
54.5099334716797 15.0983791351318\\
54.6099319458008 15.1001815795898\\
54.6148452758789 15.1453304290771\\
54.6202507019043 15.0862865447998\\
54.6261940002441 15.1190767288208\\
54.6327362060547 15.0873193740845\\
54.6399269104004 15.1174182891846\\
54.6478385925293 15.0895843505859\\
54.6565437316895 15.1153011322021\\
54.6661186218262 15.091649055481\\
54.6766510009766 15.1007175445557\\
54.6824378967285 15.1071605682373\\
54.6888046264648 15.1035499572754\\
54.6958045959473 15.10325050354\\
54.7035064697266 15.1033515930176\\
54.7119827270508 15.103401184082\\
54.7213020324707 15.1034526824951\\
54.7315559387207 15.1035146713257\\
54.7428321838379 15.1035900115967\\
54.7552375793457 15.1036815643311\\
54.7688827514648 15.1037855148315\\
54.7838935852051 15.103907585144\\
54.800407409668 15.1040496826172\\
54.8185691833496 15.104211807251\\
54.8385467529297 15.1043977737427\\
54.8605270385742 15.1046094894409\\
54.8847007751465 15.1048460006714\\
54.9112930297852 15.1051235198975\\
54.940544128418 15.1054344177246\\
54.9727210998535 15.1057929992676\\
55.008113861084 15.1062183380127\\
55.0470504760742 15.106746673584\\
55.0898742675781 15.1074590682983\\
55.1369857788086 15.1085395812988\\
55.1888084411621 15.1103639602661\\
55.2458114624023 15.1135873794556\\
55.3085136413574 15.1181612014771\\
55.3774871826172 15.1220607757568\\
55.4533576965332 15.1192579269409\\
55.5368156433105 15.1170024871826\\
55.6286201477051 15.116114616394\\
55.7286186218262 15.1155796051025\\
55.8286209106445 15.1150817871094\\
55.9286193847656 15.1163215637207\\
56.0286178588867 15.1172952651978\\
56.1286201477051 15.1186971664429\\
56.2286186218262 15.1196451187134\\
56.3286209106445 15.1208915710449\\
56.4286193847656 15.121940612793\\
56.5286178588867 15.1231203079224\\
56.6286201477051 15.1242141723633\\
56.7286186218262 15.1253643035889\\
56.8286209106445 15.1265106201172\\
56.9286193847656 15.1276264190674\\
57.0286178588867 15.1287298202515\\
57.1286201477051 15.1298685073853\\
57.2286186218262 15.1309871673584\\
57.3286209106445 15.1321210861206\\
57.4286193847656 15.1332454681396\\
57.5286178588867 15.1343765258789\\
57.6286201477051 15.1355018615723\\
57.7286186218262 15.1366319656372\\
57.8286209106445 15.1377592086792\\
57.9286193847656 15.1388883590698\\
58.0286178588867 15.1403350830078\\
58.1286201477051 15.1414136886597\\
58.2286186218262 15.1425638198853\\
58.3286209106445 15.1436834335327\\
58.4286193847656 15.1448135375977\\
58.5286178588867 15.1459407806396\\
58.6286201477051 15.147068977356\\
58.7286186218262 15.1481971740723\\
58.8286209106445 15.1493253707886\\
58.9286193847656 15.1504535675049\\
59.0286178588867 15.1515827178955\\
59.1286201477051 15.1527128219604\\
59.2286186218262 15.1538410186768\\
59.3286209106445 15.1549692153931\\
59.4286193847656 15.1560974121094\\
59.5286178588867 15.1570501327515\\
59.6286201477051 15.1580305099487\\
59.7286186218262 15.1591939926147\\
59.8286209106445 15.16028881073\\
59.9143104553223 15.1626892089844\\
60 15.1638212203979\\
60.0942611694336 15.1636819839478\\
60.1942596435547 15.1645278930664\\
60.2942581176758 15.1656789779663\\
60.3942604064941 15.1667966842651\\
60.4942588806152 15.1679286956787\\
60.5942611694336 15.1690587997437\\
60.6942596435547 15.1701946258545\\
60.7942581176758 15.1713438034058\\
60.8942604064941 15.1724891662598\\
60.9942588806152 15.1736459732056\\
61.0942611694336 15.1748208999634\\
61.1942596435547 15.1760215759277\\
61.2942581176758 15.1772804260254\\
61.3942604064941 15.1831245422363\\
61.4835205078125 15.1873540878296\\
61.5549278259277 15.1866474151611\\
61.6334800720215 15.1812295913696\\
61.7198829650879 15.1818571090698\\
61.8149299621582 15.1982460021973\\
61.8549957275391 15.1466770172119\\
61.8990707397461 15.1783599853516\\
61.9192581176758 15.1903600692749\\
61.9358787536621 15.1935930252075\\
61.9541664123535 15.1851968765259\\
61.9742813110352 15.1852073669434\\
61.9964065551758 15.1853675842285\\
62.0207443237305 15.1855373382568\\
62.0475158691406 15.1857442855835\\
62.0769653320313 15.1859922409058\\
62.1093559265137 15.1862812042236\\
62.1449890136719 15.1866149902344\\
62.1841850280762 15.1869945526123\\
62.2273025512695 15.1874237060547\\
62.274730682373 15.1879034042358\\
62.3269004821777 15.1884365081787\\
62.3842887878418 15.1890249252319\\
62.4474143981934 15.1896743774414\\
62.5168533325195 15.1903877258301\\
62.5932350158691 15.1911725997925\\
62.6772575378418 15.201229095459\\
62.7696800231934 15.1843509674072\\
62.8696823120117 15.1941270828247\\
62.9696807861328 15.1983165740967\\
63.0696792602539 15.2026996612549\\
63.1196823120117 15.2522468566895\\
63.1572036743164 15.2112684249878\\
63.1601943969727 15.1924457550049\\
63.1634902954102 15.2007713317871\\
63.1671104431152 15.2183818817139\\
63.1710968017578 15.2204084396362\\
63.175479888916 15.2076606750488\\
63.1803016662598 15.1977243423462\\
63.185604095459 15.1862592697144\\
63.1914367675781 15.1751899719238\\
63.1978569030762 15.1683444976807\\
63.2049140930176 15.1635484695435\\
63.2126808166504 15.1586484909058\\
63.2212219238281 15.1523551940918\\
63.2306175231934 15.1434078216553\\
63.2409515380859 15.1331386566162\\
63.2523231506348 15.1248407363892\\
63.2648277282715 15.1209726333618\\
63.2785835266113 15.1221284866333\\
63.2937164306641 15.1282186508179\\
63.310359954834 15.1403388977051\\
63.328670501709 15.157751083374\\
63.3488121032715 15.1830720901489\\
63.3709678649902 15.1756792068481\\
63.395336151123 15.1801557540894\\
63.4221420288086 15.1926898956299\\
63.4516296386719 15.1905107498169\\
63.4840698242188 15.1960134506226\\
63.4999580383301 15.2045049667358\\
63.5174407958984 15.2093963623047\\
63.5366668701172 15.2136011123657\\
63.5578193664551 15.2201042175293\\
63.5810852050781 15.2246379852295\\
63.6066780090332 15.2214879989624\\
63.6348304748535 15.2147998809814\\
63.6657981872559 15.211368560791\\
63.6998634338379 15.2102966308594\\
63.7373313903809 15.2098836898804\\
63.7785491943359 15.2094049453735\\
63.8238906860352 15.2092361450195\\
63.8737640380859 15.2097120285034\\
63.9286231994629 15.2107381820679\\
63.9762001037598 15.2108507156372\\
64.0218276977539 15.210844039917\\
64.0647964477539 15.2120227813721\\
64.112060546875 15.213677406311\\
64.1640548706055 15.2123365402222\\
64.2212448120117 15.2097244262695\\
64.2841567993164 15.2100276947021\\
64.3533554077148 15.2105836868286\\
64.4294738769531 15.2110958099365\\
64.5132064819336 15.2118263244629\\
64.6053161621094 15.2147092819214\\
64.7053146362305 15.2116003036499\\
64.8053131103516 15.2170057296753\\
64.9053115844727 15.2135372161865\\
65.0053176879883 15.2159252166748\\
65.0079345703125 15.3255929946899\\
65.0108184814453 15.1884622573853\\
65.0139923095703 15.1785326004028\\
65.0165023803711 15.2264699935913\\
65.0192642211914 15.2408866882324\\
65.0223007202148 15.2225103378296\\
65.025634765625 15.2178754806519\\
65.0293121337891 15.2188453674316\\
65.0333557128906 15.2192363739014\\
65.0377960205078 15.2192134857178\\
65.0426864624023 15.2191905975342\\
65.0480651855469 15.2191905975342\\
65.0539855957031 15.2191944122314\\
65.0604934692383 15.2192039489746\\
65.0676498413086 15.2192192077637\\
65.0755233764648 15.2192420959473\\
65.0841903686523 15.2192735671997\\
65.0937194824219 15.2193145751953\\
65.1042022705078 15.219367980957\\
65.1157302856445 15.2194337844849\\
65.1284103393555 15.2195148468018\\
65.1423645019531 15.2196130752563\\
65.1577072143555 15.2197294235229\\
65.1745910644531 15.2198657989502\\
65.1931610107422 15.220025062561\\
65.2135848999023 15.2202072143555\\
65.2360534667969 15.2204160690308\\
65.2607727050781 15.2206516265869\\
65.287956237793 15.2209167480469\\
65.3178634643555 15.2212142944336\\
65.3507614135742 15.2215442657471\\
65.3869476318359 15.2219114303589\\
65.4267501831055 15.2223167419434\\
65.4705352783203 15.2227640151978\\
65.5186996459961 15.2232570648193\\
65.5716781616211 15.2238006591797\\
65.6299591064453 15.2243976593018\\
65.6940612792969 15.2250556945801\\
65.7645797729492 15.2257785797119\\
65.8421478271484 15.2265739440918\\
65.9274749755859 15.2274494171143\\
66.0213317871094 15.2301216125488\\
66.1213302612305 15.227593421936\\
66.2213287353516 15.2307081222534\\
66.3213272094727 15.2396078109741\\
66.3413314819336 15.2474565505981\\
66.3490600585938 15.2748823165894\\
66.3565673828125 15.2789211273193\\
66.3648223876953 15.2754240036011\\
66.3727111816406 15.2719421386719\\
66.3813858032227 15.2776584625244\\
66.3909301757813 15.3070783615112\\
66.4014358520508 15.3377952575684\\
66.4129867553711 15.3409328460693\\
66.4256896972656 15.318549156189\\
66.4396667480469 15.2981052398682\\
66.4550399780273 15.2754669189453\\
66.4719543457031 15.2517070770264\\
66.4905548095703 15.2274150848389\\
66.5110244750977 15.196891784668\\
66.5335311889648 15.161524772644\\
66.5504302978516 15.1465663909912\\
66.5690307617188 15.1374225616455\\
66.5894775390625 15.1345796585083\\
66.6119766235352 15.1544103622437\\
66.6367263793945 15.181321144104\\
66.663948059082 15.2350606918335\\
66.6938934326172 15.2418689727783\\
66.7268295288086 15.2445688247681\\
66.7630615234375 15.2459135055542\\
66.8029174804688 15.2450160980225\\
66.8467636108398 15.2432069778442\\
66.8949890136719 15.2418699264526\\
66.9480361938477 15.2552146911621\\
67.0063934326172 15.2384090423584\\
67.0705795288086 15.2503252029419\\
67.111457824707 15.2336893081665\\
67.1564254760742 15.2372798919678\\
67.1973037719727 15.2371482849121\\
67.2330703735352 15.2380876541138\\
67.2724151611328 15.2396621704102\\
67.3156967163086 15.2417964935303\\
67.3633117675781 15.2433214187622\\
67.4156799316406 15.2448740005493\\
67.4732894897461 15.2472190856934\\
67.5366516113281 15.2498083114624\\
67.6063613891602 15.2529439926147\\
67.6830368041992 15.2557792663574\\
67.7673797607422 15.2534894943237\\
67.8601531982422 15.2534856796265\\
67.9601516723633 15.2530069351196\\
68.0601577758789 15.2527360916138\\
68.16015625 15.2524242401123\\
68.2601547241211 15.2531938552856\\
68.3601531982422 15.2543420791626\\
68.4601516723633 15.2556104660034\\
68.5601577758789 15.2566413879395\\
68.66015625 15.2578296661377\\
68.7601547241211 15.2589159011841\\
68.8601531982422 15.2600698471069\\
68.9601516723633 15.2611799240112\\
69.0601577758789 15.2623205184937\\
69.16015625 15.2634401321411\\
69.2601547241211 15.2645740509033\\
69.3601531982422 15.2656984329224\\
69.4601516723633 15.2668294906616\\
69.5601577758789 15.2679557800293\\
69.66015625 15.2690849304199\\
69.7601547241211 15.2702131271362\\
69.8601531982422 15.2713413238525\\
69.9601516723633 15.2724695205688\\
70.0601577758789 15.2735977172852\\
70.16015625 15.2747259140015\\
70.2601547241211 15.2758665084839\\
70.3601531982422 15.2770357131958\\
70.4601516723633 15.2784490585327\\
70.5601577758789 15.2795009613037\\
70.66015625 15.2806520462036\\
70.7601547241211 15.281774520874\\
70.8601531982422 15.282904624939\\
70.9601516723633 15.2840328216553\\
71.0601577758789 15.2851619720459\\
71.16015625 15.2862901687622\\
71.2601547241211 15.2874193191528\\
71.3601531982422 15.2885475158691\\
71.4601516723633 15.2896766662598\\
71.5601577758789 15.2908048629761\\
71.66015625 15.2919330596924\\
71.7601547241211 15.293062210083\\
71.8601531982422 15.2941904067993\\
71.9601516723633 15.2951078414917\\
72.0601577758789 15.2961349487305\\
72.16015625 15.2972793579102\\
72.2601547241211 15.2983818054199\\
72.3601531982422 15.2994976043701\\
72.4601516723633 15.3006057739258\\
72.5601577758789 15.3017139434814\\
72.66015625 15.3028182983398\\
72.7601547241211 15.303918838501\\
72.8601531982422 15.3050155639648\\
72.9601516723633 15.3061075210571\\
73.0601577758789 15.3071956634521\\
73.16015625 15.3117561340332\\
73.2601547241211 15.3106756210327\\
73.3601531982422 15.3108730316162\\
73.4601516723633 15.3122224807739\\
73.5601577758789 15.3133325576782\\
73.66015625 15.3145837783813\\
73.7601547241211 15.3160724639893\\
73.8601531982422 15.323655128479\\
73.943000793457 15.3270483016968\\
74.00927734375 15.3217058181763\\
74.0821685791016 15.3195009231567\\
74.162353515625 15.3200702667236\\
74.2505569458008 15.3374328613281\\
74.2926559448242 15.2870559692383\\
74.3389587402344 15.316068649292\\
74.3595275878906 15.3296318054199\\
74.3765106201172 15.3321495056152\\
74.3951873779297 15.3234100341797\\
74.4157409667969 15.3234071731567\\
74.438346862793 15.3235759735107\\
74.4632110595703 15.3237495422363\\
74.4905624389648 15.3239631652832\\
74.5206527709961 15.3242168426514\\
74.5537414550781 15.3245143890381\\
74.5901489257813 15.3248567581177\\
74.6301956176758 15.3252468109131\\
74.6742477416992 15.3256864547729\\
74.7227020263672 15.3261775970459\\
74.7760009765625 15.3267230987549\\
74.8346328735352 15.3273248672485\\
74.8991317749023 15.3279876708984\\
74.9495620727539 15.3287191390991\\
75 15.3292875289917\\
};
\addlegendentry{6 "};

\addplot [
color=black,
dashed
]
table[row sep=crcr]{
0 14.700608253479\\
9.99999971718069e-10 14.700608253479\\
2.49999998480632e-09 14.700608253479\\
4.74999994892755e-09 14.700608253479\\
8.1250002281763e-09 14.700608253479\\
1.31874999809156e-08 14.700608253479\\
2.0781250498203e-08 14.700608253479\\
3.21718758300449e-08 14.700608253479\\
4.92578138278077e-08 14.700608253479\\
7.48867208244519e-08 14.700608253479\\
1.13330081319418e-07 14.700608253479\\
1.70995122061868e-07 14.700608253479\\
2.57492672517401e-07 14.700608253479\\
3.8723902662241e-07 14.700608253479\\
5.81858500936505e-07 14.700608253479\\
8.73787769251066e-07 14.700608253479\\
1.31168167172291e-06 14.700608253479\\
1.9685226106958e-06 14.700608253479\\
2.95378367809462e-06 14.700608253479\\
4.43167573394021e-06 14.700608253479\\
6.64851359033491e-06 14.7006092071533\\
9.97376992017962e-06 14.7006092071533\\
1.49616553244414e-05 14.7006092071533\\
2.24434825213393e-05 14.7006101608276\\
3.36662233166862e-05 14.700611114502\\
5.05003372381907e-05 14.7006120681763\\
7.57515081204474e-05 14.7006130218506\\
0.000113628258986864 14.7006149291992\\
0.000170443381648511 14.7006158828735\\
0.000255666091106832 14.7006177902222\\
0.000383500126190484 14.7006187438965\\
0.000575251178815961 14.7006187438965\\
0.000862877757754177 14.7006168365479\\
0.00129431765526533 14.7006139755249\\
0.00194147753063589 14.7006101608276\\
0.00291221728548408 14.7006063461304\\
0.00436832662671804 14.7006044387817\\
0.00597004732117057 14.7006063461304\\
0.00773194013163447 14.700608253479\\
0.009670021943748 14.7006092071533\\
0.0118019115179777 14.7006092071533\\
0.0141469910740852 14.7006092071533\\
0.0167265776544809 14.700608253479\\
0.0195641238242388 14.700608253479\\
0.0226854234933853 14.700608253479\\
0.0261188540607691 14.700608253479\\
0.0298956278711557 14.700608253479\\
0.0340500771999359 14.700608253479\\
0.0386199727654457 14.700608253479\\
0.0436468608677387 14.700608253479\\
0.0491764321923256 14.700608253479\\
0.0552589632570744 14.700608253479\\
0.0619497485458851 14.700608253479\\
0.0693096145987511 14.700608253479\\
0.0774054601788521 14.7006044387817\\
0.0863108932971954 14.7006120681763\\
0.0961068719625473 14.7006092071533\\
0.106882445514202 14.700608253479\\
0.118735581636429 14.700608253479\\
0.13177402317524 14.7006092071533\\
0.146116316318512 14.7006092071533\\
0.161892831325531 14.7006092071533\\
0.179247006773949 14.7006092071533\\
0.198336601257324 14.7006092071533\\
0.21933513879776 14.7006092071533\\
0.242433547973633 14.700608253479\\
0.267841786146164 14.7006092071533\\
0.29579085111618 14.7006092071533\\
0.32653483748436 14.7006092071533\\
0.360353201627731 14.7006092071533\\
0.397553414106369 14.7006092071533\\
0.438473641872406 14.7006092071533\\
0.483485877513886 14.7006092071533\\
0.532999336719513 14.7006092071533\\
0.587464153766632 14.7006092071533\\
0.647375464439392 14.7006092071533\\
0.71327793598175 14.7006092071533\\
0.7857705950737 14.7006092071533\\
0.865512549877167 14.7006092071533\\
0.953228712081909 14.7006092071533\\
1.04971647262573 14.700608253479\\
1.14971649646759 14.700608253479\\
1.24971640110016 14.700608253479\\
1.34971642494202 14.700608253479\\
1.44971644878387 14.700608253479\\
1.54971647262573 14.700608253479\\
1.64971649646759 14.700608253479\\
1.74971640110016 14.700608253479\\
1.84971642494202 14.700608253479\\
1.94971644878387 14.700608253479\\
2.04971647262573 14.700608253479\\
2.1497163772583 14.700608253479\\
2.24971652030945 14.700608253479\\
2.34971642494202 14.700608253479\\
2.44971656799316 14.700608253479\\
2.54971647262573 14.700608253479\\
2.6497163772583 14.700608253479\\
2.74971652030945 14.700608253479\\
2.84971642494202 14.700608253479\\
2.94971656799316 14.700608253479\\
3.04971647262573 14.700608253479\\
3.1497163772583 14.700608253479\\
3.24971652030945 14.700608253479\\
3.34971642494202 14.700608253479\\
3.44971656799316 14.700608253479\\
3.54971647262573 14.700608253479\\
3.6497163772583 14.700608253479\\
3.74971652030945 14.700608253479\\
3.84971642494202 14.700608253479\\
3.94971656799316 14.700608253479\\
4.04971647262573 14.700608253479\\
4.1497163772583 14.700608253479\\
4.24971628189087 14.700608253479\\
4.3497166633606 14.700608253479\\
4.44971656799316 14.700608253479\\
4.54971647262573 14.700608253479\\
4.6497163772583 14.700608253479\\
4.74971628189087 14.700608253479\\
4.8497166633606 14.700608253479\\
4.94971656799316 14.700608253479\\
5.04971647262573 14.700608253479\\
5.1497163772583 14.700608253479\\
5.24971628189087 14.700608253479\\
5.3497166633606 14.700608253479\\
5.44971656799316 14.700608253479\\
5.54971647262573 14.700608253479\\
5.6497163772583 14.700608253479\\
5.74971628189087 14.700608253479\\
5.8497166633606 14.700608253479\\
5.94971656799316 14.700608253479\\
6.04971647262573 14.700608253479\\
6.1497163772583 14.7006072998047\\
6.24971628189087 14.7006063461304\\
6.3497166633606 14.7006044387817\\
6.44971656799316 14.7006072998047\\
6.54971647262573 14.7006101608276\\
6.6497163772583 14.7006092071533\\
6.74971628189087 14.700611114502\\
6.8497166633606 14.7006072998047\\
6.94971656799316 14.7006092071533\\
7.04971647262573 14.700608253479\\
7.1497163772583 14.700608253479\\
7.24971628189087 14.700608253479\\
7.3497166633606 14.700608253479\\
7.44971656799316 14.700608253479\\
7.54971647262573 14.700608253479\\
7.6497163772583 14.700608253479\\
7.74971628189087 14.700608253479\\
7.8497166633606 14.700608253479\\
7.94971656799316 14.700608253479\\
8.04971599578857 14.700608253479\\
8.1497163772583 14.700608253479\\
8.24971675872803 14.700608253479\\
8.34971618652344 14.700608253479\\
8.44971656799316 14.700608253479\\
8.54971599578857 14.700608253479\\
8.6497163772583 14.700608253479\\
8.74971675872803 14.700608253479\\
8.84971618652344 14.700608253479\\
8.94971656799316 14.700608253479\\
9.04971599578857 14.700608253479\\
9.1497163772583 14.700608253479\\
9.24971675872803 14.700608253479\\
9.34971618652344 14.700608253479\\
9.44971656799316 14.700608253479\\
9.54971599578857 14.700608253479\\
9.6497163772583 14.700608253479\\
9.74971675872803 14.700608253479\\
9.84971618652344 14.700608253479\\
9.94971656799316 14.700608253479\\
10.0497159957886 14.700608253479\\
10.1497163772583 14.700608253479\\
10.249716758728 14.700608253479\\
10.3497161865234 14.700608253479\\
10.4497165679932 14.700608253479\\
10.5497159957886 14.700608253479\\
10.6497163772583 14.700608253479\\
10.749716758728 14.700608253479\\
10.8497161865234 14.700608253479\\
10.9497165679932 14.700608253479\\
11.0497159957886 14.700608253479\\
11.1497163772583 14.700608253479\\
11.249716758728 14.700608253479\\
11.3497161865234 14.700608253479\\
11.4497165679932 14.700608253479\\
11.5497159957886 14.700608253479\\
11.6497163772583 14.700608253479\\
11.749716758728 14.700608253479\\
11.8497161865234 14.700608253479\\
11.9497165679932 14.700608253479\\
12.0497159957886 14.700608253479\\
12.1497163772583 14.700608253479\\
12.249716758728 14.700608253479\\
12.3471450805664 14.700608253479\\
12.4250917434692 14.700608253479\\
12.4879579544067 14.7006101608276\\
12.557110786438 14.7006072998047\\
12.6331787109375 14.7006092071533\\
12.7168531417847 14.700608253479\\
12.808895111084 14.700608253479\\
12.9088954925537 14.700608253479\\
13.0088949203491 14.700608253479\\
13.1088953018188 14.700608253479\\
13.2088947296143 14.700608253479\\
13.308895111084 14.700608253479\\
13.4088954925537 14.700608253479\\
13.5088949203491 14.700608253479\\
13.6088953018188 14.700608253479\\
13.7088947296143 14.700608253479\\
13.808895111084 14.700608253479\\
13.9088954925537 14.700608253479\\
14.0088949203491 14.700608253479\\
14.1088953018188 14.700608253479\\
14.2088947296143 14.700608253479\\
14.308895111084 14.700608253479\\
14.4088954925537 14.700608253479\\
14.5088949203491 14.700608253479\\
14.6088953018188 14.700608253479\\
14.7088947296143 14.700608253479\\
14.808895111084 14.700608253479\\
14.9088954925537 14.700608253479\\
15.0088949203491 14.700608253479\\
15.1088953018188 14.700608253479\\
15.2088947296143 14.700608253479\\
15.308895111084 14.700608253479\\
15.4088954925537 14.700608253479\\
15.5088949203491 14.700608253479\\
15.6088953018188 14.700608253479\\
15.7088947296143 14.700608253479\\
15.808895111084 14.700608253479\\
15.9088954925537 14.700608253479\\
16.0088958740234 14.700608253479\\
16.1088943481445 14.700608253479\\
16.2088947296143 14.700608253479\\
16.308895111084 14.700608253479\\
16.4088954925537 14.700608253479\\
16.5088958740234 14.700608253479\\
16.6088943481445 14.700608253479\\
16.7088947296143 14.700608253479\\
16.808895111084 14.700608253479\\
16.9088954925537 14.700608253479\\
17.0088958740234 14.700608253479\\
17.1088943481445 14.700608253479\\
17.2088947296143 14.700608253479\\
17.308895111084 14.700608253479\\
17.4088954925537 14.700608253479\\
17.5088958740234 14.700608253479\\
17.6088943481445 14.700608253479\\
17.7088947296143 14.700608253479\\
17.808895111084 14.700608253479\\
17.9088954925537 14.700608253479\\
18.0088958740234 14.700608253479\\
18.1088943481445 14.700608253479\\
18.2088947296143 14.700608253479\\
18.308895111084 14.7006072998047\\
18.4088954925537 14.700608253479\\
18.5088958740234 14.7006053924561\\
18.6088943481445 14.7006053924561\\
18.7088947296143 14.700608253479\\
18.808895111084 14.7006092071533\\
18.9088954925537 14.7006101608276\\
19.0088958740234 14.7006092071533\\
19.1088943481445 14.700608253479\\
19.2088947296143 14.700608253479\\
19.308895111084 14.700608253479\\
19.4088954925537 14.700608253479\\
19.5088958740234 14.700608253479\\
19.6088943481445 14.700608253479\\
19.7088947296143 14.700608253479\\
19.808895111084 14.700608253479\\
19.9088954925537 14.700608253479\\
20.0088958740234 14.700608253479\\
20.1088943481445 14.700608253479\\
20.2088947296143 14.700608253479\\
20.308895111084 14.700608253479\\
20.4088954925537 14.700608253479\\
20.5088958740234 14.700608253479\\
20.6088943481445 14.700608253479\\
20.7088947296143 14.700608253479\\
20.808895111084 14.700608253479\\
20.9088954925537 14.700608253479\\
21.0088958740234 14.700608253479\\
21.1088943481445 14.700608253479\\
21.2088947296143 14.700608253479\\
21.308895111084 14.700608253479\\
21.4088954925537 14.700608253479\\
21.5088958740234 14.700608253479\\
21.6088943481445 14.700608253479\\
21.7088947296143 14.700608253479\\
21.808895111084 14.700608253479\\
21.9088954925537 14.700608253479\\
22.0088958740234 14.700608253479\\
22.1088943481445 14.700608253479\\
22.2088947296143 14.700608253479\\
22.308895111084 14.700608253479\\
22.4088954925537 14.700608253479\\
22.5088958740234 14.700608253479\\
22.6088943481445 14.700608253479\\
22.7088947296143 14.700608253479\\
22.808895111084 14.700608253479\\
22.9088954925537 14.700608253479\\
23.0088958740234 14.700608253479\\
23.1088943481445 14.700608253479\\
23.2088947296143 14.700608253479\\
23.308895111084 14.700608253479\\
23.4088954925537 14.700608253479\\
23.5088958740234 14.700608253479\\
23.6088943481445 14.700608253479\\
23.7088947296143 14.700608253479\\
23.808895111084 14.700608253479\\
23.9088954925537 14.700608253479\\
24.0088958740234 14.700608253479\\
24.1088943481445 14.700608253479\\
24.2088947296143 14.700608253479\\
24.308895111084 14.700608253479\\
24.4088954925537 14.700608253479\\
24.5088958740234 14.700608253479\\
24.6048831939697 14.700608253479\\
24.681676864624 14.7006072998047\\
24.7464084625244 14.700608253479\\
24.8176136016846 14.700608253479\\
24.895938873291 14.700608253479\\
24.9820976257324 14.7006092071533\\
25.0768718719482 14.7006092071533\\
25.176872253418 14.7006092071533\\
25.2768707275391 14.7006092071533\\
25.3768711090088 14.7006092071533\\
25.4768714904785 14.7006092071533\\
25.5768718719482 14.7006092071533\\
25.676872253418 14.7006092071533\\
25.7768707275391 14.7006092071533\\
25.8768711090088 14.7006092071533\\
25.9768714904785 14.7006092071533\\
26.0768718719482 14.7006092071533\\
26.176872253418 14.7006092071533\\
26.2768707275391 14.7006092071533\\
26.3768711090088 14.7006092071533\\
26.4768714904785 14.7006092071533\\
26.5768718719482 14.7006092071533\\
26.676872253418 14.7006092071533\\
26.7768707275391 14.7006092071533\\
26.8768711090088 14.7006092071533\\
26.9768714904785 14.7006092071533\\
27.0768718719482 14.7006092071533\\
27.176872253418 14.7006092071533\\
27.2768707275391 14.7006092071533\\
27.3768711090088 14.7006092071533\\
27.4768714904785 14.7006092071533\\
27.5768718719482 14.7006092071533\\
27.676872253418 14.7006092071533\\
27.7768707275391 14.7006092071533\\
27.8768711090088 14.7006092071533\\
27.9768714904785 14.7006092071533\\
28.0768718719482 14.7006092071533\\
28.176872253418 14.7006092071533\\
28.2768707275391 14.7006092071533\\
28.3768711090088 14.7006092071533\\
28.4768714904785 14.7006092071533\\
28.5768718719482 14.7006092071533\\
28.676872253418 14.7006092071533\\
28.7768707275391 14.7006092071533\\
28.8768711090088 14.7006092071533\\
28.9768714904785 14.7006092071533\\
29.0768718719482 14.7006092071533\\
29.176872253418 14.7006092071533\\
29.2768707275391 14.7006092071533\\
29.3768711090088 14.7006092071533\\
29.4768714904785 14.7006092071533\\
29.5768718719482 14.7006092071533\\
29.676872253418 14.7006092071533\\
29.7768707275391 14.7006092071533\\
29.8768711090088 14.7006092071533\\
29.9768714904785 14.7006092071533\\
30.0768718719482 14.7006101608276\\
30.176872253418 14.7006130218506\\
30.2768707275391 14.700626373291\\
30.3768711090088 14.7006864547729\\
30.4768714904785 14.7010078430176\\
30.5768718719482 14.702657699585\\
30.676872253418 14.7090797424316\\
30.7768707275391 14.7222843170166\\
30.8768711090088 14.7311239242554\\
30.9768714904785 14.7330932617188\\
31.0768718719482 14.7321681976318\\
31.176872253418 14.732889175415\\
31.2768707275391 14.732741355896\\
31.3768711090088 14.7336473464966\\
31.4768714904785 14.7339267730713\\
31.5768718719482 14.7346258163452\\
31.676872253418 14.7350444793701\\
31.7768707275391 14.735650062561\\
31.8768711090088 14.7361316680908\\
31.9768714904785 14.7366962432861\\
32.0768699645996 14.7372055053711\\
32.176872253418 14.7377510070801\\
32.2768707275391 14.7382726669312\\
32.3768730163574 14.7388114929199\\
32.4768714904785 14.7393388748169\\
32.5768699645996 14.7398738861084\\
32.676872253418 14.7404041290283\\
32.7768707275391 14.7409372329712\\
32.8768730163574 14.7414693832397\\
32.9768714904785 14.7420024871826\\
33.0768699645996 14.7425346374512\\
33.176872253418 14.7430686950684\\
33.2768707275391 14.7436017990112\\
33.3768730163574 14.7443351745605\\
33.4768714904785 14.7450523376465\\
33.5768699645996 14.7456855773926\\
33.676872253418 14.7463312149048\\
33.7768707275391 14.7469692230225\\
33.8768730163574 14.7476081848145\\
33.9768714904785 14.7482461929321\\
34.0768699645996 14.7488822937012\\
34.176872253418 14.7495183944702\\
34.2768707275391 14.7501544952393\\
34.3768730163574 14.750789642334\\
34.4768714904785 14.7514247894287\\
34.5768699645996 14.7520589828491\\
34.676872253418 14.7526941299438\\
34.7768707275391 14.7533292770386\\
34.8768730163574 14.7539043426514\\
34.9768714904785 14.7542066574097\\
35.0768699645996 14.7548160552979\\
35.176872253418 14.7553253173828\\
35.2768707275391 14.7558679580688\\
35.3768730163574 14.7563934326172\\
35.4768714904785 14.7569208145142\\
35.5768699645996 14.7574415206909\\
35.676872253418 14.7579593658447\\
35.7768707275391 14.7584714889526\\
35.8768730163574 14.7589778900146\\
35.9768714904785 14.75954246521\\
36.0768699645996 14.7602033615112\\
36.176872253418 14.760835647583\\
36.2768707275391 14.7614469528198\\
36.3768730163574 14.7620754241943\\
36.4768714904785 14.7627305984497\\
36.5768699645996 14.7634296417236\\
36.676872253418 14.764199256897\\
36.7768707275391 14.7693824768066\\
36.8673248291016 14.774468421936\\
36.9396858215332 14.7748498916626\\
37.0192794799805 14.7683591842651\\
37.1068305969238 14.7687568664551\\
37.2031402587891 14.7692346572876\\
37.3031425476074 14.7700042724609\\
37.4031410217285 14.770959854126\\
37.5031433105469 14.7719964981079\\
37.603141784668 14.7730398178101\\
37.7031402587891 14.7741031646729\\
37.8031425476074 14.7751789093018\\
37.9031410217285 14.7762670516968\\
38.0031433105469 14.7773590087891\\
38.103141784668 14.7784605026245\\
38.2031402587891 14.7795677185059\\
38.3031425476074 14.7806787490845\\
38.4031410217285 14.781792640686\\
38.5031433105469 14.7829065322876\\
38.603141784668 14.7840204238892\\
38.7031402587891 14.7851333618164\\
38.8031425476074 14.786247253418\\
38.9031410217285 14.7873611450195\\
39.0031433105469 14.7884759902954\\
39.103141784668 14.789589881897\\
39.2031402587891 14.7907037734985\\
39.3031425476074 14.7918176651001\\
39.4031410217285 14.792932510376\\
39.5031433105469 14.7940473556519\\
39.603141784668 14.7951612472534\\
39.7031402587891 14.7962760925293\\
39.8031425476074 14.7973909378052\\
39.9031410217285 14.7985067367554\\
40.0031433105469 14.7996215820313\\
40.103141784668 14.8007373809814\\
40.2031402587891 14.8018522262573\\
40.3031425476074 14.8029680252075\\
40.4031410217285 14.8040838241577\\
40.5031433105469 14.8051996231079\\
40.603141784668 14.8063163757324\\
40.7031402587891 14.8074321746826\\
40.8031425476074 14.8085489273071\\
40.9031410217285 14.8096723556519\\
41.0031433105469 14.8107757568359\\
41.103141784668 14.8119029998779\\
41.2031402587891 14.8130235671997\\
41.3031425476074 14.8141460418701\\
41.4031410217285 14.8152685165405\\
41.5031433105469 14.8163900375366\\
41.603141784668 14.817512512207\\
41.7031402587891 14.8186349868774\\
41.8031425476074 14.8197574615479\\
41.9031410217285 14.8208799362183\\
42.0031433105469 14.8220024108887\\
42.103141784668 14.8231258392334\\
42.2031402587891 14.8242483139038\\
42.3031425476074 14.8253717422485\\
42.4031410217285 14.8264951705933\\
42.5031433105469 14.8276252746582\\
42.603141784668 14.8287801742554\\
42.7031402587891 14.8300809860229\\
42.8031425476074 14.8321323394775\\
42.9031410217285 14.8368263244629\\
43.0031433105469 14.8434324264526\\
43.103141784668 14.8446493148804\\
43.2031402587891 14.8418397903442\\
43.3031425476074 14.8384666442871\\
43.4031410217285 14.8390064239502\\
43.5031433105469 14.8392572402954\\
43.603141784668 14.8407173156738\\
43.7031402587891 14.8414878845215\\
43.8031425476074 14.8429746627808\\
43.9031410217285 14.8438577651978\\
44.0031433105469 14.845142364502\\
44.103141784668 14.8461599349976\\
44.2031402587891 14.8473558425903\\
44.3031425476074 14.8484315872192\\
44.4031410217285 14.849588394165\\
44.5031433105469 14.8506917953491\\
44.603141784668 14.8518304824829\\
44.7031402587891 14.8529453277588\\
44.8031425476074 14.854076385498\\
44.9031410217285 14.8551959991455\\
45.0031433105469 14.8563241958618\\
45.103141784668 14.8574466705322\\
45.2031402587891 14.8585720062256\\
45.3031425476074 14.8596954345703\\
45.4031410217285 14.8608207702637\\
45.5031433105469 14.8619451522827\\
45.603141784668 14.8632135391235\\
45.7031402587891 14.8645153045654\\
45.8031425476074 14.8655967712402\\
45.9031410217285 14.8667345046997\\
46.0031433105469 14.8678550720215\\
46.103141784668 14.8689804077148\\
46.2031402587891 14.8701047897339\\
46.3031425476074 14.8712301254272\\
46.4031410217285 14.8723545074463\\
46.5031433105469 14.8734788894653\\
46.603141784668 14.8746042251587\\
46.7031402587891 14.8757286071777\\
46.8031425476074 14.8768539428711\\
46.9031410217285 14.8779783248901\\
47.0031433105469 14.8791036605835\\
47.103141784668 14.880223274231\\
47.2031402587891 14.882155418396\\
47.3031425476074 14.882495880127\\
47.4031410217285 14.8833150863647\\
47.5031433105469 14.8845081329346\\
47.603141784668 14.885594367981\\
47.7031402587891 14.8867216110229\\
47.8031425476074 14.8878316879272\\
47.9031410217285 14.8889465332031\\
48.0031433105469 14.8900585174561\\
48.103141784668 14.8911695480347\\
48.2031402587891 14.8923187255859\\
48.3031425476074 14.8934936523438\\
48.4031410217285 14.8948049545288\\
48.5031433105469 14.8959341049194\\
48.603141784668 14.897068977356\\
48.7031402587891 14.8982496261597\\
48.8031425476074 14.8994474411011\\
48.9031410217285 14.9006929397583\\
49.0031433105469 14.9047355651855\\
49.098030090332 14.91015625\\
49.1739387512207 14.9143772125244\\
49.2346649169922 14.9036140441895\\
49.3014602661133 14.9232406616211\\
49.3450546264648 14.8785486221313\\
49.3930053710938 14.8940105438232\\
49.4215202331543 14.9170846939087\\
49.4451713562012 15.061466217041\\
49.47119140625 14.7788228988647\\
49.4998092651367 14.9030342102051\\
49.515323638916 14.9099149703979\\
49.5276222229004 15.2073469161987\\
49.5411529541016 14.6449298858643\\
49.5560340881348 14.8354415893555\\
49.5575561523438 15.4850368499756\\
49.5592269897461 15.0425844192505\\
49.5610656738281 14.9042062759399\\
49.5630912780762 14.928560256958\\
49.565315246582 14.9238977432251\\
49.5677642822266 14.9201231002808\\
49.5704574584961 14.9168939590454\\
49.5734214782715 14.9143028259277\\
49.5766792297363 14.9123020172119\\
49.5791168212891 14.911169052124\\
49.5817947387695 14.9102449417114\\
49.5847434997559 14.9095125198364\\
49.5879859924316 14.9089479446411\\
49.591552734375 14.9085292816162\\
49.5954742431641 14.9082307815552\\
49.5997886657715 14.9080295562744\\
49.6045341491699 14.9079055786133\\
49.6097564697266 14.9078388214111\\
49.6155014038086 14.9078149795532\\
49.6218185424805 14.9078235626221\\
49.6287651062012 14.9078559875488\\
49.6364097595215 14.9079084396362\\
49.6448211669922 14.907977104187\\
49.6540679931641 14.9080600738525\\
49.6642456054688 14.9081592559814\\
49.6754379272461 14.9083528518677\\
49.6877479553223 14.9086408615112\\
49.7012901306152 14.9088802337646\\
49.7161865234375 14.909065246582\\
49.7325744628906 14.9091968536377\\
49.7505989074707 14.9093198776245\\
49.7704238891602 14.9094800949097\\
49.792236328125 14.9098882675171\\
49.8162269592285 14.9102878570557\\
49.8426170349121 14.9106273651123\\
49.8716468811035 14.9109420776367\\
49.9035758972168 14.9112949371338\\
49.9387016296387 14.9116916656494\\
49.9773406982422 14.9121742248535\\
50.0198440551758 14.9127321243286\\
50.066593170166 14.9133682250977\\
50.1180229187012 14.9141063690186\\
50.1745910644531 14.914927482605\\
50.23681640625 14.9195871353149\\
50.3052673339844 14.9137649536133\\
50.3805618286133 14.9206027984619\\
50.4633865356445 14.9163112640381\\
50.5544929504395 14.9225578308105\\
50.6544914245605 14.9173336029053\\
50.7544898986816 14.9244241714478\\
50.8544921875 14.9190740585327\\
50.9544906616211 14.9265604019165\\
51.0544929504395 14.92103099823\\
51.1544914245605 14.9286394119263\\
51.2544898986816 14.9231719970703\\
51.3544921875 14.9254550933838\\
51.3568458557129 15.0419750213623\\
51.359432220459 14.9517049789429\\
51.3622817993164 14.9136972427368\\
51.3654136657715 14.9200401306152\\
51.3688583374023 14.9292583465576\\
51.3726463317871 14.9309577941895\\
51.3768157958984 14.930121421814\\
51.3814010620117 14.9297380447388\\
51.3864479064941 14.9297370910645\\
51.3919944763184 14.9297571182251\\
51.3980979919434 14.9297523498535\\
51.4048118591309 14.9297475814819\\
51.4121971130371 14.9297504425049\\
51.4203224182129 14.9297609329224\\
51.4292602539063 14.9297800064087\\
51.4390869140625 14.9298105239868\\
51.4499015808105 14.9298534393311\\
51.4617958068848 14.92991065979\\
51.4748802185059 14.9299831390381\\
51.4892692565918 14.9300737380981\\
51.5051002502441 14.9301843643188\\
51.522518157959 14.9303159713745\\
51.5416717529297 14.9304714202881\\
51.562744140625 14.9306516647339\\
51.5859222412109 14.9308586120605\\
51.6114196777344 14.931095123291\\
51.6394653320313 14.9313611984253\\
51.6703147888184 14.9316606521606\\
51.7042503356934 14.9319944381714\\
51.7415771484375 14.9323644638062\\
51.7826385498047 14.9327745437622\\
51.827808380127 14.9332275390625\\
51.8774948120117 14.9337272644043\\
51.9321479797363 14.9342775344849\\
51.9922637939453 14.9348840713501\\
52.0583953857422 14.9355506896973\\
52.1311378479004 14.9362850189209\\
52.2111587524414 14.9371223449707\\
52.2991790771484 14.9406538009644\\
52.3959999084473 14.9362344741821\\
52.4959983825684 14.940167427063\\
52.5960006713867 14.9473218917847\\
52.6160011291504 14.9414834976196\\
52.6228942871094 14.9590892791748\\
52.6304740905762 14.9635601043701\\
52.637622833252 14.9648132324219\\
52.6454811096191 14.9650630950928\\
52.6532897949219 14.9647989273071\\
52.6618766784668 14.9618434906006\\
52.6713256835938 14.958812713623\\
52.6817169189453 14.9584264755249\\
52.6931457519531 14.9565877914429\\
52.7057189941406 14.9521532058716\\
52.7195510864258 14.9530096054077\\
52.7347640991211 14.9620189666748\\
52.7514991760254 14.9757976531982\\
52.7699089050293 14.9650344848633\\
52.7901573181152 14.9517316818237\\
52.8124351501465 14.9441261291504\\
52.8369369506836 14.9360189437866\\
52.8638877868652 14.9278078079224\\
52.8935356140137 14.9224720001221\\
52.9261474609375 14.9195766448975\\
52.9620208740234 14.9198160171509\\
53.0014839172363 14.9238557815552\\
53.0448913574219 14.9279737472534\\
53.0926399230957 14.9255933761597\\
53.1261940002441 14.9319286346436\\
53.1631088256836 14.9485940933228\\
53.1971549987793 14.9491443634033\\
53.2346038818359 14.9516468048096\\
53.2757987976074 14.9534902572632\\
53.3211135864258 14.9544343948364\\
53.3709602355957 14.954794883728\\
53.4181785583496 14.9544429779053\\
53.4612884521484 14.9543485641479\\
53.4995498657227 14.9546136856079\\
53.5327682495117 14.9595766067505\\
53.5693130493164 14.9576807022095\\
53.6095085144043 14.955361366272\\
53.6537246704102 14.9536685943604\\
53.7023620605469 14.9540119171143\\
53.7558631896973 14.9544172286987\\
53.8147125244141 14.954779624939\\
53.8794479370117 14.9552946090698\\
53.9506568908691 14.9559516906738\\
54.0289878845215 14.9566812515259\\
54.1151542663574 14.9575061798096\\
54.2099342346191 14.9595813751221\\
54.3099327087402 14.958197593689\\
54.4099311828613 14.9619359970093\\
54.5099334716797 14.9602861404419\\
54.6099319458008 14.9620819091797\\
54.6148452758789 15.0073566436768\\
54.6202507019043 14.9479274749756\\
54.6261940002441 14.981164932251\\
54.6327362060547 14.949107170105\\
54.6399269104004 14.9794025421143\\
54.6478385925293 14.9514141082764\\
54.6565437316895 14.9772491455078\\
54.6661186218262 14.9535064697266\\
54.6766510009766 14.9626369476318\\
54.6824378967285 14.9690599441528\\
54.6888046264648 14.965428352356\\
54.6958045959473 14.9651517868042\\
54.7035064697266 14.9652481079102\\
54.7119827270508 14.9652967453003\\
54.7213020324707 14.9653482437134\\
54.7315559387207 14.9654102325439\\
54.7428321838379 14.9654846191406\\
54.7552375793457 14.965576171875\\
54.7688827514648 14.9656801223755\\
54.7838935852051 14.965802192688\\
54.800407409668 14.9659423828125\\
54.8185691833496 14.9661054611206\\
54.8385467529297 14.966290473938\\
54.8605270385742 14.9665021896362\\
54.8847007751465 14.9667377471924\\
54.9112930297852 14.9670143127441\\
54.940544128418 14.967324256897\\
54.9727210998535 14.9676828384399\\
55.008113861084 14.9681062698364\\
55.0470504760742 14.9686336517334\\
55.0898742675781 14.9693460464478\\
55.1369857788086 14.9704246520996\\
55.1888084411621 14.9722480773926\\
55.2458114624023 14.9754695892334\\
55.3085136413574 14.9800415039063\\
55.3774871826172 14.9839401245117\\
55.4533576965332 14.9811353683472\\
55.5368156433105 14.9788789749146\\
55.6286201477051 14.9779901504517\\
55.7286186218262 14.9774522781372\\
55.8286209106445 14.9769515991211\\
55.9286193847656 14.9781894683838\\
56.0286178588867 14.9791612625122\\
56.1286201477051 14.9805603027344\\
56.2286186218262 14.9815053939819\\
56.3286209106445 14.9827489852905\\
56.4286193847656 14.9837961196899\\
56.5286178588867 14.9849729537964\\
56.6286201477051 14.9860639572144\\
56.7286186218262 14.9872121810913\\
56.8286209106445 14.9883556365967\\
56.9286193847656 14.9894695281982\\
57.0286178588867 14.9905700683594\\
57.1286201477051 14.9917058944702\\
57.2286186218262 14.9928226470947\\
57.3286209106445 14.9939546585083\\
57.4286193847656 14.9950761795044\\
57.5286178588867 14.9962043762207\\
57.6286201477051 14.9973278045654\\
57.7286186218262 14.9984560012817\\
57.8286209106445 14.9995803833008\\
57.9286193847656 15.0007076263428\\
58.0286178588867 15.0021514892578\\
58.1286201477051 15.003228187561\\
58.2286186218262 15.004376411438\\
58.3286209106445 15.0054931640625\\
58.4286193847656 15.0066213607788\\
58.5286178588867 15.0077466964722\\
58.6286201477051 15.0088729858398\\
58.7286186218262 15.0099992752075\\
58.8286209106445 15.0111255645752\\
58.9286193847656 15.0122509002686\\
59.0286178588867 15.0133771896362\\
59.1286201477051 15.0145063400269\\
59.2286186218262 15.0156326293945\\
59.3286209106445 15.0167579650879\\
59.4286193847656 15.0178842544556\\
59.5286178588867 15.018835067749\\
59.6286201477051 15.0198135375977\\
59.7286186218262 15.0209741592407\\
59.8286209106445 15.0220680236816\\
59.9143104553223 15.0244665145874\\
60 15.0255966186523\\
60.0942611694336 15.0254554748535\\
60.1942596435547 15.0262994766235\\
60.2942581176758 15.0274486541748\\
60.3942604064941 15.0285654067993\\
60.4942588806152 15.0296945571899\\
60.5942611694336 15.0308237075806\\
60.6942596435547 15.0319566726685\\
60.7942581176758 15.0331039428711\\
60.8942604064941 15.0342483520508\\
60.9942588806152 15.0354022979736\\
61.0942611694336 15.0365762710571\\
61.1942596435547 15.0377750396729\\
61.2942581176758 15.0390319824219\\
61.3942604064941 15.0448741912842\\
61.4835205078125 15.0491018295288\\
61.5549278259277 15.048394203186\\
61.6334800720215 15.0429744720459\\
61.7198829650879 15.0436010360718\\
61.8149299621582 15.0599889755249\\
61.8549957275391 15.0084161758423\\
61.8990707397461 15.0401029586792\\
61.9192581176758 15.0521001815796\\
61.9358787536621 15.0553321838379\\
61.9541664123535 15.0469341278076\\
61.9742813110352 15.0469465255737\\
61.9964065551758 15.0471057891846\\
62.0207443237305 15.0472755432129\\
62.0475158691406 15.0474824905396\\
62.0769653320313 15.0477294921875\\
62.1093559265137 15.0480175018311\\
62.1449890136719 15.0483512878418\\
62.1841850280762 15.0487308502197\\
62.2273025512695 15.0491590499878\\
62.274730682373 15.0496377944946\\
62.3269004821777 15.0501699447632\\
62.3842887878418 15.0507583618164\\
62.4474143981934 15.0514059066772\\
62.5168533325195 15.0521192550659\\
62.5932350158691 15.0529022216797\\
62.6772575378418 15.0629568099976\\
62.7696800231934 15.0460777282715\\
62.8696823120117 15.0558528900146\\
62.9696807861328 15.0600414276123\\
63.0696792602539 15.0644226074219\\
63.1196823120117 15.1139688491821\\
63.1572036743164 15.0729856491089\\
63.1601943969727 15.0540952682495\\
63.1634902954102 15.0625991821289\\
63.1671104431152 15.0801277160645\\
63.1710968017578 15.0820846557617\\
63.175479888916 15.0693483352661\\
63.1803016662598 15.0594511032104\\
63.185604095459 15.0479784011841\\
63.1914367675781 15.0369110107422\\
63.1978569030762 15.0300693511963\\
63.2049140930176 15.0252704620361\\
63.2126808166504 15.0203685760498\\
63.2212219238281 15.0140743255615\\
63.2306175231934 15.0051259994507\\
63.2409515380859 14.9948577880859\\
63.2523231506348 14.9865608215332\\
63.2648277282715 14.9826936721802\\
63.2785835266113 14.9838495254517\\
63.2937164306641 14.9899387359619\\
63.310359954834 15.0020589828491\\
63.328670501709 15.0194702148438\\
63.3488121032715 15.0447902679443\\
63.3709678649902 15.0373935699463\\
63.395336151123 15.0418729782104\\
63.4221420288086 15.0544061660767\\
63.4516296386719 15.0522260665894\\
63.4840698242188 15.057728767395\\
63.4999580383301 15.0662212371826\\
63.5174407958984 15.0711097717285\\
63.5366668701172 15.0753154754639\\
63.5578193664551 15.0818176269531\\
63.5810852050781 15.0863513946533\\
63.6066780090332 15.0832014083862\\
63.6348304748535 15.076512336731\\
63.6657981872559 15.0730810165405\\
63.6998634338379 15.0720090866089\\
63.7373313903809 15.0715951919556\\
63.7785491943359 15.0711154937744\\
63.8238906860352 15.0709466934204\\
63.8737640380859 15.0714225769043\\
63.9286231994629 15.0724477767944\\
63.9762001037598 15.0725593566895\\
64.0218276977539 15.0725517272949\\
64.0647964477539 15.0737295150757\\
64.112060546875 15.0753841400146\\
64.1640548706055 15.0740423202515\\
64.2212448120117 15.0714302062988\\
64.2841567993164 15.0717325210571\\
64.3533554077148 15.0722875595093\\
64.4294738769531 15.0727987289429\\
64.5132064819336 15.0735273361206\\
64.6053161621094 15.0764093399048\\
64.7053146362305 15.0732984542847\\
64.8053131103516 15.0787038803101\\
64.9053115844727 15.0752334594727\\
65.0053176879883 15.0776205062866\\
65.0079345703125 15.1878614425659\\
65.0108184814453 15.0490379333496\\
65.0139923095703 15.0407304763794\\
65.0165023803711 15.0884857177734\\
65.0192642211914 15.1023979187012\\
65.0223007202148 15.0840692520142\\
65.025634765625 15.0796194076538\\
65.0293121337891 15.0805549621582\\
65.0333557128906 15.0809278488159\\
65.0377960205078 15.0809059143066\\
65.0426864624023 15.0808839797974\\
65.0480651855469 15.080883026123\\
65.0539855957031 15.0808877944946\\
65.0604934692383 15.0808973312378\\
65.0676498413086 15.0809125900269\\
65.0755233764648 15.0809345245361\\
65.0841903686523 15.0809659957886\\
65.0937194824219 15.0810070037842\\
65.1042022705078 15.0810604095459\\
65.1157302856445 15.0811262130737\\
65.1284103393555 15.0812072753906\\
65.1423645019531 15.0813055038452\\
65.1577072143555 15.0814208984375\\
65.1745910644531 15.0815572738647\\
65.1931610107422 15.0817165374756\\
65.2135848999023 15.08189868927\\
65.2360534667969 15.082106590271\\
65.2607727050781 15.0823421478271\\
65.287956237793 15.0826072692871\\
65.3178634643555 15.0829038619995\\
65.3507614135742 15.083233833313\\
65.3869476318359 15.0836000442505\\
65.4267501831055 15.084005355835\\
65.4705352783203 15.0844526290894\\
65.5186996459961 15.0849447250366\\
65.5716781616211 15.0854873657227\\
65.6299591064453 15.0860843658447\\
65.6940612792969 15.0867404937744\\
65.7645797729492 15.0874633789063\\
65.8421478271484 15.0882577896118\\
65.9274749755859 15.08913230896\\
66.0213317871094 15.0918035507202\\
66.1213302612305 15.0892744064331\\
66.2213287353516 15.0923871994019\\
66.3213272094727 15.1012849807739\\
66.3413314819336 15.1091337203979\\
66.3490600585938 15.1365728378296\\
66.3565673828125 15.1405839920044\\
66.3648223876953 15.137095451355\\
66.3727111816406 15.1336183547974\\
66.3813858032227 15.1393384933472\\
66.3909301757813 15.1687622070313\\
66.4014358520508 15.1994705200195\\
66.4129867553711 15.202600479126\\
66.4256896972656 15.1802177429199\\
66.4396667480469 15.1597805023193\\
66.4550399780273 15.1371421813965\\
66.4719543457031 15.1133823394775\\
66.4905548095703 15.0890893936157\\
66.5110244750977 15.0585670471191\\
66.5335311889648 15.0232000350952\\
66.5504302978516 15.0082426071167\\
66.5690307617188 14.9990978240967\\
66.5894775390625 14.9962549209595\\
66.6119766235352 15.0160865783691\\
66.6367263793945 15.0429954528809\\
66.663948059082 15.0967350006104\\
66.6938934326172 15.1035394668579\\
66.7268295288086 15.106240272522\\
66.7630615234375 15.1075849533081\\
66.8029174804688 15.1066875457764\\
66.8467636108398 15.1048774719238\\
66.8949890136719 15.1035404205322\\
66.9480361938477 15.1168851852417\\
67.0063934326172 15.1000776290894\\
67.0705795288086 15.1119947433472\\
67.111457824707 15.0953569412231\\
67.1564254760742 15.0989484786987\\
67.1973037719727 15.0988159179688\\
67.2330703735352 15.0997552871704\\
67.2724151611328 15.1013288497925\\
67.3156967163086 15.1034622192383\\
67.3633117675781 15.1049871444702\\
67.4156799316406 15.106538772583\\
67.4732894897461 15.1088838577271\\
67.5366516113281 15.1114721298218\\
67.6063613891602 15.1146068572998\\
67.6830368041992 15.1174421310425\\
67.7673797607422 15.1151504516602\\
67.8601531982422 15.1151466369629\\
67.9601516723633 15.1146669387817\\
68.0601577758789 15.1143951416016\\
68.16015625 15.1140813827515\\
68.2601547241211 15.1148500442505\\
68.3601531982422 15.1159973144531\\
68.4601516723633 15.1172647476196\\
68.5601577758789 15.1182947158813\\
68.66015625 15.1194820404053\\
68.7601547241211 15.1205673217773\\
68.8601531982422 15.1217203140259\\
68.9601516723633 15.1228294372559\\
69.0601577758789 15.1239681243896\\
69.16015625 15.1250877380371\\
69.2601547241211 15.1262197494507\\
69.3601531982422 15.1273441314697\\
69.4601516723633 15.1284732818604\\
69.5601577758789 15.1295986175537\\
69.66015625 15.13072681427\\
69.7601547241211 15.131854057312\\
69.8601531982422 15.132981300354\\
69.9601516723633 15.134108543396\\
70.0601577758789 15.135235786438\\
70.16015625 15.13636302948\\
70.2601547241211 15.1375026702881\\
70.3601531982422 15.1386709213257\\
70.4601516723633 15.1400833129883\\
70.5601577758789 15.141134262085\\
70.66015625 15.1422843933105\\
70.7601547241211 15.1434059143066\\
70.8601531982422 15.1445350646973\\
70.9601516723633 15.1456623077393\\
71.0601577758789 15.1467905044556\\
71.16015625 15.1479187011719\\
71.2601547241211 15.1490459442139\\
71.3601531982422 15.1501741409302\\
71.4601516723633 15.1513013839722\\
71.5601577758789 15.1524295806885\\
71.66015625 15.1535568237305\\
71.7601547241211 15.1546850204468\\
71.8601531982422 15.1558113098145\\
71.9601516723633 15.1567287445068\\
72.0601577758789 15.1577548980713\\
72.16015625 15.1588983535767\\
72.2601547241211 15.1600008010864\\
72.3601531982422 15.1611156463623\\
72.4601516723633 15.1622228622437\\
72.5601577758789 15.163330078125\\
72.66015625 15.1644334793091\\
72.7601547241211 15.1655330657959\\
72.8601531982422 15.1666297912598\\
72.9601516723633 15.1677207946777\\
73.0601577758789 15.1688070297241\\
73.16015625 15.1733665466309\\
73.2601547241211 15.1722860336304\\
73.3601531982422 15.1724824905396\\
73.4601516723633 15.1738319396973\\
73.5601577758789 15.1749401092529\\
73.66015625 15.1761913299561\\
73.7601547241211 15.1776790618896\\
73.8601531982422 15.1852607727051\\
73.943000793457 15.1886539459229\\
74.00927734375 15.183310508728\\
74.0821685791016 15.1811046600342\\
74.162353515625 15.1816740036011\\
74.2505569458008 15.1990365982056\\
74.2926559448242 15.1486577987671\\
74.3389587402344 15.1776723861694\\
74.3595275878906 15.1912336349487\\
74.3765106201172 15.193751335144\\
74.3951873779297 15.1850109100342\\
74.4157409667969 15.1850090026855\\
74.438346862793 15.1851768493652\\
74.4632110595703 15.1853504180908\\
74.4905624389648 15.1855640411377\\
74.5206527709961 15.1858177185059\\
74.5537414550781 15.1861152648926\\
74.5901489257813 15.1864576339722\\
74.6301956176758 15.1868476867676\\
74.6742477416992 15.1872863769531\\
74.7227020263672 15.1877775192261\\
74.7760009765625 15.1883220672607\\
74.8346328735352 15.1889238357544\\
74.8991317749023 15.1895866394043\\
74.9495620727539 15.1903171539307\\
75 15.1908855438232\\
};
\addlegendentry{10 "};

\addplot [
color=black,
dotted
]
table[row sep=crcr]{
0 14.7005214691162\\
9.99999971718069e-10 14.7005214691162\\
2.49999998480632e-09 14.7005214691162\\
4.74999994892755e-09 14.7005214691162\\
8.1250002281763e-09 14.7005214691162\\
1.31874999809156e-08 14.7005214691162\\
2.0781250498203e-08 14.7005214691162\\
3.21718758300449e-08 14.7005214691162\\
4.92578138278077e-08 14.7005214691162\\
7.48867208244519e-08 14.7005214691162\\
1.13330081319418e-07 14.7005214691162\\
1.70995122061868e-07 14.7005214691162\\
2.57492672517401e-07 14.7005214691162\\
3.8723902662241e-07 14.7005214691162\\
5.81858500936505e-07 14.7005214691162\\
8.73787769251066e-07 14.7005214691162\\
1.31168167172291e-06 14.7005214691162\\
1.9685226106958e-06 14.7005214691162\\
2.95378367809462e-06 14.7005214691162\\
4.43167573394021e-06 14.7005214691162\\
6.64851359033491e-06 14.7005214691162\\
9.97376992017962e-06 14.7005214691162\\
1.49616553244414e-05 14.7005224227905\\
2.24434825213393e-05 14.7005224227905\\
3.36662233166862e-05 14.7005233764648\\
5.05003372381907e-05 14.7005233764648\\
7.57515081204474e-05 14.7005243301392\\
0.000113628258986864 14.7005252838135\\
0.000170443381648511 14.7005262374878\\
0.000255666091106832 14.7005271911621\\
0.000383500126190484 14.7005281448364\\
0.000575251178815961 14.7005290985107\\
0.000862877757754177 14.7005290985107\\
0.00129431765526533 14.7005281448364\\
0.00194147753063589 14.7005243301392\\
0.00291221728548408 14.7005195617676\\
0.00436832662671804 14.7005176544189\\
0.00597004732117057 14.7005195617676\\
0.00773194013163447 14.7005214691162\\
0.009670021943748 14.7005224227905\\
0.0118019115179777 14.7005224227905\\
0.0141469910740852 14.7005214691162\\
0.0167265776544809 14.7005214691162\\
0.0195641238242388 14.7005214691162\\
0.0226854234933853 14.7005214691162\\
0.0261188540607691 14.7005214691162\\
0.0298956278711557 14.7005214691162\\
0.0340500771999359 14.7005214691162\\
0.0386199727654457 14.7005214691162\\
0.0436468608677387 14.7005214691162\\
0.0491764321923256 14.7005214691162\\
0.0552589632570744 14.7005214691162\\
0.0619497485458851 14.7005214691162\\
0.0693096145987511 14.7005214691162\\
0.0774054601788521 14.7005176544189\\
0.0863108932971954 14.7005243301392\\
0.0961068719625473 14.7005214691162\\
0.106882445514202 14.7005214691162\\
0.118735581636429 14.7005214691162\\
0.13177402317524 14.7005214691162\\
0.146116316318512 14.7005214691162\\
0.161892831325531 14.7005214691162\\
0.179247006773949 14.7005214691162\\
0.198336601257324 14.7005214691162\\
0.21933513879776 14.7005214691162\\
0.242433547973633 14.7005214691162\\
0.267841786146164 14.7005214691162\\
0.29579085111618 14.7005214691162\\
0.32653483748436 14.7005214691162\\
0.360353201627731 14.7005214691162\\
0.397553414106369 14.7005214691162\\
0.438473641872406 14.7005214691162\\
0.483485877513886 14.7005214691162\\
0.532999336719513 14.7005214691162\\
0.587464153766632 14.7005214691162\\
0.647375464439392 14.7005214691162\\
0.71327793598175 14.7005214691162\\
0.7857705950737 14.7005214691162\\
0.865512549877167 14.7005214691162\\
0.953228712081909 14.7005214691162\\
1.04971647262573 14.7005214691162\\
1.14971649646759 14.7005214691162\\
1.24971640110016 14.7005214691162\\
1.34971642494202 14.7005214691162\\
1.44971644878387 14.7005214691162\\
1.54971647262573 14.7005214691162\\
1.64971649646759 14.7005214691162\\
1.74971640110016 14.7005214691162\\
1.84971642494202 14.7005214691162\\
1.94971644878387 14.7005214691162\\
2.04971647262573 14.7005214691162\\
2.1497163772583 14.7005214691162\\
2.24971652030945 14.7005214691162\\
2.34971642494202 14.7005214691162\\
2.44971656799316 14.7005214691162\\
2.54971647262573 14.7005214691162\\
2.6497163772583 14.7005214691162\\
2.74971652030945 14.7005214691162\\
2.84971642494202 14.7005214691162\\
2.94971656799316 14.7005214691162\\
3.04971647262573 14.7005214691162\\
3.1497163772583 14.7005214691162\\
3.24971652030945 14.7005214691162\\
3.34971642494202 14.7005214691162\\
3.44971656799316 14.7005214691162\\
3.54971647262573 14.7005214691162\\
3.6497163772583 14.7005214691162\\
3.74971652030945 14.7005214691162\\
3.84971642494202 14.7005214691162\\
3.94971656799316 14.7005214691162\\
4.04971647262573 14.7005214691162\\
4.1497163772583 14.7005214691162\\
4.24971628189087 14.7005214691162\\
4.3497166633606 14.7005214691162\\
4.44971656799316 14.7005214691162\\
4.54971647262573 14.7005214691162\\
4.6497163772583 14.7005214691162\\
4.74971628189087 14.7005214691162\\
4.8497166633606 14.7005214691162\\
4.94971656799316 14.7005214691162\\
5.04971647262573 14.7005214691162\\
5.1497163772583 14.7005214691162\\
5.24971628189087 14.7005214691162\\
5.3497166633606 14.7005214691162\\
5.44971656799316 14.7005214691162\\
5.54971647262573 14.7005214691162\\
5.6497163772583 14.7005214691162\\
5.74971628189087 14.7005214691162\\
5.8497166633606 14.7005214691162\\
5.94971656799316 14.7005214691162\\
6.04971647262573 14.7005214691162\\
6.1497163772583 14.7005205154419\\
6.24971628189087 14.7005195617676\\
6.3497166633606 14.7005186080933\\
6.44971656799316 14.7005205154419\\
6.54971647262573 14.7005224227905\\
6.6497163772583 14.7005224227905\\
6.74971628189087 14.7005233764648\\
6.8497166633606 14.7005205154419\\
6.94971656799316 14.7005214691162\\
7.04971647262573 14.7005214691162\\
7.1497163772583 14.7005214691162\\
7.24971628189087 14.7005214691162\\
7.3497166633606 14.7005214691162\\
7.44971656799316 14.7005214691162\\
7.54971647262573 14.7005214691162\\
7.6497163772583 14.7005214691162\\
7.74971628189087 14.7005214691162\\
7.8497166633606 14.7005214691162\\
7.94971656799316 14.7005214691162\\
8.04971599578857 14.7005214691162\\
8.1497163772583 14.7005214691162\\
8.24971675872803 14.7005214691162\\
8.34971618652344 14.7005214691162\\
8.44971656799316 14.7005214691162\\
8.54971599578857 14.7005214691162\\
8.6497163772583 14.7005214691162\\
8.74971675872803 14.7005214691162\\
8.84971618652344 14.7005214691162\\
8.94971656799316 14.7005214691162\\
9.04971599578857 14.7005214691162\\
9.1497163772583 14.7005214691162\\
9.24971675872803 14.7005214691162\\
9.34971618652344 14.7005214691162\\
9.44971656799316 14.7005214691162\\
9.54971599578857 14.7005214691162\\
9.6497163772583 14.7005214691162\\
9.74971675872803 14.7005214691162\\
9.84971618652344 14.7005214691162\\
9.94971656799316 14.7005214691162\\
10.0497159957886 14.7005214691162\\
10.1497163772583 14.7005214691162\\
10.249716758728 14.7005214691162\\
10.3497161865234 14.7005214691162\\
10.4497165679932 14.7005214691162\\
10.5497159957886 14.7005214691162\\
10.6497163772583 14.7005214691162\\
10.749716758728 14.7005214691162\\
10.8497161865234 14.7005214691162\\
10.9497165679932 14.7005214691162\\
11.0497159957886 14.7005214691162\\
11.1497163772583 14.7005214691162\\
11.249716758728 14.7005214691162\\
11.3497161865234 14.7005214691162\\
11.4497165679932 14.7005214691162\\
11.5497159957886 14.7005214691162\\
11.6497163772583 14.7005214691162\\
11.749716758728 14.7005214691162\\
11.8497161865234 14.7005214691162\\
11.9497165679932 14.7005214691162\\
12.0497159957886 14.7005214691162\\
12.1497163772583 14.7005214691162\\
12.249716758728 14.7005214691162\\
12.3471450805664 14.7005214691162\\
12.4250917434692 14.7005214691162\\
12.4879579544067 14.7005224227905\\
12.557110786438 14.7005205154419\\
12.6331787109375 14.7005224227905\\
12.7168531417847 14.7005214691162\\
12.808895111084 14.7005214691162\\
12.9088954925537 14.7005214691162\\
13.0088949203491 14.7005214691162\\
13.1088953018188 14.7005214691162\\
13.2088947296143 14.7005214691162\\
13.308895111084 14.7005214691162\\
13.4088954925537 14.7005214691162\\
13.5088949203491 14.7005214691162\\
13.6088953018188 14.7005214691162\\
13.7088947296143 14.7005214691162\\
13.808895111084 14.7005214691162\\
13.9088954925537 14.7005214691162\\
14.0088949203491 14.7005214691162\\
14.1088953018188 14.7005214691162\\
14.2088947296143 14.7005214691162\\
14.308895111084 14.7005214691162\\
14.4088954925537 14.7005214691162\\
14.5088949203491 14.7005214691162\\
14.6088953018188 14.7005214691162\\
14.7088947296143 14.7005214691162\\
14.808895111084 14.7005214691162\\
14.9088954925537 14.7005214691162\\
15.0088949203491 14.7005214691162\\
15.1088953018188 14.7005214691162\\
15.2088947296143 14.7005214691162\\
15.308895111084 14.7005214691162\\
15.4088954925537 14.7005214691162\\
15.5088949203491 14.7005214691162\\
15.6088953018188 14.7005214691162\\
15.7088947296143 14.7005214691162\\
15.808895111084 14.7005214691162\\
15.9088954925537 14.7005214691162\\
16.0088958740234 14.7005214691162\\
16.1088943481445 14.7005214691162\\
16.2088947296143 14.7005214691162\\
16.308895111084 14.7005214691162\\
16.4088954925537 14.7005214691162\\
16.5088958740234 14.7005214691162\\
16.6088943481445 14.7005214691162\\
16.7088947296143 14.7005214691162\\
16.808895111084 14.7005214691162\\
16.9088954925537 14.7005214691162\\
17.0088958740234 14.7005214691162\\
17.1088943481445 14.7005214691162\\
17.2088947296143 14.7005214691162\\
17.308895111084 14.7005214691162\\
17.4088954925537 14.7005214691162\\
17.5088958740234 14.7005214691162\\
17.6088943481445 14.7005214691162\\
17.7088947296143 14.7005214691162\\
17.808895111084 14.7005214691162\\
17.9088954925537 14.7005214691162\\
18.0088958740234 14.7005214691162\\
18.1088943481445 14.7005214691162\\
18.2088947296143 14.7005214691162\\
18.308895111084 14.7005205154419\\
18.4088954925537 14.7005214691162\\
18.5088958740234 14.7005195617676\\
18.6088943481445 14.7005186080933\\
18.7088947296143 14.7005214691162\\
18.808895111084 14.7005224227905\\
18.9088954925537 14.7005233764648\\
19.0088958740234 14.7005224227905\\
19.1088943481445 14.7005214691162\\
19.2088947296143 14.7005214691162\\
19.308895111084 14.7005214691162\\
19.4088954925537 14.7005214691162\\
19.5088958740234 14.7005214691162\\
19.6088943481445 14.7005214691162\\
19.7088947296143 14.7005214691162\\
19.808895111084 14.7005214691162\\
19.9088954925537 14.7005214691162\\
20.0088958740234 14.7005214691162\\
20.1088943481445 14.7005214691162\\
20.2088947296143 14.7005214691162\\
20.308895111084 14.7005214691162\\
20.4088954925537 14.7005214691162\\
20.5088958740234 14.7005214691162\\
20.6088943481445 14.7005214691162\\
20.7088947296143 14.7005214691162\\
20.808895111084 14.7005214691162\\
20.9088954925537 14.7005214691162\\
21.0088958740234 14.7005214691162\\
21.1088943481445 14.7005214691162\\
21.2088947296143 14.7005214691162\\
21.308895111084 14.7005214691162\\
21.4088954925537 14.7005214691162\\
21.5088958740234 14.7005214691162\\
21.6088943481445 14.7005214691162\\
21.7088947296143 14.7005214691162\\
21.808895111084 14.7005214691162\\
21.9088954925537 14.7005214691162\\
22.0088958740234 14.7005214691162\\
22.1088943481445 14.7005214691162\\
22.2088947296143 14.7005214691162\\
22.308895111084 14.7005214691162\\
22.4088954925537 14.7005214691162\\
22.5088958740234 14.7005214691162\\
22.6088943481445 14.7005214691162\\
22.7088947296143 14.7005214691162\\
22.808895111084 14.7005214691162\\
22.9088954925537 14.7005214691162\\
23.0088958740234 14.7005214691162\\
23.1088943481445 14.7005214691162\\
23.2088947296143 14.7005214691162\\
23.308895111084 14.7005214691162\\
23.4088954925537 14.7005214691162\\
23.5088958740234 14.7005214691162\\
23.6088943481445 14.7005214691162\\
23.7088947296143 14.7005214691162\\
23.808895111084 14.7005214691162\\
23.9088954925537 14.7005214691162\\
24.0088958740234 14.7005214691162\\
24.1088943481445 14.7005214691162\\
24.2088947296143 14.7005214691162\\
24.308895111084 14.7005214691162\\
24.4088954925537 14.7005214691162\\
24.5088958740234 14.7005214691162\\
24.6048831939697 14.7005214691162\\
24.681676864624 14.7005214691162\\
24.7464084625244 14.7005224227905\\
24.8176136016846 14.7005214691162\\
24.895938873291 14.7005214691162\\
24.9820976257324 14.7005214691162\\
25.0768718719482 14.7005214691162\\
25.176872253418 14.7005214691162\\
25.2768707275391 14.7005214691162\\
25.3768711090088 14.7005214691162\\
25.4768714904785 14.7005214691162\\
25.5768718719482 14.7005214691162\\
25.676872253418 14.7005214691162\\
25.7768707275391 14.7005214691162\\
25.8768711090088 14.7005214691162\\
25.9768714904785 14.7005214691162\\
26.0768718719482 14.7005214691162\\
26.176872253418 14.7005214691162\\
26.2768707275391 14.7005214691162\\
26.3768711090088 14.7005214691162\\
26.4768714904785 14.7005214691162\\
26.5768718719482 14.7005214691162\\
26.676872253418 14.7005214691162\\
26.7768707275391 14.7005214691162\\
26.8768711090088 14.7005214691162\\
26.9768714904785 14.7005214691162\\
27.0768718719482 14.7005214691162\\
27.176872253418 14.7005214691162\\
27.2768707275391 14.7005214691162\\
27.3768711090088 14.7005214691162\\
27.4768714904785 14.7005214691162\\
27.5768718719482 14.7005214691162\\
27.676872253418 14.7005214691162\\
27.7768707275391 14.7005214691162\\
27.8768711090088 14.7005214691162\\
27.9768714904785 14.7005214691162\\
28.0768718719482 14.7005214691162\\
28.176872253418 14.7005214691162\\
28.2768707275391 14.7005214691162\\
28.3768711090088 14.7005214691162\\
28.4768714904785 14.7005214691162\\
28.5768718719482 14.7005214691162\\
28.676872253418 14.7005214691162\\
28.7768707275391 14.7005214691162\\
28.8768711090088 14.7005214691162\\
28.9768714904785 14.7005214691162\\
29.0768718719482 14.7005214691162\\
29.176872253418 14.7005214691162\\
29.2768707275391 14.7005214691162\\
29.3768711090088 14.7005214691162\\
29.4768714904785 14.7005214691162\\
29.5768718719482 14.7005214691162\\
29.676872253418 14.7005214691162\\
29.7768707275391 14.7005214691162\\
29.8768711090088 14.7005214691162\\
29.9768714904785 14.7005214691162\\
30.0768718719482 14.7005214691162\\
30.176872253418 14.7005214691162\\
30.2768707275391 14.7005214691162\\
30.3768711090088 14.7005214691162\\
30.4768714904785 14.7005214691162\\
30.5768718719482 14.7005195617676\\
30.676872253418 14.7005214691162\\
30.7768707275391 14.7005186080933\\
30.8768711090088 14.7005205154419\\
30.9768714904785 14.7005214691162\\
31.0768718719482 14.7005233764648\\
31.176872253418 14.7005224227905\\
31.2768707275391 14.7005214691162\\
31.3768711090088 14.7005214691162\\
31.4768714904785 14.7005214691162\\
31.5768718719482 14.7005214691162\\
31.676872253418 14.7005214691162\\
31.7768707275391 14.7005214691162\\
31.8768711090088 14.7005214691162\\
31.9768714904785 14.7005214691162\\
32.0768699645996 14.7005214691162\\
32.176872253418 14.7005214691162\\
32.2768707275391 14.7005214691162\\
32.3768730163574 14.7005214691162\\
32.4768714904785 14.7005214691162\\
32.5768699645996 14.7005214691162\\
32.676872253418 14.7005214691162\\
32.7768707275391 14.7005214691162\\
32.8768730163574 14.7005214691162\\
32.9768714904785 14.7005214691162\\
33.0768699645996 14.7005214691162\\
33.176872253418 14.7005214691162\\
33.2768707275391 14.7005214691162\\
33.3768730163574 14.7005214691162\\
33.4768714904785 14.7005214691162\\
33.5768699645996 14.7005214691162\\
33.676872253418 14.7005214691162\\
33.7768707275391 14.7005214691162\\
33.8768730163574 14.7005214691162\\
33.9768714904785 14.7005214691162\\
34.0768699645996 14.7005214691162\\
34.176872253418 14.7005214691162\\
34.2768707275391 14.7005214691162\\
34.3768730163574 14.7005214691162\\
34.4768714904785 14.7005214691162\\
34.5768699645996 14.7005214691162\\
34.676872253418 14.7005214691162\\
34.7768707275391 14.7005214691162\\
34.8768730163574 14.7005214691162\\
34.9768714904785 14.7005214691162\\
35.0768699645996 14.7005214691162\\
35.176872253418 14.7005214691162\\
35.2768707275391 14.7005214691162\\
35.3768730163574 14.7005214691162\\
35.4768714904785 14.7005214691162\\
35.5768699645996 14.7005214691162\\
35.676872253418 14.7005214691162\\
35.7768707275391 14.7005214691162\\
35.8768730163574 14.7005214691162\\
35.9768714904785 14.7005214691162\\
36.0768699645996 14.7005214691162\\
36.176872253418 14.7005214691162\\
36.2768707275391 14.7005214691162\\
36.3768730163574 14.7005214691162\\
36.4768714904785 14.7005214691162\\
36.5768699645996 14.7005214691162\\
36.676872253418 14.7005214691162\\
36.7768707275391 14.7005214691162\\
36.8673248291016 14.7005205154419\\
36.9396858215332 14.7005214691162\\
37.0192794799805 14.7005205154419\\
37.1068305969238 14.7005205154419\\
37.2031402587891 14.7005214691162\\
37.3031425476074 14.7005224227905\\
37.4031410217285 14.7005224227905\\
37.5031433105469 14.7005224227905\\
37.603141784668 14.7005224227905\\
37.7031402587891 14.7005224227905\\
37.8031425476074 14.7005224227905\\
37.9031410217285 14.7005224227905\\
38.0031433105469 14.7005224227905\\
38.103141784668 14.7005224227905\\
38.2031402587891 14.7005224227905\\
38.3031425476074 14.7005224227905\\
38.4031410217285 14.7005224227905\\
38.5031433105469 14.7005224227905\\
38.603141784668 14.7005224227905\\
38.7031402587891 14.7005224227905\\
38.8031425476074 14.7005224227905\\
38.9031410217285 14.7005224227905\\
39.0031433105469 14.7005224227905\\
39.103141784668 14.7005224227905\\
39.2031402587891 14.7005224227905\\
39.3031425476074 14.7005224227905\\
39.4031410217285 14.7005224227905\\
39.5031433105469 14.7005224227905\\
39.603141784668 14.7005224227905\\
39.7031402587891 14.7005224227905\\
39.8031425476074 14.7005224227905\\
39.9031410217285 14.7005224227905\\
40.0031433105469 14.7005224227905\\
40.103141784668 14.7005224227905\\
40.2031402587891 14.7005224227905\\
40.3031425476074 14.7005224227905\\
40.4031410217285 14.7005224227905\\
40.5031433105469 14.7005224227905\\
40.603141784668 14.7005224227905\\
40.7031402587891 14.7005224227905\\
40.8031425476074 14.7005224227905\\
40.9031410217285 14.7005224227905\\
41.0031433105469 14.7005224227905\\
41.103141784668 14.7005224227905\\
41.2031402587891 14.7005224227905\\
41.3031425476074 14.7005224227905\\
41.4031410217285 14.7005224227905\\
41.5031433105469 14.7005224227905\\
41.603141784668 14.7005224227905\\
41.7031402587891 14.7005224227905\\
41.8031425476074 14.7005224227905\\
41.9031410217285 14.7005224227905\\
42.0031433105469 14.7005224227905\\
42.103141784668 14.7005224227905\\
42.2031402587891 14.7005224227905\\
42.3031425476074 14.7005233764648\\
42.4031410217285 14.7005252838135\\
42.5031433105469 14.7005348205566\\
42.603141784668 14.7005796432495\\
42.7031402587891 14.700813293457\\
42.8031425476074 14.7020397186279\\
42.9031410217285 14.7071542739868\\
43.0031433105469 14.7185764312744\\
43.103141784668 14.7293090820313\\
43.2031402587891 14.7331857681274\\
43.3031425476074 14.7322092056274\\
43.4031410217285 14.7326755523682\\
43.5031433105469 14.7325897216797\\
43.603141784668 14.7333745956421\\
43.7031402587891 14.7335987091064\\
43.8031425476074 14.7344036102295\\
43.9031410217285 14.7347440719604\\
44.0031433105469 14.7353925704956\\
44.103141784668 14.7358360290527\\
44.2031402587891 14.7364168167114\\
44.3031425476074 14.7369050979614\\
44.4031410217285 14.737455368042\\
44.5031433105469 14.737964630127\\
44.603141784668 14.7385015487671\\
44.7031402587891 14.7390193939209\\
44.8031425476074 14.7395496368408\\
44.9031410217285 14.7400722503662\\
45.0031433105469 14.7405996322632\\
45.103141784668 14.7411241531372\\
45.2031402587891 14.7416505813599\\
45.3031425476074 14.7421751022339\\
45.4031410217285 14.7427015304565\\
45.5031433105469 14.7432260513306\\
45.603141784668 14.7438955307007\\
45.7031402587891 14.7446517944336\\
45.8031425476074 14.7452526092529\\
45.9031410217285 14.7458925247192\\
46.0031433105469 14.7465181350708\\
46.103141784668 14.747145652771\\
46.2031402587891 14.7477703094482\\
46.3031425476074 14.7483940124512\\
46.4031410217285 14.7490158081055\\
46.5031433105469 14.7496366500854\\
46.603141784668 14.7502565383911\\
46.7031402587891 14.7508745193481\\
46.8031425476074 14.7514915466309\\
46.9031410217285 14.7521076202393\\
47.0031433105469 14.752721786499\\
47.103141784668 14.7533302307129\\
47.2031402587891 14.7547473907471\\
47.3031425476074 14.7548007965088\\
47.4031410217285 14.75510597229\\
47.5031433105469 14.7557010650635\\
47.603141784668 14.7562093734741\\
47.7031402587891 14.7567462921143\\
47.8031425476074 14.757266998291\\
47.9031410217285 14.7577886581421\\
48.0031433105469 14.7583045959473\\
48.103141784668 14.7588148117065\\
48.2031402587891 14.7593603134155\\
48.3031425476074 14.7599382400513\\
48.4031410217285 14.7606639862061\\
48.5031433105469 14.7612600326538\\
48.603141784668 14.7618627548218\\
48.7031402587891 14.7625112533569\\
48.8031425476074 14.7631931304932\\
48.9031410217285 14.7639379501343\\
49.0031433105469 14.7675094604492\\
49.098030090332 14.7731132507324\\
49.1739387512207 14.7778854370117\\
49.2346649169922 14.7670297622681\\
49.3014602661133 14.7864770889282\\
49.3450546264648 14.7413110733032\\
49.3930053710938 14.7568492889404\\
49.4215202331543 14.7794589996338\\
49.4451713562012 14.9238920211792\\
49.47119140625 14.6406326293945\\
49.4998092651367 14.7655096054077\\
49.515323638916 14.7719631195068\\
49.5276222229004 15.0700016021729\\
49.5411529541016 14.5051183700562\\
49.5560340881348 14.7441434860229\\
49.5575561523438 14.9260988235474\\
49.5592269897461 14.8672685623169\\
49.5610656738281 14.7720851898193\\
49.5630912780762 14.7697839736938\\
49.565315246582 14.775013923645\\
49.5677642822266 14.7745523452759\\
49.5704574584961 14.7732105255127\\
49.5734214782715 14.7722997665405\\
49.5766792297363 14.7716579437256\\
49.5791168212891 14.7713174819946\\
49.5817947387695 14.7710256576538\\
49.5847434997559 14.7708053588867\\
49.5879859924316 14.7706470489502\\
49.591552734375 14.7705421447754\\
49.5954742431641 14.7704830169678\\
49.5997886657715 14.7704610824585\\
49.6045341491699 14.770471572876\\
49.6097564697266 14.7705078125\\
49.6155014038086 14.7705659866333\\
49.6218185424805 14.7706422805786\\
49.6287651062012 14.7707366943359\\
49.6364097595215 14.7708473205566\\
49.6448211669922 14.7709741592407\\
49.6540679931641 14.7711191177368\\
49.6642456054688 14.7712831497192\\
49.6754379272461 14.7714939117432\\
49.6877479553223 14.7717561721802\\
49.7012901306152 14.772027015686\\
49.7161865234375 14.7723045349121\\
49.7325744628906 14.7726030349731\\
49.7505989074707 14.7729425430298\\
49.7704238891602 14.7733125686646\\
49.792236328125 14.7737236022949\\
49.8162269592285 14.7742366790771\\
49.8426170349121 14.7747392654419\\
49.8716468811035 14.7752819061279\\
49.9035758972168 14.7758703231812\\
49.9387016296387 14.7765121459961\\
49.9773406982422 14.7772274017334\\
50.0198440551758 14.7779836654663\\
50.066593170166 14.7787675857544\\
50.1180229187012 14.7795886993408\\
50.1745910644531 14.7804069519043\\
50.23681640625 14.7852191925049\\
50.3052673339844 14.7783489227295\\
50.3805618286133 14.7855243682861\\
50.4633865356445 14.7805738449097\\
50.5544929504395 14.7869434356689\\
50.6544914245605 14.7814540863037\\
50.7544898986816 14.7884674072266\\
50.8544921875 14.7828998565674\\
50.9544906616211 14.7903347015381\\
51.0544929504395 14.784631729126\\
51.1544914245605 14.7922086715698\\
51.2544898986816 14.7865896224976\\
51.3544921875 14.7888717651367\\
51.3568458557129 14.9311866760254\\
51.359432220459 14.771879196167\\
51.3622817993164 14.7858896255493\\
51.3654136657715 14.7889060974121\\
51.3688583374023 14.7922592163086\\
51.3726463317871 14.7928676605225\\
51.3768157958984 14.7925300598145\\
51.3814010620117 14.7923707962036\\
51.3864479064941 14.7923583984375\\
51.3919944763184 14.7923574447632\\
51.3980979919434 14.7923488616943\\
51.4048118591309 14.7923440933228\\
51.4121971130371 14.7923460006714\\
51.4203224182129 14.7923564910889\\
51.4292602539063 14.7923755645752\\
51.4390869140625 14.7924060821533\\
51.4499015808105 14.7924480438232\\
51.4617958068848 14.7925043106079\\
51.4748802185059 14.792576789856\\
51.4892692565918 14.792667388916\\
51.5051002502441 14.7927761077881\\
51.522518157959 14.7929077148438\\
51.5416717529297 14.7930612564087\\
51.562744140625 14.7932405471802\\
51.5859222412109 14.7934465408325\\
51.6114196777344 14.7936811447144\\
51.6394653320313 14.7939462661743\\
51.6703147888184 14.794243812561\\
51.7042503356934 14.7945747375488\\
51.7415771484375 14.794942855835\\
51.7826385498047 14.7953510284424\\
51.827808380127 14.7958002090454\\
51.8774948120117 14.7962970733643\\
51.9321479797363 14.7968425750732\\
51.9922637939453 14.7974443435669\\
52.0583953857422 14.7981061935425\\
52.1311378479004 14.7988348007202\\
52.2111587524414 14.7996473312378\\
52.2991790771484 14.8031940460205\\
52.3959999084473 14.7987470626831\\
52.4959983825684 14.8027029037476\\
52.5960006713867 14.8096828460693\\
52.6160011291504 14.8044166564941\\
52.6228942871094 14.8235206604004\\
52.6304740905762 14.8275203704834\\
52.637622833252 14.8295345306396\\
52.6454811096191 14.8304376602173\\
52.6532897949219 14.8309841156006\\
52.6618766784668 14.8288736343384\\
52.6713256835938 14.827000617981\\
52.6817169189453 14.8279972076416\\
52.6931457519531 14.8277835845947\\
52.7057189941406 14.8253555297852\\
52.7195510864258 14.8286399841309\\
52.7347640991211 14.8403186798096\\
52.7514991760254 14.8563413619995\\
52.7699089050293 14.8471593856812\\
52.7901573181152 14.8343667984009\\
52.8124351501465 14.8274850845337\\
52.8369369506836 14.8198614120483\\
52.8638877868652 14.8119401931763\\
52.8935356140137 14.8066883087158\\
52.9261474609375 14.8033123016357\\
52.9620208740234 14.803032875061\\
53.0014839172363 14.8056116104126\\
53.0448913574219 14.8073863983154\\
53.0926399230957 14.8022766113281\\
53.1261940002441 14.8064365386963\\
53.1631088256836 14.821720123291\\
53.1971549987793 14.8213319778442\\
53.2346038818359 14.8228759765625\\
53.2757987976074 14.8236560821533\\
53.3211135864258 14.8235054016113\\
53.3709602355957 14.8227787017822\\
53.4181785583496 14.8213768005371\\
53.4612884521484 14.8203830718994\\
53.4995498657227 14.8198862075806\\
53.5327682495117 14.8242158889771\\
53.5693130493164 14.8219871520996\\
53.6095085144043 14.8194122314453\\
53.6537246704102 14.8175020217896\\
53.7023620605469 14.8176488876343\\
53.7558631896973 14.8178815841675\\
53.8147125244141 14.8181009292603\\
53.8794479370117 14.81849193573\\
53.9506568908691 14.8190412521362\\
54.0289878845215 14.819676399231\\
54.1151542663574 14.8204202651978\\
54.2099342346191 14.8224248886108\\
54.3099327087402 14.8209629058838\\
54.4099311828613 14.8246622085571\\
54.5099334716797 14.8229570388794\\
54.6099319458008 14.8247308731079\\
54.6148452758789 14.8701305389404\\
54.6202507019043 14.8082237243652\\
54.6261940002441 14.84446144104\\
54.6327362060547 14.8103818893433\\
54.6399269104004 14.842001914978\\
54.6478385925293 14.8129749298096\\
54.6565437316895 14.8396100997925\\
54.6661186218262 14.8152551651001\\
54.6766510009766 14.8247995376587\\
54.6824378967285 14.8310499191284\\
54.6888046264648 14.8272819519043\\
54.6958045959473 14.8271617889404\\
54.7035064697266 14.8272266387939\\
54.7119827270508 14.8272714614868\\
54.7213020324707 14.8273229598999\\
54.7315559387207 14.8273849487305\\
54.7428321838379 14.8274593353271\\
54.7552375793457 14.8275499343872\\
54.7688827514648 14.827654838562\\
54.7838935852051 14.8277759552002\\
54.800407409668 14.827917098999\\
54.8185691833496 14.8280792236328\\
54.8385467529297 14.8282642364502\\
54.8605270385742 14.8284749984741\\
54.8847007751465 14.8287105560303\\
54.9112930297852 14.828987121582\\
54.940544128418 14.8292970657349\\
54.9727210998535 14.8296546936035\\
55.008113861084 14.830078125\\
55.0470504760742 14.8306045532227\\
55.0898742675781 14.8313159942627\\
55.1369857788086 14.8323945999146\\
55.1888084411621 14.8342180252075\\
55.2458114624023 14.837438583374\\
55.3085136413574 14.8420095443726\\
55.3774871826172 14.8459062576294\\
55.4533576965332 14.8430995941162\\
55.5368156433105 14.8408432006836\\
55.6286201477051 14.8399534225464\\
55.7286186218262 14.8394145965576\\
55.8286209106445 14.8389129638672\\
55.9286193847656 14.8401489257813\\
56.0286178588867 14.841118812561\\
56.1286201477051 14.8425168991089\\
56.2286186218262 14.8434600830078\\
56.3286209106445 14.8447027206421\\
56.4286193847656 14.8457479476929\\
56.5286178588867 14.8469228744507\\
56.6286201477051 14.8480129241943\\
56.7286186218262 14.8491592407227\\
56.8286209106445 14.8503007888794\\
56.9286193847656 14.8514137268066\\
57.0286178588867 14.8525123596191\\
57.1286201477051 14.8536462783813\\
57.2286186218262 14.8547620773315\\
57.3286209106445 14.8558912277222\\
57.4286193847656 14.8570108413696\\
57.5286178588867 14.8581380844116\\
57.6286201477051 14.859260559082\\
57.7286186218262 14.8603858947754\\
57.8286209106445 14.8615093231201\\
57.9286193847656 14.8626337051392\\
58.0286178588867 14.8640766143799\\
58.1286201477051 14.8651514053345\\
58.2286186218262 14.8662977218628\\
58.3286209106445 14.867413520813\\
58.4286193847656 14.8685398101807\\
58.5286178588867 14.8696622848511\\
58.6286201477051 14.8707866668701\\
58.7286186218262 14.8719110488892\\
58.8286209106445 14.8730354309082\\
58.9286193847656 14.8741598129272\\
59.0286178588867 14.8752841949463\\
59.1286201477051 14.876410484314\\
59.2286186218262 14.877534866333\\
59.3286209106445 14.8786592483521\\
59.4286193847656 14.8797836303711\\
59.5286178588867 14.8807325363159\\
59.6286201477051 14.8817081451416\\
59.7286186218262 14.8828678131104\\
59.8286209106445 14.8839597702026\\
59.9143104553223 14.8863573074341\\
60 14.8874864578247\\
60.0942611694336 14.8873443603516\\
60.1942596435547 14.8881864547729\\
60.2942581176758 14.8893346786499\\
60.3942604064941 14.8904495239258\\
60.4942588806152 14.8915777206421\\
60.5942611694336 14.8927049636841\\
60.6942596435547 14.8938369750977\\
60.7942581176758 14.894983291626\\
60.8942604064941 14.896125793457\\
60.9942588806152 14.8972797393799\\
61.0942611694336 14.8984518051147\\
61.1942596435547 14.8996496200562\\
61.2942581176758 14.9009046554565\\
61.3942604064941 14.9067459106445\\
61.4835205078125 14.9109716415405\\
61.5549278259277 14.9102621078491\\
61.6334800720215 14.904842376709\\
61.7198829650879 14.9054689407349\\
61.8149299621582 14.9218549728394\\
61.8549957275391 14.8702697753906\\
61.8990707397461 14.9019804000854\\
61.9192581176758 14.9139614105225\\
61.9358787536621 14.9171876907349\\
61.9541664123535 14.9087867736816\\
61.9742813110352 14.9088153839111\\
61.9964065551758 14.9089679718018\\
62.0207443237305 14.9091367721558\\
62.0475158691406 14.9093437194824\\
62.0769653320313 14.9095907211304\\
62.1093559265137 14.9098787307739\\
62.1449890136719 14.9102115631104\\
62.1841850280762 14.9105911254883\\
62.2273025512695 14.911018371582\\
62.274730682373 14.9114971160889\\
62.3269004821777 14.9120292663574\\
62.3842887878418 14.9126167297363\\
62.4474143981934 14.9132633209229\\
62.5168533325195 14.9139757156372\\
62.5932350158691 14.914758682251\\
62.6772575378418 14.9248132705688\\
62.7696800231934 14.9079313278198\\
62.8696823120117 14.9177074432373\\
62.9696807861328 14.921893119812\\
63.0696792602539 14.9262704849243\\
63.1196823120117 14.9758176803589\\
63.1572036743164 14.9348125457764\\
63.1601943969727 14.915491104126\\
63.1634902954102 14.9251308441162\\
63.1671104431152 14.9421281814575\\
63.1710968017578 14.9436464309692\\
63.175479888916 14.9309911727905\\
63.1803016662598 14.9213371276855\\
63.185604095459 14.9098138809204\\
63.1914367675781 14.8987636566162\\
63.1978569030762 14.8919439315796\\
63.2049140930176 14.8871259689331\\
63.2126808166504 14.8822116851807\\
63.2212219238281 14.8759126663208\\
63.2306175231934 14.866961479187\\
63.2409515380859 14.8566970825195\\
63.2523231506348 14.8484077453613\\
63.2648277282715 14.8445434570313\\
63.2785835266113 14.8456974029541\\
63.2937164306641 14.8517847061157\\
63.310359954834 14.8639039993286\\
63.328670501709 14.8813123703003\\
63.3488121032715 14.9066333770752\\
63.3709678649902 14.8992128372192\\
63.395336151123 14.9037160873413\\
63.4221420288086 14.9162464141846\\
63.4516296386719 14.9140577316284\\
63.4840698242188 14.9195671081543\\
63.4999580383301 14.9280624389648\\
63.5174407958984 14.9329414367676\\
63.5366668701172 14.9371490478516\\
63.5578193664551 14.9436540603638\\
63.5810852050781 14.9481840133667\\
63.6066780090332 14.945032119751\\
63.6348304748535 14.9383449554443\\
63.6657981872559 14.9349155426025\\
63.6998634338379 14.9338417053223\\
63.7373313903809 14.9334278106689\\
63.7785491943359 14.9329471588135\\
63.8238906860352 14.9327774047852\\
63.8737640380859 14.9332523345947\\
63.9286231994629 14.9342765808105\\
63.9762001037598 14.9343881607056\\
64.0218276977539 14.9343795776367\\
64.0647964477539 14.9355573654175\\
64.112060546875 14.9372110366821\\
64.1640548706055 14.9358682632446\\
64.2212448120117 14.9332551956177\\
64.2841567993164 14.933557510376\\
64.3533554077148 14.9341106414795\\
64.4294738769531 14.9346218109131\\
64.5132064819336 14.9353494644165\\
64.6053161621094 14.9382305145264\\
64.7053146362305 14.9351186752319\\
64.8053131103516 14.940523147583\\
64.9053115844727 14.937050819397\\
65.0053176879883 14.9394378662109\\
65.0079345703125 15.0532741546631\\
65.0108184814453 14.9038276672363\\
65.0139923095703 14.9057626724243\\
65.0165023803711 14.9522666931152\\
65.0192642211914 14.9630641937256\\
65.0223007202148 14.9450578689575\\
65.025634765625 14.9417514801025\\
65.0293121337891 14.9424686431885\\
65.0333557128906 14.9427337646484\\
65.0377960205078 14.9427156448364\\
65.0426864624023 14.942699432373\\
65.0480651855469 14.9426984786987\\
65.0539855957031 14.942702293396\\
65.0604934692383 14.9427118301392\\
65.0676498413086 14.9427270889282\\
65.0755233764648 14.9427490234375\\
65.0841903686523 14.9427804946899\\
65.0937194824219 14.9428215026855\\
65.1042022705078 14.9428749084473\\
65.1157302856445 14.9429407119751\\
65.1284103393555 14.943021774292\\
65.1423645019531 14.9431190490723\\
65.1577072143555 14.9432353973389\\
65.1745910644531 14.9433717727661\\
65.1931610107422 14.9435300827026\\
65.2135848999023 14.9437122344971\\
65.2360534667969 14.943920135498\\
65.2607727050781 14.9441556930542\\
65.287956237793 14.9444198608398\\
65.3178634643555 14.9447164535522\\
65.3507614135742 14.9450464248657\\
65.3869476318359 14.9454126358032\\
65.4267501831055 14.9458169937134\\
65.4705352783203 14.9462633132935\\
65.5186996459961 14.9467554092407\\
65.5716781616211 14.9472970962524\\
65.6299591064453 14.9478940963745\\
65.6940612792969 14.9485502243042\\
65.7645797729492 14.9492712020874\\
65.8421478271484 14.950065612793\\
65.9274749755859 14.9509391784668\\
66.0213317871094 14.9536094665527\\
66.1213302612305 14.9510793685913\\
66.2213287353516 14.9541921615601\\
66.3213272094727 14.9630889892578\\
66.3413314819336 14.9709367752075\\
66.3490600585938 14.9984550476074\\
66.3565673828125 15.0022993087769\\
66.3648223876953 14.9988698959351\\
66.3727111816406 14.9954147338867\\
66.3813858032227 15.0011625289917\\
66.3909301757813 15.030611038208\\
66.4014358520508 15.0612630844116\\
66.4129867553711 15.0643491744995\\
66.4256896972656 15.0419807434082\\
66.4396667480469 15.021580696106\\
66.4550399780273 14.9989366531372\\
66.4719543457031 14.9751787185669\\
66.4905548095703 14.9508867263794\\
66.5110244750977 14.9203605651855\\
66.5335311889648 14.8849954605103\\
66.5504302978516 14.8700475692749\\
66.5690307617188 14.8608989715576\\
66.5894775390625 14.8580551147461\\
66.6119766235352 14.8778924942017\\
66.6367263793945 14.9047927856445\\
66.663948059082 14.9585361480713\\
66.6938934326172 14.9653215408325\\
66.7268295288086 14.9680328369141\\
66.7630615234375 14.9693775177002\\
66.8029174804688 14.9684762954712\\
66.8467636108398 14.9666681289673\\
66.8949890136719 14.9652538299561\\
66.9480361938477 14.9785985946655\\
67.0063934326172 14.9617881774902\\
67.0705795288086 14.973708152771\\
67.111457824707 14.9570655822754\\
67.1564254760742 14.9606628417969\\
67.1973037719727 14.9605274200439\\
67.2330703735352 14.9614667892456\\
67.2724151611328 14.9630403518677\\
67.3156967163086 14.9651737213135\\
67.3633117675781 14.9666976928711\\
67.4156799316406 14.9682493209839\\
67.4732894897461 14.9705944061279\\
67.5366516113281 14.9731826782227\\
67.6063613891602 14.9763174057007\\
67.6830368041992 14.979151725769\\
67.7673797607422 14.9768600463867\\
67.8601531982422 14.9768552780151\\
67.9601516723633 14.976375579834\\
68.0601577758789 14.9761028289795\\
68.16015625 14.9757890701294\\
68.2601547241211 14.9765577316284\\
68.3601531982422 14.9777040481567\\
68.4601516723633 14.9789714813232\\
68.5601577758789 14.9800004959106\\
68.66015625 14.9811868667603\\
68.7601547241211 14.9822721481323\\
68.8601531982422 14.9834251403809\\
68.9601516723633 14.9845333099365\\
69.0601577758789 14.9856719970703\\
69.16015625 14.9867906570435\\
69.2601547241211 14.987922668457\\
69.3601531982422 14.9890460968018\\
69.4601516723633 14.9901752471924\\
69.5601577758789 14.9912996292114\\
69.66015625 14.9924278259277\\
69.7601547241211 14.9935541152954\\
69.8601531982422 14.9946813583374\\
69.9601516723633 14.9958076477051\\
70.0601577758789 14.9969348907471\\
70.16015625 14.9980611801147\\
70.2601547241211 14.9991998672485\\
70.3601531982422 15.0003681182861\\
70.4601516723633 15.0017805099487\\
70.5601577758789 15.0028305053711\\
70.66015625 15.0039796829224\\
70.7601547241211 15.0051002502441\\
70.8601531982422 15.0062303543091\\
70.9601516723633 15.0073566436768\\
71.0601577758789 15.0084838867188\\
71.16015625 15.0096111297607\\
71.2601547241211 15.0107383728027\\
71.3601531982422 15.0118656158447\\
71.4601516723633 15.0129928588867\\
71.5601577758789 15.0141201019287\\
71.66015625 15.0152473449707\\
71.7601547241211 15.0163745880127\\
71.8601531982422 15.0175008773804\\
71.9601516723633 15.0184164047241\\
72.0601577758789 15.0194425582886\\
72.16015625 15.0205860137939\\
72.2601547241211 15.0216875076294\\
72.3601531982422 15.022801399231\\
72.4601516723633 15.0239086151123\\
72.5601577758789 15.0250148773193\\
72.66015625 15.0261173248291\\
72.7601547241211 15.0272178649902\\
72.8601531982422 15.0283126831055\\
72.9601516723633 15.0294036865234\\
73.0601577758789 15.0304899215698\\
73.16015625 15.0350484848022\\
73.2601547241211 15.0339670181274\\
73.3601531982422 15.0341634750366\\
73.4601516723633 15.03551197052\\
73.5601577758789 15.0366191864014\\
73.66015625 15.0378694534302\\
73.7601547241211 15.0393571853638\\
73.8601531982422 15.0469388961792\\
73.943000793457 15.0503301620483\\
74.00927734375 15.0449857711792\\
74.0821685791016 15.0427808761597\\
74.162353515625 15.0433492660522\\
74.2505569458008 15.0607118606567\\
74.2926559448242 15.0103273391724\\
74.3389587402344 15.0393514633179\\
74.3595275878906 15.0529079437256\\
74.3765106201172 15.0554208755493\\
74.3951873779297 15.0466804504395\\
74.4157409667969 15.046685218811\\
74.438346862793 15.0468502044678\\
74.4632110595703 15.0470237731934\\
74.4905624389648 15.0472373962402\\
74.5206527709961 15.0474910736084\\
74.5537414550781 15.0477876663208\\
74.5901489257813 15.0481300354004\\
74.6301956176758 15.0485200881958\\
74.6742477416992 15.0489587783813\\
74.7227020263672 15.0494499206543\\
74.7760009765625 15.0499935150146\\
74.8346328735352 15.0505952835083\\
74.8991317749023 15.0512571334839\\
74.9495620727539 15.0519876480103\\
75 15.0525560379028\\
};
\addlegendentry{14 "};

\addplot [
color=green,
solid
]
table[row sep=crcr]{
0 14.7004346847534\\
9.99999971718069e-10 14.7004346847534\\
2.49999998480632e-09 14.7004346847534\\
4.74999994892755e-09 14.7004346847534\\
8.1250002281763e-09 14.7004346847534\\
1.31874999809156e-08 14.7004346847534\\
2.0781250498203e-08 14.7004346847534\\
3.21718758300449e-08 14.7004346847534\\
4.92578138278077e-08 14.7004346847534\\
7.48867208244519e-08 14.7004346847534\\
1.13330081319418e-07 14.7004346847534\\
1.70995122061868e-07 14.7004346847534\\
2.57492672517401e-07 14.7004346847534\\
3.8723902662241e-07 14.7004346847534\\
5.81858500936505e-07 14.7004346847534\\
8.73787769251066e-07 14.7004346847534\\
1.31168167172291e-06 14.7004346847534\\
1.9685226106958e-06 14.7004346847534\\
2.95378367809462e-06 14.7004346847534\\
4.43167573394021e-06 14.7004346847534\\
6.64851359033491e-06 14.7004346847534\\
9.97376992017962e-06 14.7004346847534\\
1.49616553244414e-05 14.7004346847534\\
2.24434825213393e-05 14.7004346847534\\
3.36662233166862e-05 14.7004346847534\\
5.05003372381907e-05 14.7004356384277\\
7.57515081204474e-05 14.7004356384277\\
0.000113628258986864 14.7004356384277\\
0.000170443381648511 14.7004365921021\\
0.000255666091106832 14.7004375457764\\
0.000383500126190484 14.7004384994507\\
0.000575251178815961 14.7004404067993\\
0.000862877757754177 14.7004413604736\\
0.00129431765526533 14.7004413604736\\
0.00194147753063589 14.7004375457764\\
0.00291221728548408 14.7004327774048\\
0.00436832662671804 14.7004318237305\\
0.00597004732117057 14.7004327774048\\
0.00773194013163447 14.7004346847534\\
0.009670021943748 14.7004356384277\\
0.0118019115179777 14.7004356384277\\
0.0141469910740852 14.7004346847534\\
0.0167265776544809 14.7004346847534\\
0.0195641238242388 14.7004346847534\\
0.0226854234933853 14.7004346847534\\
0.0261188540607691 14.7004346847534\\
0.0298956278711557 14.7004346847534\\
0.0340500771999359 14.7004346847534\\
0.0386199727654457 14.7004346847534\\
0.0436468608677387 14.7004346847534\\
0.0491764321923256 14.7004346847534\\
0.0552589632570744 14.7004346847534\\
0.0619497485458851 14.7004346847534\\
0.0693096145987511 14.7004346847534\\
0.0774054601788521 14.7004318237305\\
0.0863108932971954 14.7004365921021\\
0.0961068719625473 14.7004346847534\\
0.106882445514202 14.7004346847534\\
0.118735581636429 14.7004346847534\\
0.13177402317524 14.7004346847534\\
0.146116316318512 14.7004346847534\\
0.161892831325531 14.7004346847534\\
0.179247006773949 14.7004346847534\\
0.198336601257324 14.7004346847534\\
0.21933513879776 14.7004346847534\\
0.242433547973633 14.7004346847534\\
0.267841786146164 14.7004346847534\\
0.29579085111618 14.7004346847534\\
0.32653483748436 14.7004346847534\\
0.360353201627731 14.7004346847534\\
0.397553414106369 14.7004346847534\\
0.438473641872406 14.7004346847534\\
0.483485877513886 14.7004346847534\\
0.532999336719513 14.7004346847534\\
0.587464153766632 14.7004346847534\\
0.647375464439392 14.7004346847534\\
0.71327793598175 14.7004346847534\\
0.7857705950737 14.7004346847534\\
0.865512549877167 14.7004346847534\\
0.953228712081909 14.7004346847534\\
1.04971647262573 14.7004346847534\\
1.14971649646759 14.7004346847534\\
1.24971640110016 14.7004346847534\\
1.34971642494202 14.7004346847534\\
1.44971644878387 14.7004346847534\\
1.54971647262573 14.7004346847534\\
1.64971649646759 14.7004346847534\\
1.74971640110016 14.7004346847534\\
1.84971642494202 14.7004346847534\\
1.94971644878387 14.7004346847534\\
2.04971647262573 14.7004346847534\\
2.1497163772583 14.7004346847534\\
2.24971652030945 14.7004346847534\\
2.34971642494202 14.7004346847534\\
2.44971656799316 14.7004346847534\\
2.54971647262573 14.7004346847534\\
2.6497163772583 14.7004346847534\\
2.74971652030945 14.7004346847534\\
2.84971642494202 14.7004346847534\\
2.94971656799316 14.7004346847534\\
3.04971647262573 14.7004346847534\\
3.1497163772583 14.7004346847534\\
3.24971652030945 14.7004346847534\\
3.34971642494202 14.7004346847534\\
3.44971656799316 14.7004346847534\\
3.54971647262573 14.7004346847534\\
3.6497163772583 14.7004346847534\\
3.74971652030945 14.7004346847534\\
3.84971642494202 14.7004346847534\\
3.94971656799316 14.7004346847534\\
4.04971647262573 14.7004346847534\\
4.1497163772583 14.7004346847534\\
4.24971628189087 14.7004346847534\\
4.3497166633606 14.7004346847534\\
4.44971656799316 14.7004346847534\\
4.54971647262573 14.7004346847534\\
4.6497163772583 14.7004346847534\\
4.74971628189087 14.7004346847534\\
4.8497166633606 14.7004346847534\\
4.94971656799316 14.7004346847534\\
5.04971647262573 14.7004346847534\\
5.1497163772583 14.7004346847534\\
5.24971628189087 14.7004346847534\\
5.3497166633606 14.7004346847534\\
5.44971656799316 14.7004346847534\\
5.54971647262573 14.7004346847534\\
5.6497163772583 14.7004346847534\\
5.74971628189087 14.7004346847534\\
5.8497166633606 14.7004346847534\\
5.94971656799316 14.7004346847534\\
6.04971647262573 14.7004346847534\\
6.1497163772583 14.7004337310791\\
6.24971628189087 14.7004327774048\\
6.3497166633606 14.7004318237305\\
6.44971656799316 14.7004337310791\\
6.54971647262573 14.7004356384277\\
6.6497163772583 14.7004356384277\\
6.74971628189087 14.7004365921021\\
6.8497166633606 14.7004337310791\\
6.94971656799316 14.7004346847534\\
7.04971647262573 14.7004346847534\\
7.1497163772583 14.7004346847534\\
7.24971628189087 14.7004346847534\\
7.3497166633606 14.7004346847534\\
7.44971656799316 14.7004346847534\\
7.54971647262573 14.7004346847534\\
7.6497163772583 14.7004346847534\\
7.74971628189087 14.7004346847534\\
7.8497166633606 14.7004346847534\\
7.94971656799316 14.7004346847534\\
8.04971599578857 14.7004346847534\\
8.1497163772583 14.7004346847534\\
8.24971675872803 14.7004346847534\\
8.34971618652344 14.7004346847534\\
8.44971656799316 14.7004346847534\\
8.54971599578857 14.7004346847534\\
8.6497163772583 14.7004346847534\\
8.74971675872803 14.7004346847534\\
8.84971618652344 14.7004346847534\\
8.94971656799316 14.7004346847534\\
9.04971599578857 14.7004346847534\\
9.1497163772583 14.7004346847534\\
9.24971675872803 14.7004346847534\\
9.34971618652344 14.7004346847534\\
9.44971656799316 14.7004346847534\\
9.54971599578857 14.7004346847534\\
9.6497163772583 14.7004346847534\\
9.74971675872803 14.7004346847534\\
9.84971618652344 14.7004346847534\\
9.94971656799316 14.7004346847534\\
10.0497159957886 14.7004346847534\\
10.1497163772583 14.7004346847534\\
10.249716758728 14.7004346847534\\
10.3497161865234 14.7004346847534\\
10.4497165679932 14.7004346847534\\
10.5497159957886 14.7004346847534\\
10.6497163772583 14.7004346847534\\
10.749716758728 14.7004346847534\\
10.8497161865234 14.7004346847534\\
10.9497165679932 14.7004346847534\\
11.0497159957886 14.7004346847534\\
11.1497163772583 14.7004346847534\\
11.249716758728 14.7004346847534\\
11.3497161865234 14.7004346847534\\
11.4497165679932 14.7004346847534\\
11.5497159957886 14.7004346847534\\
11.6497163772583 14.7004346847534\\
11.749716758728 14.7004346847534\\
11.8497161865234 14.7004346847534\\
11.9497165679932 14.7004346847534\\
12.0497159957886 14.7004346847534\\
12.1497163772583 14.7004346847534\\
12.249716758728 14.7004346847534\\
12.3471450805664 14.7004346847534\\
12.4250917434692 14.7004346847534\\
12.4879579544067 14.7004356384277\\
12.557110786438 14.7004337310791\\
12.6331787109375 14.7004346847534\\
12.7168531417847 14.7004346847534\\
12.808895111084 14.7004346847534\\
12.9088954925537 14.7004346847534\\
13.0088949203491 14.7004346847534\\
13.1088953018188 14.7004346847534\\
13.2088947296143 14.7004346847534\\
13.308895111084 14.7004346847534\\
13.4088954925537 14.7004346847534\\
13.5088949203491 14.7004346847534\\
13.6088953018188 14.7004346847534\\
13.7088947296143 14.7004346847534\\
13.808895111084 14.7004346847534\\
13.9088954925537 14.7004346847534\\
14.0088949203491 14.7004346847534\\
14.1088953018188 14.7004346847534\\
14.2088947296143 14.7004346847534\\
14.308895111084 14.7004346847534\\
14.4088954925537 14.7004346847534\\
14.5088949203491 14.7004346847534\\
14.6088953018188 14.7004346847534\\
14.7088947296143 14.7004346847534\\
14.808895111084 14.7004346847534\\
14.9088954925537 14.7004346847534\\
15.0088949203491 14.7004346847534\\
15.1088953018188 14.7004346847534\\
15.2088947296143 14.7004346847534\\
15.308895111084 14.7004346847534\\
15.4088954925537 14.7004346847534\\
15.5088949203491 14.7004346847534\\
15.6088953018188 14.7004346847534\\
15.7088947296143 14.7004346847534\\
15.808895111084 14.7004346847534\\
15.9088954925537 14.7004346847534\\
16.0088958740234 14.7004346847534\\
16.1088943481445 14.7004346847534\\
16.2088947296143 14.7004346847534\\
16.308895111084 14.7004346847534\\
16.4088954925537 14.7004346847534\\
16.5088958740234 14.7004346847534\\
16.6088943481445 14.7004346847534\\
16.7088947296143 14.7004346847534\\
16.808895111084 14.7004346847534\\
16.9088954925537 14.7004346847534\\
17.0088958740234 14.7004346847534\\
17.1088943481445 14.7004346847534\\
17.2088947296143 14.7004346847534\\
17.308895111084 14.7004346847534\\
17.4088954925537 14.7004346847534\\
17.5088958740234 14.7004346847534\\
17.6088943481445 14.7004346847534\\
17.7088947296143 14.7004346847534\\
17.808895111084 14.7004346847534\\
17.9088954925537 14.7004346847534\\
18.0088958740234 14.7004346847534\\
18.1088943481445 14.7004346847534\\
18.2088947296143 14.7004346847534\\
18.308895111084 14.7004337310791\\
18.4088954925537 14.7004346847534\\
18.5088958740234 14.7004327774048\\
18.6088943481445 14.7004327774048\\
18.7088947296143 14.7004346847534\\
18.808895111084 14.7004356384277\\
18.9088954925537 14.7004356384277\\
19.0088958740234 14.7004356384277\\
19.1088943481445 14.7004346847534\\
19.2088947296143 14.7004346847534\\
19.308895111084 14.7004346847534\\
19.4088954925537 14.7004346847534\\
19.5088958740234 14.7004346847534\\
19.6088943481445 14.7004346847534\\
19.7088947296143 14.7004346847534\\
19.808895111084 14.7004346847534\\
19.9088954925537 14.7004346847534\\
20.0088958740234 14.7004346847534\\
20.1088943481445 14.7004346847534\\
20.2088947296143 14.7004346847534\\
20.308895111084 14.7004346847534\\
20.4088954925537 14.7004346847534\\
20.5088958740234 14.7004346847534\\
20.6088943481445 14.7004346847534\\
20.7088947296143 14.7004346847534\\
20.808895111084 14.7004346847534\\
20.9088954925537 14.7004346847534\\
21.0088958740234 14.7004346847534\\
21.1088943481445 14.7004346847534\\
21.2088947296143 14.7004346847534\\
21.308895111084 14.7004346847534\\
21.4088954925537 14.7004346847534\\
21.5088958740234 14.7004346847534\\
21.6088943481445 14.7004346847534\\
21.7088947296143 14.7004346847534\\
21.808895111084 14.7004346847534\\
21.9088954925537 14.7004346847534\\
22.0088958740234 14.7004346847534\\
22.1088943481445 14.7004346847534\\
22.2088947296143 14.7004346847534\\
22.308895111084 14.7004346847534\\
22.4088954925537 14.7004346847534\\
22.5088958740234 14.7004346847534\\
22.6088943481445 14.7004346847534\\
22.7088947296143 14.7004346847534\\
22.808895111084 14.7004346847534\\
22.9088954925537 14.7004346847534\\
23.0088958740234 14.7004346847534\\
23.1088943481445 14.7004346847534\\
23.2088947296143 14.7004346847534\\
23.308895111084 14.7004346847534\\
23.4088954925537 14.7004346847534\\
23.5088958740234 14.7004346847534\\
23.6088943481445 14.7004346847534\\
23.7088947296143 14.7004346847534\\
23.808895111084 14.7004346847534\\
23.9088954925537 14.7004346847534\\
24.0088958740234 14.7004346847534\\
24.1088943481445 14.7004346847534\\
24.2088947296143 14.7004346847534\\
24.308895111084 14.7004346847534\\
24.4088954925537 14.7004346847534\\
24.5088958740234 14.7004346847534\\
24.6048831939697 14.7004346847534\\
24.681676864624 14.7004346847534\\
24.7464084625244 14.7004346847534\\
24.8176136016846 14.7004346847534\\
24.895938873291 14.7004346847534\\
24.9820976257324 14.7004346847534\\
25.0768718719482 14.7004346847534\\
25.176872253418 14.7004346847534\\
25.2768707275391 14.7004346847534\\
25.3768711090088 14.7004346847534\\
25.4768714904785 14.7004346847534\\
25.5768718719482 14.7004346847534\\
25.676872253418 14.7004346847534\\
25.7768707275391 14.7004346847534\\
25.8768711090088 14.7004346847534\\
25.9768714904785 14.7004346847534\\
26.0768718719482 14.7004346847534\\
26.176872253418 14.7004346847534\\
26.2768707275391 14.7004346847534\\
26.3768711090088 14.7004346847534\\
26.4768714904785 14.7004346847534\\
26.5768718719482 14.7004346847534\\
26.676872253418 14.7004346847534\\
26.7768707275391 14.7004346847534\\
26.8768711090088 14.7004346847534\\
26.9768714904785 14.7004346847534\\
27.0768718719482 14.7004346847534\\
27.176872253418 14.7004346847534\\
27.2768707275391 14.7004346847534\\
27.3768711090088 14.7004346847534\\
27.4768714904785 14.7004346847534\\
27.5768718719482 14.7004346847534\\
27.676872253418 14.7004346847534\\
27.7768707275391 14.7004346847534\\
27.8768711090088 14.7004346847534\\
27.9768714904785 14.7004346847534\\
28.0768718719482 14.7004346847534\\
28.176872253418 14.7004346847534\\
28.2768707275391 14.7004346847534\\
28.3768711090088 14.7004346847534\\
28.4768714904785 14.7004346847534\\
28.5768718719482 14.7004346847534\\
28.676872253418 14.7004346847534\\
28.7768707275391 14.7004346847534\\
28.8768711090088 14.7004346847534\\
28.9768714904785 14.7004346847534\\
29.0768718719482 14.7004346847534\\
29.176872253418 14.7004346847534\\
29.2768707275391 14.7004346847534\\
29.3768711090088 14.7004346847534\\
29.4768714904785 14.7004346847534\\
29.5768718719482 14.7004346847534\\
29.676872253418 14.7004346847534\\
29.7768707275391 14.7004346847534\\
29.8768711090088 14.7004346847534\\
29.9768714904785 14.7004346847534\\
30.0768718719482 14.7004346847534\\
30.176872253418 14.7004346847534\\
30.2768707275391 14.7004346847534\\
30.3768711090088 14.7004346847534\\
30.4768714904785 14.7004346847534\\
30.5768718719482 14.7004337310791\\
30.676872253418 14.7004346847534\\
30.7768707275391 14.7004318237305\\
30.8768711090088 14.7004337310791\\
30.9768714904785 14.7004346847534\\
31.0768718719482 14.7004356384277\\
31.176872253418 14.7004346847534\\
31.2768707275391 14.7004346847534\\
31.3768711090088 14.7004346847534\\
31.4768714904785 14.7004346847534\\
31.5768718719482 14.7004346847534\\
31.676872253418 14.7004346847534\\
31.7768707275391 14.7004346847534\\
31.8768711090088 14.7004346847534\\
31.9768714904785 14.7004346847534\\
32.0768699645996 14.7004346847534\\
32.176872253418 14.7004346847534\\
32.2768707275391 14.7004346847534\\
32.3768730163574 14.7004346847534\\
32.4768714904785 14.7004346847534\\
32.5768699645996 14.7004346847534\\
32.676872253418 14.7004346847534\\
32.7768707275391 14.7004346847534\\
32.8768730163574 14.7004346847534\\
32.9768714904785 14.7004346847534\\
33.0768699645996 14.7004346847534\\
33.176872253418 14.7004346847534\\
33.2768707275391 14.7004346847534\\
33.3768730163574 14.7004346847534\\
33.4768714904785 14.7004346847534\\
33.5768699645996 14.7004346847534\\
33.676872253418 14.7004346847534\\
33.7768707275391 14.7004346847534\\
33.8768730163574 14.7004346847534\\
33.9768714904785 14.7004346847534\\
34.0768699645996 14.7004346847534\\
34.176872253418 14.7004346847534\\
34.2768707275391 14.7004346847534\\
34.3768730163574 14.7004346847534\\
34.4768714904785 14.7004346847534\\
34.5768699645996 14.7004346847534\\
34.676872253418 14.7004346847534\\
34.7768707275391 14.7004346847534\\
34.8768730163574 14.7004346847534\\
34.9768714904785 14.7004346847534\\
35.0768699645996 14.7004346847534\\
35.176872253418 14.7004346847534\\
35.2768707275391 14.7004346847534\\
35.3768730163574 14.7004346847534\\
35.4768714904785 14.7004346847534\\
35.5768699645996 14.7004346847534\\
35.676872253418 14.7004346847534\\
35.7768707275391 14.7004346847534\\
35.8768730163574 14.7004346847534\\
35.9768714904785 14.7004346847534\\
36.0768699645996 14.7004346847534\\
36.176872253418 14.7004346847534\\
36.2768707275391 14.7004346847534\\
36.3768730163574 14.7004346847534\\
36.4768714904785 14.7004346847534\\
36.5768699645996 14.7004346847534\\
36.676872253418 14.7004346847534\\
36.7768707275391 14.7004346847534\\
36.8673248291016 14.7004346847534\\
36.9396858215332 14.7004346847534\\
37.0192794799805 14.7004346847534\\
37.1068305969238 14.7004346847534\\
37.2031402587891 14.7004346847534\\
37.3031425476074 14.7004346847534\\
37.4031410217285 14.7004346847534\\
37.5031433105469 14.7004346847534\\
37.603141784668 14.7004346847534\\
37.7031402587891 14.7004346847534\\
37.8031425476074 14.7004346847534\\
37.9031410217285 14.7004346847534\\
38.0031433105469 14.7004346847534\\
38.103141784668 14.7004346847534\\
38.2031402587891 14.7004346847534\\
38.3031425476074 14.7004346847534\\
38.4031410217285 14.7004346847534\\
38.5031433105469 14.7004346847534\\
38.603141784668 14.7004346847534\\
38.7031402587891 14.7004346847534\\
38.8031425476074 14.7004346847534\\
38.9031410217285 14.7004346847534\\
39.0031433105469 14.7004346847534\\
39.103141784668 14.7004346847534\\
39.2031402587891 14.7004346847534\\
39.3031425476074 14.7004346847534\\
39.4031410217285 14.7004346847534\\
39.5031433105469 14.7004346847534\\
39.603141784668 14.7004346847534\\
39.7031402587891 14.7004346847534\\
39.8031425476074 14.7004346847534\\
39.9031410217285 14.7004346847534\\
40.0031433105469 14.7004346847534\\
40.103141784668 14.7004346847534\\
40.2031402587891 14.7004346847534\\
40.3031425476074 14.7004346847534\\
40.4031410217285 14.7004346847534\\
40.5031433105469 14.7004346847534\\
40.603141784668 14.7004346847534\\
40.7031402587891 14.7004346847534\\
40.8031425476074 14.7004346847534\\
40.9031410217285 14.7004346847534\\
41.0031433105469 14.7004346847534\\
41.103141784668 14.7004346847534\\
41.2031402587891 14.7004346847534\\
41.3031425476074 14.7004346847534\\
41.4031410217285 14.7004346847534\\
41.5031433105469 14.7004346847534\\
41.603141784668 14.7004346847534\\
41.7031402587891 14.7004346847534\\
41.8031425476074 14.7004346847534\\
41.9031410217285 14.7004346847534\\
42.0031433105469 14.7004346847534\\
42.103141784668 14.7004346847534\\
42.2031402587891 14.7004346847534\\
42.3031425476074 14.7004346847534\\
42.4031410217285 14.7004346847534\\
42.5031433105469 14.7004346847534\\
42.603141784668 14.7004346847534\\
42.7031402587891 14.7004346847534\\
42.8031425476074 14.7004337310791\\
42.9031410217285 14.7004337310791\\
43.0031433105469 14.7004327774048\\
43.103141784668 14.7004327774048\\
43.2031402587891 14.7004346847534\\
43.3031425476074 14.7004356384277\\
43.4031410217285 14.7004356384277\\
43.5031433105469 14.7004346847534\\
43.603141784668 14.7004346847534\\
43.7031402587891 14.7004346847534\\
43.8031425476074 14.7004346847534\\
43.9031410217285 14.7004346847534\\
44.0031433105469 14.7004346847534\\
44.103141784668 14.7004346847534\\
44.2031402587891 14.7004346847534\\
44.3031425476074 14.7004346847534\\
44.4031410217285 14.7004346847534\\
44.5031433105469 14.7004346847534\\
44.603141784668 14.7004346847534\\
44.7031402587891 14.7004346847534\\
44.8031425476074 14.7004346847534\\
44.9031410217285 14.7004346847534\\
45.0031433105469 14.7004346847534\\
45.103141784668 14.7004346847534\\
45.2031402587891 14.7004346847534\\
45.3031425476074 14.7004346847534\\
45.4031410217285 14.7004346847534\\
45.5031433105469 14.7004346847534\\
45.603141784668 14.7004346847534\\
45.7031402587891 14.7004346847534\\
45.8031425476074 14.7004346847534\\
45.9031410217285 14.7004346847534\\
46.0031433105469 14.7004346847534\\
46.103141784668 14.7004346847534\\
46.2031402587891 14.7004346847534\\
46.3031425476074 14.7004346847534\\
46.4031410217285 14.7004346847534\\
46.5031433105469 14.7004346847534\\
46.603141784668 14.7004346847534\\
46.7031402587891 14.7004346847534\\
46.8031425476074 14.7004346847534\\
46.9031410217285 14.7004346847534\\
47.0031433105469 14.7004346847534\\
47.103141784668 14.7004346847534\\
47.2031402587891 14.7004346847534\\
47.3031425476074 14.7004346847534\\
47.4031410217285 14.7004346847534\\
47.5031433105469 14.7004346847534\\
47.603141784668 14.7004346847534\\
47.7031402587891 14.7004346847534\\
47.8031425476074 14.7004346847534\\
47.9031410217285 14.7004346847534\\
48.0031433105469 14.7004346847534\\
48.103141784668 14.7004346847534\\
48.2031402587891 14.7004346847534\\
48.3031425476074 14.7004346847534\\
48.4031410217285 14.7004346847534\\
48.5031433105469 14.7004346847534\\
48.603141784668 14.7004346847534\\
48.7031402587891 14.7004346847534\\
48.8031425476074 14.7004346847534\\
48.9031410217285 14.7004346847534\\
49.0031433105469 14.7004337310791\\
49.098030090332 14.7004337310791\\
49.1739387512207 14.7004337310791\\
49.2346649169922 14.7004346847534\\
49.3014602661133 14.7004327774048\\
49.3450546264648 14.7004337310791\\
49.3930053710938 14.7004280090332\\
49.4215202331543 14.7004451751709\\
49.4451713562012 14.7004337310791\\
49.47119140625 14.7004327774048\\
49.4998092651367 14.700421333313\\
49.515323638916 14.7004585266113\\
49.5276222229004 14.7004289627075\\
49.5411529541016 14.7004442214966\\
49.5560340881348 14.7002801895142\\
49.5575561523438 14.7013158798218\\
49.5592269897461 14.7010650634766\\
49.5610656738281 14.700511932373\\
49.5630912780762 14.7003774642944\\
49.565315246582 14.7004451751709\\
49.5677642822266 14.7004709243774\\
49.5704574584961 14.7004613876343\\
49.5734214782715 14.7004528045654\\
49.5766792297363 14.7004470825195\\
49.5791168212891 14.7004442214966\\
49.5817947387695 14.7004423141479\\
49.5847434997559 14.7004404067993\\
49.5879859924316 14.700439453125\\
49.591552734375 14.7004375457764\\
49.5954742431641 14.7004365921021\\
49.5997886657715 14.7004356384277\\
49.6045341491699 14.7004356384277\\
49.6097564697266 14.7004346847534\\
49.6155014038086 14.7004346847534\\
49.6218185424805 14.7004346847534\\
49.6287651062012 14.7004346847534\\
49.6364097595215 14.7004346847534\\
49.6448211669922 14.7004346847534\\
49.6540679931641 14.7004346847534\\
49.6642456054688 14.7004346847534\\
49.6754379272461 14.7004346847534\\
49.6877479553223 14.7004356384277\\
49.7012901306152 14.7004356384277\\
49.7161865234375 14.7004356384277\\
49.7325744628906 14.7004356384277\\
49.7505989074707 14.7004356384277\\
49.7704238891602 14.7004356384277\\
49.792236328125 14.7004356384277\\
49.8162269592285 14.7004356384277\\
49.8426170349121 14.7004356384277\\
49.8716468811035 14.7004356384277\\
49.9035758972168 14.7004346847534\\
49.9387016296387 14.7004346847534\\
49.9773406982422 14.7004346847534\\
50.0198440551758 14.7004346847534\\
50.066593170166 14.7004337310791\\
50.1180229187012 14.7004337310791\\
50.1745910644531 14.7004337310791\\
50.23681640625 14.7004318237305\\
50.3052673339844 14.7004365921021\\
50.3805618286133 14.7004337310791\\
50.4633865356445 14.7004365921021\\
50.5544929504395 14.7004346847534\\
50.6544914245605 14.7004356384277\\
50.7544898986816 14.7004346847534\\
50.8544921875 14.7004356384277\\
50.9544906616211 14.7004346847534\\
51.0544929504395 14.7004365921021\\
51.1544914245605 14.7004346847534\\
51.2544898986816 14.7004365921021\\
51.3544921875 14.7004356384277\\
51.3568458557129 14.700343132019\\
51.359432220459 14.7005558013916\\
51.3622817993164 14.7004327774048\\
51.3654136657715 14.7004117965698\\
51.3688583374023 14.7004337310791\\
51.3726463317871 14.7004404067993\\
51.3768157958984 14.7004384994507\\
51.3814010620117 14.7004365921021\\
51.3864479064941 14.7004365921021\\
51.3919944763184 14.7004365921021\\
51.3980979919434 14.7004365921021\\
51.4048118591309 14.7004365921021\\
51.4121971130371 14.7004365921021\\
51.4203224182129 14.7004365921021\\
51.4292602539063 14.7004365921021\\
51.4390869140625 14.7004365921021\\
51.4499015808105 14.7004365921021\\
51.4617958068848 14.7004365921021\\
51.4748802185059 14.7004356384277\\
51.4892692565918 14.7004356384277\\
51.5051002502441 14.7004356384277\\
51.522518157959 14.7004356384277\\
51.5416717529297 14.7004356384277\\
51.562744140625 14.7004356384277\\
51.5859222412109 14.7004356384277\\
51.6114196777344 14.7004356384277\\
51.6394653320313 14.7004356384277\\
51.6703147888184 14.7004356384277\\
51.7042503356934 14.7004356384277\\
51.7415771484375 14.7004356384277\\
51.7826385498047 14.7004356384277\\
51.827808380127 14.7004356384277\\
51.8774948120117 14.7004356384277\\
51.9321479797363 14.7004356384277\\
51.9922637939453 14.7004356384277\\
52.0583953857422 14.7004356384277\\
52.1311378479004 14.7004356384277\\
52.2111587524414 14.7004356384277\\
52.2991790771484 14.7004346847534\\
52.3959999084473 14.7004365921021\\
52.4959983825684 14.7004346847534\\
52.5960006713867 14.7004671096802\\
52.6160011291504 14.7005157470703\\
52.6228942871094 14.7006378173828\\
52.6304740905762 14.7006931304932\\
52.637622833252 14.7007284164429\\
52.6454811096191 14.7007627487183\\
52.6532897949219 14.7008094787598\\
52.6618766784668 14.7008581161499\\
52.6713256835938 14.700873374939\\
52.6817169189453 14.7008419036865\\
52.6931457519531 14.7008333206177\\
52.7057189941406 14.7008457183838\\
52.7195510864258 14.7162704467773\\
52.7347640991211 14.7814197540283\\
52.7514991760254 14.8313913345337\\
52.7699089050293 14.7942914962769\\
52.7901573181152 14.7505931854248\\
52.8124351501465 14.7507543563843\\
52.8369369506836 14.741641998291\\
52.8638877868652 14.7334995269775\\
52.8935356140137 14.7306871414185\\
52.9261474609375 14.7293510437012\\
52.9620208740234 14.7293262481689\\
53.0014839172363 14.7307090759277\\
53.0448913574219 14.7264175415039\\
53.0926399230957 14.7075109481812\\
53.1261940002441 14.7034454345703\\
53.1631088256836 14.7010097503662\\
53.1971549987793 14.7005605697632\\
53.2346038818359 14.7004499435425\\
53.2757987976074 14.7004337310791\\
53.3211135864258 14.7004327774048\\
53.3709602355957 14.7004337310791\\
53.4181785583496 14.7004337310791\\
53.4612884521484 14.70046043396\\
53.4995498657227 14.7006483078003\\
53.5327682495117 14.7005710601807\\
53.5693130493164 14.7004871368408\\
53.6095085144043 14.7005214691162\\
53.6537246704102 14.7005128860474\\
53.7023620605469 14.7005004882813\\
53.7558631896973 14.7004919052124\\
53.8147125244141 14.7004842758179\\
53.8794479370117 14.7004613876343\\
53.9506568908691 14.7004737854004\\
54.0289878845215 14.7004680633545\\
54.1151542663574 14.70046043396\\
54.2099342346191 14.7004537582397\\
54.3099327087402 14.7004518508911\\
54.4099311828613 14.7004461288452\\
54.5099334716797 14.7004461288452\\
54.6099319458008 14.7004432678223\\
54.6148452758789 14.7004356384277\\
54.6202507019043 14.7004528045654\\
54.6261940002441 14.7004356384277\\
54.6327362060547 14.7004489898682\\
54.6399269104004 14.7004384994507\\
54.6478385925293 14.7004470825195\\
54.6565437316895 14.700439453125\\
54.6661186218262 14.7004470825195\\
54.6766510009766 14.7004413604736\\
54.6824378967285 14.7004451751709\\
54.6888046264648 14.7004461288452\\
54.6958045959473 14.7004451751709\\
54.7035064697266 14.7004461288452\\
54.7119827270508 14.7004461288452\\
54.7213020324707 14.7004461288452\\
54.7315559387207 14.7004461288452\\
54.7428321838379 14.7004461288452\\
54.7552375793457 14.7004489898682\\
54.7688827514648 14.7004489898682\\
54.7838935852051 14.7004508972168\\
54.800407409668 14.7004528045654\\
54.8185691833496 14.7004566192627\\
54.8385467529297 14.7004613876343\\
54.8605270385742 14.7004680633545\\
54.8847007751465 14.700475692749\\
54.9112930297852 14.7004976272583\\
54.940544128418 14.7005290985107\\
54.9727210998535 14.7005844116211\\
55.008113861084 14.7006883621216\\
55.0470504760742 14.7008943557739\\
55.0898742675781 14.7013206481934\\
55.1369857788086 14.702244758606\\
55.1888084411621 14.7042741775513\\
55.2458114624023 14.708592414856\\
55.3085136413574 14.71617603302\\
55.3774871826172 14.7257823944092\\
55.4533576965332 14.7300510406494\\
55.5368156433105 14.7317171096802\\
55.6286201477051 14.73264503479\\
55.7286186218262 14.7328653335571\\
55.8286209106445 14.732367515564\\
55.9286193847656 14.7329635620117\\
56.0286178588867 14.7333345413208\\
56.1286201477051 14.7340755462646\\
56.2286186218262 14.7344665527344\\
56.3286209106445 14.73508644104\\
56.4286193847656 14.7355556488037\\
56.5286178588867 14.7361240386963\\
56.6286201477051 14.736626625061\\
56.7286186218262 14.73717212677\\
56.8286209106445 14.7377223968506\\
56.9286193847656 14.7382497787476\\
57.0286178588867 14.7387590408325\\
57.1286201477051 14.739294052124\\
57.2286186218262 14.7398147583008\\
57.3286209106445 14.740345954895\\
57.4286193847656 14.7408685684204\\
57.5286178588867 14.741397857666\\
57.6286201477051 14.7419233322144\\
57.7286186218262 14.7424516677856\\
57.8286209106445 14.742977142334\\
57.9286193847656 14.7435054779053\\
58.0286178588867 14.7443513870239\\
58.1286201477051 14.7449464797974\\
58.2286186218262 14.7455930709839\\
58.3286209106445 14.7462158203125\\
58.4286193847656 14.746844291687\\
58.5286178588867 14.7474679946899\\
58.6286201477051 14.7480926513672\\
58.7286186218262 14.7487144470215\\
58.8286209106445 14.7493352890015\\
58.9286193847656 14.7499551773071\\
59.0286178588867 14.7505731582642\\
59.1286201477051 14.7511920928955\\
59.2286186218262 14.7518091201782\\
59.3286209106445 14.7524242401123\\
59.4286193847656 14.7530374526978\\
59.5286178588867 14.753475189209\\
59.6286201477051 14.7538871765137\\
59.7286186218262 14.7544393539429\\
59.8286209106445 14.7549333572388\\
59.9143104553223 14.756724357605\\
60 14.7576465606689\\
60.0942611694336 14.7573175430298\\
60.1942596435547 14.7576713562012\\
60.2942581176758 14.7582445144653\\
60.3942604064941 14.7587900161743\\
60.4942588806152 14.7593469619751\\
60.5942611694336 14.7599039077759\\
60.6942596435547 14.7604656219482\\
60.7942581176758 14.761043548584\\
60.8942604064941 14.7616233825684\\
60.9942588806152 14.7622175216675\\
61.0942611694336 14.7628374099731\\
61.1942596435547 14.7634944915771\\
61.2942581176758 14.7642259597778\\
61.3942604064941 14.7695741653442\\
61.4835205078125 14.7742338180542\\
61.5549278259277 14.7738199234009\\
61.6334800720215 14.7681999206543\\
61.7198829650879 14.7685880661011\\
61.8149299621582 14.78480052948\\
61.8549957275391 14.7326602935791\\
61.8990707397461 14.7645215988159\\
61.9192581176758 14.7761859893799\\
61.9358787536621 14.779317855835\\
61.9541664123535 14.7709083557129\\
61.9742813110352 14.7709636688232\\
61.9964065551758 14.7711057662964\\
62.0207443237305 14.7712736129761\\
62.0475158691406 14.7714796066284\\
62.0769653320313 14.7717247009277\\
62.1093559265137 14.772011756897\\
62.1449890136719 14.7723426818848\\
62.1841850280762 14.7727203369141\\
62.2273025512695 14.7731466293335\\
62.274730682373 14.7736234664917\\
62.3269004821777 14.7741527557373\\
62.3842887878418 14.7747383117676\\
62.4474143981934 14.7753829956055\\
62.5168533325195 14.7760925292969\\
62.5932350158691 14.7768716812134\\
62.6772575378418 14.7869234085083\\
62.7696800231934 14.7699384689331\\
62.8696823120117 14.7797727584839\\
62.9696807861328 14.7835788726807\\
63.0696792602539 14.7879552841187\\
63.1196823120117 14.8459377288818\\
63.1572036743164 14.8215522766113\\
63.1601943969727 14.8122129440308\\
63.1634902954102 14.8215837478638\\
63.1671104431152 14.8384027481079\\
63.1710968017578 14.8398990631104\\
63.175479888916 14.827826499939\\
63.1803016662598 14.8189325332642\\
63.185604095459 14.8077821731567\\
63.1914367675781 14.7971477508545\\
63.1978569030762 14.7906932830811\\
63.2049140930176 14.7861375808716\\
63.2126808166504 14.7814483642578\\
63.2212219238281 14.775354385376\\
63.2306175231934 14.7665920257568\\
63.2409515380859 14.7565002441406\\
63.2523231506348 14.7483263015747\\
63.2648277282715 14.7444591522217\\
63.2785835266113 14.74538230896\\
63.2937164306641 14.7508277893066\\
63.310359954834 14.7618427276611\\
63.328670501709 14.7775745391846\\
63.3488121032715 14.8004503250122\\
63.3709678649902 14.7910413742065\\
63.395336151123 14.794038772583\\
63.4221420288086 14.8046808242798\\
63.4516296386719 14.8002023696899\\
63.4840698242188 14.8033075332642\\
63.4999580383301 14.8089237213135\\
63.5174407958984 14.8124809265137\\
63.5366668701172 14.8153057098389\\
63.5578193664551 14.820484161377\\
63.5810852050781 14.8240013122559\\
63.6066780090332 14.8201875686646\\
63.6348304748535 14.8128347396851\\
63.6657981872559 14.8085241317749\\
63.6998634338379 14.8064451217651\\
63.7373313903809 14.8049812316895\\
63.7785491943359 14.8034334182739\\
63.8238906860352 14.8021631240845\\
63.8737640380859 14.8015108108521\\
63.9286231994629 14.8014316558838\\
63.9762001037598 14.8005447387695\\
64.0218276977539 14.7997951507568\\
64.0647964477539 14.8003396987915\\
64.112060546875 14.8015193939209\\
64.1640548706055 14.7998294830322\\
64.2212448120117 14.7968893051147\\
64.2841567993164 14.7969312667847\\
64.3533554077148 14.7972631454468\\
64.4294738769531 14.7976064682007\\
64.5132064819336 14.7981977462769\\
64.6053161621094 14.800968170166\\
64.7053146362305 14.7977294921875\\
64.8053131103516 14.8030796051025\\
64.9053115844727 14.799521446228\\
65.0053176879883 14.8018865585327\\
65.0079345703125 14.9211874008179\\
65.0108184814453 14.7532529830933\\
65.0139923095703 14.7732934951782\\
65.0165023803711 14.8170938491821\\
65.0192642211914 14.8227758407593\\
65.0223007202148 14.8054294586182\\
65.025634765625 14.8040208816528\\
65.0293121337891 14.8043365478516\\
65.0333557128906 14.8044385910034\\
65.0377960205078 14.8044281005859\\
65.0426864624023 14.8044195175171\\
65.0480651855469 14.8044185638428\\
65.0539855957031 14.80442237854\\
65.0604934692383 14.8044319152832\\
65.0676498413086 14.8044471740723\\
65.0755233764648 14.8044691085815\\
65.0841903686523 14.804500579834\\
65.0937194824219 14.8045415878296\\
65.1042022705078 14.8045949935913\\
65.1157302856445 14.8046607971191\\
65.1284103393555 14.804741859436\\
65.1423645019531 14.8048400878906\\
65.1577072143555 14.8049554824829\\
65.1745910644531 14.8050918579102\\
65.1931610107422 14.8052501678467\\
65.2135848999023 14.8054323196411\\
65.2360534667969 14.8056402206421\\
65.2607727050781 14.8058757781982\\
65.287956237793 14.8061408996582\\
65.3178634643555 14.8064374923706\\
65.3507614135742 14.8067674636841\\
65.3869476318359 14.8071336746216\\
65.4267501831055 14.8075380325317\\
65.4705352783203 14.8079853057861\\
65.5186996459961 14.8084774017334\\
65.5716781616211 14.8090200424194\\
65.6299591064453 14.8096160888672\\
65.6940612792969 14.8102731704712\\
65.7645797729492 14.8109951019287\\
65.8421478271484 14.81178855896\\
65.9274749755859 14.8126621246338\\
66.0213317871094 14.8153343200684\\
66.1213302612305 14.8127927780151\\
66.2213287353516 14.8159160614014\\
66.3213272094727 14.8247833251953\\
66.3413314819336 14.8341493606567\\
66.3490600585938 14.8635358810425\\
66.3565673828125 14.8677759170532\\
66.3648223876953 14.8654117584229\\
66.3727111816406 14.8633127212524\\
66.3813858032227 14.8705043792725\\
66.3909301757813 14.901686668396\\
66.4014358520508 14.9339475631714\\
66.4129867553711 14.9385004043579\\
66.4256896972656 14.9176511764526\\
66.4396667480469 14.8991689682007\\
66.4550399780273 14.8786306381226\\
66.4719543457031 14.8571538925171\\
66.4905548095703 14.8352222442627\\
66.5110244750977 14.8070163726807\\
66.5335311889648 14.7739591598511\\
66.5504302978516 14.7611961364746\\
66.5690307617188 14.75328540802\\
66.5894775390625 14.7513961791992\\
66.6119766235352 14.7717170715332\\
66.6367263793945 14.7980251312256\\
66.663948059082 14.8486166000366\\
66.6938934326172 14.8523149490356\\
66.7268295288086 14.8523588180542\\
66.7630615234375 14.8497171401978\\
66.8029174804688 14.8483695983887\\
66.8467636108398 14.8427743911743\\
66.8949890136719 14.8378305435181\\
66.9480361938477 14.850546836853\\
67.0063934326172 14.8312520980835\\
67.0705795288086 14.8423528671265\\
67.111457824707 14.823787689209\\
67.1564254760742 14.8267183303833\\
67.1973037719727 14.8257665634155\\
67.2330703735352 14.8260259628296\\
67.2724151611328 14.827169418335\\
67.3156967163086 14.8289728164673\\
67.3633117675781 14.8302440643311\\
67.4156799316406 14.8315734863281\\
67.4732894897461 14.8337287902832\\
67.5366516113281 14.8361644744873\\
67.6063613891602 14.8391675949097\\
67.6830368041992 14.841890335083\\
67.7673797607422 14.8395042419434\\
67.8601531982422 14.8394212722778\\
67.9601516723633 14.8388748168945\\
68.0601577758789 14.838547706604\\
68.16015625 14.8381910324097\\
68.2601547241211 14.8389263153076\\
68.3601531982422 14.8400468826294\\
68.4601516723633 14.8412933349609\\
68.5601577758789 14.8423051834106\\
68.66015625 14.8434772491455\\
68.7601547241211 14.8445472717285\\
68.8601531982422 14.8456869125366\\
68.9601516723633 14.8467817306519\\
69.0601577758789 14.8479070663452\\
69.16015625 14.8490133285522\\
69.2601547241211 14.8501329421997\\
69.3601531982422 14.8512439727783\\
69.4601516723633 14.8523607254028\\
69.5601577758789 14.8534736633301\\
69.66015625 14.8545894622803\\
69.7601547241211 14.8557043075562\\
69.8601531982422 14.8568201065063\\
69.9601516723633 14.8579359054565\\
70.0601577758789 14.8590526580811\\
70.16015625 14.8601684570313\\
70.2601547241211 14.8612966537476\\
70.3601531982422 14.8624601364136\\
70.4601516723633 14.8638668060303\\
70.5601577758789 14.8649110794067\\
70.66015625 14.8660554885864\\
70.7601547241211 14.8671712875366\\
70.8601531982422 14.8682956695557\\
70.9601516723633 14.8694171905518\\
71.0601577758789 14.8705396652222\\
71.16015625 14.8716621398926\\
71.2601547241211 14.872784614563\\
71.3601531982422 14.8739070892334\\
71.4601516723633 14.8750295639038\\
71.5601577758789 14.8761520385742\\
71.66015625 14.8772745132446\\
71.7601547241211 14.878396987915\\
71.8601531982422 14.8795185089111\\
71.9601516723633 14.8804302215576\\
72.0601577758789 14.8814516067505\\
72.16015625 14.8825902938843\\
72.2601547241211 14.8836879730225\\
72.3601531982422 14.8847980499268\\
72.4601516723633 14.8859014511108\\
72.5601577758789 14.8870038986206\\
72.66015625 14.8881025314331\\
72.7601547241211 14.8891983032227\\
72.8601531982422 14.8902902603149\\
72.9601516723633 14.8913774490356\\
73.0601577758789 14.8924598693848\\
73.16015625 14.8970155715942\\
73.2601547241211 14.8959302902222\\
73.3601531982422 14.8961229324341\\
73.4601516723633 14.8974685668945\\
73.5601577758789 14.8985738754272\\
73.66015625 14.8998203277588\\
73.7601547241211 14.9013042449951\\
73.8601531982422 14.9088830947876\\
73.943000793457 14.9122714996338\\
74.00927734375 14.9069242477417\\
74.0821685791016 14.9047174453735\\
74.162353515625 14.9052839279175\\
74.2505569458008 14.9226446151733\\
74.2926559448242 14.8722496032715\\
74.3389587402344 14.9012889862061\\
74.3595275878906 14.9148330688477\\
74.3765106201172 14.9173383712769\\
74.3951873779297 14.9085969924927\\
74.4157409667969 14.9086132049561\\
74.438346862793 14.9087734222412\\
74.4632110595703 14.9089460372925\\
74.4905624389648 14.909158706665\\
74.5206527709961 14.9094123840332\\
74.5537414550781 14.9097080230713\\
74.5901489257813 14.9100503921509\\
74.6301956176758 14.9104385375977\\
74.6742477416992 14.9108772277832\\
74.7227020263672 14.9113664627075\\
74.7760009765625 14.9119100570679\\
74.8346328735352 14.9125108718872\\
74.8991317749023 14.9131717681885\\
74.9495620727539 14.9139003753662\\
75 14.9144678115845\\
};
\addlegendentry{18 "};

\addplot [
color=green,
dash pattern=on 1pt off 3pt on 3pt off 3pt
]
table[row sep=crcr]{
0 14.7003479003906\\
9.99999971718069e-10 14.7003479003906\\
2.49999998480632e-09 14.7003479003906\\
4.74999994892755e-09 14.7003479003906\\
8.1250002281763e-09 14.7003479003906\\
1.31874999809156e-08 14.7003479003906\\
2.0781250498203e-08 14.7003479003906\\
3.21718758300449e-08 14.7003479003906\\
4.92578138278077e-08 14.7003479003906\\
7.48867208244519e-08 14.7003479003906\\
1.13330081319418e-07 14.7003479003906\\
1.70995122061868e-07 14.7003479003906\\
2.57492672517401e-07 14.7003479003906\\
3.8723902662241e-07 14.7003479003906\\
5.81858500936505e-07 14.7003479003906\\
8.73787769251066e-07 14.7003479003906\\
1.31168167172291e-06 14.7003479003906\\
1.9685226106958e-06 14.7003479003906\\
2.95378367809462e-06 14.7003479003906\\
4.43167573394021e-06 14.7003479003906\\
6.64851359033491e-06 14.7003479003906\\
9.97376992017962e-06 14.7003479003906\\
1.49616553244414e-05 14.7003479003906\\
2.24434825213393e-05 14.7003479003906\\
3.36662233166862e-05 14.7003479003906\\
5.05003372381907e-05 14.7003479003906\\
7.57515081204474e-05 14.7003479003906\\
0.000113628258986864 14.7003479003906\\
0.000170443381648511 14.7003479003906\\
0.000255666091106832 14.7003488540649\\
0.000383500126190484 14.7003498077393\\
0.000575251178815961 14.7003507614136\\
0.000862877757754177 14.7003536224365\\
0.00129431765526533 14.7003536224365\\
0.00194147753063589 14.7003507614136\\
0.00291221728548408 14.7003469467163\\
0.00436832662671804 14.7003450393677\\
0.00597004732117057 14.700345993042\\
0.00773194013163447 14.7003479003906\\
0.009670021943748 14.7003479003906\\
0.0118019115179777 14.7003479003906\\
0.0141469910740852 14.7003479003906\\
0.0167265776544809 14.7003479003906\\
0.0195641238242388 14.7003479003906\\
0.0226854234933853 14.7003479003906\\
0.0261188540607691 14.7003479003906\\
0.0298956278711557 14.7003479003906\\
0.0340500771999359 14.7003479003906\\
0.0386199727654457 14.7003479003906\\
0.0436468608677387 14.7003479003906\\
0.0491764321923256 14.7003479003906\\
0.0552589632570744 14.7003479003906\\
0.0619497485458851 14.7003479003906\\
0.0693096145987511 14.7003479003906\\
0.0774054601788521 14.7003450393677\\
0.0863108932971954 14.7003498077393\\
0.0961068719625473 14.7003479003906\\
0.106882445514202 14.7003479003906\\
0.118735581636429 14.7003479003906\\
0.13177402317524 14.7003479003906\\
0.146116316318512 14.7003479003906\\
0.161892831325531 14.7003479003906\\
0.179247006773949 14.7003479003906\\
0.198336601257324 14.7003479003906\\
0.21933513879776 14.7003479003906\\
0.242433547973633 14.7003479003906\\
0.267841786146164 14.7003479003906\\
0.29579085111618 14.7003479003906\\
0.32653483748436 14.7003479003906\\
0.360353201627731 14.7003479003906\\
0.397553414106369 14.7003479003906\\
0.438473641872406 14.7003479003906\\
0.483485877513886 14.7003479003906\\
0.532999336719513 14.7003479003906\\
0.587464153766632 14.7003479003906\\
0.647375464439392 14.7003479003906\\
0.71327793598175 14.7003479003906\\
0.7857705950737 14.7003479003906\\
0.865512549877167 14.7003479003906\\
0.953228712081909 14.7003479003906\\
1.04971647262573 14.7003479003906\\
1.14971649646759 14.7003479003906\\
1.24971640110016 14.7003479003906\\
1.34971642494202 14.7003479003906\\
1.44971644878387 14.7003479003906\\
1.54971647262573 14.7003479003906\\
1.64971649646759 14.7003479003906\\
1.74971640110016 14.7003479003906\\
1.84971642494202 14.7003479003906\\
1.94971644878387 14.7003479003906\\
2.04971647262573 14.7003479003906\\
2.1497163772583 14.7003479003906\\
2.24971652030945 14.7003479003906\\
2.34971642494202 14.7003479003906\\
2.44971656799316 14.7003479003906\\
2.54971647262573 14.7003479003906\\
2.6497163772583 14.7003479003906\\
2.74971652030945 14.7003479003906\\
2.84971642494202 14.7003479003906\\
2.94971656799316 14.7003479003906\\
3.04971647262573 14.7003479003906\\
3.1497163772583 14.7003479003906\\
3.24971652030945 14.7003479003906\\
3.34971642494202 14.7003479003906\\
3.44971656799316 14.7003479003906\\
3.54971647262573 14.7003479003906\\
3.6497163772583 14.7003479003906\\
3.74971652030945 14.7003479003906\\
3.84971642494202 14.7003479003906\\
3.94971656799316 14.7003479003906\\
4.04971647262573 14.7003479003906\\
4.1497163772583 14.7003479003906\\
4.24971628189087 14.7003479003906\\
4.3497166633606 14.7003479003906\\
4.44971656799316 14.7003479003906\\
4.54971647262573 14.7003479003906\\
4.6497163772583 14.7003479003906\\
4.74971628189087 14.7003479003906\\
4.8497166633606 14.7003479003906\\
4.94971656799316 14.7003479003906\\
5.04971647262573 14.7003479003906\\
5.1497163772583 14.7003479003906\\
5.24971628189087 14.7003479003906\\
5.3497166633606 14.7003479003906\\
5.44971656799316 14.7003479003906\\
5.54971647262573 14.7003479003906\\
5.6497163772583 14.7003479003906\\
5.74971628189087 14.7003479003906\\
5.8497166633606 14.7003479003906\\
5.94971656799316 14.7003479003906\\
6.04971647262573 14.7003479003906\\
6.1497163772583 14.7003469467163\\
6.24971628189087 14.700345993042\\
6.3497166633606 14.700345993042\\
6.44971656799316 14.7003469467163\\
6.54971647262573 14.7003488540649\\
6.6497163772583 14.7003479003906\\
6.74971628189087 14.7003488540649\\
6.8497166633606 14.7003469467163\\
6.94971656799316 14.7003479003906\\
7.04971647262573 14.7003469467163\\
7.1497163772583 14.7003479003906\\
7.24971628189087 14.7003479003906\\
7.3497166633606 14.7003479003906\\
7.44971656799316 14.7003479003906\\
7.54971647262573 14.7003479003906\\
7.6497163772583 14.7003479003906\\
7.74971628189087 14.7003479003906\\
7.8497166633606 14.7003479003906\\
7.94971656799316 14.7003479003906\\
8.04971599578857 14.7003479003906\\
8.1497163772583 14.7003479003906\\
8.24971675872803 14.7003479003906\\
8.34971618652344 14.7003479003906\\
8.44971656799316 14.7003479003906\\
8.54971599578857 14.7003479003906\\
8.6497163772583 14.7003479003906\\
8.74971675872803 14.7003479003906\\
8.84971618652344 14.7003479003906\\
8.94971656799316 14.7003479003906\\
9.04971599578857 14.7003479003906\\
9.1497163772583 14.7003479003906\\
9.24971675872803 14.7003479003906\\
9.34971618652344 14.7003479003906\\
9.44971656799316 14.7003479003906\\
9.54971599578857 14.7003479003906\\
9.6497163772583 14.7003479003906\\
9.74971675872803 14.7003479003906\\
9.84971618652344 14.7003479003906\\
9.94971656799316 14.7003479003906\\
10.0497159957886 14.7003479003906\\
10.1497163772583 14.7003479003906\\
10.249716758728 14.7003479003906\\
10.3497161865234 14.7003479003906\\
10.4497165679932 14.7003479003906\\
10.5497159957886 14.7003479003906\\
10.6497163772583 14.7003479003906\\
10.749716758728 14.7003479003906\\
10.8497161865234 14.7003479003906\\
10.9497165679932 14.7003479003906\\
11.0497159957886 14.7003479003906\\
11.1497163772583 14.7003479003906\\
11.249716758728 14.7003479003906\\
11.3497161865234 14.7003479003906\\
11.4497165679932 14.7003479003906\\
11.5497159957886 14.7003479003906\\
11.6497163772583 14.7003479003906\\
11.749716758728 14.7003479003906\\
11.8497161865234 14.7003479003906\\
11.9497165679932 14.7003479003906\\
12.0497159957886 14.7003479003906\\
12.1497163772583 14.7003479003906\\
12.249716758728 14.7003479003906\\
12.3471450805664 14.7003479003906\\
12.4250917434692 14.7003479003906\\
12.4879579544067 14.7003488540649\\
12.557110786438 14.7003469467163\\
12.6331787109375 14.7003479003906\\
12.7168531417847 14.7003479003906\\
12.808895111084 14.7003479003906\\
12.9088954925537 14.7003479003906\\
13.0088949203491 14.7003479003906\\
13.1088953018188 14.7003479003906\\
13.2088947296143 14.7003479003906\\
13.308895111084 14.7003479003906\\
13.4088954925537 14.7003479003906\\
13.5088949203491 14.7003479003906\\
13.6088953018188 14.7003479003906\\
13.7088947296143 14.7003479003906\\
13.808895111084 14.7003479003906\\
13.9088954925537 14.7003479003906\\
14.0088949203491 14.7003479003906\\
14.1088953018188 14.7003479003906\\
14.2088947296143 14.7003479003906\\
14.308895111084 14.7003479003906\\
14.4088954925537 14.7003479003906\\
14.5088949203491 14.7003479003906\\
14.6088953018188 14.7003479003906\\
14.7088947296143 14.7003479003906\\
14.808895111084 14.7003479003906\\
14.9088954925537 14.7003479003906\\
15.0088949203491 14.7003479003906\\
15.1088953018188 14.7003479003906\\
15.2088947296143 14.7003479003906\\
15.308895111084 14.7003479003906\\
15.4088954925537 14.7003479003906\\
15.5088949203491 14.7003479003906\\
15.6088953018188 14.7003479003906\\
15.7088947296143 14.7003479003906\\
15.808895111084 14.7003479003906\\
15.9088954925537 14.7003479003906\\
16.0088958740234 14.7003479003906\\
16.1088943481445 14.7003479003906\\
16.2088947296143 14.7003479003906\\
16.308895111084 14.7003479003906\\
16.4088954925537 14.7003479003906\\
16.5088958740234 14.7003479003906\\
16.6088943481445 14.7003479003906\\
16.7088947296143 14.7003479003906\\
16.808895111084 14.7003479003906\\
16.9088954925537 14.7003479003906\\
17.0088958740234 14.7003479003906\\
17.1088943481445 14.7003479003906\\
17.2088947296143 14.7003479003906\\
17.308895111084 14.7003479003906\\
17.4088954925537 14.7003479003906\\
17.5088958740234 14.7003479003906\\
17.6088943481445 14.7003479003906\\
17.7088947296143 14.7003479003906\\
17.808895111084 14.7003479003906\\
17.9088954925537 14.7003479003906\\
18.0088958740234 14.7003479003906\\
18.1088943481445 14.7003479003906\\
18.2088947296143 14.7003479003906\\
18.308895111084 14.7003469467163\\
18.4088954925537 14.7003479003906\\
18.5088958740234 14.700345993042\\
18.6088943481445 14.700345993042\\
18.7088947296143 14.7003479003906\\
18.808895111084 14.7003479003906\\
18.9088954925537 14.7003488540649\\
19.0088958740234 14.7003479003906\\
19.1088943481445 14.7003479003906\\
19.2088947296143 14.7003479003906\\
19.308895111084 14.7003479003906\\
19.4088954925537 14.7003479003906\\
19.5088958740234 14.7003479003906\\
19.6088943481445 14.7003479003906\\
19.7088947296143 14.7003479003906\\
19.808895111084 14.7003479003906\\
19.9088954925537 14.7003479003906\\
20.0088958740234 14.7003479003906\\
20.1088943481445 14.7003479003906\\
20.2088947296143 14.7003479003906\\
20.308895111084 14.7003479003906\\
20.4088954925537 14.7003479003906\\
20.5088958740234 14.7003479003906\\
20.6088943481445 14.7003479003906\\
20.7088947296143 14.7003479003906\\
20.808895111084 14.7003479003906\\
20.9088954925537 14.7003479003906\\
21.0088958740234 14.7003479003906\\
21.1088943481445 14.7003479003906\\
21.2088947296143 14.7003479003906\\
21.308895111084 14.7003479003906\\
21.4088954925537 14.7003479003906\\
21.5088958740234 14.7003479003906\\
21.6088943481445 14.7003479003906\\
21.7088947296143 14.7003479003906\\
21.808895111084 14.7003479003906\\
21.9088954925537 14.7003479003906\\
22.0088958740234 14.7003479003906\\
22.1088943481445 14.7003479003906\\
22.2088947296143 14.7003479003906\\
22.308895111084 14.7003479003906\\
22.4088954925537 14.7003479003906\\
22.5088958740234 14.7003479003906\\
22.6088943481445 14.7003479003906\\
22.7088947296143 14.7003479003906\\
22.808895111084 14.7003479003906\\
22.9088954925537 14.7003479003906\\
23.0088958740234 14.7003479003906\\
23.1088943481445 14.7003479003906\\
23.2088947296143 14.7003479003906\\
23.308895111084 14.7003479003906\\
23.4088954925537 14.7003479003906\\
23.5088958740234 14.7003479003906\\
23.6088943481445 14.7003479003906\\
23.7088947296143 14.7003479003906\\
23.808895111084 14.7003479003906\\
23.9088954925537 14.7003479003906\\
24.0088958740234 14.7003479003906\\
24.1088943481445 14.7003479003906\\
24.2088947296143 14.7003479003906\\
24.308895111084 14.7003479003906\\
24.4088954925537 14.7003479003906\\
24.5088958740234 14.7003479003906\\
24.6048831939697 14.7003479003906\\
24.681676864624 14.7003479003906\\
24.7464084625244 14.7003479003906\\
24.8176136016846 14.7003479003906\\
24.895938873291 14.7003479003906\\
24.9820976257324 14.7003479003906\\
25.0768718719482 14.7003479003906\\
25.176872253418 14.7003479003906\\
25.2768707275391 14.7003479003906\\
25.3768711090088 14.7003479003906\\
25.4768714904785 14.7003479003906\\
25.5768718719482 14.7003479003906\\
25.676872253418 14.7003479003906\\
25.7768707275391 14.7003479003906\\
25.8768711090088 14.7003479003906\\
25.9768714904785 14.7003479003906\\
26.0768718719482 14.7003479003906\\
26.176872253418 14.7003479003906\\
26.2768707275391 14.7003479003906\\
26.3768711090088 14.7003479003906\\
26.4768714904785 14.7003479003906\\
26.5768718719482 14.7003479003906\\
26.676872253418 14.7003479003906\\
26.7768707275391 14.7003479003906\\
26.8768711090088 14.7003479003906\\
26.9768714904785 14.7003479003906\\
27.0768718719482 14.7003479003906\\
27.176872253418 14.7003479003906\\
27.2768707275391 14.7003479003906\\
27.3768711090088 14.7003479003906\\
27.4768714904785 14.7003479003906\\
27.5768718719482 14.7003479003906\\
27.676872253418 14.7003479003906\\
27.7768707275391 14.7003479003906\\
27.8768711090088 14.7003479003906\\
27.9768714904785 14.7003479003906\\
28.0768718719482 14.7003479003906\\
28.176872253418 14.7003479003906\\
28.2768707275391 14.7003479003906\\
28.3768711090088 14.7003479003906\\
28.4768714904785 14.7003479003906\\
28.5768718719482 14.7003479003906\\
28.676872253418 14.7003479003906\\
28.7768707275391 14.7003479003906\\
28.8768711090088 14.7003479003906\\
28.9768714904785 14.7003479003906\\
29.0768718719482 14.7003479003906\\
29.176872253418 14.7003479003906\\
29.2768707275391 14.7003479003906\\
29.3768711090088 14.7003479003906\\
29.4768714904785 14.7003479003906\\
29.5768718719482 14.7003479003906\\
29.676872253418 14.7003479003906\\
29.7768707275391 14.7003479003906\\
29.8768711090088 14.7003479003906\\
29.9768714904785 14.7003479003906\\
30.0768718719482 14.7003479003906\\
30.176872253418 14.7003479003906\\
30.2768707275391 14.7003479003906\\
30.3768711090088 14.7003479003906\\
30.4768714904785 14.7003469467163\\
30.5768718719482 14.7003469467163\\
30.676872253418 14.7003469467163\\
30.7768707275391 14.700345993042\\
30.8768711090088 14.7003469467163\\
30.9768714904785 14.7003479003906\\
31.0768718719482 14.7003488540649\\
31.176872253418 14.7003479003906\\
31.2768707275391 14.7003479003906\\
31.3768711090088 14.7003479003906\\
31.4768714904785 14.7003479003906\\
31.5768718719482 14.7003479003906\\
31.676872253418 14.7003479003906\\
31.7768707275391 14.7003479003906\\
31.8768711090088 14.7003479003906\\
31.9768714904785 14.7003479003906\\
32.0768699645996 14.7003479003906\\
32.176872253418 14.7003479003906\\
32.2768707275391 14.7003479003906\\
32.3768730163574 14.7003479003906\\
32.4768714904785 14.7003479003906\\
32.5768699645996 14.7003479003906\\
32.676872253418 14.7003479003906\\
32.7768707275391 14.7003479003906\\
32.8768730163574 14.7003479003906\\
32.9768714904785 14.7003479003906\\
33.0768699645996 14.7003479003906\\
33.176872253418 14.7003479003906\\
33.2768707275391 14.7003479003906\\
33.3768730163574 14.7003479003906\\
33.4768714904785 14.7003479003906\\
33.5768699645996 14.7003479003906\\
33.676872253418 14.7003479003906\\
33.7768707275391 14.7003479003906\\
33.8768730163574 14.7003479003906\\
33.9768714904785 14.7003479003906\\
34.0768699645996 14.7003479003906\\
34.176872253418 14.7003479003906\\
34.2768707275391 14.7003479003906\\
34.3768730163574 14.7003479003906\\
34.4768714904785 14.7003479003906\\
34.5768699645996 14.7003479003906\\
34.676872253418 14.7003479003906\\
34.7768707275391 14.7003479003906\\
34.8768730163574 14.7003479003906\\
34.9768714904785 14.7003479003906\\
35.0768699645996 14.7003479003906\\
35.176872253418 14.7003479003906\\
35.2768707275391 14.7003479003906\\
35.3768730163574 14.7003479003906\\
35.4768714904785 14.7003479003906\\
35.5768699645996 14.7003479003906\\
35.676872253418 14.7003479003906\\
35.7768707275391 14.7003479003906\\
35.8768730163574 14.7003479003906\\
35.9768714904785 14.7003479003906\\
36.0768699645996 14.7003479003906\\
36.176872253418 14.7003479003906\\
36.2768707275391 14.7003479003906\\
36.3768730163574 14.7003479003906\\
36.4768714904785 14.7003479003906\\
36.5768699645996 14.7003479003906\\
36.676872253418 14.7003479003906\\
36.7768707275391 14.7003479003906\\
36.8673248291016 14.7003479003906\\
36.9396858215332 14.7003479003906\\
37.0192794799805 14.7003479003906\\
37.1068305969238 14.7003469467163\\
37.2031402587891 14.7003479003906\\
37.3031425476074 14.7003479003906\\
37.4031410217285 14.7003479003906\\
37.5031433105469 14.7003479003906\\
37.603141784668 14.7003479003906\\
37.7031402587891 14.7003479003906\\
37.8031425476074 14.7003479003906\\
37.9031410217285 14.7003479003906\\
38.0031433105469 14.7003479003906\\
38.103141784668 14.7003479003906\\
38.2031402587891 14.7003479003906\\
38.3031425476074 14.7003479003906\\
38.4031410217285 14.7003479003906\\
38.5031433105469 14.7003479003906\\
38.603141784668 14.7003479003906\\
38.7031402587891 14.7003479003906\\
38.8031425476074 14.7003479003906\\
38.9031410217285 14.7003479003906\\
39.0031433105469 14.7003479003906\\
39.103141784668 14.7003479003906\\
39.2031402587891 14.7003479003906\\
39.3031425476074 14.7003479003906\\
39.4031410217285 14.7003479003906\\
39.5031433105469 14.7003479003906\\
39.603141784668 14.7003479003906\\
39.7031402587891 14.7003479003906\\
39.8031425476074 14.7003479003906\\
39.9031410217285 14.7003479003906\\
40.0031433105469 14.7003479003906\\
40.103141784668 14.7003479003906\\
40.2031402587891 14.7003479003906\\
40.3031425476074 14.7003479003906\\
40.4031410217285 14.7003479003906\\
40.5031433105469 14.7003479003906\\
40.603141784668 14.7003479003906\\
40.7031402587891 14.7003479003906\\
40.8031425476074 14.7003479003906\\
40.9031410217285 14.7003479003906\\
41.0031433105469 14.7003479003906\\
41.103141784668 14.7003479003906\\
41.2031402587891 14.7003479003906\\
41.3031425476074 14.7003479003906\\
41.4031410217285 14.7003479003906\\
41.5031433105469 14.7003479003906\\
41.603141784668 14.7003479003906\\
41.7031402587891 14.7003479003906\\
41.8031425476074 14.7003479003906\\
41.9031410217285 14.7003479003906\\
42.0031433105469 14.7003479003906\\
42.103141784668 14.7003479003906\\
42.2031402587891 14.7003479003906\\
42.3031425476074 14.7003479003906\\
42.4031410217285 14.7003479003906\\
42.5031433105469 14.7003479003906\\
42.603141784668 14.7003479003906\\
42.7031402587891 14.7003479003906\\
42.8031425476074 14.7003469467163\\
42.9031410217285 14.7003469467163\\
43.0031433105469 14.700345993042\\
43.103141784668 14.700345993042\\
43.2031402587891 14.7003479003906\\
43.3031425476074 14.7003479003906\\
43.4031410217285 14.7003488540649\\
43.5031433105469 14.7003479003906\\
43.603141784668 14.7003479003906\\
43.7031402587891 14.7003479003906\\
43.8031425476074 14.7003479003906\\
43.9031410217285 14.7003479003906\\
44.0031433105469 14.7003479003906\\
44.103141784668 14.7003479003906\\
44.2031402587891 14.7003479003906\\
44.3031425476074 14.7003479003906\\
44.4031410217285 14.7003479003906\\
44.5031433105469 14.7003479003906\\
44.603141784668 14.7003479003906\\
44.7031402587891 14.7003479003906\\
44.8031425476074 14.7003479003906\\
44.9031410217285 14.7003479003906\\
45.0031433105469 14.7003479003906\\
45.103141784668 14.7003479003906\\
45.2031402587891 14.7003479003906\\
45.3031425476074 14.7003479003906\\
45.4031410217285 14.7003479003906\\
45.5031433105469 14.7003479003906\\
45.603141784668 14.7003479003906\\
45.7031402587891 14.7003479003906\\
45.8031425476074 14.7003479003906\\
45.9031410217285 14.7003479003906\\
46.0031433105469 14.7003479003906\\
46.103141784668 14.7003479003906\\
46.2031402587891 14.7003479003906\\
46.3031425476074 14.7003479003906\\
46.4031410217285 14.7003479003906\\
46.5031433105469 14.7003479003906\\
46.603141784668 14.7003479003906\\
46.7031402587891 14.7003479003906\\
46.8031425476074 14.7003479003906\\
46.9031410217285 14.7003479003906\\
47.0031433105469 14.7003479003906\\
47.103141784668 14.7003479003906\\
47.2031402587891 14.7003479003906\\
47.3031425476074 14.7003479003906\\
47.4031410217285 14.7003479003906\\
47.5031433105469 14.7003479003906\\
47.603141784668 14.7003479003906\\
47.7031402587891 14.7003479003906\\
47.8031425476074 14.7003479003906\\
47.9031410217285 14.7003479003906\\
48.0031433105469 14.7003479003906\\
48.103141784668 14.7003479003906\\
48.2031402587891 14.7003479003906\\
48.3031425476074 14.7003479003906\\
48.4031410217285 14.7003479003906\\
48.5031433105469 14.7003479003906\\
48.603141784668 14.7003479003906\\
48.7031402587891 14.7003479003906\\
48.8031425476074 14.7003479003906\\
48.9031410217285 14.7003479003906\\
49.0031433105469 14.7003479003906\\
49.098030090332 14.7003479003906\\
49.1739387512207 14.7003479003906\\
49.2346649169922 14.7003488540649\\
49.3014602661133 14.7003469467163\\
49.3450546264648 14.7003488540649\\
49.3930053710938 14.7003421783447\\
49.4215202331543 14.7003555297852\\
49.4451713562012 14.7003469467163\\
49.47119140625 14.7003498077393\\
49.4998092651367 14.7003364562988\\
49.515323638916 14.7003660202026\\
49.5276222229004 14.700343132019\\
49.5411529541016 14.7003602981567\\
49.5560340881348 14.7002239227295\\
49.5575561523438 14.7010269165039\\
49.5592269897461 14.7008752822876\\
49.5610656738281 14.7004156112671\\
49.5630912780762 14.7002944946289\\
49.565315246582 14.7003526687622\\
49.5677642822266 14.7003765106201\\
49.5704574584961 14.7003688812256\\
49.5734214782715 14.7003612518311\\
49.5766792297363 14.7003574371338\\
49.5791168212891 14.7003555297852\\
49.5817947387695 14.7003536224365\\
49.5847434997559 14.7003517150879\\
49.5879859924316 14.7003507614136\\
49.591552734375 14.7003498077393\\
49.5954742431641 14.7003488540649\\
49.5997886657715 14.7003488540649\\
49.6045341491699 14.7003479003906\\
49.6097564697266 14.7003479003906\\
49.6155014038086 14.7003479003906\\
49.6218185424805 14.7003469467163\\
49.6287651062012 14.7003469467163\\
49.6364097595215 14.7003469467163\\
49.6448211669922 14.7003469467163\\
49.6540679931641 14.7003469467163\\
49.6642456054688 14.7003469467163\\
49.6754379272461 14.7003469467163\\
49.6877479553223 14.7003479003906\\
49.7012901306152 14.7003479003906\\
49.7161865234375 14.7003479003906\\
49.7325744628906 14.7003479003906\\
49.7505989074707 14.7003479003906\\
49.7704238891602 14.7003479003906\\
49.792236328125 14.7003479003906\\
49.8162269592285 14.7003479003906\\
49.8426170349121 14.7003479003906\\
49.8716468811035 14.7003479003906\\
49.9035758972168 14.7003469467163\\
49.9387016296387 14.7003469467163\\
49.9773406982422 14.7003469467163\\
50.0198440551758 14.700345993042\\
50.066593170166 14.700345993042\\
50.1180229187012 14.700345993042\\
50.1745910644531 14.700345993042\\
50.23681640625 14.7003450393677\\
50.3052673339844 14.7003479003906\\
50.3805618286133 14.7003469467163\\
50.4633865356445 14.7003488540649\\
50.5544929504395 14.7003469467163\\
50.6544914245605 14.7003479003906\\
50.7544898986816 14.7003469467163\\
50.8544921875 14.7003479003906\\
50.9544906616211 14.7003469467163\\
51.0544929504395 14.7003479003906\\
51.1544914245605 14.7003469467163\\
51.2544898986816 14.7003479003906\\
51.3544921875 14.7003469467163\\
51.3568458557129 14.7002773284912\\
51.359432220459 14.7004432678223\\
51.3622817993164 14.700343132019\\
51.3654136657715 14.7003288269043\\
51.3688583374023 14.700345993042\\
51.3726463317871 14.7003507614136\\
51.3768157958984 14.7003498077393\\
51.3814010620117 14.7003488540649\\
51.3864479064941 14.7003488540649\\
51.3919944763184 14.7003479003906\\
51.3980979919434 14.7003479003906\\
51.4048118591309 14.7003479003906\\
51.4121971130371 14.7003479003906\\
51.4203224182129 14.7003479003906\\
51.4292602539063 14.7003479003906\\
51.4390869140625 14.7003479003906\\
51.4499015808105 14.7003479003906\\
51.4617958068848 14.7003479003906\\
51.4748802185059 14.7003479003906\\
51.4892692565918 14.7003479003906\\
51.5051002502441 14.7003479003906\\
51.522518157959 14.7003479003906\\
51.5416717529297 14.7003479003906\\
51.562744140625 14.7003479003906\\
51.5859222412109 14.7003479003906\\
51.6114196777344 14.7003479003906\\
51.6394653320313 14.7003479003906\\
51.6703147888184 14.7003479003906\\
51.7042503356934 14.7003479003906\\
51.7415771484375 14.7003479003906\\
51.7826385498047 14.7003479003906\\
51.827808380127 14.7003479003906\\
51.8774948120117 14.7003479003906\\
51.9321479797363 14.7003479003906\\
51.9922637939453 14.7003479003906\\
52.0583953857422 14.7003479003906\\
52.1311378479004 14.7003479003906\\
52.2111587524414 14.7003479003906\\
52.2991790771484 14.7003469467163\\
52.3959999084473 14.7003479003906\\
52.4959983825684 14.7003469467163\\
52.5960006713867 14.7003707885742\\
52.6160011291504 14.7004060745239\\
52.6228942871094 14.7004947662354\\
52.6304740905762 14.7005338668823\\
52.637622833252 14.7005567550659\\
52.6454811096191 14.7005786895752\\
52.6532897949219 14.7006063461304\\
52.6618766784668 14.7006349563599\\
52.6713256835938 14.7006368637085\\
52.6817169189453 14.7005996704102\\
52.6931457519531 14.7005710601807\\
52.7057189941406 14.7005233764648\\
52.7195510864258 14.700234413147\\
52.7347640991211 14.6992664337158\\
52.7514991760254 14.7002639770508\\
52.7699089050293 14.7004365921021\\
52.7901573181152 14.7004327774048\\
52.8124351501465 14.7004852294922\\
52.8369369506836 14.7003126144409\\
52.8638877868652 14.7002782821655\\
52.8935356140137 14.7002391815186\\
52.9261474609375 14.7002658843994\\
52.9620208740234 14.7002820968628\\
53.0014839172363 14.7002935409546\\
53.0448913574219 14.7003383636475\\
53.0926399230957 14.7003927230835\\
53.1261940002441 14.7003040313721\\
53.1631088256836 14.7003421783447\\
53.1971549987793 14.7003364562988\\
53.2346038818359 14.7003383636475\\
53.2757987976074 14.7003440856934\\
53.3211135864258 14.700345993042\\
53.3709602355957 14.7003469467163\\
53.4181785583496 14.7003469467163\\
53.4612884521484 14.7003479003906\\
53.4995498657227 14.7003469467163\\
53.5327682495117 14.7003479003906\\
53.5693130493164 14.7003479003906\\
53.6095085144043 14.7003469467163\\
53.6537246704102 14.7003479003906\\
53.7023620605469 14.7003479003906\\
53.7558631896973 14.7003479003906\\
53.8147125244141 14.7003479003906\\
53.8794479370117 14.7003479003906\\
53.9506568908691 14.7003479003906\\
54.0289878845215 14.7003479003906\\
54.1151542663574 14.7003479003906\\
54.2099342346191 14.7003469467163\\
54.3099327087402 14.7003479003906\\
54.4099311828613 14.7003469467163\\
54.5099334716797 14.7003488540649\\
54.6099319458008 14.7003469467163\\
54.6148452758789 14.700343132019\\
54.6202507019043 14.7003555297852\\
54.6261940002441 14.700343132019\\
54.6327362060547 14.7003517150879\\
54.6399269104004 14.7003450393677\\
54.6478385925293 14.7003507614136\\
54.6565437316895 14.7003450393677\\
54.6661186218262 14.7003498077393\\
54.6766510009766 14.700345993042\\
54.6824378967285 14.7003488540649\\
54.6888046264648 14.7003498077393\\
54.6958045959473 14.7003488540649\\
54.7035064697266 14.7003488540649\\
54.7119827270508 14.7003488540649\\
54.7213020324707 14.7003488540649\\
54.7315559387207 14.7003488540649\\
54.7428321838379 14.7003479003906\\
54.7552375793457 14.7003479003906\\
54.7688827514648 14.7003479003906\\
54.7838935852051 14.7003479003906\\
54.800407409668 14.7003479003906\\
54.8185691833496 14.7003479003906\\
54.8385467529297 14.7003479003906\\
54.8605270385742 14.7003479003906\\
54.8847007751465 14.7003479003906\\
54.9112930297852 14.7003479003906\\
54.940544128418 14.7003479003906\\
54.9727210998535 14.7003479003906\\
55.008113861084 14.7003479003906\\
55.0470504760742 14.7003469467163\\
55.0898742675781 14.7003469467163\\
55.1369857788086 14.700345993042\\
55.1888084411621 14.7003450393677\\
55.2458114624023 14.7003479003906\\
55.3085136413574 14.700345993042\\
55.3774871826172 14.700345993042\\
55.4533576965332 14.7003479003906\\
55.5368156433105 14.7003479003906\\
55.6286201477051 14.7003479003906\\
55.7286186218262 14.7003479003906\\
55.8286209106445 14.7003479003906\\
55.9286193847656 14.7003479003906\\
56.0286178588867 14.7003479003906\\
56.1286201477051 14.7003479003906\\
56.2286186218262 14.7003479003906\\
56.3286209106445 14.7003479003906\\
56.4286193847656 14.7003479003906\\
56.5286178588867 14.7003479003906\\
56.6286201477051 14.7003479003906\\
56.7286186218262 14.7003479003906\\
56.8286209106445 14.7003479003906\\
56.9286193847656 14.7003479003906\\
57.0286178588867 14.7003479003906\\
57.1286201477051 14.7003479003906\\
57.2286186218262 14.7003479003906\\
57.3286209106445 14.7003479003906\\
57.4286193847656 14.7003479003906\\
57.5286178588867 14.7003479003906\\
57.6286201477051 14.7003479003906\\
57.7286186218262 14.7003479003906\\
57.8286209106445 14.7003479003906\\
57.9286193847656 14.7003479003906\\
58.0286178588867 14.7003469467163\\
58.1286201477051 14.7003479003906\\
58.2286186218262 14.7003479003906\\
58.3286209106445 14.7003479003906\\
58.4286193847656 14.7003479003906\\
58.5286178588867 14.7003479003906\\
58.6286201477051 14.7003479003906\\
58.7286186218262 14.7003479003906\\
58.8286209106445 14.7003479003906\\
58.9286193847656 14.7003479003906\\
59.0286178588867 14.7003479003906\\
59.1286201477051 14.7003479003906\\
59.2286186218262 14.7003479003906\\
59.3286209106445 14.7003479003906\\
59.4286193847656 14.7003479003906\\
59.5286178588867 14.7003479003906\\
59.6286201477051 14.7003479003906\\
59.7286186218262 14.7003479003906\\
59.8286209106445 14.7003479003906\\
59.9143104553223 14.700345993042\\
60 14.7003488540649\\
60.0942611694336 14.7003479003906\\
60.1942596435547 14.7003479003906\\
60.2942581176758 14.7003479003906\\
60.3942604064941 14.7003479003906\\
60.4942588806152 14.7003479003906\\
60.5942611694336 14.7003479003906\\
60.6942596435547 14.7003479003906\\
60.7942581176758 14.7003479003906\\
60.8942604064941 14.7003479003906\\
60.9942588806152 14.7003479003906\\
61.0942611694336 14.7003479003906\\
61.1942596435547 14.7003479003906\\
61.2942581176758 14.7003479003906\\
61.3942604064941 14.7003479003906\\
61.4835205078125 14.7003479003906\\
61.5549278259277 14.7003479003906\\
61.6334800720215 14.7003479003906\\
61.7198829650879 14.7003479003906\\
61.8149299621582 14.7003479003906\\
61.8549957275391 14.7003479003906\\
61.8990707397461 14.7003488540649\\
61.9192581176758 14.7003488540649\\
61.9358787536621 14.7003517150879\\
61.9541664123535 14.7003507614136\\
61.9742813110352 14.7003498077393\\
61.9964065551758 14.7003498077393\\
62.0207443237305 14.7003498077393\\
62.0475158691406 14.7003498077393\\
62.0769653320313 14.7003498077393\\
62.1093559265137 14.7003498077393\\
62.1449890136719 14.7003498077393\\
62.1841850280762 14.7003498077393\\
62.2273025512695 14.7003498077393\\
62.274730682373 14.7003498077393\\
62.3269004821777 14.7003498077393\\
62.3842887878418 14.7003498077393\\
62.4474143981934 14.7003498077393\\
62.5168533325195 14.7003498077393\\
62.5932350158691 14.7003498077393\\
62.6772575378418 14.7003488540649\\
62.7696800231934 14.7003498077393\\
62.8696823120117 14.7003488540649\\
62.9696807861328 14.7003612518311\\
63.0696792602539 14.70046043396\\
63.1196823120117 14.7011070251465\\
63.1572036743164 14.7501459121704\\
63.1601943969727 14.8082304000854\\
63.1634902954102 14.8275775909424\\
63.1671104431152 14.8587884902954\\
63.1710968017578 14.8385429382324\\
63.175479888916 14.81520652771\\
63.1803016662598 14.809139251709\\
63.185604095459 14.7970771789551\\
63.1914367675781 14.7816677093506\\
63.1978569030762 14.7636184692383\\
63.2049140930176 14.7450733184814\\
63.2126808166504 14.7287282943726\\
63.2212219238281 14.7167320251465\\
63.2306175231934 14.7091121673584\\
63.2409515380859 14.7047214508057\\
63.2523231506348 14.7030181884766\\
63.2648277282715 14.7023553848267\\
63.2785835266113 14.7026662826538\\
63.2937164306641 14.7052640914917\\
63.310359954834 14.7136421203613\\
63.328670501709 14.727481842041\\
63.3488121032715 14.7294492721558\\
63.3709678649902 14.7086715698242\\
63.395336151123 14.7093486785889\\
63.4221420288086 14.7154245376587\\
63.4516296386719 14.7043342590332\\
63.4840698242188 14.7007217407227\\
63.4999580383301 14.700325012207\\
63.5174407958984 14.7003307342529\\
63.5366668701172 14.7003364562988\\
63.5578193664551 14.7003393173218\\
63.5810852050781 14.700345993042\\
63.6066780090332 14.7003650665283\\
63.6348304748535 14.700366973877\\
63.6657981872559 14.7003660202026\\
63.6998634338379 14.7003622055054\\
63.7373313903809 14.7003612518311\\
63.7785491943359 14.7003479003906\\
63.8238906860352 14.7003479003906\\
63.8737640380859 14.7003479003906\\
63.9286231994629 14.7003488540649\\
63.9762001037598 14.7003479003906\\
64.0218276977539 14.7003479003906\\
64.0647964477539 14.7003469467163\\
64.112060546875 14.7003469467163\\
64.1640548706055 14.7003469467163\\
64.2212448120117 14.7003479003906\\
64.2841567993164 14.7003488540649\\
64.3533554077148 14.7003488540649\\
64.4294738769531 14.7003488540649\\
64.5132064819336 14.7003488540649\\
64.6053161621094 14.7003488540649\\
64.7053146362305 14.7003498077393\\
64.8053131103516 14.7003488540649\\
64.9053115844727 14.7003498077393\\
65.0053176879883 14.7003488540649\\
65.0079345703125 14.7003269195557\\
65.0108184814453 14.7003898620605\\
65.0139923095703 14.7002935409546\\
65.0165023803711 14.7003793716431\\
65.0192642211914 14.7003622055054\\
65.0223007202148 14.7003545761108\\
65.025634765625 14.7003488540649\\
65.0293121337891 14.7003498077393\\
65.0333557128906 14.7003498077393\\
65.0377960205078 14.7003498077393\\
65.0426864624023 14.7003498077393\\
65.0480651855469 14.7003498077393\\
65.0539855957031 14.7003498077393\\
65.0604934692383 14.7003498077393\\
65.0676498413086 14.7003498077393\\
65.0755233764648 14.7003498077393\\
65.0841903686523 14.7003498077393\\
65.0937194824219 14.7003498077393\\
65.1042022705078 14.7003498077393\\
65.1157302856445 14.7003498077393\\
65.1284103393555 14.7003498077393\\
65.1423645019531 14.7003488540649\\
65.1577072143555 14.7003488540649\\
65.1745910644531 14.7003488540649\\
65.1931610107422 14.7003488540649\\
65.2135848999023 14.7003488540649\\
65.2360534667969 14.7003488540649\\
65.2607727050781 14.7003488540649\\
65.287956237793 14.7003488540649\\
65.3178634643555 14.7003488540649\\
65.3507614135742 14.7003488540649\\
65.3869476318359 14.7003488540649\\
65.4267501831055 14.7003488540649\\
65.4705352783203 14.7003488540649\\
65.5186996459961 14.7003488540649\\
65.5716781616211 14.7003488540649\\
65.6299591064453 14.7003488540649\\
65.6940612792969 14.7003488540649\\
65.7645797729492 14.7003488540649\\
65.8421478271484 14.7003488540649\\
65.9274749755859 14.7003488540649\\
66.0213317871094 14.7003479003906\\
66.1213302612305 14.7003498077393\\
66.2213287353516 14.7003488540649\\
66.3213272094727 14.700382232666\\
66.3413314819336 14.7005805969238\\
66.3490600585938 14.7007627487183\\
66.3565673828125 14.7008609771729\\
66.3648223876953 14.7009506225586\\
66.3727111816406 14.702073097229\\
66.3813858032227 14.7257041931152\\
66.3909301757813 14.8084630966187\\
66.4014358520508 14.8818635940552\\
66.4129867553711 14.861626625061\\
66.4256896972656 14.791332244873\\
66.4396667480469 14.7891397476196\\
66.4550399780273 14.7783813476563\\
66.4719543457031 14.7659130096436\\
66.4905548095703 14.753342628479\\
66.5110244750977 14.7273330688477\\
66.5335311889648 14.7030553817749\\
66.5504302978516 14.7008428573608\\
66.5690307617188 14.7004652023315\\
66.5894775390625 14.7029008865356\\
66.6119766235352 14.7356910705566\\
66.6367263793945 14.7598648071289\\
66.663948059082 14.774320602417\\
66.6938934326172 14.7670192718506\\
66.7268295288086 14.7647190093994\\
66.7630615234375 14.7602691650391\\
66.8029174804688 14.7569589614868\\
66.8467636108398 14.7471675872803\\
66.8949890136719 14.7358732223511\\
66.9480361938477 14.7454681396484\\
67.0063934326172 14.7260799407959\\
67.0705795288086 14.7364339828491\\
67.111457824707 14.7149486541748\\
67.1564254760742 14.7174549102783\\
67.1973037719727 14.7130289077759\\
67.2330703735352 14.7074823379517\\
67.2724151611328 14.7050952911377\\
67.3156967163086 14.7045469284058\\
67.3633117675781 14.7049865722656\\
67.4156799316406 14.7060270309448\\
67.4732894897461 14.7079210281372\\
67.5366516113281 14.7116451263428\\
67.6063613891602 14.7174367904663\\
67.6830368041992 14.7247123718262\\
67.7673797607422 14.7278480529785\\
67.8601531982422 14.7306604385376\\
67.9601516723633 14.7320785522461\\
68.0601577758789 14.7326364517212\\
68.16015625 14.7323799133301\\
68.2601547241211 14.7326145172119\\
68.3601531982422 14.7331008911133\\
68.4601516723633 14.7337217330933\\
68.5601577758789 14.7341613769531\\
68.66015625 14.7347240447998\\
68.7601547241211 14.7352075576782\\
68.8601531982422 14.7357444763184\\
68.9601516723633 14.7362461090088\\
69.0601577758789 14.7367715835571\\
69.16015625 14.7372827529907\\
69.2601547241211 14.7378034591675\\
69.3601531982422 14.738317489624\\
69.4601516723633 14.7388362884521\\
69.5601577758789 14.7393522262573\\
69.66015625 14.7398710250854\\
69.7601547241211 14.7403879165649\\
69.8601531982422 14.7409067153931\\
69.9601516723633 14.7414245605469\\
70.0601577758789 14.7419424057007\\
70.16015625 14.7424612045288\\
70.2601547241211 14.7429828643799\\
70.3601531982422 14.7435598373413\\
70.4601516723633 14.7443885803223\\
70.5601577758789 14.7449607849121\\
70.66015625 14.7456026077271\\
70.7601547241211 14.7462224960327\\
70.8601531982422 14.7468461990356\\
70.9601516723633 14.7474679946899\\
71.0601577758789 14.7480878829956\\
71.16015625 14.748706817627\\
71.2601547241211 14.7493238449097\\
71.3601531982422 14.7499399185181\\
71.4601516723633 14.7505550384521\\
71.5601577758789 14.7511682510376\\
71.66015625 14.751781463623\\
71.7601547241211 14.7523927688599\\
71.8601531982422 14.753002166748\\
71.9601516723633 14.753399848938\\
72.0601577758789 14.7538452148438\\
72.16015625 14.7543783187866\\
72.2601547241211 14.7548751831055\\
72.3601531982422 14.7553758621216\\
72.4601516723633 14.7558660507202\\
72.5601577758789 14.7563505172729\\
72.66015625 14.7568244934082\\
72.7601547241211 14.757287979126\\
72.8601531982422 14.7577381134033\\
72.9601516723633 14.7581739425659\\
73.0601577758789 14.7585926055908\\
73.16015625 14.7624711990356\\
73.2601547241211 14.7617177963257\\
73.3601531982422 14.761568069458\\
73.4601516723633 14.7623281478882\\
73.5601577758789 14.762903213501\\
73.66015625 14.7636156082153\\
73.7601547241211 14.764594078064\\
73.8601531982422 14.7717514038086\\
73.943000793457 14.7758617401123\\
74.00927734375 14.7706327438354\\
74.0821685791016 14.7681941986084\\
74.162353515625 14.7685651779175\\
74.2505569458008 14.7857704162598\\
74.2926559448242 14.7348537445068\\
74.3389587402344 14.7640619277954\\
74.3595275878906 14.7772178649902\\
74.3765106201172 14.7796487808228\\
74.3951873779297 14.7709054946899\\
74.4157409667969 14.770956993103\\
74.438346862793 14.7711029052734\\
74.4632110595703 14.7712745666504\\
74.4905624389648 14.7714853286743\\
74.5206527709961 14.7717370986938\\
74.5537414550781 14.7720317840576\\
74.5901489257813 14.7723712921143\\
74.6301956176758 14.7727584838867\\
74.6742477416992 14.7731943130493\\
74.7227020263672 14.7736806869507\\
74.7760009765625 14.7742214202881\\
74.8346328735352 14.7748184204102\\
74.8991317749023 14.7754764556885\\
74.9495620727539 14.7762012481689\\
75 14.7767648696899\\
};
\addlegendentry{22 "};

\addplot [
color=mycolor1,
dotted,
forget plot
]
table[row sep=crcr]{
37.3019608320264 14.0\\
37.3019608320264 18.0\\
};
\addplot [
color=mycolor1,
dotted,
forget plot
]
table[row sep=crcr]{
49.5692811093685 14.0\\
49.5692811093685 18.0\\
};
\end{axis}
\end{tikzpicture}%
\caption{Filling problem with the bottom channel in the nonlinear domain.}
\label{fig:vmpDDBotChan}
\end{figure}

While the water front is in the nonlinear domain, the pressures everywhere are in alignment with the analytic solution.
However, once the water front enters the linear domain, the unresolved nonlinearities in the linear domain induce nonphysical pressure oscillations in both domains.
This result is mirrored in the case where the top channel is in the nonlinear domain, \fig{fig:vmpDDTopChan}.

\begin{figure}[h!tb]
\centering
% This file was created by matlab2tikz v0.4.3.
% Copyright (c) 2008--2013, Nico Schlömer <nico.schloemer@gmail.com>
% All rights reserved.
% 
\tikzsetnextfilename{plots/vmpTopNLN1em1_eps}
\begin{tikzpicture}

\begin{axis}[%
width=0.8\textwidth,
height=0.508815326102268\textwidth,
scale only axis,
xmin=0.0,
xmax=75.0,
xlabel={Time [s]},
ymin=14.5,
ymax=15.8,
ylabel={Pressure [psia]},
legend style={at={(0.03,0.97)},anchor=north west,draw=black,fill=white,legend cell align=left}
]
\addplot [
color=black,
solid
]
table[row sep=crcr]{
0 14.7007818222046\\
9.99999971718069e-10 14.7007818222046\\
2.49999998480632e-09 14.7007818222046\\
4.74999994892755e-09 14.7007818222046\\
8.1250002281763e-09 14.7007818222046\\
1.31874999809156e-08 14.7007818222046\\
2.0781250498203e-08 14.7007818222046\\
3.21718758300449e-08 14.7007818222046\\
4.92578138278077e-08 14.7007818222046\\
7.48867208244519e-08 14.7007818222046\\
1.13330081319418e-07 14.7007818222046\\
1.70995122061868e-07 14.7007818222046\\
2.57492672517401e-07 14.7007818222046\\
3.8723902662241e-07 14.7007818222046\\
5.81858500936505e-07 14.7007818222046\\
8.73787769251066e-07 14.7007818222046\\
1.31168167172291e-06 14.7007818222046\\
1.9685226106958e-06 14.7007827758789\\
2.95378367809462e-06 14.7007827758789\\
4.43167573394021e-06 14.7007827758789\\
6.64851359033491e-06 14.7007827758789\\
9.97376992017962e-06 14.7007837295532\\
1.49616553244414e-05 14.7007846832275\\
2.24434825213393e-05 14.7007856369019\\
3.36662233166862e-05 14.7007865905762\\
5.05003372381907e-05 14.7007884979248\\
7.57515081204474e-05 14.7007904052734\\
0.000113628258986864 14.7007932662964\\
0.000170443381648511 14.700795173645\\
0.000255666091106832 14.7007970809937\\
0.000383500126190484 14.7007970809937\\
0.000575251178815961 14.7007942199707\\
0.000862877757754177 14.7007904052734\\
0.00129431765526533 14.7007865905762\\
0.00194147753063589 14.7007837295532\\
0.00291221728548408 14.700779914856\\
0.00436832662671804 14.7007780075073\\
0.00597004732117057 14.700779914856\\
0.00773194013163447 14.7007827758789\\
0.009670021943748 14.7007837295532\\
0.0118019115179777 14.7007837295532\\
0.0141469910740852 14.7007827758789\\
0.0167265776544809 14.7007827758789\\
0.0195641238242388 14.7007827758789\\
0.0226854234933853 14.7007827758789\\
0.0261188540607691 14.7007827758789\\
0.0298956278711557 14.7007827758789\\
0.0340500771999359 14.7007827758789\\
0.0386199727654457 14.7007827758789\\
0.0436468608677387 14.7007827758789\\
0.0491764321923256 14.7007827758789\\
0.0552589632570744 14.7007827758789\\
0.0619497485458851 14.7007827758789\\
0.0693096145987511 14.7007827758789\\
0.0774054601788521 14.7007827758789\\
0.0863108932971954 14.700779914856\\
0.0961068719625473 14.7007827758789\\
0.106882445514202 14.7007827758789\\
0.118735581636429 14.7007827758789\\
0.13177402317524 14.7007827758789\\
0.146116316318512 14.7007827758789\\
0.161892831325531 14.7007827758789\\
0.179247006773949 14.7007827758789\\
0.198336601257324 14.7007827758789\\
0.21933513879776 14.7007827758789\\
0.242433547973633 14.7007827758789\\
0.267841786146164 14.7007827758789\\
0.29579085111618 14.7007827758789\\
0.32653483748436 14.7007827758789\\
0.360353201627731 14.7007827758789\\
0.397553414106369 14.7007837295532\\
0.438473641872406 14.7007837295532\\
0.483485877513886 14.7007837295532\\
0.532999336719513 14.7007837295532\\
0.587464153766632 14.7007837295532\\
0.647375464439392 14.7007837295532\\
0.71327793598175 14.7007837295532\\
0.7857705950737 14.7007837295532\\
0.865512549877167 14.7007837295532\\
0.953228712081909 14.7007837295532\\
1.04971647262573 14.7007837295532\\
1.14971649646759 14.7007827758789\\
1.24971640110016 14.7007827758789\\
1.34971642494202 14.7007827758789\\
1.44971644878387 14.7007827758789\\
1.54971647262573 14.7007827758789\\
1.64971649646759 14.7007827758789\\
1.74971640110016 14.7007827758789\\
1.84971642494202 14.7007827758789\\
1.94971644878387 14.7007827758789\\
2.04971647262573 14.7007827758789\\
2.1497163772583 14.7007827758789\\
2.24971652030945 14.7007827758789\\
2.34971642494202 14.7007827758789\\
2.44971656799316 14.7007827758789\\
2.54971647262573 14.7007827758789\\
2.6497163772583 14.7007827758789\\
2.74971652030945 14.7007827758789\\
2.84971642494202 14.7007827758789\\
2.94971656799316 14.7007827758789\\
3.04971647262573 14.7007827758789\\
3.1497163772583 14.7007827758789\\
3.24971652030945 14.7007827758789\\
3.34971642494202 14.7007827758789\\
3.44971656799316 14.7007827758789\\
3.54971647262573 14.7007827758789\\
3.6497163772583 14.7007827758789\\
3.74971652030945 14.7007827758789\\
3.84971642494202 14.7007827758789\\
3.94971656799316 14.7007827758789\\
4.04971647262573 14.7007827758789\\
4.1497163772583 14.7007827758789\\
4.24971628189087 14.7007827758789\\
4.3497166633606 14.7007827758789\\
4.44971656799316 14.7007827758789\\
4.54971647262573 14.7007827758789\\
4.6497163772583 14.7007827758789\\
4.74971628189087 14.7007827758789\\
4.8497166633606 14.7007827758789\\
4.94971656799316 14.7007827758789\\
5.04971647262573 14.7007837295532\\
5.1497163772583 14.7007837295532\\
5.24971628189087 14.7007837295532\\
5.3497166633606 14.7007837295532\\
5.44971656799316 14.7007837295532\\
5.54971647262573 14.7007837295532\\
5.6497163772583 14.7007837295532\\
5.74971628189087 14.7007884979248\\
5.8497166633606 14.7008085250854\\
5.94971656799316 14.70090675354\\
6.04971647262573 14.7014398574829\\
6.1497163772583 14.704080581665\\
6.24971628189087 14.7125511169434\\
6.3497166633606 14.725396156311\\
6.44971656799316 14.7330207824707\\
6.54971647262573 14.732325553894\\
6.6497163772583 14.733006477356\\
6.74971628189087 14.732626914978\\
6.8497166633606 14.7333736419678\\
6.94971656799316 14.7336568832397\\
7.04971647262573 14.734393119812\\
7.1497163772583 14.7347822189331\\
7.24971628189087 14.7354030609131\\
7.3497166633606 14.7358684539795\\
7.44971656799316 14.7364387512207\\
7.54971647262573 14.7369375228882\\
7.6497163772583 14.7374849319458\\
7.74971628189087 14.7379999160767\\
7.8497166633606 14.7385368347168\\
7.94971656799316 14.7390584945679\\
8.04971599578857 14.7395906448364\\
8.1497163772583 14.7401151657104\\
8.24971675872803 14.7406454086304\\
8.34971618652344 14.741171836853\\
8.44971656799316 14.7417011260986\\
8.54971599578857 14.7422285079956\\
8.6497163772583 14.7427568435669\\
8.74971675872803 14.7432851791382\\
8.84971618652344 14.743842124939\\
8.94971656799316 14.744704246521\\
9.04971599578857 14.7452869415283\\
9.1497163772583 14.7459402084351\\
9.24971675872803 14.7465677261353\\
9.34971618652344 14.7472019195557\\
9.44971656799316 14.7478313446045\\
9.54971599578857 14.748459815979\\
9.6497163772583 14.7490863800049\\
9.74971675872803 14.7497119903564\\
9.84971618652344 14.7503356933594\\
9.94971656799316 14.750958442688\\
10.0497159957886 14.751579284668\\
10.1497163772583 14.7521991729736\\
10.249716758728 14.752818107605\\
10.3497161865234 14.7534303665161\\
10.4497165679932 14.7537994384766\\
10.5497159957886 14.7542762756348\\
10.6497163772583 14.7548017501831\\
10.749716758728 14.7553043365479\\
10.8497161865234 14.7558059692383\\
10.9497165679932 14.7562980651855\\
11.0497159957886 14.7567825317383\\
11.1497163772583 14.7572574615479\\
11.249716758728 14.7577199935913\\
11.3497161865234 14.7581701278687\\
11.4497165679932 14.7586069107056\\
11.5497159957886 14.7590341567993\\
11.6497163772583 14.7594566345215\\
11.749716758728 14.7598266601563\\
11.8497161865234 14.7601566314697\\
11.9497165679932 14.7643003463745\\
12.0497159957886 14.7638158798218\\
12.1497163772583 14.7637815475464\\
12.249716758728 14.7665519714355\\
12.347146987915 14.7727012634277\\
12.4250946044922 14.7783174514771\\
12.4874591827393 14.7692184448242\\
12.5560598373413 14.7686729431152\\
12.6315202713013 14.7690563201904\\
12.714527130127 14.7695341110229\\
12.8058338165283 14.7866821289063\\
12.8497476577759 14.7367706298828\\
12.8980522155762 14.7679605484009\\
12.9143037796021 14.7766189575195\\
12.9277105331421 14.7797164916992\\
12.9424571990967 14.7722702026367\\
12.9586782455444 14.7723064422607\\
12.9765224456787 14.772406578064\\
12.9961500167847 14.7725305557251\\
13.0177412033081 14.77268409729\\
13.0414915084839 14.7728700637817\\
13.0676164627075 14.7730894088745\\
13.0963535308838 14.7733449935913\\
13.1279649734497 14.7736377716064\\
13.1627368927002 14.7739715576172\\
13.2009868621826 14.7743463516235\\
13.2430610656738 14.7747659683228\\
13.289342880249 14.7752323150635\\
13.3402528762817 14.7757482528687\\
13.3962545394897 14.7763185501099\\
13.4578561782837 14.7769460678101\\
13.5256175994873 14.7776365280151\\
13.600154876709 14.778395652771\\
13.6821460723877 14.7879543304443\\
13.7723369598389 14.7718334197998\\
13.871545791626 14.7812509536743\\
13.9715461730957 14.7859354019165\\
14.0715456008911 14.7874317169189\\
14.1215467453003 14.8222885131836\\
14.147439956665 14.8224458694458\\
14.1759233474731 14.8214206695557\\
14.1775197982788 14.8230495452881\\
14.1792755126953 14.8388214111328\\
14.1812076568604 14.8541812896729\\
14.1833324432373 14.8612461090088\\
14.1856698989868 14.8579320907593\\
14.1882400512695 14.848048210144\\
14.191068649292 14.8357906341553\\
14.1941795349121 14.8241014480591\\
14.1976013183594 14.8225889205933\\
14.2013654708862 14.8215818405151\\
14.2055063247681 14.8197231292725\\
14.2100610733032 14.8171873092651\\
14.2150716781616 14.8140468597412\\
14.2205829620361 14.81028175354\\
14.2266454696655 14.8058986663818\\
14.2333135604858 14.8009548187256\\
14.2406492233276 14.7955570220947\\
14.2487182617188 14.7895851135254\\
14.2575941085815 14.7806301116943\\
14.2673578262329 14.7690601348877\\
14.27809715271 14.7588787078857\\
14.2899112701416 14.7508697509766\\
14.3029069900513 14.7460966110229\\
14.3172016143799 14.7458801269531\\
14.3329257965088 14.7500343322754\\
14.3502225875854 14.7575302124023\\
14.3692483901978 14.764627456665\\
14.3901777267456 14.7696876525879\\
14.4131994247437 14.7905673980713\\
14.4385232925415 14.8013181686401\\
14.4663791656494 14.8087797164917\\
14.4970216751099 14.8139886856079\\
14.520975112915 14.8184442520142\\
14.5473232269287 14.8209476470947\\
14.5763063430786 14.8280334472656\\
14.6081876754761 14.8234481811523\\
14.6432571411133 14.8154821395874\\
14.6818342208862 14.8090295791626\\
14.7242679595947 14.8064222335815\\
14.7709455490112 14.8045387268066\\
14.8222913742065 14.8033390045166\\
14.8712491989136 14.8034515380859\\
14.9185028076172 14.8024024963379\\
14.9638071060181 14.8014860153198\\
15.0066013336182 14.8008108139038\\
15.0464649200439 14.8033781051636\\
15.0903158187866 14.8021049499512\\
15.1385507583618 14.8002147674561\\
15.1916093826294 14.7980537414551\\
15.2499742507935 14.7981424331665\\
15.3141756057739 14.7984142303467\\
15.3847970962524 14.7987470626831\\
15.4624805450439 14.7992877960205\\
15.5479326248169 14.7999620437622\\
15.6419296264648 14.8027620315552\\
15.7419300079346 14.7995977401733\\
15.7919301986694 14.8068828582764\\
15.8469295501709 14.7996120452881\\
15.9074296951294 14.8025875091553\\
15.920783996582 14.8267011642456\\
15.9354724884033 14.7910566329956\\
15.9516305923462 14.8026876449585\\
15.9645795822144 14.8078603744507\\
15.9788236618042 14.8051252365112\\
15.9944925308228 14.8051795959473\\
16.0117282867432 14.8052787780762\\
16.0306873321533 14.8054027557373\\
16.0515422821045 14.8055582046509\\
16.0744819641113 14.8057441711426\\
16.0997161865234 14.8059644699097\\
16.1274738311768 14.8062191009521\\
16.1580085754395 14.8065099716187\\
16.1915950775146 14.8068380355835\\
16.2285404205322 14.8072052001953\\
16.2691822052002 14.8076133728027\\
16.3138847351074 14.8080654144287\\
16.3630599975586 14.8085641860962\\
16.4171524047852 14.8091144561768\\
16.4766540527344 14.8097200393677\\
16.542106628418 14.8103857040405\\
16.6141033172607 14.8111190795898\\
16.6932983398438 14.8119249343872\\
16.7804145812988 14.8146600723267\\
16.87624168396 14.8118324279785\\
16.9762420654297 14.8149604797363\\
17.0762424468994 14.8186702728271\\
17.0962505340576 14.8169832229614\\
17.1021957397461 14.8160438537598\\
17.1087341308594 14.8157091140747\\
17.1159267425537 14.8163604736328\\
17.1238384246826 14.8173379898071\\
17.1325416564941 14.8179922103882\\
17.1421165466309 14.8181858062744\\
17.1526470184326 14.8177623748779\\
17.1642322540283 14.8177585601807\\
17.1769752502441 14.8181552886963\\
17.1909923553467 14.8182220458984\\
17.2064094543457 14.8181581497192\\
17.2233715057373 14.8179349899292\\
17.2420272827148 14.8172750473022\\
17.2625503540039 14.817795753479\\
17.2851238250732 14.8186664581299\\
17.3099555969238 14.8194894790649\\
17.3372707366943 14.8202610015869\\
17.367317199707 14.8214244842529\\
17.400369644165 14.824499130249\\
17.4367256164551 14.8309907913208\\
17.4767169952393 14.8342952728271\\
17.5207099914551 14.8310642242432\\
17.5690994262695 14.8282308578491\\
17.6223297119141 14.8265428543091\\
17.6808815002441 14.8251361846924\\
17.745288848877 14.8237962722778\\
17.8161373138428 14.8237657546997\\
17.8624935150146 14.8278837203979\\
17.9028415679932 14.8306369781494\\
17.9472217559814 14.8317584991455\\
17.9960422515869 14.8284797668457\\
18.0497417449951 14.8292589187622\\
18.108814239502 14.8296241760254\\
18.173791885376 14.8301248550415\\
18.245267868042 14.8310031890869\\
18.3238925933838 14.8325843811035\\
18.41037940979 14.8356885910034\\
18.5055141448975 14.8405809402466\\
18.6055145263672 14.8439178466797\\
18.7055149078369 14.8405122756958\\
18.805513381958 14.8397369384766\\
18.9055137634277 14.8389377593994\\
19.0055141448975 14.8386125564575\\
19.1055145263672 14.8394756317139\\
19.2055149078369 14.8405332565308\\
19.305513381958 14.8418436050415\\
19.4055137634277 14.8428249359131\\
19.5055141448975 14.8440217971802\\
19.6055145263672 14.8450775146484\\
19.7055149078369 14.846227645874\\
19.805513381958 14.8473148345947\\
19.9055137634277 14.8484697341919\\
20.0055141448975 14.8495569229126\\
20.1055145263672 14.8506593704224\\
20.2055149078369 14.8517713546753\\
20.305513381958 14.8528823852539\\
20.4055137634277 14.8539962768555\\
20.5055141448975 14.8551092147827\\
20.6055145263672 14.8562231063843\\
20.7055149078369 14.8573360443115\\
20.805513381958 14.8584508895874\\
20.9055137634277 14.859564781189\\
21.0055141448975 14.8606786727905\\
21.1055145263672 14.8617935180664\\
21.2055149078369 14.862982749939\\
21.305513381958 14.8643550872803\\
21.4055137634277 14.8654136657715\\
21.5055141448975 14.8665447235107\\
21.6055145263672 14.867657661438\\
21.7055149078369 14.8687791824341\\
21.805513381958 14.8698987960815\\
21.9055137634277 14.871018409729\\
22.0055141448975 14.8721380233765\\
22.1055145263672 14.8732585906982\\
22.2055149078369 14.87437915802\\
22.305513381958 14.8754987716675\\
22.4055137634277 14.8766193389893\\
22.5055141448975 14.8777408599854\\
22.6055145263672 14.8788614273071\\
22.7055149078369 14.8799819946289\\
22.805513381958 14.8809013366699\\
22.9055137634277 14.8819103240967\\
23.0055141448975 14.8830518722534\\
23.1055145263672 14.8841466903687\\
23.2055149078369 14.8852558135986\\
23.305513381958 14.8863582611084\\
23.4055137634277 14.8874597549438\\
23.5055141448975 14.888557434082\\
23.6055145263672 14.8896532058716\\
23.7055149078369 14.8907442092896\\
23.805513381958 14.8918313980103\\
23.9055137634277 14.8929147720337\\
24.0055141448975 14.8973426818848\\
24.1055145263672 14.8963432312012\\
24.2055149078369 14.8965749740601\\
24.305513381958 14.897912979126\\
24.4055137634277 14.899019241333\\
24.5055141448975 14.9002637863159\\
24.6055145263672 14.9015207290649\\
24.7055149078369 14.9091014862061\\
24.789867401123 14.9124031066895\\
24.8573379516602 14.908429145813\\
24.9315567016602 14.9052228927612\\
25.0131969451904 14.9058017730713\\
25.1030006408691 14.9229736328125\\
25.1443557739258 14.8719282150269\\
25.189847946167 14.9017992019653\\
25.2099170684814 14.9156122207642\\
25.2266445159912 14.9176540374756\\
25.245044708252 14.9090824127197\\
25.2652854919434 14.9091358184814\\
25.2875499725342 14.9092798233032\\
25.3120403289795 14.9094495773315\\
25.3389797210693 14.9096584320068\\
25.3686122894287 14.9099073410034\\
25.4012107849121 14.9101982116699\\
25.4370670318604 14.9105339050293\\
25.4765090942383 14.9109153747559\\
25.5198955535889 14.9113464355469\\
25.5676212310791 14.911828994751\\
25.6201171875 14.9123640060425\\
25.6778659820557 14.9129552841187\\
25.7413883209229 14.9136066436768\\
25.8112621307373 14.9143228530884\\
25.8881244659424 14.9151105880737\\
25.9726715087891 14.9250268936157\\
26.065673828125 14.9084243774414\\
26.1656742095947 14.9180812835693\\
26.2656745910645 14.9219942092896\\
26.3656749725342 14.9271039962769\\
26.409065246582 14.9905433654785\\
26.4240894317627 14.9428100585938\\
26.4268970489502 14.9467554092407\\
26.4299831390381 14.9844131469727\\
26.4333801269531 15.0116138458252\\
26.4371166229248 15.006160736084\\
26.4412250518799 14.9833459854126\\
26.4457454681396 14.9567747116089\\
26.4507179260254 14.9507646560669\\
26.4561862945557 14.9450531005859\\
26.4622020721436 14.9374942779541\\
26.4688205718994 14.9286851882935\\
26.4761009216309 14.9186964035034\\
26.4841079711914 14.9024829864502\\
26.4929161071777 14.8745708465576\\
26.5026054382324 14.8438882827759\\
26.5132637023926 14.8130626678467\\
26.5249881744385 14.8001461029053\\
26.5378837585449 14.8013877868652\\
26.5520706176758 14.8034315109253\\
26.5676746368408 14.8064813613892\\
26.5848388671875 14.8114957809448\\
26.6037216186523 14.823561668396\\
26.624490737915 14.9041938781738\\
26.6473369598389 14.9308948516846\\
26.6724681854248 14.9534959793091\\
26.7001132965088 14.9321537017822\\
26.7305221557617 14.934362411499\\
26.7594413757324 14.9430532455444\\
26.7881412506104 14.9447593688965\\
26.8180274963379 14.9418487548828\\
26.848762512207 14.9389333724976\\
26.8805618286133 14.9358825683594\\
26.9128971099854 14.9342393875122\\
26.9456844329834 14.9331693649292\\
26.97873878479 14.9324254989624\\
27.0120601654053 14.9320163726807\\
27.0457935333252 14.9321355819702\\
27.0821418762207 14.932279586792\\
27.1221237182617 14.9328956604004\\
27.1524639129639 14.9334154129028\\
27.1838111877441 14.9332399368286\\
27.2182941436768 14.9331483840942\\
27.2558403015137 14.9333639144897\\
27.2955665588379 14.9349718093872\\
27.3392658233643 14.9365282058716\\
27.3873329162598 14.9355592727661\\
27.4402103424072 14.932731628418\\
27.4983730316162 14.9329652786255\\
27.5623531341553 14.9334697723389\\
27.6327304840088 14.9339361190796\\
27.7101459503174 14.9345626831055\\
27.7953014373779 14.935338973999\\
27.8889751434326 14.9383172988892\\
27.9889755249023 14.935079574585\\
28.0889739990234 14.9406566619873\\
28.1889743804932 14.9370632171631\\
28.2889747619629 14.9397621154785\\
28.2927665710449 15.029182434082\\
28.2969398498535 14.8933029174805\\
28.3015289306641 14.9227323532104\\
28.304780960083 14.9578008651733\\
28.307991027832 14.9553022384644\\
28.3115215301514 14.9428262710571\\
28.3154048919678 14.9426984786987\\
28.3196773529053 14.9427585601807\\
28.3243770599365 14.942756652832\\
28.3295459747314 14.9427576065063\\
28.3352317810059 14.9427623748779\\
28.3414859771729 14.9427719116211\\
28.3483657836914 14.9427871704102\\
28.3559341430664 14.9428091049194\\
28.3642578125 14.9428386688232\\
28.3734169006348 14.9428787231445\\
28.3834896087646 14.9429292678833\\
28.3945693969727 14.9429922103882\\
28.4067573547363 14.9430694580078\\
28.4201641082764 14.9431619644165\\
28.4349136352539 14.9432725906372\\
28.451135635376 14.9434022903442\\
28.4689807891846 14.9435529708862\\
28.4886093139648 14.9437265396118\\
28.5102024078369 14.9439249038696\\
28.5339546203613 14.9441499710083\\
28.5600814819336 14.9444036483765\\
28.5888214111328 14.9446878433228\\
28.620433807373 14.9450044631958\\
28.6552085876465 14.9453554153442\\
28.6934604644775 14.9457445144653\\
28.735538482666 14.946174621582\\
28.7818241119385 14.9466485977173\\
28.8327369689941 14.9471702575684\\
28.8887424468994 14.9477443695068\\
28.9503479003906 14.9483757019043\\
29.0181159973145 14.949070930481\\
29.092658996582 14.9498357772827\\
29.1746559143066 14.9506769180298\\
29.2648525238037 14.9534139633179\\
29.3640689849854 14.9506978988647\\
29.4640693664551 14.9538469314575\\
29.4847068786621 14.990306854248\\
29.4888343811035 14.9767799377441\\
29.4933738708496 14.9730243682861\\
29.4983692169189 14.9749917984009\\
29.5038623809814 14.9799671173096\\
29.5099067687988 14.9851007461548\\
29.5165538787842 14.9893236160278\\
29.5230560302734 14.9923229217529\\
29.5302066802979 14.992057800293\\
29.5380744934082 14.9914808273315\\
29.5467281341553 14.9924287796021\\
29.556245803833 15.0068359375\\
29.5667171478271 15.0434789657593\\
29.5782356262207 15.0608205795288\\
29.5909042358398 15.048900604248\\
29.6048412322998 15.0223159790039\\
29.6201705932617 15.0033016204834\\
29.637035369873 14.98193359375\\
29.6555843353271 14.9575176239014\\
29.675989151001 14.9270372390747\\
29.6984329223633 14.8900175094604\\
29.7114696502686 14.8749370574951\\
29.7258071899414 14.865478515625\\
29.7415809631348 14.8589191436768\\
29.7589302062988 14.8572874069214\\
29.7780151367188 14.8754825592041\\
29.7990093231201 14.9062395095825\\
29.8221015930176 14.9403438568115\\
29.8475036621094 14.9625034332275\\
29.8754444122314 14.9619789123535\\
29.9061813354492 14.9667282104492\\
29.9399909973145 14.9624214172363\\
29.9771823883057 14.9616079330444\\
30.0180931091309 14.9706830978394\\
30.0630931854248 14.9721250534058\\
30.1125946044922 14.9658699035645\\
30.1670475006104 14.9701681137085\\
30.2269439697266 14.9626846313477\\
30.2706527709961 14.9616565704346\\
30.318733215332 14.9579372406006\\
30.3593273162842 14.9569540023804\\
30.3947944641113 14.9615793228149\\
30.4330863952637 14.9652519226074\\
30.4652576446533 14.9671697616577\\
30.5006484985352 14.9657430648804\\
30.5330505371094 14.9660358428955\\
30.5677318572998 14.9668111801147\\
30.6004276275635 14.967357635498\\
30.6349658966064 14.9677438735962\\
30.6682662963867 14.9683675765991\\
30.7048969268799 14.9690294265747\\
30.7451915740967 14.9700117111206\\
30.789514541626 14.9714412689209\\
30.8382682800293 14.9735927581787\\
30.8918991088867 14.9766263961792\\
30.9508934020996 14.9801692962646\\
31.0157871246338 14.9810523986816\\
31.0871696472168 14.9822988510132\\
31.1656913757324 14.9743814468384\\
31.2520637512207 14.9756555557251\\
31.347074508667 14.9767618179321\\
31.4470748901367 14.976035118103\\
31.5470752716064 14.9768695831299\\
31.6470737457275 14.9779739379883\\
31.7470741271973 14.9792556762695\\
31.847074508667 14.9802827835083\\
31.9470748901367 14.9814729690552\\
32.0470733642578 14.9825601577759\\
32.1470756530762 14.9837141036987\\
32.2470741271973 14.984842300415\\
32.3470764160156 14.9859609603882\\
32.4470748901367 14.9870824813843\\
32.5470733642578 14.9882154464722\\
32.6470756530762 14.9893407821655\\
32.7470741271973 14.9904708862305\\
32.8470764160156 14.9915981292725\\
32.9470748901367 14.9927272796631\\
33.0470733642578 14.9938554763794\\
33.1470756530762 14.99498462677\\
33.2470741271973 14.9961128234863\\
33.3470764160156 14.9972410202026\\
33.4470748901367 14.9983692169189\\
33.5470733642578 14.9994983673096\\
33.6470756530762 15.0007028579712\\
33.7470741271973 15.0020895004272\\
33.8470764160156 15.0031490325928\\
33.9470748901367 15.0042991638184\\
34.0470733642578 15.0054216384888\\
34.1470756530762 15.0065517425537\\
34.2470741271973 15.0076808929443\\
34.3470764160156 15.0088090896606\\
34.4470748901367 15.0099382400513\\
34.5470733642578 15.0110673904419\\
34.6470756530762 15.0121955871582\\
34.7470741271973 15.0133247375488\\
34.8470764160156 15.0144529342651\\
34.9470748901367 15.0155820846558\\
35.0470733642578 15.0167102813721\\
35.1470756530762 15.0178375244141\\
35.2470741271973 15.0187406539917\\
35.3470764160156 15.0197877883911\\
35.4470748901367 15.0209245681763\\
35.5470733642578 15.0220308303833\\
35.6470756530762 15.0231447219849\\
35.7470741271973 15.0242538452148\\
35.8470764160156 15.0253620147705\\
35.9470748901367 15.0264663696289\\
36.0470733642578 15.02756690979\\
36.1470756530762 15.0286645889282\\
36.2470741271973 15.0297565460205\\
36.3470764160156 15.0308456420898\\
36.4470748901367 15.0319299697876\\
36.5470733642578 15.0367088317871\\
36.6470756530762 15.0355014801025\\
36.7470741271973 15.0357141494751\\
36.8470764160156 15.0370750427246\\
36.9470748901367 15.0382385253906\\
37.0470733642578 15.0399618148804\\
37.1470756530762 15.0478219985962\\
37.2293395996094 15.0507116317749\\
37.2951393127441 15.0446720123291\\
37.3675231933594 15.0430946350098\\
37.4471435546875 15.0437049865723\\
37.534725189209 15.0443515777588\\
37.6310653686523 15.0602254867554\\
37.6772041320801 15.0154504776001\\
37.7279586791992 15.0434455871582\\
37.7453880310059 15.0559968948364\\
37.7602653503418 15.0504083633423\\
37.7766304016113 15.0480985641479\\
37.7946319580078 15.0481758117676\\
37.8144378662109 15.0482902526855\\
37.8362197875977 15.0484342575073\\
37.860179901123 15.0486125946045\\
37.8865356445313 15.048828125\\
37.9155311584473 15.0490818023682\\
37.9474220275879 15.0493755340576\\
37.9825019836426 15.0497102737427\\
38.0210914611816 15.0500898361206\\
38.0635414123535 15.0505151748657\\
38.1102333068848 15.050989151001\\
38.1615982055664 15.051513671875\\
38.2180976867676 15.0520925521851\\
38.2802429199219 15.0527305603027\\
38.3486099243164 15.0534334182739\\
38.4238090515137 15.0542049407959\\
38.5065269470215 15.0635442733765\\
38.5975189208984 15.0479984283447\\
38.6975212097168 15.0571489334106\\
38.7975196838379 15.0615015029907\\
38.8975219726563 15.0646104812622\\
38.9475212097168 15.1068248748779\\
38.9856338500977 15.0554523468018\\
39.0002326965332 15.0596685409546\\
39.0162887573242 15.074875831604\\
39.0339508056641 15.0196542739868\\
39.0533828735352 15.0236587524414\\
39.0747528076172 15.0281391143799\\
39.0982627868652 15.0386371612549\\
39.1241226196289 15.0343189239502\\
39.1525688171387 15.0356178283691\\
39.1838569641113 15.0318927764893\\
39.2182769775391 15.0263814926147\\
39.2382202148438 15.0288171768188\\
39.2601547241211 15.0320692062378\\
39.2842864990234 15.0438995361328\\
39.3108291625977 15.0539636611938\\
39.3400230407715 15.0636329650879\\
39.3721389770508 15.071759223938\\
39.4047470092773 15.0824699401855\\
39.4365348815918 15.0848035812378\\
39.4690742492676 15.0786380767822\\
39.5013847351074 15.073860168457\\
39.5341873168945 15.0730228424072\\
39.5668563842773 15.0727167129517\\
39.5998191833496 15.0723686218262\\
39.632698059082 15.0722303390503\\
39.6657829284668 15.0726518630981\\
39.6988639831543 15.0733270645142\\
39.7321624755859 15.0734090805054\\
39.7687873840332 15.0733156204224\\
39.8090744018555 15.0733938217163\\
39.8533935546875 15.0735673904419\\
39.8968162536621 15.0747079849243\\
39.939453125 15.0775203704834\\
39.9863548278809 15.0754022598267\\
40.0379447937012 15.0727977752686\\
40.0946960449219 15.0729818344116\\
40.1571197509766 15.0735454559326\\
40.2257843017578 15.0739345550537\\
40.3013191223145 15.0745630264282\\
40.384407043457 15.0753135681152\\
40.4758033752441 15.0781593322754\\
40.5758018493652 15.0751533508301\\
40.6758041381836 15.0804147720337\\
40.7758026123047 15.0771484375\\
40.875804901123 15.0795545578003\\
40.8792419433594 15.1705026626587\\
40.8830223083496 15.0418367385864\\
40.8871841430664 15.0566053390503\\
40.8903007507324 15.0976390838623\\
40.8937301635742 15.0959091186523\\
40.8975028991699 15.0830898284912\\
40.9016532897949 15.0823287963867\\
40.9062156677246 15.0826759338379\\
40.9112396240234 15.0827026367188\\
40.9167633056641 15.0826959609985\\
40.9228363037109 15.0826978683472\\
40.9295196533203 15.082706451416\\
40.9368743896484 15.0827217102051\\
40.9449615478516 15.0827445983887\\
40.953857421875 15.0827770233154\\
40.9636421203613 15.0828189849854\\
40.9744033813477 15.0828733444214\\
40.9862442016602 15.0829420089722\\
40.999267578125 15.083025932312\\
41.0135955810547 15.0831279754639\\
41.029354095459 15.0832481384277\\
41.046688079834 15.0833892822266\\
41.0657577514648 15.083553314209\\
41.0867347717285 15.0837421417236\\
41.109806060791 15.0839576721191\\
41.1351852416992 15.0842008590698\\
41.1631050109863 15.0844745635986\\
41.1938133239746 15.0847806930542\\
41.227596282959 15.0851202011108\\
41.2647552490234 15.0854978561401\\
41.3056297302246 15.0859146118164\\
41.3505935668945 15.0863742828369\\
41.4000511169434 15.0868816375732\\
41.4544563293457 15.087438583374\\
41.5143013000488 15.0880527496338\\
41.5801315307617 15.0887279510498\\
41.6525421142578 15.0894708633423\\
41.7321968078613 15.0902881622314\\
41.8198165893555 15.0929698944092\\
41.9161987304688 15.0902862548828\\
42.0161972045898 15.0933780670166\\
42.1161994934082 15.1014089584351\\
42.1361999511719 15.1044054031372\\
42.1442070007324 15.1263380050659\\
42.1520309448242 15.1286678314209\\
42.1606330871582 15.1251840591431\\
42.1689529418945 15.1250610351563\\
42.1781005859375 15.1263246536255\\
42.1881637573242 15.1490697860718\\
42.1992378234863 15.196873664856\\
42.2114143371582 15.189733505249\\
42.2248077392578 15.1753234863281\\
42.2395439147949 15.1590957641602\\
42.2557525634766 15.1399297714233\\
42.2735824584961 15.1175594329834\\
42.2931938171387 15.0951271057129\\
42.3147659301758 15.0735559463501\\
42.3384971618652 15.0424995422363\\
42.3646011352539 15.0191345214844\\
42.3847808837891 15.0164680480957\\
42.4069786071777 15.0296792984009\\
42.431396484375 15.0430812835693\\
42.4582557678223 15.0570974349976\\
42.4878005981445 15.0924634933472\\
42.5203018188477 15.102765083313\\
42.5560493469238 15.104564666748\\
42.5953750610352 15.1070861816406\\
42.6386299133301 15.1000471115112\\
42.6862144470215 15.1255187988281\\
42.7385559082031 15.1007146835327\\
42.7959022521973 15.1126689910889\\
42.8589859008789 15.1065330505371\\
42.9028167724609 15.099627494812\\
42.9510307312012 15.1031665802002\\
42.9925117492676 15.1003246307373\\
43.0297622680664 15.1022920608521\\
43.0707359313965 15.1042146682739\\
43.1158065795898 15.1059608459473\\
43.1579742431641 15.108437538147\\
43.203685760498 15.1111583709717\\
43.2497863769531 15.1133432388306\\
43.3005027770996 15.112265586853\\
43.3562850952148 15.1151332855225\\
43.4176483154297 15.1133279800415\\
43.4851493835449 15.115701675415\\
43.559398651123 15.1133918762207\\
43.6410713195801 15.1153106689453\\
43.7309150695801 15.1132593154907\\
43.8297386169434 15.1149082183838\\
43.9297409057617 15.1136474609375\\
44.0297393798828 15.1167106628418\\
44.1297416687012 15.1160287857056\\
44.2297401428223 15.1178121566772\\
44.2499618530273 15.1207723617554\\
44.2722015380859 15.1194009780884\\
44.2966690063477 15.1196460723877\\
44.3235816955566 15.1199216842651\\
44.3531875610352 15.1202249526978\\
44.3857498168945 15.1205577850342\\
44.421573638916 15.120924949646\\
44.4609756469727 15.1213293075562\\
44.5043182373047 15.1217737197876\\
44.5519981384277 15.1222620010376\\
44.6044425964355 15.1227998733521\\
44.6621360778809 15.1233921051025\\
44.7255935668945 15.1240425109863\\
44.7953987121582 15.1247587203979\\
44.872184753418 15.1255464553833\\
44.9566497802734 15.1264123916626\\
45.0495643615723 15.127366065979\\
45.1495628356934 15.1290063858032\\
45.2495613098145 15.1289091110229\\
45.3495635986328 15.1305780410767\\
45.4495620727539 15.1325855255127\\
45.5495643615723 15.1322498321533\\
45.6495628356934 15.1340141296387\\
45.7495613098145 15.143045425415\\
45.7695655822754 15.1518993377686\\
45.7816352844238 15.1519794464111\\
45.7949142456055 15.1501445770264\\
45.8095207214355 15.1481800079346\\
45.8255882263184 15.1468181610107\\
45.8432655334473 15.145131111145\\
45.8627052307129 15.1430063247681\\
45.884090423584 15.1402645111084\\
45.9076156616211 15.1387748718262\\
45.9334945678711 15.137110710144\\
45.9619598388672 15.1353178024292\\
45.9932708740234 15.1338310241699\\
46.0277099609375 15.1329689025879\\
46.0655975341797 15.1322021484375\\
46.1072731018066 15.1316795349121\\
46.1531181335449 15.131516456604\\
46.2035446166992 15.1321287155151\\
46.2590141296387 15.1336517333984\\
46.3200302124023 15.1395454406738\\
46.387149810791 15.1426134109497\\
46.4504814147949 15.1439905166626\\
46.5130958557129 15.1447420120239\\
46.581974029541 15.1458339691162\\
46.6577415466309 15.1459465026855\\
46.7410850524902 15.1467313766479\\
46.8327598571777 15.1475620269775\\
46.9327621459961 15.148699760437\\
47.0327606201172 15.1497564315796\\
47.1327629089355 15.1508131027222\\
47.2327613830566 15.1519069671631\\
47.3327598571777 15.1530160903931\\
47.4327621459961 15.1541423797607\\
47.5327606201172 15.1552734375\\
47.6327629089355 15.1562194824219\\
47.7327613830566 15.1572999954224\\
47.8327598571777 15.1583862304688\\
47.9327621459961 15.1595058441162\\
48.0327606201172 15.160626411438\\
48.1327629089355 15.16175365448\\
48.2327613830566 15.1628799438477\\
48.3327598571777 15.1640071868896\\
48.4327621459961 15.1651334762573\\
48.5327606201172 15.1662607192993\\
48.6327629089355 15.1673879623413\\
48.7327613830566 15.1685171127319\\
48.8327598571777 15.1696453094482\\
48.9327621459961 15.1707582473755\\
49.0327606201172 15.1719007492065\\
49.1327629089355 15.1730527877808\\
49.2327613830566 15.1742172241211\\
49.3327598571777 15.175404548645\\
49.4327621459961 15.1766338348389\\
49.5327606201172 15.1835803985596\\
49.6257057189941 15.1864309310913\\
49.6982688903809 15.1847476959229\\
49.7780876159668 15.1806678771973\\
49.8658905029297 15.1813354492188\\
49.962474822998 15.1820306777954\\
50.0624732971191 15.1829824447632\\
50.1624755859375 15.1840620040894\\
50.2624740600586 15.1851768493652\\
50.3624725341797 15.1862812042236\\
50.462474822998 15.1873931884766\\
50.5624732971191 15.1885080337524\\
50.6624755859375 15.1896276473999\\
50.7624740600586 15.190749168396\\
50.8624725341797 15.1918678283691\\
50.962474822998 15.1929903030396\\
51.0624732971191 15.1941137313843\\
51.1624755859375 15.195240020752\\
51.2624740600586 15.1963663101196\\
51.3624725341797 15.1974925994873\\
51.462474822998 15.1986207962036\\
51.5624732971191 15.1997499465942\\
51.6624755859375 15.2008781433105\\
51.7624740600586 15.2020063400269\\
51.8624725341797 15.2031345367432\\
51.962474822998 15.2042627334595\\
52.0624732971191 15.2053909301758\\
52.1624755859375 15.2065200805664\\
52.2624740600586 15.2076482772827\\
52.3624725341797 15.208776473999\\
52.462474822998 15.2099046707153\\
52.5624732971191 15.2110328674316\\
52.6624755859375 15.2121610641479\\
52.7624740600586 15.2132892608643\\
52.8624725341797 15.2144174575806\\
52.962474822998 15.2155456542969\\
53.0624732971191 15.2166738510132\\
53.1624755859375 15.2178020477295\\
53.2624740600586 15.2189311981201\\
53.3624725341797 15.2200593948364\\
53.462474822998 15.2211875915527\\
53.5624732971191 15.222315788269\\
53.6624755859375 15.2234439849854\\
53.7624740600586 15.2245721817017\\
53.8624725341797 15.225700378418\\
53.962474822998 15.2268285751343\\
54.0624732971191 15.2279567718506\\
54.1624755859375 15.2290849685669\\
54.2624740600586 15.2302131652832\\
54.3624725341797 15.2313413619995\\
54.462474822998 15.2324695587158\\
54.5624732971191 15.2335977554321\\
54.6624755859375 15.2347259521484\\
54.7624740600586 15.2358541488647\\
54.8624725341797 15.2369823455811\\
54.962474822998 15.2381105422974\\
55.0624732971191 15.2392387390137\\
55.1624755859375 15.2403688430786\\
55.2624740600586 15.2415037155151\\
55.3624725341797 15.2426662445068\\
55.462474822998 15.2439756393433\\
55.5624732971191 15.2460346221924\\
55.6624755859375 15.2506475448608\\
55.7624740600586 15.2581014633179\\
55.8624725341797 15.2591276168823\\
55.962474822998 15.2546482086182\\
56.0624732971191 15.2517595291138\\
56.1624755859375 15.253833770752\\
56.2624740600586 15.2526025772095\\
56.3624725341797 15.2550840377808\\
56.462474822998 15.2553071975708\\
56.5624732971191 15.2570285797119\\
56.6624755859375 15.2577657699585\\
56.7624740600586 15.2591514587402\\
56.8624725341797 15.2601108551025\\
56.962474822998 15.2613506317139\\
57.0624732971191 15.2624063491821\\
57.1624755859375 15.2635831832886\\
57.2624740600586 15.2646799087524\\
57.3624725341797 15.265830039978\\
57.462474822998 15.2669591903687\\
57.5624732971191 15.2680835723877\\
57.6624755859375 15.2692041397095\\
57.7624740600586 15.270339012146\\
57.8624725341797 15.2714653015137\\
57.962474822998 15.2725963592529\\
58.0624732971191 15.2737245559692\\
58.1624755859375 15.2748556137085\\
58.2624740600586 15.2759847640991\\
58.3624725341797 15.2771787643433\\
58.462474822998 15.2785768508911\\
58.5624732971191 15.2796430587769\\
58.6624755859375 15.2807950973511\\
58.7624740600586 15.2819194793701\\
58.8624725341797 15.2830533981323\\
58.962474822998 15.2841835021973\\
59.0624732971191 15.2853164672852\\
59.1624755859375 15.2864484786987\\
59.2624740600586 15.2875804901123\\
59.3624725341797 15.2887134552002\\
59.462474822998 15.2898473739624\\
59.5624732971191 15.2909803390503\\
59.6624755859375 15.2921152114868\\
59.7624740600586 15.2932500839233\\
59.8624725341797 15.2943830490112\\
59.9312362670898 15.2952404022217\\
60 15.2960224151611\\
60.0756378173828 15.2967901229858\\
60.1588439941406 15.2976398468018\\
60.2503662109375 15.2985744476318\\
60.3503646850586 15.2996025085449\\
60.450366973877 15.3007154464722\\
60.550365447998 15.3018445968628\\
60.6503677368164 15.3029661178589\\
60.7503662109375 15.3040895462036\\
60.8503646850586 15.305212020874\\
60.950366973877 15.3063402175903\\
61.050365447998 15.3075017929077\\
61.1503677368164 15.3086757659912\\
61.2503662109375 15.3098239898682\\
61.3503646850586 15.310977935791\\
61.450366973877 15.3121452331543\\
61.550365447998 15.3133325576782\\
61.6503677368164 15.3145561218262\\
61.7503662109375 15.3162097930908\\
61.8503646850586 15.3241939544678\\
61.9325637817383 15.3269824981689\\
61.998291015625 15.3208036422729\\
62.0705947875977 15.3194065093994\\
62.1501274108887 15.3200178146362\\
62.2376098632813 15.3206624984741\\
62.3338470458984 15.3215112686157\\
62.4338455200195 15.3225431442261\\
62.5338439941406 15.3236522674561\\
62.633846282959 15.3247575759888\\
62.7338447570801 15.3258695602417\\
62.8338470458984 15.3269834518433\\
62.9338455200195 15.3281030654907\\
63.0338439941406 15.3292217254639\\
63.133846282959 15.3303413391113\\
63.2338447570801 15.3314638137817\\
63.3338470458984 15.3325891494751\\
63.4338455200195 15.3337135314941\\
63.5338439941406 15.3348426818848\\
63.633846282959 15.3359708786011\\
63.7338447570801 15.3371000289917\\
63.8338470458984 15.338228225708\\
63.9338455200195 15.3393564224243\\
64.0338439941406 15.3404846191406\\
64.1338424682617 15.3416128158569\\
64.2338485717773 15.3427419662476\\
64.3338470458984 15.3438701629639\\
64.4338455200195 15.3449983596802\\
64.5338439941406 15.3461265563965\\
64.6338424682617 15.3472547531128\\
64.7338485717773 15.3483829498291\\
64.8338470458984 15.3495111465454\\
64.9338455200195 15.3506393432617\\
65.0338439941406 15.351767539978\\
65.1338424682617 15.3528966903687\\
65.2338485717773 15.354024887085\\
65.3338470458984 15.3551530838013\\
65.4338455200195 15.3562812805176\\
65.5338439941406 15.3574094772339\\
65.6338424682617 15.3585376739502\\
65.7338485717773 15.3596658706665\\
65.8338470458984 15.3607940673828\\
65.9338455200195 15.3619222640991\\
66.0338439941406 15.3630504608154\\
66.1338424682617 15.3641786575317\\
66.2338485717773 15.365306854248\\
66.3338470458984 15.3664350509644\\
66.4338455200195 15.3675632476807\\
66.5338439941406 15.368691444397\\
66.6338424682617 15.3698196411133\\
66.7338485717773 15.3709478378296\\
66.8338470458984 15.3720760345459\\
66.9338455200195 15.3732042312622\\
67.0338439941406 15.3743324279785\\
67.1338424682617 15.3754606246948\\
67.2338485717773 15.3765888214111\\
67.3338470458984 15.3777179718018\\
67.4338455200195 15.3788471221924\\
67.5338439941406 15.3799810409546\\
67.6338424682617 15.3811378479004\\
67.7338485717773 15.3824148178101\\
67.8338470458984 15.3843259811401\\
67.9338455200195 15.3885641098022\\
68.0338439941406 15.3959884643555\\
68.1338424682617 15.3985404968262\\
68.2338485717773 15.3932781219482\\
68.3338470458984 15.3902626037598\\
68.4338455200195 15.3925457000732\\
68.5338439941406 15.3910255432129\\
68.6338424682617 15.3936004638672\\
68.7338485717773 15.3937511444092\\
68.8338470458984 15.3955230712891\\
68.9338455200195 15.3962259292603\\
69.0338439941406 15.3976345062256\\
69.1338424682617 15.3985786437988\\
69.2338485717773 15.399829864502\\
69.3338470458984 15.4008779525757\\
69.4338455200195 15.4020595550537\\
69.5338439941406 15.4031534194946\\
69.6338424682617 15.4043064117432\\
69.7338485717773 15.4054193496704\\
69.8338470458984 15.4065589904785\\
69.9338455200195 15.4076814651489\\
70.0338439941406 15.4088153839111\\
70.1338424682617 15.4099426269531\\
70.2338485717773 15.4110736846924\\
70.3338470458984 15.4122018814087\\
70.4338455200195 15.4133329391479\\
70.5338439941406 15.4144620895386\\
70.6338424682617 15.4156684875488\\
70.7338485717773 15.4170522689819\\
70.8338470458984 15.4181261062622\\
70.9338455200195 15.4192762374878\\
71.0338439941406 15.4204006195068\\
71.1338424682617 15.421534538269\\
71.2338485717773 15.422664642334\\
71.3338470458984 15.4237966537476\\
71.4338455200195 15.4249286651611\\
71.5338439941406 15.4260606765747\\
71.6338424682617 15.4271936416626\\
71.7338485717773 15.4283275604248\\
71.8338470458984 15.4294605255127\\
71.9338455200195 15.4305953979492\\
72.0338439941406 15.4317302703857\\
72.1338424682617 15.4328594207764\\
72.2338485717773 15.4337282180786\\
72.3338470458984 15.4348306655884\\
72.4338455200195 15.4359569549561\\
72.5338439941406 15.4370794296265\\
72.6338424682617 15.4382019042969\\
72.7338485717773 15.4393243789673\\
72.8338470458984 15.4404458999634\\
72.9338455200195 15.4415674209595\\
73.0338439941406 15.4426879882813\\
73.1338424682617 15.4438095092773\\
73.2338485717773 15.4449443817139\\
73.3338470458984 15.4461278915405\\
73.4338455200195 15.4473123550415\\
73.5338439941406 15.44846534729\\
73.6338424682617 15.4496212005615\\
73.7338485717773 15.4507942199707\\
73.8338470458984 15.4519844055176\\
73.9338455200195 15.4532203674316\\
74.0338439941406 15.4560813903809\\
74.1323699951172 15.4629106521606\\
74.2112121582031 15.4660730361938\\
74.2755126953125 15.457633972168\\
74.3462524414063 15.4579305648804\\
74.4240570068359 15.458517074585\\
74.509651184082 15.4591579437256\\
74.6037979125977 15.4599933624268\\
74.7037963867188 15.4610061645508\\
74.8037948608398 15.4621171951294\\
74.9019012451172 15.4632234573364\\
75 15.4643135070801\\
};
\addlegendentry{2"};

\addplot [
color=black,
dash pattern=on 1pt off 3pt on 3pt off 3pt
]
table[row sep=crcr]{
0 14.7006950378418\\
9.99999971718069e-10 14.7006950378418\\
2.49999998480632e-09 14.7006950378418\\
4.74999994892755e-09 14.7006950378418\\
8.1250002281763e-09 14.7006950378418\\
1.31874999809156e-08 14.7006950378418\\
2.0781250498203e-08 14.7006950378418\\
3.21718758300449e-08 14.7006950378418\\
4.92578138278077e-08 14.7006950378418\\
7.48867208244519e-08 14.7006950378418\\
1.13330081319418e-07 14.7006950378418\\
1.70995122061868e-07 14.7006950378418\\
2.57492672517401e-07 14.7006950378418\\
3.8723902662241e-07 14.7006950378418\\
5.81858500936505e-07 14.7006950378418\\
8.73787769251066e-07 14.7006950378418\\
1.31168167172291e-06 14.7006950378418\\
1.9685226106958e-06 14.7006950378418\\
2.95378367809462e-06 14.7006959915161\\
4.43167573394021e-06 14.7006959915161\\
6.64851359033491e-06 14.7006959915161\\
9.97376992017962e-06 14.7006959915161\\
1.49616553244414e-05 14.7006969451904\\
2.24434825213393e-05 14.7006978988647\\
3.36662233166862e-05 14.7006988525391\\
5.05003372381907e-05 14.7006998062134\\
7.57515081204474e-05 14.700701713562\\
0.000113628258986864 14.7007036209106\\
0.000170443381648511 14.7007055282593\\
0.000255666091106832 14.7007074356079\\
0.000383500126190484 14.7007083892822\\
0.000575251178815961 14.7007074356079\\
0.000862877757754177 14.7007036209106\\
0.00129431765526533 14.7007007598877\\
0.00194147753063589 14.7006969451904\\
0.00291221728548408 14.7006931304932\\
0.00436832662671804 14.7006912231445\\
0.00597004732117057 14.7006931304932\\
0.00773194013163447 14.7006950378418\\
0.009670021943748 14.7006959915161\\
0.0118019115179777 14.7006959915161\\
0.0141469910740852 14.7006959915161\\
0.0167265776544809 14.7006959915161\\
0.0195641238242388 14.7006959915161\\
0.0226854234933853 14.7006959915161\\
0.0261188540607691 14.7006959915161\\
0.0298956278711557 14.7006959915161\\
0.0340500771999359 14.7006950378418\\
0.0386199727654457 14.7006950378418\\
0.0436468608677387 14.7006950378418\\
0.0491764321923256 14.7006950378418\\
0.0552589632570744 14.7006950378418\\
0.0619497485458851 14.7006959915161\\
0.0693096145987511 14.7006959915161\\
0.0774054601788521 14.7006959915161\\
0.0863108932971954 14.7006931304932\\
0.0961068719625473 14.7006959915161\\
0.106882445514202 14.7006959915161\\
0.118735581636429 14.7006959915161\\
0.13177402317524 14.7006959915161\\
0.146116316318512 14.7006959915161\\
0.161892831325531 14.7006959915161\\
0.179247006773949 14.7006959915161\\
0.198336601257324 14.7006959915161\\
0.21933513879776 14.7006959915161\\
0.242433547973633 14.7006950378418\\
0.267841786146164 14.7006959915161\\
0.29579085111618 14.7006959915161\\
0.32653483748436 14.7006959915161\\
0.360353201627731 14.7006959915161\\
0.397553414106369 14.7006959915161\\
0.438473641872406 14.7006959915161\\
0.483485877513886 14.7006959915161\\
0.532999336719513 14.7006959915161\\
0.587464153766632 14.7006959915161\\
0.647375464439392 14.7006959915161\\
0.71327793598175 14.7006959915161\\
0.7857705950737 14.7006959915161\\
0.865512549877167 14.7006959915161\\
0.953228712081909 14.7006959915161\\
1.04971647262573 14.7006950378418\\
1.14971649646759 14.7006950378418\\
1.24971640110016 14.7006950378418\\
1.34971642494202 14.7006950378418\\
1.44971644878387 14.7006950378418\\
1.54971647262573 14.7006950378418\\
1.64971649646759 14.7006950378418\\
1.74971640110016 14.7006950378418\\
1.84971642494202 14.7006950378418\\
1.94971644878387 14.7006950378418\\
2.04971647262573 14.7006950378418\\
2.1497163772583 14.7006950378418\\
2.24971652030945 14.7006950378418\\
2.34971642494202 14.7006950378418\\
2.44971656799316 14.7006950378418\\
2.54971647262573 14.7006950378418\\
2.6497163772583 14.7006950378418\\
2.74971652030945 14.7006950378418\\
2.84971642494202 14.7006950378418\\
2.94971656799316 14.7006950378418\\
3.04971647262573 14.7006950378418\\
3.1497163772583 14.7006950378418\\
3.24971652030945 14.7006950378418\\
3.34971642494202 14.7006950378418\\
3.44971656799316 14.7006950378418\\
3.54971647262573 14.7006950378418\\
3.6497163772583 14.7006950378418\\
3.74971652030945 14.7006950378418\\
3.84971642494202 14.7006950378418\\
3.94971656799316 14.7006950378418\\
4.04971647262573 14.7006950378418\\
4.1497163772583 14.7006950378418\\
4.24971628189087 14.7006950378418\\
4.3497166633606 14.7006950378418\\
4.44971656799316 14.7006950378418\\
4.54971647262573 14.7006950378418\\
4.6497163772583 14.7006950378418\\
4.74971628189087 14.7006950378418\\
4.8497166633606 14.7006950378418\\
4.94971656799316 14.7006950378418\\
5.04971647262573 14.7006950378418\\
5.1497163772583 14.7006950378418\\
5.24971628189087 14.7006950378418\\
5.3497166633606 14.7006950378418\\
5.44971656799316 14.7006950378418\\
5.54971647262573 14.7006950378418\\
5.6497163772583 14.7006950378418\\
5.74971628189087 14.7006950378418\\
5.8497166633606 14.7006950378418\\
5.94971656799316 14.7006950378418\\
6.04971647262573 14.7006950378418\\
6.1497163772583 14.7006940841675\\
6.24971628189087 14.7006931304932\\
6.3497166633606 14.7006921768188\\
6.44971656799316 14.7006940841675\\
6.54971647262573 14.7006969451904\\
6.6497163772583 14.7006969451904\\
6.74971628189087 14.7006969451904\\
6.8497166633606 14.7006950378418\\
6.94971656799316 14.7006959915161\\
7.04971647262573 14.7006950378418\\
7.1497163772583 14.7006950378418\\
7.24971628189087 14.7006950378418\\
7.3497166633606 14.7006950378418\\
7.44971656799316 14.7006950378418\\
7.54971647262573 14.7006950378418\\
7.6497163772583 14.7006950378418\\
7.74971628189087 14.7006950378418\\
7.8497166633606 14.7006950378418\\
7.94971656799316 14.7006950378418\\
8.04971599578857 14.7006950378418\\
8.1497163772583 14.7006950378418\\
8.24971675872803 14.7006950378418\\
8.34971618652344 14.7006950378418\\
8.44971656799316 14.7006950378418\\
8.54971599578857 14.7006950378418\\
8.6497163772583 14.7006950378418\\
8.74971675872803 14.7006950378418\\
8.84971618652344 14.7006950378418\\
8.94971656799316 14.7006950378418\\
9.04971599578857 14.7006950378418\\
9.1497163772583 14.7006950378418\\
9.24971675872803 14.7006950378418\\
9.34971618652344 14.7006950378418\\
9.44971656799316 14.7006950378418\\
9.54971599578857 14.7006950378418\\
9.6497163772583 14.7006950378418\\
9.74971675872803 14.7006950378418\\
9.84971618652344 14.7006950378418\\
9.94971656799316 14.7006950378418\\
10.0497159957886 14.7006950378418\\
10.1497163772583 14.7006950378418\\
10.249716758728 14.7006950378418\\
10.3497161865234 14.7006950378418\\
10.4497165679932 14.7006950378418\\
10.5497159957886 14.7006950378418\\
10.6497163772583 14.7006950378418\\
10.749716758728 14.7006950378418\\
10.8497161865234 14.7006950378418\\
10.9497165679932 14.7006950378418\\
11.0497159957886 14.7006950378418\\
11.1497163772583 14.7006950378418\\
11.249716758728 14.7006950378418\\
11.3497161865234 14.7006950378418\\
11.4497165679932 14.7006950378418\\
11.5497159957886 14.7006950378418\\
11.6497163772583 14.7006959915161\\
11.749716758728 14.7006959915161\\
11.8497161865234 14.7006959915161\\
11.9497165679932 14.7006950378418\\
12.0497159957886 14.7006950378418\\
12.1497163772583 14.7006950378418\\
12.249716758728 14.7006950378418\\
12.347146987915 14.7006950378418\\
12.4250946044922 14.7006950378418\\
12.4874591827393 14.7006959915161\\
12.5560598373413 14.7006940841675\\
12.6315202713013 14.7006940841675\\
12.714527130127 14.7006950378418\\
12.8058338165283 14.7006959915161\\
12.8497476577759 14.7006959915161\\
12.8980522155762 14.7006950378418\\
12.9143037796021 14.7006959915161\\
12.9277105331421 14.7006978988647\\
12.9424571990967 14.7006978988647\\
12.9586782455444 14.7006969451904\\
12.9765224456787 14.7006969451904\\
12.9961500167847 14.7006959915161\\
13.0177412033081 14.7006959915161\\
13.0414915084839 14.7006959915161\\
13.0676164627075 14.7006959915161\\
13.0963535308838 14.7006959915161\\
13.1279649734497 14.7006959915161\\
13.1627368927002 14.7006959915161\\
13.2009868621826 14.7006959915161\\
13.2430610656738 14.7006959915161\\
13.289342880249 14.7006959915161\\
13.3402528762817 14.7006959915161\\
13.3962545394897 14.7006959915161\\
13.4578561782837 14.7006959915161\\
13.5256175994873 14.7006959915161\\
13.600154876709 14.7006959915161\\
13.6821460723877 14.7006959915161\\
13.7723369598389 14.7006969451904\\
13.871545791626 14.7006959915161\\
13.9715461730957 14.7007293701172\\
14.0715456008911 14.7008724212646\\
14.1215467453003 14.7015933990479\\
14.147439956665 14.7021436691284\\
14.1759233474731 14.7775974273682\\
14.1775197982788 14.8057174682617\\
14.1792755126953 14.8608150482178\\
14.1812076568604 14.878981590271\\
14.1833324432373 14.8715009689331\\
14.1856698989868 14.8500652313232\\
14.1882400512695 14.8300609588623\\
14.191068649292 14.8133125305176\\
14.1941795349121 14.7969388961792\\
14.1976013183594 14.7934789657593\\
14.2013654708862 14.7855930328369\\
14.2055063247681 14.7780284881592\\
14.2100610733032 14.7707757949829\\
14.2150716781616 14.7635793685913\\
14.2205829620361 14.7563791275024\\
14.2266454696655 14.7492876052856\\
14.2333135604858 14.7424945831299\\
14.2406492233276 14.7362041473389\\
14.2487182617188 14.7298812866211\\
14.2575941085815 14.718391418457\\
14.2673578262329 14.7074756622314\\
14.27809715271 14.7032098770142\\
14.2899112701416 14.7014875411987\\
14.3029069900513 14.7006855010986\\
14.3172016143799 14.7004013061523\\
14.3329257965088 14.7003364562988\\
14.3502225875854 14.7003593444824\\
14.3692483901978 14.7004528045654\\
14.3901777267456 14.7005615234375\\
14.4131994247437 14.7007923126221\\
14.4385232925415 14.7030200958252\\
14.4663791656494 14.7026329040527\\
14.4970216751099 14.7008476257324\\
14.520975112915 14.7006750106812\\
14.5473232269287 14.7006874084473\\
14.5763063430786 14.7007102966309\\
14.6081876754761 14.7007188796997\\
14.6432571411133 14.7007179260254\\
14.6818342208862 14.7007179260254\\
14.7242679595947 14.7007083892822\\
14.7709455490112 14.7007083892822\\
14.8222913742065 14.7007064819336\\
14.8712491989136 14.7007093429565\\
14.9185028076172 14.7006978988647\\
14.9638071060181 14.7006969451904\\
15.0066013336182 14.7006959915161\\
15.0464649200439 14.7006940841675\\
15.0903158187866 14.7006950378418\\
15.1385507583618 14.7006950378418\\
15.1916093826294 14.7006959915161\\
15.2499742507935 14.7006959915161\\
15.3141756057739 14.7006959915161\\
15.3847970962524 14.7006959915161\\
15.4624805450439 14.7006969451904\\
15.5479326248169 14.7006969451904\\
15.6419296264648 14.7006959915161\\
15.7419300079346 14.7006969451904\\
15.7919301986694 14.7006950378418\\
15.8469295501709 14.7006978988647\\
15.9074296951294 14.7006959915161\\
15.920783996582 14.7006950378418\\
15.9354724884033 14.7006998062134\\
15.9516305923462 14.7006950378418\\
15.9645795822144 14.7006978988647\\
15.9788236618042 14.7006978988647\\
15.9944925308228 14.7006978988647\\
16.0117282867432 14.7006969451904\\
16.0306873321533 14.7006969451904\\
16.0515422821045 14.7006969451904\\
16.0744819641113 14.7006969451904\\
16.0997161865234 14.7006969451904\\
16.1274738311768 14.7006969451904\\
16.1580085754395 14.7006969451904\\
16.1915950775146 14.7006969451904\\
16.2285404205322 14.7006969451904\\
16.2691822052002 14.7006969451904\\
16.3138847351074 14.7006969451904\\
16.3630599975586 14.7006969451904\\
16.4171524047852 14.7006969451904\\
16.4766540527344 14.7006969451904\\
16.542106628418 14.7006969451904\\
16.6141033172607 14.7006969451904\\
16.6932983398438 14.7006969451904\\
16.7804145812988 14.7006959915161\\
16.87624168396 14.7006969451904\\
16.9762420654297 14.7006959915161\\
17.0762424468994 14.7007169723511\\
17.0962505340576 14.700737953186\\
17.1021957397461 14.7007350921631\\
17.1087341308594 14.7007303237915\\
17.1159267425537 14.7007312774658\\
17.1238384246826 14.7007360458374\\
17.1325416564941 14.700740814209\\
17.1421165466309 14.7007436752319\\
17.1526470184326 14.7007446289063\\
17.1642322540283 14.700740814209\\
17.1769752502441 14.7007369995117\\
17.1909923553467 14.7007350921631\\
17.2064094543457 14.7007312774658\\
17.2233715057373 14.7007265090942\\
17.2420272827148 14.7007188796997\\
17.2625503540039 14.700704574585\\
17.2851238250732 14.7006978988647\\
17.3099555969238 14.7007093429565\\
17.3372707366943 14.7007665634155\\
17.367317199707 14.7014036178589\\
17.400369644165 14.7046594619751\\
17.4367256164551 14.7127485275269\\
17.4767169952393 14.7173728942871\\
17.5207099914551 14.7151708602905\\
17.5690994262695 14.713773727417\\
17.6223297119141 14.7127704620361\\
17.6808815002441 14.7103567123413\\
17.745288848877 14.7067518234253\\
17.8161373138428 14.7031755447388\\
17.8624935150146 14.7012090682983\\
17.9028415679932 14.7008924484253\\
17.9472217559814 14.700831413269\\
17.9960422515869 14.7008447647095\\
18.0497417449951 14.700891494751\\
18.108814239502 14.7009944915771\\
18.173791885376 14.7012376785278\\
18.245267868042 14.7018308639526\\
18.3238925933838 14.7033967971802\\
18.41037940979 14.7074222564697\\
18.5055141448975 14.7157716751099\\
18.6055145263672 14.7264890670776\\
18.7055149078369 14.7307796478271\\
18.805513381958 14.7326993942261\\
18.9055137634277 14.7329883575439\\
19.0055141448975 14.7327213287354\\
19.1055145263672 14.7330560684204\\
19.2055149078369 14.733494758606\\
19.305513381958 14.7341642379761\\
19.4055137634277 14.7345790863037\\
19.5055141448975 14.7351608276367\\
19.6055145263672 14.7356328964233\\
19.7055149078369 14.736177444458\\
19.805513381958 14.7366733551025\\
19.9055137634277 14.7372274398804\\
20.0055141448975 14.7377290725708\\
20.1055145263672 14.7382364273071\\
20.2055149078369 14.7387504577637\\
20.305513381958 14.7392625808716\\
20.4055137634277 14.7397756576538\\
20.5055141448975 14.740288734436\\
20.6055145263672 14.7408027648926\\
20.7055149078369 14.7413158416748\\
20.805513381958 14.7418308258057\\
20.9055137634277 14.7423448562622\\
21.0055141448975 14.7428598403931\\
21.1055145263672 14.7433738708496\\
21.2055149078369 14.7439641952515\\
21.305513381958 14.7447643280029\\
21.4055137634277 14.7453393936157\\
21.5055141448975 14.7459745407104\\
21.6055145263672 14.7465906143188\\
21.7055149078369 14.7472105026245\\
21.805513381958 14.7478284835815\\
21.9055137634277 14.7484445571899\\
22.0055141448975 14.749059677124\\
22.1055145263672 14.7496738433838\\
22.2055149078369 14.7502870559692\\
22.305513381958 14.7508983612061\\
22.4055137634277 14.7515096664429\\
22.5055141448975 14.7521190643311\\
22.6055145263672 14.7527275085449\\
22.7055149078369 14.7533349990845\\
22.805513381958 14.7537393569946\\
22.9055137634277 14.7541742324829\\
23.0055141448975 14.7547082901001\\
23.1055145263672 14.7552013397217\\
23.2055149078369 14.7557010650635\\
23.305513381958 14.7561902999878\\
23.4055137634277 14.7566728591919\\
23.5055141448975 14.7571449279785\\
23.6055145263672 14.757607460022\\
23.7055149078369 14.7580575942993\\
23.805513381958 14.7584924697876\\
23.9055137634277 14.7589120864868\\
24.0055141448975 14.7626638412476\\
24.1055145263672 14.7619619369507\\
24.2055149078369 14.7618417739868\\
24.305513381958 14.7625942230225\\
24.4055137634277 14.7631692886353\\
24.5055141448975 14.7638778686523\\
24.6055145263672 14.764627456665\\
24.7055149078369 14.7717323303223\\
24.789867401123 14.7757654190063\\
24.8573379516602 14.7719659805298\\
24.9315567016602 14.7685289382935\\
25.0131969451904 14.7688999176025\\
25.1030006408691 14.78590965271\\
25.1443557739258 14.7343425750732\\
25.189847946167 14.7643365859985\\
25.2099170684814 14.7778463363647\\
25.2266445159912 14.7797927856445\\
25.245044708252 14.7712144851685\\
25.2652854919434 14.7712774276733\\
25.2875499725342 14.7714157104492\\
25.3120403289795 14.7715845108032\\
25.3389797210693 14.7717905044556\\
25.3686122894287 14.7720375061035\\
25.4012107849121 14.7723255157471\\
25.4370670318604 14.7726583480835\\
25.4765090942383 14.7730369567871\\
25.5198955535889 14.7734642028809\\
25.5676212310791 14.7739429473877\\
25.6201171875 14.7744731903076\\
25.6778659820557 14.7750606536865\\
25.7413883209229 14.7757062911987\\
25.8112621307373 14.7764177322388\\
25.8881244659424 14.7771997451782\\
25.9726715087891 14.7871103286743\\
26.065673828125 14.7704029083252\\
26.1656742095947 14.7801122665405\\
26.2656745910645 14.7836456298828\\
26.3656749725342 14.7887620925903\\
26.409065246582 14.8607425689697\\
26.4240894317627 14.82839012146\\
26.4268970489502 14.8374729156494\\
26.4299831390381 14.8749341964722\\
26.4333801269531 14.9023380279541\\
26.4371166229248 14.8974323272705\\
26.4412250518799 14.8754510879517\\
26.4457454681396 14.8496561050415\\
26.4507179260254 14.8444328308105\\
26.4561862945557 14.8393049240112\\
26.4622020721436 14.8323955535889\\
26.4688205718994 14.8242340087891\\
26.4761009216309 14.8148803710938\\
26.4841079711914 14.7992858886719\\
26.4929161071777 14.7720003128052\\
26.5026054382324 14.7419347763062\\
26.5132637023926 14.7116203308105\\
26.5249881744385 14.6989736557007\\
26.5378837585449 14.7000589370728\\
26.5520706176758 14.7008085250854\\
26.5676746368408 14.700888633728\\
26.5848388671875 14.7009897232056\\
26.6037216186523 14.7068471908569\\
26.624490737915 14.7774820327759\\
26.6473369598389 14.8038082122803\\
26.6724681854248 14.8295211791992\\
26.7001132965088 14.8111772537231\\
26.7305221557617 14.8129739761353\\
26.7594413757324 14.8204298019409\\
26.7881412506104 14.8210220336914\\
26.8180274963379 14.8170986175537\\
26.848762512207 14.8131685256958\\
26.8805618286133 14.8091526031494\\
26.9128971099854 14.8065423965454\\
26.9456844329834 14.8045520782471\\
26.97873878479 14.8029336929321\\
27.0120601654053 14.8016910552979\\
27.0457935333252 14.8010110855103\\
27.0821418762207 14.8004035949707\\
27.1221237182617 14.8002653121948\\
27.1524639129639 14.8000402450562\\
27.1838111877441 14.7993679046631\\
27.2182941436768 14.7987899780273\\
27.2558403015137 14.7985029220581\\
27.2955665588379 14.7996072769165\\
27.3392658233643 14.8007402420044\\
27.3873329162598 14.79944896698\\
27.4402103424072 14.7963056564331\\
27.4983730316162 14.7962818145752\\
27.5623531341553 14.7965641021729\\
27.6327304840088 14.7968549728394\\
27.7101459503174 14.7973413467407\\
27.7953014373779 14.797999382019\\
27.8889751434326 14.8008794784546\\
27.9889755249023 14.7975244522095\\
28.0889739990234 14.8030557632446\\
28.1889743804932 14.7993812561035\\
28.2889747619629 14.8020620346069\\
28.2927665710449 14.8907194137573\\
28.2969398498535 14.7548770904541\\
28.3015289306641 14.7844171524048\\
28.304780960083 14.8194255828857\\
28.307991027832 14.8168888092041\\
28.3115215301514 14.8044271469116\\
28.3154048919678 14.8043069839478\\
28.3196773529053 14.8043622970581\\
28.3243770599365 14.8043603897095\\
28.3295459747314 14.8043613433838\\
28.3352317810059 14.8043661117554\\
28.3414859771729 14.8043756484985\\
28.3483657836914 14.8043899536133\\
28.3559341430664 14.8044128417969\\
28.3642578125 14.8044424057007\\
28.3734169006348 14.804482460022\\
28.3834896087646 14.8045330047607\\
28.3945693969727 14.8045949935913\\
28.4067573547363 14.8046722412109\\
28.4201641082764 14.8047647476196\\
28.4349136352539 14.8048753738403\\
28.451135635376 14.8050050735474\\
28.4689807891846 14.805154800415\\
28.4886093139648 14.8053283691406\\
28.5102024078369 14.8055267333984\\
28.5339546203613 14.8057518005371\\
28.5600814819336 14.8060054779053\\
28.5888214111328 14.8062887191772\\
28.620433807373 14.8066053390503\\
28.6552085876465 14.8069562911987\\
28.6934604644775 14.8073453903198\\
28.735538482666 14.8077745437622\\
28.7818241119385 14.8082475662231\\
28.8327369689941 14.8087692260742\\
28.8887424468994 14.8093423843384\\
28.9503479003906 14.8099737167358\\
29.0181159973145 14.8106679916382\\
29.092658996582 14.8114318847656\\
29.1746559143066 14.8122720718384\\
29.2648525238037 14.8150081634521\\
29.3640689849854 14.8122806549072\\
29.4640693664551 14.8154392242432\\
29.4847068786621 14.8518590927124\\
29.4888343811035 14.839527130127\\
29.4933738708496 14.8358459472656\\
29.4983692169189 14.8380994796753\\
29.5038623809814 14.8434314727783\\
29.5099067687988 14.8490161895752\\
29.5165538787842 14.8538227081299\\
29.5230560302734 14.857590675354\\
29.5302066802979 14.8581743240356\\
29.5380744934082 14.8586883544922\\
29.5467281341553 14.8610038757324\\
29.556245803833 14.8771142959595\\
29.5667171478271 14.9156627655029\\
29.5782356262207 14.9348030090332\\
29.5909042358398 14.9245138168335\\
29.6048412322998 14.8996801376343\\
29.6201705932617 14.8828029632568\\
29.637035369873 14.8636388778687\\
29.6555843353271 14.8414516448975\\
29.675989151001 14.8131999969482\\
29.6984329223633 14.7784404754639\\
29.7114696502686 14.7655954360962\\
29.7258071899414 14.7571582794189\\
29.7415809631348 14.7515525817871\\
29.7589302062988 14.7506418228149\\
29.7780151367188 14.7692098617554\\
29.7990093231201 14.7995586395264\\
29.8221015930176 14.8309669494629\\
29.8475036621094 14.8500289916992\\
29.8754444122314 14.8471164703369\\
29.9061813354492 14.8495998382568\\
29.9399909973145 14.8428249359131\\
29.9771823883057 14.8395290374756\\
30.0180931091309 14.8460636138916\\
30.0630931854248 14.8451824188232\\
30.1125946044922 14.8372097015381\\
30.1670475006104 14.8400993347168\\
30.2269439697266 14.831151008606\\
30.2706527709961 14.8286352157593\\
30.318733215332 14.8240871429443\\
30.3593273162842 14.8222198486328\\
30.3947944641113 14.8261480331421\\
30.4330863952637 14.8293514251709\\
30.4652576446533 14.8309202194214\\
30.5006484985352 14.8292961120605\\
30.5330505371094 14.8294124603271\\
30.5677318572998 14.8300466537476\\
30.6004276275635 14.8304691314697\\
30.6349658966064 14.8307609558105\\
30.6682662963867 14.8312969207764\\
30.7048969268799 14.8318872451782\\
30.7451915740967 14.8327989578247\\
30.789514541626 14.8341617584229\\
30.8382682800293 14.8362503051758\\
30.8918991088867 14.8392238616943\\
30.9508934020996 14.8427104949951\\
31.0157871246338 14.843542098999\\
31.0871696472168 14.8447504043579\\
31.1656913757324 14.8367958068848\\
31.2520637512207 14.8380374908447\\
31.347074508667 14.8391056060791\\
31.4470748901367 14.8383436203003\\
31.5470752716064 14.8391494750977\\
31.6470737457275 14.8402299880981\\
31.7470741271973 14.8414936065674\\
31.847074508667 14.8425025939941\\
31.9470748901367 14.843677520752\\
32.0470733642578 14.8447484970093\\
32.1470756530762 14.8458871841431\\
32.2470741271973 14.8470010757446\\
32.3470764160156 14.8481121063232\\
32.4470748901367 14.8492269515991\\
32.5470733642578 14.8503522872925\\
32.6470756530762 14.8514709472656\\
32.7470741271973 14.852593421936\\
32.8470764160156 14.8537139892578\\
32.9470748901367 14.8548364639282\\
33.0470733642578 14.8559579849243\\
33.1470756530762 14.8570795059204\\
33.2470741271973 14.8582010269165\\
33.3470764160156 14.8593225479126\\
33.4470748901367 14.8604440689087\\
33.5470733642578 14.8615655899048\\
33.6470756530762 14.8627634048462\\
33.7470741271973 14.864143371582\\
33.8470764160156 14.8651962280273\\
33.9470748901367 14.8663396835327\\
34.0470733642578 14.8674554824829\\
34.1470756530762 14.8685789108276\\
34.2470741271973 14.869701385498\\
34.3470764160156 14.8708238601685\\
34.4470748901367 14.8719453811646\\
34.5470733642578 14.873067855835\\
34.6470756530762 14.8741903305054\\
34.7470741271973 14.8753128051758\\
34.8470764160156 14.8764352798462\\
34.9470748901367 14.8775577545166\\
35.0470733642578 14.878680229187\\
35.1470756530762 14.8798027038574\\
35.2470741271973 14.8806991577148\\
35.3470764160156 14.8817405700684\\
35.4470748901367 14.8828716278076\\
35.5470733642578 14.8839721679688\\
35.6470756530762 14.8850812911987\\
35.7470741271973 14.8861856460571\\
35.8470764160156 14.8872880935669\\
35.9470748901367 14.8883876800537\\
36.0470733642578 14.8894834518433\\
36.1470756530762 14.8905754089355\\
36.2470741271973 14.8916635513306\\
36.3470764160156 14.8927478790283\\
36.4470748901367 14.8938283920288\\
36.5470733642578 14.8986015319824\\
36.6470756530762 14.8973903656006\\
36.7470741271973 14.8975992202759\\
36.8470764160156 14.8989562988281\\
36.9470748901367 14.9001150131226\\
37.0470733642578 14.901834487915\\
37.1470756530762 14.9096908569336\\
37.2293395996094 14.912576675415\\
37.2951393127441 14.9065322875977\\
37.3675231933594 14.904953956604\\
37.4471435546875 14.9055604934692\\
37.534725189209 14.9062042236328\\
37.6310653686523 14.9220752716064\\
37.6772041320801 14.8772964477539\\
37.7279586791992 14.905291557312\\
37.7453880310059 14.9178409576416\\
37.7602653503418 14.9122505187988\\
37.7766304016113 14.9099416732788\\
37.7946319580078 14.9100179672241\\
37.8144378662109 14.9101324081421\\
37.8362197875977 14.9102745056152\\
37.860179901123 14.9104528427124\\
37.8865356445313 14.9106674194336\\
37.9155311584473 14.9109201431274\\
37.9474220275879 14.9112129211426\\
37.9825019836426 14.9115476608276\\
38.0210914611816 14.9119262695313\\
38.0635414123535 14.9123506546021\\
38.1102333068848 14.9128227233887\\
38.1615982055664 14.9133462905884\\
38.2180976867676 14.9139242172241\\
38.2802429199219 14.9145603179932\\
38.3486099243164 14.9152612686157\\
38.4238090515137 14.9160318374634\\
38.5065269470215 14.925368309021\\
38.5975189208984 14.9098205566406\\
38.6975212097168 14.9189682006836\\
38.7975196838379 14.9233179092407\\
38.8975219726563 14.9264163970947\\
38.9475212097168 14.9686193466187\\
38.9856338500977 14.9172391891479\\
39.0002326965332 14.9214525222778\\
39.0162887573242 14.9366588592529\\
39.0339508056641 14.8814353942871\\
39.0533828735352 14.8854417800903\\
39.0747528076172 14.8899202346802\\
39.0982627868652 14.9004173278809\\
39.1241226196289 14.8960981369019\\
39.1525688171387 14.8973960876465\\
39.1838569641113 14.8936700820923\\
39.2182769775391 14.8881578445435\\
39.2382202148438 14.8905925750732\\
39.2601547241211 14.8938446044922\\
39.2842864990234 14.9056739807129\\
39.3108291625977 14.9157371520996\\
39.3400230407715 14.925404548645\\
39.3721389770508 14.9335298538208\\
39.4047470092773 14.944239616394\\
39.4365348815918 14.9465713500977\\
39.4690742492676 14.9404048919678\\
39.5013847351074 14.9356250762939\\
39.5341873168945 14.9347867965698\\
39.5668563842773 14.9344797134399\\
39.5998191833496 14.9341306686401\\
39.632698059082 14.9339904785156\\
39.6657829284668 14.9344120025635\\
39.6988639831543 14.9350862503052\\
39.7321624755859 14.9351663589478\\
39.7687873840332 14.9350719451904\\
39.8090744018555 14.9351491928101\\
39.8533935546875 14.9353218078613\\
39.8968162536621 14.9364604949951\\
39.939453125 14.9392719268799\\
39.9863548278809 14.9371528625488\\
40.0379447937012 14.9345474243164\\
40.0946960449219 14.9347305297852\\
40.1571197509766 14.9352922439575\\
40.2257843017578 14.9356803894043\\
40.3013191223145 14.9363069534302\\
40.384407043457 14.9370546340942\\
40.4758033752441 14.9398994445801\\
40.5758018493652 14.9368915557861\\
40.6758041381836 14.9421501159668\\
40.7758026123047 14.9388818740845\\
40.875804901123 14.9412860870361\\
40.8792419433594 15.032265663147\\
40.8830223083496 14.9034938812256\\
40.8871841430664 14.9183750152588\\
40.8903007507324 14.9593820571899\\
40.8937301635742 14.9576196670532\\
40.8975028991699 14.9448156356812\\
40.9016532897949 14.9440612792969\\
40.9062156677246 14.9444046020508\\
40.9112396240234 14.9444313049316\\
40.9167633056641 14.9444246292114\\
40.9228363037109 14.9444265365601\\
40.9295196533203 14.9444351196289\\
40.9368743896484 14.944450378418\\
40.9449615478516 14.9444723129272\\
40.953857421875 14.944504737854\\
40.9636421203613 14.9445466995239\\
40.9744033813477 14.94460105896\\
40.9862442016602 14.9446697235107\\
40.999267578125 14.9447536468506\\
41.0135955810547 14.9448547363281\\
41.029354095459 14.944974899292\\
41.046688079834 14.9451160430908\\
41.0657577514648 14.9452800750732\\
41.0867347717285 14.9454679489136\\
41.109806060791 14.9456825256348\\
41.1351852416992 14.9459257125854\\
41.1631050109863 14.9461994171143\\
41.1938133239746 14.9465045928955\\
41.227596282959 14.9468441009521\\
41.2647552490234 14.9472208023071\\
41.3056297302246 14.9476366043091\\
41.3505935668945 14.9480962753296\\
41.4000511169434 14.9486017227173\\
41.4544563293457 14.9491586685181\\
41.5143013000488 14.9497718811035\\
41.5801315307617 14.9504461288452\\
41.6525421142578 14.9511871337891\\
41.7321968078613 14.9520034790039\\
41.8198165893555 14.9546842575073\\
41.9161987304688 14.9519987106323\\
42.0161972045898 14.9550886154175\\
42.1161994934082 14.9631185531616\\
42.1361999511719 14.9661073684692\\
42.1442070007324 14.9880399703979\\
42.1520309448242 14.99036693573\\
42.1606330871582 14.9868831634521\\
42.1689529418945 14.9867601394653\\
42.1781005859375 14.9880237579346\\
42.1881637573242 15.0107698440552\\
42.1992378234863 15.0585737228394\\
42.2114143371582 15.0514297485352\\
42.2248077392578 15.0370216369629\\
42.2395439147949 15.0207929611206\\
42.2557525634766 15.0016269683838\\
42.2735824584961 14.9792566299438\\
42.2931938171387 14.9568243026733\\
42.3147659301758 14.9352531433105\\
42.3384971618652 14.9041957855225\\
42.3646011352539 14.8808307647705\\
42.3847808837891 14.8781642913818\\
42.4069786071777 14.891375541687\\
42.431396484375 14.9047765731812\\
42.4582557678223 14.9187927246094\\
42.4878005981445 14.954158782959\\
42.5203018188477 14.9644584655762\\
42.5560493469238 14.9662570953369\\
42.5953750610352 14.9687767028809\\
42.6386299133301 14.9617357254028\\
42.6862144470215 14.9872074127197\\
42.7385559082031 14.9624004364014\\
42.7959022521973 14.9743537902832\\
42.8589859008789 14.9682159423828\\
42.9028167724609 14.9613094329834\\
42.9510307312012 14.9648466110229\\
42.9925117492676 14.9620046615601\\
43.0297622680664 14.9639711380005\\
43.0707359313965 14.965892791748\\
43.1158065795898 14.9676380157471\\
43.1579742431641 14.9701137542725\\
43.203685760498 14.9728336334229\\
43.2497863769531 14.9750175476074\\
43.3005027770996 14.9739389419556\\
43.3562850952148 14.9768056869507\\
43.4176483154297 14.9750003814697\\
43.4851493835449 14.9773731231689\\
43.559398651123 14.9750623703003\\
43.6410713195801 14.9769802093506\\
43.7309150695801 14.9749279022217\\
43.8297386169434 14.9765748977661\\
43.9297409057617 14.9753122329712\\
44.0297393798828 14.9783744812012\\
44.1297416687012 14.9776916503906\\
44.2297401428223 14.979474067688\\
44.2499618530273 14.9824323654175\\
44.2722015380859 14.9810600280762\\
44.2966690063477 14.9813051223755\\
44.3235816955566 14.9815797805786\\
44.3531875610352 14.9818830490112\\
44.3857498168945 14.9822158813477\\
44.421573638916 14.9825820922852\\
44.4609756469727 14.9829864501953\\
44.5043182373047 14.9834299087524\\
44.5519981384277 14.9839181900024\\
44.6044425964355 14.9844551086426\\
44.6621360778809 14.9850463867188\\
44.7255935668945 14.9856967926025\\
44.7953987121582 14.9864120483398\\
44.872184753418 14.9871988296509\\
44.9566497802734 14.9880638122559\\
45.0495643615723 14.9890165328979\\
45.1495628356934 14.9906549453735\\
45.2495613098145 14.9905576705933\\
45.3495635986328 14.9922256469727\\
45.4495620727539 14.9942312240601\\
45.5495643615723 14.9938936233521\\
45.6495628356934 14.9956569671631\\
45.7495613098145 15.0046863555908\\
45.7695655822754 15.0135383605957\\
45.7816352844238 15.0136184692383\\
45.7949142456055 15.0117835998535\\
45.8095207214355 15.0098190307617\\
45.8255882263184 15.0084562301636\\
45.8432655334473 15.0067701339722\\
45.8627052307129 15.0046443939209\\
45.884090423584 15.0019025802612\\
45.9076156616211 15.000412940979\\
45.9334945678711 14.9987478256226\\
45.9619598388672 14.996955871582\\
45.9932708740234 14.9954690933228\\
46.0277099609375 14.9946060180664\\
46.0655975341797 14.9938383102417\\
46.1072731018066 14.9933156967163\\
46.1531181335449 14.9931516647339\\
46.2035446166992 14.9937629699707\\
46.2590141296387 14.995285987854\\
46.3200302124023 15.0011787414551\\
46.387149810791 15.0042457580566\\
46.4504814147949 15.0056219100952\\
46.5130958557129 15.0063724517822\\
46.581974029541 15.0074634552002\\
46.6577415466309 15.0075750350952\\
46.7410850524902 15.0083589553833\\
46.8327598571777 15.0091886520386\\
46.9327621459961 15.0103254318237\\
47.0327606201172 15.0113801956177\\
47.1327629089355 15.0124368667603\\
47.2327613830566 15.0135288238525\\
47.3327598571777 15.0146369934082\\
47.4327621459961 15.0157623291016\\
47.5327606201172 15.0168914794922\\
47.6327629089355 15.0178375244141\\
47.7327613830566 15.0189170837402\\
47.8327598571777 15.0200023651123\\
47.9327621459961 15.0211210250854\\
48.0327606201172 15.0222406387329\\
48.1327629089355 15.0233669281006\\
48.2327613830566 15.0244922637939\\
48.3327598571777 15.0256175994873\\
48.4327621459961 15.026743888855\\
48.5327606201172 15.0278692245483\\
48.6327629089355 15.0289964675903\\
48.7327613830566 15.0301246643066\\
48.8327598571777 15.0312528610229\\
48.9327621459961 15.0323638916016\\
49.0327606201172 15.0335054397583\\
49.1327629089355 15.0346574783325\\
49.2327613830566 15.0358200073242\\
49.3327598571777 15.0370063781738\\
49.4327621459961 15.0382356643677\\
49.5327606201172 15.0451812744141\\
49.6257057189941 15.0480308532715\\
49.6982688903809 15.0463466644287\\
49.7780876159668 15.0422668457031\\
49.8658905029297 15.0429334640503\\
49.962474822998 15.043628692627\\
50.0624732971191 15.0445795059204\\
50.1624755859375 15.0456581115723\\
50.2624740600586 15.0467720031738\\
50.3624725341797 15.0478754043579\\
50.462474822998 15.0489873886108\\
50.5624732971191 15.0501012802124\\
50.6624755859375 15.0512199401855\\
50.7624740600586 15.0523405075073\\
50.8624725341797 15.0534582138062\\
50.962474822998 15.0545797348022\\
51.0624732971191 15.055703163147\\
51.1624755859375 15.0568284988403\\
51.2624740600586 15.0579538345337\\
51.3624725341797 15.0590801239014\\
51.462474822998 15.0602083206177\\
51.5624732971191 15.0613355636597\\
51.6624755859375 15.0624628067017\\
51.7624740600586 15.063591003418\\
51.8624725341797 15.06471824646\\
51.962474822998 15.0658464431763\\
52.0624732971191 15.0669736862183\\
52.1624755859375 15.0681018829346\\
52.2624740600586 15.0692291259766\\
52.3624725341797 15.0703573226929\\
52.462474822998 15.0714845657349\\
52.5624732971191 15.0726127624512\\
52.6624755859375 15.0737400054932\\
52.7624740600586 15.0748682022095\\
52.8624725341797 15.0759954452515\\
52.962474822998 15.0771226882935\\
53.0624732971191 15.0782508850098\\
53.1624755859375 15.0793781280518\\
53.2624740600586 15.0805063247681\\
53.3624725341797 15.0816335678101\\
53.462474822998 15.0827617645264\\
53.5624732971191 15.0838890075684\\
53.6624755859375 15.0850172042847\\
53.7624740600586 15.0861444473267\\
53.8624725341797 15.0872716903687\\
53.962474822998 15.088399887085\\
54.0624732971191 15.089527130127\\
54.1624755859375 15.0906553268433\\
54.2624740600586 15.0917825698853\\
54.3624725341797 15.0929107666016\\
54.462474822998 15.0940380096436\\
54.5624732971191 15.0951662063599\\
54.6624755859375 15.0962934494019\\
54.7624740600586 15.0974206924438\\
54.8624725341797 15.0985488891602\\
54.962474822998 15.0996761322021\\
55.0624732971191 15.1008043289185\\
55.1624755859375 15.1019344329834\\
55.2624740600586 15.1030683517456\\
55.3624725341797 15.104229927063\\
55.462474822998 15.1055393218994\\
55.5624732971191 15.1075983047485\\
55.6624755859375 15.1122102737427\\
55.7624740600586 15.1196641921997\\
55.8624725341797 15.1206893920898\\
55.962474822998 15.1162099838257\\
56.0624732971191 15.113320350647\\
56.1624755859375 15.1153945922852\\
56.2624740600586 15.1141624450684\\
56.3624725341797 15.1166439056396\\
56.462474822998 15.1168661117554\\
56.5624732971191 15.1185874938965\\
56.6624755859375 15.1193237304688\\
56.7624740600586 15.1207084655762\\
56.8624725341797 15.1216678619385\\
56.962474822998 15.1229066848755\\
57.0624732971191 15.1239624023438\\
57.1624755859375 15.1251392364502\\
57.2624740600586 15.1262350082397\\
57.3624725341797 15.1273851394653\\
57.462474822998 15.1285133361816\\
57.5624732971191 15.1296377182007\\
57.6624755859375 15.1307582855225\\
57.7624740600586 15.1318922042847\\
57.8624725341797 15.133017539978\\
57.962474822998 15.1341495513916\\
58.0624732971191 15.1352767944336\\
58.1624755859375 15.1364078521729\\
58.2624740600586 15.1375360488892\\
58.3624725341797 15.1387300491333\\
58.462474822998 15.1401271820068\\
58.5624732971191 15.1411933898926\\
58.6624755859375 15.1423444747925\\
58.7624740600586 15.1434688568115\\
58.8624725341797 15.1446018218994\\
58.962474822998 15.1457328796387\\
59.0624732971191 15.1468648910522\\
59.1624755859375 15.1479959487915\\
59.2624740600586 15.1491289138794\\
59.3624725341797 15.150260925293\\
59.462474822998 15.1513938903809\\
59.5624732971191 15.1525278091431\\
59.6624755859375 15.1536617279053\\
59.7624740600586 15.1547966003418\\
59.8624725341797 15.1559286117554\\
59.9312362670898 15.1567859649658\\
60 15.157567024231\\
60.0756378173828 15.1583347320557\\
60.1588439941406 15.1591844558716\\
60.2503662109375 15.1601190567017\\
60.3503646850586 15.1611461639404\\
60.450366973877 15.1622591018677\\
60.550365447998 15.1633882522583\\
60.6503677368164 15.1645088195801\\
60.7503662109375 15.1656322479248\\
60.8503646850586 15.1667547225952\\
60.950366973877 15.1678819656372\\
61.050365447998 15.1690435409546\\
61.1503677368164 15.1702175140381\\
61.2503662109375 15.171365737915\\
61.3503646850586 15.1725187301636\\
61.450366973877 15.1736860275269\\
61.550365447998 15.1748733520508\\
61.6503677368164 15.1760959625244\\
61.7503662109375 15.1777496337891\\
61.8503646850586 15.1857328414917\\
61.9325637817383 15.1885213851929\\
61.998291015625 15.1823425292969\\
62.0705947875977 15.1809453964233\\
62.1501274108887 15.1815567016602\\
62.2376098632813 15.182201385498\\
62.3338470458984 15.1830501556396\\
62.4338455200195 15.1840810775757\\
62.5338439941406 15.1851902008057\\
62.633846282959 15.1862955093384\\
62.7338447570801 15.187406539917\\
62.8338470458984 15.1885204315186\\
62.9338455200195 15.189640045166\\
63.0338439941406 15.1907587051392\\
63.133846282959 15.1918773651123\\
63.2338447570801 15.1929998397827\\
63.3338470458984 15.1941242218018\\
63.4338455200195 15.1952495574951\\
63.5338439941406 15.1963777542114\\
63.633846282959 15.1975059509277\\
63.7338447570801 15.198634147644\\
63.8338470458984 15.1997623443604\\
63.9338455200195 15.2008905410767\\
64.0338439941406 15.202018737793\\
64.1338424682617 15.2031469345093\\
64.2338485717773 15.2042751312256\\
64.3338470458984 15.2054033279419\\
64.4338455200195 15.2065315246582\\
64.5338439941406 15.2076587677002\\
64.6338424682617 15.2087869644165\\
64.7338485717773 15.2099151611328\\
64.8338470458984 15.2110433578491\\
64.9338455200195 15.2121715545654\\
65.0338439941406 15.2132997512817\\
65.1338424682617 15.214427947998\\
65.2338485717773 15.21555519104\\
65.3338470458984 15.2166833877563\\
65.4338455200195 15.2178115844727\\
65.5338439941406 15.218939781189\\
65.6338424682617 15.2200679779053\\
65.7338485717773 15.2211961746216\\
65.8338470458984 15.2223234176636\\
65.9338455200195 15.2234516143799\\
66.0338439941406 15.2245798110962\\
66.1338424682617 15.2257080078125\\
66.2338485717773 15.2268362045288\\
66.3338470458984 15.2279634475708\\
66.4338455200195 15.2290916442871\\
66.5338439941406 15.2302198410034\\
66.6338424682617 15.2313480377197\\
66.7338485717773 15.232476234436\\
66.8338470458984 15.233603477478\\
66.9338455200195 15.2347316741943\\
67.0338439941406 15.2358598709106\\
67.1338424682617 15.236988067627\\
67.2338485717773 15.2381162643433\\
67.3338470458984 15.2392444610596\\
67.4338455200195 15.2403736114502\\
67.5338439941406 15.2415075302124\\
67.6338424682617 15.2426633834839\\
67.7338485717773 15.2439413070679\\
67.8338470458984 15.2458524703979\\
67.9338455200195 15.2500896453857\\
68.0338439941406 15.2575149536133\\
68.1338424682617 15.2600660324097\\
68.2338485717773 15.2548036575317\\
68.3338470458984 15.2517881393433\\
68.4338455200195 15.2540702819824\\
68.5338439941406 15.2525510787964\\
68.6338424682617 15.2551250457764\\
68.7338485717773 15.2552757263184\\
68.8338470458984 15.2570466995239\\
68.9338455200195 15.2577505111694\\
69.0338439941406 15.2591590881348\\
69.1338424682617 15.2601022720337\\
69.2338485717773 15.2613534927368\\
69.3338470458984 15.2624015808105\\
69.4338455200195 15.2635831832886\\
69.5338439941406 15.2646770477295\\
69.6338424682617 15.2658290863037\\
69.7338485717773 15.2669429779053\\
69.8338470458984 15.2680816650391\\
69.9338455200195 15.2692041397095\\
70.0338439941406 15.2703380584717\\
70.1338424682617 15.2714643478394\\
70.2338485717773 15.2725963592529\\
70.3338470458984 15.2737236022949\\
70.4338455200195 15.2748546600342\\
70.5338439941406 15.2759838104248\\
70.6338424682617 15.2771902084351\\
70.7338485717773 15.2785730361938\\
70.8338470458984 15.2796478271484\\
70.9338455200195 15.2807970046997\\
71.0338439941406 15.2819213867188\\
71.1338424682617 15.283055305481\\
71.2338485717773 15.2841854095459\\
71.3338470458984 15.2853174209595\\
71.4338455200195 15.286449432373\\
71.5338439941406 15.2875814437866\\
71.6338424682617 15.2887144088745\\
71.7338485717773 15.2898473739624\\
71.8338470458984 15.2909803390503\\
71.9338455200195 15.2921152114868\\
72.0338439941406 15.2932500839233\\
72.1338424682617 15.294379234314\\
72.2338485717773 15.2952470779419\\
72.3338470458984 15.296350479126\\
72.4338455200195 15.2974758148193\\
72.5338439941406 15.2985982894897\\
72.6338424682617 15.2997207641602\\
72.7338485717773 15.3008432388306\\
72.8338470458984 15.3019647598267\\
72.9338455200195 15.3030862808228\\
73.0338439941406 15.3042068481445\\
73.1338424682617 15.3053274154663\\
73.2338485717773 15.3064622879028\\
73.3338470458984 15.3076457977295\\
73.4338455200195 15.3088302612305\\
73.5338439941406 15.309983253479\\
73.6338424682617 15.3111391067505\\
73.7338485717773 15.3123121261597\\
73.8338470458984 15.3135023117065\\
73.9338455200195 15.3147373199463\\
74.0338439941406 15.3175983428955\\
74.1323699951172 15.3244285583496\\
74.2112121582031 15.3275899887085\\
74.2755126953125 15.3191509246826\\
74.3462524414063 15.319447517395\\
74.4240570068359 15.3200340270996\\
74.509651184082 15.3206748962402\\
74.6037979125977 15.3215103149414\\
74.7037963867188 15.3225231170654\\
74.8037948608398 15.323634147644\\
74.9019012451172 15.3247394561768\\
75 15.3258295059204\\
};
\addlegendentry{6 "};

\addplot [
color=black,
dashed
]
table[row sep=crcr]{
0 14.700608253479\\
9.99999971718069e-10 14.700608253479\\
2.49999998480632e-09 14.700608253479\\
4.74999994892755e-09 14.700608253479\\
8.1250002281763e-09 14.700608253479\\
1.31874999809156e-08 14.700608253479\\
2.0781250498203e-08 14.700608253479\\
3.21718758300449e-08 14.700608253479\\
4.92578138278077e-08 14.700608253479\\
7.48867208244519e-08 14.700608253479\\
1.13330081319418e-07 14.700608253479\\
1.70995122061868e-07 14.700608253479\\
2.57492672517401e-07 14.700608253479\\
3.8723902662241e-07 14.700608253479\\
5.81858500936505e-07 14.700608253479\\
8.73787769251066e-07 14.700608253479\\
1.31168167172291e-06 14.700608253479\\
1.9685226106958e-06 14.700608253479\\
2.95378367809462e-06 14.700608253479\\
4.43167573394021e-06 14.700608253479\\
6.64851359033491e-06 14.7006092071533\\
9.97376992017962e-06 14.7006092071533\\
1.49616553244414e-05 14.7006092071533\\
2.24434825213393e-05 14.7006101608276\\
3.36662233166862e-05 14.700611114502\\
5.05003372381907e-05 14.7006120681763\\
7.57515081204474e-05 14.7006130218506\\
0.000113628258986864 14.7006149291992\\
0.000170443381648511 14.7006158828735\\
0.000255666091106832 14.7006177902222\\
0.000383500126190484 14.7006187438965\\
0.000575251178815961 14.7006187438965\\
0.000862877757754177 14.7006168365479\\
0.00129431765526533 14.7006139755249\\
0.00194147753063589 14.7006101608276\\
0.00291221728548408 14.7006063461304\\
0.00436832662671804 14.7006044387817\\
0.00597004732117057 14.7006063461304\\
0.00773194013163447 14.700608253479\\
0.009670021943748 14.7006092071533\\
0.0118019115179777 14.7006092071533\\
0.0141469910740852 14.7006092071533\\
0.0167265776544809 14.700608253479\\
0.0195641238242388 14.700608253479\\
0.0226854234933853 14.700608253479\\
0.0261188540607691 14.700608253479\\
0.0298956278711557 14.700608253479\\
0.0340500771999359 14.700608253479\\
0.0386199727654457 14.700608253479\\
0.0436468608677387 14.700608253479\\
0.0491764321923256 14.700608253479\\
0.0552589632570744 14.700608253479\\
0.0619497485458851 14.700608253479\\
0.0693096145987511 14.700608253479\\
0.0774054601788521 14.700608253479\\
0.0863108932971954 14.7006072998047\\
0.0961068719625473 14.7006092071533\\
0.106882445514202 14.7006092071533\\
0.118735581636429 14.7006092071533\\
0.13177402317524 14.7006092071533\\
0.146116316318512 14.7006092071533\\
0.161892831325531 14.7006092071533\\
0.179247006773949 14.7006092071533\\
0.198336601257324 14.7006092071533\\
0.21933513879776 14.7006092071533\\
0.242433547973633 14.700608253479\\
0.267841786146164 14.7006092071533\\
0.29579085111618 14.7006092071533\\
0.32653483748436 14.7006092071533\\
0.360353201627731 14.7006092071533\\
0.397553414106369 14.7006092071533\\
0.438473641872406 14.7006092071533\\
0.483485877513886 14.7006092071533\\
0.532999336719513 14.7006092071533\\
0.587464153766632 14.7006092071533\\
0.647375464439392 14.7006092071533\\
0.71327793598175 14.7006092071533\\
0.7857705950737 14.7006092071533\\
0.865512549877167 14.7006092071533\\
0.953228712081909 14.7006092071533\\
1.04971647262573 14.700608253479\\
1.14971649646759 14.700608253479\\
1.24971640110016 14.700608253479\\
1.34971642494202 14.700608253479\\
1.44971644878387 14.700608253479\\
1.54971647262573 14.700608253479\\
1.64971649646759 14.700608253479\\
1.74971640110016 14.700608253479\\
1.84971642494202 14.700608253479\\
1.94971644878387 14.700608253479\\
2.04971647262573 14.700608253479\\
2.1497163772583 14.700608253479\\
2.24971652030945 14.700608253479\\
2.34971642494202 14.700608253479\\
2.44971656799316 14.700608253479\\
2.54971647262573 14.700608253479\\
2.6497163772583 14.700608253479\\
2.74971652030945 14.700608253479\\
2.84971642494202 14.700608253479\\
2.94971656799316 14.700608253479\\
3.04971647262573 14.700608253479\\
3.1497163772583 14.700608253479\\
3.24971652030945 14.700608253479\\
3.34971642494202 14.700608253479\\
3.44971656799316 14.700608253479\\
3.54971647262573 14.700608253479\\
3.6497163772583 14.700608253479\\
3.74971652030945 14.700608253479\\
3.84971642494202 14.700608253479\\
3.94971656799316 14.700608253479\\
4.04971647262573 14.700608253479\\
4.1497163772583 14.700608253479\\
4.24971628189087 14.700608253479\\
4.3497166633606 14.700608253479\\
4.44971656799316 14.700608253479\\
4.54971647262573 14.700608253479\\
4.6497163772583 14.700608253479\\
4.74971628189087 14.700608253479\\
4.8497166633606 14.700608253479\\
4.94971656799316 14.700608253479\\
5.04971647262573 14.700608253479\\
5.1497163772583 14.700608253479\\
5.24971628189087 14.700608253479\\
5.3497166633606 14.700608253479\\
5.44971656799316 14.700608253479\\
5.54971647262573 14.700608253479\\
5.6497163772583 14.700608253479\\
5.74971628189087 14.700608253479\\
5.8497166633606 14.700608253479\\
5.94971656799316 14.700608253479\\
6.04971647262573 14.700608253479\\
6.1497163772583 14.7006072998047\\
6.24971628189087 14.7006063461304\\
6.3497166633606 14.7006053924561\\
6.44971656799316 14.7006072998047\\
6.54971647262573 14.7006092071533\\
6.6497163772583 14.7006101608276\\
6.74971628189087 14.7006101608276\\
6.8497166633606 14.700608253479\\
6.94971656799316 14.700608253479\\
7.04971647262573 14.700608253479\\
7.1497163772583 14.700608253479\\
7.24971628189087 14.700608253479\\
7.3497166633606 14.700608253479\\
7.44971656799316 14.700608253479\\
7.54971647262573 14.700608253479\\
7.6497163772583 14.700608253479\\
7.74971628189087 14.700608253479\\
7.8497166633606 14.700608253479\\
7.94971656799316 14.700608253479\\
8.04971599578857 14.700608253479\\
8.1497163772583 14.700608253479\\
8.24971675872803 14.700608253479\\
8.34971618652344 14.700608253479\\
8.44971656799316 14.700608253479\\
8.54971599578857 14.700608253479\\
8.6497163772583 14.700608253479\\
8.74971675872803 14.700608253479\\
8.84971618652344 14.700608253479\\
8.94971656799316 14.700608253479\\
9.04971599578857 14.700608253479\\
9.1497163772583 14.700608253479\\
9.24971675872803 14.700608253479\\
9.34971618652344 14.700608253479\\
9.44971656799316 14.700608253479\\
9.54971599578857 14.700608253479\\
9.6497163772583 14.700608253479\\
9.74971675872803 14.700608253479\\
9.84971618652344 14.700608253479\\
9.94971656799316 14.700608253479\\
10.0497159957886 14.700608253479\\
10.1497163772583 14.700608253479\\
10.249716758728 14.700608253479\\
10.3497161865234 14.700608253479\\
10.4497165679932 14.700608253479\\
10.5497159957886 14.700608253479\\
10.6497163772583 14.700608253479\\
10.749716758728 14.700608253479\\
10.8497161865234 14.700608253479\\
10.9497165679932 14.700608253479\\
11.0497159957886 14.700608253479\\
11.1497163772583 14.700608253479\\
11.249716758728 14.700608253479\\
11.3497161865234 14.700608253479\\
11.4497165679932 14.700608253479\\
11.5497159957886 14.700608253479\\
11.6497163772583 14.700608253479\\
11.749716758728 14.700608253479\\
11.8497161865234 14.700608253479\\
11.9497165679932 14.700608253479\\
12.0497159957886 14.700608253479\\
12.1497163772583 14.700608253479\\
12.249716758728 14.700608253479\\
12.347146987915 14.700608253479\\
12.4250946044922 14.7006092071533\\
12.4874591827393 14.7006101608276\\
12.5560598373413 14.700608253479\\
12.6315202713013 14.7006072998047\\
12.714527130127 14.700608253479\\
12.8058338165283 14.700608253479\\
12.8497476577759 14.7006101608276\\
12.8980522155762 14.7006072998047\\
12.9143037796021 14.700608253479\\
12.9277105331421 14.7006101608276\\
12.9424571990967 14.7006092071533\\
12.9586782455444 14.7006092071533\\
12.9765224456787 14.700608253479\\
12.9961500167847 14.700608253479\\
13.0177412033081 14.700608253479\\
13.0414915084839 14.700608253479\\
13.0676164627075 14.700608253479\\
13.0963535308838 14.700608253479\\
13.1279649734497 14.700608253479\\
13.1627368927002 14.700608253479\\
13.2009868621826 14.700608253479\\
13.2430610656738 14.700608253479\\
13.289342880249 14.700608253479\\
13.3402528762817 14.700608253479\\
13.3962545394897 14.700608253479\\
13.4578561782837 14.700608253479\\
13.5256175994873 14.700608253479\\
13.600154876709 14.700608253479\\
13.6821460723877 14.7006072998047\\
13.7723369598389 14.700608253479\\
13.871545791626 14.700608253479\\
13.9715461730957 14.7006368637085\\
14.0715456008911 14.7007598876953\\
14.1215467453003 14.7013568878174\\
14.147439956665 14.701621055603\\
14.1759233474731 14.7025394439697\\
14.1775197982788 14.6906423568726\\
14.1792755126953 14.6931867599487\\
14.1812076568604 14.6985626220703\\
14.1833324432373 14.7019205093384\\
14.1856698989868 14.7030448913574\\
14.1882400512695 14.7021379470825\\
14.191068649292 14.7021017074585\\
14.1941795349121 14.7017002105713\\
14.1976013183594 14.7017459869385\\
14.2013654708862 14.701545715332\\
14.2055063247681 14.7012624740601\\
14.2100610733032 14.7010364532471\\
14.2150716781616 14.7008752822876\\
14.2205829620361 14.7007360458374\\
14.2266454696655 14.7006206512451\\
14.2333135604858 14.7005023956299\\
14.2406492233276 14.7003955841064\\
14.2487182617188 14.7003631591797\\
14.2575941085815 14.7004070281982\\
14.2673578262329 14.7001333236694\\
14.27809715271 14.7002182006836\\
14.2899112701416 14.7003011703491\\
14.3029069900513 14.7003002166748\\
14.3172016143799 14.7002716064453\\
14.3329257965088 14.7002735137939\\
14.3502225875854 14.700309753418\\
14.3692483901978 14.7003936767578\\
14.3901777267456 14.7004871368408\\
14.4131994247437 14.7006158828735\\
14.4385232925415 14.7005462646484\\
14.4663791656494 14.7005920410156\\
14.4970216751099 14.7006120681763\\
14.520975112915 14.700590133667\\
14.5473232269287 14.7006053924561\\
14.5763063430786 14.7006092071533\\
14.6081876754761 14.7006158828735\\
14.6432571411133 14.7006158828735\\
14.6818342208862 14.7006168365479\\
14.7242679595947 14.7006101608276\\
14.7709455490112 14.7006101608276\\
14.8222913742065 14.7006092071533\\
14.8712491989136 14.7006130218506\\
14.9185028076172 14.700611114502\\
14.9638071060181 14.7006101608276\\
15.0066013336182 14.7006101608276\\
15.0464649200439 14.7006092071533\\
15.0903158187866 14.7006092071533\\
15.1385507583618 14.700608253479\\
15.1916093826294 14.700608253479\\
15.2499742507935 14.7006092071533\\
15.3141756057739 14.700608253479\\
15.3847970962524 14.700608253479\\
15.4624805450439 14.700608253479\\
15.5479326248169 14.700608253479\\
15.6419296264648 14.7006072998047\\
15.7419300079346 14.7006092071533\\
15.7919301986694 14.7006072998047\\
15.8469295501709 14.7006092071533\\
15.9074296951294 14.7006072998047\\
15.920783996582 14.7006072998047\\
15.9354724884033 14.700611114502\\
15.9516305923462 14.7006072998047\\
15.9645795822144 14.7006092071533\\
15.9788236618042 14.7006092071533\\
15.9944925308228 14.7006092071533\\
16.0117282867432 14.7006092071533\\
16.0306873321533 14.7006092071533\\
16.0515422821045 14.700608253479\\
16.0744819641113 14.700608253479\\
16.0997161865234 14.700608253479\\
16.1274738311768 14.700608253479\\
16.1580085754395 14.700608253479\\
16.1915950775146 14.700608253479\\
16.2285404205322 14.700608253479\\
16.2691822052002 14.700608253479\\
16.3138847351074 14.700608253479\\
16.3630599975586 14.700608253479\\
16.4171524047852 14.700608253479\\
16.4766540527344 14.700608253479\\
16.542106628418 14.700608253479\\
16.6141033172607 14.700608253479\\
16.6932983398438 14.700608253479\\
16.7804145812988 14.7006072998047\\
16.87624168396 14.7006092071533\\
16.9762420654297 14.700608253479\\
17.0762424468994 14.7006254196167\\
17.0962505340576 14.7006425857544\\
17.1021957397461 14.7006397247314\\
17.1087341308594 14.7006349563599\\
17.1159267425537 14.7006349563599\\
17.1238384246826 14.7006387710571\\
17.1325416564941 14.7006425857544\\
17.1421165466309 14.700644493103\\
17.1526470184326 14.7006454467773\\
17.1642322540283 14.7006425857544\\
17.1769752502441 14.7006387710571\\
17.1909923553467 14.7006378173828\\
17.2064094543457 14.7006349563599\\
17.2233715057373 14.7006301879883\\
17.2420272827148 14.7006244659424\\
17.2625503540039 14.7006130218506\\
17.2851238250732 14.7006053924561\\
17.3099555969238 14.7006006240845\\
17.3372707366943 14.7005958557129\\
17.367317199707 14.700587272644\\
17.400369644165 14.7005615234375\\
17.4367256164551 14.7005290985107\\
17.4767169952393 14.7006139755249\\
17.5207099914551 14.700611114502\\
17.5690994262695 14.700608253479\\
17.6223297119141 14.7006053924561\\
17.6808815002441 14.7006053924561\\
17.745288848877 14.7006139755249\\
17.8161373138428 14.7006053924561\\
17.8624935150146 14.7005977630615\\
17.9028415679932 14.7005968093872\\
17.9472217559814 14.7006063461304\\
17.9960422515869 14.7006072998047\\
18.0497417449951 14.700608253479\\
18.108814239502 14.700608253479\\
18.173791885376 14.700608253479\\
18.245267868042 14.700608253479\\
18.3238925933838 14.7006063461304\\
18.41037940979 14.7006044387817\\
18.5055141448975 14.7006092071533\\
18.6055145263672 14.7006063461304\\
18.7055149078369 14.700608253479\\
18.805513381958 14.7006092071533\\
18.9055137634277 14.7006092071533\\
19.0055141448975 14.7006092071533\\
19.1055145263672 14.700608253479\\
19.2055149078369 14.700608253479\\
19.305513381958 14.700608253479\\
19.4055137634277 14.700608253479\\
19.5055141448975 14.700608253479\\
19.6055145263672 14.700608253479\\
19.7055149078369 14.700608253479\\
19.805513381958 14.700608253479\\
19.9055137634277 14.7006092071533\\
20.0055141448975 14.700608253479\\
20.1055145263672 14.700608253479\\
20.2055149078369 14.700608253479\\
20.305513381958 14.700608253479\\
20.4055137634277 14.700608253479\\
20.5055141448975 14.700608253479\\
20.6055145263672 14.700608253479\\
20.7055149078369 14.700608253479\\
20.805513381958 14.700608253479\\
20.9055137634277 14.700608253479\\
21.0055141448975 14.700608253479\\
21.1055145263672 14.700608253479\\
21.2055149078369 14.700608253479\\
21.305513381958 14.700608253479\\
21.4055137634277 14.700608253479\\
21.5055141448975 14.700608253479\\
21.6055145263672 14.700608253479\\
21.7055149078369 14.700608253479\\
21.805513381958 14.700608253479\\
21.9055137634277 14.700608253479\\
22.0055141448975 14.700608253479\\
22.1055145263672 14.700608253479\\
22.2055149078369 14.700608253479\\
22.305513381958 14.700608253479\\
22.4055137634277 14.700608253479\\
22.5055141448975 14.700608253479\\
22.6055145263672 14.700608253479\\
22.7055149078369 14.700608253479\\
22.805513381958 14.700608253479\\
22.9055137634277 14.700608253479\\
23.0055141448975 14.700608253479\\
23.1055145263672 14.700608253479\\
23.2055149078369 14.700608253479\\
23.305513381958 14.700608253479\\
23.4055137634277 14.700608253479\\
23.5055141448975 14.700608253479\\
23.6055145263672 14.700608253479\\
23.7055149078369 14.700608253479\\
23.805513381958 14.700608253479\\
23.9055137634277 14.700608253479\\
24.0055141448975 14.700608253479\\
24.1055145263672 14.700608253479\\
24.2055149078369 14.700608253479\\
24.305513381958 14.700608253479\\
24.4055137634277 14.700608253479\\
24.5055141448975 14.700608253479\\
24.6055145263672 14.700608253479\\
24.7055149078369 14.7006072998047\\
24.789867401123 14.7006072998047\\
24.8573379516602 14.700608253479\\
24.9315567016602 14.700608253479\\
25.0131969451904 14.7006072998047\\
25.1030006408691 14.700608253479\\
25.1443557739258 14.700608253479\\
25.189847946167 14.7006044387817\\
25.2099170684814 14.7006187438965\\
25.2266445159912 14.7006130218506\\
25.245044708252 14.700611114502\\
25.2652854919434 14.7006101608276\\
25.2875499725342 14.7006092071533\\
25.3120403289795 14.7006092071533\\
25.3389797210693 14.7006092071533\\
25.3686122894287 14.7006092071533\\
25.4012107849121 14.7006092071533\\
25.4370670318604 14.7006092071533\\
25.4765090942383 14.7006092071533\\
25.5198955535889 14.7006092071533\\
25.5676212310791 14.7006092071533\\
25.6201171875 14.7006092071533\\
25.6778659820557 14.7006092071533\\
25.7413883209229 14.7006092071533\\
25.8112621307373 14.7006092071533\\
25.8881244659424 14.7006101608276\\
25.9726715087891 14.700608253479\\
26.065673828125 14.7006101608276\\
26.1656742095947 14.7006092071533\\
26.2656745910645 14.7006282806396\\
26.3656749725342 14.7008085250854\\
26.409065246582 14.7020082473755\\
26.4240894317627 14.7075719833374\\
26.4268970489502 14.8330392837524\\
26.4299831390381 14.9396123886108\\
26.4333801269531 14.9484939575195\\
26.4371166229248 14.8965311050415\\
26.4412250518799 14.8519287109375\\
26.4457454681396 14.8220481872559\\
26.4507179260254 14.8179616928101\\
26.4561862945557 14.8049793243408\\
26.4622020721436 14.7912492752075\\
26.4688205718994 14.7775077819824\\
26.4761009216309 14.7641344070435\\
26.4841079711914 14.7443590164185\\
26.4929161071777 14.7199792861938\\
26.5026054382324 14.7085657119751\\
26.5132637023926 14.7012510299683\\
26.5249881744385 14.6981964111328\\
26.5378837585449 14.6994886398315\\
26.5520706176758 14.7004013061523\\
26.5676746368408 14.7006120681763\\
26.5848388671875 14.7006921768188\\
26.6037216186523 14.7020883560181\\
26.624490737915 14.6995515823364\\
26.6473369598389 14.7001657485962\\
26.6724681854248 14.7007608413696\\
26.7001132965088 14.7006225585938\\
26.7305221557617 14.7005672454834\\
26.7594413757324 14.7005615234375\\
26.7881412506104 14.7005786895752\\
26.8180274963379 14.7006015777588\\
26.848762512207 14.7006063461304\\
26.8805618286133 14.7006101608276\\
26.9128971099854 14.7006092071533\\
26.9456844329834 14.7006092071533\\
26.97873878479 14.7006092071533\\
27.0120601654053 14.700608253479\\
27.0457935333252 14.7006092071533\\
27.0821418762207 14.7006101608276\\
27.1221237182617 14.700611114502\\
27.1524639129639 14.700611114502\\
27.1838111877441 14.7006101608276\\
27.2182941436768 14.7006092071533\\
27.2558403015137 14.700608253479\\
27.2955665588379 14.7006072998047\\
27.3392658233643 14.7006072998047\\
27.3873329162598 14.700608253479\\
27.4402103424072 14.700608253479\\
27.4983730316162 14.700608253479\\
27.5623531341553 14.7006101608276\\
27.6327304840088 14.7006092071533\\
27.7101459503174 14.7006092071533\\
27.7953014373779 14.7006092071533\\
27.8889751434326 14.700608253479\\
27.9889755249023 14.7006101608276\\
28.0889739990234 14.7006092071533\\
28.1889743804932 14.7006101608276\\
28.2889747619629 14.7006092071533\\
28.2927665710449 14.7006034851074\\
28.2969398498535 14.7006206512451\\
28.3015289306641 14.7005386352539\\
28.304780960083 14.7006778717041\\
28.307991027832 14.7006387710571\\
28.3115215301514 14.7006120681763\\
28.3154048919678 14.7006101608276\\
28.3196773529053 14.7006120681763\\
28.3243770599365 14.7006120681763\\
28.3295459747314 14.700611114502\\
28.3352317810059 14.700611114502\\
28.3414859771729 14.700611114502\\
28.3483657836914 14.700611114502\\
28.3559341430664 14.700611114502\\
28.3642578125 14.700611114502\\
28.3734169006348 14.7006101608276\\
28.3834896087646 14.7006101608276\\
28.3945693969727 14.7006101608276\\
28.4067573547363 14.7006101608276\\
28.4201641082764 14.7006101608276\\
28.4349136352539 14.7006101608276\\
28.451135635376 14.7006101608276\\
28.4689807891846 14.7006101608276\\
28.4886093139648 14.7006101608276\\
28.5102024078369 14.7006101608276\\
28.5339546203613 14.7006101608276\\
28.5600814819336 14.7006101608276\\
28.5888214111328 14.7006101608276\\
28.620433807373 14.7006101608276\\
28.6552085876465 14.7006101608276\\
28.6934604644775 14.7006101608276\\
28.735538482666 14.7006101608276\\
28.7818241119385 14.7006101608276\\
28.8327369689941 14.7006101608276\\
28.8887424468994 14.7006101608276\\
28.9503479003906 14.7006101608276\\
29.0181159973145 14.7006101608276\\
29.092658996582 14.7006101608276\\
29.1746559143066 14.7006101608276\\
29.2648525238037 14.700608253479\\
29.3640689849854 14.7006101608276\\
29.4640693664551 14.7006092071533\\
29.4847068786621 14.7009754180908\\
29.4888343811035 14.7010774612427\\
29.4933738708496 14.700966835022\\
29.4983692169189 14.7009925842285\\
29.5038623809814 14.7010583877563\\
29.5099067687988 14.7011394500732\\
29.5165538787842 14.7012329101563\\
29.5230560302734 14.7013444900513\\
29.5302066802979 14.701452255249\\
29.5380744934082 14.7015533447266\\
29.5467281341553 14.7033586502075\\
29.556245803833 14.7535715103149\\
29.5667171478271 14.8463010787964\\
29.5782356262207 14.8755788803101\\
29.5909042358398 14.8282041549683\\
29.6048412322998 14.7773714065552\\
29.6201705932617 14.7907190322876\\
29.637035369873 14.7759046554565\\
29.6555843353271 14.7576093673706\\
29.675989151001 14.7312831878662\\
29.6984329223633 14.704517364502\\
29.7114696502686 14.7016630172729\\
29.7258071899414 14.7007369995117\\
29.7415809631348 14.7005338668823\\
29.7589302062988 14.7033634185791\\
29.7780151367188 14.7350234985352\\
29.7990093231201 14.7702569961548\\
29.8221015930176 14.7679309844971\\
29.8475036621094 14.7670278549194\\
29.8754444122314 14.7600269317627\\
29.9061813354492 14.7619876861572\\
29.9399909973145 14.7530488967896\\
29.9771823883057 14.747127532959\\
30.0180931091309 14.7481756210327\\
30.0630931854248 14.7414245605469\\
30.1125946044922 14.7313919067383\\
30.1670475006104 14.7335319519043\\
30.2269439697266 14.7248487472534\\
30.2706527709961 14.7174825668335\\
30.318733215332 14.7115631103516\\
30.3593273162842 14.7052364349365\\
30.3947944641113 14.7023334503174\\
30.4330863952637 14.701434135437\\
30.4652576446533 14.701192855835\\
30.5006484985352 14.7011852264404\\
30.5330505371094 14.7012434005737\\
30.5677318572998 14.7013483047485\\
30.6004276275635 14.7015352249146\\
30.6349658966064 14.7018280029297\\
30.6682662963867 14.7022867202759\\
30.7048969268799 14.7029676437378\\
30.7451915740967 14.7041110992432\\
30.789514541626 14.7060413360596\\
30.8382682800293 14.7093057632446\\
30.8918991088867 14.7145195007324\\
30.9508934020996 14.7218151092529\\
31.0157871246338 14.7280025482178\\
31.0871696472168 14.7341451644897\\
31.1656913757324 14.7303953170776\\
31.2520637512207 14.7317905426025\\
31.347074508667 14.7330884933472\\
31.4470748901367 14.7325534820557\\
31.5470752716064 14.7328481674194\\
31.6470737457275 14.7333068847656\\
31.7470741271973 14.7339391708374\\
31.847074508667 14.7343769073486\\
31.9470748901367 14.7349405288696\\
32.0470733642578 14.735424041748\\
32.1470756530762 14.7359590530396\\
32.2470741271973 14.7364673614502\\
32.3470764160156 14.7369928359985\\
32.4470748901367 14.7375106811523\\
32.5470733642578 14.7380361557007\\
32.6470756530762 14.7385568618774\\
32.7470741271973 14.7390813827515\\
32.8470764160156 14.7396030426025\\
32.9470748901367 14.7401266098022\\
33.0470733642578 14.7406482696533\\
33.1470756530762 14.7411708831787\\
33.2470741271973 14.7416934967041\\
33.3470764160156 14.7422161102295\\
33.4470748901367 14.7427387237549\\
33.5470733642578 14.7432613372803\\
33.6470756530762 14.743860244751\\
33.7470741271973 14.7446699142456\\
33.8470764160156 14.745246887207\\
33.9470748901367 14.7458877563477\\
34.0470733642578 14.746506690979\\
34.1470756530762 14.7471303939819\\
34.2470741271973 14.7477502822876\\
34.3470764160156 14.7483701705933\\
34.4470748901367 14.7489881515503\\
34.5470733642578 14.7496042251587\\
34.6470756530762 14.7502202987671\\
34.7470741271973 14.7508344650269\\
34.8470764160156 14.7514476776123\\
34.9470748901367 14.7520599365234\\
35.0470733642578 14.7526702880859\\
35.1470756530762 14.7532787322998\\
35.2470741271973 14.7536611557007\\
35.3470764160156 14.7541217803955\\
35.4470748901367 14.7546472549438\\
35.5470733642578 14.755145072937\\
35.6470756530762 14.7556447982788\\
35.7470741271973 14.7561349868774\\
35.8470764160156 14.7566175460815\\
35.9470748901367 14.7570905685425\\
36.0470733642578 14.7575531005859\\
36.1470756530762 14.758002281189\\
36.2470741271973 14.7584371566772\\
36.3470764160156 14.7588577270508\\
36.4470748901367 14.759259223938\\
36.5470733642578 14.7633399963379\\
36.6470756530762 14.7625246047974\\
36.7470741271973 14.7624053955078\\
36.8470764160156 14.7631921768188\\
36.9470748901367 14.7638397216797\\
37.0470733642578 14.7650547027588\\
37.1470756530762 14.7725477218628\\
37.2293395996094 14.776219367981\\
37.2951393127441 14.7702379226685\\
37.3675231933594 14.7684335708618\\
37.4471435546875 14.7688436508179\\
37.534725189209 14.7693328857422\\
37.6310653686523 14.7850761413574\\
37.6772041320801 14.7398271560669\\
37.7279586791992 14.7679586410522\\
37.7453880310059 14.7801971435547\\
37.7602653503418 14.7744770050049\\
37.7766304016113 14.7722024917603\\
37.7946319580078 14.7722778320313\\
37.8144378662109 14.7723894119263\\
37.8362197875977 14.7725305557251\\
37.860179901123 14.7727079391479\\
37.8865356445313 14.7729215621948\\
37.9155311584473 14.77317237854\\
37.9474220275879 14.7734632492065\\
37.9825019836426 14.773796081543\\
38.0210914611816 14.7741727828979\\
38.0635414123535 14.7745943069458\\
38.1102333068848 14.7750644683838\\
38.1615982055664 14.7755842208862\\
38.2180976867676 14.776159286499\\
38.2802429199219 14.7767925262451\\
38.3486099243164 14.7774887084961\\
38.4238090515137 14.7782554626465\\
38.5065269470215 14.7875881195068\\
38.5975189208984 14.7719497680664\\
38.6975212097168 14.7811489105225\\
38.7975196838379 14.7851705551147\\
38.8975219726563 14.7882423400879\\
38.9475212097168 14.8382053375244\\
38.9856338500977 14.8013620376587\\
39.0002326965332 14.8161401748657\\
39.0162887573242 14.833155632019\\
39.0339508056641 14.7793445587158\\
39.0533828735352 14.7840824127197\\
39.0747528076172 14.7882699966431\\
39.0982627868652 14.7975759506226\\
39.1241226196289 14.7917604446411\\
39.1525688171387 14.7914037704468\\
39.1838569641113 14.7857885360718\\
39.2182769775391 14.7782182693481\\
39.2382202148438 14.7788038253784\\
39.2601547241211 14.7807712554932\\
39.2842864990234 14.7908363342285\\
39.3108291625977 14.7989864349365\\
39.3400230407715 14.8065252304077\\
39.3721389770508 14.8123645782471\\
39.4047470092773 14.8208017349243\\
39.4365348815918 14.8217630386353\\
39.4690742492676 14.8147678375244\\
39.5013847351074 14.8090543746948\\
39.5341873168945 14.8072032928467\\
39.5668563842773 14.8059225082397\\
39.5998191833496 14.8046703338623\\
39.632698059082 14.8036670684814\\
39.6657829284668 14.8032712936401\\
39.6988639831543 14.8031902313232\\
39.7321624755859 14.8025960922241\\
39.7687873840332 14.8018817901611\\
39.8090744018555 14.8013277053833\\
39.8533935546875 14.8008670806885\\
39.8968162536621 14.8013782501221\\
39.939453125 14.8036880493164\\
39.9863548278809 14.8012666702271\\
40.0379447937012 14.7983522415161\\
40.0946960449219 14.7982845306396\\
40.1571197509766 14.7986211776733\\
40.2257843017578 14.7988386154175\\
40.3013191223145 14.7993240356445\\
40.384407043457 14.7999534606934\\
40.4758033752441 14.802698135376\\
40.5758018493652 14.7995738983154\\
40.6758041381836 14.804783821106\\
40.7758026123047 14.8014316558838\\
40.875804901123 14.8038167953491\\
40.8792419433594 14.8960933685303\\
40.8830223083496 14.7611074447632\\
40.8871841430664 14.7825517654419\\
40.8903007507324 14.8220062255859\\
40.8937301635742 14.8183765411377\\
40.8975028991699 14.8064222335815\\
40.9016532897949 14.8060741424561\\
40.9062156677246 14.8062314987183\\
40.9112396240234 14.8062400817871\\
40.9167633056641 14.8062362670898\\
40.9228363037109 14.8062391281128\\
40.9295196533203 14.8062477111816\\
40.9368743896484 14.8062629699707\\
40.9449615478516 14.8062858581543\\
40.953857421875 14.8063173294067\\
40.9636421203613 14.8063592910767\\
40.9744033813477 14.8064136505127\\
40.9862442016602 14.8064823150635\\
40.999267578125 14.8065662384033\\
41.0135955810547 14.8066663742065\\
41.029354095459 14.8067865371704\\
41.046688079834 14.8069276809692\\
41.0657577514648 14.8070917129517\\
41.0867347717285 14.807279586792\\
41.109806060791 14.8074941635132\\
41.1351852416992 14.8077373504639\\
41.1631050109863 14.8080101013184\\
41.1938133239746 14.8083152770996\\
41.227596282959 14.8086538314819\\
41.2647552490234 14.8090305328369\\
41.3056297302246 14.8094463348389\\
41.3505935668945 14.8099050521851\\
41.4000511169434 14.8104104995728\\
41.4544563293457 14.8109664916992\\
41.5143013000488 14.8115787506104\\
41.5801315307617 14.8122520446777\\
41.6525421142578 14.8129930496216\\
41.7321968078613 14.8138084411621\\
41.8198165893555 14.8164882659912\\
41.9161987304688 14.8137912750244\\
42.0161972045898 14.8168897628784\\
42.1161994934082 14.8248729705811\\
42.1361999511719 14.8291673660278\\
42.1442070007324 14.852593421936\\
42.1520309448242 14.8552503585815\\
42.1606330871582 14.8526744842529\\
42.1689529418945 14.8536882400513\\
42.1781005859375 14.8561382293701\\
42.1881637573242 14.8804187774658\\
42.1992378234863 14.9297533035278\\
42.2114143371582 14.92360496521\\
42.2248077392578 14.9104614257813\\
42.2395439147949 14.8958578109741\\
42.2557525634766 14.8787059783936\\
42.2735824584961 14.8586225509644\\
42.2931938171387 14.8386106491089\\
42.3147659301758 14.8194217681885\\
42.3384971618652 14.790548324585\\
42.3646011352539 14.7692842483521\\
42.3847808837891 14.7683639526367\\
42.4069786071777 14.7822370529175\\
42.431396484375 14.7954044342041\\
42.4582557678223 14.8076066970825\\
42.4878005981445 14.8400793075562\\
42.5203018188477 14.848087310791\\
42.5560493469238 14.8473196029663\\
42.5953750610352 14.8462696075439\\
42.6386299133301 14.8382730484009\\
42.6862144470215 14.8601264953613\\
42.7385559082031 14.8342657089233\\
42.7959022521973 14.8444900512695\\
42.8589859008789 14.8366861343384\\
42.9028167724609 14.8282566070557\\
42.9510307312012 14.831036567688\\
42.9925117492676 14.8273391723633\\
43.0297622680664 14.8286476135254\\
43.0707359313965 14.830132484436\\
43.1158065795898 14.8315591812134\\
43.1579742431641 14.8337850570679\\
43.203685760498 14.8362989425659\\
43.2497863769531 14.8383131027222\\
43.3005027770996 14.8371000289917\\
43.3562850952148 14.8398532867432\\
43.4176483154297 14.8379344940186\\
43.4851493835449 14.840220451355\\
43.559398651123 14.837818145752\\
43.6410713195801 14.8396701812744\\
43.7309150695801 14.8375434875488\\
43.8297386169434 14.8391427993774\\
43.9297409057617 14.8378210067749\\
44.0297393798828 14.8408517837524\\
44.1297416687012 14.8401279449463\\
44.2297401428223 14.841893196106\\
44.2499618530273 14.8444700241089\\
44.2722015380859 14.8431005477905\\
44.2966690063477 14.8433475494385\\
44.3235816955566 14.8436212539673\\
44.3531875610352 14.843921661377\\
44.3857498168945 14.8442525863647\\
44.421573638916 14.8446178436279\\
44.4609756469727 14.8450183868408\\
44.5043182373047 14.8454599380493\\
44.5519981384277 14.8459444046021\\
44.6044425964355 14.8464784622192\\
44.6621360778809 14.8470659255981\\
44.7255935668945 14.8477125167847\\
44.7953987121582 14.8484230041504\\
44.872184753418 14.8492050170898\\
44.9566497802734 14.8500652313232\\
45.0495643615723 14.8510122299194\\
45.1495628356934 14.8526449203491\\
45.2495613098145 14.8525362014771\\
45.3495635986328 14.8542003631592\\
45.4495620727539 14.8559513092041\\
45.5495643615723 14.8556089401245\\
45.6495628356934 14.8573741912842\\
45.7495613098145 14.8663511276245\\
45.7695655822754 14.8766441345215\\
45.7816352844238 14.8774070739746\\
45.7949142456055 14.8760204315186\\
45.8095207214355 14.8745975494385\\
45.8255882263184 14.8738279342651\\
45.8432655334473 14.872784614563\\
45.8627052307129 14.8713493347168\\
45.884090423584 14.8693227767944\\
45.9076156616211 14.868540763855\\
45.9334945678711 14.8674955368042\\
45.9619598388672 14.8662166595459\\
45.9932708740234 14.8651332855225\\
46.0277099609375 14.8645372390747\\
46.0655975341797 14.8638582229614\\
46.1072731018066 14.8632478713989\\
46.1531181335449 14.8628215789795\\
46.2035446166992 14.8629741668701\\
46.2590141296387 14.8637828826904\\
46.3200302124023 14.8686790466309\\
46.387149810791 14.8708410263062\\
46.4504814147949 14.8713541030884\\
46.5130958557129 14.8714208602905\\
46.581974029541 14.8719568252563\\
46.6577415466309 14.8716735839844\\
46.7410850524902 14.8721294403076\\
46.8327598571777 14.8727016448975\\
46.9327621459961 14.8736228942871\\
47.0327606201172 14.8745012283325\\
47.1327629089355 14.8754215240479\\
47.2327613830566 14.8764047622681\\
47.3327598571777 14.8774242401123\\
47.4327621459961 14.8784761428833\\
47.5327606201172 14.8795461654663\\
47.6327629089355 14.8804426193237\\
47.7327613830566 14.8814821243286\\
47.8327598571777 14.8825330734253\\
47.9327621459961 14.8836221694946\\
48.0327606201172 14.8847169876099\\
48.1327629089355 14.8858203887939\\
48.2327613830566 14.8869256973267\\
48.3327598571777 14.8880338668823\\
48.4327621459961 14.8891429901123\\
48.5327606201172 14.8902540206909\\
48.6327629089355 14.8913679122925\\
48.7327613830566 14.8924827575684\\
48.8327598571777 14.8935985565186\\
48.9327621459961 14.8946990966797\\
49.0327606201172 14.8958292007446\\
49.1327629089355 14.8969688415527\\
49.2327613830566 14.898120880127\\
49.3327598571777 14.8992958068848\\
49.4327621459961 14.9005136489868\\
49.5327606201172 14.9074468612671\\
49.6257057189941 14.9102649688721\\
49.6982688903809 14.9085607528687\\
49.7780876159668 14.9044771194458\\
49.8658905029297 14.9051399230957\\
49.962474822998 14.9058303833008\\
50.0624732971191 14.9067764282227\\
50.1624755859375 14.9078502655029\\
50.2624740600586 14.9089584350586\\
50.3624725341797 14.9100570678711\\
50.462474822998 14.9111642837524\\
50.5624732971191 14.9122734069824\\
50.6624755859375 14.9133863449097\\
50.7624740600586 14.9145030975342\\
50.8624725341797 14.9156150817871\\
50.962474822998 14.9167327880859\\
51.0624732971191 14.9178514480591\\
51.1624755859375 14.9189720153809\\
51.2624740600586 14.9200925827026\\
51.3624725341797 14.9212131500244\\
51.462474822998 14.9223365783691\\
51.5624732971191 14.9234600067139\\
51.6624755859375 14.9245834350586\\
51.7624740600586 14.9257068634033\\
51.8624725341797 14.9268293380737\\
51.962474822998 14.9279527664185\\
52.0624732971191 14.9290761947632\\
52.1624755859375 14.9301996231079\\
52.2624740600586 14.9313230514526\\
52.3624725341797 14.9324464797974\\
52.462474822998 14.9335699081421\\
52.5624732971191 14.9346933364868\\
52.6624755859375 14.9358167648315\\
52.7624740600586 14.9369401931763\\
52.8624725341797 14.938063621521\\
52.962474822998 14.9391870498657\\
53.0624732971191 14.9403104782104\\
53.1624755859375 14.9414339065552\\
53.2624740600586 14.9425573348999\\
53.3624725341797 14.9436817169189\\
53.462474822998 14.9448051452637\\
53.5624732971191 14.9459285736084\\
53.6624755859375 14.9470520019531\\
53.7624740600586 14.9481763839722\\
53.8624725341797 14.9492998123169\\
53.962474822998 14.9504241943359\\
54.0624732971191 14.9515476226807\\
54.1624755859375 14.9526720046997\\
54.2624740600586 14.9537954330444\\
54.3624725341797 14.9549198150635\\
54.462474822998 14.9560432434082\\
54.5624732971191 14.9571676254272\\
54.6624755859375 14.958291053772\\
54.7624740600586 14.959415435791\\
54.8624725341797 14.9605398178101\\
54.962474822998 14.9616632461548\\
55.0624732971191 14.9627885818481\\
55.1624755859375 14.9639139175415\\
55.2624740600586 14.9650449752808\\
55.3624725341797 14.9662036895752\\
55.462474822998 14.9675092697144\\
55.5624732971191 14.9695634841919\\
55.6624755859375 14.9741716384888\\
55.7624740600586 14.9816217422485\\
55.8624725341797 14.9826431274414\\
55.962474822998 14.97815990448\\
56.0624732971191 14.9752674102783\\
56.1624755859375 14.9773387908936\\
56.2624740600586 14.9761028289795\\
56.3624725341797 14.9785804748535\\
56.462474822998 14.9787998199463\\
56.5624732971191 14.9805183410645\\
56.6624755859375 14.9812507629395\\
56.7624740600586 14.9826326370239\\
56.8624725341797 14.983588218689\\
56.962474822998 14.984824180603\\
57.0624732971191 14.985876083374\\
57.1624755859375 14.9870500564575\\
57.2624740600586 14.9881429672241\\
57.3624725341797 14.9892902374268\\
57.462474822998 14.9904155731201\\
57.5624732971191 14.9915370941162\\
57.6624755859375 14.992654800415\\
57.7624740600586 14.9937858581543\\
57.8624725341797 14.9949083328247\\
57.962474822998 14.9960374832153\\
58.0624732971191 14.9971618652344\\
58.1624755859375 14.9982900619507\\
58.2624740600586 14.999415397644\\
58.3624725341797 15.0006074905396\\
58.462474822998 15.0020008087158\\
58.5624732971191 15.0030641555786\\
58.6624755859375 15.0042123794556\\
58.7624740600586 15.0053339004517\\
58.8624725341797 15.0064649581909\\
58.962474822998 15.0075931549072\\
59.0624732971191 15.0087223052979\\
59.1624755859375 15.0098514556885\\
59.2624740600586 15.0109806060791\\
59.3624725341797 15.012110710144\\
59.462474822998 15.013240814209\\
59.5624732971191 15.0143718719482\\
59.6624755859375 15.0155038833618\\
59.7624740600586 15.0166358947754\\
59.8624725341797 15.0177659988403\\
59.9312362670898 15.0186204910278\\
60 15.0194005966187\\
60.0756378173828 15.0201663970947\\
60.1588439941406 15.021014213562\\
60.2503662109375 15.0219469070435\\
60.3503646850586 15.0229721069336\\
60.450366973877 15.0240821838379\\
60.550365447998 15.0252084732056\\
60.6503677368164 15.0263271331787\\
60.7503662109375 15.0274486541748\\
60.8503646850586 15.0285692214966\\
60.950366973877 15.0296936035156\\
61.050365447998 15.0308532714844\\
61.1503677368164 15.0320243835449\\
61.2503662109375 15.0331707000732\\
61.3503646850586 15.0343217849731\\
61.450366973877 15.0354862213135\\
61.550365447998 15.0366716384888\\
61.6503677368164 15.0378913879395\\
61.7503662109375 15.0395431518555\\
61.8503646850586 15.0475234985352\\
61.9325637817383 15.0503091812134\\
61.998291015625 15.0441284179688\\
62.0705947875977 15.0427303314209\\
62.1501274108887 15.0433397293091\\
62.2376098632813 15.0439825057983\\
62.3338470458984 15.0448293685913\\
62.4338455200195 15.0458583831787\\
62.5338439941406 15.0469655990601\\
62.633846282959 15.0480690002441\\
62.7338447570801 15.0491781234741\\
62.8338470458984 15.0502901077271\\
62.9338455200195 15.0514078140259\\
63.0338439941406 15.0525236129761\\
63.133846282959 15.0536413192749\\
63.2338447570801 15.0547618865967\\
63.3338470458984 15.0558843612671\\
63.4338455200195 15.0570077896118\\
63.5338439941406 15.0581340789795\\
63.633846282959 15.0592603683472\\
63.7338447570801 15.0603866577148\\
63.8338470458984 15.0615129470825\\
63.9338455200195 15.0626392364502\\
64.0338439941406 15.0637655258179\\
64.1338424682617 15.0648918151855\\
64.2338485717773 15.0660181045532\\
64.3338470458984 15.0671453475952\\
64.4338455200195 15.0682716369629\\
64.5338439941406 15.0693979263306\\
64.6338424682617 15.0705242156982\\
64.7338485717773 15.0716505050659\\
64.8338470458984 15.0727767944336\\
64.9338455200195 15.0739030838013\\
65.0338439941406 15.0750293731689\\
65.1338424682617 15.0761556625366\\
65.2338485717773 15.0772819519043\\
65.3338470458984 15.078408241272\\
65.4338455200195 15.079535484314\\
65.5338439941406 15.0806617736816\\
65.6338424682617 15.0817880630493\\
65.7338485717773 15.082914352417\\
65.8338470458984 15.0840406417847\\
65.9338455200195 15.0851669311523\\
66.0338439941406 15.0862941741943\\
66.1338424682617 15.087420463562\\
66.2338485717773 15.0885467529297\\
66.3338470458984 15.0896730422974\\
66.4338455200195 15.090799331665\\
66.5338439941406 15.091926574707\\
66.6338424682617 15.0930528640747\\
66.7338485717773 15.0941791534424\\
66.8338470458984 15.0953054428101\\
66.9338455200195 15.0964326858521\\
67.0338439941406 15.0975589752197\\
67.1338424682617 15.0986852645874\\
67.2338485717773 15.0998125076294\\
67.3338470458984 15.1009397506714\\
67.4338455200195 15.1020669937134\\
67.5338439941406 15.1031999588013\\
67.6338424682617 15.1043548583984\\
67.7338485717773 15.1056308746338\\
67.8338470458984 15.1075391769409\\
67.9338455200195 15.1117744445801\\
68.0338439941406 15.1191968917847\\
68.1338424682617 15.1217460632324\\
68.2338485717773 15.1164817810059\\
68.3338470458984 15.1134653091431\\
68.4338455200195 15.1157464981079\\
68.5338439941406 15.1142244338989\\
68.6338424682617 15.1167974472046\\
68.7338485717773 15.1169462203979\\
68.8338470458984 15.1187152862549\\
68.9338455200195 15.1194171905518\\
69.0338439941406 15.1208238601685\\
69.1338424682617 15.1217660903931\\
69.2338485717773 15.1230154037476\\
69.3338470458984 15.1240606307983\\
69.4338455200195 15.1252422332764\\
69.5338439941406 15.1263332366943\\
69.6338424682617 15.1274843215942\\
69.7338485717773 15.1285972595215\\
69.8338470458984 15.129734992981\\
69.9338455200195 15.1308565139771\\
70.0338439941406 15.1319894790649\\
70.1338424682617 15.1331148147583\\
70.2338485717773 15.1342449188232\\
70.3338470458984 15.1353721618652\\
70.4338455200195 15.1365013122559\\
70.5338439941406 15.1376295089722\\
70.6338424682617 15.1388349533081\\
70.7338485717773 15.1402158737183\\
70.8338470458984 15.1412878036499\\
70.9338455200195 15.1424369812012\\
71.0338439941406 15.1435604095459\\
71.1338424682617 15.1446924209595\\
71.2338485717773 15.1458225250244\\
71.3338470458984 15.1469535827637\\
71.4338455200195 15.1480836868286\\
71.5338439941406 15.1492156982422\\
71.6338424682617 15.1503467559814\\
71.7338485717773 15.151478767395\\
71.8338470458984 15.1526117324829\\
71.9338455200195 15.1537446975708\\
72.0338439941406 15.154878616333\\
72.1338424682617 15.1560077667236\\
72.2338485717773 15.1568746566772\\
72.3338470458984 15.157977104187\\
72.4338455200195 15.1591014862061\\
72.5338439941406 15.1602230072021\\
72.6338424682617 15.1613445281982\\
72.7338485717773 15.1624660491943\\
72.8338470458984 15.1635866165161\\
72.9338455200195 15.1647071838379\\
73.0338439941406 15.1658277511597\\
73.1338424682617 15.1669473648071\\
73.2338485717773 15.1680812835693\\
73.3338470458984 15.1692638397217\\
73.4338455200195 15.170446395874\\
73.5338439941406 15.1715974807739\\
73.6338424682617 15.1727523803711\\
73.7338485717773 15.173924446106\\
73.8338470458984 15.1751136779785\\
73.9338455200195 15.1763477325439\\
74.0338439941406 15.1792068481445\\
74.1323699951172 15.18603515625\\
74.2112121582031 15.1891946792603\\
74.2755126953125 15.1807537078857\\
74.3462524414063 15.1810503005981\\
74.4240570068359 15.1816368103027\\
74.509651184082 15.182276725769\\
74.6037979125977 15.1831111907959\\
74.7037963867188 15.1841239929199\\
74.8037948608398 15.1852331161499\\
74.9019012451172 15.1863384246826\\
75 15.1874284744263\\
};
\addlegendentry{10 "};

\addplot [
color=black,
dotted
]
table[row sep=crcr]{
0 14.7005214691162\\
9.99999971718069e-10 14.7005214691162\\
2.49999998480632e-09 14.7005214691162\\
4.74999994892755e-09 14.7005214691162\\
8.1250002281763e-09 14.7005214691162\\
1.31874999809156e-08 14.7005214691162\\
2.0781250498203e-08 14.7005214691162\\
3.21718758300449e-08 14.7005214691162\\
4.92578138278077e-08 14.7005214691162\\
7.48867208244519e-08 14.7005214691162\\
1.13330081319418e-07 14.7005214691162\\
1.70995122061868e-07 14.7005214691162\\
2.57492672517401e-07 14.7005214691162\\
3.8723902662241e-07 14.7005214691162\\
5.81858500936505e-07 14.7005214691162\\
8.73787769251066e-07 14.7005214691162\\
1.31168167172291e-06 14.7005214691162\\
1.9685226106958e-06 14.7005214691162\\
2.95378367809462e-06 14.7005214691162\\
4.43167573394021e-06 14.7005214691162\\
6.64851359033491e-06 14.7005214691162\\
9.97376992017962e-06 14.7005214691162\\
1.49616553244414e-05 14.7005224227905\\
2.24434825213393e-05 14.7005224227905\\
3.36662233166862e-05 14.7005233764648\\
5.05003372381907e-05 14.7005233764648\\
7.57515081204474e-05 14.7005243301392\\
0.000113628258986864 14.7005252838135\\
0.000170443381648511 14.7005262374878\\
0.000255666091106832 14.7005271911621\\
0.000383500126190484 14.7005281448364\\
0.000575251178815961 14.7005290985107\\
0.000862877757754177 14.7005290985107\\
0.00129431765526533 14.7005281448364\\
0.00194147753063589 14.7005243301392\\
0.00291221728548408 14.7005195617676\\
0.00436832662671804 14.7005176544189\\
0.00597004732117057 14.7005195617676\\
0.00773194013163447 14.7005214691162\\
0.009670021943748 14.7005224227905\\
0.0118019115179777 14.7005224227905\\
0.0141469910740852 14.7005214691162\\
0.0167265776544809 14.7005214691162\\
0.0195641238242388 14.7005214691162\\
0.0226854234933853 14.7005214691162\\
0.0261188540607691 14.7005214691162\\
0.0298956278711557 14.7005214691162\\
0.0340500771999359 14.7005214691162\\
0.0386199727654457 14.7005214691162\\
0.0436468608677387 14.7005214691162\\
0.0491764321923256 14.7005214691162\\
0.0552589632570744 14.7005214691162\\
0.0619497485458851 14.7005214691162\\
0.0693096145987511 14.7005214691162\\
0.0774054601788521 14.7005214691162\\
0.0863108932971954 14.7005205154419\\
0.0961068719625473 14.7005224227905\\
0.106882445514202 14.7005224227905\\
0.118735581636429 14.7005214691162\\
0.13177402317524 14.7005214691162\\
0.146116316318512 14.7005214691162\\
0.161892831325531 14.7005214691162\\
0.179247006773949 14.7005214691162\\
0.198336601257324 14.7005214691162\\
0.21933513879776 14.7005214691162\\
0.242433547973633 14.7005214691162\\
0.267841786146164 14.7005214691162\\
0.29579085111618 14.7005214691162\\
0.32653483748436 14.7005214691162\\
0.360353201627731 14.7005214691162\\
0.397553414106369 14.7005214691162\\
0.438473641872406 14.7005214691162\\
0.483485877513886 14.7005214691162\\
0.532999336719513 14.7005214691162\\
0.587464153766632 14.7005214691162\\
0.647375464439392 14.7005214691162\\
0.71327793598175 14.7005214691162\\
0.7857705950737 14.7005214691162\\
0.865512549877167 14.7005214691162\\
0.953228712081909 14.7005214691162\\
1.04971647262573 14.7005214691162\\
1.14971649646759 14.7005214691162\\
1.24971640110016 14.7005214691162\\
1.34971642494202 14.7005214691162\\
1.44971644878387 14.7005214691162\\
1.54971647262573 14.7005214691162\\
1.64971649646759 14.7005214691162\\
1.74971640110016 14.7005214691162\\
1.84971642494202 14.7005214691162\\
1.94971644878387 14.7005214691162\\
2.04971647262573 14.7005214691162\\
2.1497163772583 14.7005214691162\\
2.24971652030945 14.7005214691162\\
2.34971642494202 14.7005214691162\\
2.44971656799316 14.7005214691162\\
2.54971647262573 14.7005214691162\\
2.6497163772583 14.7005214691162\\
2.74971652030945 14.7005214691162\\
2.84971642494202 14.7005214691162\\
2.94971656799316 14.7005214691162\\
3.04971647262573 14.7005214691162\\
3.1497163772583 14.7005214691162\\
3.24971652030945 14.7005214691162\\
3.34971642494202 14.7005214691162\\
3.44971656799316 14.7005214691162\\
3.54971647262573 14.7005214691162\\
3.6497163772583 14.7005214691162\\
3.74971652030945 14.7005214691162\\
3.84971642494202 14.7005214691162\\
3.94971656799316 14.7005214691162\\
4.04971647262573 14.7005214691162\\
4.1497163772583 14.7005214691162\\
4.24971628189087 14.7005214691162\\
4.3497166633606 14.7005214691162\\
4.44971656799316 14.7005214691162\\
4.54971647262573 14.7005214691162\\
4.6497163772583 14.7005214691162\\
4.74971628189087 14.7005214691162\\
4.8497166633606 14.7005214691162\\
4.94971656799316 14.7005214691162\\
5.04971647262573 14.7005214691162\\
5.1497163772583 14.7005214691162\\
5.24971628189087 14.7005214691162\\
5.3497166633606 14.7005214691162\\
5.44971656799316 14.7005214691162\\
5.54971647262573 14.7005214691162\\
5.6497163772583 14.7005214691162\\
5.74971628189087 14.7005214691162\\
5.8497166633606 14.7005214691162\\
5.94971656799316 14.7005214691162\\
6.04971647262573 14.7005214691162\\
6.1497163772583 14.7005205154419\\
6.24971628189087 14.7005195617676\\
6.3497166633606 14.7005186080933\\
6.44971656799316 14.7005205154419\\
6.54971647262573 14.7005224227905\\
6.6497163772583 14.7005233764648\\
6.74971628189087 14.7005224227905\\
6.8497166633606 14.7005214691162\\
6.94971656799316 14.7005214691162\\
7.04971647262573 14.7005214691162\\
7.1497163772583 14.7005214691162\\
7.24971628189087 14.7005214691162\\
7.3497166633606 14.7005214691162\\
7.44971656799316 14.7005214691162\\
7.54971647262573 14.7005214691162\\
7.6497163772583 14.7005214691162\\
7.74971628189087 14.7005214691162\\
7.8497166633606 14.7005214691162\\
7.94971656799316 14.7005214691162\\
8.04971599578857 14.7005214691162\\
8.1497163772583 14.7005214691162\\
8.24971675872803 14.7005214691162\\
8.34971618652344 14.7005214691162\\
8.44971656799316 14.7005214691162\\
8.54971599578857 14.7005214691162\\
8.6497163772583 14.7005214691162\\
8.74971675872803 14.7005214691162\\
8.84971618652344 14.7005214691162\\
8.94971656799316 14.7005214691162\\
9.04971599578857 14.7005214691162\\
9.1497163772583 14.7005214691162\\
9.24971675872803 14.7005214691162\\
9.34971618652344 14.7005214691162\\
9.44971656799316 14.7005214691162\\
9.54971599578857 14.7005214691162\\
9.6497163772583 14.7005214691162\\
9.74971675872803 14.7005214691162\\
9.84971618652344 14.7005214691162\\
9.94971656799316 14.7005214691162\\
10.0497159957886 14.7005214691162\\
10.1497163772583 14.7005214691162\\
10.249716758728 14.7005214691162\\
10.3497161865234 14.7005214691162\\
10.4497165679932 14.7005214691162\\
10.5497159957886 14.7005214691162\\
10.6497163772583 14.7005214691162\\
10.749716758728 14.7005214691162\\
10.8497161865234 14.7005214691162\\
10.9497165679932 14.7005214691162\\
11.0497159957886 14.7005214691162\\
11.1497163772583 14.7005214691162\\
11.249716758728 14.7005214691162\\
11.3497161865234 14.7005214691162\\
11.4497165679932 14.7005214691162\\
11.5497159957886 14.7005214691162\\
11.6497163772583 14.7005214691162\\
11.749716758728 14.7005214691162\\
11.8497161865234 14.7005214691162\\
11.9497165679932 14.7005214691162\\
12.0497159957886 14.7005214691162\\
12.1497163772583 14.7005214691162\\
12.249716758728 14.7005214691162\\
12.347146987915 14.7005214691162\\
12.4250946044922 14.7005214691162\\
12.4874591827393 14.7005224227905\\
12.5560598373413 14.7005214691162\\
12.6315202713013 14.7005205154419\\
12.714527130127 14.7005214691162\\
12.8058338165283 14.7005214691162\\
12.8497476577759 14.7005224227905\\
12.8980522155762 14.7005205154419\\
12.9143037796021 14.7005214691162\\
12.9277105331421 14.7005224227905\\
12.9424571990967 14.7005224227905\\
12.9586782455444 14.7005214691162\\
12.9765224456787 14.7005214691162\\
12.9961500167847 14.7005214691162\\
13.0177412033081 14.7005214691162\\
13.0414915084839 14.7005214691162\\
13.0676164627075 14.7005214691162\\
13.0963535308838 14.7005214691162\\
13.1279649734497 14.7005214691162\\
13.1627368927002 14.7005214691162\\
13.2009868621826 14.7005214691162\\
13.2430610656738 14.7005214691162\\
13.289342880249 14.7005214691162\\
13.3402528762817 14.7005214691162\\
13.3962545394897 14.7005214691162\\
13.4578561782837 14.7005214691162\\
13.5256175994873 14.7005214691162\\
13.600154876709 14.7005214691162\\
13.6821460723877 14.7005205154419\\
13.7723369598389 14.7005214691162\\
13.871545791626 14.7005214691162\\
13.9715461730957 14.7005462646484\\
14.0715456008911 14.7006511688232\\
14.1215467453003 14.7011642456055\\
14.147439956665 14.7014102935791\\
14.1759233474731 14.6993856430054\\
14.1775197982788 14.6922054290771\\
14.1792755126953 14.6938762664795\\
14.1812076568604 14.6986751556396\\
14.1833324432373 14.7015905380249\\
14.1856698989868 14.7026424407959\\
14.1882400512695 14.7018547058105\\
14.191068649292 14.7017850875854\\
14.1941795349121 14.7014532089233\\
14.1976013183594 14.7014856338501\\
14.2013654708862 14.701319694519\\
14.2055063247681 14.701075553894\\
14.2100610733032 14.7008810043335\\
14.2150716781616 14.700740814209\\
14.2205829620361 14.7006196975708\\
14.2266454696655 14.7005205154419\\
14.2333135604858 14.7004175186157\\
14.2406492233276 14.70032787323\\
14.2487182617188 14.7003011703491\\
14.2575941085815 14.7003393173218\\
14.2673578262329 14.7001152038574\\
14.27809715271 14.7001924514771\\
14.2899112701416 14.7002668380737\\
14.3029069900513 14.7002716064453\\
14.3172016143799 14.7002477645874\\
14.3329257965088 14.7002487182617\\
14.3502225875854 14.7002773284912\\
14.3692483901978 14.7003488540649\\
14.3901777267456 14.7004194259644\\
14.4131994247437 14.7005281448364\\
14.4385232925415 14.7004671096802\\
14.4663791656494 14.7005062103271\\
14.4970216751099 14.7005243301392\\
14.520975112915 14.7005052566528\\
14.5473232269287 14.7005186080933\\
14.5763063430786 14.7005214691162\\
14.6081876754761 14.7005271911621\\
14.6432571411133 14.7005271911621\\
14.6818342208862 14.7005281448364\\
14.7242679595947 14.7005224227905\\
14.7709455490112 14.7005224227905\\
14.8222913742065 14.7005224227905\\
14.8712491989136 14.7005252838135\\
14.9185028076172 14.7005233764648\\
14.9638071060181 14.7005233764648\\
15.0066013336182 14.7005224227905\\
15.0464649200439 14.7005224227905\\
15.0903158187866 14.7005214691162\\
15.1385507583618 14.7005214691162\\
15.1916093826294 14.7005214691162\\
15.2499742507935 14.7005224227905\\
15.3141756057739 14.7005214691162\\
15.3847970962524 14.7005214691162\\
15.4624805450439 14.7005214691162\\
15.5479326248169 14.7005214691162\\
15.6419296264648 14.7005205154419\\
15.7419300079346 14.7005214691162\\
15.7919301986694 14.7005205154419\\
15.8469295501709 14.7005224227905\\
15.9074296951294 14.7005205154419\\
15.920783996582 14.7005205154419\\
15.9354724884033 14.7005233764648\\
15.9516305923462 14.7005205154419\\
15.9645795822144 14.7005224227905\\
15.9788236618042 14.7005224227905\\
15.9944925308228 14.7005224227905\\
16.0117282867432 14.7005214691162\\
16.0306873321533 14.7005214691162\\
16.0515422821045 14.7005214691162\\
16.0744819641113 14.7005214691162\\
16.0997161865234 14.7005214691162\\
16.1274738311768 14.7005214691162\\
16.1580085754395 14.7005214691162\\
16.1915950775146 14.7005214691162\\
16.2285404205322 14.7005214691162\\
16.2691822052002 14.7005214691162\\
16.3138847351074 14.7005214691162\\
16.3630599975586 14.7005214691162\\
16.4171524047852 14.7005214691162\\
16.4766540527344 14.7005214691162\\
16.542106628418 14.7005214691162\\
16.6141033172607 14.7005214691162\\
16.6932983398438 14.7005214691162\\
16.7804145812988 14.7005205154419\\
16.87624168396 14.7005214691162\\
16.9762420654297 14.7005205154419\\
17.0762424468994 14.700535774231\\
17.0962505340576 14.70055103302\\
17.1021957397461 14.7005481719971\\
17.1087341308594 14.7005443572998\\
17.1159267425537 14.7005443572998\\
17.1238384246826 14.7005481719971\\
17.1325416564941 14.70055103302\\
17.1421165466309 14.7005529403687\\
17.1526470184326 14.700553894043\\
17.1642322540283 14.70055103302\\
17.1769752502441 14.7005481719971\\
17.1909923553467 14.7005472183228\\
17.2064094543457 14.7005443572998\\
17.2233715057373 14.7005414962769\\
17.2420272827148 14.7005367279053\\
17.2625503540039 14.7005262374878\\
17.2851238250732 14.7005205154419\\
17.3099555969238 14.7005157470703\\
17.3372707366943 14.700511932373\\
17.367317199707 14.7005033493042\\
17.400369644165 14.7004804611206\\
17.4367256164551 14.7004528045654\\
17.4767169952393 14.7005262374878\\
17.5207099914551 14.7005233764648\\
17.5690994262695 14.7005214691162\\
17.6223297119141 14.7005186080933\\
17.6808815002441 14.7005195617676\\
17.745288848877 14.7005262374878\\
17.8161373138428 14.7005186080933\\
17.8624935150146 14.7005128860474\\
17.9028415679932 14.700511932373\\
17.9472217559814 14.7005195617676\\
17.9960422515869 14.7005205154419\\
18.0497417449951 14.7005214691162\\
18.108814239502 14.7005214691162\\
18.173791885376 14.7005214691162\\
18.245267868042 14.7005214691162\\
18.3238925933838 14.7005205154419\\
18.41037940979 14.7005186080933\\
18.5055141448975 14.7005224227905\\
18.6055145263672 14.7005195617676\\
18.7055149078369 14.7005214691162\\
18.805513381958 14.7005214691162\\
18.9055137634277 14.7005224227905\\
19.0055141448975 14.7005224227905\\
19.1055145263672 14.7005214691162\\
19.2055149078369 14.7005214691162\\
19.305513381958 14.7005214691162\\
19.4055137634277 14.7005214691162\\
19.5055141448975 14.7005214691162\\
19.6055145263672 14.7005214691162\\
19.7055149078369 14.7005214691162\\
19.805513381958 14.7005214691162\\
19.9055137634277 14.7005214691162\\
20.0055141448975 14.7005214691162\\
20.1055145263672 14.7005214691162\\
20.2055149078369 14.7005214691162\\
20.305513381958 14.7005214691162\\
20.4055137634277 14.7005214691162\\
20.5055141448975 14.7005214691162\\
20.6055145263672 14.7005214691162\\
20.7055149078369 14.7005214691162\\
20.805513381958 14.7005214691162\\
20.9055137634277 14.7005214691162\\
21.0055141448975 14.7005214691162\\
21.1055145263672 14.7005214691162\\
21.2055149078369 14.7005214691162\\
21.305513381958 14.7005214691162\\
21.4055137634277 14.7005214691162\\
21.5055141448975 14.7005214691162\\
21.6055145263672 14.7005214691162\\
21.7055149078369 14.7005214691162\\
21.805513381958 14.7005214691162\\
21.9055137634277 14.7005214691162\\
22.0055141448975 14.7005214691162\\
22.1055145263672 14.7005214691162\\
22.2055149078369 14.7005214691162\\
22.305513381958 14.7005214691162\\
22.4055137634277 14.7005214691162\\
22.5055141448975 14.7005214691162\\
22.6055145263672 14.7005214691162\\
22.7055149078369 14.7005214691162\\
22.805513381958 14.7005214691162\\
22.9055137634277 14.7005214691162\\
23.0055141448975 14.7005214691162\\
23.1055145263672 14.7005214691162\\
23.2055149078369 14.7005214691162\\
23.305513381958 14.7005214691162\\
23.4055137634277 14.7005214691162\\
23.5055141448975 14.7005214691162\\
23.6055145263672 14.7005214691162\\
23.7055149078369 14.7005214691162\\
23.805513381958 14.7005214691162\\
23.9055137634277 14.7005214691162\\
24.0055141448975 14.7005214691162\\
24.1055145263672 14.7005214691162\\
24.2055149078369 14.7005214691162\\
24.305513381958 14.7005214691162\\
24.4055137634277 14.7005214691162\\
24.5055141448975 14.7005214691162\\
24.6055145263672 14.7005214691162\\
24.7055149078369 14.7005214691162\\
24.789867401123 14.7005214691162\\
24.8573379516602 14.7005224227905\\
24.9315567016602 14.7005214691162\\
25.0131969451904 14.7005205154419\\
25.1030006408691 14.7005205154419\\
25.1443557739258 14.7005224227905\\
25.189847946167 14.7005176544189\\
25.2099170684814 14.7005290985107\\
25.2266445159912 14.7005243301392\\
25.245044708252 14.7005224227905\\
25.2652854919434 14.7005214691162\\
25.2875499725342 14.7005214691162\\
25.3120403289795 14.7005214691162\\
25.3389797210693 14.7005214691162\\
25.3686122894287 14.7005214691162\\
25.4012107849121 14.7005214691162\\
25.4370670318604 14.7005214691162\\
25.4765090942383 14.7005214691162\\
25.5198955535889 14.7005214691162\\
25.5676212310791 14.7005214691162\\
25.6201171875 14.7005214691162\\
25.6778659820557 14.7005214691162\\
25.7413883209229 14.7005214691162\\
25.8112621307373 14.7005214691162\\
25.8881244659424 14.7005214691162\\
25.9726715087891 14.7005205154419\\
26.065673828125 14.7005224227905\\
26.1656742095947 14.7005205154419\\
26.2656745910645 14.7005367279053\\
26.3656749725342 14.7006921768188\\
26.409065246582 14.7017135620117\\
26.4240894317627 14.7040319442749\\
26.4268970489502 14.679594039917\\
26.4299831390381 14.6945314407349\\
26.4333801269531 14.7002611160278\\
26.4371166229248 14.7028903961182\\
26.4412250518799 14.7019510269165\\
26.4457454681396 14.7024879455566\\
26.4507179260254 14.7020864486694\\
26.4561862945557 14.7014570236206\\
26.4622020721436 14.7011957168579\\
26.4688205718994 14.70094871521\\
26.4761009216309 14.7006912231445\\
26.4841079711914 14.7009782791138\\
26.4929161071777 14.6998996734619\\
26.5026054382324 14.7000255584717\\
26.5132637023926 14.6993074417114\\
26.5249881744385 14.6985559463501\\
26.5378837585449 14.6996326446533\\
26.5520706176758 14.7003126144409\\
26.5676746368408 14.7004976272583\\
26.5848388671875 14.7005586624146\\
26.6037216186523 14.7017478942871\\
26.624490737915 14.6995964050293\\
26.6473369598389 14.7001476287842\\
26.6724681854248 14.7006454467773\\
26.7001132965088 14.7005205154419\\
26.7305221557617 14.7004957199097\\
26.7594413757324 14.7004947662354\\
26.7881412506104 14.7005081176758\\
26.8180274963379 14.7005176544189\\
26.848762512207 14.7005205154419\\
26.8805618286133 14.7005233764648\\
26.9128971099854 14.7005224227905\\
26.9456844329834 14.7005224227905\\
26.97873878479 14.7005224227905\\
27.0120601654053 14.7005224227905\\
27.0457935333252 14.7005224227905\\
27.0821418762207 14.7005233764648\\
27.1221237182617 14.7005243301392\\
27.1524639129639 14.7005243301392\\
27.1838111877441 14.7005233764648\\
27.2182941436768 14.7005233764648\\
27.2558403015137 14.7005224227905\\
27.2955665588379 14.7005224227905\\
27.3392658233643 14.7005224227905\\
27.3873329162598 14.7005214691162\\
27.4402103424072 14.7005214691162\\
27.4983730316162 14.7005214691162\\
27.5623531341553 14.7005224227905\\
27.6327304840088 14.7005214691162\\
27.7101459503174 14.7005214691162\\
27.7953014373779 14.7005214691162\\
27.8889751434326 14.7005205154419\\
27.9889755249023 14.7005224227905\\
28.0889739990234 14.7005205154419\\
28.1889743804932 14.7005224227905\\
28.2889747619629 14.7005205154419\\
28.2927665710449 14.7005167007446\\
28.2969398498535 14.7005300521851\\
28.3015289306641 14.7004632949829\\
28.304780960083 14.7005815505981\\
28.307991027832 14.7005405426025\\
28.3115215301514 14.7005233764648\\
28.3154048919678 14.7005214691162\\
28.3196773529053 14.7005233764648\\
28.3243770599365 14.7005233764648\\
28.3295459747314 14.7005233764648\\
28.3352317810059 14.7005224227905\\
28.3414859771729 14.7005224227905\\
28.3483657836914 14.7005224227905\\
28.3559341430664 14.7005224227905\\
28.3642578125 14.7005224227905\\
28.3734169006348 14.7005224227905\\
28.3834896087646 14.7005224227905\\
28.3945693969727 14.7005214691162\\
28.4067573547363 14.7005214691162\\
28.4201641082764 14.7005214691162\\
28.4349136352539 14.7005214691162\\
28.451135635376 14.7005214691162\\
28.4689807891846 14.7005214691162\\
28.4886093139648 14.7005214691162\\
28.5102024078369 14.7005214691162\\
28.5339546203613 14.7005214691162\\
28.5600814819336 14.7005214691162\\
28.5888214111328 14.7005214691162\\
28.620433807373 14.7005214691162\\
28.6552085876465 14.7005214691162\\
28.6934604644775 14.7005214691162\\
28.735538482666 14.7005214691162\\
28.7818241119385 14.7005214691162\\
28.8327369689941 14.7005214691162\\
28.8887424468994 14.7005214691162\\
28.9503479003906 14.7005214691162\\
29.0181159973145 14.7005214691162\\
29.092658996582 14.7005214691162\\
29.1746559143066 14.7005214691162\\
29.2648525238037 14.7005205154419\\
29.3640689849854 14.7005224227905\\
29.4640693664551 14.7005205154419\\
29.4847068786621 14.7008361816406\\
29.4888343811035 14.7007741928101\\
29.4933738708496 14.7007970809937\\
29.4983692169189 14.70081615448\\
29.5038623809814 14.7008647918701\\
29.5099067687988 14.7009239196777\\
29.5165538787842 14.7009906768799\\
29.5230560302734 14.7010650634766\\
29.5302066802979 14.7011346817017\\
29.5380744934082 14.7011613845825\\
29.5467281341553 14.7010793685913\\
29.556245803833 14.6992692947388\\
29.5667171478271 14.6978635787964\\
29.5782356262207 14.7019662857056\\
29.5909042358398 14.701153755188\\
29.6048412322998 14.7009496688843\\
29.6201705932617 14.7015628814697\\
29.637035369873 14.7008209228516\\
29.6555843353271 14.7006845474243\\
29.675989151001 14.7006616592407\\
29.6984329223633 14.7003402709961\\
29.7114696502686 14.7002182006836\\
29.7258071899414 14.7001867294312\\
29.7415809631348 14.7001638412476\\
29.7589302062988 14.7002286911011\\
29.7780151367188 14.6995000839233\\
29.7990093231201 14.7001218795776\\
29.8221015930176 14.700608253479\\
29.8475036621094 14.7004766464233\\
29.8754444122314 14.7005472183228\\
29.9061813354492 14.7005195617676\\
29.9399909973145 14.7005462646484\\
29.9771823883057 14.7005319595337\\
30.0180931091309 14.7005195617676\\
30.0630931854248 14.7005300521851\\
30.1125946044922 14.7005338668823\\
30.1670475006104 14.7005167007446\\
30.2269439697266 14.7005348205566\\
30.2706527709961 14.7005710601807\\
30.318733215332 14.7005176544189\\
30.3593273162842 14.7005252838135\\
30.3947944641113 14.7005109786987\\
30.4330863952637 14.7005071640015\\
30.4652576446533 14.7005109786987\\
30.5006484985352 14.7005157470703\\
30.5330505371094 14.7005186080933\\
30.5677318572998 14.7005195617676\\
30.6004276275635 14.7005205154419\\
30.6349658966064 14.7005205154419\\
30.6682662963867 14.7005205154419\\
30.7048969268799 14.7005205154419\\
30.7451915740967 14.7005224227905\\
30.789514541626 14.7005205154419\\
30.8382682800293 14.7005205154419\\
30.8918991088867 14.7005195617676\\
30.9508934020996 14.7005186080933\\
31.0157871246338 14.7005205154419\\
31.0871696472168 14.7005205154419\\
31.1656913757324 14.7005233764648\\
31.2520637512207 14.7005214691162\\
31.347074508667 14.7005214691162\\
31.4470748901367 14.7005224227905\\
31.5470752716064 14.7005214691162\\
31.6470737457275 14.7005214691162\\
31.7470741271973 14.7005214691162\\
31.847074508667 14.7005214691162\\
31.9470748901367 14.7005214691162\\
32.0470733642578 14.7005214691162\\
32.1470756530762 14.7005214691162\\
32.2470741271973 14.7005214691162\\
32.3470764160156 14.7005214691162\\
32.4470748901367 14.7005214691162\\
32.5470733642578 14.7005214691162\\
32.6470756530762 14.7005214691162\\
32.7470741271973 14.7005214691162\\
32.8470764160156 14.7005214691162\\
32.9470748901367 14.7005214691162\\
33.0470733642578 14.7005214691162\\
33.1470756530762 14.7005214691162\\
33.2470741271973 14.7005214691162\\
33.3470764160156 14.7005214691162\\
33.4470748901367 14.7005214691162\\
33.5470733642578 14.7005214691162\\
33.6470756530762 14.7005214691162\\
33.7470741271973 14.7005214691162\\
33.8470764160156 14.7005214691162\\
33.9470748901367 14.7005214691162\\
34.0470733642578 14.7005214691162\\
34.1470756530762 14.7005214691162\\
34.2470741271973 14.7005214691162\\
34.3470764160156 14.7005214691162\\
34.4470748901367 14.7005214691162\\
34.5470733642578 14.7005214691162\\
34.6470756530762 14.7005214691162\\
34.7470741271973 14.7005214691162\\
34.8470764160156 14.7005214691162\\
34.9470748901367 14.7005214691162\\
35.0470733642578 14.7005214691162\\
35.1470756530762 14.7005214691162\\
35.2470741271973 14.7005214691162\\
35.3470764160156 14.7005214691162\\
35.4470748901367 14.7005214691162\\
35.5470733642578 14.7005214691162\\
35.6470756530762 14.7005214691162\\
35.7470741271973 14.7005214691162\\
35.8470764160156 14.7005214691162\\
35.9470748901367 14.7005214691162\\
36.0470733642578 14.7005214691162\\
36.1470756530762 14.7005214691162\\
36.2470741271973 14.7005214691162\\
36.3470764160156 14.7005214691162\\
36.4470748901367 14.7005214691162\\
36.5470733642578 14.7005214691162\\
36.6470756530762 14.7005214691162\\
36.7470741271973 14.7005214691162\\
36.8470764160156 14.7005214691162\\
36.9470748901367 14.7005214691162\\
37.0470733642578 14.7005214691162\\
37.1470756530762 14.7005205154419\\
37.2293395996094 14.7005205154419\\
37.2951393127441 14.7005214691162\\
37.3675231933594 14.7005205154419\\
37.4471435546875 14.7005224227905\\
37.534725189209 14.7005224227905\\
37.6310653686523 14.7005214691162\\
37.6772041320801 14.7005224227905\\
37.7279586791992 14.7005157470703\\
37.7453880310059 14.7005386352539\\
37.7602653503418 14.7005252838135\\
37.7766304016113 14.7005233764648\\
37.7946319580078 14.7005233764648\\
37.8144378662109 14.7005233764648\\
37.8362197875977 14.7005224227905\\
37.860179901123 14.7005224227905\\
37.8865356445313 14.7005224227905\\
37.9155311584473 14.7005224227905\\
37.9474220275879 14.7005224227905\\
37.9825019836426 14.7005224227905\\
38.0210914611816 14.7005224227905\\
38.0635414123535 14.7005224227905\\
38.1102333068848 14.7005224227905\\
38.1615982055664 14.7005224227905\\
38.2180976867676 14.7005224227905\\
38.2802429199219 14.7005224227905\\
38.3486099243164 14.7005224227905\\
38.4238090515137 14.7005224227905\\
38.5065269470215 14.7005214691162\\
38.5975189208984 14.7005233764648\\
38.6975212097168 14.7005224227905\\
38.7975196838379 14.7005424499512\\
38.8975219726563 14.7006349563599\\
38.9475212097168 14.7014837265015\\
38.9856338500977 14.7027235031128\\
39.0002326965332 14.8218641281128\\
39.0162887573242 14.8336391448975\\
39.0339508056641 14.7699060440063\\
39.0533828735352 14.7605018615723\\
39.0747528076172 14.7409057617188\\
39.0982627868652 14.7352437973022\\
39.1241226196289 14.7261028289795\\
39.1525688171387 14.724986076355\\
39.1838569641113 14.7174682617188\\
39.2182769775391 14.7027702331543\\
39.2382202148438 14.7020587921143\\
39.2601547241211 14.7016553878784\\
39.2842864990234 14.7009305953979\\
39.3108291625977 14.7005643844604\\
39.3400230407715 14.7005043029785\\
39.3721389770508 14.7005071640015\\
39.4047470092773 14.7005128860474\\
39.4365348815918 14.7005224227905\\
39.4690742492676 14.7005414962769\\
39.5013847351074 14.7005424499512\\
39.5341873168945 14.7005233764648\\
39.5668563842773 14.7005243301392\\
39.5998191833496 14.7005224227905\\
39.632698059082 14.7005224227905\\
39.6657829284668 14.7005224227905\\
39.6988639831543 14.7005243301392\\
39.7321624755859 14.7005233764648\\
39.7687873840332 14.7005224227905\\
39.8090744018555 14.7005214691162\\
39.8533935546875 14.7005214691162\\
39.8968162536621 14.7005205154419\\
39.939453125 14.7005195617676\\
39.9863548278809 14.7005205154419\\
40.0379447937012 14.7005214691162\\
40.0946960449219 14.7005224227905\\
40.1571197509766 14.7005224227905\\
40.2257843017578 14.7005224227905\\
40.3013191223145 14.7005224227905\\
40.384407043457 14.7005224227905\\
40.4758033752441 14.7005214691162\\
40.5758018493652 14.7005233764648\\
40.6758041381836 14.7005224227905\\
40.7758026123047 14.7005233764648\\
40.875804901123 14.7005224227905\\
40.8792419433594 14.7005090713501\\
40.8830223083496 14.7005481719971\\
40.8871841430664 14.7004261016846\\
40.8903007507324 14.7006130218506\\
40.8937301635742 14.7005472183228\\
40.8975028991699 14.7005252838135\\
40.9016532897949 14.7005224227905\\
40.9062156677246 14.7005243301392\\
40.9112396240234 14.7005243301392\\
40.9167633056641 14.7005243301392\\
40.9228363037109 14.7005243301392\\
40.9295196533203 14.7005243301392\\
40.9368743896484 14.7005233764648\\
40.9449615478516 14.7005233764648\\
40.953857421875 14.7005233764648\\
40.9636421203613 14.7005233764648\\
40.9744033813477 14.7005233764648\\
40.9862442016602 14.7005233764648\\
40.999267578125 14.7005233764648\\
41.0135955810547 14.7005233764648\\
41.029354095459 14.7005233764648\\
41.046688079834 14.7005233764648\\
41.0657577514648 14.7005233764648\\
41.0867347717285 14.7005224227905\\
41.109806060791 14.7005224227905\\
41.1351852416992 14.7005224227905\\
41.1631050109863 14.7005224227905\\
41.1938133239746 14.7005224227905\\
41.227596282959 14.7005224227905\\
41.2647552490234 14.7005224227905\\
41.3056297302246 14.7005224227905\\
41.3505935668945 14.7005224227905\\
41.4000511169434 14.7005224227905\\
41.4544563293457 14.7005224227905\\
41.5143013000488 14.7005224227905\\
41.5801315307617 14.7005224227905\\
41.6525421142578 14.7005224227905\\
41.7321968078613 14.7005224227905\\
41.8198165893555 14.7005214691162\\
41.9161987304688 14.7005233764648\\
42.0161972045898 14.7005224227905\\
42.1161994934082 14.7005662918091\\
42.1361999511719 14.700756072998\\
42.1442070007324 14.7009811401367\\
42.1520309448242 14.7010622024536\\
42.1606330871582 14.7011251449585\\
42.1689529418945 14.70130443573\\
42.1781005859375 14.7069730758667\\
42.1881637573242 14.7789726257324\\
42.1992378234863 14.8953943252563\\
42.2114143371582 14.8294239044189\\
42.2248077392578 14.7825241088867\\
42.2395439147949 14.7758989334106\\
42.2557525634766 14.771656036377\\
42.2735824584961 14.7628574371338\\
42.2931938171387 14.7536773681641\\
42.3147659301758 14.7434539794922\\
42.3384971618652 14.7132425308228\\
42.3646011352539 14.7018718719482\\
42.3847808837891 14.7178812026978\\
42.4069786071777 14.7434091567993\\
42.431396484375 14.7503242492676\\
42.4582557678223 14.7466602325439\\
42.4878005981445 14.7564325332642\\
42.5203018188477 14.7627782821655\\
42.5560493469238 14.7591323852539\\
42.5953750610352 14.7552919387817\\
42.6386299133301 14.7415943145752\\
42.6862144470215 14.758056640625\\
42.7385559082031 14.7251062393188\\
42.7959022521973 14.7442684173584\\
42.8589859008789 14.7337284088135\\
42.9028167724609 14.7218246459961\\
42.9510307312012 14.7238111495972\\
42.9925117492676 14.7177438735962\\
43.0297622680664 14.7141418457031\\
43.0707359313965 14.7126760482788\\
43.1158065795898 14.7130670547485\\
43.1579742431641 14.7154331207275\\
43.203685760498 14.7177610397339\\
43.2497863769531 14.7213020324707\\
43.3005027770996 14.7234125137329\\
43.3562850952148 14.7260751724243\\
43.4176483154297 14.7283363342285\\
43.4851493835449 14.7303762435913\\
43.559398651123 14.7315559387207\\
43.6410713195801 14.7324800491333\\
43.7309150695801 14.7327299118042\\
43.8297386169434 14.7327213287354\\
43.9297409057617 14.7326316833496\\
44.0297393798828 14.7334089279175\\
44.1297416687012 14.7337350845337\\
44.2297401428223 14.7343606948853\\
44.2499618530273 14.7344675064087\\
44.2722015380859 14.7345676422119\\
44.2966690063477 14.7346906661987\\
44.3235816955566 14.7348213195801\\
44.3531875610352 14.7349653244019\\
44.3857498168945 14.7351236343384\\
44.421573638916 14.735297203064\\
44.4609756469727 14.7354888916016\\
44.5043182373047 14.7356996536255\\
44.5519981384277 14.7359323501587\\
44.6044425964355 14.7361869812012\\
44.6621360778809 14.7364683151245\\
44.7255935668945 14.7367763519287\\
44.7953987121582 14.7371168136597\\
44.872184753418 14.7374897003174\\
44.9566497802734 14.7379007339478\\
45.0495643615723 14.738353729248\\
45.1495628356934 14.7388324737549\\
45.2495613098145 14.7393960952759\\
45.3495635986328 14.7399053573608\\
45.4495620727539 14.740270614624\\
45.5495643615723 14.7406969070435\\
45.6495628356934 14.7413082122803\\
45.7495613098145 14.7454128265381\\
45.7695655822754 14.7518892288208\\
45.7816352844238 14.7558612823486\\
45.7949142456055 14.7579107284546\\
45.8095207214355 14.7593936920166\\
45.8255882263184 14.7592372894287\\
45.8432655334473 14.7587327957153\\
45.8627052307129 14.758355140686\\
45.884090423584 14.7574462890625\\
45.9076156616211 14.7587394714355\\
45.9334945678711 14.7569437026978\\
45.9619598388672 14.7548732757568\\
45.9932708740234 14.7536277770996\\
46.0277099609375 14.752965927124\\
46.0655975341797 14.7512979507446\\
46.1072731018066 14.7498359680176\\
46.1531181335449 14.7489671707153\\
46.2035446166992 14.7484302520752\\
46.2590141296387 14.7481298446655\\
46.3200302124023 14.7493534088135\\
46.387149810791 14.7507944107056\\
46.4504814147949 14.750659942627\\
46.5130958557129 14.7503032684326\\
46.581974029541 14.7500820159912\\
46.6577415466309 14.7501611709595\\
46.7410850524902 14.7502708435059\\
46.8327598571777 14.7505140304565\\
46.9327621459961 14.7509918212891\\
47.0327606201172 14.7514228820801\\
47.1327629089355 14.751880645752\\
47.2327613830566 14.7523813247681\\
47.3327598571777 14.7529096603394\\
47.4327621459961 14.7534608840942\\
47.5327606201172 14.7540311813354\\
47.6327629089355 14.7544317245483\\
47.7327613830566 14.7549276351929\\
47.8327598571777 14.7554235458374\\
47.9327621459961 14.7559471130371\\
48.0327606201172 14.7564735412598\\
48.1327629089355 14.757007598877\\
48.2327613830566 14.7575435638428\\
48.3327598571777 14.7580814361572\\
48.4327621459961 14.758620262146\\
48.5327606201172 14.7591609954834\\
48.6327629089355 14.7597026824951\\
48.7327613830566 14.7602472305298\\
48.8327598571777 14.7607927322388\\
48.9327621459961 14.7613220214844\\
49.0327606201172 14.7618780136108\\
49.1327629089355 14.7624464035034\\
49.2327613830566 14.763032913208\\
49.3327598571777 14.7636499404907\\
49.4327621459961 14.7643251419067\\
49.5327606201172 14.7707414627075\\
49.6257057189941 14.7741250991821\\
49.6982688903809 14.7725954055786\\
49.7780876159668 14.7682867050171\\
49.8658905029297 14.7687177658081\\
49.962474822998 14.7692365646362\\
50.0624732971191 14.7700414657593\\
50.1624755859375 14.7710103988647\\
50.2624740600586 14.7720394134521\\
50.3624725341797 14.7730770111084\\
50.462474822998 14.7741355895996\\
50.5624732971191 14.7752056121826\\
50.6624755859375 14.7762870788574\\
50.7624740600586 14.7773761749268\\
50.8624725341797 14.778468132019\\
50.962474822998 14.7795686721802\\
51.0624732971191 14.7806739807129\\
51.1624755859375 14.7817821502686\\
51.2624740600586 14.7828922271729\\
51.3624725341797 14.7840032577515\\
51.462474822998 14.785117149353\\
51.5624732971191 14.7862300872803\\
51.6624755859375 14.7873430252075\\
51.7624740600586 14.7884569168091\\
51.8624725341797 14.7895698547363\\
51.962474822998 14.7906837463379\\
52.0624732971191 14.7917966842651\\
52.1624755859375 14.7929105758667\\
52.2624740600586 14.7940235137939\\
52.3624725341797 14.7951374053955\\
52.462474822998 14.7962512969971\\
52.5624732971191 14.7973642349243\\
52.6624755859375 14.7984781265259\\
52.7624740600586 14.7995910644531\\
52.8624725341797 14.8007049560547\\
52.962474822998 14.8018188476563\\
53.0624732971191 14.8029327392578\\
53.1624755859375 14.8040466308594\\
53.2624740600586 14.8051595687866\\
53.3624725341797 14.8062734603882\\
53.462474822998 14.8073873519897\\
53.5624732971191 14.8085012435913\\
53.6624755859375 14.8096151351929\\
53.7624740600586 14.8107290267944\\
53.8624725341797 14.8118438720703\\
53.962474822998 14.8129577636719\\
54.0624732971191 14.8140716552734\\
54.1624755859375 14.8151865005493\\
54.2624740600586 14.8163003921509\\
54.3624725341797 14.8174142837524\\
54.462474822998 14.8185291290283\\
54.5624732971191 14.8196439743042\\
54.6624755859375 14.8207578659058\\
54.7624740600586 14.8218727111816\\
54.8624725341797 14.8229875564575\\
54.962474822998 14.8241024017334\\
55.0624732971191 14.8252182006836\\
55.1624755859375 14.8263339996338\\
55.2624740600586 14.8274564743042\\
55.3624725341797 14.8286056518555\\
55.462474822998 14.8299016952515\\
55.5624732971191 14.8319473266602\\
55.6624755859375 14.8365468978882\\
55.7624740600586 14.8439874649048\\
55.8624725341797 14.8449983596802\\
55.962474822998 14.8405065536499\\
56.0624732971191 14.8376092910767\\
56.1624755859375 14.8396768569946\\
56.2624740600586 14.8384313583374\\
56.3624725341797 14.8409013748169\\
56.462474822998 14.8411121368408\\
56.5624732971191 14.8428220748901\\
56.6624755859375 14.8435468673706\\
56.7624740600586 14.8449201583862\\
56.8624725341797 14.8458681106567\\
56.962474822998 14.8470964431763\\
57.0624732971191 14.8481407165527\\
57.1624755859375 14.8493061065674\\
57.2624740600586 14.8503923416138\\
57.3624725341797 14.8515310287476\\
57.462474822998 14.8526496887207\\
57.5624732971191 14.8537664413452\\
57.6624755859375 14.8548803329468\\
57.7624740600586 14.8560085296631\\
57.8624725341797 14.8571271896362\\
57.962474822998 14.8582515716553\\
58.0624732971191 14.8593730926514\\
58.1624755859375 14.8604965209961\\
58.2624740600586 14.8616189956665\\
58.3624725341797 14.8628063201904\\
58.462474822998 14.8641967773438\\
58.5624732971191 14.865255355835\\
58.6624755859375 14.866400718689\\
58.7624740600586 14.8675174713135\\
58.8624725341797 14.8686447143555\\
58.962474822998 14.8697681427002\\
59.0624732971191 14.8708934783936\\
59.1624755859375 14.8720188140869\\
59.2624740600586 14.8731451034546\\
59.3624725341797 14.8742713928223\\
59.462474822998 14.8753976821899\\
59.5624732971191 14.8765249252319\\
59.6624755859375 14.8776531219482\\
59.7624740600586 14.8787813186646\\
59.8624725341797 14.8799076080322\\
59.9312362670898 14.8807582855225\\
60 14.8815355300903\\
60.0756378173828 14.8822994232178\\
60.1588439941406 14.8831443786621\\
60.2503662109375 14.8840742111206\\
60.3503646850586 14.8850965499878\\
60.450366973877 14.8862037658691\\
60.550365447998 14.8873262405396\\
60.6503677368164 14.8884420394897\\
60.7503662109375 14.8895597457886\\
60.8503646850586 14.8906764984131\\
60.950366973877 14.8917989730835\\
61.050365447998 14.8929538726807\\
61.1503677368164 14.8941221237183\\
61.2503662109375 14.8952674865723\\
61.3503646850586 14.8964147567749\\
61.450366973877 14.8975763320923\\
61.550365447998 14.8987579345703\\
61.6503677368164 14.8999757766724\\
61.7503662109375 14.9016246795654\\
61.8503646850586 14.9096021652222\\
61.9325637817383 14.9123849868774\\
61.998291015625 14.9062013626099\\
62.0705947875977 14.9048023223877\\
62.1501274108887 14.9054098129272\\
62.2376098632813 14.9060506820679\\
62.3338470458984 14.9068946838379\\
62.4338455200195 14.9079217910767\\
62.5338439941406 14.9090270996094\\
62.633846282959 14.9101276397705\\
62.7338447570801 14.9112339019775\\
62.8338470458984 14.9123439788818\\
62.9338455200195 14.9134588241577\\
63.0338439941406 14.9145727157593\\
63.133846282959 14.9156875610352\\
63.2338447570801 14.9168062210083\\
63.3338470458984 14.9179258346558\\
63.4338455200195 14.9190464019775\\
63.5338439941406 14.9201707839966\\
63.633846282959 14.9212951660156\\
63.7338447570801 14.9224195480347\\
63.8338470458984 14.9235429763794\\
63.9338455200195 14.9246673583984\\
64.0338439941406 14.9257907867432\\
64.1338424682617 14.9269151687622\\
64.2338485717773 14.9280385971069\\
64.3338470458984 14.929162979126\\
64.4338455200195 14.930287361145\\
64.5338439941406 14.9314107894897\\
64.6338424682617 14.9325351715088\\
64.7338485717773 14.9336595535278\\
64.8338470458984 14.9347829818726\\
64.9338455200195 14.9359073638916\\
65.0338439941406 14.9370317459106\\
65.1338424682617 14.9381551742554\\
65.2338485717773 14.9392795562744\\
65.3338470458984 14.9404039382935\\
65.4338455200195 14.9415283203125\\
65.5338439941406 14.9426517486572\\
65.6338424682617 14.9437761306763\\
65.7338485717773 14.9449005126953\\
65.8338470458984 14.9460248947144\\
65.9338455200195 14.9471483230591\\
66.0338439941406 14.9482727050781\\
66.1338424682617 14.9493970870972\\
66.2338485717773 14.9505214691162\\
66.3338470458984 14.9516458511353\\
66.4338455200195 14.9527702331543\\
66.5338439941406 14.9538946151733\\
66.6338424682617 14.9550189971924\\
66.7338485717773 14.9561433792114\\
66.8338470458984 14.9572677612305\\
66.9338455200195 14.9583921432495\\
67.0338439941406 14.9595165252686\\
67.1338424682617 14.9606409072876\\
67.2338485717773 14.9617652893066\\
67.3338470458984 14.9628896713257\\
67.4338455200195 14.9640159606934\\
67.5338439941406 14.9651460647583\\
67.6338424682617 14.9662990570068\\
67.7338485717773 14.9675731658936\\
67.8338470458984 14.9694795608521\\
67.9338455200195 14.9737119674683\\
68.0338439941406 14.9811325073242\\
68.1338424682617 14.9836797714233\\
68.2338485717773 14.9784126281738\\
68.3338470458984 14.9753942489624\\
68.4338455200195 14.9776735305786\\
68.5338439941406 14.976149559021\\
68.6338424682617 14.978720664978\\
68.7338485717773 14.9788665771484\\
68.8338470458984 14.9806346893311\\
68.9338455200195 14.981333732605\\
69.0338439941406 14.982738494873\\
69.1338424682617 14.983678817749\\
69.2338485717773 14.9849252700806\\
69.3338470458984 14.985969543457\\
69.4338455200195 14.9871482849121\\
69.5338439941406 14.9882383346558\\
69.6338424682617 14.9893865585327\\
69.7338485717773 14.9904975891113\\
69.8338470458984 14.9916334152222\\
69.9338455200195 14.9927530288696\\
70.0338439941406 14.9938831329346\\
70.1338424682617 14.9950065612793\\
70.2338485717773 14.9961347579956\\
70.3338470458984 14.997260093689\\
70.4338455200195 14.998387336731\\
70.5338439941406 14.9995136260986\\
70.6338424682617 15.0007171630859\\
70.7338485717773 15.0020961761475\\
70.8338470458984 15.0031661987305\\
70.9338455200195 15.0043134689331\\
71.0338439941406 15.0054349899292\\
71.1338424682617 15.0065650939941\\
71.2338485717773 15.0076923370361\\
71.3338470458984 15.0088214874268\\
71.4338455200195 15.0099506378174\\
71.5338439941406 15.011079788208\\
71.6338424682617 15.0122098922729\\
71.7338485717773 15.0133399963379\\
71.8338470458984 15.0144701004028\\
71.9338455200195 15.0156021118164\\
72.0338439941406 15.01673412323\\
72.1338424682617 15.0178604125977\\
72.2338485717773 15.018726348877\\
72.3338470458984 15.0198259353638\\
72.4338455200195 15.0209493637085\\
72.5338439941406 15.022068977356\\
72.6338424682617 15.0231885910034\\
72.7338485717773 15.0243082046509\\
72.8338470458984 15.025426864624\\
72.9338455200195 15.0265455245972\\
73.0338439941406 15.0276641845703\\
73.1338424682617 15.0287818908691\\
73.2338485717773 15.0299139022827\\
73.3338470458984 15.0310945510864\\
73.4338455200195 15.0322761535645\\
73.5338439941406 15.0334253311157\\
73.6338424682617 15.0345783233643\\
73.7338485717773 15.0357465744019\\
73.8338470458984 15.0369386672974\\
73.9338455200195 15.0381679534912\\
74.0338439941406 15.0410251617432\\
74.1323699951172 15.0478515625\\
74.2112121582031 15.0510091781616\\
74.2755126953125 15.0425672531128\\
74.3462524414063 15.0428628921509\\
74.4240570068359 15.0434484481812\\
74.509651184082 15.0440874099731\\
74.6037979125977 15.0449199676514\\
74.7037963867188 15.0459308624268\\
74.8037948608398 15.0470390319824\\
74.9019012451172 15.0481424331665\\
75 15.0492305755615\\
};
\addlegendentry{14 "};

\addplot [
color=green,
solid
]
table[row sep=crcr]{
0 14.7004346847534\\
9.99999971718069e-10 14.7004346847534\\
2.49999998480632e-09 14.7004346847534\\
4.74999994892755e-09 14.7004346847534\\
8.1250002281763e-09 14.7004346847534\\
1.31874999809156e-08 14.7004346847534\\
2.0781250498203e-08 14.7004346847534\\
3.21718758300449e-08 14.7004346847534\\
4.92578138278077e-08 14.7004346847534\\
7.48867208244519e-08 14.7004346847534\\
1.13330081319418e-07 14.7004346847534\\
1.70995122061868e-07 14.7004346847534\\
2.57492672517401e-07 14.7004346847534\\
3.8723902662241e-07 14.7004346847534\\
5.81858500936505e-07 14.7004346847534\\
8.73787769251066e-07 14.7004346847534\\
1.31168167172291e-06 14.7004346847534\\
1.9685226106958e-06 14.7004346847534\\
2.95378367809462e-06 14.7004346847534\\
4.43167573394021e-06 14.7004346847534\\
6.64851359033491e-06 14.7004346847534\\
9.97376992017962e-06 14.7004346847534\\
1.49616553244414e-05 14.7004346847534\\
2.24434825213393e-05 14.7004346847534\\
3.36662233166862e-05 14.7004346847534\\
5.05003372381907e-05 14.7004356384277\\
7.57515081204474e-05 14.7004356384277\\
0.000113628258986864 14.7004356384277\\
0.000170443381648511 14.7004365921021\\
0.000255666091106832 14.7004375457764\\
0.000383500126190484 14.7004384994507\\
0.000575251178815961 14.7004404067993\\
0.000862877757754177 14.7004413604736\\
0.00129431765526533 14.7004413604736\\
0.00194147753063589 14.7004375457764\\
0.00291221728548408 14.7004327774048\\
0.00436832662671804 14.7004318237305\\
0.00597004732117057 14.7004327774048\\
0.00773194013163447 14.7004346847534\\
0.009670021943748 14.7004356384277\\
0.0118019115179777 14.7004356384277\\
0.0141469910740852 14.7004346847534\\
0.0167265776544809 14.7004346847534\\
0.0195641238242388 14.7004346847534\\
0.0226854234933853 14.7004346847534\\
0.0261188540607691 14.7004346847534\\
0.0298956278711557 14.7004346847534\\
0.0340500771999359 14.7004346847534\\
0.0386199727654457 14.7004346847534\\
0.0436468608677387 14.7004346847534\\
0.0491764321923256 14.7004346847534\\
0.0552589632570744 14.7004346847534\\
0.0619497485458851 14.7004346847534\\
0.0693096145987511 14.7004346847534\\
0.0774054601788521 14.7004346847534\\
0.0863108932971954 14.7004337310791\\
0.0961068719625473 14.7004346847534\\
0.106882445514202 14.7004346847534\\
0.118735581636429 14.7004346847534\\
0.13177402317524 14.7004346847534\\
0.146116316318512 14.7004346847534\\
0.161892831325531 14.7004346847534\\
0.179247006773949 14.7004346847534\\
0.198336601257324 14.7004346847534\\
0.21933513879776 14.7004346847534\\
0.242433547973633 14.7004346847534\\
0.267841786146164 14.7004346847534\\
0.29579085111618 14.7004346847534\\
0.32653483748436 14.7004346847534\\
0.360353201627731 14.7004346847534\\
0.397553414106369 14.7004346847534\\
0.438473641872406 14.7004346847534\\
0.483485877513886 14.7004346847534\\
0.532999336719513 14.7004346847534\\
0.587464153766632 14.7004346847534\\
0.647375464439392 14.7004346847534\\
0.71327793598175 14.7004346847534\\
0.7857705950737 14.7004346847534\\
0.865512549877167 14.7004346847534\\
0.953228712081909 14.7004346847534\\
1.04971647262573 14.7004346847534\\
1.14971649646759 14.7004346847534\\
1.24971640110016 14.7004346847534\\
1.34971642494202 14.7004346847534\\
1.44971644878387 14.7004346847534\\
1.54971647262573 14.7004346847534\\
1.64971649646759 14.7004346847534\\
1.74971640110016 14.7004346847534\\
1.84971642494202 14.7004346847534\\
1.94971644878387 14.7004346847534\\
2.04971647262573 14.7004346847534\\
2.1497163772583 14.7004346847534\\
2.24971652030945 14.7004346847534\\
2.34971642494202 14.7004346847534\\
2.44971656799316 14.7004346847534\\
2.54971647262573 14.7004346847534\\
2.6497163772583 14.7004346847534\\
2.74971652030945 14.7004346847534\\
2.84971642494202 14.7004346847534\\
2.94971656799316 14.7004346847534\\
3.04971647262573 14.7004346847534\\
3.1497163772583 14.7004346847534\\
3.24971652030945 14.7004346847534\\
3.34971642494202 14.7004346847534\\
3.44971656799316 14.7004346847534\\
3.54971647262573 14.7004346847534\\
3.6497163772583 14.7004346847534\\
3.74971652030945 14.7004346847534\\
3.84971642494202 14.7004346847534\\
3.94971656799316 14.7004346847534\\
4.04971647262573 14.7004346847534\\
4.1497163772583 14.7004346847534\\
4.24971628189087 14.7004346847534\\
4.3497166633606 14.7004346847534\\
4.44971656799316 14.7004346847534\\
4.54971647262573 14.7004346847534\\
4.6497163772583 14.7004346847534\\
4.74971628189087 14.7004346847534\\
4.8497166633606 14.7004346847534\\
4.94971656799316 14.7004346847534\\
5.04971647262573 14.7004346847534\\
5.1497163772583 14.7004346847534\\
5.24971628189087 14.7004346847534\\
5.3497166633606 14.7004346847534\\
5.44971656799316 14.7004346847534\\
5.54971647262573 14.7004346847534\\
5.6497163772583 14.7004346847534\\
5.74971628189087 14.7004346847534\\
5.8497166633606 14.7004346847534\\
5.94971656799316 14.7004346847534\\
6.04971647262573 14.7004346847534\\
6.1497163772583 14.7004337310791\\
6.24971628189087 14.7004327774048\\
6.3497166633606 14.7004327774048\\
6.44971656799316 14.7004337310791\\
6.54971647262573 14.7004356384277\\
6.6497163772583 14.7004356384277\\
6.74971628189087 14.7004356384277\\
6.8497166633606 14.7004346847534\\
6.94971656799316 14.7004346847534\\
7.04971647262573 14.7004346847534\\
7.1497163772583 14.7004346847534\\
7.24971628189087 14.7004346847534\\
7.3497166633606 14.7004346847534\\
7.44971656799316 14.7004346847534\\
7.54971647262573 14.7004346847534\\
7.6497163772583 14.7004346847534\\
7.74971628189087 14.7004346847534\\
7.8497166633606 14.7004346847534\\
7.94971656799316 14.7004346847534\\
8.04971599578857 14.7004346847534\\
8.1497163772583 14.7004346847534\\
8.24971675872803 14.7004346847534\\
8.34971618652344 14.7004346847534\\
8.44971656799316 14.7004346847534\\
8.54971599578857 14.7004346847534\\
8.6497163772583 14.7004346847534\\
8.74971675872803 14.7004346847534\\
8.84971618652344 14.7004346847534\\
8.94971656799316 14.7004346847534\\
9.04971599578857 14.7004346847534\\
9.1497163772583 14.7004346847534\\
9.24971675872803 14.7004346847534\\
9.34971618652344 14.7004346847534\\
9.44971656799316 14.7004346847534\\
9.54971599578857 14.7004346847534\\
9.6497163772583 14.7004346847534\\
9.74971675872803 14.7004346847534\\
9.84971618652344 14.7004346847534\\
9.94971656799316 14.7004346847534\\
10.0497159957886 14.7004346847534\\
10.1497163772583 14.7004346847534\\
10.249716758728 14.7004346847534\\
10.3497161865234 14.7004346847534\\
10.4497165679932 14.7004346847534\\
10.5497159957886 14.7004346847534\\
10.6497163772583 14.7004346847534\\
10.749716758728 14.7004346847534\\
10.8497161865234 14.7004346847534\\
10.9497165679932 14.7004346847534\\
11.0497159957886 14.7004346847534\\
11.1497163772583 14.7004346847534\\
11.249716758728 14.7004346847534\\
11.3497161865234 14.7004346847534\\
11.4497165679932 14.7004346847534\\
11.5497159957886 14.7004346847534\\
11.6497163772583 14.7004346847534\\
11.749716758728 14.7004346847534\\
11.8497161865234 14.7004346847534\\
11.9497165679932 14.7004346847534\\
12.0497159957886 14.7004346847534\\
12.1497163772583 14.7004346847534\\
12.249716758728 14.7004346847534\\
12.347146987915 14.7004346847534\\
12.4250946044922 14.7004346847534\\
12.4874591827393 14.7004356384277\\
12.5560598373413 14.7004346847534\\
12.6315202713013 14.7004337310791\\
12.714527130127 14.7004346847534\\
12.8058338165283 14.7004346847534\\
12.8497476577759 14.7004356384277\\
12.8980522155762 14.7004337310791\\
12.9143037796021 14.7004346847534\\
12.9277105331421 14.7004356384277\\
12.9424571990967 14.7004356384277\\
12.9586782455444 14.7004346847534\\
12.9765224456787 14.7004346847534\\
12.9961500167847 14.7004346847534\\
13.0177412033081 14.7004346847534\\
13.0414915084839 14.7004346847534\\
13.0676164627075 14.7004346847534\\
13.0963535308838 14.7004346847534\\
13.1279649734497 14.7004346847534\\
13.1627368927002 14.7004346847534\\
13.2009868621826 14.7004346847534\\
13.2430610656738 14.7004346847534\\
13.289342880249 14.7004346847534\\
13.3402528762817 14.7004346847534\\
13.3962545394897 14.7004346847534\\
13.4578561782837 14.7004346847534\\
13.5256175994873 14.7004346847534\\
13.600154876709 14.7004346847534\\
13.6821460723877 14.7004337310791\\
13.7723369598389 14.7004346847534\\
13.871545791626 14.7004346847534\\
13.9715461730957 14.7004547119141\\
14.0715456008911 14.7005434036255\\
14.1215467453003 14.7009696960449\\
14.147439956665 14.7011756896973\\
14.1759233474731 14.6994876861572\\
14.1775197982788 14.6936931610107\\
14.1792755126953 14.6947050094604\\
14.1812076568604 14.6988182067871\\
14.1833324432373 14.7013626098633\\
14.1856698989868 14.7022333145142\\
14.1882400512695 14.7015657424927\\
14.191068649292 14.7014808654785\\
14.1941795349121 14.7012119293213\\
14.1976013183594 14.7012357711792\\
14.2013654708862 14.7011013031006\\
14.2055063247681 14.7008972167969\\
14.2100610733032 14.7007341384888\\
14.2150716781616 14.7006177902222\\
14.2205829620361 14.7005176544189\\
14.2266454696655 14.7004356384277\\
14.2333135604858 14.7003498077393\\
14.2406492233276 14.7002763748169\\
14.2487182617188 14.7002534866333\\
14.2575941085815 14.7002859115601\\
14.2673578262329 14.7000999450684\\
14.27809715271 14.7001628875732\\
14.2899112701416 14.7002248764038\\
14.3029069900513 14.7002277374268\\
14.3172016143799 14.7002077102661\\
14.3329257965088 14.7002077102661\\
14.3502225875854 14.7002305984497\\
14.3692483901978 14.7002897262573\\
14.3901777267456 14.7003488540649\\
14.4131994247437 14.700439453125\\
14.4385232925415 14.7003889083862\\
14.4663791656494 14.700421333313\\
14.4970216751099 14.7004365921021\\
14.520975112915 14.7004203796387\\
14.5473232269287 14.7004318237305\\
14.5763063430786 14.7004337310791\\
14.6081876754761 14.700439453125\\
14.6432571411133 14.700439453125\\
14.6818342208862 14.7004404067993\\
14.7242679595947 14.7004356384277\\
14.7709455490112 14.7004356384277\\
14.8222913742065 14.7004346847534\\
14.8712491989136 14.7004375457764\\
14.9185028076172 14.7004365921021\\
14.9638071060181 14.7004356384277\\
15.0066013336182 14.7004356384277\\
15.0464649200439 14.7004346847534\\
15.0903158187866 14.7004346847534\\
15.1385507583618 14.7004346847534\\
15.1916093826294 14.7004346847534\\
15.2499742507935 14.7004346847534\\
15.3141756057739 14.7004346847534\\
15.3847970962524 14.7004346847534\\
15.4624805450439 14.7004346847534\\
15.5479326248169 14.7004346847534\\
15.6419296264648 14.7004337310791\\
15.7419300079346 14.7004346847534\\
15.7919301986694 14.7004337310791\\
15.8469295501709 14.7004346847534\\
15.9074296951294 14.7004337310791\\
15.920783996582 14.7004337310791\\
15.9354724884033 14.7004365921021\\
15.9516305923462 14.7004337310791\\
15.9645795822144 14.7004356384277\\
15.9788236618042 14.7004356384277\\
15.9944925308228 14.7004346847534\\
16.0117282867432 14.7004346847534\\
16.0306873321533 14.7004346847534\\
16.0515422821045 14.7004346847534\\
16.0744819641113 14.7004346847534\\
16.0997161865234 14.7004346847534\\
16.1274738311768 14.7004346847534\\
16.1580085754395 14.7004346847534\\
16.1915950775146 14.7004346847534\\
16.2285404205322 14.7004346847534\\
16.2691822052002 14.7004346847534\\
16.3138847351074 14.7004346847534\\
16.3630599975586 14.7004346847534\\
16.4171524047852 14.7004346847534\\
16.4766540527344 14.7004346847534\\
16.542106628418 14.7004346847534\\
16.6141033172607 14.7004346847534\\
16.6932983398438 14.7004346847534\\
16.7804145812988 14.7004337310791\\
16.87624168396 14.7004346847534\\
16.9762420654297 14.7004337310791\\
17.0762424468994 14.7004461288452\\
17.0962505340576 14.7004585266113\\
17.1021957397461 14.7004566192627\\
17.1087341308594 14.7004537582397\\
17.1159267425537 14.7004537582397\\
17.1238384246826 14.7004566192627\\
17.1325416564941 14.7004594802856\\
17.1421165466309 14.70046043396\\
17.1526470184326 14.7004613876343\\
17.1642322540283 14.7004594802856\\
17.1769752502441 14.7004566192627\\
17.1909923553467 14.7004556655884\\
17.2064094543457 14.7004537582397\\
17.2233715057373 14.7004508972168\\
17.2420272827148 14.7004470825195\\
17.2625503540039 14.7004384994507\\
17.2851238250732 14.7004337310791\\
17.3099555969238 14.7004299163818\\
17.3372707366943 14.7004261016846\\
17.367317199707 14.7004194259644\\
17.400369644165 14.700400352478\\
17.4367256164551 14.7003774642944\\
17.4767169952393 14.7004384994507\\
17.5207099914551 14.7004365921021\\
17.5690994262695 14.7004346847534\\
17.6223297119141 14.7004327774048\\
17.6808815002441 14.7004327774048\\
17.745288848877 14.7004384994507\\
17.8161373138428 14.7004327774048\\
17.8624935150146 14.7004270553589\\
17.9028415679932 14.7004261016846\\
17.9472217559814 14.7004337310791\\
17.9960422515869 14.7004337310791\\
18.0497417449951 14.7004346847534\\
18.108814239502 14.7004346847534\\
18.173791885376 14.7004346847534\\
18.245267868042 14.7004337310791\\
18.3238925933838 14.7004337310791\\
18.41037940979 14.7004318237305\\
18.5055141448975 14.7004346847534\\
18.6055145263672 14.7004327774048\\
18.7055149078369 14.7004346847534\\
18.805513381958 14.7004346847534\\
18.9055137634277 14.7004356384277\\
19.0055141448975 14.7004356384277\\
19.1055145263672 14.7004346847534\\
19.2055149078369 14.7004346847534\\
19.305513381958 14.7004346847534\\
19.4055137634277 14.7004346847534\\
19.5055141448975 14.7004346847534\\
19.6055145263672 14.7004346847534\\
19.7055149078369 14.7004346847534\\
19.805513381958 14.7004346847534\\
19.9055137634277 14.7004346847534\\
20.0055141448975 14.7004346847534\\
20.1055145263672 14.7004346847534\\
20.2055149078369 14.7004346847534\\
20.305513381958 14.7004346847534\\
20.4055137634277 14.7004346847534\\
20.5055141448975 14.7004346847534\\
20.6055145263672 14.7004346847534\\
20.7055149078369 14.7004346847534\\
20.805513381958 14.7004346847534\\
20.9055137634277 14.7004346847534\\
21.0055141448975 14.7004346847534\\
21.1055145263672 14.7004346847534\\
21.2055149078369 14.7004346847534\\
21.305513381958 14.7004346847534\\
21.4055137634277 14.7004346847534\\
21.5055141448975 14.7004346847534\\
21.6055145263672 14.7004346847534\\
21.7055149078369 14.7004346847534\\
21.805513381958 14.7004346847534\\
21.9055137634277 14.7004346847534\\
22.0055141448975 14.7004346847534\\
22.1055145263672 14.7004346847534\\
22.2055149078369 14.7004346847534\\
22.305513381958 14.7004346847534\\
22.4055137634277 14.7004346847534\\
22.5055141448975 14.7004346847534\\
22.6055145263672 14.7004346847534\\
22.7055149078369 14.7004346847534\\
22.805513381958 14.7004346847534\\
22.9055137634277 14.7004346847534\\
23.0055141448975 14.7004346847534\\
23.1055145263672 14.7004346847534\\
23.2055149078369 14.7004346847534\\
23.305513381958 14.7004346847534\\
23.4055137634277 14.7004346847534\\
23.5055141448975 14.7004346847534\\
23.6055145263672 14.7004346847534\\
23.7055149078369 14.7004346847534\\
23.805513381958 14.7004346847534\\
23.9055137634277 14.7004346847534\\
24.0055141448975 14.7004346847534\\
24.1055145263672 14.7004346847534\\
24.2055149078369 14.7004346847534\\
24.305513381958 14.7004346847534\\
24.4055137634277 14.7004346847534\\
24.5055141448975 14.7004346847534\\
24.6055145263672 14.7004346847534\\
24.7055149078369 14.7004346847534\\
24.789867401123 14.7004346847534\\
24.8573379516602 14.7004356384277\\
24.9315567016602 14.7004346847534\\
25.0131969451904 14.7004337310791\\
25.1030006408691 14.7004337310791\\
25.1443557739258 14.7004356384277\\
25.189847946167 14.7004308700562\\
25.2099170684814 14.7004413604736\\
25.2266445159912 14.7004365921021\\
25.245044708252 14.7004356384277\\
25.2652854919434 14.7004346847534\\
25.2875499725342 14.7004346847534\\
25.3120403289795 14.7004346847534\\
25.3389797210693 14.7004346847534\\
25.3686122894287 14.7004346847534\\
25.4012107849121 14.7004346847534\\
25.4370670318604 14.7004346847534\\
25.4765090942383 14.7004346847534\\
25.5198955535889 14.7004346847534\\
25.5676212310791 14.7004346847534\\
25.6201171875 14.7004346847534\\
25.6778659820557 14.7004346847534\\
25.7413883209229 14.7004346847534\\
25.8112621307373 14.7004346847534\\
25.8881244659424 14.7004346847534\\
25.9726715087891 14.7004337310791\\
26.065673828125 14.7004346847534\\
26.1656742095947 14.7004337310791\\
26.2656745910645 14.7004480361938\\
26.3656749725342 14.7005767822266\\
26.409065246582 14.7014284133911\\
26.4240894317627 14.7017812728882\\
26.4268970489502 14.683367729187\\
26.4299831390381 14.6951837539673\\
26.4333801269531 14.7002420425415\\
26.4371166229248 14.7024335861206\\
26.4412250518799 14.7016401290894\\
26.4457454681396 14.7020626068115\\
26.4507179260254 14.7017259597778\\
26.4561862945557 14.7011966705322\\
26.4622020721436 14.7009696960449\\
26.4688205718994 14.700758934021\\
26.4761009216309 14.7005376815796\\
26.4841079711914 14.7007627487183\\
26.4929161071777 14.6998863220215\\
26.5026054382324 14.6999998092651\\
26.5132637023926 14.6994352340698\\
26.5249881744385 14.6988401412964\\
26.5378837585449 14.6997184753418\\
26.5520706176758 14.7002658843994\\
26.5676746368408 14.7004146575928\\
26.5848388671875 14.7004632949829\\
26.6037216186523 14.7014112472534\\
26.624490737915 14.6997003555298\\
26.6473369598389 14.7001371383667\\
26.6724681854248 14.700532913208\\
26.7001132965088 14.7004356384277\\
26.7305221557617 14.7004146575928\\
26.7594413757324 14.7004117965698\\
26.7881412506104 14.7004222869873\\
26.8180274963379 14.7004308700562\\
26.848762512207 14.7004337310791\\
26.8805618286133 14.7004356384277\\
26.9128971099854 14.7004346847534\\
26.9456844329834 14.7004356384277\\
26.97873878479 14.7004356384277\\
27.0120601654053 14.7004346847534\\
27.0457935333252 14.7004356384277\\
27.0821418762207 14.7004356384277\\
27.1221237182617 14.7004365921021\\
27.1524639129639 14.7004356384277\\
27.1838111877441 14.7004356384277\\
27.2182941436768 14.7004356384277\\
27.2558403015137 14.7004346847534\\
27.2955665588379 14.7004346847534\\
27.3392658233643 14.7004346847534\\
27.3873329162598 14.7004346847534\\
27.4402103424072 14.7004346847534\\
27.4983730316162 14.7004346847534\\
27.5623531341553 14.7004356384277\\
27.6327304840088 14.7004346847534\\
27.7101459503174 14.7004346847534\\
27.7953014373779 14.7004346847534\\
27.8889751434326 14.7004337310791\\
27.9889755249023 14.7004346847534\\
28.0889739990234 14.7004337310791\\
28.1889743804932 14.7004356384277\\
28.2889747619629 14.7004337310791\\
28.2927665710449 14.7004299163818\\
28.2969398498535 14.7004423141479\\
28.3015289306641 14.7003860473633\\
28.304780960083 14.7004842758179\\
28.307991027832 14.7004518508911\\
28.3115215301514 14.7004356384277\\
28.3154048919678 14.7004346847534\\
28.3196773529053 14.7004356384277\\
28.3243770599365 14.7004356384277\\
28.3295459747314 14.7004356384277\\
28.3352317810059 14.7004356384277\\
28.3414859771729 14.7004356384277\\
28.3483657836914 14.7004356384277\\
28.3559341430664 14.7004356384277\\
28.3642578125 14.7004356384277\\
28.3734169006348 14.7004346847534\\
28.3834896087646 14.7004346847534\\
28.3945693969727 14.7004346847534\\
28.4067573547363 14.7004346847534\\
28.4201641082764 14.7004346847534\\
28.4349136352539 14.7004346847534\\
28.451135635376 14.7004346847534\\
28.4689807891846 14.7004346847534\\
28.4886093139648 14.7004346847534\\
28.5102024078369 14.7004346847534\\
28.5339546203613 14.7004346847534\\
28.5600814819336 14.7004346847534\\
28.5888214111328 14.7004346847534\\
28.620433807373 14.7004346847534\\
28.6552085876465 14.7004346847534\\
28.6934604644775 14.7004346847534\\
28.735538482666 14.7004346847534\\
28.7818241119385 14.7004346847534\\
28.8327369689941 14.7004346847534\\
28.8887424468994 14.7004346847534\\
28.9503479003906 14.7004346847534\\
29.0181159973145 14.7004346847534\\
29.092658996582 14.7004346847534\\
29.1746559143066 14.7004346847534\\
29.2648525238037 14.7004337310791\\
29.3640689849854 14.7004346847534\\
29.4640693664551 14.7004337310791\\
29.4847068786621 14.7006969451904\\
29.4888343811035 14.7006454467773\\
29.4933738708496 14.700665473938\\
29.4983692169189 14.7006816864014\\
29.5038623809814 14.700722694397\\
29.5099067687988 14.7007732391357\\
29.5165538787842 14.7008295059204\\
29.5230560302734 14.7008943557739\\
29.5302066802979 14.7009553909302\\
29.5380744934082 14.700982093811\\
29.5467281341553 14.7009220123291\\
29.556245803833 14.6994256973267\\
29.5667171478271 14.6982154846191\\
29.5782356262207 14.7016353607178\\
29.5909042358398 14.7009706497192\\
29.6048412322998 14.7007942199707\\
29.6201705932617 14.7012987136841\\
29.637035369873 14.7006692886353\\
29.6555843353271 14.700532913208\\
29.675989151001 14.7004976272583\\
29.6984329223633 14.7003059387207\\
29.7114696502686 14.7001695632935\\
29.7258071899414 14.7001609802246\\
29.7415809631348 14.700156211853\\
29.7589302062988 14.700216293335\\
29.7780151367188 14.699649810791\\
29.7990093231201 14.7001132965088\\
29.8221015930176 14.7005014419556\\
29.8475036621094 14.7003917694092\\
29.8754444122314 14.7004499435425\\
29.9061813354492 14.7004280090332\\
29.9399909973145 14.7004499435425\\
29.9771823883057 14.700439453125\\
30.0180931091309 14.7004299163818\\
30.0630931854248 14.700439453125\\
30.1125946044922 14.7004442214966\\
30.1670475006104 14.7004299163818\\
30.2269439697266 14.7004451751709\\
30.2706527709961 14.700475692749\\
30.318733215332 14.7004308700562\\
30.3593273162842 14.7004375457764\\
30.3947944641113 14.7004251480103\\
30.4330863952637 14.7004222869873\\
30.4652576446533 14.7004251480103\\
30.5006484985352 14.7004289627075\\
30.5330505371094 14.7004308700562\\
30.5677318572998 14.7004327774048\\
30.6004276275635 14.7004337310791\\
30.6349658966064 14.7004337310791\\
30.6682662963867 14.7004337310791\\
30.7048969268799 14.7004327774048\\
30.7451915740967 14.7004346847534\\
30.789514541626 14.7004337310791\\
30.8382682800293 14.7004337310791\\
30.8918991088867 14.7004327774048\\
30.9508934020996 14.7004327774048\\
31.0157871246338 14.7004337310791\\
31.0871696472168 14.7004337310791\\
31.1656913757324 14.7004365921021\\
31.2520637512207 14.7004346847534\\
31.347074508667 14.7004346847534\\
31.4470748901367 14.7004356384277\\
31.5470752716064 14.7004346847534\\
31.6470737457275 14.7004346847534\\
31.7470741271973 14.7004346847534\\
31.847074508667 14.7004346847534\\
31.9470748901367 14.7004346847534\\
32.0470733642578 14.7004346847534\\
32.1470756530762 14.7004346847534\\
32.2470741271973 14.7004346847534\\
32.3470764160156 14.7004346847534\\
32.4470748901367 14.7004346847534\\
32.5470733642578 14.7004346847534\\
32.6470756530762 14.7004346847534\\
32.7470741271973 14.7004346847534\\
32.8470764160156 14.7004346847534\\
32.9470748901367 14.7004346847534\\
33.0470733642578 14.7004346847534\\
33.1470756530762 14.7004346847534\\
33.2470741271973 14.7004346847534\\
33.3470764160156 14.7004346847534\\
33.4470748901367 14.7004346847534\\
33.5470733642578 14.7004346847534\\
33.6470756530762 14.7004346847534\\
33.7470741271973 14.7004346847534\\
33.8470764160156 14.7004346847534\\
33.9470748901367 14.7004346847534\\
34.0470733642578 14.7004346847534\\
34.1470756530762 14.7004346847534\\
34.2470741271973 14.7004346847534\\
34.3470764160156 14.7004346847534\\
34.4470748901367 14.7004346847534\\
34.5470733642578 14.7004346847534\\
34.6470756530762 14.7004346847534\\
34.7470741271973 14.7004346847534\\
34.8470764160156 14.7004346847534\\
34.9470748901367 14.7004346847534\\
35.0470733642578 14.7004346847534\\
35.1470756530762 14.7004346847534\\
35.2470741271973 14.7004346847534\\
35.3470764160156 14.7004346847534\\
35.4470748901367 14.7004346847534\\
35.5470733642578 14.7004346847534\\
35.6470756530762 14.7004346847534\\
35.7470741271973 14.7004346847534\\
35.8470764160156 14.7004346847534\\
35.9470748901367 14.7004346847534\\
36.0470733642578 14.7004346847534\\
36.1470756530762 14.7004346847534\\
36.2470741271973 14.7004346847534\\
36.3470764160156 14.7004346847534\\
36.4470748901367 14.7004346847534\\
36.5470733642578 14.7004346847534\\
36.6470756530762 14.7004346847534\\
36.7470741271973 14.7004346847534\\
36.8470764160156 14.7004346847534\\
36.9470748901367 14.7004346847534\\
37.0470733642578 14.7004346847534\\
37.1470756530762 14.7004346847534\\
37.2293395996094 14.7004346847534\\
37.2951393127441 14.7004356384277\\
37.3675231933594 14.7004337310791\\
37.4471435546875 14.7004346847534\\
37.534725189209 14.7004346847534\\
37.6310653686523 14.7004337310791\\
37.6772041320801 14.7004356384277\\
37.7279586791992 14.7004289627075\\
37.7453880310059 14.7004480361938\\
37.7602653503418 14.7004365921021\\
37.7766304016113 14.7004356384277\\
37.7946319580078 14.7004356384277\\
37.8144378662109 14.7004346847534\\
37.8362197875977 14.7004346847534\\
37.860179901123 14.7004346847534\\
37.8865356445313 14.7004346847534\\
37.9155311584473 14.7004346847534\\
37.9474220275879 14.7004346847534\\
37.9825019836426 14.7004346847534\\
38.0210914611816 14.7004346847534\\
38.0635414123535 14.7004346847534\\
38.1102333068848 14.7004346847534\\
38.1615982055664 14.7004346847534\\
38.2180976867676 14.7004346847534\\
38.2802429199219 14.7004346847534\\
38.3486099243164 14.7004346847534\\
38.4238090515137 14.7004346847534\\
38.5065269470215 14.7004337310791\\
38.5975189208984 14.7004346847534\\
38.6975212097168 14.7004337310791\\
38.7975196838379 14.7004508972168\\
38.8975219726563 14.7005262374878\\
38.9475212097168 14.7011890411377\\
38.9856338500977 14.7012968063354\\
39.0002326965332 14.6960802078247\\
39.0162887573242 14.7003889083862\\
39.0339508056641 14.7009620666504\\
39.0533828735352 14.7008686065674\\
39.0747528076172 14.7001876831055\\
39.0982627868652 14.7000579833984\\
39.1241226196289 14.7002983093262\\
39.1525688171387 14.7002506256104\\
39.1838569641113 14.7004060745239\\
39.2182769775391 14.700457572937\\
39.2382202148438 14.7003240585327\\
39.2601547241211 14.7003412246704\\
39.2842864990234 14.7004013061523\\
39.3108291625977 14.7004098892212\\
39.3400230407715 14.700421333313\\
39.3721389770508 14.7004280090332\\
39.4047470092773 14.7004318237305\\
39.4365348815918 14.7004375457764\\
39.4690742492676 14.700439453125\\
39.5013847351074 14.7004404067993\\
39.5341873168945 14.7004356384277\\
39.5668563842773 14.7004365921021\\
39.5998191833496 14.7004356384277\\
39.632698059082 14.7004356384277\\
39.6657829284668 14.7004356384277\\
39.6988639831543 14.7004365921021\\
39.7321624755859 14.7004365921021\\
39.7687873840332 14.7004356384277\\
39.8090744018555 14.7004356384277\\
39.8533935546875 14.7004356384277\\
39.8968162536621 14.7004346847534\\
39.939453125 14.7004346847534\\
39.9863548278809 14.7004346847534\\
40.0379447937012 14.7004346847534\\
40.0946960449219 14.7004346847534\\
40.1571197509766 14.7004346847534\\
40.2257843017578 14.7004346847534\\
40.3013191223145 14.7004346847534\\
40.384407043457 14.7004346847534\\
40.4758033752441 14.7004337310791\\
40.5758018493652 14.7004346847534\\
40.6758041381836 14.7004337310791\\
40.7758026123047 14.7004346847534\\
40.875804901123 14.7004337310791\\
40.8792419433594 14.7004241943359\\
40.8830223083496 14.7004537582397\\
40.8871841430664 14.7003593444824\\
40.8903007507324 14.7005033493042\\
40.8937301635742 14.7004556655884\\
40.8975028991699 14.7004356384277\\
40.9016532897949 14.7004346847534\\
40.9062156677246 14.7004356384277\\
40.9112396240234 14.7004356384277\\
40.9167633056641 14.7004356384277\\
40.9228363037109 14.7004356384277\\
40.9295196533203 14.7004356384277\\
40.9368743896484 14.7004356384277\\
40.9449615478516 14.7004356384277\\
40.953857421875 14.7004346847534\\
40.9636421203613 14.7004346847534\\
40.9744033813477 14.7004346847534\\
40.9862442016602 14.7004346847534\\
40.999267578125 14.7004346847534\\
41.0135955810547 14.7004346847534\\
41.029354095459 14.7004346847534\\
41.046688079834 14.7004346847534\\
41.0657577514648 14.7004346847534\\
41.0867347717285 14.7004346847534\\
41.109806060791 14.7004346847534\\
41.1351852416992 14.7004346847534\\
41.1631050109863 14.7004346847534\\
41.1938133239746 14.7004346847534\\
41.227596282959 14.7004346847534\\
41.2647552490234 14.7004346847534\\
41.3056297302246 14.7004346847534\\
41.3505935668945 14.7004346847534\\
41.4000511169434 14.7004346847534\\
41.4544563293457 14.7004346847534\\
41.5143013000488 14.7004346847534\\
41.5801315307617 14.7004346847534\\
41.6525421142578 14.7004346847534\\
41.7321968078613 14.7004346847534\\
41.8198165893555 14.7004337310791\\
41.9161987304688 14.7004346847534\\
42.0161972045898 14.7004337310791\\
42.1161994934082 14.7004671096802\\
42.1361999511719 14.7006092071533\\
42.1442070007324 14.7007675170898\\
42.1520309448242 14.7008237838745\\
42.1606330871582 14.700855255127\\
42.1689529418945 14.7008533477783\\
42.1781005859375 14.70068359375\\
42.1881637573242 14.6987838745117\\
42.1992378234863 14.700231552124\\
42.2114143371582 14.7009496688843\\
42.2248077392578 14.7008018493652\\
42.2395439147949 14.700909614563\\
42.2557525634766 14.7008361816406\\
42.2735824584961 14.7006721496582\\
42.2931938171387 14.7005052566528\\
42.3147659301758 14.7003908157349\\
42.3384971618652 14.7005567550659\\
42.3646011352539 14.700098991394\\
42.3847808837891 14.7000894546509\\
42.4069786071777 14.6997203826904\\
42.431396484375 14.7003383636475\\
42.4582557678223 14.7003240585327\\
42.4878005981445 14.7004423141479\\
42.5203018188477 14.7004308700562\\
42.5560493469238 14.7004470825195\\
42.5953750610352 14.7004423141479\\
42.6386299133301 14.7004508972168\\
42.6862144470215 14.7004175186157\\
42.7385559082031 14.7004642486572\\
42.7959022521973 14.700400352478\\
42.8589859008789 14.7004652023315\\
42.9028167724609 14.7004852294922\\
42.9510307312012 14.7004060745239\\
42.9925117492676 14.7004480361938\\
43.0297622680664 14.7004346847534\\
43.0707359313965 14.7004270553589\\
43.1158065795898 14.7004261016846\\
43.1579742431641 14.7004241943359\\
43.203685760498 14.7004365921021\\
43.2497863769531 14.7004337310791\\
43.3005027770996 14.7004346847534\\
43.3562850952148 14.7004346847534\\
43.4176483154297 14.7004346847534\\
43.4851493835449 14.7004346847534\\
43.559398651123 14.7004356384277\\
43.6410713195801 14.7004346847534\\
43.7309150695801 14.7004356384277\\
43.8297386169434 14.7004356384277\\
43.9297409057617 14.7004346847534\\
44.0297393798828 14.7004346847534\\
44.1297416687012 14.7004346847534\\
44.2297401428223 14.7004346847534\\
44.2499618530273 14.7004346847534\\
44.2722015380859 14.7004346847534\\
44.2966690063477 14.7004346847534\\
44.3235816955566 14.7004346847534\\
44.3531875610352 14.7004346847534\\
44.3857498168945 14.7004346847534\\
44.421573638916 14.7004346847534\\
44.4609756469727 14.7004346847534\\
44.5043182373047 14.7004346847534\\
44.5519981384277 14.7004346847534\\
44.6044425964355 14.7004346847534\\
44.6621360778809 14.7004346847534\\
44.7255935668945 14.7004346847534\\
44.7953987121582 14.7004346847534\\
44.872184753418 14.7004346847534\\
44.9566497802734 14.7004346847534\\
45.0495643615723 14.7004346847534\\
45.1495628356934 14.7004346847534\\
45.2495613098145 14.7004346847534\\
45.3495635986328 14.7004346847534\\
45.4495620727539 14.7004346847534\\
45.5495643615723 14.7004346847534\\
45.6495628356934 14.7004346847534\\
45.7495613098145 14.7004508972168\\
45.7695655822754 14.7004632949829\\
45.7816352844238 14.700457572937\\
45.7949142456055 14.70046043396\\
45.8095207214355 14.7004518508911\\
45.8255882263184 14.7004432678223\\
45.8432655334473 14.700439453125\\
45.8627052307129 14.7004356384277\\
45.884090423584 14.7004299163818\\
45.9076156616211 14.7004346847534\\
45.9334945678711 14.7004461288452\\
45.9619598388672 14.7004299163818\\
45.9932708740234 14.700421333313\\
46.0277099609375 14.7004108428955\\
46.0655975341797 14.700421333313\\
46.1072731018066 14.700421333313\\
46.1531181335449 14.7004222869873\\
46.2035446166992 14.7004232406616\\
46.2590141296387 14.7004251480103\\
46.3200302124023 14.7004308700562\\
46.387149810791 14.7004346847534\\
46.4504814147949 14.7004356384277\\
46.5130958557129 14.7004356384277\\
46.581974029541 14.7004356384277\\
46.6577415466309 14.7004356384277\\
46.7410850524902 14.7004356384277\\
46.8327598571777 14.7004356384277\\
46.9327621459961 14.7004356384277\\
47.0327606201172 14.7004356384277\\
47.1327629089355 14.7004356384277\\
47.2327613830566 14.7004356384277\\
47.3327598571777 14.7004346847534\\
47.4327621459961 14.7004346847534\\
47.5327606201172 14.7004346847534\\
47.6327629089355 14.7004356384277\\
47.7327613830566 14.7004346847534\\
47.8327598571777 14.7004346847534\\
47.9327621459961 14.7004346847534\\
48.0327606201172 14.7004346847534\\
48.1327629089355 14.7004346847534\\
48.2327613830566 14.7004346847534\\
48.3327598571777 14.7004346847534\\
48.4327621459961 14.7004346847534\\
48.5327606201172 14.7004346847534\\
48.6327629089355 14.7004346847534\\
48.7327613830566 14.7004346847534\\
48.8327598571777 14.7004346847534\\
48.9327621459961 14.7004346847534\\
49.0327606201172 14.7004346847534\\
49.1327629089355 14.7004346847534\\
49.2327613830566 14.7004346847534\\
49.3327598571777 14.7004346847534\\
49.4327621459961 14.7004337310791\\
49.5327606201172 14.7004337310791\\
49.6257057189941 14.7004337310791\\
49.6982688903809 14.7004346847534\\
49.7780876159668 14.7004337310791\\
49.8658905029297 14.7004346847534\\
49.962474822998 14.7004356384277\\
50.0624732971191 14.7004356384277\\
50.1624755859375 14.7004356384277\\
50.2624740600586 14.7004356384277\\
50.3624725341797 14.7004356384277\\
50.462474822998 14.7004356384277\\
50.5624732971191 14.7004356384277\\
50.6624755859375 14.7004356384277\\
50.7624740600586 14.7004356384277\\
50.8624725341797 14.7004356384277\\
50.962474822998 14.7004356384277\\
51.0624732971191 14.7004356384277\\
51.1624755859375 14.7004356384277\\
51.2624740600586 14.7004356384277\\
51.3624725341797 14.7004356384277\\
51.462474822998 14.7004356384277\\
51.5624732971191 14.7004356384277\\
51.6624755859375 14.7004356384277\\
51.7624740600586 14.7004356384277\\
51.8624725341797 14.7004356384277\\
51.962474822998 14.7004356384277\\
52.0624732971191 14.7004356384277\\
52.1624755859375 14.7004356384277\\
52.2624740600586 14.7004356384277\\
52.3624725341797 14.7004356384277\\
52.462474822998 14.7004356384277\\
52.5624732971191 14.7004356384277\\
52.6624755859375 14.7004356384277\\
52.7624740600586 14.7004356384277\\
52.8624725341797 14.7004356384277\\
52.962474822998 14.7004356384277\\
53.0624732971191 14.7004356384277\\
53.1624755859375 14.7004356384277\\
53.2624740600586 14.7004356384277\\
53.3624725341797 14.7004356384277\\
53.462474822998 14.7004356384277\\
53.5624732971191 14.7004356384277\\
53.6624755859375 14.7004356384277\\
53.7624740600586 14.7004356384277\\
53.8624725341797 14.7004356384277\\
53.962474822998 14.7004356384277\\
54.0624732971191 14.7004356384277\\
54.1624755859375 14.7004356384277\\
54.2624740600586 14.7004356384277\\
54.3624725341797 14.7004356384277\\
54.462474822998 14.7004356384277\\
54.5624732971191 14.7004356384277\\
54.6624755859375 14.7004356384277\\
54.7624740600586 14.7004356384277\\
54.8624725341797 14.7004356384277\\
54.962474822998 14.7004356384277\\
55.0624732971191 14.7004365921021\\
55.1624755859375 14.700439453125\\
55.2624740600586 14.7004489898682\\
55.3624725341797 14.7004957199097\\
55.462474822998 14.7007360458374\\
55.5624732971191 14.7019739151001\\
55.6624755859375 14.7070140838623\\
55.7624740600586 14.7191886901855\\
55.8624725341797 14.7304124832153\\
55.962474822998 14.7327260971069\\
56.0624732971191 14.7315616607666\\
56.1624755859375 14.7333087921143\\
56.2624740600586 14.7320785522461\\
56.3624725341797 14.733660697937\\
56.462474822998 14.7334928512573\\
56.5624732971191 14.7344827651978\\
56.6624755859375 14.7347106933594\\
56.7624740600586 14.7354402542114\\
56.8624725341797 14.73583984375\\
56.962474822998 14.7364568710327\\
57.0624732971191 14.736930847168\\
57.1624755859375 14.7374992370605\\
57.2624740600586 14.7380065917969\\
57.3624725341797 14.7385540008545\\
57.462474822998 14.7390823364258\\
57.5624732971191 14.7396211624146\\
57.6624755859375 14.7401504516602\\
57.7624740600586 14.7406911849976\\
57.8624725341797 14.7412242889404\\
57.962474822998 14.7417631149292\\
58.0624732971191 14.7423000335693\\
58.1624755859375 14.7428379058838\\
58.2624740600586 14.7433757781982\\
58.3624725341797 14.7439794540405\\
58.462474822998 14.7448091506958\\
58.5624732971191 14.7454013824463\\
58.6624755859375 14.7460565567017\\
58.7624740600586 14.7466897964478\\
58.8624725341797 14.7473287582397\\
58.962474822998 14.7479648590088\\
59.0624732971191 14.7486009597778\\
59.1624755859375 14.7492361068726\\
59.2624740600586 14.749870300293\\
59.3624725341797 14.7505054473877\\
59.462474822998 14.7511396408081\\
59.5624732971191 14.7517738342285\\
59.6624755859375 14.7524089813232\\
59.7624740600586 14.753044128418\\
59.8624725341797 14.7536773681641\\
59.9312362670898 14.7540311813354\\
60 14.7544212341309\\
60.0756378173828 14.7547960281372\\
60.1588439941406 14.7552108764648\\
60.2503662109375 14.7556676864624\\
60.3503646850586 14.7561683654785\\
60.450366973877 14.75670337677\\
60.550365447998 14.7572507858276\\
60.6503677368164 14.7577905654907\\
60.7503662109375 14.7583303451538\\
60.8503646850586 14.7588691711426\\
60.950366973877 14.7594108581543\\
61.050365447998 14.7599859237671\\
61.1503677368164 14.7605819702148\\
61.2503662109375 14.7611665725708\\
61.3503646850586 14.7617588043213\\
61.450366973877 14.7623710632324\\
61.550365447998 14.7630138397217\\
61.6503677368164 14.7637071609497\\
61.7503662109375 14.764853477478\\
61.8503646850586 14.7724523544312\\
61.9325637817383 14.7760171890259\\
61.998291015625 14.7698783874512\\
62.0705947875977 14.768256187439\\
62.1501274108887 14.768669128418\\
62.2376098632813 14.7691583633423\\
62.3338470458984 14.7698764801025\\
62.4338455200195 14.7708015441895\\
62.5338439941406 14.7718257904053\\
62.633846282959 14.7728662490845\\
62.7338447570801 14.77392578125\\
62.8338470458984 14.774998664856\\
62.9338455200195 14.776083946228\\
63.0338439941406 14.7771739959717\\
63.133846282959 14.7782716751099\\
63.2338447570801 14.7793760299683\\
63.3338470458984 14.7804841995239\\
63.4338455200195 14.7815952301025\\
63.5338439941406 14.7827091217041\\
63.633846282959 14.78382396698\\
63.7338447570801 14.7849378585815\\
63.8338470458984 14.7860517501831\\
63.9338455200195 14.787166595459\\
64.0338439941406 14.7882804870605\\
64.1338424682617 14.7893953323364\\
64.2338485717773 14.790509223938\\
64.3338470458984 14.7916240692139\\
64.4338455200195 14.7927379608154\\
64.5338439941406 14.7938528060913\\
64.6338424682617 14.7949666976929\\
64.7338485717773 14.7960815429688\\
64.8338470458984 14.7971963882446\\
64.9338455200195 14.7983112335205\\
65.0338439941406 14.7994260787964\\
65.1338424682617 14.8005409240723\\
65.2338485717773 14.8016557693481\\
65.3338470458984 14.802770614624\\
65.4338455200195 14.8038854598999\\
65.5338439941406 14.8050003051758\\
65.6338424682617 14.8061151504517\\
65.7338485717773 14.8072309494019\\
65.8338470458984 14.8083457946777\\
65.9338455200195 14.8094615936279\\
66.0338439941406 14.8105773925781\\
66.1338424682617 14.811692237854\\
66.2338485717773 14.8128080368042\\
66.3338470458984 14.8139238357544\\
66.4338455200195 14.8150396347046\\
66.5338439941406 14.8161563873291\\
66.6338424682617 14.8172721862793\\
66.7338485717773 14.8183879852295\\
66.8338470458984 14.819504737854\\
66.9338455200195 14.8206205368042\\
67.0338439941406 14.8217372894287\\
67.1338424682617 14.8228540420532\\
67.2338485717773 14.8239707946777\\
67.3338470458984 14.8250875473022\\
67.4338455200195 14.8262062072754\\
67.5338439941406 14.8273286819458\\
67.6338424682617 14.8284740447998\\
67.7338485717773 14.829740524292\\
67.8338470458984 14.8316411972046\\
67.9338455200195 14.8358678817749\\
68.0338439941406 14.8432817459106\\
68.1338424682617 14.8458261489868\\
68.2338485717773 14.8405570983887\\
68.3338470458984 14.8375368118286\\
68.4338455200195 14.8398132324219\\
68.5338439941406 14.8382863998413\\
68.6338424682617 14.8408555984497\\
68.7338485717773 14.8409996032715\\
68.8338470458984 14.8427648544312\\
68.9338455200195 14.8434619903564\\
69.0338439941406 14.8448648452759\\
69.1338424682617 14.8458023071289\\
69.2338485717773 14.8470468521118\\
69.3338470458984 14.8480892181396\\
69.4338455200195 14.8492660522461\\
69.5338439941406 14.8503532409668\\
69.6338424682617 14.8515005111694\\
69.7338485717773 14.8526077270508\\
69.8338470458984 14.853741645813\\
69.9338455200195 14.8548574447632\\
70.0338439941406 14.8559856414795\\
70.1338424682617 14.8571062088013\\
70.2338485717773 14.8582324981689\\
70.3338470458984 14.859354019165\\
70.4338455200195 14.8604793548584\\
70.5338439941406 14.8616018295288\\
70.6338424682617 14.8628034591675\\
70.7338485717773 14.8641805648804\\
70.8338470458984 14.8652486801147\\
70.9338455200195 14.8663930892944\\
71.0338439941406 14.8675107955933\\
71.1338424682617 14.8686389923096\\
71.2338485717773 14.8697633743286\\
71.3338470458984 14.8708896636963\\
71.4338455200195 14.872015953064\\
71.5338439941406 14.8731422424316\\
71.6338424682617 14.8742694854736\\
71.7338485717773 14.8753967285156\\
71.8338470458984 14.8765239715576\\
71.9338455200195 14.8776531219482\\
72.0338439941406 14.8787822723389\\
72.1338424682617 14.8799057006836\\
72.2338485717773 14.8807687759399\\
72.3338470458984 14.8818664550781\\
72.4338455200195 14.8829870223999\\
72.5338439941406 14.8841037750244\\
72.6338424682617 14.8852214813232\\
72.7338485717773 14.8863382339478\\
72.8338470458984 14.8874549865723\\
72.9338455200195 14.8885717391968\\
73.0338439941406 14.889687538147\\
73.1338424682617 14.8908033370972\\
73.2338485717773 14.8919334411621\\
73.3338470458984 14.8931112289429\\
73.4338455200195 14.8942918777466\\
73.5338439941406 14.8954401016235\\
73.6338424682617 14.8965911865234\\
73.7338485717773 14.8977575302124\\
73.8338470458984 14.8989477157593\\
73.9338455200195 14.9001760482788\\
74.0338439941406 14.9030332565308\\
74.1323699951172 14.9098587036133\\
74.2112121582031 14.9130144119263\\
74.2755126953125 14.9045715332031\\
74.3462524414063 14.9048681259155\\
74.4240570068359 14.9054508209229\\
74.509651184082 14.9060888290405\\
74.6037979125977 14.9069204330444\\
74.7037963867188 14.9079294204712\\
74.8037948608398 14.9090356826782\\
74.9019012451172 14.9101371765137\\
75 14.9112234115601\\
};
\addlegendentry{18 "};

\addplot [
color=green,
dash pattern=on 1pt off 3pt on 3pt off 3pt
]
table[row sep=crcr]{
0 14.7003479003906\\
9.99999971718069e-10 14.7003479003906\\
2.49999998480632e-09 14.7003479003906\\
4.74999994892755e-09 14.7003479003906\\
8.1250002281763e-09 14.7003479003906\\
1.31874999809156e-08 14.7003479003906\\
2.0781250498203e-08 14.7003479003906\\
3.21718758300449e-08 14.7003479003906\\
4.92578138278077e-08 14.7003479003906\\
7.48867208244519e-08 14.7003479003906\\
1.13330081319418e-07 14.7003479003906\\
1.70995122061868e-07 14.7003479003906\\
2.57492672517401e-07 14.7003479003906\\
3.8723902662241e-07 14.7003479003906\\
5.81858500936505e-07 14.7003479003906\\
8.73787769251066e-07 14.7003479003906\\
1.31168167172291e-06 14.7003479003906\\
1.9685226106958e-06 14.7003479003906\\
2.95378367809462e-06 14.7003479003906\\
4.43167573394021e-06 14.7003479003906\\
6.64851359033491e-06 14.7003479003906\\
9.97376992017962e-06 14.7003479003906\\
1.49616553244414e-05 14.7003479003906\\
2.24434825213393e-05 14.7003479003906\\
3.36662233166862e-05 14.7003479003906\\
5.05003372381907e-05 14.7003479003906\\
7.57515081204474e-05 14.7003479003906\\
0.000113628258986864 14.7003479003906\\
0.000170443381648511 14.7003479003906\\
0.000255666091106832 14.7003488540649\\
0.000383500126190484 14.7003498077393\\
0.000575251178815961 14.7003507614136\\
0.000862877757754177 14.7003536224365\\
0.00129431765526533 14.7003536224365\\
0.00194147753063589 14.7003507614136\\
0.00291221728548408 14.7003469467163\\
0.00436832662671804 14.7003450393677\\
0.00597004732117057 14.700345993042\\
0.00773194013163447 14.7003479003906\\
0.009670021943748 14.7003479003906\\
0.0118019115179777 14.7003479003906\\
0.0141469910740852 14.7003479003906\\
0.0167265776544809 14.7003479003906\\
0.0195641238242388 14.7003479003906\\
0.0226854234933853 14.7003479003906\\
0.0261188540607691 14.7003479003906\\
0.0298956278711557 14.7003479003906\\
0.0340500771999359 14.7003479003906\\
0.0386199727654457 14.7003479003906\\
0.0436468608677387 14.7003479003906\\
0.0491764321923256 14.7003479003906\\
0.0552589632570744 14.7003479003906\\
0.0619497485458851 14.7003479003906\\
0.0693096145987511 14.7003479003906\\
0.0774054601788521 14.7003479003906\\
0.0863108932971954 14.7003469467163\\
0.0961068719625473 14.7003479003906\\
0.106882445514202 14.7003479003906\\
0.118735581636429 14.7003479003906\\
0.13177402317524 14.7003479003906\\
0.146116316318512 14.7003479003906\\
0.161892831325531 14.7003479003906\\
0.179247006773949 14.7003479003906\\
0.198336601257324 14.7003479003906\\
0.21933513879776 14.7003479003906\\
0.242433547973633 14.7003479003906\\
0.267841786146164 14.7003479003906\\
0.29579085111618 14.7003479003906\\
0.32653483748436 14.7003479003906\\
0.360353201627731 14.7003479003906\\
0.397553414106369 14.7003479003906\\
0.438473641872406 14.7003479003906\\
0.483485877513886 14.7003479003906\\
0.532999336719513 14.7003479003906\\
0.587464153766632 14.7003479003906\\
0.647375464439392 14.7003479003906\\
0.71327793598175 14.7003479003906\\
0.7857705950737 14.7003479003906\\
0.865512549877167 14.7003479003906\\
0.953228712081909 14.7003479003906\\
1.04971647262573 14.7003479003906\\
1.14971649646759 14.7003479003906\\
1.24971640110016 14.7003479003906\\
1.34971642494202 14.7003479003906\\
1.44971644878387 14.7003479003906\\
1.54971647262573 14.7003479003906\\
1.64971649646759 14.7003479003906\\
1.74971640110016 14.7003479003906\\
1.84971642494202 14.7003479003906\\
1.94971644878387 14.7003479003906\\
2.04971647262573 14.7003479003906\\
2.1497163772583 14.7003479003906\\
2.24971652030945 14.7003479003906\\
2.34971642494202 14.7003479003906\\
2.44971656799316 14.7003479003906\\
2.54971647262573 14.7003479003906\\
2.6497163772583 14.7003479003906\\
2.74971652030945 14.7003479003906\\
2.84971642494202 14.7003479003906\\
2.94971656799316 14.7003479003906\\
3.04971647262573 14.7003479003906\\
3.1497163772583 14.7003479003906\\
3.24971652030945 14.7003479003906\\
3.34971642494202 14.7003479003906\\
3.44971656799316 14.7003479003906\\
3.54971647262573 14.7003479003906\\
3.6497163772583 14.7003479003906\\
3.74971652030945 14.7003479003906\\
3.84971642494202 14.7003479003906\\
3.94971656799316 14.7003479003906\\
4.04971647262573 14.7003479003906\\
4.1497163772583 14.7003479003906\\
4.24971628189087 14.7003479003906\\
4.3497166633606 14.7003479003906\\
4.44971656799316 14.7003479003906\\
4.54971647262573 14.7003479003906\\
4.6497163772583 14.7003479003906\\
4.74971628189087 14.7003479003906\\
4.8497166633606 14.7003479003906\\
4.94971656799316 14.7003479003906\\
5.04971647262573 14.7003479003906\\
5.1497163772583 14.7003479003906\\
5.24971628189087 14.7003479003906\\
5.3497166633606 14.7003479003906\\
5.44971656799316 14.7003479003906\\
5.54971647262573 14.7003479003906\\
5.6497163772583 14.7003479003906\\
5.74971628189087 14.7003479003906\\
5.8497166633606 14.7003479003906\\
5.94971656799316 14.7003479003906\\
6.04971647262573 14.7003479003906\\
6.1497163772583 14.7003469467163\\
6.24971628189087 14.700345993042\\
6.3497166633606 14.700345993042\\
6.44971656799316 14.7003469467163\\
6.54971647262573 14.7003479003906\\
6.6497163772583 14.7003488540649\\
6.74971628189087 14.7003488540649\\
6.8497166633606 14.7003479003906\\
6.94971656799316 14.7003479003906\\
7.04971647262573 14.7003479003906\\
7.1497163772583 14.7003479003906\\
7.24971628189087 14.7003479003906\\
7.3497166633606 14.7003479003906\\
7.44971656799316 14.7003479003906\\
7.54971647262573 14.7003479003906\\
7.6497163772583 14.7003479003906\\
7.74971628189087 14.7003479003906\\
7.8497166633606 14.7003479003906\\
7.94971656799316 14.7003479003906\\
8.04971599578857 14.7003479003906\\
8.1497163772583 14.7003479003906\\
8.24971675872803 14.7003479003906\\
8.34971618652344 14.7003479003906\\
8.44971656799316 14.7003479003906\\
8.54971599578857 14.7003479003906\\
8.6497163772583 14.7003479003906\\
8.74971675872803 14.7003479003906\\
8.84971618652344 14.7003479003906\\
8.94971656799316 14.7003479003906\\
9.04971599578857 14.7003479003906\\
9.1497163772583 14.7003479003906\\
9.24971675872803 14.7003479003906\\
9.34971618652344 14.7003479003906\\
9.44971656799316 14.7003479003906\\
9.54971599578857 14.7003479003906\\
9.6497163772583 14.7003479003906\\
9.74971675872803 14.7003479003906\\
9.84971618652344 14.7003479003906\\
9.94971656799316 14.7003479003906\\
10.0497159957886 14.7003479003906\\
10.1497163772583 14.7003479003906\\
10.249716758728 14.7003479003906\\
10.3497161865234 14.7003479003906\\
10.4497165679932 14.7003479003906\\
10.5497159957886 14.7003479003906\\
10.6497163772583 14.7003479003906\\
10.749716758728 14.7003479003906\\
10.8497161865234 14.7003479003906\\
10.9497165679932 14.7003479003906\\
11.0497159957886 14.7003479003906\\
11.1497163772583 14.7003479003906\\
11.249716758728 14.7003479003906\\
11.3497161865234 14.7003479003906\\
11.4497165679932 14.7003479003906\\
11.5497159957886 14.7003479003906\\
11.6497163772583 14.7003479003906\\
11.749716758728 14.7003479003906\\
11.8497161865234 14.7003479003906\\
11.9497165679932 14.7003479003906\\
12.0497159957886 14.7003479003906\\
12.1497163772583 14.7003479003906\\
12.249716758728 14.7003479003906\\
12.347146987915 14.7003479003906\\
12.4250946044922 14.7003479003906\\
12.4874591827393 14.7003488540649\\
12.5560598373413 14.7003479003906\\
12.6315202713013 14.7003469467163\\
12.714527130127 14.7003479003906\\
12.8058338165283 14.7003479003906\\
12.8497476577759 14.7003488540649\\
12.8980522155762 14.7003469467163\\
12.9143037796021 14.7003479003906\\
12.9277105331421 14.7003488540649\\
12.9424571990967 14.7003479003906\\
12.9586782455444 14.7003479003906\\
12.9765224456787 14.7003479003906\\
12.9961500167847 14.7003479003906\\
13.0177412033081 14.7003479003906\\
13.0414915084839 14.7003479003906\\
13.0676164627075 14.7003479003906\\
13.0963535308838 14.7003479003906\\
13.1279649734497 14.7003479003906\\
13.1627368927002 14.7003479003906\\
13.2009868621826 14.7003479003906\\
13.2430610656738 14.7003479003906\\
13.289342880249 14.7003479003906\\
13.3402528762817 14.7003479003906\\
13.3962545394897 14.7003479003906\\
13.4578561782837 14.7003479003906\\
13.5256175994873 14.7003479003906\\
13.600154876709 14.7003479003906\\
13.6821460723877 14.7003469467163\\
13.7723369598389 14.7003479003906\\
13.871545791626 14.7003469467163\\
13.9715461730957 14.700364112854\\
14.0715456008911 14.7004346847534\\
14.1215467453003 14.7007761001587\\
14.147439956665 14.7009401321411\\
14.1759233474731 14.6995906829834\\
14.1775197982788 14.6950759887695\\
14.1792755126953 14.6956443786621\\
14.1812076568604 14.6990003585815\\
14.1833324432373 14.7011137008667\\
14.1856698989868 14.7018070220947\\
14.1882400512695 14.7012662887573\\
14.191068649292 14.7011804580688\\
14.1941795349121 14.7009706497192\\
14.1976013183594 14.7009878158569\\
14.2013654708862 14.7008810043335\\
14.2055063247681 14.7007179260254\\
14.2100610733032 14.700587272644\\
14.2150716781616 14.700493812561\\
14.2205829620361 14.7004137039185\\
14.2266454696655 14.7003479003906\\
14.2333135604858 14.7002801895142\\
14.2406492233276 14.7002201080322\\
14.2487182617188 14.7002029418945\\
14.2575941085815 14.7002286911011\\
14.2673578262329 14.7000799179077\\
14.27809715271 14.7001304626465\\
14.2899112701416 14.7001800537109\\
14.3029069900513 14.7001829147339\\
14.3172016143799 14.7001667022705\\
14.3329257965088 14.7001657485962\\
14.3502225875854 14.7001848220825\\
14.3692483901978 14.7002325057983\\
14.3901777267456 14.7002782821655\\
14.4131994247437 14.7003517150879\\
14.4385232925415 14.7003107070923\\
14.4663791656494 14.7003364562988\\
14.4970216751099 14.7003488540649\\
14.520975112915 14.7003364562988\\
14.5473232269287 14.7003450393677\\
14.5763063430786 14.7003469467163\\
14.6081876754761 14.7003517150879\\
14.6432571411133 14.7003517150879\\
14.6818342208862 14.7003517150879\\
14.7242679595947 14.7003479003906\\
14.7709455490112 14.7003479003906\\
14.8222913742065 14.7003479003906\\
14.8712491989136 14.7003498077393\\
14.9185028076172 14.7003488540649\\
14.9638071060181 14.7003488540649\\
15.0066013336182 14.7003479003906\\
15.0464649200439 14.7003479003906\\
15.0903158187866 14.7003479003906\\
15.1385507583618 14.7003479003906\\
15.1916093826294 14.7003479003906\\
15.2499742507935 14.7003479003906\\
15.3141756057739 14.7003479003906\\
15.3847970962524 14.7003479003906\\
15.4624805450439 14.7003479003906\\
15.5479326248169 14.7003479003906\\
15.6419296264648 14.7003469467163\\
15.7419300079346 14.7003479003906\\
15.7919301986694 14.7003469467163\\
15.8469295501709 14.7003479003906\\
15.9074296951294 14.7003469467163\\
15.920783996582 14.7003469467163\\
15.9354724884033 14.7003488540649\\
15.9516305923462 14.7003469467163\\
15.9645795822144 14.7003479003906\\
15.9788236618042 14.7003479003906\\
15.9944925308228 14.7003479003906\\
16.0117282867432 14.7003479003906\\
16.0306873321533 14.7003479003906\\
16.0515422821045 14.7003479003906\\
16.0744819641113 14.7003479003906\\
16.0997161865234 14.7003479003906\\
16.1274738311768 14.7003479003906\\
16.1580085754395 14.7003479003906\\
16.1915950775146 14.7003479003906\\
16.2285404205322 14.7003479003906\\
16.2691822052002 14.7003479003906\\
16.3138847351074 14.7003479003906\\
16.3630599975586 14.7003479003906\\
16.4171524047852 14.7003479003906\\
16.4766540527344 14.7003479003906\\
16.542106628418 14.7003479003906\\
16.6141033172607 14.7003479003906\\
16.6932983398438 14.7003479003906\\
16.7804145812988 14.7003469467163\\
16.87624168396 14.7003479003906\\
16.9762420654297 14.7003469467163\\
17.0762424468994 14.7003574371338\\
17.0962505340576 14.700366973877\\
17.1021957397461 14.7003650665283\\
17.1087341308594 14.7003631591797\\
17.1159267425537 14.7003631591797\\
17.1238384246826 14.7003650665283\\
17.1325416564941 14.700366973877\\
17.1421165466309 14.7003679275513\\
17.1526470184326 14.7003688812256\\
17.1642322540283 14.700366973877\\
17.1769752502441 14.7003650665283\\
17.1909923553467 14.700364112854\\
17.2064094543457 14.7003631591797\\
17.2233715057373 14.7003602981567\\
17.2420272827148 14.7003574371338\\
17.2625503540039 14.7003507614136\\
17.2851238250732 14.7003469467163\\
17.3099555969238 14.7003440856934\\
17.3372707366943 14.7003412246704\\
17.367317199707 14.7003355026245\\
17.400369644165 14.7003202438354\\
17.4367256164551 14.7003021240234\\
17.4767169952393 14.7003507614136\\
17.5207099914551 14.7003488540649\\
17.5690994262695 14.7003469467163\\
17.6223297119141 14.700345993042\\
17.6808815002441 14.700345993042\\
17.745288848877 14.7003507614136\\
17.8161373138428 14.700345993042\\
17.8624935150146 14.7003421783447\\
17.9028415679932 14.7003412246704\\
17.9472217559814 14.7003469467163\\
17.9960422515869 14.7003469467163\\
18.0497417449951 14.7003479003906\\
18.108814239502 14.7003479003906\\
18.173791885376 14.7003479003906\\
18.245267868042 14.7003469467163\\
18.3238925933838 14.7003469467163\\
18.41037940979 14.7003450393677\\
18.5055141448975 14.7003479003906\\
18.6055145263672 14.700345993042\\
18.7055149078369 14.7003479003906\\
18.805513381958 14.7003479003906\\
18.9055137634277 14.7003479003906\\
19.0055141448975 14.7003479003906\\
19.1055145263672 14.7003479003906\\
19.2055149078369 14.7003479003906\\
19.305513381958 14.7003479003906\\
19.4055137634277 14.7003479003906\\
19.5055141448975 14.7003479003906\\
19.6055145263672 14.7003479003906\\
19.7055149078369 14.7003479003906\\
19.805513381958 14.7003479003906\\
19.9055137634277 14.7003479003906\\
20.0055141448975 14.7003479003906\\
20.1055145263672 14.7003479003906\\
20.2055149078369 14.7003479003906\\
20.305513381958 14.7003479003906\\
20.4055137634277 14.7003479003906\\
20.5055141448975 14.7003479003906\\
20.6055145263672 14.7003479003906\\
20.7055149078369 14.7003479003906\\
20.805513381958 14.7003479003906\\
20.9055137634277 14.7003479003906\\
21.0055141448975 14.7003479003906\\
21.1055145263672 14.7003479003906\\
21.2055149078369 14.7003479003906\\
21.305513381958 14.7003479003906\\
21.4055137634277 14.7003479003906\\
21.5055141448975 14.7003479003906\\
21.6055145263672 14.7003479003906\\
21.7055149078369 14.7003479003906\\
21.805513381958 14.7003479003906\\
21.9055137634277 14.7003479003906\\
22.0055141448975 14.7003479003906\\
22.1055145263672 14.7003479003906\\
22.2055149078369 14.7003479003906\\
22.305513381958 14.7003479003906\\
22.4055137634277 14.7003479003906\\
22.5055141448975 14.7003479003906\\
22.6055145263672 14.7003479003906\\
22.7055149078369 14.7003479003906\\
22.805513381958 14.7003479003906\\
22.9055137634277 14.7003479003906\\
23.0055141448975 14.7003479003906\\
23.1055145263672 14.7003479003906\\
23.2055149078369 14.7003479003906\\
23.305513381958 14.7003479003906\\
23.4055137634277 14.7003479003906\\
23.5055141448975 14.7003479003906\\
23.6055145263672 14.7003479003906\\
23.7055149078369 14.7003479003906\\
23.805513381958 14.7003479003906\\
23.9055137634277 14.7003479003906\\
24.0055141448975 14.7003469467163\\
24.1055145263672 14.7003479003906\\
24.2055149078369 14.7003479003906\\
24.305513381958 14.7003479003906\\
24.4055137634277 14.7003479003906\\
24.5055141448975 14.7003479003906\\
24.6055145263672 14.7003479003906\\
24.7055149078369 14.7003479003906\\
24.789867401123 14.7003479003906\\
24.8573379516602 14.7003488540649\\
24.9315567016602 14.7003479003906\\
25.0131969451904 14.7003469467163\\
25.1030006408691 14.7003469467163\\
25.1443557739258 14.7003488540649\\
25.189847946167 14.7003450393677\\
25.2099170684814 14.7003526687622\\
25.2266445159912 14.7003498077393\\
25.245044708252 14.7003488540649\\
25.2652854919434 14.7003479003906\\
25.2875499725342 14.7003479003906\\
25.3120403289795 14.7003479003906\\
25.3389797210693 14.7003479003906\\
25.3686122894287 14.7003479003906\\
25.4012107849121 14.7003479003906\\
25.4370670318604 14.7003479003906\\
25.4765090942383 14.7003479003906\\
25.5198955535889 14.7003479003906\\
25.5676212310791 14.7003479003906\\
25.6201171875 14.7003479003906\\
25.6778659820557 14.7003479003906\\
25.7413883209229 14.7003479003906\\
25.8112621307373 14.7003479003906\\
25.8881244659424 14.7003479003906\\
25.9726715087891 14.7003469467163\\
26.065673828125 14.7003479003906\\
26.1656742095947 14.7003469467163\\
26.2656745910645 14.7003583908081\\
26.3656749725342 14.7004613876343\\
26.409065246582 14.7011423110962\\
26.4240894317627 14.7014245986938\\
26.4268970489502 14.686824798584\\
26.4299831390381 14.6959629058838\\
26.4333801269531 14.7002267837524\\
26.4371166229248 14.7019624710083\\
26.4412250518799 14.701322555542\\
26.4457454681396 14.7016468048096\\
26.4507179260254 14.7013826370239\\
26.4561862945557 14.7009582519531\\
26.4622020721436 14.7007761001587\\
26.4688205718994 14.7006101608276\\
26.4761009216309 14.7004356384277\\
26.4841079711914 14.7006168365479\\
26.4929161071777 14.6999206542969\\
26.5026054382324 14.7000093460083\\
26.5132637023926 14.6995620727539\\
26.5249881744385 14.6990823745728\\
26.5378837585449 14.6997766494751\\
26.5520706176758 14.700213432312\\
26.5676746368408 14.7003316879272\\
26.5848388671875 14.7003707885742\\
26.6037216186523 14.7011280059814\\
26.624490737915 14.6997661590576\\
26.6473369598389 14.7001094818115\\
26.6724681854248 14.7004261016846\\
26.7001132965088 14.7003479003906\\
26.7305221557617 14.7003316879272\\
26.7594413757324 14.7003297805786\\
26.7881412506104 14.7003374099731\\
26.8180274963379 14.7003440856934\\
26.848762512207 14.700345993042\\
26.8805618286133 14.7003479003906\\
26.9128971099854 14.7003479003906\\
26.9456844329834 14.7003479003906\\
26.97873878479 14.7003479003906\\
27.0120601654053 14.7003479003906\\
27.0457935333252 14.7003479003906\\
27.0821418762207 14.7003479003906\\
27.1221237182617 14.7003488540649\\
27.1524639129639 14.7003479003906\\
27.1838111877441 14.7003479003906\\
27.2182941436768 14.7003479003906\\
27.2558403015137 14.7003479003906\\
27.2955665588379 14.7003479003906\\
27.3392658233643 14.7003479003906\\
27.3873329162598 14.7003479003906\\
27.4402103424072 14.7003479003906\\
27.4983730316162 14.7003479003906\\
27.5623531341553 14.7003479003906\\
27.6327304840088 14.7003479003906\\
27.7101459503174 14.7003479003906\\
27.7953014373779 14.7003479003906\\
27.8889751434326 14.7003469467163\\
27.9889755249023 14.7003479003906\\
28.0889739990234 14.7003469467163\\
28.1889743804932 14.7003479003906\\
28.2889747619629 14.7003469467163\\
28.2927665710449 14.7003440856934\\
28.2969398498535 14.7003536224365\\
28.3015289306641 14.700309753418\\
28.304780960083 14.7003860473633\\
28.307991027832 14.7003622055054\\
28.3115215301514 14.7003488540649\\
28.3154048919678 14.7003479003906\\
28.3196773529053 14.7003488540649\\
28.3243770599365 14.7003488540649\\
28.3295459747314 14.7003488540649\\
28.3352317810059 14.7003488540649\\
28.3414859771729 14.7003488540649\\
28.3483657836914 14.7003479003906\\
28.3559341430664 14.7003479003906\\
28.3642578125 14.7003479003906\\
28.3734169006348 14.7003479003906\\
28.3834896087646 14.7003479003906\\
28.3945693969727 14.7003479003906\\
28.4067573547363 14.7003479003906\\
28.4201641082764 14.7003479003906\\
28.4349136352539 14.7003479003906\\
28.451135635376 14.7003479003906\\
28.4689807891846 14.7003479003906\\
28.4886093139648 14.7003479003906\\
28.5102024078369 14.7003479003906\\
28.5339546203613 14.7003479003906\\
28.5600814819336 14.7003479003906\\
28.5888214111328 14.7003479003906\\
28.620433807373 14.7003479003906\\
28.6552085876465 14.7003479003906\\
28.6934604644775 14.7003479003906\\
28.735538482666 14.7003479003906\\
28.7818241119385 14.7003479003906\\
28.8327369689941 14.7003479003906\\
28.8887424468994 14.7003479003906\\
28.9503479003906 14.7003479003906\\
29.0181159973145 14.7003479003906\\
29.092658996582 14.7003479003906\\
29.1746559143066 14.7003479003906\\
29.2648525238037 14.7003469467163\\
29.3640689849854 14.7003479003906\\
29.4640693664551 14.7003469467163\\
29.4847068786621 14.7005577087402\\
29.4888343811035 14.7005167007446\\
29.4933738708496 14.7005319595337\\
29.4983692169189 14.7005453109741\\
29.5038623809814 14.7005777359009\\
29.5099067687988 14.7006187438965\\
29.5165538787842 14.7006635665894\\
29.5230560302734 14.7007150650024\\
29.5302066802979 14.7007637023926\\
29.5380744934082 14.7007856369019\\
29.5467281341553 14.700737953186\\
29.556245803833 14.6995410919189\\
29.5667171478271 14.6985721588135\\
29.5782356262207 14.7013063430786\\
29.5909042358398 14.7007780075073\\
29.6048412322998 14.7006349563599\\
29.6201705932617 14.7010383605957\\
29.637035369873 14.7005300521851\\
29.6555843353271 14.7004308700562\\
29.675989151001 14.7004089355469\\
29.6984329223633 14.7000226974487\\
29.7114696502686 14.7001466751099\\
29.7258071899414 14.7001419067383\\
29.7415809631348 14.7001361846924\\
29.7589302062988 14.7001810073853\\
29.7780151367188 14.699725151062\\
29.7990093231201 14.7000904083252\\
29.8221015930176 14.7004013061523\\
29.8475036621094 14.7003135681152\\
29.8754444122314 14.7003602981567\\
29.9061813354492 14.7003421783447\\
29.9399909973145 14.7003602981567\\
29.9771823883057 14.7003517150879\\
30.0180931091309 14.7003440856934\\
30.0630931854248 14.7003517150879\\
30.1125946044922 14.7003555297852\\
30.1670475006104 14.7003440856934\\
30.2269439697266 14.7003564834595\\
30.2706527709961 14.7003803253174\\
30.318733215332 14.7003450393677\\
30.3593273162842 14.7003498077393\\
30.3947944641113 14.7003402709961\\
30.4330863952637 14.7003374099731\\
30.4652576446533 14.7003402709961\\
30.5006484985352 14.700343132019\\
30.5330505371094 14.7003450393677\\
30.5677318572998 14.700345993042\\
30.6004276275635 14.7003469467163\\
30.6349658966064 14.7003469467163\\
30.6682662963867 14.7003469467163\\
30.7048969268799 14.7003469467163\\
30.7451915740967 14.7003479003906\\
30.789514541626 14.7003469467163\\
30.8382682800293 14.7003469467163\\
30.8918991088867 14.700345993042\\
30.9508934020996 14.700345993042\\
31.0157871246338 14.7003469467163\\
31.0871696472168 14.7003469467163\\
31.1656913757324 14.7003488540649\\
31.2520637512207 14.7003479003906\\
31.347074508667 14.7003479003906\\
31.4470748901367 14.7003488540649\\
31.5470752716064 14.7003479003906\\
31.6470737457275 14.7003479003906\\
31.7470741271973 14.7003479003906\\
31.847074508667 14.7003479003906\\
31.9470748901367 14.7003479003906\\
32.0470733642578 14.7003479003906\\
32.1470756530762 14.7003479003906\\
32.2470741271973 14.7003479003906\\
32.3470764160156 14.7003479003906\\
32.4470748901367 14.7003479003906\\
32.5470733642578 14.7003479003906\\
32.6470756530762 14.7003479003906\\
32.7470741271973 14.7003479003906\\
32.8470764160156 14.7003479003906\\
32.9470748901367 14.7003479003906\\
33.0470733642578 14.7003479003906\\
33.1470756530762 14.7003479003906\\
33.2470741271973 14.7003479003906\\
33.3470764160156 14.7003479003906\\
33.4470748901367 14.7003479003906\\
33.5470733642578 14.7003479003906\\
33.6470756530762 14.7003479003906\\
33.7470741271973 14.7003479003906\\
33.8470764160156 14.7003479003906\\
33.9470748901367 14.7003479003906\\
34.0470733642578 14.7003479003906\\
34.1470756530762 14.7003479003906\\
34.2470741271973 14.7003479003906\\
34.3470764160156 14.7003479003906\\
34.4470748901367 14.7003479003906\\
34.5470733642578 14.7003479003906\\
34.6470756530762 14.7003479003906\\
34.7470741271973 14.7003479003906\\
34.8470764160156 14.7003479003906\\
34.9470748901367 14.7003479003906\\
35.0470733642578 14.7003479003906\\
35.1470756530762 14.7003479003906\\
35.2470741271973 14.7003479003906\\
35.3470764160156 14.7003479003906\\
35.4470748901367 14.7003479003906\\
35.5470733642578 14.7003479003906\\
35.6470756530762 14.7003479003906\\
35.7470741271973 14.7003479003906\\
35.8470764160156 14.7003479003906\\
35.9470748901367 14.7003479003906\\
36.0470733642578 14.7003479003906\\
36.1470756530762 14.7003479003906\\
36.2470741271973 14.7003479003906\\
36.3470764160156 14.7003479003906\\
36.4470748901367 14.7003479003906\\
36.5470733642578 14.7003479003906\\
36.6470756530762 14.7003479003906\\
36.7470741271973 14.7003479003906\\
36.8470764160156 14.7003479003906\\
36.9470748901367 14.7003479003906\\
37.0470733642578 14.7003479003906\\
37.1470756530762 14.7003479003906\\
37.2293395996094 14.7003479003906\\
37.2951393127441 14.7003488540649\\
37.3675231933594 14.7003469467163\\
37.4471435546875 14.7003479003906\\
37.534725189209 14.7003479003906\\
37.6310653686523 14.7003469467163\\
37.6772041320801 14.7003488540649\\
37.7279586791992 14.700343132019\\
37.7453880310059 14.7003583908081\\
37.7602653503418 14.7003498077393\\
37.7766304016113 14.7003488540649\\
37.7946319580078 14.7003479003906\\
37.8144378662109 14.7003479003906\\
37.8362197875977 14.7003479003906\\
37.860179901123 14.7003479003906\\
37.8865356445313 14.7003479003906\\
37.9155311584473 14.7003479003906\\
37.9474220275879 14.7003479003906\\
37.9825019836426 14.7003479003906\\
38.0210914611816 14.7003479003906\\
38.0635414123535 14.7003479003906\\
38.1102333068848 14.7003479003906\\
38.1615982055664 14.7003479003906\\
38.2180976867676 14.7003479003906\\
38.2802429199219 14.7003479003906\\
38.3486099243164 14.7003479003906\\
38.4238090515137 14.7003479003906\\
38.5065269470215 14.7003469467163\\
38.5975189208984 14.7003479003906\\
38.6975212097168 14.7003469467163\\
38.7975196838379 14.7003602981567\\
38.8975219726563 14.700421333313\\
38.9475212097168 14.7009515762329\\
38.9856338500977 14.700831413269\\
39.0002326965332 14.6970434188843\\
39.0162887573242 14.7003164291382\\
39.0339508056641 14.7007751464844\\
39.0533828735352 14.7007064819336\\
39.0747528076172 14.7001695632935\\
39.0982627868652 14.70006275177\\
39.1241226196289 14.7002410888672\\
39.1525688171387 14.7002058029175\\
39.1838569641113 14.7003259658813\\
39.2182769775391 14.700366973877\\
39.2382202148438 14.7002630233765\\
39.2601547241211 14.7002744674683\\
39.2842864990234 14.7003211975098\\
39.3108291625977 14.70032787323\\
39.3400230407715 14.7003364562988\\
39.3721389770508 14.7003421783447\\
39.4047470092773 14.7003450393677\\
39.4365348815918 14.7003488540649\\
39.4690742492676 14.7003507614136\\
39.5013847351074 14.7003517150879\\
39.5341873168945 14.7003479003906\\
39.5668563842773 14.7003488540649\\
39.5998191833496 14.7003479003906\\
39.632698059082 14.7003479003906\\
39.6657829284668 14.7003479003906\\
39.6988639831543 14.7003488540649\\
39.7321624755859 14.7003488540649\\
39.7687873840332 14.7003488540649\\
39.8090744018555 14.7003479003906\\
39.8533935546875 14.7003479003906\\
39.8968162536621 14.7003479003906\\
39.939453125 14.7003479003906\\
39.9863548278809 14.7003479003906\\
40.0379447937012 14.7003479003906\\
40.0946960449219 14.7003479003906\\
40.1571197509766 14.7003479003906\\
40.2257843017578 14.7003479003906\\
40.3013191223145 14.7003479003906\\
40.384407043457 14.7003479003906\\
40.4758033752441 14.7003469467163\\
40.5758018493652 14.7003479003906\\
40.6758041381836 14.7003469467163\\
40.7758026123047 14.7003479003906\\
40.875804901123 14.7003469467163\\
40.8792419433594 14.7003393173218\\
40.8830223083496 14.7003631591797\\
40.8871841430664 14.7002878189087\\
40.8903007507324 14.7004013061523\\
40.8937301635742 14.7003660202026\\
40.8975028991699 14.7003488540649\\
40.9016532897949 14.7003469467163\\
40.9062156677246 14.7003488540649\\
40.9112396240234 14.7003488540649\\
40.9167633056641 14.7003488540649\\
40.9228363037109 14.7003488540649\\
40.9295196533203 14.7003479003906\\
40.9368743896484 14.7003479003906\\
40.9449615478516 14.7003479003906\\
40.953857421875 14.7003479003906\\
40.9636421203613 14.7003479003906\\
40.9744033813477 14.7003479003906\\
40.9862442016602 14.7003479003906\\
40.999267578125 14.7003479003906\\
41.0135955810547 14.7003479003906\\
41.029354095459 14.7003479003906\\
41.046688079834 14.7003479003906\\
41.0657577514648 14.7003479003906\\
41.0867347717285 14.7003479003906\\
41.109806060791 14.7003479003906\\
41.1351852416992 14.7003479003906\\
41.1631050109863 14.7003479003906\\
41.1938133239746 14.7003479003906\\
41.227596282959 14.7003479003906\\
41.2647552490234 14.7003479003906\\
41.3056297302246 14.7003479003906\\
41.3505935668945 14.7003479003906\\
41.4000511169434 14.7003479003906\\
41.4544563293457 14.7003479003906\\
41.5143013000488 14.7003479003906\\
41.5801315307617 14.7003479003906\\
41.6525421142578 14.7003479003906\\
41.7321968078613 14.7003479003906\\
41.8198165893555 14.7003469467163\\
41.9161987304688 14.7003479003906\\
42.0161972045898 14.7003469467163\\
42.1161994934082 14.7003736495972\\
42.1361999511719 14.7004880905151\\
42.1442070007324 14.7006158828735\\
42.1520309448242 14.7006616592407\\
42.1606330871582 14.7006893157959\\
42.1689529418945 14.7006912231445\\
42.1781005859375 14.7005605697632\\
42.1881637573242 14.699047088623\\
42.1992378234863 14.7001857757568\\
42.2114143371582 14.7007608413696\\
42.2248077392578 14.7006464004517\\
42.2395439147949 14.7007312774658\\
42.2557525634766 14.7006711959839\\
42.2735824584961 14.7005300521851\\
42.2931938171387 14.7003879547119\\
42.3147659301758 14.7002897262573\\
42.3384971618652 14.7004270553589\\
42.3646011352539 14.7000827789307\\
42.3847808837891 14.7000951766968\\
42.4069786071777 14.6998195648193\\
42.431396484375 14.7002754211426\\
42.4582557678223 14.7002630233765\\
42.4878005981445 14.7003488540649\\
42.5203018188477 14.7003393173218\\
42.5560493469238 14.7003536224365\\
42.5953750610352 14.7003498077393\\
42.6386299133301 14.7003574371338\\
42.6862144470215 14.7003316879272\\
42.7385559082031 14.7003698348999\\
42.7959022521973 14.7003202438354\\
42.8589859008789 14.7003717422485\\
42.9028167724609 14.7003870010376\\
42.9510307312012 14.7003240585327\\
42.9925117492676 14.7003574371338\\
43.0297622680664 14.7003469467163\\
43.0707359313965 14.7003412246704\\
43.1158065795898 14.7003402709961\\
43.1579742431641 14.7003383636475\\
43.203685760498 14.7003488540649\\
43.2497863769531 14.700345993042\\
43.3005027770996 14.7003469467163\\
43.3562850952148 14.7003469467163\\
43.4176483154297 14.7003479003906\\
43.4851493835449 14.7003469467163\\
43.559398651123 14.7003479003906\\
43.6410713195801 14.7003479003906\\
43.7309150695801 14.7003479003906\\
43.8297386169434 14.7003479003906\\
43.9297409057617 14.7003479003906\\
44.0297393798828 14.7003479003906\\
44.1297416687012 14.7003479003906\\
44.2297401428223 14.7003479003906\\
44.2499618530273 14.7003479003906\\
44.2722015380859 14.7003479003906\\
44.2966690063477 14.7003479003906\\
44.3235816955566 14.7003479003906\\
44.3531875610352 14.7003479003906\\
44.3857498168945 14.7003479003906\\
44.421573638916 14.7003479003906\\
44.4609756469727 14.7003479003906\\
44.5043182373047 14.7003479003906\\
44.5519981384277 14.7003479003906\\
44.6044425964355 14.7003479003906\\
44.6621360778809 14.7003479003906\\
44.7255935668945 14.7003479003906\\
44.7953987121582 14.7003479003906\\
44.872184753418 14.7003479003906\\
44.9566497802734 14.7003479003906\\
45.0495643615723 14.7003479003906\\
45.1495628356934 14.7003479003906\\
45.2495613098145 14.7003479003906\\
45.3495635986328 14.7003479003906\\
45.4495620727539 14.7003479003906\\
45.5495643615723 14.7003479003906\\
45.6495628356934 14.7003479003906\\
45.7495613098145 14.7003612518311\\
45.7695655822754 14.7003707885742\\
45.7816352844238 14.700366973877\\
45.7949142456055 14.7003688812256\\
45.8095207214355 14.7003631591797\\
45.8255882263184 14.7003564834595\\
45.8432655334473 14.7003536224365\\
45.8627052307129 14.7003498077393\\
45.884090423584 14.700345993042\\
45.9076156616211 14.7003507614136\\
45.9334945678711 14.7003602981567\\
45.9619598388672 14.700345993042\\
45.9932708740234 14.7003402709961\\
46.0277099609375 14.7003307342529\\
46.0655975341797 14.7003393173218\\
46.1072731018066 14.7003383636475\\
46.1531181335449 14.7003383636475\\
46.2035446166992 14.7003383636475\\
46.2590141296387 14.7003393173218\\
46.3200302124023 14.7003440856934\\
46.387149810791 14.7003469467163\\
46.4504814147949 14.7003479003906\\
46.5130958557129 14.7003479003906\\
46.581974029541 14.7003479003906\\
46.6577415466309 14.7003479003906\\
46.7410850524902 14.7003479003906\\
46.8327598571777 14.7003479003906\\
46.9327621459961 14.7003479003906\\
47.0327606201172 14.7003479003906\\
47.1327629089355 14.7003479003906\\
47.2327613830566 14.7003479003906\\
47.3327598571777 14.7003479003906\\
47.4327621459961 14.7003479003906\\
47.5327606201172 14.7003479003906\\
47.6327629089355 14.7003479003906\\
47.7327613830566 14.7003479003906\\
47.8327598571777 14.7003479003906\\
47.9327621459961 14.7003479003906\\
48.0327606201172 14.7003479003906\\
48.1327629089355 14.7003479003906\\
48.2327613830566 14.7003479003906\\
48.3327598571777 14.7003479003906\\
48.4327621459961 14.7003479003906\\
48.5327606201172 14.7003479003906\\
48.6327629089355 14.7003479003906\\
48.7327613830566 14.7003479003906\\
48.8327598571777 14.7003479003906\\
48.9327621459961 14.7003479003906\\
49.0327606201172 14.7003479003906\\
49.1327629089355 14.7003479003906\\
49.2327613830566 14.7003479003906\\
49.3327598571777 14.7003479003906\\
49.4327621459961 14.7003479003906\\
49.5327606201172 14.7003479003906\\
49.6257057189941 14.7003479003906\\
49.6982688903809 14.7003479003906\\
49.7780876159668 14.7003469467163\\
49.8658905029297 14.7003479003906\\
49.962474822998 14.7003479003906\\
50.0624732971191 14.7003479003906\\
50.1624755859375 14.7003479003906\\
50.2624740600586 14.7003479003906\\
50.3624725341797 14.7003479003906\\
50.462474822998 14.7003479003906\\
50.5624732971191 14.7003479003906\\
50.6624755859375 14.7003479003906\\
50.7624740600586 14.7003479003906\\
50.8624725341797 14.7003479003906\\
50.962474822998 14.7003479003906\\
51.0624732971191 14.7003479003906\\
51.1624755859375 14.7003479003906\\
51.2624740600586 14.7003479003906\\
51.3624725341797 14.7003479003906\\
51.462474822998 14.7003479003906\\
51.5624732971191 14.7003479003906\\
51.6624755859375 14.7003479003906\\
51.7624740600586 14.7003479003906\\
51.8624725341797 14.7003479003906\\
51.962474822998 14.7003479003906\\
52.0624732971191 14.7003479003906\\
52.1624755859375 14.7003479003906\\
52.2624740600586 14.7003479003906\\
52.3624725341797 14.7003479003906\\
52.462474822998 14.7003479003906\\
52.5624732971191 14.7003479003906\\
52.6624755859375 14.7003479003906\\
52.7624740600586 14.7003479003906\\
52.8624725341797 14.7003479003906\\
52.962474822998 14.7003479003906\\
53.0624732971191 14.7003479003906\\
53.1624755859375 14.7003479003906\\
53.2624740600586 14.7003479003906\\
53.3624725341797 14.7003479003906\\
53.462474822998 14.7003479003906\\
53.5624732971191 14.7003479003906\\
53.6624755859375 14.7003479003906\\
53.7624740600586 14.7003479003906\\
53.8624725341797 14.7003479003906\\
53.962474822998 14.7003479003906\\
54.0624732971191 14.7003479003906\\
54.1624755859375 14.7003479003906\\
54.2624740600586 14.7003479003906\\
54.3624725341797 14.7003479003906\\
54.462474822998 14.7003479003906\\
54.5624732971191 14.7003479003906\\
54.6624755859375 14.7003479003906\\
54.7624740600586 14.7003479003906\\
54.8624725341797 14.7003479003906\\
54.962474822998 14.7003479003906\\
55.0624732971191 14.7003479003906\\
55.1624755859375 14.7003479003906\\
55.2624740600586 14.7003479003906\\
55.3624725341797 14.7003479003906\\
55.462474822998 14.7003479003906\\
55.5624732971191 14.7003469467163\\
55.6624755859375 14.7003450393677\\
55.7624740600586 14.7003479003906\\
55.8624725341797 14.700345993042\\
55.962474822998 14.7003479003906\\
56.0624732971191 14.7003488540649\\
56.1624755859375 14.7003479003906\\
56.2624740600586 14.7003479003906\\
56.3624725341797 14.7003469467163\\
56.462474822998 14.7003479003906\\
56.5624732971191 14.7003479003906\\
56.6624755859375 14.7003479003906\\
56.7624740600586 14.7003479003906\\
56.8624725341797 14.7003479003906\\
56.962474822998 14.7003479003906\\
57.0624732971191 14.7003479003906\\
57.1624755859375 14.7003479003906\\
57.2624740600586 14.7003479003906\\
57.3624725341797 14.7003479003906\\
57.462474822998 14.7003479003906\\
57.5624732971191 14.7003479003906\\
57.6624755859375 14.7003479003906\\
57.7624740600586 14.7003479003906\\
57.8624725341797 14.7003479003906\\
57.962474822998 14.7003479003906\\
58.0624732971191 14.7003479003906\\
58.1624755859375 14.7003479003906\\
58.2624740600586 14.7003479003906\\
58.3624725341797 14.7003479003906\\
58.462474822998 14.7003479003906\\
58.5624732971191 14.7003479003906\\
58.6624755859375 14.7003479003906\\
58.7624740600586 14.7003479003906\\
58.8624725341797 14.7003479003906\\
58.962474822998 14.7003479003906\\
59.0624732971191 14.7003479003906\\
59.1624755859375 14.7003479003906\\
59.2624740600586 14.7003479003906\\
59.3624725341797 14.7003479003906\\
59.462474822998 14.7003479003906\\
59.5624732971191 14.7003479003906\\
59.6624755859375 14.7003479003906\\
59.7624740600586 14.7003479003906\\
59.8624725341797 14.7003479003906\\
59.9312362670898 14.7003479003906\\
60 14.7003479003906\\
60.0756378173828 14.7003479003906\\
60.1588439941406 14.7003479003906\\
60.2503662109375 14.7003479003906\\
60.3503646850586 14.7003479003906\\
60.450366973877 14.7003479003906\\
60.550365447998 14.7003479003906\\
60.6503677368164 14.7003479003906\\
60.7503662109375 14.7003479003906\\
60.8503646850586 14.7003479003906\\
60.950366973877 14.7003479003906\\
61.050365447998 14.7003479003906\\
61.1503677368164 14.7003479003906\\
61.2503662109375 14.7003479003906\\
61.3503646850586 14.7003479003906\\
61.450366973877 14.7003479003906\\
61.550365447998 14.7003469467163\\
61.6503677368164 14.7003469467163\\
61.7503662109375 14.7003469467163\\
61.8503646850586 14.7003469467163\\
61.9325637817383 14.7003469467163\\
61.998291015625 14.7003469467163\\
62.0705947875977 14.7003469467163\\
62.1501274108887 14.7003479003906\\
62.2376098632813 14.7003479003906\\
62.3338470458984 14.7003488540649\\
62.4338455200195 14.7003488540649\\
62.5338439941406 14.7003488540649\\
62.633846282959 14.7003488540649\\
62.7338447570801 14.7003488540649\\
62.8338470458984 14.7003488540649\\
62.9338455200195 14.7003488540649\\
63.0338439941406 14.7003488540649\\
63.133846282959 14.7003488540649\\
63.2338447570801 14.7003488540649\\
63.3338470458984 14.7003488540649\\
63.4338455200195 14.7003488540649\\
63.5338439941406 14.7003488540649\\
63.633846282959 14.7003488540649\\
63.7338447570801 14.7003488540649\\
63.8338470458984 14.7003488540649\\
63.9338455200195 14.7003488540649\\
64.0338439941406 14.7003488540649\\
64.1338424682617 14.7003488540649\\
64.2338485717773 14.7003488540649\\
64.3338470458984 14.7003488540649\\
64.4338455200195 14.7003488540649\\
64.5338439941406 14.7003488540649\\
64.6338424682617 14.7003488540649\\
64.7338485717773 14.7003488540649\\
64.8338470458984 14.7003488540649\\
64.9338455200195 14.7003488540649\\
65.0338439941406 14.7003488540649\\
65.1338424682617 14.7003488540649\\
65.2338485717773 14.7003488540649\\
65.3338470458984 14.7003488540649\\
65.4338455200195 14.7003488540649\\
65.5338439941406 14.7003488540649\\
65.6338424682617 14.7003488540649\\
65.7338485717773 14.7003488540649\\
65.8338470458984 14.7003488540649\\
65.9338455200195 14.7003488540649\\
66.0338439941406 14.7003488540649\\
66.1338424682617 14.7003488540649\\
66.2338485717773 14.7003488540649\\
66.3338470458984 14.7003488540649\\
66.4338455200195 14.7003488540649\\
66.5338439941406 14.7003488540649\\
66.6338424682617 14.7003488540649\\
66.7338485717773 14.7003488540649\\
66.8338470458984 14.7003488540649\\
66.9338455200195 14.7003488540649\\
67.0338439941406 14.7003488540649\\
67.1338424682617 14.7003488540649\\
67.2338485717773 14.7003488540649\\
67.3338470458984 14.7003498077393\\
67.4338455200195 14.7003517150879\\
67.5338439941406 14.7003602981567\\
67.6338424682617 14.7003984451294\\
67.7338485717773 14.7005977630615\\
67.8338470458984 14.7016334533691\\
67.9338455200195 14.7060508728027\\
68.0338439941406 14.7175426483154\\
68.1338424682617 14.7300148010254\\
68.2338485717773 14.7325229644775\\
68.3338470458984 14.7313499450684\\
68.4338455200195 14.7333145141602\\
68.5338439941406 14.7319011688232\\
68.6338424682617 14.7335596084595\\
68.7338485717773 14.7333393096924\\
68.8338470458984 14.7343702316284\\
68.9338455200195 14.7345752716064\\
69.0338439941406 14.7353248596191\\
69.1338424682617 14.7357158660889\\
69.2338485717773 14.7363433837891\\
69.3338470458984 14.7368144989014\\
69.4338455200195 14.7373905181885\\
69.5338439941406 14.7378969192505\\
69.6338424682617 14.7384490966797\\
69.7338485717773 14.7389717102051\\
69.8338470458984 14.7395143508911\\
69.9338455200195 14.740044593811\\
70.0338439941406 14.7405834197998\\
70.1338424682617 14.7411165237427\\
70.2338485717773 14.7416534423828\\
70.3338470458984 14.7421894073486\\
70.4338455200195 14.7427263259888\\
70.5338439941406 14.7432632446289\\
70.6338424682617 14.7438774108887\\
70.7338485717773 14.7446966171265\\
70.8338470458984 14.7452974319458\\
70.9338455200195 14.7459516525269\\
71.0338439941406 14.7465858459473\\
71.1338424682617 14.7472248077393\\
71.2338485717773 14.7478618621826\\
71.3338470458984 14.748498916626\\
71.4338455200195 14.7491340637207\\
71.5338439941406 14.7497701644897\\
71.6338424682617 14.7504043579102\\
71.7338485717773 14.7510395050049\\
71.8338470458984 14.7516746520996\\
71.9338455200195 14.7523097991943\\
72.0338439941406 14.7529449462891\\
72.1338424682617 14.7535753250122\\
72.2338485717773 14.753942489624\\
72.3338470458984 14.7544736862183\\
72.4338455200195 14.7550210952759\\
72.5338439941406 14.7555646896362\\
72.6338424682617 14.7561063766479\\
72.7338485717773 14.756646156311\\
72.8338470458984 14.7571849822998\\
72.9338455200195 14.7577199935913\\
73.0338439941406 14.7582540512085\\
73.1338424682617 14.7587852478027\\
73.2338485717773 14.7593288421631\\
73.3338470458984 14.7599220275879\\
73.4338455200195 14.7605323791504\\
73.5338439941406 14.7611246109009\\
73.6338424682617 14.7617254257202\\
73.7338485717773 14.7623500823975\\
73.8338470458984 14.7630071640015\\
73.9338455200195 14.7637195587158\\
74.0338439941406 14.7660856246948\\
74.1323699951172 14.7728147506714\\
74.2112121582031 14.7766342163086\\
74.2755126953125 14.7681369781494\\
74.3462524414063 14.7682304382324\\
74.4240570068359 14.7686338424683\\
74.509651184082 14.7691287994385\\
74.6037979125977 14.7698411941528\\
74.7037963867188 14.7707529067993\\
74.8037948608398 14.7717819213867\\
74.9019012451172 14.7728242874146\\
75 14.7738656997681\\
};
\addlegendentry{22 "};

\end{axis}
\end{tikzpicture}%
\caption{Filling problem with the top channel in the nonlinear domain.}
\label{fig:vmpDDTopChan}
\end{figure}

\fig{fig:vmpDDTopChan} shows that when the water level is in the bottom channel, the nonphysical solution propagates globally.
In addition, once the water level enters the nonlinear domain, the solution is out of sync with the analytic solution.
This discrepancy is due to the time lag introduced by the linear solver while the water level is in the linear domain.

This problem highlights the impact of unresolved local nonlinearities on the global solution.
Isolating places where the nonlinearities are present and solving those regions with the nonlinear solver allows for a global solution that more accurately solves the nonlinear equations.

%============================================================================
% Capstone problem that how the domain decomposition is useful in a quasi-realistic model.
%============================================================================
\section{Refill Problem}
\label{sect:refillProblem}

The final test problem is meant to reflect a typical LWR scenario.
The model is a simplified two-loop, LWR core model with upper-head safety injection that is going through the refill phase of a simulated accident scenario.
The intent of this refill problem is to bring together the individual assertions from the previous test problems.
In particular, if there are localized nonlinearities, then the use of the domain decomposition algorithm can produce results that are not only more accurate than those produced by the linear solver but are also less computationally expensive than those produced by the nonlinear solver.

\subsection{Model}
\label{sect:refillModel}

The physical basis for the problem is a model of a nuclear reactor vessel.
\fig{fig:refillModel} contains a schematic of the discrete representation of this problem.
There are three sections in the problem.
In the first section there is the inlet pleneum to the core and two hemispheres of the downcomer.
The downcomers are connected to each other and the inlet plenum, as indicated by dotted lines.
The second section contains a three subchannel representation of the reactor core.
One subchannel represents the inner ring of the core, which produces more steam than the outer ring.
Another subchannel represents the outer core ring, which has a lower rate of steam production.
And, a third represents the control rod channels, which are represented by the grey circle and have neither steam production nor hydrodynamic communication with the two core rings.
There is no hydrodynamic communication between the core subchannels and the two downcomer subchannels, as indicated by solid lines.
The third section represents the upper head of the reactor.
In this model, there are transverse flow paths between all adjacent subchannels in the upper head.
In the third section there is also a safety injection inlet at point B, and a pressure connection to containment at point A.

\begin{figure}[h!tb]
\centering
\tikzsetnextfilename{images/refillModel_pdf}
{
\begin{tikzpicture}

%Boundary Lines
\draw (0,12) circle (2 and 1);
\draw (-2,7) arc (180:360:2 and 1);
\draw [dashed] (2,7) arc (0:180:2 and 1);
\draw (-2,2) arc (180:360:2 and 1);
\draw [dashed] (2,2) arc (0:180:2 and 1);
\draw (-2,-3) arc (180:360:2 and 1);
\draw [dashed] (2,-3) arc (0:180:2 and 1);

%Vertical Lines
\draw (-2,-3) -- (-2,12);
\draw (2,-3) -- (2,12);

%Section 1 Flow Pattern
%\draw (0,9.5) circle (2 and 1);
%\draw [dotted] (0,9.5) circle (1.6 and 0.8);
%\draw (0,9) circle (0.4 and 0.2);
%\filldraw [gray!75] (0.5,9.5) circle (2pt);
%\filldraw [black] (-1.8,9.5) circle (2pt);

%Section 3 Flow Pattern
\draw [dotted] (7.25, 11.9) -- (7.25, 9.2);
\draw [dotted] (7.25, 8.6) -- (7.25, 7.1);
\draw [dotted] (7.25, 9.5) circle (0.6);
\draw [dotted] (7.25, 9.5) circle (1.4);
\draw [dashdotted, color=red]  (7.25, 9.5) circle (2.0);
\draw [solid]  (7.25,9.5) circle (2.4);
\filldraw [dotted, fill=white, draw=black] (7.25, 8.9) circle (0.3);
\filldraw [black] ( 8.2, 9.5) circle (2pt) node[anchor=north]{B};
\filldraw [black] (5.05, 9.5) circle (2pt) node[anchor=north]{A};	

%Section 2 Flow Pattern
\draw [solid] (7.25, 4.5) circle (0.6);
\draw [solid] (7.25, 4.5) circle (1.4);
\draw [solid] (7.25, 4.5) circle (2.0);
\draw [solid] (7.25, 4.5) circle (2.4);
\filldraw [fill=white, draw=black] (7.25, 3.9) circle (0.3);
\draw [dotted] (7.25, 6.9) -- (7.25, 6.5);
\draw [dotted] (7.25, 2.1) -- (7.25, 2.5);
\filldraw [black] (7.25, 4.5) circle (2pt) node[anchor=east]{C};	
\filldraw [black] ( 8.2, 4.5) circle (2pt) node[anchor=north]{D};

%Section 1 Flow Pattern
\draw [solid]  (7.25, -0.5) circle (2.4);
\draw [dotted] (7.25, -0.5) circle (2.0);
\draw [dotted] (7.25,  1.9) -- (7.25,  1.5);
\draw [dotted] (7.25, -2.9) -- (7.25, -2.5);

%Arrows & Labels
\draw (2.5,12) -- (3,12);
\draw [<->] (2.75,12) -- (2.75,7);
\draw (2.5,7) -- (3,7);
\draw [<->] (2.75,7) -- (2.75,2);
\draw (2.5,2) -- (3,2);
\draw [<->] (2.75,2) -- (2.75,-3);
\draw (2.5,-3) -- (3,-3);
\foreach \y/\ytext in {-0.5/Section 1,4.5/Section 2,9.5/Section 3}
	\draw (3,\y) node [anchor=west] {\ytext};

%Extras
\end{tikzpicture}
}
\caption{\cobra{} model for the refill problem.}
\label{fig:refillModel}
\end{figure}

The lower plenum and the bottom of the downcomer in section one are initially full of saturated water while both the core region in section two and the upper head in section three are full of saturated steam.
Of note, this model does not have solid fuel structures to produce heat in the reactor core.
The fuel rods were omitted from this work because the evaluation of the impact of the nonlinear solver on wall heat is beyond the scope of this work.
To simulate the presence of the fuel rods, a mass source produces steam within the reactor core.
This source is located in the outer core ring near the top of the middle section.
This steam rises into the upper head and condensation will be caused by the safety injection inlet flow.
The total amount of this condensation will be of interest for this problem.

\subsection{Results}
\label{sect:refillResults}

Like previous problems, this simulation was subject to timestep sensitivity studies.
The set of timestep sizes analyzed is is given by \eqref{eqn:timeStepAlgo}.
For this problem, $\dt{}_{0}$, $r_{f}$, and ${n}$ were 0.02, 2.0, and 4 respectively.
Additionally, the nonlinear convergence parameters \kmax{}, $f_{tol}$, and $d_{tol}$ were 35, \expneg{1.0}{6}, and \expneg{1.0}{8}, respectively.

Since this model was intended to model a core refill, the parameters of interest are related to core cooling.
The primary variable of engineering interest in this problem is the time-dependent behavior of the condensation in the region of the safety injection inlet and the maximum condensation over the course of the transient.

\begin{figure}[h!tb]
\centering
% This file was created by matlab2tikz v0.4.3.
% Copyright (c) 2008--2013, Nico Schlömer <nico.schloemer@gmail.com>
% All rights reserved.
% 
\tikzsetnextfilename{plots/refillGammaLin_eps}
%
% defining custom colors
\definecolor{mycolor1}{rgb}{0,0.75,0.75}%
%
\begin{tikzpicture}

\begin{axis}[%
width=\mytikzpicwidth,
height=\mytikzpicheight,
scale only axis,
xmin=0.0,
xmax=150.0,
xlabel={Time $[\text{s}]$},
ymin=-30.0,
ymax=2.0,
ylabel={$\Gamma\, [\,\lbm{}\,]$},
legend style={at={(0.03,0.03)},anchor=south west,draw=black,fill=white,legend cell align=left}
]
\addplot [
color=blue,
solid
]
table[row sep=crcr]{
0.0 0.0\\
1.01201021671295 0\\
2.01201009750366 0\\
3.01201009750366 0\\
4.01201009750366 0\\
5.01201009750366 0\\
6.01201009750366 0\\
7.01201009750366 0\\
8.01201057434082 0\\
9.01201057434082 0\\
10.0120105743408 0\\
11.0120105743408 0\\
12.0120105743408 0\\
13.0120105743408 0\\
14.0120105743408 0\\
15.0120105743408 0\\
16.0120105743408 0\\
17.0120105743408 0\\
18.0120105743408 0\\
19.0120105743408 0\\
20.0120105743408 -1.21348051607129e-08\\
21.0314502716064 -0.532965123653412\\
22.0433254241943 -0.624448001384735\\
23.0584888458252 -0.658934414386749\\
24.07008934021 -0.683407843112946\\
25.0927181243896 -0.705884695053101\\
26.1346302032471 -0.734544813632965\\
27.1794548034668 -0.771136164665222\\
28.1794548034668 -0.816832304000854\\
29.188533782959 -0.876108884811401\\
30.228401184082 -0.956134438514709\\
31.2366123199463 -1.05676209926605\\
32.2744636535645 -1.18002927303314\\
33.2744636535645 -1.31126856803894\\
34.2976188659668 -1.45138323307037\\
35.3207893371582 -1.59519135951996\\
36.373119354248 -1.74001812934875\\
37.373119354248 -1.87065887451172\\
38.4339485168457 -2.00622224807739\\
39.4667282104492 -2.14011359214783\\
40.5255661010742 -2.28325772285461\\
41.5883941650391 -2.43073511123657\\
42.5883941650391 -2.57034349441528\\
43.5883941650391 -2.70764064788818\\
44.5883941650391 -2.84412050247192\\
45.6154632568359 -2.98709297180176\\
46.6388092041016 -3.12025618553162\\
47.6388092041016 -3.24078631401062\\
48.6482276916504 -3.36742520332336\\
49.6864128112793 -3.49009990692139\\
50.7217788696289 -3.59640502929688\\
51.7511787414551 -3.69487762451172\\
52.7880630493164 -3.79088473320007\\
53.8089599609375 -3.8817036151886\\
54.8371696472168 -3.97048687934875\\
55.8578186035156 -4.057701587677\\
56.8790626525879 -4.14304828643799\\
57.9082298278809 -4.22821474075317\\
58.9352798461914 -4.31396722793579\\
59.966667175293 -4.40267038345337\\
60.9977836608887 -4.49007320404053\\
62.0255928039551 -4.57624340057373\\
63.0614433288574 -4.66685390472412\\
64.0809707641602 -4.757737159729\\
65.1093215942383 -4.85024070739746\\
66.1471328735352 -4.94423151016235\\
67.1665725708008 -5.03737306594849\\
68.1982803344727 -5.13245630264282\\
69.2267456054688 -5.22712898254395\\
70.2057189941406 -5.31623983383179\\
71.2327880859375 -5.41298913955688\\
72.2550582885742 -5.50677824020386\\
73.2859039306641 -5.60302829742432\\
74.3311157226563 -5.70086526870728\\
75.3711395263672 -5.79715299606323\\
76.4078521728516 -5.89410924911499\\
77.4475555419922 -5.99232292175293\\
78.456787109375 -6.08849811553955\\
79.456787109375 -6.18407106399536\\
80.456787109375 -6.27926540374756\\
81.456787109375 -6.37464427947998\\
82.456787109375 -6.46886014938354\\
83.4757308959961 -6.56578826904297\\
84.4836273193359 -6.6608772277832\\
85.4836273193359 -6.75448751449585\\
86.4836273193359 -6.84781312942505\\
87.4836273193359 -6.94193983078003\\
88.4836273193359 -7.03611850738525\\
89.4836273193359 -7.12886190414429\\
90.4836273193359 -7.22113847732544\\
91.4836273193359 -7.31396102905273\\
92.4836273193359 -7.40770101547241\\
93.5179214477539 -7.5047082901001\\
94.5476379394531 -7.60030603408813\\
95.5800857543945 -7.69130611419678\\
96.6179428100586 -7.77754640579224\\
97.6640930175781 -7.86106109619141\\
98.6932067871094 -7.94069290161133\\
99.733642578125 -8.01907730102539\\
100.068885803223 -8.04431247711182\\
100.162666320801 -8.05110549926758\\
100.245223999023 -8.05701446533203\\
100.344909667969 -8.06423759460449\\
100.44490814209 -8.07159328460693\\
100.544906616211 -8.07897186279297\\
100.644905090332 -8.08636379241943\\
100.744903564453 -8.09387111663818\\
100.844909667969 -8.10142421722412\\
100.94490814209 -8.10893154144287\\
101.044906616211 -8.11630153656006\\
101.144905090332 -8.12347888946533\\
101.244903564453 -8.13053131103516\\
101.344909667969 -8.13754940032959\\
101.44490814209 -8.14458560943604\\
101.544906616211 -8.15168952941895\\
101.644905090332 -8.15887451171875\\
101.744903564453 -8.16615772247314\\
101.844909667969 -8.17348766326904\\
101.94490814209 -8.18079280853271\\
102.044906616211 -8.18797588348389\\
102.144905090332 -8.19498729705811\\
102.244903564453 -8.2018871307373\\
102.344909667969 -8.20873737335205\\
102.44490814209 -8.21558380126953\\
102.544906616211 -8.22246932983398\\
102.644905090332 -8.22942447662354\\
102.744903564453 -8.23647880554199\\
102.844909667969 -8.24362754821777\\
102.94490814209 -8.25074195861816\\
103.044906616211 -8.25774669647217\\
103.144905090332 -8.26461410522461\\
103.244903564453 -8.27137565612793\\
103.344909667969 -8.27809810638428\\
103.44490814209 -8.28481864929199\\
103.544906616211 -8.29156970977783\\
103.644905090332 -8.29837799072266\\
103.744903564453 -8.30528545379639\\
103.844909667969 -8.31225967407227\\
103.94490814209 -8.31924057006836\\
104.044906616211 -8.32612228393555\\
104.144905090332 -8.33288955688477\\
104.244903564453 -8.33957672119141\\
104.344909667969 -8.34620952606201\\
104.44490814209 -8.35282802581787\\
104.544906616211 -8.35945701599121\\
104.644905090332 -8.36612606048584\\
104.744903564453 -8.37287902832031\\
104.844909667969 -8.37971687316895\\
104.94490814209 -8.38655948638916\\
105.044906616211 -8.39333724975586\\
105.144905090332 -8.40001392364502\\
105.244903564453 -8.4066162109375\\
105.344909667969 -8.41316032409668\\
105.44490814209 -8.41967391967773\\
105.544906616211 -8.42617893218994\\
105.644905090332 -8.43270492553711\\
105.744903564453 -8.43931770324707\\
105.844909667969 -8.44601726531982\\
105.94490814209 -8.45275211334229\\
106.044906616211 -8.45943641662598\\
106.144905090332 -8.46602725982666\\
106.244903564453 -8.47254276275635\\
106.344909667969 -8.47900581359863\\
106.44490814209 -8.48545360565186\\
106.544906616211 -8.4918966293335\\
106.644905090332 -8.49832248687744\\
106.744903564453 -8.50478172302246\\
106.844909667969 -8.5112886428833\\
106.94490814209 -8.51785278320313\\
107.044906616211 -8.52444839477539\\
107.144905090332 -8.53103160858154\\
107.244903564453 -8.53755569458008\\
107.344909667969 -8.54402923583984\\
107.44490814209 -8.55047130584717\\
107.544906616211 -8.55686378479004\\
107.644905090332 -8.56322479248047\\
107.744903564453 -8.56958293914795\\
107.844909667969 -8.57598400115967\\
107.94490814209 -8.58244228363037\\
108.044906616211 -8.58894824981689\\
108.144905090332 -8.59545993804932\\
108.244903564453 -8.60195064544678\\
108.344909667969 -8.6083927154541\\
108.44490814209 -8.61477661132813\\
108.544906616211 -8.62110996246338\\
108.644905090332 -8.62740707397461\\
108.744903564453 -8.63369274139404\\
108.844909667969 -8.6400032043457\\
108.94490814209 -8.64633941650391\\
109.044906616211 -8.65272903442383\\
109.144905090332 -8.65915679931641\\
109.244903564453 -8.66556835174561\\
109.344909667969 -8.67193412780762\\
109.44490814209 -8.6782398223877\\
109.544906616211 -8.68450164794922\\
109.644905090332 -8.69075107574463\\
109.744903564453 -8.69699001312256\\
109.844909667969 -8.70324897766113\\
109.94490814209 -8.709547996521\\
110.044906616211 -8.71589088439941\\
110.144905090332 -8.72226619720459\\
110.244903564453 -8.72863674163818\\
110.344909667969 -8.73496437072754\\
110.44490814209 -8.74126052856445\\
110.544906616211 -8.74752902984619\\
110.644905090332 -8.75374603271484\\
110.744903564453 -8.75992488861084\\
110.844909667969 -8.76608753204346\\
110.94490814209 -8.77225017547607\\
111.044906616211 -8.77840614318848\\
111.144905090332 -8.7845344543457\\
111.244903564453 -8.79062938690186\\
111.344909667969 -8.79670143127441\\
111.44490814209 -8.80277442932129\\
111.544906616211 -8.80883502960205\\
111.644905090332 -8.8148832321167\\
111.744903564453 -8.82091522216797\\
111.844909667969 -8.82693195343018\\
111.94490814209 -8.83291625976563\\
112.044906616211 -8.83888053894043\\
112.144905090332 -8.84481620788574\\
112.244903564453 -8.85072422027588\\
112.344909667969 -8.85660934448242\\
112.44490814209 -8.86246871948242\\
112.544906616211 -8.86830806732178\\
112.644905090332 -8.87412548065186\\
112.744903564453 -8.87991142272949\\
112.844909667969 -8.88565826416016\\
112.94490814209 -8.89136219024658\\
113.044906616211 -8.89703941345215\\
113.144905090332 -8.90271472930908\\
113.244903564453 -8.90837955474854\\
113.344909667969 -8.91405487060547\\
113.44490814209 -8.91973209381104\\
113.544906616211 -8.9254207611084\\
113.644905090332 -8.93110179901123\\
113.744903564453 -8.93678283691406\\
113.844909667969 -8.94244575500488\\
113.94490814209 -8.94808006286621\\
114.044906616211 -8.95369052886963\\
114.144905090332 -8.95925331115723\\
114.244903564453 -8.96476268768311\\
114.344909667969 -8.97022914886475\\
114.44490814209 -8.97566032409668\\
114.544906616211 -8.98108196258545\\
114.644905090332 -8.98652744293213\\
114.744903564453 -8.99200248718262\\
114.844909667969 -8.99748516082764\\
114.94490814209 -9.00294208526611\\
115.044906616211 -9.00838851928711\\
115.144905090332 -9.0137996673584\\
115.244903564453 -9.01916313171387\\
115.344909667969 -9.02449035644531\\
115.44490814209 -9.02976608276367\\
115.544906616211 -9.03502178192139\\
115.644905090332 -9.04026699066162\\
115.744903564453 -9.04550457000732\\
115.844909667969 -9.05075359344482\\
115.94490814209 -9.05601024627686\\
116.044906616211 -9.06126403808594\\
116.144905090332 -9.06650352478027\\
116.244903564453 -9.07172012329102\\
116.344909667969 -9.07688617706299\\
116.44490814209 -9.08200454711914\\
116.544906616211 -9.08708667755127\\
116.644905090332 -9.09213829040527\\
116.744903564453 -9.09717750549316\\
116.844909667969 -9.10221385955811\\
116.94490814209 -9.107253074646\\
117.044906616211 -9.1123104095459\\
117.144905090332 -9.11738967895508\\
117.244903564453 -9.12246131896973\\
117.344909667969 -9.12751483917236\\
117.44490814209 -9.13254165649414\\
117.544906616211 -9.13753795623779\\
117.644905090332 -9.14249324798584\\
117.744903564453 -9.14741516113281\\
117.844909667969 -9.15230941772461\\
117.94490814209 -9.15718173980713\\
118.044906616211 -9.16203594207764\\
118.144905090332 -9.16689872741699\\
118.244903564453 -9.17176914215088\\
118.344909667969 -9.17664337158203\\
118.44490814209 -9.18149948120117\\
118.544906616211 -9.1863489151001\\
118.644905090332 -9.19117736816406\\
118.744903564453 -9.195969581604\\
118.844909667969 -9.20071601867676\\
118.94490814209 -9.20541858673096\\
119.044906616211 -9.21008014678955\\
119.144905090332 -9.21472549438477\\
119.244903564453 -9.21936702728271\\
119.344909667969 -9.22399044036865\\
119.44490814209 -9.22862243652344\\
119.544906616211 -9.23326587677002\\
119.644905090332 -9.23788738250732\\
119.744903564453 -9.24247550964355\\
119.844909667969 -9.24703788757324\\
119.94490814209 -9.25155735015869\\
120.044906616211 -9.25603008270264\\
120.144905090332 -9.26047420501709\\
120.244903564453 -9.26487922668457\\
120.344909667969 -9.26927471160889\\
120.44490814209 -9.27366733551025\\
120.544906616211 -9.27808094024658\\
120.644905090332 -9.28250694274902\\
120.744903564453 -9.28691864013672\\
120.844909667969 -9.29131412506104\\
120.94490814209 -9.29566287994385\\
121.044906616211 -9.29996109008789\\
121.144905090332 -9.30422782897949\\
121.244903564453 -9.30842399597168\\
121.344909667969 -9.31259059906006\\
121.44490814209 -9.316725730896\\
121.544906616211 -9.32086753845215\\
121.644905090332 -9.32502555847168\\
121.744903564453 -9.32920932769775\\
121.844909667969 -9.33339500427246\\
121.94490814209 -9.33754730224609\\
122.044906616211 -9.34164905548096\\
122.144905090332 -9.34569072723389\\
122.244903564453 -9.34967613220215\\
122.344909667969 -9.35362148284912\\
122.44490814209 -9.35754776000977\\
122.544906616211 -9.36149501800537\\
122.644905090332 -9.36543464660645\\
122.744903564453 -9.36939239501953\\
122.844909667969 -9.37337017059326\\
122.94490814209 -9.37734699249268\\
123.044906616211 -9.38130664825439\\
123.144905090332 -9.38524341583252\\
123.244903564453 -9.38914775848389\\
123.344909667969 -9.39305877685547\\
123.44490814209 -9.39695358276367\\
123.544906616211 -9.40082168579102\\
123.644905090332 -9.40467834472656\\
123.744903564453 -9.40853404998779\\
123.844909667969 -9.41241455078125\\
123.94490814209 -9.41629981994629\\
124.044906616211 -9.42017364501953\\
124.144905090332 -9.42401218414307\\
124.244903564453 -9.42781639099121\\
124.344909667969 -9.43155288696289\\
124.44490814209 -9.43522453308105\\
124.544906616211 -9.43886184692383\\
124.644905090332 -9.44250011444092\\
124.744903564453 -9.44605731964111\\
124.844909667969 -9.4495325088501\\
124.94490814209 -9.45297050476074\\
125.044906616211 -9.45639038085938\\
125.144905090332 -9.45980453491211\\
125.244903564453 -9.46322059631348\\
125.344909667969 -9.46660137176514\\
125.44490814209 -9.46989345550537\\
125.544906616211 -9.47311115264893\\
125.644905090332 -9.47624111175537\\
125.744903564453 -9.47931957244873\\
125.844909667969 -9.48237991333008\\
125.94490814209 -9.48541164398193\\
126.044906616211 -9.48842239379883\\
126.144905090332 -9.49140357971191\\
126.244903564453 -9.49435424804688\\
126.344909667969 -9.49726867675781\\
126.44490814209 -9.50009727478027\\
126.544906616211 -9.50284671783447\\
126.644905090332 -9.50552749633789\\
126.744903564453 -9.50818347930908\\
126.842742919922 -9.51076984405518\\
126.942741394043 -9.513427734375\\
127.042739868164 -9.51607227325439\\
127.142738342285 -9.51869487762451\\
127.237205505371 -9.5211238861084\\
127.333419799805 -9.52352809906006\\
127.397598266602 -9.52509593963623\\
127.470085144043 -9.52684688568115\\
127.528678894043 -9.52824878692627\\
127.594779968262 -9.52982139587402\\
127.694778442383 -9.53222751617432\\
127.793579101563 -9.5345516204834\\
127.893264770508 -9.5368824005127\\
127.993263244629 -9.53921127319336\\
128.09326171875 -9.54152870178223\\
128.193267822266 -9.54382991790771\\
128.293258666992 -9.54610633850098\\
128.387588500977 -9.54824447631836\\
128.47526550293 -9.55024147033691\\
128.564590454102 -9.55222511291504\\
128.638717651367 -9.55383682250977\\
128.715942382813 -9.55549335479736\\
128.789047241211 -9.55704593658447\\
128.864959716797 -9.55865955352783\\
128.958312988281 -9.56064796447754\\
129.043411254883 -9.5624828338623\\
129.135528564453 -9.56448841094971\\
129.234756469727 -9.5666389465332\\
129.333938598633 -9.56877040863037\\
129.430023193359 -9.57081317901611\\
129.524444580078 -9.57282066345215\\
129.624435424805 -9.57506847381592\\
129.72444152832 -9.57717990875244\\
129.823654174805 -9.57913208007813\\
129.876678466797 -9.58017826080322\\
129.973388671875 -9.58209037780762\\
130.073379516602 -9.58408832550049\\
130.173385620117 -9.58623027801514\\
130.273376464844 -9.58851623535156\\
130.373382568359 -9.59074592590332\\
130.473388671875 -9.59281063079834\\
130.571273803711 -9.59469032287598\\
130.663940429688 -9.59640407562256\\
130.748779296875 -9.59796619415283\\
130.847290039063 -9.59982204437256\\
130.947280883789 -9.6018238067627\\
131.047286987305 -9.60394191741943\\
131.14729309082 -9.60608959197998\\
131.247283935547 -9.60818767547607\\
131.347290039063 -9.61018657684326\\
131.447280883789 -9.6121072769165\\
131.547286987305 -9.61400699615479\\
131.64729309082 -9.61594009399414\\
131.747283935547 -9.61795711517334\\
131.847290039063 -9.62005424499512\\
131.947280883789 -9.62217330932617\\
132.047286987305 -9.62425804138184\\
132.14729309082 -9.62651443481445\\
132.247283935547 -9.62888717651367\\
132.347290039063 -9.63402652740479\\
132.426193237305 -9.76993560791016\\
132.49235534668 -10.6432313919067\\
132.522872924805 -10.9557418823242\\
132.553268432617 -11.178539276123\\
132.589706420898 -11.3824663162231\\
132.633071899414 -11.5726776123047\\
132.693908691406 -20.1003818511963\\
132.714920043945 -20.3567581176758\\
132.732955932617 -20.555061340332\\
132.763000488281 -21.2921733856201\\
132.777435302734 -21.3285465240479\\
132.807403564453 -21.348705291748\\
132.865142822266 -21.3856678009033\\
132.946960449219 -21.571475982666\\
132.978622436523 -21.6475391387939\\
133.035949707031 -22.6605281829834\\
133.065948486328 -22.9387683868408\\
133.083038330078 -23.0525093078613\\
133.127380371094 -23.8359794616699\\
133.22087097168 -23.9987106323242\\
133.275665283203 -24.1917934417725\\
133.310455322266 -25.1708164215088\\
133.386871337891 -25.4841251373291\\
133.45475769043 -25.5084037780762\\
133.463104248047 -25.6802005767822\\
133.484024047852 -26.4167957305908\\
133.517013549805 -27.2719573974609\\
133.568054199219 -27.7294616699219\\
133.632583618164 -27.7337970733643\\
133.68913269043 -27.7334003448486\\
133.709533691406 -27.7344455718994\\
133.762451171875 -27.7402534484863\\
133.852386474609 -27.8401679992676\\
133.918258666992 -28.6140575408936\\
133.967834472656 -28.6380748748779\\
134.031677246094 -28.6354732513428\\
134.079193115234 -28.6357345581055\\
134.136657714844 -28.6365528106689\\
134.219955444336 -28.6519393920898\\
134.319961547852 -28.6559925079346\\
134.419967651367 -28.6554622650146\\
134.519958496094 -28.6571273803711\\
134.619964599609 -28.6582832336426\\
134.719955444336 -28.6619338989258\\
134.819961547852 -28.6695461273193\\
134.919967651367 -28.6784400939941\\
135.019958496094 -28.6950817108154\\
135.119964599609 -28.7089385986328\\
135.196380615234 -28.7586040496826\\
135.207015991211 -28.9995498657227\\
135.209915161133 -29.0314769744873\\
135.212783813477 -29.0614547729492\\
135.2158203125 -29.0910968780518\\
135.22248840332 -29.1183395385742\\
135.244247436523 -29.1266593933105\\
135.295700073242 -29.1305084228516\\
135.384262084961 -29.1338195800781\\
135.484252929688 -29.1365032196045\\
135.584259033203 -29.1392707824707\\
135.68424987793 -29.141773223877\\
135.784255981445 -29.1439399719238\\
135.843490600586 -29.1450023651123\\
135.881881713867 -29.1457958221436\\
135.912017822266 -29.1464366912842\\
135.942184448242 -29.1470413208008\\
135.970932006836 -29.1475963592529\\
135.998947143555 -29.1481418609619\\
136.026748657227 -29.148681640625\\
136.054885864258 -29.1492233276367\\
136.082855224609 -29.1497325897217\\
136.11083984375 -29.1502285003662\\
136.139221191406 -29.150728225708\\
136.168060302734 -29.1512336730957\\
136.197250366211 -29.151741027832\\
136.226577758789 -29.1522464752197\\
136.255920410156 -29.1527538299561\\
136.285263061523 -29.1532611846924\\
136.314575195313 -29.1537666320801\\
136.343872070313 -29.1542720794678\\
136.373153686523 -29.15478515625\\
136.402404785156 -29.1553115844727\\
136.431716918945 -29.1558513641357\\
136.461120605469 -29.1564064025879\\
136.490646362305 -29.1569843292236\\
136.520278930664 -29.1575946807861\\
136.550003051758 -29.1582164764404\\
136.579803466797 -29.1588554382324\\
136.609619140625 -29.1594982147217\\
136.639526367188 -29.1602039337158\\
136.669509887695 -29.160961151123\\
136.697875976563 -29.1616840362549\\
136.718032836914 -29.1622619628906\\
136.737640380859 -29.16286277771\\
136.758010864258 -29.1634654998779\\
136.778854370117 -29.164083480835\\
136.815048217773 -29.1651821136475\\
136.900497436523 -29.1679363250732\\
137.000503540039 -29.17112159729\\
137.070922851563 -29.1742515563965\\
137.089157104492 -29.1749820709229\\
137.109191894531 -29.1758880615234\\
137.14697265625 -29.1777725219727\\
137.172027587891 -29.1790924072266\\
137.193786621094 -29.180233001709\\
137.237945556641 -29.1825084686279\\
137.262603759766 -29.1838665008545\\
137.287475585938 -29.1850624084473\\
137.347274780273 -29.1868591308594\\
137.408355712891 -29.1889762878418\\
137.442123413086 -29.1902751922607\\
137.473175048828 -29.1914386749268\\
137.5048828125 -29.1925945281982\\
137.536727905273 -29.1937160491943\\
137.568283081055 -29.1947898864746\\
137.600036621094 -29.1958065032959\\
137.6318359375 -29.1967868804932\\
137.663696289063 -29.1977615356445\\
137.685424804688 -29.1984252929688\\
137.705291748047 -29.1990413665771\\
137.731048583984 -29.1998481750488\\
137.762222290039 -29.2008190155029\\
137.788970947266 -29.2016296386719\\
137.820404052734 -29.2025814056396\\
137.851760864258 -29.2035179138184\\
137.884460449219 -29.2044811248779\\
137.91682434082 -29.2054023742676\\
137.951049804688 -29.2063255310059\\
137.985107421875 -29.207202911377\\
138.020965576172 -29.2081146240234\\
138.056289672852 -29.2089939117432\\
138.093399047852 -29.2098922729492\\
138.129959106445 -29.2108001708984\\
138.176620483398 -29.2119693756104\\
138.221237182617 -29.2130470275879\\
138.264984130859 -29.2140712738037\\
138.306991577148 -29.214958190918\\
138.348876953125 -29.2156887054443\\
138.390747070313 -29.2164134979248\\
138.42936706543 -29.2172355651855\\
138.465698242188 -29.2181034088135\\
138.501831054688 -29.2189826965332\\
138.539566040039 -29.2199420928955\\
138.580383300781 -29.2209186553955\\
138.617630004883 -29.2217578887939\\
138.652969360352 -29.2225761413574\\
138.686859130859 -29.2233772277832\\
138.719360351563 -29.224142074585\\
138.75080871582 -29.2248516082764\\
138.781829833984 -29.2255477905273\\
138.812438964844 -29.226261138916\\
138.842788696289 -29.2270221710205\\
138.873001098633 -29.227897644043\\
138.903350830078 -29.2290668487549\\
138.933731079102 -29.2303943634033\\
138.964157104492 -29.2317981719971\\
138.994567871094 -29.2328910827637\\
139.025024414063 -29.2339038848877\\
139.055328369141 -29.2351512908936\\
139.085464477539 -29.2366886138916\\
139.115234375 -29.2385540008545\\
139.144638061523 -29.2407932281494\\
139.173675537109 -29.2432918548584\\
139.202484130859 -29.2458381652832\\
139.231552124023 -29.2481155395508\\
139.254531860352 -29.2493362426758\\
139.279418945313 -29.2505435943604\\
139.307189941406 -29.2520389556885\\
139.378570556641 -29.2568550109863\\
139.478561401367 -29.2636642456055\\
139.578567504883 -29.2698764801025\\
139.678558349609 -29.2760543823242\\
139.778564453125 -29.2822818756104\\
139.878570556641 -29.2891941070557\\
139.978561401367 -29.2955284118652\\
140.078567504883 -29.302417755127\\
140.178558349609 -29.3090572357178\\
140.278564453125 -29.3155765533447\\
140.378570556641 -29.3220767974854\\
140.478561401367 -29.3282432556152\\
140.578567504883 -29.3341846466064\\
140.678558349609 -29.339864730835\\
140.778564453125 -29.3452415466309\\
140.878570556641 -29.3503513336182\\
140.971420288086 -29.3548679351807\\
141.021377563477 -29.3572120666504\\
141.118301391602 -29.3616333007813\\
141.195297241211 -29.3650074005127\\
141.295303344727 -29.3692302703857\\
141.395309448242 -29.3732624053955\\
141.495300292969 -29.3771381378174\\
141.595306396484 -29.3809394836426\\
141.695297241211 -29.3844509124756\\
141.795303344727 -29.3875064849854\\
141.895309448242 -29.3902282714844\\
141.995300292969 -29.3926124572754\\
142.095306396484 -29.3941688537598\\
142.195297241211 -29.3957843780518\\
142.295303344727 -29.3979873657227\\
142.395309448242 -29.400936126709\\
142.495300292969 -29.40260887146\\
142.595306396484 -29.4047203063965\\
142.695297241211 -29.4067039489746\\
142.795303344727 -29.4080448150635\\
142.895309448242 -29.4095249176025\\
142.995300292969 -29.4110527038574\\
143.095306396484 -29.4126605987549\\
143.195297241211 -29.4140129089355\\
143.295303344727 -29.4158592224121\\
143.395309448242 -29.4175319671631\\
143.495300292969 -29.4190483093262\\
143.595306396484 -29.4204406738281\\
143.695297241211 -29.4217357635498\\
143.795303344727 -29.4231586456299\\
143.895309448242 -29.4247379302979\\
143.995300292969 -29.427562713623\\
144.095306396484 -29.432466506958\\
144.195297241211 -29.4377136230469\\
144.295303344727 -29.4443798065186\\
144.395309448242 -29.4507331848145\\
144.495300292969 -29.4576606750488\\
144.595306396484 -29.4626445770264\\
144.695297241211 -29.464448928833\\
144.795303344727 -29.4664001464844\\
144.895309448242 -29.4687747955322\\
144.995300292969 -29.4705600738525\\
145.095306396484 -29.4724922180176\\
145.195297241211 -29.4759922027588\\
145.295303344727 -29.4808235168457\\
145.395309448242 -29.485689163208\\
145.495300292969 -29.4908447265625\\
145.595306396484 -29.4966011047363\\
145.695297241211 -29.5019111633301\\
145.795303344727 -29.5062694549561\\
145.895309448242 -29.5107097625732\\
145.995300292969 -29.5159587860107\\
146.095306396484 -29.5210494995117\\
146.195297241211 -29.5259876251221\\
146.295303344727 -29.5310459136963\\
146.395309448242 -29.5358943939209\\
146.495300292969 -29.5406036376953\\
146.595306396484 -29.5451278686523\\
146.695297241211 -29.5494899749756\\
146.795303344727 -29.5536670684814\\
146.895309448242 -29.5576133728027\\
146.995300292969 -29.5613269805908\\
147.095306396484 -29.564811706543\\
147.195297241211 -29.5680980682373\\
147.295303344727 -29.5712013244629\\
147.395309448242 -29.5741424560547\\
147.495300292969 -29.57692527771\\
147.595306396484 -29.5795574188232\\
147.695297241211 -29.5820350646973\\
147.795303344727 -29.584342956543\\
147.895309448242 -29.5865020751953\\
147.995300292969 -29.5885581970215\\
148.095306396484 -29.5905456542969\\
148.195297241211 -29.5924682617188\\
148.295303344727 -29.5943336486816\\
148.395309448242 -29.5961208343506\\
148.495300292969 -29.5978336334229\\
148.595306396484 -29.5994758605957\\
148.695297241211 -29.601053237915\\
148.795303344727 -29.6025791168213\\
148.895309448242 -29.6040649414063\\
148.995300292969 -29.6055297851563\\
149.095306396484 -29.6069774627686\\
149.195297241211 -29.6084041595459\\
149.295303344727 -29.6098003387451\\
149.395309448242 -29.6111335754395\\
149.495300292969 -29.6123962402344\\
149.595306396484 -29.6135845184326\\
149.695297241211 -29.6147003173828\\
149.795303344727 -29.6157531738281\\
149.895309448242 -29.6167545318604\\
149.992645263672 -29.6177024841309\\
};
\addlegendentry{$\dtmax{} = \expneg{1.0}{2}{[s]}$};

\addplot [
color=green!50!black,
solid
]
table[row sep=crcr]{
0 0\\
1.00349032878876 0\\
2.00349020957947 0\\
3.00349020957947 0\\
4.00349044799805 0\\
5.00349044799805 0\\
6.00349044799805 0\\
7.00349044799805 0\\
8.00349044799805 0\\
9.00349044799805 0\\
10.003490447998 0\\
11.003490447998 0\\
12.003490447998 0\\
13.003490447998 0\\
14.003490447998 0\\
15.003490447998 0\\
16.003490447998 0\\
17.003490447998 0\\
18.003490447998 0\\
19.003490447998 0\\
20.003490447998 -2.73030849484712e-07\\
21.0106086730957 -0.540697038173676\\
22.0216331481934 -0.642518162727356\\
23.0372295379639 -0.676991403102875\\
24.0441570281982 -0.701713085174561\\
25.0543155670166 -0.723991513252258\\
26.0543155670166 -0.749794900417328\\
27.0543155670166 -0.782914578914642\\
28.0543155670166 -0.825319349765778\\
29.0543155670166 -0.879932165145874\\
30.0543155670166 -0.950437903404236\\
31.0543155670166 -1.04184293746948\\
32.0543174743652 -1.15105020999908\\
33.0543174743652 -1.26922833919525\\
34.0543174743652 -1.39405059814453\\
35.0043144226074 -1.51563405990601\\
36.0043144226074 -1.64204668998718\\
37.0043144226074 -1.76512444019318\\
38.0043144226074 -1.88360321521759\\
39.0043144226074 -2.0022988319397\\
40.0043144226074 -2.12198328971863\\
41.0043144226074 -2.24313473701477\\
42.0043144226074 -2.36421632766724\\
43.0043144226074 -2.4872133731842\\
44.0043144226074 -2.60979247093201\\
45.0043144226074 -2.73429942131042\\
46.010124206543 -2.85383200645447\\
47.010124206543 -2.96735715866089\\
48.010124206543 -3.07932090759277\\
49.010124206543 -3.18865132331848\\
50.010124206543 -3.29776787757874\\
51.0248336791992 -3.40574336051941\\
52.0329208374023 -3.50847125053406\\
53.0473403930664 -3.59802508354187\\
54.0686645507813 -3.68802165985107\\
55.0859222412109 -3.7742714881897\\
56.1051292419434 -3.85979676246643\\
57.1237754821777 -3.94434309005737\\
58.1356887817383 -4.02810573577881\\
59.1512565612793 -4.11327123641968\\
60.1731109619141 -4.2017240524292\\
61.1811447143555 -4.29140758514404\\
62.2059059143066 -4.38262987136841\\
63.2253608703613 -4.47213554382324\\
64.2454605102539 -4.56157255172729\\
65.2518615722656 -4.65173816680908\\
66.2753295898438 -4.74429225921631\\
67.2869567871094 -4.83678531646729\\
68.304084777832 -4.93041276931763\\
69.3234558105469 -5.02512216567993\\
70.3017196655273 -5.11530160903931\\
71.318489074707 -5.20823192596436\\
72.3318710327148 -5.3030104637146\\
73.3434066772461 -5.39947605133057\\
74.3449249267578 -5.49222040176392\\
75.3493804931641 -5.58443593978882\\
76.3493804931641 -5.67713737487793\\
77.3493804931641 -5.77146816253662\\
78.3493804931641 -5.86557912826538\\
79.3493804931641 -5.96419715881348\\
80.3493804931641 -6.06363391876221\\
81.3493804931641 -6.1623649597168\\
82.3493804931641 -6.26070880889893\\
83.3493804931641 -6.35822534561157\\
84.3493804931641 -6.45623540878296\\
85.3493804931641 -6.55272817611694\\
86.3493804931641 -6.6473183631897\\
87.3493804931641 -6.73946762084961\\
88.3493804931641 -6.83043670654297\\
89.3493804931641 -6.92046308517456\\
90.3493804931641 -7.00965166091919\\
91.3493804931641 -7.09882736206055\\
92.3493804931641 -7.18795108795166\\
93.3493804931641 -7.27720546722412\\
94.354377746582 -7.36481094360352\\
95.354377746582 -7.44804191589355\\
96.354377746582 -7.52755546569824\\
97.354377746582 -7.60391283035278\\
98.354377746582 -7.67739152908325\\
99.354377746582 -7.74849462509155\\
100.010231018066 -7.7939510345459\\
100.058853149414 -7.79731035232544\\
100.108856201172 -7.80073261260986\\
100.158851623535 -7.80417728424072\\
100.208854675293 -7.80767011642456\\
100.258857727051 -7.8111457824707\\
100.308853149414 -7.81457281112671\\
100.358856201172 -7.81794214248657\\
100.408851623535 -7.82126379013062\\
100.458854675293 -7.8245701789856\\
100.508857727051 -7.82789516448975\\
100.558853149414 -7.83126354217529\\
100.608856201172 -7.8346586227417\\
100.658851623535 -7.83806991577148\\
100.708854675293 -7.84149122238159\\
100.758857727051 -7.84491109848022\\
100.808853149414 -7.84832382202148\\
100.858856201172 -7.85173082351685\\
100.908851623535 -7.85513830184937\\
100.958854675293 -7.85855054855347\\
101.008857727051 -7.86196327209473\\
101.058853149414 -7.8653736114502\\
101.108856201172 -7.86876249313354\\
101.158851623535 -7.8721604347229\\
101.208854675293 -7.87559843063354\\
101.258857727051 -7.87904214859009\\
101.308853149414 -7.88245248794556\\
101.358856201172 -7.88581371307373\\
101.408851623535 -7.88912630081177\\
101.458854675293 -7.89239501953125\\
101.508857727051 -7.8956446647644\\
101.558853149414 -7.89890718460083\\
101.608856201172 -7.9022216796875\\
101.658851623535 -7.90560913085938\\
101.708854675293 -7.90902423858643\\
101.758857727051 -7.9124436378479\\
101.808853149414 -7.91585111618042\\
101.858856201172 -7.91925144195557\\
101.908851623535 -7.92264795303345\\
101.958854675293 -7.92604827880859\\
102.008857727051 -7.92945432662964\\
102.058853149414 -7.93286991119385\\
102.108856201172 -7.93628311157227\\
102.158851623535 -7.93967485427856\\
102.208854675293 -7.94311714172363\\
102.258857727051 -7.94654655456543\\
102.308853149414 -7.94994497299194\\
102.358856201172 -7.95329856872559\\
102.408851623535 -7.95659732818604\\
102.458854675293 -7.9598560333252\\
102.508857727051 -7.96307992935181\\
102.558853149414 -7.96629285812378\\
102.608856201172 -7.9695143699646\\
102.658851623535 -7.97275590896606\\
102.708854675293 -7.97602415084839\\
102.758857727051 -7.97931337356567\\
102.808853149414 -7.98262023925781\\
102.858856201172 -7.98594093322754\\
102.908851623535 -7.98928022384644\\
102.958854675293 -7.99263095855713\\
103.008857727051 -7.99598979949951\\
103.058853149414 -7.99934816360474\\
103.108856201172 -8.00270080566406\\
103.158851623535 -8.00610256195068\\
103.208854675293 -8.0095043182373\\
103.258857727051 -8.01288509368896\\
103.308853149414 -8.01624011993408\\
103.358856201172 -8.01955032348633\\
103.408851623535 -8.0228099822998\\
103.458854675293 -8.02602577209473\\
103.508857727051 -8.02921199798584\\
103.558853149414 -8.03237533569336\\
103.608856201172 -8.03553581237793\\
103.658851623535 -8.0386962890625\\
103.708854675293 -8.04186153411865\\
103.758857727051 -8.04504108428955\\
103.808853149414 -8.04823303222656\\
103.858856201172 -8.05143928527832\\
103.908851623535 -8.05466938018799\\
103.958854675293 -8.0579252243042\\
104.008857727051 -8.061203956604\\
104.058853149414 -8.06449508666992\\
104.108856201172 -8.06779861450195\\
104.158851623535 -8.07111549377441\\
104.208854675293 -8.07441234588623\\
104.258857727051 -8.07769966125488\\
104.308853149414 -8.08096218109131\\
104.358856201172 -8.08418941497803\\
104.408851623535 -8.08737468719482\\
104.458854675293 -8.09053611755371\\
104.508857727051 -8.09367275238037\\
104.558853149414 -8.09678745269775\\
104.608856201172 -8.09988880157471\\
104.658851623535 -8.10298824310303\\
104.708854675293 -8.10608959197998\\
104.758857727051 -8.10919952392578\\
104.808853149414 -8.11231422424316\\
104.858856201172 -8.11544132232666\\
104.908851623535 -8.11858749389648\\
104.958854675293 -8.12176036834717\\
105.008857727051 -8.12495517730713\\
105.058853149414 -8.12817668914795\\
105.108856201172 -8.1314115524292\\
105.158851623535 -8.13465404510498\\
105.208854675293 -8.13790416717529\\
105.258857727051 -8.14113903045654\\
105.308853149414 -8.14435195922852\\
105.358856201172 -8.14752769470215\\
105.408851623535 -8.15067481994629\\
105.458854675293 -8.15380096435547\\
105.508857727051 -8.15691089630127\\
105.558853149414 -8.1600170135498\\
105.608856201172 -8.16311168670654\\
105.658851623535 -8.16619396209717\\
105.708854675293 -8.16926383972168\\
105.758857727051 -8.17232894897461\\
105.808853149414 -8.17539119720459\\
105.858856201172 -8.17845630645752\\
105.908851623535 -8.18152904510498\\
105.958854675293 -8.18460655212402\\
106.008857727051 -8.18770313262939\\
106.058853149414 -8.19081878662109\\
106.108856201172 -8.19396114349365\\
106.158851623535 -8.19711875915527\\
106.208854675293 -8.20029258728027\\
106.258857727051 -8.20347785949707\\
106.308853149414 -8.2066535949707\\
106.358856201172 -8.20979404449463\\
106.408851623535 -8.21291732788086\\
106.458854675293 -8.21602916717529\\
106.508857727051 -8.21913433074951\\
106.558853149414 -8.22222423553467\\
106.608856201172 -8.22530746459961\\
106.658851623535 -8.22837734222412\\
106.708854675293 -8.23143577575684\\
106.758857727051 -8.23447895050049\\
106.808853149414 -8.23751258850098\\
106.858856201172 -8.24053859710693\\
106.908851623535 -8.24355220794678\\
106.958854675293 -8.24656581878662\\
107.008857727051 -8.24957752227783\\
107.058853149414 -8.25259304046631\\
107.108856201172 -8.25562000274658\\
107.158851623535 -8.25866317749023\\
107.208854675293 -8.26172161102295\\
107.258857727051 -8.26479339599609\\
107.308853149414 -8.26788139343262\\
107.358856201172 -8.27098178863525\\
107.408851623535 -8.27408409118652\\
107.458854675293 -8.27718257904053\\
107.508857727051 -8.28027153015137\\
107.558853149414 -8.28334140777588\\
107.608856201172 -8.28639221191406\\
107.658851623535 -8.28943538665771\\
107.708854675293 -8.29246616363525\\
107.758857727051 -8.295485496521\\
107.808853149414 -8.29849433898926\\
107.858856201172 -8.30149364471436\\
107.908851623535 -8.30449390411377\\
107.958854675293 -8.30749607086182\\
108.008857727051 -8.31049156188965\\
108.058853149414 -8.31347846984863\\
108.108856201172 -8.31646347045898\\
108.158851623535 -8.31944847106934\\
108.208854675293 -8.32242870330811\\
108.258857727051 -8.32541656494141\\
108.308853149414 -8.32841396331787\\
108.358856201172 -8.33142566680908\\
108.408851623535 -8.33444690704346\\
108.458854675293 -8.33747482299805\\
108.508857727051 -8.34051132202148\\
108.558853149414 -8.34355545043945\\
108.608856201172 -8.34660148620605\\
108.658851623535 -8.34964466094971\\
108.708854675293 -8.35268306732178\\
108.758857727051 -8.35571193695068\\
108.808853149414 -8.35872554779053\\
108.858856201172 -8.36172866821289\\
108.908851623535 -8.36472606658936\\
108.958854675293 -8.36771488189697\\
109.008857727051 -8.37069225311279\\
109.058853149414 -8.3736629486084\\
109.108856201172 -8.37662601470947\\
109.158851623535 -8.37957954406738\\
109.208854675293 -8.38252353668213\\
109.258857727051 -8.38546085357666\\
109.308853149414 -8.38839435577393\\
109.358856201172 -8.39132213592529\\
109.408851623535 -8.39425563812256\\
109.458854675293 -8.39718532562256\\
109.508857727051 -8.40012359619141\\
109.558853149414 -8.40306949615479\\
109.608856201172 -8.40602779388428\\
109.658851623535 -8.4089937210083\\
109.708854675293 -8.4119701385498\\
109.758857727051 -8.41495132446289\\
109.808853149414 -8.41793441772461\\
109.858856201172 -8.42090892791748\\
109.908851623535 -8.42387676239014\\
109.958854675293 -8.42684173583984\\
110.008857727051 -8.42980480194092\\
110.058853149414 -8.43275737762451\\
110.108856201172 -8.43570423126221\\
110.158851623535 -8.43864631652832\\
110.208854675293 -8.44158363342285\\
110.258857727051 -8.44451427459717\\
110.308853149414 -8.44743919372559\\
110.358856201172 -8.45035362243652\\
110.408851623535 -8.45326232910156\\
110.458854675293 -8.45617008209229\\
110.508857727051 -8.45907974243164\\
110.558853149414 -8.46198844909668\\
110.608856201172 -8.46489715576172\\
110.658851623535 -8.46779918670654\\
110.708854675293 -8.4707088470459\\
110.758857727051 -8.4736270904541\\
110.808853149414 -8.47654724121094\\
110.858856201172 -8.47946548461914\\
110.908851623535 -8.48238372802734\\
110.958854675293 -8.48530292510986\\
111.008857727051 -8.48821926116943\\
111.058853149414 -8.49112796783447\\
111.108856201172 -8.49403285980225\\
111.158851623535 -8.49693489074707\\
111.208854675293 -8.49982452392578\\
111.258857727051 -8.50270462036133\\
111.308853149414 -8.50557518005371\\
111.358856201172 -8.50843906402588\\
111.408851623535 -8.5113000869751\\
111.458854675293 -8.51415538787842\\
111.508857727051 -8.51700592041016\\
111.558853149414 -8.51984405517578\\
111.608856201172 -8.52267837524414\\
111.658851623535 -8.52549648284912\\
111.708854675293 -8.52830123901367\\
111.758857727051 -8.53108787536621\\
111.808853149414 -8.5338659286499\\
111.858856201172 -8.53665161132813\\
111.908851623535 -8.53944492340088\\
111.958854675293 -8.54224967956543\\
112.008857727051 -8.54507064819336\\
112.058853149414 -8.54790592193604\\
112.108856201172 -8.55074405670166\\
112.158851623535 -8.55358505249023\\
112.208854675293 -8.55643081665039\\
112.258857727051 -8.55927467346191\\
112.308853149414 -8.56210708618164\\
112.358856201172 -8.5649356842041\\
112.408851623535 -8.56775760650635\\
112.458854675293 -8.5705680847168\\
112.508857727051 -8.57336902618408\\
112.558853149414 -8.57616424560547\\
112.608856201172 -8.57894229888916\\
112.658851623535 -8.58171081542969\\
112.708854675293 -8.5844783782959\\
112.758857727051 -8.58723926544189\\
112.808853149414 -8.58999061584473\\
112.858856201172 -8.59273052215576\\
112.908851623535 -8.59545421600342\\
112.958854675293 -8.59816741943359\\
113.008857727051 -8.60086727142334\\
113.058853149414 -8.60356616973877\\
113.108856201172 -8.60626602172852\\
113.158851623535 -8.60896492004395\\
113.208854675293 -8.61166858673096\\
113.258857727051 -8.61438179016113\\
113.308853149414 -8.61710834503174\\
113.358856201172 -8.61984539031982\\
113.408851623535 -8.62258625030518\\
113.458854675293 -8.62533664703369\\
113.508857727051 -8.62808322906494\\
113.558853149414 -8.63082695007324\\
113.608856201172 -8.6335620880127\\
113.658851623535 -8.63629150390625\\
113.708854675293 -8.63900852203369\\
113.758857727051 -8.64171314239502\\
113.808853149414 -8.64440536499023\\
113.858856201172 -8.64708709716797\\
113.908851623535 -8.64975929260254\\
113.958854675293 -8.65242385864258\\
114.008857727051 -8.65507698059082\\
114.058853149414 -8.65771961212158\\
114.108856201172 -8.66035270690918\\
114.158851623535 -8.66297340393066\\
114.208854675293 -8.66558361053467\\
114.258857727051 -8.66818237304688\\
114.308853149414 -8.6707763671875\\
114.358856201172 -8.67337512969971\\
114.408851623535 -8.67597484588623\\
114.458854675293 -8.67857933044434\\
114.508857727051 -8.68119525909424\\
114.558853149414 -8.68382358551025\\
114.608856201172 -8.68645858764648\\
114.658851623535 -8.68910026550293\\
114.708854675293 -8.69174194335938\\
114.758857727051 -8.69438362121582\\
114.808853149414 -8.69702243804932\\
114.858856201172 -8.69965362548828\\
114.908851623535 -8.7022762298584\\
114.958854675293 -8.7048921585083\\
115.008857727051 -8.70749473571777\\
115.058853149414 -8.71008586883545\\
115.108856201172 -8.71266555786133\\
115.158851623535 -8.71523284912109\\
115.208854675293 -8.71778583526611\\
115.258857727051 -8.72032356262207\\
115.308853149414 -8.72284889221191\\
115.358856201172 -8.72536087036133\\
115.408851623535 -8.72786903381348\\
115.458854675293 -8.73037147521973\\
115.508857727051 -8.73287391662598\\
115.558853149414 -8.73538780212402\\
115.608856201172 -8.73791790008545\\
115.658851623535 -8.74046421051025\\
115.708854675293 -8.74301719665527\\
115.758857727051 -8.74557971954346\\
115.808853149414 -8.7481517791748\\
115.858856201172 -8.75072765350342\\
115.908851623535 -8.75330257415771\\
115.958854675293 -8.75587368011475\\
116.008857727051 -8.75844192504883\\
116.058853149414 -8.76100540161133\\
116.108856201172 -8.76355743408203\\
116.158851623535 -8.76609802246094\\
116.208854675293 -8.76862716674805\\
116.258857727051 -8.77114200592041\\
116.308853149414 -8.77364635467529\\
116.358856201172 -8.77613830566406\\
116.408851623535 -8.77861499786377\\
116.458854675293 -8.78108501434326\\
116.508857727051 -8.78354835510254\\
116.558853149414 -8.78600311279297\\
116.608856201172 -8.78844833374023\\
116.658851623535 -8.79089736938477\\
116.708854675293 -8.79334449768066\\
116.758857727051 -8.79580211639404\\
116.808853149414 -8.79827213287354\\
116.858856201172 -8.80075454711914\\
116.908851623535 -8.80325126647949\\
116.958854675293 -8.80575942993164\\
117.008857727051 -8.80827331542969\\
117.058853149414 -8.810791015625\\
117.108856201172 -8.81330585479736\\
117.158851623535 -8.81581401824951\\
117.208854675293 -8.81831169128418\\
117.258857727051 -8.82079410552979\\
117.308853149414 -8.82326412200928\\
117.358856201172 -8.82572269439697\\
117.408851623535 -8.82816600799561\\
117.458854675293 -8.83059597015381\\
117.508857727051 -8.83300971984863\\
117.558853149414 -8.83540725708008\\
117.608856201172 -8.83779525756836\\
117.658851623535 -8.84017562866211\\
117.708854675293 -8.84254932403564\\
117.758857727051 -8.84492111206055\\
117.808853149414 -8.84729671478271\\
117.858856201172 -8.8496789932251\\
117.908851623535 -8.8520622253418\\
117.958854675293 -8.85445976257324\\
118.008857727051 -8.85687351226807\\
118.058853149414 -8.85929870605469\\
118.108856201172 -8.86173057556152\\
118.158851623535 -8.86416244506836\\
118.208854675293 -8.8665943145752\\
118.258857727051 -8.8690242767334\\
118.308853149414 -8.87145042419434\\
118.358856201172 -8.87386035919189\\
118.408851623535 -8.87625789642334\\
118.458854675293 -8.87864208221436\\
118.508857727051 -8.88101387023926\\
118.558853149414 -8.88337135314941\\
118.608856201172 -8.88571357727051\\
118.658851623535 -8.88804817199707\\
118.708854675293 -8.890380859375\\
118.758857727051 -8.89269924163818\\
118.808853149414 -8.8950080871582\\
118.858856201172 -8.89731025695801\\
118.908851623535 -8.89961051940918\\
118.958854675293 -8.90190505981445\\
119.008857727051 -8.90419673919678\\
119.058853149414 -8.90649604797363\\
119.108856201172 -8.90880584716797\\
119.158851623535 -8.9111213684082\\
119.208854675293 -8.91344261169434\\
119.258857727051 -8.915771484375\\
119.308853149414 -8.9181079864502\\
119.358856201172 -8.92044448852539\\
119.408851623535 -8.92277336120605\\
119.458854675293 -8.92509651184082\\
119.508857727051 -8.92740726470947\\
119.558853149414 -8.92970657348633\\
119.608856201172 -8.9319953918457\\
119.658851623535 -8.9342679977417\\
119.708854675293 -8.93652534484863\\
119.758857727051 -8.93876934051514\\
119.808853149414 -8.94099712371826\\
119.858856201172 -8.94321060180664\\
119.908851623535 -8.94541358947754\\
119.958854675293 -8.94761657714844\\
120.008857727051 -8.94981956481934\\
120.058853149414 -8.95201301574707\\
120.108856201172 -8.95420742034912\\
120.158851623535 -8.95640850067139\\
120.208854675293 -8.95861721038818\\
120.258857727051 -8.96083068847656\\
120.308853149414 -8.96305084228516\\
120.358856201172 -8.96526718139648\\
120.408851623535 -8.96748638153076\\
120.458854675293 -8.96970653533936\\
120.508857727051 -8.97191905975342\\
120.558853149414 -8.97412300109863\\
120.608856201172 -8.97632026672363\\
120.658851623535 -8.9785041809082\\
120.708854675293 -8.98067569732666\\
120.758857727051 -8.98283004760742\\
120.808853149414 -8.98497295379639\\
120.858856201172 -8.98710346221924\\
120.908851623535 -8.98922348022461\\
120.958854675293 -8.99133205413818\\
121.008857727051 -8.99343395233154\\
121.058853149414 -8.99554634094238\\
121.108856201172 -8.99766063690186\\
121.158851623535 -8.99976348876953\\
121.208854675293 -9.00186252593994\\
121.258857727051 -9.00396537780762\\
121.308853149414 -9.00606632232666\\
121.358856201172 -9.00817394256592\\
121.408851623535 -9.01027965545654\\
121.458854675293 -9.01237869262695\\
121.508857727051 -9.0144681930542\\
121.558853149414 -9.01655578613281\\
121.608856201172 -9.01863765716553\\
121.658851623535 -9.02071189880371\\
121.708854675293 -9.02277851104736\\
121.758857727051 -9.0248384475708\\
121.808853149414 -9.02688980102539\\
121.858856201172 -9.02893924713135\\
121.908851623535 -9.03097343444824\\
121.958854675293 -9.03299617767334\\
122.008857727051 -9.03501224517822\\
122.058853149414 -9.03702259063721\\
122.108856201172 -9.03902912139893\\
122.158851623535 -9.04103374481201\\
122.208854675293 -9.04303646087646\\
122.258857727051 -9.04503631591797\\
122.308853149414 -9.04703521728516\\
122.358856201172 -9.04903602600098\\
122.408851623535 -9.05103874206543\\
122.458854675293 -9.05304431915283\\
122.508857727051 -9.05505561828613\\
122.558853149414 -9.05706596374512\\
122.608856201172 -9.05906772613525\\
122.658851623535 -9.06106090545654\\
122.708854675293 -9.06304836273193\\
122.758857727051 -9.06502628326416\\
122.808853149414 -9.0669994354248\\
122.858856201172 -9.06896591186523\\
122.908851623535 -9.07093334197998\\
122.958854675293 -9.07289600372314\\
123.008857727051 -9.07485485076904\\
123.058853149414 -9.07681941986084\\
123.108856201172 -9.07878494262695\\
123.158851623535 -9.08075046539307\\
123.208854675293 -9.08271789550781\\
123.258857727051 -9.08468246459961\\
123.308853149414 -9.08664894104004\\
123.358856201172 -9.08862400054932\\
123.408851623535 -9.09060287475586\\
123.458854675293 -9.09258270263672\\
123.508857727051 -9.09456634521484\\
123.558853149414 -9.09654331207275\\
123.608856201172 -9.09850692749023\\
123.658851623535 -9.10045623779297\\
123.708854675293 -9.10239410400391\\
123.758857727051 -9.10432147979736\\
123.808853149414 -9.10622978210449\\
123.858856201172 -9.10813140869141\\
123.908851623535 -9.11002063751221\\
123.958854675293 -9.11188983917236\\
124.008857727051 -9.1137523651123\\
124.058853149414 -9.11559867858887\\
124.108856201172 -9.11743927001953\\
124.158851623535 -9.1192741394043\\
124.208854675293 -9.1211051940918\\
124.258857727051 -9.12293338775635\\
124.308853149414 -9.12475299835205\\
124.358856201172 -9.12656688690186\\
124.408851623535 -9.12837886810303\\
124.458854675293 -9.13018703460693\\
124.508857727051 -9.13198375701904\\
124.558853149414 -9.13376712799072\\
124.608856201172 -9.13554096221924\\
124.658851623535 -9.13729476928711\\
124.708854675293 -9.13903331756592\\
124.758857727051 -9.14077091217041\\
124.808853149414 -9.14249229431152\\
124.858856201172 -9.14417362213135\\
124.908851623535 -9.14582061767578\\
124.958854675293 -9.14744853973389\\
125.008857727051 -9.1490592956543\\
125.058853149414 -9.15065956115723\\
125.108856201172 -9.15225219726563\\
125.158851623535 -9.15385055541992\\
125.208854675293 -9.1554536819458\\
125.258857727051 -9.15707302093506\\
125.308853149414 -9.15870571136475\\
125.358856201172 -9.16034603118896\\
125.408851623535 -9.16198253631592\\
125.458854675293 -9.16359806060791\\
125.508857727051 -9.16519546508789\\
125.558853149414 -9.16676712036133\\
125.608856201172 -9.16831398010254\\
125.658851623535 -9.16983509063721\\
125.708854675293 -9.17133140563965\\
125.758857727051 -9.17280101776123\\
125.808853149414 -9.17426013946533\\
125.858856201172 -9.17570781707764\\
125.908851623535 -9.17714595794678\\
125.958854675293 -9.17857551574707\\
126.008857727051 -9.17998695373535\\
126.058853149414 -9.18139839172363\\
126.108856201172 -9.18280982971191\\
126.158851623535 -9.1842212677002\\
126.208854675293 -9.18563079833984\\
126.258857727051 -9.18704319000244\\
126.308853149414 -9.18845272064209\\
126.358856201172 -9.18986415863037\\
126.408851623535 -9.19127082824707\\
126.458854675293 -9.19266414642334\\
126.508857727051 -9.19403266906738\\
126.558853149414 -9.1953821182251\\
126.608856201172 -9.19670581817627\\
126.658851623535 -9.19801235198975\\
126.708854675293 -9.19929504394531\\
126.758857727051 -9.2005558013916\\
126.808853149414 -9.20180606842041\\
126.858856201172 -9.203049659729\\
126.908851623535 -9.20428276062012\\
126.958854675293 -9.2055139541626\\
127.008857727051 -9.20674896240234\\
127.058853149414 -9.2079906463623\\
127.108856201172 -9.2092399597168\\
127.158851623535 -9.21049213409424\\
127.208854675293 -9.21174240112305\\
127.258857727051 -9.21298217773438\\
127.308853149414 -9.21421146392822\\
127.358856201172 -9.21542739868164\\
127.408851623535 -9.21662998199463\\
127.458854675293 -9.21782207489014\\
127.508857727051 -9.21899509429932\\
127.558853149414 -9.22015190124512\\
127.608856201172 -9.22129154205322\\
127.658851623535 -9.22242641448975\\
127.708854675293 -9.22356605529785\\
127.758857727051 -9.22471618652344\\
127.808853149414 -9.22583770751953\\
127.858856201172 -9.22694206237793\\
127.908851623535 -9.2280387878418\\
127.958854675293 -9.22912979125977\\
128.008850097656 -9.23022270202637\\
128.058853149414 -9.2313232421875\\
128.108856201172 -9.23243141174316\\
128.15885925293 -9.23354434967041\\
128.208847045898 -9.23466491699219\\
128.258850097656 -9.23579025268555\\
128.308853149414 -9.23691463470459\\
128.358856201172 -9.23803329467773\\
128.40885925293 -9.23914909362793\\
128.458847045898 -9.24026393890381\\
128.508850097656 -9.24137783050537\\
128.558853149414 -9.24247646331787\\
128.608856201172 -9.24354553222656\\
128.65885925293 -9.24458789825439\\
128.708847045898 -9.2456111907959\\
128.758850097656 -9.24662303924561\\
128.808853149414 -9.24762439727783\\
128.858856201172 -9.24862575531006\\
128.90885925293 -9.24962902069092\\
128.958847045898 -9.25063991546631\\
129.008850097656 -9.25166606903076\\
129.058853149414 -9.25270652770996\\
129.108856201172 -9.25376033782959\\
129.15885925293 -9.25482749938965\\
129.208847045898 -9.25590133666992\\
129.258850097656 -9.25697708129883\\
129.308853149414 -9.25804996490479\\
129.358856201172 -9.25911712646484\\
129.40885925293 -9.26017475128174\\
129.458847045898 -9.26123237609863\\
129.508850097656 -9.26230049133301\\
129.558853149414 -9.26340675354004\\
129.608856201172 -9.26449584960938\\
129.65885925293 -9.26548099517822\\
129.708847045898 -9.26643466949463\\
129.756820678711 -9.2673511505127\\
129.801651000977 -9.26821231842041\\
129.851654052734 -9.2691822052002\\
129.901657104492 -9.27015495300293\\
129.951644897461 -9.27112674713135\\
130.001647949219 -9.27210235595703\\
130.051651000977 -9.27310371398926\\
130.101654052734 -9.27416133880615\\
130.151657104492 -9.2752685546875\\
130.201644897461 -9.27643299102783\\
130.251647949219 -9.27759647369385\\
130.301651000977 -9.27871704101563\\
130.351654052734 -9.27977275848389\\
130.401657104492 -9.28077507019043\\
130.451644897461 -9.28173351287842\\
130.501647949219 -9.28265380859375\\
130.551651000977 -9.28354644775391\\
130.601654052734 -9.28442478179932\\
130.651657104492 -9.28530406951904\\
130.701644897461 -9.28619766235352\\
130.751647949219 -9.28711223602295\\
130.801651000977 -9.28805732727051\\
130.851654052734 -9.2890510559082\\
130.901657104492 -9.2900857925415\\
130.951644897461 -9.29115962982178\\
131.001647949219 -9.29225635528564\\
131.051651000977 -9.29334354400635\\
131.101654052734 -9.29440498352051\\
131.151657104492 -9.29543399810791\\
131.201644897461 -9.29641914367676\\
131.251647949219 -9.29736328125\\
131.301651000977 -9.29828453063965\\
131.351654052734 -9.29919052124023\\
131.401657104492 -9.30008697509766\\
131.451644897461 -9.30099010467529\\
131.501647949219 -9.30190753936768\\
131.551651000977 -9.3028507232666\\
131.601654052734 -9.30382633209229\\
131.651657104492 -9.30483245849609\\
131.701644897461 -9.30586051940918\\
131.751647949219 -9.30690002441406\\
131.801651000977 -9.30793857574463\\
131.851654052734 -9.30896759033203\\
131.901657104492 -9.30998420715332\\
131.951644897461 -9.31098175048828\\
132.001647949219 -9.3120174407959\\
132.051651000977 -9.31308841705322\\
132.101654052734 -9.31418037414551\\
132.151657104492 -9.31524181365967\\
132.201644897461 -9.31637477874756\\
132.251647949219 -9.31762027740479\\
132.301651000977 -9.31943893432617\\
132.351654052734 -9.33881950378418\\
132.399063110352 -9.59718799591064\\
132.449066162109 -10.3238258361816\\
132.488479614258 -10.7745790481567\\
132.517471313477 -11.0445594787598\\
132.550643920898 -11.3497371673584\\
132.591033935547 -11.6514368057251\\
132.638305664063 -11.860743522644\\
132.68830871582 -14.0141801834106\\
132.709854125977 -14.3180704116821\\
132.736358642578 -14.7210302352905\\
132.75927734375 -15.4034357070923\\
132.783203125 -15.4156341552734\\
132.831085205078 -15.4250774383545\\
132.881088256836 -15.4381380081177\\
132.919631958008 -15.4575252532959\\
132.942733764648 -15.4722108840942\\
132.989501953125 -15.790093421936\\
133.036010742188 -16.7283401489258\\
133.057739257813 -16.906530380249\\
133.103927612305 -17.6809177398682\\
133.153915405273 -17.9481296539307\\
133.203353881836 -18.0170078277588\\
133.248291015625 -18.769323348999\\
133.296310424805 -19.4464874267578\\
133.346298217773 -19.4795627593994\\
133.396301269531 -19.4927539825439\\
133.422164916992 -19.8758449554443\\
133.463790893555 -20.8120594024658\\
133.513793945313 -20.9664974212646\\
133.56379699707 -20.9658622741699\\
133.613784790039 -20.9659976959229\\
133.663787841797 -20.9765510559082\\
133.713790893555 -21.0123081207275\\
133.763793945313 -21.0225086212158\\
133.81379699707 -21.0251407623291\\
133.863784790039 -21.0276584625244\\
133.913787841797 -21.0451164245605\\
133.952331542969 -21.1279964447021\\
133.971298217773 -21.2445106506348\\
134.01139831543 -21.2701625823975\\
134.061401367188 -21.2705383300781\\
134.110595703125 -21.2699146270752\\
134.160125732422 -21.27073097229\\
134.21012878418 -21.2714824676514\\
134.260131835938 -21.2738914489746\\
134.310134887695 -21.27565574646\\
134.360122680664 -21.2765884399414\\
134.410125732422 -21.2765445709229\\
134.46012878418 -21.2767524719238\\
134.510131835938 -21.2795581817627\\
134.560134887695 -21.294376373291\\
134.593399047852 -21.4076690673828\\
134.594650268555 -21.4420394897461\\
134.595703125 -21.4693717956543\\
134.596771240234 -21.4982261657715\\
134.597778320313 -21.526195526123\\
134.598785400391 -21.5536727905273\\
134.599822998047 -21.5817222595215\\
134.600875854492 -21.6095504760742\\
134.601974487305 -21.6379814147949\\
134.603118896484 -21.6666145324707\\
134.604309082031 -21.6955623626709\\
134.605545043945 -21.724422454834\\
134.606857299805 -21.753963470459\\
134.608215332031 -21.7830715179443\\
134.609680175781 -21.8131427764893\\
134.611221313477 -21.8430328369141\\
134.612869262695 -21.8732376098633\\
134.614654541016 -21.903507232666\\
134.616607666016 -21.9341888427734\\
134.618759155273 -21.9647979736328\\
134.621170043945 -21.9957866668701\\
134.623901367188 -22.0268611907959\\
134.627059936523 -22.0584659576416\\
134.630798339844 -22.0905075073242\\
134.635391235352 -22.1230316162109\\
134.641159057617 -22.1571083068848\\
134.648010253906 -22.1905307769775\\
134.658340454102 -22.2232818603516\\
134.67985534668 -22.2501029968262\\
134.691192626953 -22.2825813293457\\
134.696212768555 -22.3641891479492\\
134.697998046875 -22.3838310241699\\
134.700500488281 -22.3864059448242\\
134.707580566406 -22.3893432617188\\
134.726531982422 -22.3919296264648\\
134.766204833984 -22.3945960998535\\
134.808532714844 -22.3983592987061\\
134.853454589844 -22.4012756347656\\
134.899154663086 -22.4028987884521\\
134.948272705078 -22.4042549133301\\
134.998275756836 -22.4061717987061\\
135.048278808594 -22.407564163208\\
135.098281860352 -22.4084453582764\\
135.14826965332 -22.4091339111328\\
135.198272705078 -22.4097156524658\\
135.244552612305 -22.4101257324219\\
135.293548583984 -22.4106903076172\\
135.331344604492 -22.4112873077393\\
135.363616943359 -22.4118900299072\\
135.41325378418 -22.4130058288574\\
135.463256835938 -22.4143562316895\\
135.513259887695 -22.4159412384033\\
135.563262939453 -22.4177074432373\\
135.613250732422 -22.4196166992188\\
135.66325378418 -22.4215564727783\\
135.713256835938 -22.4234886169434\\
135.74528503418 -22.4246940612793\\
135.774383544922 -22.425745010376\\
135.803100585938 -22.4267253875732\\
135.831848144531 -22.4276485443115\\
135.860641479492 -22.428503036499\\
135.889511108398 -22.4293022155762\\
135.918563842773 -22.4300594329834\\
135.947799682617 -22.430793762207\\
135.977233886719 -22.431510925293\\
136.006851196289 -22.4322032928467\\
136.03662109375 -22.432861328125\\
136.066482543945 -22.4334850311279\\
136.096450805664 -22.4340648651123\\
136.126388549805 -22.4346084594727\\
136.156341552734 -22.4351215362549\\
136.186279296875 -22.435604095459\\
136.216156005859 -22.4360542297363\\
136.246032714844 -22.4364757537842\\
136.275894165039 -22.4368762969971\\
136.305801391602 -22.4372787475586\\
136.33576965332 -22.4376964569092\\
136.365829467773 -22.4381160736084\\
136.39599609375 -22.4385452270508\\
136.426345825195 -22.4389953613281\\
136.456741333008 -22.4394607543945\\
136.487258911133 -22.4399452209473\\
136.517913818359 -22.4404525756836\\
136.548629760742 -22.440975189209\\
136.579360961914 -22.441499710083\\
136.610107421875 -22.4420375823975\\
136.640869140625 -22.4425811767578\\
136.67155456543 -22.4431304931641\\
136.702239990234 -22.4436912536621\\
136.732971191406 -22.4442539215088\\
136.763717651367 -22.4448165893555\\
136.794525146484 -22.4453792572021\\
136.825408935547 -22.4459419250488\\
136.856384277344 -22.4465045928955\\
136.887481689453 -22.4470672607422\\
136.918670654297 -22.447624206543\\
136.949966430664 -22.448184967041\\
136.981307983398 -22.4487476348877\\
137.012680053711 -22.4493255615234\\
137.044082641602 -22.4499053955078\\
137.075546264648 -22.4504699707031\\
137.107040405273 -22.4510326385498\\
137.138626098633 -22.4515953063965\\
137.170318603516 -22.4521484375\\
137.202133178711 -22.4526882171631\\
137.234130859375 -22.4532165527344\\
137.266311645508 -22.4537258148193\\
137.298706054688 -22.4542083740234\\
137.331420898438 -22.4546604156494\\
137.364349365234 -22.4550743103027\\
137.397476196289 -22.4555034637451\\
137.430725097656 -22.4560089111328\\
137.464111328125 -22.4566020965576\\
137.497680664063 -22.4572639465332\\
137.531341552734 -22.4579811096191\\
137.565063476563 -22.4587574005127\\
137.598892211914 -22.4595718383789\\
137.632751464844 -22.4605617523193\\
137.664688110352 -22.4615116119385\\
137.686798095703 -22.4621543884277\\
137.705932617188 -22.4627151489258\\
137.728256225586 -22.4633827209473\\
137.759414672852 -22.4643440246582\\
137.790267944336 -22.4653072357178\\
137.814239501953 -22.4660472869873\\
137.834365844727 -22.4666213989258\\
137.864456176758 -22.467456817627\\
137.895660400391 -22.4682884216309\\
137.919982910156 -22.4689254760742\\
137.940353393555 -22.4694480895996\\
137.970275878906 -22.4701480865479\\
138.001678466797 -22.4709358215332\\
138.032989501953 -22.4719848632813\\
138.064208984375 -22.4733200073242\\
138.088577270508 -22.4743480682373\\
138.111175537109 -22.4751319885254\\
138.139999389648 -22.4758796691895\\
138.17155456543 -22.476541519165\\
138.202758789063 -22.4772834777832\\
138.233932495117 -22.4781341552734\\
138.260604858398 -22.478910446167\\
138.285491943359 -22.4796504974365\\
138.309707641602 -22.4803581237793\\
138.337066650391 -22.481128692627\\
138.374252319336 -22.482120513916\\
138.392700195313 -22.4825954437256\\
138.425262451172 -22.4833927154541\\
138.461929321289 -22.4842357635498\\
138.493881225586 -22.484935760498\\
138.52766418457 -22.4856395721436\\
138.56005859375 -22.4862823486328\\
138.59260559082 -22.4869003295898\\
138.624969482422 -22.4874897003174\\
138.6572265625 -22.488073348999\\
138.689361572266 -22.4886856079102\\
138.721313476563 -22.4892807006836\\
138.753051757813 -22.4898319244385\\
138.783340454102 -22.4903030395508\\
138.812240600586 -22.4907264709473\\
138.839721679688 -22.4911441802979\\
138.870788574219 -22.4916038513184\\
138.901794433594 -22.4920291900635\\
138.932571411133 -22.4924449920654\\
138.963226318359 -22.492883682251\\
138.99382019043 -22.493350982666\\
139.024337768555 -22.493839263916\\
139.054656982422 -22.4943656921387\\
139.085006713867 -22.4949150085449\\
139.115203857422 -22.4955387115479\\
139.145263671875 -22.4962348937988\\
139.17529296875 -22.4969730377197\\
139.205322265625 -22.4976501464844\\
139.235382080078 -22.4982624053955\\
139.265396118164 -22.4988498687744\\
139.295501708984 -22.4994373321533\\
139.325637817383 -22.4999256134033\\
139.355880737305 -22.500337600708\\
139.386154174805 -22.5007190704346\\
139.416519165039 -22.5011215209961\\
139.44694519043 -22.5015525817871\\
139.47737121582 -22.5019950866699\\
139.507736206055 -22.5024719238281\\
139.538101196289 -22.5029754638672\\
139.568450927734 -22.5034580230713\\
139.598739624023 -22.5039081573486\\
139.629043579102 -22.5043296813965\\
139.659378051758 -22.5047492980957\\
139.689758300781 -22.5051879882813\\
139.720199584961 -22.5056533813477\\
139.750732421875 -22.5061111450195\\
139.781326293945 -22.5066070556641\\
139.81201171875 -22.5072631835938\\
139.842803955078 -22.5080699920654\\
139.867553710938 -22.5087966918945\\
139.886856079102 -22.5094013214111\\
139.904815673828 -22.5099964141846\\
139.922988891602 -22.510627746582\\
139.941986083984 -22.5113162994385\\
139.961761474609 -22.5120677947998\\
139.981750488281 -22.5128593444824\\
140.018447875977 -22.5143852233887\\
140.068450927734 -22.5166187286377\\
140.114990234375 -22.5188331604004\\
140.152893066406 -22.5207099914551\\
140.18928527832 -22.5225620269775\\
140.216110229492 -22.5239505767822\\
140.262878417969 -22.5263900756836\\
140.310623168945 -22.5289154052734\\
140.353820800781 -22.5312194824219\\
140.403823852539 -22.5339183807373\\
140.453826904297 -22.5366287231445\\
140.503829956055 -22.5393142700195\\
140.553817749023 -22.5420074462891\\
140.591079711914 -22.5439949035645\\
140.61750793457 -22.545389175415\\
140.647613525391 -22.5469627380371\\
140.677612304688 -22.5485153198242\\
140.707641601563 -22.5500583648682\\
140.737747192383 -22.5515918731689\\
140.767852783203 -22.5531101226807\\
140.798034667969 -22.5546073913574\\
140.828247070313 -22.5560894012451\\
140.858505249023 -22.5575523376465\\
140.888793945313 -22.5590000152588\\
140.919143676758 -22.5604343414307\\
140.949508666992 -22.5618476867676\\
140.979904174805 -22.5632419586182\\
141.010314941406 -22.5646190643311\\
141.040771484375 -22.5659770965576\\
141.071258544922 -22.5673160552979\\
141.101791381836 -22.5686359405518\\
141.132400512695 -22.5699329376221\\
141.163101196289 -22.5712146759033\\
141.193923950195 -22.5724792480469\\
141.224945068359 -22.57373046875\\
141.256256103516 -22.5749702453613\\
141.287963867188 -22.5762100219727\\
141.320404052734 -22.577449798584\\
141.353958129883 -22.5787105560303\\
141.389129638672 -22.580005645752\\
141.437835693359 -22.5817546844482\\
141.487823486328 -22.5834732055664\\
141.537826538086 -22.5851249694824\\
141.587829589844 -22.5866851806641\\
141.637832641602 -22.5883884429932\\
141.687835693359 -22.5900573730469\\
141.737823486328 -22.5916786193848\\
141.787826538086 -22.5932922363281\\
141.837829589844 -22.5948905944824\\
141.887832641602 -22.5964794158936\\
141.937835693359 -22.5980091094971\\
141.987243652344 -22.5994529724121\\
142.035125732422 -22.6007709503174\\
142.084182739258 -22.6020050048828\\
142.132675170898 -22.6030197143555\\
142.182052612305 -22.6039772033691\\
142.231628417969 -22.6049880981445\\
142.273376464844 -22.6058731079102\\
142.323364257813 -22.6069622039795\\
142.37336730957 -22.6080055236816\\
142.423355102539 -22.6090755462646\\
142.473358154297 -22.6101608276367\\
142.523361206055 -22.6111946105957\\
142.573364257813 -22.6121101379395\\
142.622451782227 -22.6129188537598\\
142.672454833984 -22.6134662628174\\
142.722457885742 -22.6139259338379\\
142.76985168457 -22.6146202087402\\
142.811462402344 -22.6152572631836\\
142.861465454102 -22.6160202026367\\
142.911468505859 -22.6168212890625\\
142.961471557617 -22.6176929473877\\
143.011474609375 -22.618616104126\\
143.061462402344 -22.6195983886719\\
143.111465454102 -22.6206035614014\\
143.161468505859 -22.621639251709\\
143.211471557617 -22.6222095489502\\
143.261474609375 -22.6224975585938\\
143.311462402344 -22.622932434082\\
143.361465454102 -22.6236000061035\\
143.411468505859 -22.6246528625488\\
143.461471557617 -22.6259384155273\\
143.511474609375 -22.6277751922607\\
143.561462402344 -22.630199432373\\
143.611465454102 -22.6329460144043\\
143.661468505859 -22.6359081268311\\
143.711471557617 -22.6384601593018\\
143.761474609375 -22.641565322876\\
143.811462402344 -22.6446857452393\\
143.861465454102 -22.648588180542\\
143.911468505859 -22.6530170440674\\
143.961471557617 -22.6570949554443\\
144.011474609375 -22.6611614227295\\
144.061462402344 -22.6653537750244\\
144.111465454102 -22.6698303222656\\
144.161468505859 -22.6743755340576\\
144.211471557617 -22.6787853240967\\
144.261474609375 -22.6831665039063\\
144.311462402344 -22.6874771118164\\
144.361465454102 -22.6913948059082\\
144.411468505859 -22.6950092315674\\
144.461471557617 -22.6987152099609\\
144.511474609375 -22.7026195526123\\
144.561462402344 -22.7066516876221\\
144.611465454102 -22.7105312347412\\
144.661468505859 -22.7141742706299\\
144.711471557617 -22.7174072265625\\
144.761474609375 -22.7207717895508\\
144.811462402344 -22.7240695953369\\
144.861465454102 -22.7274227142334\\
144.911468505859 -22.7308387756348\\
144.961471557617 -22.7342548370361\\
145.011474609375 -22.7376289367676\\
145.061462402344 -22.7404022216797\\
145.111465454102 -22.7424983978271\\
145.161468505859 -22.7446956634521\\
145.211471557617 -22.7470073699951\\
145.261474609375 -22.7493419647217\\
145.311462402344 -22.7511005401611\\
145.361465454102 -22.7527542114258\\
145.411468505859 -22.7545948028564\\
145.461471557617 -22.7565765380859\\
145.511474609375 -22.7586212158203\\
145.561462402344 -22.7607078552246\\
145.611465454102 -22.7626838684082\\
145.661468505859 -22.7646942138672\\
145.711471557617 -22.7667007446289\\
145.761474609375 -22.7687816619873\\
145.811462402344 -22.7709674835205\\
145.861465454102 -22.7731227874756\\
145.911468505859 -22.7753715515137\\
145.961471557617 -22.7776679992676\\
146.011474609375 -22.7798938751221\\
146.061462402344 -22.782133102417\\
146.111465454102 -22.784330368042\\
146.161468505859 -22.786491394043\\
146.211471557617 -22.7886161804199\\
146.261474609375 -22.7906970977783\\
146.311462402344 -22.7927341461182\\
146.361465454102 -22.794713973999\\
146.411468505859 -22.7966251373291\\
146.461471557617 -22.7984676361084\\
146.511474609375 -22.8002510070801\\
146.561462402344 -22.8019733428955\\
146.611465454102 -22.8036518096924\\
146.661468505859 -22.8052921295166\\
146.711471557617 -22.8069076538086\\
146.761474609375 -22.8084964752197\\
146.811462402344 -22.8100662231445\\
146.861465454102 -22.811616897583\\
146.911468505859 -22.8131523132324\\
146.961471557617 -22.8146667480469\\
147.011474609375 -22.8161602020264\\
147.061462402344 -22.817626953125\\
147.111465454102 -22.8190612792969\\
147.161468505859 -22.8204574584961\\
147.211471557617 -22.8218097686768\\
147.261474609375 -22.8231067657471\\
147.311462402344 -22.8243465423584\\
147.361465454102 -22.825532913208\\
147.411468505859 -22.8266677856445\\
147.461471557617 -22.8277606964111\\
147.511474609375 -22.828821182251\\
147.561462402344 -22.8298625946045\\
147.611465454102 -22.8308925628662\\
147.661468505859 -22.8319072723389\\
147.711471557617 -22.8329181671143\\
147.761474609375 -22.8339233398438\\
147.811462402344 -22.8349151611328\\
147.861465454102 -22.8358917236328\\
147.911468505859 -22.8368530273438\\
147.961471557617 -22.837776184082\\
148.011474609375 -22.8386611938477\\
148.061462402344 -22.8395042419434\\
148.111465454102 -22.8403072357178\\
148.161468505859 -22.8410720825195\\
148.211471557617 -22.8418045043945\\
148.261474609375 -22.84250831604\\
148.311462402344 -22.8431930541992\\
148.361465454102 -22.8438625335693\\
148.411468505859 -22.8445262908936\\
148.461471557617 -22.8451747894287\\
148.511474609375 -22.8458232879639\\
148.561462402344 -22.8464660644531\\
148.611465454102 -22.847095489502\\
148.661468505859 -22.8476886749268\\
148.711471557617 -22.8482532501221\\
148.761474609375 -22.8487987518311\\
148.811462402344 -22.8493251800537\\
148.861465454102 -22.8498344421387\\
148.911468505859 -22.8503303527832\\
148.961471557617 -22.8508148193359\\
149.011474609375 -22.8512916564941\\
149.061462402344 -22.8517665863037\\
149.111465454102 -22.8522243499756\\
149.161468505859 -22.8526821136475\\
149.211471557617 -22.8531227111816\\
149.261474609375 -22.8535499572754\\
149.311462402344 -22.8539562225342\\
149.361465454102 -22.8543224334717\\
149.411468505859 -22.8546905517578\\
149.461471557617 -22.8550224304199\\
149.511474609375 -22.8553276062012\\
149.561462402344 -22.8556060791016\\
149.611465454102 -22.8558483123779\\
149.661468505859 -22.8560962677002\\
149.711471557617 -22.8563575744629\\
149.761474609375 -22.8565464019775\\
149.811462402344 -22.8567581176758\\
149.861465454102 -22.8570919036865\\
149.911468505859 -22.8575839996338\\
149.961471557617 -22.8579483032227\\
};
\addlegendentry{$\dtmax{} = \expneg{5.0}{3}{[s]}$};

\addplot [
color=red,
solid
]
table[row sep=crcr]{
0 0\\
1.00125873088837 0\\
2.00125861167908 0\\
3.00125861167908 0\\
4.00125885009766 0\\
5.00125885009766 0\\
6.00125885009766 0\\
7.00125885009766 0\\
8.00125885009766 0\\
9.00125885009766 0\\
10.0012588500977 0\\
11.0012588500977 0\\
12.0012588500977 0\\
13.0012588500977 0\\
14.0012588500977 0\\
15.0012588500977 0\\
16.0012588500977 0\\
17.0012588500977 0\\
18.0012588500977 0\\
19.0012588500977 0\\
20.0012588500977 -6.40699113318988e-07\\
21.0027179718018 -0.54533064365387\\
22.0114879608154 -0.647120714187622\\
23.0191459655762 -0.682085752487183\\
24.0288333892822 -0.706474602222443\\
25.031099319458 -0.728024959564209\\
26.031099319458 -0.754050970077515\\
27.031099319458 -0.786733090877533\\
28.031099319458 -0.828033864498138\\
29.031099319458 -0.881752073764801\\
30.031099319458 -0.951272785663605\\
31.031099319458 -1.04249548912048\\
32.0310974121094 -1.15545642375946\\
33.0310974121094 -1.2824410200119\\
34.0310974121094 -1.41679072380066\\
35.0010986328125 -1.55218243598938\\
36.0010986328125 -1.6894314289093\\
37.0010986328125 -1.82237696647644\\
38.0010986328125 -1.94850885868073\\
39.0010986328125 -2.07345080375671\\
40.0010986328125 -2.19880151748657\\
41.0010986328125 -2.3241822719574\\
42.0010986328125 -2.44855570793152\\
43.0010986328125 -2.57333970069885\\
44.0010986328125 -2.70048999786377\\
45.0010986328125 -2.82953810691834\\
46.0010986328125 -2.95576953887939\\
47.0010986328125 -3.07543992996216\\
48.0010986328125 -3.19168424606323\\
49.0010986328125 -3.30715131759644\\
50.0010986328125 -3.42086625099182\\
51.0010986328125 -3.53039288520813\\
52.0010986328125 -3.62599897384644\\
53.0010986328125 -3.71748447418213\\
54.0010986328125 -3.80550861358643\\
55.0010986328125 -3.89148640632629\\
56.0010986328125 -3.97675061225891\\
57.0010986328125 -4.06211280822754\\
58.0010986328125 -4.14786577224731\\
59.0010986328125 -4.23421096801758\\
60.0010986328125 -4.32158374786377\\
61.0010986328125 -4.41166973114014\\
62.0010986328125 -4.50269317626953\\
63.0010986328125 -4.59248113632202\\
64.0010986328125 -4.68359231948853\\
65.0010986328125 -4.77367973327637\\
66.0010986328125 -4.86511182785034\\
67.0010986328125 -4.95840406417847\\
68.0010986328125 -5.05229425430298\\
69.0010986328125 -5.14599943161011\\
70.0010986328125 -5.2413501739502\\
71.0010986328125 -5.33550786972046\\
72.0010986328125 -5.42925834655762\\
73.0010986328125 -5.52535629272461\\
74.0010986328125 -5.62437772750854\\
75.0010986328125 -5.72004413604736\\
76.0010986328125 -5.81302213668823\\
77.0010986328125 -5.90825080871582\\
78.0010986328125 -6.00795364379883\\
79.0010986328125 -6.10745191574097\\
80.0010986328125 -6.21083831787109\\
81.0010986328125 -6.31373691558838\\
82.0010986328125 -6.41339635848999\\
83.0010986328125 -6.51313209533691\\
84.0010986328125 -6.61153364181519\\
85.0010986328125 -6.71013212203979\\
86.0010986328125 -6.80636072158813\\
87.0010986328125 -6.90033006668091\\
88.0010986328125 -6.99256420135498\\
89.0010986328125 -7.08304166793823\\
90.0010986328125 -7.17208576202393\\
91.0010986328125 -7.26068496704102\\
92.0010986328125 -7.34971618652344\\
93.0010986328125 -7.43731832504272\\
94.0010986328125 -7.52138471603394\\
95.0010986328125 -7.60158109664917\\
96.0010986328125 -7.67838191986084\\
97.0010986328125 -7.75192356109619\\
98.0010986328125 -7.82260465621948\\
99.0010986328125 -7.89101314544678\\
100 -7.9577751159668\\
100.024505615234 -7.95939493179321\\
100.049507141113 -7.96104145050049\\
100.074508666992 -7.96269559860229\\
100.099502563477 -7.96436738967896\\
100.124504089355 -7.96605157852173\\
100.149505615234 -7.96773099899292\\
100.174507141113 -7.9694037437439\\
100.199508666992 -7.97106266021729\\
100.224502563477 -7.97270536422729\\
100.249504089355 -7.97433567047119\\
100.274505615234 -7.97595357894897\\
100.299507141113 -7.97755813598633\\
100.324508666992 -7.9791522026062\\
100.349502563477 -7.98074293136597\\
100.374504089355 -7.98233938217163\\
100.399505615234 -7.98394632339478\\
100.424507141113 -7.98556852340698\\
100.449508666992 -7.98720359802246\\
100.474502563477 -7.98884630203247\\
100.499504089355 -7.99049186706543\\
100.524505615234 -7.99214267730713\\
100.549507141113 -7.99379348754883\\
100.574508666992 -7.99545431137085\\
100.599502563477 -7.99711942672729\\
100.624504089355 -7.99878692626953\\
100.649505615234 -8.00045871734619\\
100.674507141113 -8.0021276473999\\
100.699508666992 -8.00379371643066\\
100.724502563477 -8.00545501708984\\
100.749504089355 -8.00711154937744\\
100.774505615234 -8.00876522064209\\
100.799507141113 -8.01041412353516\\
100.824508666992 -8.01206111907959\\
100.849502563477 -8.01370143890381\\
100.874504089355 -8.01533889770508\\
100.899505615234 -8.01697063446045\\
100.924507141113 -8.01859760284424\\
100.949508666992 -8.02024936676025\\
100.974502563477 -8.02190494537354\\
100.999504089355 -8.02355766296387\\
101.024505615234 -8.02520275115967\\
101.049507141113 -8.02683639526367\\
101.074508666992 -8.02847576141357\\
101.099502563477 -8.03011417388916\\
101.124504089355 -8.03177261352539\\
101.149505615234 -8.03343963623047\\
101.174507141113 -8.03510189056396\\
101.199508666992 -8.03675842285156\\
101.224502563477 -8.03840351104736\\
101.249504089355 -8.04003620147705\\
101.274505615234 -8.04165554046631\\
101.299507141113 -8.04326057434082\\
101.324508666992 -8.04485607147217\\
101.349502563477 -8.0464391708374\\
101.374504089355 -8.04801559448242\\
101.399505615234 -8.04958915710449\\
101.424507141113 -8.05116844177246\\
101.449508666992 -8.05275344848633\\
101.474502563477 -8.05434989929199\\
101.499504089355 -8.0559663772583\\
101.524505615234 -8.05760288238525\\
101.549507141113 -8.05926132202148\\
101.574508666992 -8.06092739105225\\
101.599502563477 -8.0626049041748\\
101.624504089355 -8.06428337097168\\
101.649505615234 -8.06596183776855\\
101.674507141113 -8.0676383972168\\
101.699508666992 -8.06930923461914\\
101.724502563477 -8.07097625732422\\
101.749504089355 -8.07264232635498\\
101.774505615234 -8.07430171966553\\
101.799507141113 -8.07595825195313\\
101.824508666992 -8.07760906219482\\
101.849502563477 -8.07925701141357\\
101.874504089355 -8.08090305328369\\
101.899505615234 -8.08254337310791\\
101.924507141113 -8.08418369293213\\
101.949508666992 -8.08582401275635\\
101.974502563477 -8.08746433258057\\
101.999504089355 -8.08910274505615\\
102.024505615234 -8.09073352813721\\
102.049507141113 -8.09237670898438\\
102.074508666992 -8.09402275085449\\
102.099502563477 -8.09565925598145\\
102.124504089355 -8.09731769561768\\
102.149505615234 -8.09898471832275\\
102.174507141113 -8.10064506530762\\
102.199508666992 -8.10229969024658\\
102.224502563477 -8.10394477844238\\
102.249504089355 -8.10557842254639\\
102.274505615234 -8.10719585418701\\
102.299507141113 -8.10879898071289\\
102.324508666992 -8.11039352416992\\
102.349502563477 -8.11197280883789\\
102.374504089355 -8.11354160308838\\
102.399505615234 -8.11510276794434\\
102.424507141113 -8.11666393280029\\
102.449508666992 -8.11822414398193\\
102.474502563477 -8.11978626251221\\
102.499504089355 -8.12135982513428\\
102.524505615234 -8.12294292449951\\
102.549507141113 -8.12453460693359\\
102.574508666992 -8.12613487243652\\
102.599502563477 -8.12773418426514\\
102.624504089355 -8.12934112548828\\
102.649505615234 -8.13095188140869\\
102.674507141113 -8.13256359100342\\
102.699508666992 -8.13417530059814\\
102.724502563477 -8.13578701019287\\
102.749504089355 -8.13740253448486\\
102.774505615234 -8.1390209197998\\
102.799507141113 -8.1406421661377\\
102.824508666992 -8.14226341247559\\
102.849502563477 -8.14388465881348\\
102.874504089355 -8.14551258087158\\
102.899505615234 -8.14713382720947\\
102.924507141113 -8.14875411987305\\
102.949508666992 -8.15037250518799\\
102.974502563477 -8.15200805664063\\
102.999504089355 -8.15365505218506\\
103.024505615234 -8.15529727935791\\
103.049507141113 -8.1569356918335\\
103.074508666992 -8.15856552124023\\
103.099502563477 -8.16018962860107\\
103.124504089355 -8.16182804107666\\
103.149505615234 -8.16347312927246\\
103.174507141113 -8.16511535644531\\
103.199508666992 -8.16675281524658\\
103.224502563477 -8.168381690979\\
103.249504089355 -8.16999816894531\\
103.274505615234 -8.17160224914551\\
103.299507141113 -8.17319011688232\\
103.324508666992 -8.17476463317871\\
103.349502563477 -8.17632579803467\\
103.374504089355 -8.17787742614746\\
103.399505615234 -8.17942047119141\\
103.424507141113 -8.18095588684082\\
103.449508666992 -8.18248844146729\\
103.474502563477 -8.18402004241943\\
103.499504089355 -8.18555450439453\\
103.524505615234 -8.18709373474121\\
103.549507141113 -8.18863868713379\\
103.574508666992 -8.19019222259521\\
103.599502563477 -8.19175243377686\\
103.624504089355 -8.19331645965576\\
103.649505615234 -8.19488620758057\\
103.674507141113 -8.196457862854\\
103.699508666992 -8.19803142547607\\
103.724502563477 -8.19960594177246\\
103.749504089355 -8.20118618011475\\
103.774505615234 -8.20276832580566\\
103.799507141113 -8.2043514251709\\
103.824508666992 -8.20594024658203\\
103.849502563477 -8.20752906799316\\
103.874504089355 -8.20912075042725\\
103.899505615234 -8.21071338653564\\
103.924507141113 -8.21230792999268\\
103.949508666992 -8.21390724182129\\
103.974502563477 -8.21549987792969\\
103.999504089355 -8.21709251403809\\
104.024505615234 -8.21867942810059\\
104.049507141113 -8.2202615737915\\
104.074508666992 -8.22185897827148\\
104.099502563477 -8.22346687316895\\
104.124504089355 -8.22507572174072\\
104.149505615234 -8.22668170928955\\
104.174507141113 -8.22828197479248\\
104.199508666992 -8.22987461090088\\
104.224502563477 -8.23145771026611\\
104.249504089355 -8.23303318023682\\
104.274505615234 -8.23459815979004\\
104.299507141113 -8.23614978790283\\
104.324508666992 -8.23769378662109\\
104.349502563477 -8.23922729492188\\
104.374504089355 -8.24075317382813\\
104.399505615234 -8.24227142333984\\
104.424507141113 -8.24378395080566\\
104.449508666992 -8.24529075622559\\
104.474502563477 -8.24679565429688\\
104.499504089355 -8.24830055236816\\
104.524505615234 -8.24980449676514\\
104.549507141113 -8.25130844116211\\
104.574508666992 -8.25281524658203\\
104.599502563477 -8.25432586669922\\
104.624504089355 -8.2558422088623\\
104.649505615234 -8.25736236572266\\
104.674507141113 -8.25888633728027\\
104.699508666992 -8.26041221618652\\
104.724502563477 -8.26194190979004\\
104.749504089355 -8.2634744644165\\
104.774505615234 -8.26500701904297\\
104.799507141113 -8.26653957366943\\
104.824508666992 -8.2680721282959\\
104.849502563477 -8.26960754394531\\
104.874504089355 -8.27115249633789\\
104.899505615234 -8.27270984649658\\
104.924507141113 -8.27427291870117\\
104.949508666992 -8.27584171295166\\
104.974502563477 -8.27741622924805\\
104.999504089355 -8.27899646759033\\
105.024505615234 -8.2805757522583\\
105.049507141113 -8.28214836120605\\
105.074508666992 -8.28372001647949\\
105.099502563477 -8.28529930114746\\
105.124504089355 -8.28687572479248\\
105.149505615234 -8.2884464263916\\
105.174507141113 -8.29001712799072\\
105.199508666992 -8.29158210754395\\
105.224502563477 -8.29314231872559\\
105.249504089355 -8.2946949005127\\
105.274505615234 -8.29623699188232\\
105.299507141113 -8.29776954650879\\
105.324508666992 -8.29929447174072\\
105.349502563477 -8.30081272125244\\
105.374504089355 -8.30232620239258\\
105.399505615234 -8.30383682250977\\
105.424507141113 -8.30534362792969\\
105.449508666992 -8.30684852600098\\
105.474502563477 -8.30834579467773\\
105.499504089355 -8.30984210968018\\
105.524505615234 -8.31133651733398\\
105.549507141113 -8.31282711029053\\
105.574508666992 -8.31431484222412\\
105.599502563477 -8.31580257415771\\
105.624504089355 -8.31729030609131\\
105.649505615234 -8.3187780380249\\
105.674507141113 -8.32026863098145\\
105.699508666992 -8.32175827026367\\
105.724502563477 -8.32324600219727\\
105.749504089355 -8.32474040985107\\
105.774505615234 -8.32623672485352\\
105.799507141113 -8.32773494720459\\
105.824508666992 -8.32923889160156\\
105.849502563477 -8.33074569702148\\
105.874504089355 -8.33225250244141\\
105.899505615234 -8.33376598358154\\
105.924507141113 -8.33528327941895\\
105.949508666992 -8.33680629730225\\
105.974502563477 -8.33833599090576\\
105.999504089355 -8.33987140655518\\
106.024505615234 -8.34141159057617\\
106.049507141113 -8.3429536819458\\
106.074508666992 -8.34449577331543\\
106.099502563477 -8.34604072570801\\
106.124504089355 -8.3475923538208\\
106.149505615234 -8.34914684295654\\
106.174507141113 -8.35069942474365\\
106.199508666992 -8.35224723815918\\
106.224502563477 -8.35378932952881\\
106.249504089355 -8.35533142089844\\
106.274505615234 -8.35686779022217\\
106.299507141113 -8.35839366912842\\
106.324508666992 -8.35991287231445\\
106.349502563477 -8.36142635345459\\
106.374504089355 -8.36293888092041\\
106.399505615234 -8.36444568634033\\
106.424507141113 -8.3659496307373\\
106.449508666992 -8.3674488067627\\
106.474502563477 -8.36894607543945\\
106.499504089355 -8.37044143676758\\
106.524505615234 -8.37193298339844\\
106.549507141113 -8.37342071533203\\
106.574508666992 -8.37490558624268\\
106.599502563477 -8.37638568878174\\
106.624504089355 -8.37786197662354\\
106.649505615234 -8.37933731079102\\
106.674507141113 -8.38080787658691\\
106.699508666992 -8.38227653503418\\
106.724502563477 -8.38374805450439\\
106.749504089355 -8.38522338867188\\
106.774505615234 -8.38669872283936\\
106.799507141113 -8.38817405700684\\
106.824508666992 -8.38964939117432\\
106.849502563477 -8.39112663269043\\
106.874504089355 -8.39260292053223\\
106.899505615234 -8.39407825469971\\
106.924507141113 -8.3955545425415\\
106.949508666992 -8.39703273773193\\
106.974502563477 -8.39851093292236\\
106.999504089355 -8.39999294281006\\
107.024505615234 -8.4014778137207\\
107.049507141113 -8.40296268463135\\
107.074508666992 -8.40445041656494\\
107.099502563477 -8.40593910217285\\
107.124504089355 -8.40743350982666\\
107.149505615234 -8.40893077850342\\
107.174507141113 -8.41043376922607\\
107.199508666992 -8.411940574646\\
107.224502563477 -8.41345405578613\\
107.249504089355 -8.41497039794922\\
107.274505615234 -8.41649150848389\\
107.299507141113 -8.41801452636719\\
107.324508666992 -8.41954040527344\\
107.349502563477 -8.42106628417969\\
107.374504089355 -8.4225902557373\\
107.399505615234 -8.42411327362061\\
107.424507141113 -8.42562961578369\\
107.449508666992 -8.42713832855225\\
107.474502563477 -8.42864036560059\\
107.499504089355 -8.43013668060303\\
107.524505615234 -8.43163108825684\\
107.549507141113 -8.43312454223633\\
107.574508666992 -8.43461227416992\\
107.599502563477 -8.43609809875488\\
107.624504089355 -8.43758201599121\\
107.649505615234 -8.43906021118164\\
107.674507141113 -8.44053840637207\\
107.699508666992 -8.44201374053955\\
107.724502563477 -8.44348812103271\\
107.749504089355 -8.4449634552002\\
107.774505615234 -8.44643497467041\\
107.799507141113 -8.44790363311768\\
107.824508666992 -8.44937133789063\\
107.849502563477 -8.45083618164063\\
107.874504089355 -8.45229816436768\\
107.899505615234 -8.4537525177002\\
107.924507141113 -8.45520210266113\\
107.949508666992 -8.45665168762207\\
107.974502563477 -8.45809841156006\\
107.999504089355 -8.45954513549805\\
108.024505615234 -8.46099090576172\\
108.049507141113 -8.46243095397949\\
108.074508666992 -8.46387195587158\\
108.099502563477 -8.46531391143799\\
108.124504089355 -8.46675777435303\\
108.149505615234 -8.46820449829102\\
108.174507141113 -8.469651222229\\
108.199508666992 -8.47110080718994\\
108.224502563477 -8.4725513458252\\
108.249504089355 -8.47400951385498\\
108.274505615234 -8.47547149658203\\
108.299507141113 -8.47693920135498\\
108.324508666992 -8.47840976715088\\
108.349502563477 -8.47988605499268\\
108.374504089355 -8.48136711120605\\
108.399505615234 -8.48285388946533\\
108.424507141113 -8.48434162139893\\
108.449508666992 -8.48583316802979\\
108.474502563477 -8.48732662200928\\
108.499504089355 -8.48882102966309\\
108.524505615234 -8.49031543731689\\
108.549507141113 -8.4918098449707\\
108.574508666992 -8.49330425262451\\
108.599502563477 -8.49479389190674\\
108.624504089355 -8.49627876281738\\
108.649505615234 -8.49775886535645\\
108.674507141113 -8.49923706054688\\
108.699508666992 -8.50071334838867\\
108.724502563477 -8.50218772888184\\
108.749504089355 -8.50365924835205\\
108.774505615234 -8.50512790679932\\
108.799507141113 -8.50659656524658\\
108.824508666992 -8.50806331634521\\
108.849502563477 -8.50952625274658\\
108.874504089355 -8.51098537445068\\
108.899505615234 -8.51244258880615\\
108.924507141113 -8.51389598846436\\
108.949508666992 -8.51534557342529\\
108.974502563477 -8.51679515838623\\
108.999504089355 -8.51824188232422\\
109.024505615234 -8.51968383789063\\
109.049507141113 -8.52112293243408\\
109.074508666992 -8.52256011962891\\
109.099502563477 -8.52399635314941\\
109.124504089355 -8.52542686462402\\
109.149505615234 -8.52685737609863\\
109.174507141113 -8.52828788757324\\
109.199508666992 -8.5297155380249\\
109.224502563477 -8.53114414215088\\
109.249504089355 -8.53257465362549\\
109.274505615234 -8.5340051651001\\
109.299507141113 -8.53543949127197\\
109.324508666992 -8.53687763214111\\
109.349502563477 -8.5383186340332\\
109.374504089355 -8.53976345062256\\
109.399505615234 -8.54121112823486\\
109.424507141113 -8.5426607131958\\
109.449508666992 -8.54411697387695\\
109.474502563477 -8.54557704925537\\
109.499504089355 -8.54704284667969\\
109.524505615234 -8.54851150512695\\
109.549507141113 -8.54998016357422\\
109.574508666992 -8.5514554977417\\
109.599502563477 -8.55292987823486\\
109.624504089355 -8.55440521240234\\
109.649505615234 -8.55588150024414\\
109.674507141113 -8.5573558807373\\
109.699508666992 -8.55882453918457\\
109.724502563477 -8.56028842926025\\
109.749504089355 -8.56174755096436\\
109.774505615234 -8.56320667266846\\
109.799507141113 -8.56466579437256\\
109.824508666992 -8.56612396240234\\
109.849502563477 -8.56757926940918\\
109.874504089355 -8.56903076171875\\
109.899505615234 -8.57048034667969\\
109.924507141113 -8.57192993164063\\
109.949508666992 -8.57337665557861\\
109.974502563477 -8.57482051849365\\
109.999504089355 -8.57626056671143\\
110.024505615234 -8.5777006149292\\
110.049507141113 -8.57913970947266\\
110.074508666992 -8.58057689666748\\
110.099502563477 -8.58200836181641\\
110.124504089355 -8.5834379196167\\
110.149505615234 -8.58486557006836\\
110.174507141113 -8.58629322052002\\
110.199508666992 -8.58771800994873\\
110.224502563477 -8.58913898468018\\
110.249504089355 -8.59055995941162\\
110.274505615234 -8.59198093414307\\
110.299507141113 -8.59340190887451\\
110.324508666992 -8.59482288360596\\
110.349502563477 -8.5962438583374\\
110.374504089355 -8.59766864776611\\
110.399505615234 -8.59909629821777\\
110.424507141113 -8.60052394866943\\
110.449508666992 -8.60195350646973\\
110.474502563477 -8.60338401794434\\
110.499504089355 -8.60481548309326\\
110.524505615234 -8.60625267028809\\
110.549507141113 -8.60769081115723\\
110.574508666992 -8.609130859375\\
110.599502563477 -8.61057662963867\\
110.624504089355 -8.61202335357666\\
110.649505615234 -8.6134729385376\\
110.674507141113 -8.61492252349854\\
110.699508666992 -8.61637210845947\\
110.724502563477 -8.61782073974609\\
110.749504089355 -8.61926746368408\\
110.774505615234 -8.62070941925049\\
110.799507141113 -8.62214946746826\\
110.824508666992 -8.62358951568604\\
110.849502563477 -8.62502956390381\\
110.874504089355 -8.62646961212158\\
110.899505615234 -8.62790966033936\\
110.924507141113 -8.62934684753418\\
110.949508666992 -8.630784034729\\
110.974502563477 -8.63222026824951\\
110.999504089355 -8.63365745544434\\
111.024505615234 -8.63509464263916\\
111.049507141113 -8.63652896881104\\
111.074508666992 -8.63795948028564\\
111.099502563477 -8.63938999176025\\
111.124504089355 -8.64081859588623\\
111.149505615234 -8.64224624633789\\
111.174507141113 -8.64367389678955\\
111.199508666992 -8.64509773254395\\
111.224502563477 -8.64651870727539\\
111.249504089355 -8.64793968200684\\
111.274505615234 -8.64936065673828\\
111.299507141113 -8.65077877044678\\
111.324508666992 -8.65219688415527\\
111.349502563477 -8.65361404418945\\
111.374504089355 -8.65503215789795\\
111.399505615234 -8.65644836425781\\
111.424507141113 -8.65785884857178\\
111.449508666992 -8.65926265716553\\
111.474502563477 -8.66065979003906\\
111.499504089355 -8.66205215454102\\
111.524505615234 -8.66344261169434\\
111.549507141113 -8.66483116149902\\
111.574508666992 -8.66621398925781\\
111.599502563477 -8.6675968170166\\
111.624504089355 -8.66897964477539\\
111.649505615234 -8.67036247253418\\
111.674507141113 -8.67174625396729\\
111.699508666992 -8.67313575744629\\
111.724502563477 -8.67452716827393\\
111.749504089355 -8.67592334747314\\
111.774505615234 -8.67732238769531\\
111.799507141113 -8.67872428894043\\
111.824508666992 -8.68013095855713\\
111.849502563477 -8.68154048919678\\
111.874504089355 -8.68295574188232\\
111.899505615234 -8.68437576293945\\
111.924507141113 -8.6857967376709\\
111.949508666992 -8.68721771240234\\
111.974502563477 -8.68863868713379\\
111.999504089355 -8.69005966186523\\
112.024505615234 -8.69148063659668\\
112.049507141113 -8.69290161132813\\
112.074508666992 -8.69432258605957\\
112.099502563477 -8.69574069976807\\
112.124504089355 -8.69715881347656\\
112.149505615234 -8.69857025146484\\
112.174507141113 -8.69998073577881\\
112.199508666992 -8.70139217376709\\
112.224502563477 -8.70279979705811\\
112.249504089355 -8.70420455932617\\
112.274505615234 -8.70560646057129\\
112.299507141113 -8.70700550079346\\
112.324508666992 -8.70839977264404\\
112.349502563477 -8.70979022979736\\
112.374504089355 -8.71117973327637\\
112.399505615234 -8.71256351470947\\
112.424507141113 -8.71394634246826\\
112.449508666992 -8.71532821655273\\
112.474502563477 -8.71670722961426\\
112.499504089355 -8.71808338165283\\
112.524505615234 -8.71945667266846\\
112.549507141113 -8.72082996368408\\
112.574508666992 -8.72220134735107\\
112.599502563477 -8.72357082366943\\
112.624504089355 -8.72494125366211\\
112.649505615234 -8.72631168365479\\
112.674507141113 -8.72767543792725\\
112.699508666992 -8.72903919219971\\
112.724502563477 -8.7303991317749\\
112.749504089355 -8.73176002502441\\
112.774505615234 -8.73311614990234\\
112.799507141113 -8.73447036743164\\
112.824508666992 -8.73582458496094\\
112.849502563477 -8.73717880249023\\
112.874504089355 -8.73853397369385\\
112.899505615234 -8.73988914489746\\
112.924507141113 -8.74124336242676\\
112.949508666992 -8.74259757995605\\
112.974502563477 -8.74395751953125\\
112.999504089355 -8.74532032012939\\
113.024505615234 -8.74668502807617\\
113.049507141113 -8.74805641174316\\
113.074508666992 -8.74942970275879\\
113.099502563477 -8.75080585479736\\
113.124504089355 -8.7521858215332\\
113.149505615234 -8.75356578826904\\
113.174507141113 -8.75494575500488\\
113.199508666992 -8.75632572174072\\
113.224502563477 -8.75770473480225\\
113.249504089355 -8.75908470153809\\
113.274505615234 -8.76046466827393\\
113.299507141113 -8.76184463500977\\
113.324508666992 -8.76322078704834\\
113.349502563477 -8.76459407806396\\
113.374504089355 -8.76596546173096\\
113.399505615234 -8.76733303070068\\
113.424507141113 -8.76869678497314\\
113.449508666992 -8.77006053924561\\
113.474502563477 -8.77142238616943\\
113.499504089355 -8.77278327941895\\
113.524505615234 -8.77413845062256\\
113.549507141113 -8.77549266815186\\
113.574508666992 -8.7768440246582\\
113.599502563477 -8.77818870544434\\
113.624504089355 -8.77953052520752\\
113.649505615234 -8.78087043762207\\
113.674507141113 -8.78220558166504\\
113.699508666992 -8.78354072570801\\
113.724502563477 -8.78487300872803\\
113.749504089355 -8.78620529174805\\
113.774505615234 -8.78753280639648\\
113.799507141113 -8.78885555267334\\
113.824508666992 -8.79017353057861\\
113.849502563477 -8.79148960113525\\
113.874504089355 -8.79280567169189\\
113.899505615234 -8.7941198348999\\
113.924507141113 -8.79543113708496\\
113.949508666992 -8.79673767089844\\
113.974502563477 -8.79804420471191\\
113.999504089355 -8.79935073852539\\
114.024505615234 -8.80065727233887\\
114.049507141113 -8.80196380615234\\
114.074508666992 -8.80327606201172\\
114.099502563477 -8.80458831787109\\
114.124504089355 -8.80590152740479\\
114.149505615234 -8.80721473693848\\
114.174507141113 -8.8085298538208\\
114.199508666992 -8.80984783172607\\
114.224502563477 -8.81117057800293\\
114.249504089355 -8.81249618530273\\
114.274505615234 -8.81382465362549\\
114.299507141113 -8.81515693664551\\
114.324508666992 -8.81649017333984\\
114.349502563477 -8.81782531738281\\
114.374504089355 -8.81916046142578\\
114.399505615234 -8.82049560546875\\
114.424507141113 -8.82183074951172\\
114.449508666992 -8.82316589355469\\
114.474502563477 -8.82450103759766\\
114.499504089355 -8.82583618164063\\
114.524505615234 -8.82716846466064\\
114.549507141113 -8.82850074768066\\
114.574508666992 -8.82983303070068\\
114.599502563477 -8.83116054534912\\
114.624504089355 -8.83248615264893\\
114.649505615234 -8.83380889892578\\
114.674507141113 -8.83512592315674\\
114.699508666992 -8.83644199371338\\
114.724502563477 -8.83775424957275\\
114.749504089355 -8.83906269073486\\
114.774505615234 -8.84036922454834\\
114.799507141113 -8.84167289733887\\
114.824508666992 -8.84297657012939\\
114.849502563477 -8.84427547454834\\
114.874504089355 -8.84557151794434\\
114.899505615234 -8.84686470031738\\
114.924507141113 -8.84815120697021\\
114.949508666992 -8.8494291305542\\
114.974502563477 -8.85070419311523\\
114.999504089355 -8.851975440979\\
115.024505615234 -8.85324382781982\\
115.049507141113 -8.85451221466064\\
115.074508666992 -8.85577774047852\\
115.099502563477 -8.85704231262207\\
115.124504089355 -8.85830783843994\\
115.149505615234 -8.85957336425781\\
115.174507141113 -8.86084079742432\\
115.199508666992 -8.86210918426514\\
115.224502563477 -8.86337757110596\\
115.249504089355 -8.86465263366699\\
115.274505615234 -8.86592769622803\\
115.299507141113 -8.86720371246338\\
115.324508666992 -8.86848068237305\\
115.349502563477 -8.86975860595703\\
115.374504089355 -8.87104320526123\\
115.399505615234 -8.87233066558838\\
115.424507141113 -8.87361812591553\\
115.449508666992 -8.87491226196289\\
115.474502563477 -8.87620830535889\\
115.499504089355 -8.8775053024292\\
115.524505615234 -8.87880706787109\\
115.549507141113 -8.88011074066162\\
115.574508666992 -8.88141441345215\\
115.599502563477 -8.88271713256836\\
115.624504089355 -8.88402080535889\\
115.649505615234 -8.88532447814941\\
115.674507141113 -8.88662528991699\\
115.699508666992 -8.8879222869873\\
115.724502563477 -8.8892183303833\\
115.749504089355 -8.89051532745361\\
115.774505615234 -8.89180850982666\\
115.799507141113 -8.89309597015381\\
115.824508666992 -8.89437866210938\\
115.849502563477 -8.89565658569336\\
115.874504089355 -8.89693355560303\\
115.899505615234 -8.89820861816406\\
115.924507141113 -8.89947891235352\\
115.949508666992 -8.90074729919434\\
115.974502563477 -8.90201377868652\\
115.999504089355 -8.90327739715576\\
116.024505615234 -8.90453624725342\\
116.049507141113 -8.90579319000244\\
116.074508666992 -8.90704536437988\\
116.099502563477 -8.90829467773438\\
116.124504089355 -8.90954113006592\\
116.149505615234 -8.91078281402588\\
116.174507141113 -8.91202259063721\\
116.199508666992 -8.91326236724854\\
116.224502563477 -8.91450214385986\\
116.249504089355 -8.91574001312256\\
116.274505615234 -8.91697692871094\\
116.299507141113 -8.91821384429932\\
116.324508666992 -8.9194507598877\\
116.349502563477 -8.92068862915039\\
116.374504089355 -8.92192840576172\\
116.399505615234 -8.92316818237305\\
116.424507141113 -8.92440795898438\\
116.449508666992 -8.92565441131592\\
116.474502563477 -8.92690372467041\\
116.499504089355 -8.92815494537354\\
116.524505615234 -8.92941093444824\\
116.549507141113 -8.9306697845459\\
116.574508666992 -8.93193435668945\\
116.599502563477 -8.93320274353027\\
116.624504089355 -8.93447113037109\\
116.649505615234 -8.93573951721191\\
116.674507141113 -8.93701171875\\
116.699508666992 -8.93828678131104\\
116.724502563477 -8.93956089019775\\
116.749504089355 -8.94083595275879\\
116.774505615234 -8.94211101531982\\
116.799507141113 -8.94338512420654\\
116.824508666992 -8.94465351104736\\
116.849502563477 -8.94592189788818\\
116.874504089355 -8.94718837738037\\
116.899505615234 -8.94845294952393\\
116.924507141113 -8.94971179962158\\
116.949508666992 -8.95096874237061\\
116.974502563477 -8.95221996307373\\
116.999504089355 -8.95346927642822\\
117.024505615234 -8.95471572875977\\
117.049507141113 -8.95595550537109\\
117.074508666992 -8.95719528198242\\
117.099502563477 -8.95843124389648\\
117.124504089355 -8.95966148376465\\
117.149505615234 -8.96088886260986\\
117.174507141113 -8.96210956573486\\
117.199508666992 -8.96332931518555\\
117.224502563477 -8.96454524993896\\
117.249504089355 -8.9657564163208\\
117.274505615234 -8.96696758270264\\
117.299507141113 -8.96817684173584\\
117.324508666992 -8.96938037872314\\
117.349502563477 -8.97058200836182\\
117.374504089355 -8.97178363800049\\
117.399505615234 -8.97298526763916\\
117.424507141113 -8.97418689727783\\
117.449508666992 -8.9753885269165\\
117.474502563477 -8.97659206390381\\
117.499504089355 -8.97779369354248\\
117.524505615234 -8.97899627685547\\
117.549507141113 -8.98020553588867\\
117.574508666992 -8.98141670227051\\
117.599502563477 -8.98263168334961\\
117.624504089355 -8.98385238647461\\
117.649505615234 -8.98507595062256\\
117.674507141113 -8.98630332946777\\
117.699508666992 -8.98753261566162\\
117.724502563477 -8.9887638092041\\
117.749504089355 -8.9900016784668\\
117.774505615234 -8.99124145507813\\
117.799507141113 -8.99248123168945\\
117.824508666992 -8.99372100830078\\
117.849502563477 -8.99496078491211\\
117.874504089355 -8.99620056152344\\
117.899505615234 -8.99744033813477\\
117.924507141113 -8.99867725372314\\
117.949508666992 -8.99991321563721\\
117.974502563477 -9.00114345550537\\
117.999504089355 -9.00237369537354\\
118.024505615234 -9.0036039352417\\
118.049507141113 -9.00483226776123\\
118.074508666992 -9.00605583190918\\
118.099502563477 -9.00727558135986\\
118.124504089355 -9.0084924697876\\
118.149505615234 -9.00970363616943\\
118.174507141113 -9.01091289520264\\
118.199508666992 -9.01211643218994\\
118.224502563477 -9.01331806182861\\
118.249504089355 -9.01451683044434\\
118.274505615234 -9.01570892333984\\
118.299507141113 -9.01690101623535\\
118.324508666992 -9.01809310913086\\
118.349502563477 -9.01928520202637\\
118.374504089355 -9.02047729492188\\
118.399505615234 -9.02166652679443\\
118.424507141113 -9.02285003662109\\
118.449508666992 -9.0240306854248\\
118.474502563477 -9.02520942687988\\
118.499504089355 -9.02638530731201\\
118.524505615234 -9.02755832672119\\
118.549507141113 -9.02873134613037\\
118.574508666992 -9.02990436553955\\
118.599502563477 -9.03107738494873\\
118.624504089355 -9.03225040435791\\
118.649505615234 -9.03342056274414\\
118.674507141113 -9.03459072113037\\
118.699508666992 -9.03576278686523\\
118.724502563477 -9.03693580627441\\
118.749504089355 -9.03811168670654\\
118.774505615234 -9.03929138183594\\
118.799507141113 -9.04047107696533\\
118.824508666992 -9.04165267944336\\
118.849502563477 -9.0428352355957\\
118.874504089355 -9.04401779174805\\
118.899505615234 -9.04520702362061\\
118.924507141113 -9.04639720916748\\
118.949508666992 -9.04758930206299\\
118.974502563477 -9.0487813949585\\
118.999504089355 -9.04997730255127\\
119.024505615234 -9.05117607116699\\
119.049507141113 -9.05237579345703\\
119.074508666992 -9.05357456207275\\
119.099502563477 -9.05477237701416\\
119.124504089355 -9.05596542358398\\
119.149505615234 -9.05715656280518\\
119.174507141113 -9.05834197998047\\
119.199508666992 -9.05952453613281\\
119.224502563477 -9.06070423126221\\
119.249504089355 -9.06188106536865\\
119.274505615234 -9.06305408477783\\
119.299507141113 -9.0642261505127\\
119.324508666992 -9.06539344787598\\
119.349502563477 -9.06655597686768\\
119.374504089355 -9.06771659851074\\
119.399505615234 -9.06887054443359\\
119.424507141113 -9.07002353668213\\
119.449508666992 -9.07117080688477\\
119.474502563477 -9.07231426239014\\
119.499504089355 -9.07345581054688\\
119.524505615234 -9.0745906829834\\
119.549507141113 -9.07572555541992\\
119.574508666992 -9.07685852050781\\
119.599502563477 -9.07798480987549\\
119.624504089355 -9.07911014556885\\
119.649505615234 -9.08023357391357\\
119.674507141113 -9.08135604858398\\
119.699508666992 -9.08247852325439\\
119.724502563477 -9.08360290527344\\
119.749504089355 -9.0847282409668\\
119.774505615234 -9.08585357666016\\
119.799507141113 -9.08697891235352\\
119.824508666992 -9.08810424804688\\
119.849502563477 -9.08923625946045\\
119.874504089355 -9.09037113189697\\
119.899505615234 -9.0915060043335\\
119.924507141113 -9.0926456451416\\
119.949508666992 -9.09378910064697\\
119.974502563477 -9.09492969512939\\
119.999504089355 -9.09607124328613\\
120.024505615234 -9.0972146987915\\
120.049507141113 -9.09835910797119\\
120.074508666992 -9.09950351715088\\
120.099502563477 -9.10064792633057\\
120.124504089355 -9.10179233551025\\
120.149505615234 -9.10293674468994\\
120.174507141113 -9.10408115386963\\
120.199508666992 -9.10522556304932\\
120.224502563477 -9.10636806488037\\
120.249504089355 -9.10750770568848\\
120.274505615234 -9.108642578125\\
120.299507141113 -9.10977745056152\\
120.324508666992 -9.11091136932373\\
120.349502563477 -9.11204051971436\\
120.374504089355 -9.11316585540771\\
120.399505615234 -9.11428928375244\\
120.424507141113 -9.11541366577148\\
120.449508666992 -9.11653232574463\\
120.474502563477 -9.11764812469482\\
120.499504089355 -9.1187629699707\\
120.524505615234 -9.11987590789795\\
120.549507141113 -9.12098217010498\\
120.574508666992 -9.12208843231201\\
120.599502563477 -9.12318992614746\\
120.624504089355 -9.12428665161133\\
120.649505615234 -9.1253833770752\\
120.674507141113 -9.12647819519043\\
120.699508666992 -9.12756824493408\\
120.724502563477 -9.12865543365479\\
120.749504089355 -9.12974262237549\\
120.774505615234 -9.13083171844482\\
120.799507141113 -9.13192749023438\\
120.824508666992 -9.13302421569824\\
120.849502563477 -9.13411903381348\\
120.874504089355 -9.13521289825439\\
120.899505615234 -9.13630676269531\\
120.924507141113 -9.13740062713623\\
120.949508666992 -9.13849449157715\\
120.974502563477 -9.13958835601807\\
120.999504089355 -9.14068508148193\\
121.024505615234 -9.1417818069458\\
121.049507141113 -9.14287853240967\\
121.074508666992 -9.14398002624512\\
121.099502563477 -9.14507961273193\\
121.124504089355 -9.1461763381958\\
121.149505615234 -9.1472749710083\\
121.174507141113 -9.14837837219238\\
121.199508666992 -9.1494836807251\\
121.224502563477 -9.15058994293213\\
121.249504089355 -9.15169620513916\\
121.274505615234 -9.15280246734619\\
121.299507141113 -9.15390872955322\\
121.324508666992 -9.15501499176025\\
121.349502563477 -9.15612125396729\\
121.374504089355 -9.15722751617432\\
121.399505615234 -9.15833377838135\\
121.424507141113 -9.15944004058838\\
121.449508666992 -9.16054630279541\\
121.474502563477 -9.16165065765381\\
121.499504089355 -9.16275405883789\\
121.524505615234 -9.16385459899902\\
121.549507141113 -9.16495132446289\\
121.574508666992 -9.16604804992676\\
121.599502563477 -9.16714000701904\\
121.624504089355 -9.16822624206543\\
121.649505615234 -9.16931056976318\\
121.674507141113 -9.17039012908936\\
121.699508666992 -9.17146778106689\\
121.724502563477 -9.17254447937012\\
121.749504089355 -9.17361927032471\\
121.774505615234 -9.17469120025635\\
121.799507141113 -9.17575931549072\\
121.824508666992 -9.1768274307251\\
121.849502563477 -9.17789554595947\\
121.874504089355 -9.1789608001709\\
121.899505615234 -9.18002319335938\\
121.924507141113 -9.18108177185059\\
121.949508666992 -9.1821403503418\\
121.974502563477 -9.18319892883301\\
121.999504089355 -9.18425750732422\\
122.024505615234 -9.18531608581543\\
122.049507141113 -9.18637466430664\\
122.074508666992 -9.18743324279785\\
122.099502563477 -9.18849182128906\\
122.124504089355 -9.18955135345459\\
122.149505615234 -9.19061660766602\\
122.174507141113 -9.19167804718018\\
122.199508666992 -9.19273662567139\\
122.224502563477 -9.19379329681396\\
122.249504089355 -9.19484901428223\\
122.274505615234 -9.19590473175049\\
122.299507141113 -9.19696044921875\\
122.324508666992 -9.19801044464111\\
122.349502563477 -9.19905948638916\\
122.374504089355 -9.20010852813721\\
122.399505615234 -9.20115756988525\\
122.424507141113 -9.20220565795898\\
122.449508666992 -9.20325183868408\\
122.474502563477 -9.20429515838623\\
122.499504089355 -9.20533275604248\\
122.524505615234 -9.20636653900146\\
122.549507141113 -9.20739650726318\\
122.574508666992 -9.20842552185059\\
122.599502563477 -9.20945167541504\\
122.624504089355 -9.21047210693359\\
122.649505615234 -9.2114896774292\\
122.674507141113 -9.21250343322754\\
122.699508666992 -9.21351432800293\\
122.724502563477 -9.21451950073242\\
122.749504089355 -9.21551895141602\\
122.774505615234 -9.21651744842529\\
122.799507141113 -9.21751594543457\\
122.824508666992 -9.21850776672363\\
122.849502563477 -9.21949195861816\\
122.874504089355 -9.22047424316406\\
122.899505615234 -9.22145652770996\\
122.924507141113 -9.22243881225586\\
122.949508666992 -9.22342109680176\\
122.974502563477 -9.22440338134766\\
122.999504089355 -9.22538566589355\\
123.024505615234 -9.22636413574219\\
123.049507141113 -9.22733688354492\\
123.074508666992 -9.22830963134766\\
123.099502563477 -9.22927856445313\\
123.124504089355 -9.23024463653564\\
123.149505615234 -9.23120784759521\\
123.174507141113 -9.23216819763184\\
123.199508666992 -9.23312664031982\\
123.224502563477 -9.23407936096191\\
123.249504089355 -9.23503017425537\\
123.274505615234 -9.23598003387451\\
123.299507141113 -9.23692417144775\\
123.324508666992 -9.237868309021\\
123.349502563477 -9.23880481719971\\
123.374504089355 -9.23973941802979\\
123.399505615234 -9.24067401885986\\
123.424507141113 -9.24160671234131\\
123.449508666992 -9.24253273010254\\
123.474502563477 -9.24345779418945\\
123.499504089355 -9.24438285827637\\
123.524505615234 -9.24530601501465\\
123.549507141113 -9.24622344970703\\
123.574508666992 -9.24713897705078\\
123.599502563477 -9.24805450439453\\
123.624504089355 -9.24896335601807\\
123.649505615234 -9.24986934661865\\
123.674507141113 -9.25077438354492\\
123.699508666992 -9.25167751312256\\
123.724502563477 -9.25257396697998\\
123.749504089355 -9.25346946716309\\
123.774505615234 -9.25436305999756\\
123.799507141113 -9.25525093078613\\
123.824508666992 -9.25613689422607\\
123.849502563477 -9.25701427459717\\
123.874504089355 -9.25789165496826\\
123.899505615234 -9.25876903533936\\
123.924507141113 -9.25964546203613\\
123.949508666992 -9.26051330566406\\
123.974502563477 -9.26138114929199\\
123.999504089355 -9.26224613189697\\
124.024505615234 -9.26310634613037\\
124.049507141113 -9.26396465301514\\
124.074508666992 -9.26482105255127\\
124.099502563477 -9.26567554473877\\
124.124504089355 -9.26653003692627\\
124.149505615234 -9.26737880706787\\
124.174507141113 -9.26822757720947\\
124.199508666992 -9.26907348632813\\
124.224502563477 -9.2699146270752\\
124.249504089355 -9.27075386047363\\
124.274505615234 -9.27159309387207\\
124.299507141113 -9.27243232727051\\
124.324508666992 -9.27327919006348\\
124.349502563477 -9.27412796020508\\
124.374504089355 -9.27497005462646\\
124.399505615234 -9.27580261230469\\
124.424507141113 -9.27663135528564\\
124.449508666992 -9.27745151519775\\
124.474502563477 -9.27826976776123\\
124.499504089355 -9.27908039093018\\
124.524505615234 -9.27988624572754\\
124.549507141113 -9.28068733215332\\
124.574508666992 -9.2814884185791\\
124.599502563477 -9.28228569030762\\
124.624504089355 -9.28307723999023\\
124.649505615234 -9.28386878967285\\
124.674507141113 -9.28465843200684\\
124.699508666992 -9.2854471206665\\
124.724502563477 -9.28623008728027\\
124.749504089355 -9.28701210021973\\
124.774505615234 -9.28779411315918\\
124.799507141113 -9.28857612609863\\
124.824508666992 -9.28935813903809\\
124.849502563477 -9.29014015197754\\
124.874504089355 -9.29092597961426\\
124.899505615234 -9.29171752929688\\
124.924507141113 -9.29250907897949\\
124.949508666992 -9.29330062866211\\
124.974502563477 -9.29409408569336\\
124.999504089355 -9.29488658905029\\
125.024505615234 -9.29567718505859\\
125.049507141113 -9.29646015167236\\
125.074508666992 -9.29724025726318\\
125.099502563477 -9.29801273345947\\
125.124504089355 -9.29878520965576\\
125.149505615234 -9.29955577850342\\
125.174507141113 -9.30031871795654\\
125.199508666992 -9.30107498168945\\
125.224502563477 -9.30182838439941\\
125.249504089355 -9.30257892608643\\
125.274505615234 -9.30332279205322\\
125.299507141113 -9.30405902862549\\
125.324508666992 -9.30479335784912\\
125.349502563477 -9.30552387237549\\
125.374504089355 -9.30624866485596\\
125.399505615234 -9.30697154998779\\
125.424507141113 -9.30769348144531\\
125.449508666992 -9.30841159820557\\
125.474502563477 -9.30913257598877\\
125.499504089355 -9.30985069274902\\
125.524505615234 -9.31056594848633\\
125.549507141113 -9.31128120422363\\
125.574508666992 -9.31199645996094\\
125.599502563477 -9.31271171569824\\
125.624504089355 -9.31342697143555\\
125.649505615234 -9.31414127349854\\
125.674507141113 -9.31485557556152\\
125.699508666992 -9.31557083129883\\
125.724502563477 -9.31628227233887\\
125.749504089355 -9.31698799133301\\
125.774505615234 -9.31769371032715\\
125.799507141113 -9.31839942932129\\
125.824508666992 -9.31910514831543\\
125.849502563477 -9.31981086730957\\
125.874504089355 -9.32051658630371\\
125.899505615234 -9.32122230529785\\
125.924507141113 -9.32192897796631\\
125.949508666992 -9.32263469696045\\
125.974502563477 -9.32334041595459\\
125.999504089355 -9.32404041290283\\
126.024505615234 -9.32473659515381\\
126.049507141113 -9.32542705535889\\
126.074508666992 -9.3261137008667\\
126.099502563477 -9.32679843902588\\
126.124504089355 -9.32747745513916\\
126.149505615234 -9.32815265655518\\
126.174507141113 -9.32881927490234\\
126.199508666992 -9.32947731018066\\
126.224502563477 -9.33013534545898\\
126.249504089355 -9.3307933807373\\
126.274505615234 -9.33144378662109\\
126.299507141113 -9.33209228515625\\
126.324508666992 -9.33273792266846\\
126.349502563477 -9.33338260650635\\
126.374504089355 -9.33402347564697\\
126.399505615234 -9.33466243743896\\
126.424507141113 -9.33530139923096\\
126.449508666992 -9.33594036102295\\
126.474502563477 -9.33657550811768\\
126.499504089355 -9.33721256256104\\
126.524505615234 -9.33785152435303\\
126.549507141113 -9.33849048614502\\
126.574508666992 -9.33912944793701\\
126.599502563477 -9.339768409729\\
126.624504089355 -9.340407371521\\
126.649505615234 -9.34104633331299\\
126.674507141113 -9.34168529510498\\
126.699508666992 -9.34232425689697\\
126.724502563477 -9.34296321868896\\
126.749504089355 -9.34360218048096\\
126.774505615234 -9.34424114227295\\
126.799507141113 -9.34488010406494\\
126.824508666992 -9.34551239013672\\
126.849502563477 -9.34614181518555\\
126.874504089355 -9.34677124023438\\
126.899505615234 -9.34739971160889\\
126.924507141113 -9.34802055358887\\
126.949508666992 -9.3486385345459\\
126.974502563477 -9.3492488861084\\
126.999504089355 -9.34985828399658\\
127.024505615234 -9.35046100616455\\
127.049507141113 -9.35106182098389\\
127.074508666992 -9.35165691375732\\
127.099502563477 -9.3522481918335\\
127.124504089355 -9.35283946990967\\
127.149505615234 -9.35342884063721\\
127.174507141113 -9.35401725769043\\
127.199508666992 -9.35460758209229\\
127.224502563477 -9.35519886016846\\
127.249504089355 -9.35579013824463\\
127.274505615234 -9.35637664794922\\
127.299507141113 -9.35695838928223\\
127.324508666992 -9.35753345489502\\
127.349502563477 -9.35810565948486\\
127.374504089355 -9.35867595672607\\
127.399505615234 -9.35924530029297\\
127.424507141113 -9.3598108291626\\
127.449508666992 -9.36037349700928\\
127.474502563477 -9.36093616485596\\
127.499504089355 -9.36149883270264\\
127.524505615234 -9.36206150054932\\
127.549507141113 -9.36262512207031\\
127.574508666992 -9.36319446563721\\
127.599502563477 -9.36376571655273\\
127.624504089355 -9.36433792114258\\
127.649505615234 -9.36491012573242\\
127.674507141113 -9.36548233032227\\
127.699508666992 -9.36605453491211\\
127.724502563477 -9.36662673950195\\
127.749504089355 -9.36720561981201\\
127.774505615234 -9.36777782440186\\
127.799507141113 -9.36835098266602\\
127.824508666992 -9.36892318725586\\
127.849502563477 -9.3694953918457\\
127.874504089355 -9.37006759643555\\
127.899505615234 -9.37063980102539\\
127.924507141113 -9.3712100982666\\
127.949508666992 -9.37177562713623\\
127.974502563477 -9.37233829498291\\
127.999504089355 -9.37290096282959\\
128.024505615234 -9.37346363067627\\
128.049499511719 -9.37402629852295\\
128.074508666992 -9.37458896636963\\
128.099502563477 -9.37515163421631\\
128.12451171875 -9.37570858001709\\
128.149505615234 -9.37625980377197\\
128.174499511719 -9.37680244445801\\
128.199508666992 -9.37734127044678\\
128.224502563477 -9.37787342071533\\
128.24951171875 -9.37839794158936\\
128.274505615234 -9.37892150878906\\
128.299499511719 -9.37944412231445\\
128.324508666992 -9.37995910644531\\
128.349502563477 -9.38047409057617\\
128.37451171875 -9.38098907470703\\
128.399505615234 -9.38150978088379\\
128.424499511719 -9.38203239440918\\
128.449508666992 -9.38254928588867\\
128.474502563477 -9.38306617736816\\
128.49951171875 -9.38358974456787\\
128.524505615234 -9.38411331176758\\
128.549499511719 -9.3846378326416\\
128.574508666992 -9.38516521453857\\
128.599502563477 -9.38569831848145\\
128.62451171875 -9.38623237609863\\
128.649505615234 -9.38676929473877\\
128.674499511719 -9.3873119354248\\
128.699508666992 -9.38785552978516\\
128.724502563477 -9.38839912414551\\
128.74951171875 -9.38894271850586\\
128.774505615234 -9.38948631286621\\
128.799499511719 -9.39002990722656\\
128.824508666992 -9.39057350158691\\
128.849502563477 -9.39111614227295\\
128.87451171875 -9.39165878295898\\
128.899505615234 -9.39219570159912\\
128.924499511719 -9.39272975921631\\
128.949508666992 -9.3932638168335\\
128.974502563477 -9.39379787445068\\
128.99951171875 -9.39433193206787\\
129.024505615234 -9.39486789703369\\
129.049499511719 -9.39541149139404\\
129.074508666992 -9.39596557617188\\
129.099502563477 -9.3965311050415\\
129.12451171875 -9.39709854125977\\
129.149505615234 -9.39764785766602\\
129.174499511719 -9.39817047119141\\
129.199508666992 -9.3986701965332\\
129.224502563477 -9.39915466308594\\
129.24951171875 -9.39962673187256\\
129.274505615234 -9.40009880065918\\
129.299499511719 -9.40058994293213\\
129.324508666992 -9.40108489990234\\
129.349502563477 -9.40157890319824\\
129.37451171875 -9.40207481384277\\
129.399505615234 -9.4025707244873\\
129.424499511719 -9.40306663513184\\
129.449508666992 -9.40356254577637\\
129.474502563477 -9.4040584564209\\
129.49951171875 -9.40455436706543\\
129.524505615234 -9.40505027770996\\
129.549499511719 -9.40555477142334\\
129.574508666992 -9.40606307983398\\
129.599502563477 -9.40657711029053\\
129.62451171875 -9.40709781646729\\
129.649505615234 -9.40763282775879\\
129.674499511719 -9.40818405151367\\
129.699508666992 -9.40875053405762\\
129.724502563477 -9.40932846069336\\
129.74951171875 -9.40991020202637\\
129.774505615234 -9.41049194335938\\
129.799499511719 -9.41107368469238\\
129.824508666992 -9.41165065765381\\
129.849502563477 -9.4122142791748\\
129.87451171875 -9.41276264190674\\
129.899505615234 -9.41329574584961\\
129.924499511719 -9.41381740570068\\
129.949508666992 -9.41433048248291\\
129.974502563477 -9.41483402252197\\
129.99951171875 -9.41532897949219\\
130.024505615234 -9.41581344604492\\
130.049499511719 -9.41628932952881\\
130.074508666992 -9.41675662994385\\
130.099502563477 -9.41721439361572\\
130.12451171875 -9.4176664352417\\
130.149505615234 -9.41811466217041\\
130.174499511719 -9.41856288909912\\
130.199508666992 -9.41901111602783\\
130.224502563477 -9.41945934295654\\
130.24951171875 -9.41991233825684\\
130.274505615234 -9.42036914825439\\
130.299499511719 -9.42083549499512\\
130.324508666992 -9.421311378479\\
130.349502563477 -9.42179489135742\\
130.37451171875 -9.42228603363037\\
130.399505615234 -9.42278480529785\\
130.424499511719 -9.42329502105713\\
130.449508666992 -9.42381763458252\\
130.474502563477 -9.42435264587402\\
130.49951171875 -9.42489624023438\\
130.524505615234 -9.42544460296631\\
130.549499511719 -9.42599582672119\\
130.574508666992 -9.42654514312744\\
130.599502563477 -9.42708873748779\\
130.62451171875 -9.42763137817383\\
130.649505615234 -9.42816543579102\\
130.674499511719 -9.42869663238525\\
130.699508666992 -9.42922115325928\\
130.724502563477 -9.42973518371582\\
130.74951171875 -9.43023872375488\\
130.774505615234 -9.43073272705078\\
130.799499511719 -9.43121814727783\\
130.824508666992 -9.43169689178467\\
130.849502563477 -9.43217182159424\\
130.87451171875 -9.43263912200928\\
130.899505615234 -9.43310546875\\
130.924499511719 -9.43356513977051\\
130.949508666992 -9.43402290344238\\
130.974502563477 -9.43448066711426\\
130.99951171875 -9.43494129180908\\
131.024505615234 -9.4354076385498\\
131.049499511719 -9.43587493896484\\
131.074508666992 -9.43634605407715\\
131.099502563477 -9.43682289123535\\
131.12451171875 -9.43730640411377\\
131.149505615234 -9.43779468536377\\
131.174499511719 -9.4382905960083\\
131.199508666992 -9.43879222869873\\
131.224502563477 -9.43929862976074\\
131.24951171875 -9.43981266021729\\
131.274505615234 -9.44032955169678\\
131.299499511719 -9.44085311889648\\
131.324508666992 -9.44137763977051\\
131.349502563477 -9.44190216064453\\
131.37451171875 -9.44242668151855\\
131.399505615234 -9.44295120239258\\
131.424499511719 -9.4434757232666\\
131.449508666992 -9.44399929046631\\
131.474502563477 -9.44451808929443\\
131.49951171875 -9.44503307342529\\
131.524505615234 -9.44554424285889\\
131.549499511719 -9.4460506439209\\
131.574508666992 -9.44657039642334\\
131.599502563477 -9.44710636138916\\
131.62451171875 -9.44765377044678\\
131.649505615234 -9.44820594787598\\
131.674499511719 -9.44876480102539\\
131.699508666992 -9.44932270050049\\
131.724502563477 -9.44987010955811\\
131.74951171875 -9.45042610168457\\
131.774505615234 -9.45099639892578\\
131.799499511719 -9.45158195495605\\
131.824508666992 -9.45219898223877\\
131.849502563477 -9.45289707183838\\
131.87451171875 -9.45393657684326\\
131.899505615234 -9.45722103118896\\
131.924499511719 -9.47256183624268\\
131.949508666992 -9.64586639404297\\
131.974502563477 -10.0631074905396\\
131.99951171875 -10.4420013427734\\
132.024505615234 -10.7785997390747\\
132.049499511719 -11.0687284469604\\
132.074508666992 -11.3105611801147\\
132.099502563477 -11.5389890670776\\
132.12451171875 -11.7696905136108\\
132.149505615234 -11.9757375717163\\
132.174499511719 -12.143292427063\\
132.199508666992 -12.2639417648315\\
132.224502563477 -12.361123085022\\
132.24951171875 -12.7582092285156\\
132.274047851563 -13.0830516815186\\
132.290390014648 -13.2731809616089\\
132.315338134766 -13.6620054244995\\
132.33154296875 -13.7537336349487\\
132.35188293457 -13.7603950500488\\
132.376876831055 -13.7657165527344\\
132.401870727539 -13.7700853347778\\
132.426879882813 -13.7743692398071\\
132.451873779297 -13.7784795761108\\
132.47688293457 -13.7824487686157\\
132.495971679688 -13.7856044769287\\
132.520980834961 -13.7902936935425\\
132.545974731445 -13.8811073303223\\
132.57096862793 -14.4701499938965\\
132.595977783203 -14.6366672515869\\
132.620971679688 -14.6798048019409\\
132.645980834961 -14.7252893447876\\
132.670974731445 -14.980486869812\\
132.69596862793 -15.2522764205933\\
132.720977783203 -15.4208059310913\\
132.745971679688 -15.5543918609619\\
132.770980834961 -15.6223735809326\\
132.795974731445 -15.8481874465942\\
132.82096862793 -16.494556427002\\
132.845977783203 -16.8941612243652\\
132.870971679688 -17.0088520050049\\
132.895980834961 -17.0115756988525\\
132.920974731445 -17.0178699493408\\
132.94596862793 -17.1103496551514\\
132.970977783203 -17.7168788909912\\
132.995971679688 -17.992338180542\\
133.020980834961 -18.0122261047363\\
133.045974731445 -18.0128936767578\\
133.07096862793 -18.0131454467773\\
133.095977783203 -18.0139980316162\\
133.120819091797 -18.0671653747559\\
133.145690917969 -18.6833190917969\\
133.170700073242 -19.2264995574951\\
133.195693969727 -19.4008483886719\\
133.220687866211 -19.4051074981689\\
133.245697021484 -19.4057369232178\\
133.270690917969 -19.4062004089355\\
133.295700073242 -19.4067897796631\\
133.320037841797 -19.4072895050049\\
133.345031738281 -19.4079113006592\\
133.370040893555 -19.4089508056641\\
133.395034790039 -19.4097480773926\\
133.420028686523 -19.4103412628174\\
133.445037841797 -19.4109115600586\\
133.470031738281 -19.4115333557129\\
133.495040893555 -19.4121913909912\\
133.520034790039 -19.41282081604\\
133.545028686523 -19.413387298584\\
133.570037841797 -19.4139347076416\\
133.595031738281 -19.4144687652588\\
133.620040893555 -19.4149913787842\\
133.645034790039 -19.4154720306396\\
133.670028686523 -19.4158630371094\\
133.695037841797 -19.4162883758545\\
133.720031738281 -19.4167098999023\\
133.745040893555 -19.4171619415283\\
133.770034790039 -19.4175548553467\\
133.795028686523 -19.4178581237793\\
133.820037841797 -19.4181537628174\\
133.845031738281 -19.4185352325439\\
133.870040893555 -19.4190216064453\\
133.895034790039 -19.4196090698242\\
133.920028686523 -19.4203071594238\\
133.945037841797 -19.4212703704834\\
133.970031738281 -19.4231300354004\\
133.995040893555 -19.4295597076416\\
134.020034790039 -19.4574222564697\\
134.042602539063 -19.5898571014404\\
134.045257568359 -19.6176586151123\\
134.048431396484 -19.6454982757568\\
134.053085327148 -19.6743087768555\\
134.064117431641 -19.6945152282715\\
134.079299926758 -19.7057094573975\\
134.081481933594 -19.7072505950928\\
134.090911865234 -19.7101650238037\\
134.112213134766 -19.7119979858398\\
134.13720703125 -19.7129669189453\\
134.162200927734 -19.7142295837402\\
134.187210083008 -19.7153930664063\\
134.212203979492 -19.7161769866943\\
134.237213134766 -19.7164936065674\\
134.26220703125 -19.7164478302002\\
134.287200927734 -19.7162246704102\\
134.312210083008 -19.7160530090332\\
134.337203979492 -19.7160930633545\\
134.362213134766 -19.7162570953369\\
134.38720703125 -19.7166023254395\\
134.412200927734 -19.7170600891113\\
134.437210083008 -19.7175464630127\\
134.462203979492 -19.7180137634277\\
134.487213134766 -19.7184352874756\\
134.51220703125 -19.7188014984131\\
134.537200927734 -19.7190933227539\\
134.562210083008 -19.719367980957\\
134.587203979492 -19.7195663452148\\
134.612213134766 -19.7197265625\\
134.63720703125 -19.7198791503906\\
134.662200927734 -19.7200317382813\\
134.687210083008 -19.7202033996582\\
134.712203979492 -19.7204093933105\\
134.737213134766 -19.7206535339355\\
134.76220703125 -19.7209167480469\\
134.787200927734 -19.7211837768555\\
134.812210083008 -19.7214508056641\\
134.837203979492 -19.7217292785645\\
134.862213134766 -19.7220497131348\\
134.88720703125 -19.7224044799805\\
134.912200927734 -19.7227840423584\\
134.937210083008 -19.7231845855713\\
134.962203979492 -19.723596572876\\
134.987213134766 -19.7240161895752\\
135.01220703125 -19.7244358062744\\
135.037200927734 -19.7248382568359\\
135.062210083008 -19.7252254486084\\
135.087203979492 -19.7256031036377\\
135.112213134766 -19.7259654998779\\
135.13720703125 -19.7263088226318\\
135.162200927734 -19.7266368865967\\
135.187210083008 -19.7269477844238\\
135.212203979492 -19.7272357940674\\
135.237213134766 -19.7275066375732\\
135.26220703125 -19.7277584075928\\
135.287200927734 -19.727991104126\\
135.312210083008 -19.7282047271729\\
135.337203979492 -19.7283973693848\\
135.362213134766 -19.7285823822021\\
135.38720703125 -19.7287731170654\\
135.412200927734 -19.7289638519287\\
135.437210083008 -19.729154586792\\
135.462203979492 -19.7293453216553\\
135.487213134766 -19.7295360565186\\
135.51220703125 -19.7297267913818\\
135.537200927734 -19.7299175262451\\
135.562210083008 -19.7301082611084\\
135.587203979492 -19.7303104400635\\
135.612213134766 -19.7305202484131\\
135.63720703125 -19.7307300567627\\
135.662200927734 -19.7309589385986\\
135.687210083008 -19.7311954498291\\
135.712203979492 -19.7314434051514\\
135.737213134766 -19.7317008972168\\
135.76220703125 -19.7319679260254\\
135.787200927734 -19.7322368621826\\
135.812210083008 -19.7325229644775\\
135.837203979492 -19.7328090667725\\
135.862213134766 -19.7330951690674\\
135.88720703125 -19.7333812713623\\
135.912200927734 -19.7336711883545\\
135.937210083008 -19.7339763641357\\
135.962203979492 -19.734281539917\\
135.987213134766 -19.7345867156982\\
136.01220703125 -19.7348918914795\\
136.037200927734 -19.7351970672607\\
136.062210083008 -19.735502243042\\
136.087203979492 -19.7358074188232\\
136.112213134766 -19.7361125946045\\
136.13720703125 -19.7364177703857\\
136.162200927734 -19.736722946167\\
136.187210083008 -19.7370281219482\\
136.212203979492 -19.7373332977295\\
136.237213134766 -19.7376384735107\\
136.26220703125 -19.737943649292\\
136.287200927734 -19.7382316589355\\
136.312210083008 -19.7385177612305\\
136.337203979492 -19.7387847900391\\
136.362213134766 -19.7390518188477\\
136.38720703125 -19.7393188476563\\
136.412200927734 -19.7395858764648\\
136.437210083008 -19.7398529052734\\
136.462203979492 -19.7401161193848\\
136.487213134766 -19.7403812408447\\
136.51220703125 -19.7406444549561\\
136.537200927734 -19.7409114837646\\
136.562210083008 -19.741174697876\\
136.587203979492 -19.74143409729\\
136.612213134766 -19.7416496276855\\
136.63720703125 -19.7418880462646\\
136.662200927734 -19.7421703338623\\
136.687210083008 -19.7424716949463\\
136.712203979492 -19.7427883148193\\
136.729965209961 -19.743013381958\\
136.748107910156 -19.743257522583\\
136.767135620117 -19.7435169219971\\
136.786834716797 -19.7437896728516\\
136.809158325195 -19.7441062927246\\
136.83415222168 -19.7444686889648\\
136.859146118164 -19.7448387145996\\
136.884155273438 -19.745210647583\\
136.909149169922 -19.7455825805664\\
136.934158325195 -19.7459621429443\\
136.95915222168 -19.7463436126709\\
136.984146118164 -19.7467365264893\\
137.009155273438 -19.7471618652344\\
137.034149169922 -19.7476196289063\\
137.059158325195 -19.7481079101563\\
137.08415222168 -19.7486267089844\\
137.109146118164 -19.7491626739502\\
137.134155273438 -19.7497596740723\\
137.159149169922 -19.7503662109375\\
137.175140380859 -19.7507381439209\\
137.199813842773 -19.7513275146484\\
137.224822998047 -19.7519626617432\\
137.248687744141 -19.7525539398193\\
137.270263671875 -19.753059387207\\
137.295257568359 -19.7536964416504\\
137.320251464844 -19.7543392181396\\
137.343276977539 -19.7548713684082\\
137.368286132813 -19.755443572998\\
137.393280029297 -19.7560291290283\\
137.417526245117 -19.756591796875\\
137.442520141602 -19.7570877075195\\
137.467529296875 -19.7576065063477\\
137.492523193359 -19.758150100708\\
137.517532348633 -19.7588195800781\\
137.542526245117 -19.7596054077148\\
137.567443847656 -19.7603549957275\\
137.591659545898 -19.761043548584\\
137.616653442383 -19.7617340087891\\
137.641662597656 -19.7624244689941\\
137.666656494141 -19.7631130218506\\
137.691665649414 -19.7637920379639\\
137.716659545898 -19.7645015716553\\
137.741653442383 -19.7652816772461\\
137.766662597656 -19.7658939361572\\
137.791656494141 -19.7663993835449\\
137.816665649414 -19.7669658660889\\
137.841659545898 -19.7676010131836\\
137.866653442383 -19.76828956604\\
137.891418457031 -19.768970489502\\
137.914993286133 -19.7695922851563\\
137.939849853516 -19.7702217102051\\
137.964614868164 -19.7708225250244\\
137.988235473633 -19.7713794708252\\
138.013107299805 -19.7719535827637\\
138.038101196289 -19.7725028991699\\
138.063095092773 -19.7729930877686\\
138.088104248047 -19.7733116149902\\
138.113098144531 -19.7735786437988\\
138.138107299805 -19.7738456726074\\
138.163101196289 -19.7741031646729\\
138.188095092773 -19.7743511199951\\
138.213104248047 -19.7745990753174\\
138.238098144531 -19.7748622894287\\
138.263107299805 -19.7751331329346\\
138.288101196289 -19.7754249572754\\
138.313095092773 -19.7757415771484\\
138.338104248047 -19.7760906219482\\
138.363098144531 -19.776481628418\\
138.388107299805 -19.7768688201904\\
138.413101196289 -19.7772331237793\\
138.438095092773 -19.7775783538818\\
138.463104248047 -19.7779197692871\\
138.488098144531 -19.778263092041\\
138.513107299805 -19.7786235809326\\
138.538101196289 -19.7790012359619\\
138.563095092773 -19.7794017791748\\
138.588104248047 -19.7797927856445\\
138.613098144531 -19.7801570892334\\
138.638107299805 -19.7805213928223\\
138.663101196289 -19.7809143066406\\
138.688095092773 -19.7813148498535\\
138.713104248047 -19.7817211151123\\
138.738098144531 -19.7821388244629\\
138.763107299805 -19.7825584411621\\
138.788101196289 -19.782958984375\\
138.813095092773 -19.7833442687988\\
138.838104248047 -19.7837028503418\\
138.863098144531 -19.7840366363525\\
138.888107299805 -19.7843608856201\\
138.913101196289 -19.7847023010254\\
138.938095092773 -19.7850742340088\\
138.963104248047 -19.7854824066162\\
138.988098144531 -19.7859020233154\\
139.013107299805 -19.7863235473633\\
139.038101196289 -19.7867603302002\\
139.063095092773 -19.7871990203857\\
139.088104248047 -19.7876377105713\\
139.113098144531 -19.7880840301514\\
139.138107299805 -19.7885684967041\\
139.163101196289 -19.7891426086426\\
139.188095092773 -19.7899074554443\\
139.212219238281 -19.7907752990723\\
139.232727050781 -19.791540145874\\
139.254531860352 -19.7924137115479\\
139.276596069336 -19.7933807373047\\
139.299652099609 -19.794454574585\\
139.324661254883 -19.7956409454346\\
139.349655151367 -19.7968406677246\\
139.374664306641 -19.7980632781982\\
139.399658203125 -19.7993202209473\\
139.424652099609 -19.8006191253662\\
139.449661254883 -19.80198097229\\
139.474655151367 -19.8034152984619\\
139.499664306641 -19.8049240112305\\
139.524658203125 -19.8064365386963\\
139.549652099609 -19.8079414367676\\
139.574661254883 -19.8094329833984\\
139.599655151367 -19.8109683990479\\
139.624664306641 -19.8125915527344\\
139.649658203125 -19.8141975402832\\
139.674652099609 -19.8157844543457\\
139.699661254883 -19.8173446655273\\
139.724655151367 -19.8188781738281\\
139.749664306641 -19.8204021453857\\
139.774658203125 -19.8219318389893\\
139.799652099609 -19.8234901428223\\
139.824661254883 -19.8251037597656\\
139.849655151367 -19.826789855957\\
139.874664306641 -19.8284759521484\\
139.899658203125 -19.8301391601563\\
139.924652099609 -19.8317794799805\\
139.949661254883 -19.8334178924561\\
139.974655151367 -19.835054397583\\
139.999664306641 -19.8366966247559\\
140.024658203125 -19.8383750915527\\
140.049652099609 -19.8400917053223\\
140.074661254883 -19.8418083190918\\
140.099655151367 -19.843505859375\\
140.124664306641 -19.8451881408691\\
140.149658203125 -19.8468494415283\\
140.174652099609 -19.8485164642334\\
140.199661254883 -19.8502082824707\\
140.224655151367 -19.8519020080566\\
140.249664306641 -19.8535804748535\\
140.274658203125 -19.8552474975586\\
140.299652099609 -19.856897354126\\
140.324661254883 -19.8585319519043\\
140.349655151367 -19.8601512908936\\
140.374664306641 -19.8617553710938\\
140.399658203125 -19.8633422851563\\
140.424652099609 -19.8649253845215\\
140.449661254883 -19.8665046691895\\
140.474655151367 -19.8680667877197\\
140.499664306641 -19.8696117401123\\
140.524658203125 -19.8711357116699\\
140.549652099609 -19.8726425170898\\
140.574661254883 -19.8741397857666\\
140.599655151367 -19.875617980957\\
140.624664306641 -19.8770771026611\\
140.649658203125 -19.8785133361816\\
140.674652099609 -19.8799228668213\\
140.699661254883 -19.881311416626\\
140.724655151367 -19.882682800293\\
140.749664306641 -19.8840446472168\\
140.774658203125 -19.8853855133057\\
140.799652099609 -19.8867111206055\\
140.824661254883 -19.8880176544189\\
140.849655151367 -19.889310836792\\
140.874664306641 -19.890588760376\\
140.899658203125 -19.891845703125\\
140.924652099609 -19.8930854797363\\
140.949661254883 -19.8943099975586\\
140.974655151367 -19.8955154418945\\
140.999664306641 -19.8967056274414\\
141.024658203125 -19.8978805541992\\
141.049652099609 -19.8990345001221\\
141.074661254883 -19.9001712799072\\
141.099655151367 -19.901294708252\\
141.124664306641 -19.9023990631104\\
141.149658203125 -19.9034767150879\\
141.174652099609 -19.9045333862305\\
141.199661254883 -19.9055728912354\\
141.224655151367 -19.9065914154053\\
141.249664306641 -19.9075355529785\\
141.274658203125 -19.9085502624512\\
141.299652099609 -19.9096031188965\\
141.324661254883 -19.9106121063232\\
141.349655151367 -19.9115867614746\\
141.374664306641 -19.912540435791\\
141.399658203125 -19.9134788513184\\
141.424652099609 -19.9144115447998\\
141.449661254883 -19.9153232574463\\
141.474655151367 -19.9161987304688\\
141.499664306641 -19.9170589447021\\
141.524658203125 -19.9179172515869\\
141.549652099609 -19.9187622070313\\
141.574661254883 -19.9195899963379\\
141.599655151367 -19.9203968048096\\
141.624664306641 -19.9211845397949\\
141.649658203125 -19.9219493865967\\
141.674652099609 -19.9226932525635\\
141.699661254883 -19.9234142303467\\
141.724655151367 -19.9241123199463\\
141.749664306641 -19.9247894287109\\
141.774658203125 -19.9254474639893\\
141.799652099609 -19.9260768890381\\
141.824661254883 -19.9266700744629\\
141.849655151367 -19.9272155761719\\
141.874664306641 -19.9277191162109\\
141.899658203125 -19.9281959533691\\
141.924652099609 -19.9286613464355\\
141.949661254883 -19.9291191101074\\
141.974655151367 -19.9295635223389\\
141.999664306641 -19.9299869537354\\
142.024658203125 -19.9303894042969\\
142.049652099609 -19.9308204650879\\
142.074661254883 -19.9312992095947\\
142.099655151367 -19.9318103790283\\
142.124664306641 -19.9323558807373\\
142.149658203125 -19.9329471588135\\
142.174652099609 -19.9335746765137\\
142.199661254883 -19.9342002868652\\
142.224655151367 -19.9347839355469\\
142.249664306641 -19.9353141784668\\
142.274658203125 -19.9358005523682\\
142.299652099609 -19.9362640380859\\
142.324661254883 -19.9367218017578\\
142.349655151367 -19.9371795654297\\
142.374664306641 -19.9376335144043\\
142.399658203125 -19.938009262085\\
142.424652099609 -19.9382305145264\\
142.449661254883 -19.9384746551514\\
142.474655151367 -19.9387683868408\\
142.499664306641 -19.9390735626221\\
142.524658203125 -19.9393730163574\\
142.549652099609 -19.9396419525146\\
142.574661254883 -19.9398975372314\\
142.599655151367 -19.9401931762695\\
142.624664306641 -19.9404983520508\\
142.649658203125 -19.940788269043\\
142.674652099609 -19.9410743713379\\
142.699661254883 -19.9413776397705\\
142.724655151367 -19.9416828155518\\
142.749664306641 -19.941987991333\\
142.774658203125 -19.9422931671143\\
142.799652099609 -19.9425983428955\\
142.824661254883 -19.9429035186768\\
142.849655151367 -19.943208694458\\
142.874664306641 -19.9435234069824\\
142.899658203125 -19.9438514709473\\
142.924652099609 -19.9442043304443\\
142.949661254883 -19.9445858001709\\
142.974655151367 -19.9450016021729\\
142.999664306641 -19.945463180542\\
143.024658203125 -19.9459800720215\\
143.049652099609 -19.9465732574463\\
143.074661254883 -19.9472389221191\\
143.099655151367 -19.9479656219482\\
143.124664306641 -19.9487705230713\\
143.149658203125 -19.9496955871582\\
143.174652099609 -19.9507713317871\\
143.199661254883 -19.9520263671875\\
143.224655151367 -19.9535465240479\\
143.249664306641 -19.9554328918457\\
143.274658203125 -19.9577140808105\\
143.299652099609 -19.9602432250977\\
143.324661254883 -19.9625148773193\\
143.349655151367 -19.9642848968506\\
143.374664306641 -19.9660968780518\\
143.399658203125 -19.9679908752441\\
143.424652099609 -19.9698829650879\\
143.449661254883 -19.9717750549316\\
143.474655151367 -19.9736938476563\\
143.499664306641 -19.9756488800049\\
143.524658203125 -19.977632522583\\
143.549652099609 -19.9796352386475\\
143.574661254883 -19.9816799163818\\
143.599655151367 -19.983793258667\\
143.624664306641 -19.9859580993652\\
143.649658203125 -19.9881706237793\\
143.674652099609 -19.9905319213867\\
143.699661254883 -19.9929027557373\\
143.724655151367 -19.9951553344727\\
143.749664306641 -19.9971046447754\\
143.774658203125 -19.9988460540771\\
143.799652099609 -20.0005397796631\\
143.824661254883 -20.0021686553955\\
143.849655151367 -20.0038108825684\\
143.874664306641 -20.0054798126221\\
143.899658203125 -20.0072135925293\\
143.924652099609 -20.009126663208\\
143.949661254883 -20.0110893249512\\
143.974655151367 -20.0130443572998\\
143.999664306641 -20.0149803161621\\
144.024658203125 -20.0169048309326\\
144.049652099609 -20.01881980896\\
144.074661254883 -20.020715713501\\
144.099655151367 -20.0225028991699\\
144.124664306641 -20.0240688323975\\
144.149658203125 -20.025634765625\\
144.174652099609 -20.0272235870361\\
144.199661254883 -20.0288238525391\\
144.224655151367 -20.0304260253906\\
144.249664306641 -20.0320262908936\\
144.274658203125 -20.0336074829102\\
144.299652099609 -20.0351982116699\\
144.324661254883 -20.0368824005127\\
144.349655151367 -20.0385990142822\\
144.374664306641 -20.0401840209961\\
144.399658203125 -20.0412616729736\\
144.424652099609 -20.0419311523438\\
144.449661254883 -20.0425300598145\\
144.474655151367 -20.0430908203125\\
144.499664306641 -20.0436382293701\\
144.524658203125 -20.0442428588867\\
144.549652099609 -20.0449161529541\\
144.574661254883 -20.0456562042236\\
144.599655151367 -20.0464324951172\\
144.624664306641 -20.0472545623779\\
144.649658203125 -20.048152923584\\
144.674652099609 -20.0490703582764\\
144.699661254883 -20.0499439239502\\
144.724655151367 -20.0507850646973\\
144.749664306641 -20.0516662597656\\
144.774658203125 -20.0525970458984\\
144.799652099609 -20.053560256958\\
144.824661254883 -20.0545501708984\\
144.849655151367 -20.0555667877197\\
144.874664306641 -20.0566024780273\\
144.899658203125 -20.0576515197754\\
144.924652099609 -20.0586948394775\\
144.949661254883 -20.0596694946289\\
144.974655151367 -20.060640335083\\
144.999664306641 -20.0616397857666\\
145.024658203125 -20.062650680542\\
145.049652099609 -20.0636787414551\\
145.074661254883 -20.0647201538086\\
145.099655151367 -20.0657978057861\\
145.124664306641 -20.0669136047363\\
145.149658203125 -20.0680503845215\\
145.174652099609 -20.0691585540771\\
145.199661254883 -20.070276260376\\
145.224655151367 -20.0714302062988\\
145.249664306641 -20.0726127624512\\
145.274658203125 -20.0738124847412\\
145.299652099609 -20.0749931335449\\
145.324661254883 -20.076135635376\\
145.349655151367 -20.0772800445557\\
145.374664306641 -20.078441619873\\
145.399658203125 -20.0796070098877\\
145.424652099609 -20.0807857513428\\
145.449661254883 -20.0819511413574\\
145.474655151367 -20.0830745697021\\
145.499664306641 -20.0841789245605\\
145.524658203125 -20.0852794647217\\
145.549652099609 -20.0863666534424\\
145.574661254883 -20.087438583374\\
145.599655151367 -20.0885028839111\\
145.624664306641 -20.089563369751\\
145.649658203125 -20.0906105041504\\
145.674652099609 -20.0916423797607\\
145.699661254883 -20.0926704406738\\
145.724655151367 -20.0936851501465\\
145.749664306641 -20.0946960449219\\
145.774658203125 -20.0957050323486\\
145.799652099609 -20.096700668335\\
145.824661254883 -20.0976905822754\\
145.849655151367 -20.0986652374268\\
145.874664306641 -20.0996284484863\\
145.899658203125 -20.1005802154541\\
145.924652099609 -20.1015167236328\\
145.949661254883 -20.1024417877197\\
145.974655151367 -20.1033535003662\\
145.999664306641 -20.1042499542236\\
146.024658203125 -20.1051330566406\\
146.049652099609 -20.1060085296631\\
146.074661254883 -20.1068668365479\\
146.099655151367 -20.1077156066895\\
146.124664306641 -20.1085548400879\\
146.149658203125 -20.1093864440918\\
146.174652099609 -20.1102066040039\\
146.199661254883 -20.1110191345215\\
146.224655151367 -20.1118202209473\\
146.249664306641 -20.1126174926758\\
146.274658203125 -20.1133995056152\\
146.299652099609 -20.1141796112061\\
146.324661254883 -20.1149425506592\\
146.349655151367 -20.115701675415\\
146.374664306641 -20.1164455413818\\
146.399658203125 -20.1171798706055\\
146.424652099609 -20.1179027557373\\
146.449661254883 -20.1186084747314\\
146.474655151367 -20.1192970275879\\
146.499664306641 -20.119966506958\\
146.524658203125 -20.1206207275391\\
146.549652099609 -20.1212558746338\\
146.574661254883 -20.1218738555908\\
146.599655151367 -20.1224746704102\\
146.624664306641 -20.1230602264404\\
146.649658203125 -20.1236324310303\\
146.674652099609 -20.1241874694824\\
146.699661254883 -20.1247367858887\\
146.724655151367 -20.1252708435059\\
146.749664306641 -20.125804901123\\
146.774658203125 -20.1263275146484\\
146.799652099609 -20.1268424987793\\
146.824661254883 -20.1273574829102\\
146.849655151367 -20.127872467041\\
146.874664306641 -20.1283874511719\\
146.899658203125 -20.1289024353027\\
146.924652099609 -20.1294174194336\\
146.949661254883 -20.1299324035645\\
146.974655151367 -20.130443572998\\
146.999664306641 -20.1309471130371\\
147.024658203125 -20.1314430236816\\
147.049652099609 -20.1319389343262\\
147.074661254883 -20.1324348449707\\
147.099655151367 -20.1329307556152\\
147.124664306641 -20.1334247589111\\
147.149658203125 -20.1339015960693\\
147.174652099609 -20.1343784332275\\
147.199661254883 -20.1348552703857\\
147.224655151367 -20.1353187561035\\
147.249664306641 -20.1357765197754\\
147.274658203125 -20.1362342834473\\
147.299652099609 -20.1366786956787\\
147.324661254883 -20.1371173858643\\
147.349655151367 -20.1375560760498\\
147.374664306641 -20.1379833221436\\
147.399658203125 -20.1384029388428\\
147.424652099609 -20.138822555542\\
147.449661254883 -20.1392307281494\\
147.474655151367 -20.1396312713623\\
147.499664306641 -20.1400318145752\\
147.524658203125 -20.1404151916504\\
147.549652099609 -20.140796661377\\
147.574661254883 -20.1411666870117\\
147.599655151367 -20.141529083252\\
147.624664306641 -20.1418838500977\\
147.649658203125 -20.1422271728516\\
147.674652099609 -20.1425704956055\\
147.699661254883 -20.1429042816162\\
147.724655151367 -20.1432285308838\\
147.749664306641 -20.1435527801514\\
147.774658203125 -20.1438770294189\\
147.799652099609 -20.1441955566406\\
147.824661254883 -20.1445007324219\\
147.849655151367 -20.1448059082031\\
147.874664306641 -20.1451110839844\\
147.899658203125 -20.1454162597656\\
147.924652099609 -20.1457214355469\\
147.949661254883 -20.1460266113281\\
147.974655151367 -20.1463279724121\\
147.999664306641 -20.146614074707\\
148.024658203125 -20.146900177002\\
148.049652099609 -20.1471843719482\\
148.074661254883 -20.1474590301514\\
148.099655151367 -20.14772605896\\
148.124664306641 -20.1479930877686\\
148.149658203125 -20.1482601165771\\
148.174652099609 -20.1485252380371\\
148.199661254883 -20.1487789154053\\
148.224655151367 -20.1490268707275\\
148.249664306641 -20.1492748260498\\
148.274658203125 -20.1495227813721\\
148.299652099609 -20.1497707366943\\
148.324661254883 -20.1500186920166\\
148.349655151367 -20.1502532958984\\
148.374664306641 -20.1504821777344\\
148.399658203125 -20.1507110595703\\
148.424652099609 -20.1509399414063\\
148.449661254883 -20.1511688232422\\
148.474655151367 -20.1513977050781\\
148.499664306641 -20.1516265869141\\
148.524658203125 -20.15185546875\\
148.549652099609 -20.1520843505859\\
148.574661254883 -20.1523132324219\\
148.599655151367 -20.1525268554688\\
148.624664306641 -20.1527366638184\\
148.649658203125 -20.152946472168\\
148.674652099609 -20.1531562805176\\
148.699661254883 -20.1533660888672\\
148.724655151367 -20.1535758972168\\
148.749664306641 -20.1537857055664\\
148.774658203125 -20.153980255127\\
148.799652099609 -20.1541709899902\\
148.824661254883 -20.1543617248535\\
148.849655151367 -20.1545524597168\\
148.874664306641 -20.1547431945801\\
148.899658203125 -20.1549282073975\\
148.924652099609 -20.1550998687744\\
148.949661254883 -20.1552715301514\\
148.974655151367 -20.1554279327393\\
148.999664306641 -20.155590057373\\
149.024658203125 -20.1557521820068\\
149.049652099609 -20.1558971405029\\
149.074661254883 -20.1560306549072\\
149.099655151367 -20.1561641693115\\
149.124664306641 -20.1562976837158\\
149.149658203125 -20.1564311981201\\
149.174652099609 -20.1565628051758\\
149.199661254883 -20.1566772460938\\
149.224655151367 -20.1567916870117\\
149.249664306641 -20.1569061279297\\
149.274658203125 -20.1570205688477\\
149.299652099609 -20.1571350097656\\
149.324661254883 -20.1572494506836\\
149.349655151367 -20.1573638916016\\
149.374664306641 -20.1574859619141\\
149.399658203125 -20.1576385498047\\
149.424652099609 -20.1578216552734\\
149.449661254883 -20.1580123901367\\
149.474655151367 -20.1581916809082\\
149.499664306641 -20.1583385467529\\
149.524658203125 -20.1584739685059\\
149.549652099609 -20.1586265563965\\
149.574661254883 -20.1587982177734\\
149.599655151367 -20.158992767334\\
149.624664306641 -20.1592063903809\\
149.649658203125 -20.1594352722168\\
149.674652099609 -20.1596641540527\\
149.699661254883 -20.1598987579346\\
149.724655151367 -20.1602592468262\\
149.749664306641 -20.1607704162598\\
149.774658203125 -20.1613540649414\\
149.799652099609 -20.161979675293\\
149.824661254883 -20.1626472473145\\
149.849655151367 -20.1633453369141\\
149.874664306641 -20.1640396118164\\
149.899658203125 -20.1647129058838\\
149.924652099609 -20.1653518676758\\
149.949661254883 -20.1659641265869\\
149.974655151367 -20.1665554046631\\
149.998580932617 -20.167121887207\\
};
\addlegendentry{$\dtmax{} = \expneg{2.5}{3}{[s]}$};

\addplot [
color=mycolor1,
solid,
forget plot
]
table[row sep=crcr]{
0 0\\
1.0001015663147 0\\
2.0001015663147 0\\
3.0001015663147 0\\
4.0001015663147 0\\
5.0001015663147 0\\
6.0001015663147 0\\
7.0001015663147 0\\
8.00010108947754 0\\
9.00010108947754 0\\
10.0001010894775 0\\
11.0001010894775 0\\
12.0001010894775 0\\
13.0001010894775 0\\
14.0001010894775 0\\
15.0001010894775 0\\
16.0001010894775 0\\
17.0001010894775 0\\
18.0001010894775 0\\
19.0001010894775 -4.85835052188577e-08\\
20.0001010894775 -8.24629694307077e-07\\
21.0001010894775 -0.544735968112946\\
22.0001010894775 -0.64683586359024\\
23.0010185241699 -0.681989908218384\\
24.0045852661133 -0.707359611988068\\
25.0045852661133 -0.730478823184967\\
26.0045852661133 -0.757599413394928\\
27.0045852661133 -0.791127383708954\\
28.0045852661133 -0.83398300409317\\
29.0045852661133 -0.889559090137482\\
30.0045852661133 -0.96105831861496\\
31.0045852661133 -1.05419182777405\\
32.0045852661133 -1.16891384124756\\
33.0045852661133 -1.29769241809845\\
34.0045852661133 -1.4343546628952\\
35.0008354187012 -1.57461762428284\\
36.0008354187012 -1.71396934986115\\
37.0008354187012 -1.84861445426941\\
38.0008354187012 -1.97657644748688\\
39.0008354187012 -2.10267043113709\\
40.0008354187012 -2.22802758216858\\
41.0008354187012 -2.3532338142395\\
42.0008354187012 -2.47667694091797\\
43.0008354187012 -2.60142254829407\\
44.0008354187012 -2.72885036468506\\
45.0008354187012 -2.85970997810364\\
46.0008354187012 -2.98391962051392\\
47.0008354187012 -3.10062551498413\\
48.0008354187012 -3.21554613113403\\
49.0008354187012 -3.33204674720764\\
50.0008354187012 -3.44828081130981\\
51.0008354187012 -3.55808043479919\\
52.0008354187012 -3.65300488471985\\
53.0008354187012 -3.74433994293213\\
54.0008354187012 -3.83207893371582\\
55.0008354187012 -3.9187593460083\\
56.0008354187012 -4.00384902954102\\
57.0008354187012 -4.08855104446411\\
58.0008354187012 -4.17496681213379\\
59.0008354187012 -4.26092290878296\\
60.0008354187012 -4.34975910186768\\
61.0008354187012 -4.43921899795532\\
62.0008354187012 -4.5299015045166\\
63.0008354187012 -4.62167263031006\\
64.0008316040039 -4.71487045288086\\
65.0008316040039 -4.80753326416016\\
66.0008316040039 -4.90136766433716\\
67.0008316040039 -4.99564027786255\\
68.0008316040039 -5.08896160125732\\
69.0008316040039 -5.18189859390259\\
70.0008316040039 -5.27736139297485\\
71.0008316040039 -5.3721866607666\\
72.0008316040039 -5.46830797195435\\
73.0008316040039 -5.56602048873901\\
74.0008316040039 -5.66490888595581\\
75.0008316040039 -5.75968265533447\\
76.0008316040039 -5.85432577133179\\
77.0008316040039 -5.95189332962036\\
78.0008316040039 -6.05033349990845\\
79.0008316040039 -6.14729022979736\\
80.0008316040039 -6.24646139144897\\
81.0008316040039 -6.34660720825195\\
82.0008316040039 -6.44676399230957\\
83.0008316040039 -6.5465726852417\\
84.0008316040039 -6.64717149734497\\
85.0008316040039 -6.74692487716675\\
86.0008316040039 -6.84474325180054\\
87.0008316040039 -6.94109773635864\\
88.0008316040039 -7.03610706329346\\
89.0008316040039 -7.12964344024658\\
90.0008316040039 -7.22221517562866\\
91.0008316040039 -7.31422996520996\\
92.0008316040039 -7.40689182281494\\
93.0008316040039 -7.49616718292236\\
94.0008316040039 -7.57983589172363\\
95.0008316040039 -7.65848922729492\\
96.0008316040039 -7.73313140869141\\
97.0008316040039 -7.80465412139893\\
98.0008316040039 -7.87353229522705\\
99.0008316040039 -7.94060182571411\\
100 -8.00656318664551\\
100.012496948242 -8.0073766708374\\
100.025001525879 -8.00818634033203\\
100.037498474121 -8.00899314880371\\
100.049995422363 -8.00979423522949\\
100.0625 -8.01059436798096\\
100.074996948242 -8.01139259338379\\
100.087501525879 -8.01218414306641\\
100.099998474121 -8.01297569274902\\
100.112495422363 -8.01376724243164\\
100.125 -8.01455879211426\\
100.137496948242 -8.01535034179688\\
100.150001525879 -8.01614570617676\\
100.162498474121 -8.01694488525391\\
100.174995422363 -8.01774501800537\\
100.1875 -8.0185489654541\\
100.199996948242 -8.01935768127441\\
100.212501525879 -8.02016830444336\\
100.224998474121 -8.0209789276123\\
100.237495422363 -8.02179050445557\\
100.25 -8.02260971069336\\
100.262496948242 -8.02342891693115\\
100.275001525879 -8.02424907684326\\
100.287498474121 -8.02506923675537\\
100.299995422363 -8.02588939666748\\
100.3125 -8.02671432495117\\
100.324996948242 -8.02754402160645\\
100.337501525879 -8.02838134765625\\
100.349998474121 -8.02921867370605\\
100.362495422363 -8.03005695343018\\
100.375 -8.03089618682861\\
100.387496948242 -8.03173542022705\\
100.400001525879 -8.03257465362549\\
100.412498474121 -8.03341865539551\\
100.424995422363 -8.03426551818848\\
100.4375 -8.03511333465576\\
100.449996948242 -8.03595924377441\\
100.462501525879 -8.03679847717285\\
100.474998474121 -8.03763771057129\\
100.487495422363 -8.03847694396973\\
100.5 -8.03931617736816\\
100.512496948242 -8.04015350341797\\
100.525001525879 -8.04099082946777\\
100.537498474121 -8.04182052612305\\
100.549995422363 -8.042649269104\\
100.5625 -8.04347801208496\\
100.574996948242 -8.04430484771729\\
100.587501525879 -8.04512500762939\\
100.599998474121 -8.0459451675415\\
100.612495422363 -8.0467643737793\\
100.625 -8.04758358001709\\
100.637496948242 -8.04840183258057\\
100.650001525879 -8.04922103881836\\
100.662498474121 -8.05003929138184\\
100.674995422363 -8.05085754394531\\
100.6875 -8.05167293548584\\
100.699996948242 -8.05248355865479\\
100.712501525879 -8.05329418182373\\
100.724998474121 -8.05410480499268\\
100.737495422363 -8.05491542816162\\
100.75 -8.05572605133057\\
100.762496948242 -8.05653476715088\\
100.775001525879 -8.05734539031982\\
100.787498474121 -8.05816459655762\\
100.799995422363 -8.05898475646973\\
100.8125 -8.05980491638184\\
100.824996948242 -8.06062412261963\\
100.837501525879 -8.06143951416016\\
100.849998474121 -8.0622501373291\\
100.862495422363 -8.06306552886963\\
100.875 -8.06388473510742\\
100.887496948242 -8.064697265625\\
100.900001525879 -8.06551361083984\\
100.912498474121 -8.0663366317749\\
100.924995422363 -8.06716442108154\\
100.9375 -8.06799411773682\\
100.949996948242 -8.06882190704346\\
100.962501525879 -8.06965065002441\\
100.974998474121 -8.07047843933105\\
100.987495422363 -8.07130146026611\\
101 -8.07212162017822\\
101.012496948242 -8.0729398727417\\
101.025001525879 -8.07375431060791\\
101.037498474121 -8.07456493377686\\
101.049995422363 -8.07537364959717\\
101.0625 -8.07617950439453\\
101.074996948242 -8.07698059082031\\
101.087501525879 -8.07778072357178\\
101.099998474121 -8.07857704162598\\
101.112495422363 -8.07936859130859\\
101.125 -8.08015918731689\\
101.137496948242 -8.08094882965088\\
101.150001525879 -8.08173942565918\\
101.162498474121 -8.08252906799316\\
101.174995422363 -8.08331871032715\\
101.1875 -8.08410930633545\\
101.199996948242 -8.08489990234375\\
101.212501525879 -8.08569526672363\\
101.224998474121 -8.0864953994751\\
101.237495422363 -8.08730316162109\\
101.25 -8.08811664581299\\
101.262496948242 -8.08893585205078\\
101.275001525879 -8.08976078033447\\
101.287498474121 -8.09058856964111\\
101.299995422363 -8.09141826629639\\
101.3125 -8.09224796295166\\
101.324996948242 -8.09307670593262\\
101.337501525879 -8.09390640258789\\
101.349998474121 -8.09474277496338\\
101.362495422363 -8.09558010101318\\
101.375 -8.0964183807373\\
101.387496948242 -8.09725570678711\\
101.400001525879 -8.09809398651123\\
101.412498474121 -8.09893131256104\\
101.424995422363 -8.09976863861084\\
101.4375 -8.10060024261475\\
101.449996948242 -8.10142993927002\\
101.462501525879 -8.10225963592529\\
101.474998474121 -8.10308933258057\\
101.487495422363 -8.10391902923584\\
101.5 -8.1047477722168\\
101.512496948242 -8.10557556152344\\
101.525001525879 -8.10640430450439\\
101.537498474121 -8.10723209381104\\
101.549995422363 -8.10805797576904\\
101.5625 -8.10887813568115\\
101.574996948242 -8.10969829559326\\
101.587501525879 -8.11051845550537\\
101.599998474121 -8.11133861541748\\
101.612495422363 -8.11215782165527\\
101.625 -8.11297798156738\\
101.637496948242 -8.11379814147949\\
101.650001525879 -8.1146183013916\\
101.662498474121 -8.11543846130371\\
101.674995422363 -8.11625862121582\\
101.6875 -8.11707878112793\\
101.699996948242 -8.11789894104004\\
101.712501525879 -8.11871910095215\\
101.724998474121 -8.11953926086426\\
101.737495422363 -8.12035942077637\\
101.75 -8.12117958068848\\
101.762496948242 -8.12199974060059\\
101.775001525879 -8.1228199005127\\
101.787498474121 -8.1236400604248\\
101.799995422363 -8.1244592666626\\
101.8125 -8.12527847290039\\
101.824996948242 -8.12609577178955\\
101.837501525879 -8.1269063949585\\
101.849998474121 -8.12772178649902\\
101.862495422363 -8.12854194641113\\
101.875 -8.12936210632324\\
101.887496948242 -8.13018035888672\\
101.900001525879 -8.13100242614746\\
101.912498474121 -8.13183212280273\\
101.924995422363 -8.13266944885254\\
101.9375 -8.13350772857666\\
101.949996948242 -8.13434505462646\\
101.962501525879 -8.13518333435059\\
101.974998474121 -8.13601970672607\\
101.987495422363 -8.13684940338135\\
102 -8.1376781463623\\
102.012496948242 -8.13850593566895\\
102.025001525879 -8.13932609558105\\
102.037498474121 -8.14014530181885\\
102.049995422363 -8.14096164703369\\
102.0625 -8.14177227020264\\
102.074996948242 -8.14258003234863\\
102.087501525879 -8.14338111877441\\
102.099998474121 -8.14418029785156\\
102.112495422363 -8.14497661590576\\
102.125 -8.14576816558838\\
102.137496948242 -8.14655780792236\\
102.150001525879 -8.14734649658203\\
102.162498474121 -8.14812850952148\\
102.174995422363 -8.14890956878662\\
102.1875 -8.14969062805176\\
102.199996948242 -8.15047073364258\\
102.212501525879 -8.15125274658203\\
102.224998474121 -8.15203475952148\\
102.237495422363 -8.15281772613525\\
102.25 -8.15360832214355\\
102.262496948242 -8.15439891815186\\
102.275001525879 -8.15519046783447\\
102.287498474121 -8.15598773956299\\
102.299995422363 -8.15678691864014\\
102.3125 -8.15758800506592\\
102.324996948242 -8.1583890914917\\
102.337501525879 -8.15919589996338\\
102.349998474121 -8.16000175476074\\
102.362495422363 -8.16080760955811\\
102.375 -8.16161727905273\\
102.387496948242 -8.16242790222168\\
102.400001525879 -8.16323852539063\\
102.412498474121 -8.16404914855957\\
102.424995422363 -8.16485977172852\\
102.4375 -8.16567039489746\\
102.449996948242 -8.16648101806641\\
102.462501525879 -8.16729164123535\\
102.474998474121 -8.1681022644043\\
102.487495422363 -8.16891288757324\\
102.5 -8.16972351074219\\
102.512496948242 -8.17053413391113\\
102.525001525879 -8.17134475708008\\
102.537498474121 -8.17215538024902\\
102.549995422363 -8.17296886444092\\
102.5625 -8.17378807067871\\
102.574996948242 -8.17460632324219\\
102.587501525879 -8.17542552947998\\
102.599998474121 -8.17624378204346\\
102.612495422363 -8.17706203460693\\
102.625 -8.17788124084473\\
102.637496948242 -8.1786994934082\\
102.650001525879 -8.179518699646\\
102.662498474121 -8.18033695220947\\
102.674995422363 -8.18115520477295\\
102.6875 -8.18197441101074\\
102.699996948242 -8.18279266357422\\
102.712501525879 -8.18361186981201\\
102.724998474121 -8.18443012237549\\
102.737495422363 -8.18524837493896\\
102.75 -8.18606758117676\\
102.762496948242 -8.18688583374023\\
102.775001525879 -8.18770503997803\\
102.787498474121 -8.1885232925415\\
102.799995422363 -8.18934154510498\\
102.8125 -8.19015502929688\\
102.824996948242 -8.19096565246582\\
102.837501525879 -8.19178485870361\\
102.849998474121 -8.19261074066162\\
102.862495422363 -8.19343280792236\\
102.875 -8.19425201416016\\
102.887496948242 -8.19507026672363\\
102.900001525879 -8.19588565826416\\
102.912498474121 -8.19670486450195\\
102.924995422363 -8.19753170013428\\
102.9375 -8.19836044311523\\
102.949996948242 -8.19918823242188\\
102.962501525879 -8.20001697540283\\
102.974998474121 -8.20083713531494\\
102.987495422363 -8.20165729522705\\
103 -8.20247745513916\\
103.012496948242 -8.20329570770264\\
103.025001525879 -8.20411014556885\\
103.037498474121 -8.20492076873779\\
103.049995422363 -8.20572948455811\\
103.0625 -8.20653629302979\\
103.074996948242 -8.20733642578125\\
103.087501525879 -8.20813655853271\\
103.099998474121 -8.20892906188965\\
103.112495422363 -8.20971965789795\\
103.125 -8.21051025390625\\
103.137496948242 -8.21129512786865\\
103.150001525879 -8.21207714080811\\
103.162498474121 -8.21285724639893\\
103.174995422363 -8.21363735198975\\
103.1875 -8.21441459655762\\
103.199996948242 -8.21518707275391\\
103.212501525879 -8.2159595489502\\
103.224998474121 -8.21673202514648\\
103.237495422363 -8.21750450134277\\
103.25 -8.21827697753906\\
103.262496948242 -8.21904945373535\\
103.275001525879 -8.21982955932617\\
103.287498474121 -8.22060966491699\\
103.299995422363 -8.22139072418213\\
103.3125 -8.22217273712158\\
103.324996948242 -8.22295475006104\\
103.337501525879 -8.22374153137207\\
103.349998474121 -8.22453117370605\\
103.362495422363 -8.22532081604004\\
103.375 -8.22611236572266\\
103.387496948242 -8.22690391540527\\
103.400001525879 -8.22769546508789\\
103.412498474121 -8.22848701477051\\
103.424995422363 -8.22927856445313\\
103.4375 -8.23007011413574\\
103.449996948242 -8.23086166381836\\
103.462501525879 -8.23165321350098\\
103.474998474121 -8.23244667053223\\
103.487495422363 -8.23324584960938\\
103.5 -8.23404598236084\\
103.512496948242 -8.23484516143799\\
103.525001525879 -8.23564529418945\\
103.537498474121 -8.2364444732666\\
103.549995422363 -8.23724365234375\\
103.5625 -8.23804378509521\\
103.574996948242 -8.23884296417236\\
103.587501525879 -8.23964309692383\\
103.599998474121 -8.24044322967529\\
103.612495422363 -8.24124431610107\\
103.625 -8.24204540252686\\
103.637496948242 -8.24284648895264\\
103.650001525879 -8.24364757537842\\
103.662498474121 -8.2444486618042\\
103.674995422363 -8.2452507019043\\
103.6875 -8.24606037139893\\
103.699996948242 -8.24686908721924\\
103.712501525879 -8.24767875671387\\
103.724998474121 -8.24848747253418\\
103.737495422363 -8.24929618835449\\
103.75 -8.25010585784912\\
103.762496948242 -8.25091457366943\\
103.775001525879 -8.25172424316406\\
103.787498474121 -8.25253295898438\\
103.799995422363 -8.25334167480469\\
103.8125 -8.25414276123047\\
103.824996948242 -8.25494384765625\\
103.837501525879 -8.25574493408203\\
103.849998474121 -8.25654411315918\\
103.862495422363 -8.25734329223633\\
103.875 -8.25814342498779\\
103.887496948242 -8.25894355773926\\
103.900001525879 -8.25974941253662\\
103.912498474121 -8.26055812835693\\
103.924995422363 -8.26136684417725\\
103.9375 -8.26217651367188\\
103.949996948242 -8.26298522949219\\
103.962501525879 -8.26379489898682\\
103.974998474121 -8.26459693908691\\
103.987495422363 -8.2653980255127\\
104 -8.26619911193848\\
104.012496948242 -8.26699924468994\\
104.025001525879 -8.26779937744141\\
104.037498474121 -8.26859855651855\\
104.049995422363 -8.26939010620117\\
104.0625 -8.27018165588379\\
104.074996948242 -8.27097129821777\\
104.087501525879 -8.27175712585449\\
104.099998474121 -8.27253913879395\\
104.112495422363 -8.27331924438477\\
104.125 -8.2741003036499\\
104.137496948242 -8.27487373352051\\
104.150001525879 -8.2756462097168\\
104.162498474121 -8.27641677856445\\
104.174995422363 -8.27718734741211\\
104.1875 -8.2779541015625\\
104.199996948242 -8.27871704101563\\
104.212501525879 -8.27947998046875\\
104.224998474121 -8.28024291992188\\
104.237495422363 -8.28100395202637\\
104.25 -8.28176593780518\\
104.262496948242 -8.28252696990967\\
104.275001525879 -8.28328895568848\\
104.287498474121 -8.28404998779297\\
104.299995422363 -8.28481292724609\\
104.3125 -8.28557586669922\\
104.324996948242 -8.28633880615234\\
104.337501525879 -8.28710174560547\\
104.349998474121 -8.28786468505859\\
104.362495422363 -8.28863525390625\\
104.375 -8.28940677642822\\
104.387496948242 -8.29017734527588\\
104.400001525879 -8.29094886779785\\
104.412498474121 -8.29172039031982\\
104.424995422363 -8.29249286651611\\
104.4375 -8.2932653427124\\
104.449996948242 -8.29403781890869\\
104.462501525879 -8.29481029510498\\
104.474998474121 -8.29558277130127\\
104.487495422363 -8.29636001586914\\
104.5 -8.29714107513428\\
104.512496948242 -8.2979211807251\\
104.525001525879 -8.29870223999023\\
104.537498474121 -8.29948234558105\\
104.549995422363 -8.30026245117188\\
104.5625 -8.30104351043701\\
104.574996948242 -8.30182361602783\\
104.587501525879 -8.30260467529297\\
104.599998474121 -8.30338478088379\\
104.612495422363 -8.30416679382324\\
104.625 -8.3049488067627\\
104.637496948242 -8.30573081970215\\
104.650001525879 -8.3065128326416\\
104.662498474121 -8.30729484558105\\
104.674995422363 -8.30807685852051\\
104.6875 -8.30886173248291\\
104.699996948242 -8.30964374542236\\
104.712501525879 -8.31042575836182\\
104.724998474121 -8.31120777130127\\
104.737495422363 -8.31198978424072\\
104.75 -8.31277179718018\\
104.762496948242 -8.31355381011963\\
104.775001525879 -8.31433582305908\\
104.787498474121 -8.31511783599854\\
104.799995422363 -8.31590461730957\\
104.8125 -8.31669521331787\\
104.824996948242 -8.31748485565186\\
104.837501525879 -8.31827735900879\\
104.849998474121 -8.31907653808594\\
104.862495422363 -8.31987571716309\\
104.875 -8.32067584991455\\
104.887496948242 -8.32146835327148\\
104.900001525879 -8.3222599029541\\
104.912498474121 -8.32305145263672\\
104.924995422363 -8.32384777069092\\
104.9375 -8.32464790344238\\
104.949996948242 -8.32544708251953\\
104.962501525879 -8.32624053955078\\
104.974998474121 -8.3270320892334\\
104.987495422363 -8.32782363891602\\
105 -8.32861518859863\\
105.012496948242 -8.32940578460693\\
105.025001525879 -8.33019638061523\\
105.037498474121 -8.33098602294922\\
105.049995422363 -8.3317756652832\\
105.0625 -8.33255958557129\\
105.074996948242 -8.33334159851074\\
105.087501525879 -8.33412265777588\\
105.099998474121 -8.3349027633667\\
105.112495422363 -8.33567523956299\\
105.125 -8.33644771575928\\
105.137496948242 -8.33721923828125\\
105.150001525879 -8.33799076080322\\
105.162498474121 -8.33875846862793\\
105.174995422363 -8.33952140808105\\
105.1875 -8.34028434753418\\
105.199996948242 -8.34104633331299\\
105.212501525879 -8.3418083190918\\
105.224998474121 -8.34256935119629\\
105.237495422363 -8.34333038330078\\
105.25 -8.34409046173096\\
105.262496948242 -8.34484386444092\\
105.275001525879 -8.34559726715088\\
105.287498474121 -8.34635066986084\\
105.299995422363 -8.3471040725708\\
105.3125 -8.34785747528076\\
105.324996948242 -8.34860897064209\\
105.337501525879 -8.34936141967773\\
105.349998474121 -8.3501148223877\\
105.362495422363 -8.35086822509766\\
105.375 -8.35162162780762\\
105.387496948242 -8.35237503051758\\
105.400001525879 -8.35312843322754\\
105.412498474121 -8.3538818359375\\
105.424995422363 -8.35463523864746\\
105.4375 -8.35538864135742\\
105.449996948242 -8.35614204406738\\
105.462501525879 -8.35689544677734\\
105.474998474121 -8.3576488494873\\
105.487495422363 -8.35840225219727\\
105.5 -8.35915565490723\\
105.512496948242 -8.35990905761719\\
105.525001525879 -8.36066246032715\\
105.537498474121 -8.36141586303711\\
105.549995422363 -8.36216926574707\\
105.5625 -8.36292266845703\\
105.574996948242 -8.36367607116699\\
105.587501525879 -8.36442947387695\\
105.599998474121 -8.3651876449585\\
105.612495422363 -8.36594867706299\\
105.625 -8.3667106628418\\
105.637496948242 -8.36747169494629\\
105.650001525879 -8.3682336807251\\
105.662498474121 -8.36899471282959\\
105.674995422363 -8.3697566986084\\
105.6875 -8.37051963806152\\
105.699996948242 -8.37128257751465\\
105.712501525879 -8.37204551696777\\
105.724998474121 -8.37281036376953\\
105.737495422363 -8.37358093261719\\
105.75 -8.37435245513916\\
105.762496948242 -8.37512302398682\\
105.775001525879 -8.37589454650879\\
105.787498474121 -8.37666702270508\\
105.799995422363 -8.37743949890137\\
105.8125 -8.37821197509766\\
105.824996948242 -8.37898445129395\\
105.837501525879 -8.37976264953613\\
105.849998474121 -8.38054275512695\\
105.862495422363 -8.38132286071777\\
105.875 -8.38210391998291\\
105.887496948242 -8.38288593292236\\
105.900001525879 -8.38366794586182\\
105.912498474121 -8.38444900512695\\
105.924995422363 -8.38523006439209\\
105.9375 -8.38601112365723\\
105.949996948242 -8.38679122924805\\
105.962501525879 -8.38757228851318\\
105.974998474121 -8.388352394104\\
105.987495422363 -8.38913249969482\\
106 -8.38991355895996\\
106.012496948242 -8.3906946182251\\
106.025001525879 -8.39147567749023\\
106.037498474121 -8.39225578308105\\
106.049995422363 -8.39303398132324\\
106.0625 -8.39380645751953\\
106.074996948242 -8.39457893371582\\
106.087501525879 -8.39535140991211\\
106.099998474121 -8.3961238861084\\
106.112495422363 -8.39689636230469\\
106.125 -8.39766788482666\\
106.137496948242 -8.39843845367432\\
106.150001525879 -8.39920330047607\\
106.162498474121 -8.3999662399292\\
106.174995422363 -8.40072822570801\\
106.1875 -8.40149021148682\\
106.199996948242 -8.40225028991699\\
106.212501525879 -8.40300369262695\\
106.224998474121 -8.40375709533691\\
106.237495422363 -8.40450954437256\\
106.25 -8.4052619934082\\
106.262496948242 -8.40601348876953\\
106.275001525879 -8.40676593780518\\
106.287498474121 -8.40751647949219\\
106.299995422363 -8.40826034545898\\
106.3125 -8.40900421142578\\
106.324996948242 -8.40974807739258\\
106.337501525879 -8.41049194335938\\
106.349998474121 -8.41123390197754\\
106.362495422363 -8.4119758605957\\
106.375 -8.41271877288818\\
106.387496948242 -8.41346073150635\\
106.400001525879 -8.41420364379883\\
106.412498474121 -8.41494560241699\\
106.424995422363 -8.41568756103516\\
106.4375 -8.41643047332764\\
106.449996948242 -8.4171724319458\\
106.462501525879 -8.41791534423828\\
106.474998474121 -8.41865730285645\\
106.487495422363 -8.41939926147461\\
106.5 -8.42014217376709\\
106.512496948242 -8.42088413238525\\
106.525001525879 -8.42162704467773\\
106.537498474121 -8.4223690032959\\
106.549995422363 -8.42311096191406\\
106.5625 -8.42385387420654\\
106.574996948242 -8.42459011077881\\
106.587501525879 -8.42532825469971\\
106.599998474121 -8.42606258392334\\
106.612495422363 -8.42679691314697\\
106.625 -8.42753410339355\\
106.637496948242 -8.42827606201172\\
106.650001525879 -8.4290189743042\\
106.662498474121 -8.42976093292236\\
106.674995422363 -8.43050289154053\\
106.6875 -8.43124580383301\\
106.699996948242 -8.4319896697998\\
106.712501525879 -8.4327335357666\\
106.724998474121 -8.4334774017334\\
106.737495422363 -8.4342212677002\\
106.75 -8.43497371673584\\
106.762496948242 -8.43572521209717\\
106.775001525879 -8.43647766113281\\
106.787498474121 -8.43722915649414\\
106.799995422363 -8.43798065185547\\
106.8125 -8.43873310089111\\
106.824996948242 -8.43948650360107\\
106.837501525879 -8.44023990631104\\
106.849998474121 -8.44099426269531\\
106.862495422363 -8.4417552947998\\
106.875 -8.44251728057861\\
106.887496948242 -8.44328022003174\\
106.900001525879 -8.44404888153076\\
106.912498474121 -8.44481945037842\\
106.924995422363 -8.44559001922607\\
106.9375 -8.44636154174805\\
106.949996948242 -8.4471321105957\\
106.962501525879 -8.44790363311768\\
106.974998474121 -8.44867515563965\\
106.987495422363 -8.44944763183594\\
107 -8.45022010803223\\
107.012496948242 -8.45099258422852\\
107.025001525879 -8.4517650604248\\
107.037498474121 -8.45253753662109\\
107.049995422363 -8.45331001281738\\
107.0625 -8.45408248901367\\
107.074996948242 -8.45486259460449\\
107.087501525879 -8.45564365386963\\
107.099998474121 -8.45642375946045\\
107.112495422363 -8.45720100402832\\
107.125 -8.45797348022461\\
107.137496948242 -8.4587459564209\\
107.150001525879 -8.45951843261719\\
107.162498474121 -8.46028900146484\\
107.174995422363 -8.4610595703125\\
107.1875 -8.46182346343994\\
107.199996948242 -8.46258640289307\\
107.212501525879 -8.46334838867188\\
107.224998474121 -8.46410942077637\\
107.237495422363 -8.46486568450928\\
107.25 -8.46561908721924\\
107.262496948242 -8.46637153625488\\
107.275001525879 -8.46712398529053\\
107.287498474121 -8.46787548065186\\
107.299995422363 -8.46862697601318\\
107.3125 -8.46937084197998\\
107.324996948242 -8.47011470794678\\
107.337501525879 -8.47085857391357\\
107.349998474121 -8.47160148620605\\
107.362495422363 -8.47234344482422\\
107.375 -8.4730863571167\\
107.387496948242 -8.47382831573486\\
107.400001525879 -8.47456645965576\\
107.412498474121 -8.47530078887939\\
107.424995422363 -8.47603511810303\\
107.4375 -8.47676944732666\\
107.449996948242 -8.47750282287598\\
107.462501525879 -8.47823619842529\\
107.474998474121 -8.47896862030029\\
107.487495422363 -8.47970104217529\\
107.5 -8.48043441772461\\
107.512496948242 -8.48116683959961\\
107.525001525879 -8.48189258575439\\
107.537498474121 -8.48261737823486\\
107.549995422363 -8.48334217071533\\
107.5625 -8.4840669631958\\
107.574996948242 -8.48479175567627\\
107.587501525879 -8.48551654815674\\
107.599998474121 -8.48624134063721\\
107.612495422363 -8.48696613311768\\
107.625 -8.48769092559814\\
107.637496948242 -8.48841571807861\\
107.650001525879 -8.48914051055908\\
107.662498474121 -8.48986530303955\\
107.674995422363 -8.49059009552002\\
107.6875 -8.49131488800049\\
107.699996948242 -8.49203968048096\\
107.712501525879 -8.49276447296143\\
107.724998474121 -8.49349117279053\\
107.737495422363 -8.49422359466553\\
107.75 -8.49495697021484\\
107.762496948242 -8.49568939208984\\
107.775001525879 -8.49642276763916\\
107.787498474121 -8.49715709686279\\
107.799995422363 -8.49789142608643\\
107.8125 -8.49862575531006\\
107.824996948242 -8.49936294555664\\
107.837501525879 -8.50010585784912\\
107.849998474121 -8.50084781646729\\
107.862495422363 -8.50159072875977\\
107.875 -8.50233459472656\\
107.887496948242 -8.50307846069336\\
107.900001525879 -8.50382804870605\\
107.912498474121 -8.50457954406738\\
107.924995422363 -8.50533103942871\\
107.9375 -8.50608348846436\\
107.949996948242 -8.50683498382568\\
107.962501525879 -8.50758743286133\\
107.974998474121 -8.50834083557129\\
107.987495422363 -8.50909423828125\\
108 -8.50985240936279\\
108.012496948242 -8.51061344146729\\
108.025001525879 -8.51137542724609\\
108.037498474121 -8.51213645935059\\
108.049995422363 -8.51289749145508\\
108.0625 -8.51365947723389\\
108.074996948242 -8.51442050933838\\
108.087501525879 -8.51518249511719\\
108.099998474121 -8.51594543457031\\
108.112495422363 -8.51670837402344\\
108.125 -8.51747131347656\\
108.137496948242 -8.51823425292969\\
108.150001525879 -8.51899719238281\\
108.162498474121 -8.51976013183594\\
108.174995422363 -8.52052211761475\\
108.1875 -8.52128410339355\\
108.199996948242 -8.52204513549805\\
108.212501525879 -8.52280044555664\\
108.224998474121 -8.5235538482666\\
108.237495422363 -8.52430629730225\\
108.25 -8.52505874633789\\
108.262496948242 -8.52581024169922\\
108.275001525879 -8.52655601501465\\
108.287498474121 -8.52729988098145\\
108.299995422363 -8.52804279327393\\
108.3125 -8.52878570556641\\
108.324996948242 -8.52952766418457\\
108.337501525879 -8.53026294708252\\
108.349998474121 -8.53099727630615\\
108.362495422363 -8.53173160552979\\
108.375 -8.53246593475342\\
108.387496948242 -8.53320026397705\\
108.400001525879 -8.53393459320068\\
108.412498474121 -8.53466796875\\
108.424995422363 -8.535400390625\\
108.4375 -8.53613376617432\\
108.449996948242 -8.536865234375\\
108.462501525879 -8.53759002685547\\
108.474998474121 -8.53831481933594\\
108.487495422363 -8.53903961181641\\
108.5 -8.53976345062256\\
108.512496948242 -8.54048633575439\\
108.525001525879 -8.54121017456055\\
108.537498474121 -8.54193305969238\\
108.549995422363 -8.54265594482422\\
108.5625 -8.54337310791016\\
108.574996948242 -8.54408836364746\\
108.587501525879 -8.54480361938477\\
108.599998474121 -8.54551887512207\\
108.612495422363 -8.54623413085938\\
108.625 -8.54694938659668\\
108.637496948242 -8.54766464233398\\
108.650001525879 -8.54837989807129\\
108.662498474121 -8.54909515380859\\
108.674995422363 -8.5498104095459\\
108.6875 -8.5505256652832\\
108.699996948242 -8.55124092102051\\
108.712501525879 -8.55195617675781\\
108.724998474121 -8.55267143249512\\
108.737495422363 -8.55338668823242\\
108.75 -8.55410194396973\\
108.762496948242 -8.55481719970703\\
108.775001525879 -8.55553531646729\\
108.787498474121 -8.55625438690186\\
108.799995422363 -8.55697727203369\\
108.8125 -8.55770111083984\\
108.824996948242 -8.55842399597168\\
108.837501525879 -8.55914878845215\\
108.849998474121 -8.55987358093262\\
108.862495422363 -8.5605993270874\\
108.875 -8.56133270263672\\
108.887496948242 -8.56206512451172\\
108.900001525879 -8.56279850006104\\
108.912498474121 -8.56353187561035\\
108.924995422363 -8.56426620483398\\
108.9375 -8.56500053405762\\
108.949996948242 -8.56573486328125\\
108.962501525879 -8.5664758682251\\
108.974998474121 -8.56721782684326\\
108.987495422363 -8.56795978546143\\
109 -8.56870269775391\\
109.012496948242 -8.5694465637207\\
109.025001525879 -8.5701904296875\\
109.037498474121 -8.5709342956543\\
109.049995422363 -8.57167816162109\\
109.0625 -8.57243061065674\\
109.074996948242 -8.57318210601807\\
109.087501525879 -8.57393455505371\\
109.099998474121 -8.57468605041504\\
109.112495422363 -8.57543754577637\\
109.125 -8.57619094848633\\
109.137496948242 -8.57694435119629\\
109.150001525879 -8.57769775390625\\
109.162498474121 -8.57845878601074\\
109.174995422363 -8.57921504974365\\
109.1875 -8.57996845245361\\
109.199996948242 -8.58072185516357\\
109.212501525879 -8.58147430419922\\
109.224998474121 -8.58222389221191\\
109.237495422363 -8.58296775817871\\
109.25 -8.58371067047119\\
109.262496948242 -8.58445262908936\\
109.275001525879 -8.58519554138184\\
109.287498474121 -8.58593463897705\\
109.299995422363 -8.58666896820068\\
109.3125 -8.58740329742432\\
109.324996948242 -8.58813762664795\\
109.337501525879 -8.58887195587158\\
109.349998474121 -8.58960437774658\\
109.362495422363 -8.59033679962158\\
109.375 -8.5910701751709\\
109.387496948242 -8.5918025970459\\
109.400001525879 -8.59253597259521\\
109.412498474121 -8.59326171875\\
109.424995422363 -8.59398651123047\\
109.4375 -8.59471130371094\\
109.449996948242 -8.59543609619141\\
109.462501525879 -8.59616088867188\\
109.474998474121 -8.59688377380371\\
109.487495422363 -8.59760665893555\\
109.5 -8.5983304977417\\
109.512496948242 -8.59905242919922\\
109.525001525879 -8.59976768493652\\
109.537498474121 -8.60048294067383\\
109.549995422363 -8.60119819641113\\
109.5625 -8.60191345214844\\
109.574996948242 -8.60262870788574\\
109.587501525879 -8.60334396362305\\
109.599998474121 -8.60405731201172\\
109.612495422363 -8.60477066040039\\
109.625 -8.60548496246338\\
109.637496948242 -8.60619831085205\\
109.650001525879 -8.60690879821777\\
109.662498474121 -8.60761451721191\\
109.674995422363 -8.60832023620605\\
109.6875 -8.6090259552002\\
109.699996948242 -8.60973167419434\\
109.712501525879 -8.61043739318848\\
109.724998474121 -8.61114311218262\\
109.737495422363 -8.61184883117676\\
109.75 -8.6125545501709\\
109.762496948242 -8.61325931549072\\
109.775001525879 -8.61396408081055\\
109.787498474121 -8.61466789245605\\
109.799995422363 -8.6153736114502\\
109.8125 -8.61607933044434\\
109.824996948242 -8.61678504943848\\
109.837501525879 -8.61749076843262\\
109.849998474121 -8.61819648742676\\
109.862495422363 -8.6189022064209\\
109.875 -8.61961555480957\\
109.887496948242 -8.62032985687256\\
109.900001525879 -8.62104511260986\\
109.912498474121 -8.62176513671875\\
109.924995422363 -8.62248802185059\\
109.9375 -8.62321186065674\\
109.949996948242 -8.62393474578857\\
109.962501525879 -8.62465858459473\\
109.974998474121 -8.62538146972656\\
109.987495422363 -8.62610626220703\\
110 -8.6268310546875\\
110.012496948242 -8.62755584716797\\
110.025001525879 -8.62828063964844\\
110.037498474121 -8.62900543212891\\
110.049995422363 -8.62973785400391\\
110.0625 -8.63047122955322\\
110.074996948242 -8.63120365142822\\
110.087501525879 -8.63193798065186\\
110.099998474121 -8.63267230987549\\
110.112495422363 -8.63340663909912\\
110.125 -8.63414096832275\\
110.137496948242 -8.63487529754639\\
110.150001525879 -8.63560962677002\\
110.162498474121 -8.63634395599365\\
110.174995422363 -8.63707828521729\\
110.1875 -8.63781452178955\\
110.199996948242 -8.63855648040771\\
110.212501525879 -8.6392993927002\\
110.224998474121 -8.64004135131836\\
110.237495422363 -8.64078330993652\\
110.25 -8.641526222229\\
110.262496948242 -8.64226531982422\\
110.275001525879 -8.64299964904785\\
110.287498474121 -8.64373397827148\\
110.299995422363 -8.64446830749512\\
110.3125 -8.64520168304443\\
110.324996948242 -8.64593410491943\\
110.337501525879 -8.64666748046875\\
110.349998474121 -8.64739990234375\\
110.362495422363 -8.64813232421875\\
110.375 -8.64886569976807\\
110.387496948242 -8.64959812164307\\
110.400001525879 -8.65033149719238\\
110.412498474121 -8.65105819702148\\
110.424995422363 -8.65178298950195\\
110.4375 -8.65250778198242\\
110.449996948242 -8.65323257446289\\
110.462501525879 -8.65395736694336\\
110.474998474121 -8.6546802520752\\
110.487495422363 -8.65540313720703\\
110.5 -8.65612697601318\\
110.512496948242 -8.65684986114502\\
110.525001525879 -8.65757369995117\\
110.537498474121 -8.65829658508301\\
110.549995422363 -8.65901565551758\\
110.5625 -8.65973091125488\\
110.574996948242 -8.66044616699219\\
110.587501525879 -8.66116142272949\\
110.599998474121 -8.6618766784668\\
110.612495422363 -8.6625919342041\\
110.625 -8.66330623626709\\
110.637496948242 -8.66401958465576\\
110.650001525879 -8.66473388671875\\
110.662498474121 -8.66544723510742\\
110.674995422363 -8.66616058349609\\
110.6875 -8.66687488555908\\
110.699996948242 -8.66758823394775\\
110.712501525879 -8.66829967498779\\
110.724998474121 -8.66900539398193\\
110.737495422363 -8.66971111297607\\
110.75 -8.67041683197021\\
110.762496948242 -8.67112255096436\\
110.775001525879 -8.6718282699585\\
110.787498474121 -8.67253398895264\\
110.799995422363 -8.67323970794678\\
110.8125 -8.67394542694092\\
110.824996948242 -8.67465114593506\\
110.837501525879 -8.6753568649292\\
110.849998474121 -8.67606258392334\\
110.862495422363 -8.67676830291748\\
110.875 -8.67747402191162\\
110.887496948242 -8.67817974090576\\
110.900001525879 -8.6788854598999\\
110.912498474121 -8.67959117889404\\
110.924995422363 -8.68029689788818\\
110.9375 -8.68100166320801\\
110.949996948242 -8.68170547485352\\
110.962501525879 -8.68241024017334\\
110.974998474121 -8.68311405181885\\
110.987495422363 -8.68381786346436\\
111 -8.68452262878418\\
111.012496948242 -8.68522644042969\\
111.025001525879 -8.68593120574951\\
111.037498474121 -8.68663501739502\\
111.049995422363 -8.68733882904053\\
111.0625 -8.68804359436035\\
111.074996948242 -8.68874740600586\\
111.087501525879 -8.68945217132568\\
111.099998474121 -8.69015598297119\\
111.112495422363 -8.69086170196533\\
111.125 -8.69156742095947\\
111.137496948242 -8.69227313995361\\
111.150001525879 -8.69297885894775\\
111.162498474121 -8.69368934631348\\
111.174995422363 -8.69440269470215\\
111.1875 -8.69511795043945\\
111.199996948242 -8.69583320617676\\
111.212501525879 -8.69655227661133\\
111.224998474121 -8.69727516174316\\
111.237495422363 -8.69799900054932\\
111.25 -8.69872379302979\\
111.262496948242 -8.69944858551025\\
111.275001525879 -8.70017337799072\\
111.287498474121 -8.70090103149414\\
111.299995422363 -8.70163345336914\\
111.3125 -8.70236682891846\\
111.324996948242 -8.70309925079346\\
111.337501525879 -8.70383262634277\\
111.349998474121 -8.70456504821777\\
111.362495422363 -8.70529747009277\\
111.375 -8.70603084564209\\
111.387496948242 -8.70676326751709\\
111.400001525879 -8.70749664306641\\
111.412498474121 -8.70822906494141\\
111.424995422363 -8.70896148681641\\
111.4375 -8.70969009399414\\
111.449996948242 -8.71041488647461\\
111.462501525879 -8.71113967895508\\
111.474998474121 -8.71186447143555\\
111.487495422363 -8.7125883102417\\
111.5 -8.71331214904785\\
111.512496948242 -8.71403503417969\\
111.525001525879 -8.71475887298584\\
111.537498474121 -8.71548175811768\\
111.549995422363 -8.71620464324951\\
111.5625 -8.71692848205566\\
111.574996948242 -8.7176513671875\\
111.587501525879 -8.7183723449707\\
111.599998474121 -8.71908760070801\\
111.612495422363 -8.71980285644531\\
111.625 -8.72051811218262\\
111.637496948242 -8.72123336791992\\
111.650001525879 -8.72194766998291\\
111.662498474121 -8.72266101837158\\
111.674995422363 -8.72337436676025\\
111.6875 -8.72408103942871\\
111.699996948242 -8.72478675842285\\
111.712501525879 -8.72549247741699\\
111.724998474121 -8.72619819641113\\
111.737495422363 -8.72690391540527\\
111.75 -8.72760963439941\\
111.762496948242 -8.72831535339355\\
111.775001525879 -8.7290210723877\\
111.787498474121 -8.72972679138184\\
111.799995422363 -8.73043060302734\\
111.8125 -8.73113536834717\\
111.824996948242 -8.73183917999268\\
111.837501525879 -8.7325439453125\\
111.849998474121 -8.73324775695801\\
111.862495422363 -8.73394775390625\\
111.875 -8.73464393615723\\
111.887496948242 -8.7353401184082\\
111.900001525879 -8.73603630065918\\
111.912498474121 -8.73673248291016\\
111.924995422363 -8.73742866516113\\
111.9375 -8.73812484741211\\
111.949996948242 -8.73882102966309\\
111.962501525879 -8.73951721191406\\
111.974998474121 -8.74021244049072\\
111.987495422363 -8.74090671539307\\
112 -8.74160194396973\\
112.012496948242 -8.74229621887207\\
112.025001525879 -8.74299144744873\\
112.037498474121 -8.74368572235107\\
112.049995422363 -8.74437999725342\\
112.0625 -8.74507522583008\\
112.074996948242 -8.74576950073242\\
112.087501525879 -8.74646472930908\\
112.099998474121 -8.74715900421143\\
112.112495422363 -8.74785327911377\\
112.125 -8.74854850769043\\
112.137496948242 -8.74924278259277\\
112.150001525879 -8.7499303817749\\
112.162498474121 -8.75061702728271\\
112.174995422363 -8.75130367279053\\
112.1875 -8.75199031829834\\
112.199996948242 -8.7526798248291\\
112.212501525879 -8.75337505340576\\
112.224998474121 -8.75406932830811\\
112.237495422363 -8.75476360321045\\
112.25 -8.75545883178711\\
112.262496948242 -8.75615310668945\\
112.275001525879 -8.75684833526611\\
112.287498474121 -8.75754451751709\\
112.299995422363 -8.75824069976807\\
112.3125 -8.75893688201904\\
112.324996948242 -8.75963306427002\\
112.337501525879 -8.76033687591553\\
112.349998474121 -8.76104068756104\\
112.362495422363 -8.76174449920654\\
112.375 -8.76244926452637\\
112.387496948242 -8.76315307617188\\
112.400001525879 -8.76385879516602\\
112.412498474121 -8.76456451416016\\
112.424995422363 -8.7652702331543\\
112.4375 -8.76597595214844\\
112.449996948242 -8.76668167114258\\
112.462501525879 -8.76738739013672\\
112.474998474121 -8.76810073852539\\
112.487495422363 -8.76881408691406\\
112.5 -8.76952838897705\\
112.512496948242 -8.77024173736572\\
112.525001525879 -8.77095603942871\\
112.537498474121 -8.77166938781738\\
112.549995422363 -8.77238273620605\\
112.5625 -8.77309417724609\\
112.574996948242 -8.77379989624023\\
112.587501525879 -8.77450561523438\\
112.599998474121 -8.77521133422852\\
112.612495422363 -8.77591705322266\\
112.625 -8.7766227722168\\
112.637496948242 -8.77732849121094\\
112.650001525879 -8.77803421020508\\
112.662498474121 -8.77873992919922\\
112.674995422363 -8.77944564819336\\
112.6875 -8.7801513671875\\
112.699996948242 -8.78085708618164\\
112.712501525879 -8.78156185150146\\
112.724998474121 -8.78226566314697\\
112.737495422363 -8.78296947479248\\
112.75 -8.7836742401123\\
112.762496948242 -8.78437805175781\\
112.775001525879 -8.78508281707764\\
112.787498474121 -8.78577899932861\\
112.799995422363 -8.78647518157959\\
112.8125 -8.78717136383057\\
112.824996948242 -8.78786754608154\\
112.837501525879 -8.78856372833252\\
112.849998474121 -8.7892599105835\\
112.862495422363 -8.78995418548584\\
112.875 -8.7906494140625\\
112.887496948242 -8.79134178161621\\
112.900001525879 -8.79202842712402\\
112.912498474121 -8.79271507263184\\
112.924995422363 -8.79340171813965\\
112.9375 -8.79408836364746\\
112.949996948242 -8.79477500915527\\
112.962501525879 -8.79546165466309\\
112.974998474121 -8.7961483001709\\
112.987495422363 -8.79683303833008\\
113 -8.79751873016357\\
113.012496948242 -8.79820346832275\\
113.025001525879 -8.79888916015625\\
113.037498474121 -8.79957389831543\\
113.049995422363 -8.80025863647461\\
113.0625 -8.80094432830811\\
113.074996948242 -8.80162334442139\\
113.087501525879 -8.80230045318604\\
113.099998474121 -8.80297756195068\\
113.112495422363 -8.80365467071533\\
113.125 -8.80433177947998\\
113.137496948242 -8.80500888824463\\
113.150001525879 -8.80568599700928\\
113.162498474121 -8.80636310577393\\
113.174995422363 -8.80703926086426\\
113.1875 -8.80771541595459\\
113.199996948242 -8.80839061737061\\
113.212501525879 -8.80906677246094\\
113.224998474121 -8.80974197387695\\
113.237495422363 -8.81041717529297\\
113.25 -8.8110933303833\\
113.262496948242 -8.81176853179932\\
113.275001525879 -8.81244468688965\\
113.287498474121 -8.81311321258545\\
113.299995422363 -8.81378078460693\\
113.3125 -8.81444835662842\\
113.324996948242 -8.8151159286499\\
113.337501525879 -8.81578350067139\\
113.349998474121 -8.81645107269287\\
113.362495422363 -8.81711864471436\\
113.375 -8.81778621673584\\
113.387496948242 -8.81845378875732\\
113.400001525879 -8.81912517547607\\
113.412498474121 -8.81980037689209\\
113.424995422363 -8.82047557830811\\
113.4375 -8.82115173339844\\
113.449996948242 -8.82182693481445\\
113.462501525879 -8.82250308990479\\
113.474998474121 -8.82318019866943\\
113.487495422363 -8.82385730743408\\
113.5 -8.82453441619873\\
113.512496948242 -8.82521152496338\\
113.525001525879 -8.82588863372803\\
113.537498474121 -8.82657337188721\\
113.549995422363 -8.82725811004639\\
113.5625 -8.82794380187988\\
113.574996948242 -8.82862854003906\\
113.587501525879 -8.82931518554688\\
113.599998474121 -8.83000183105469\\
113.612495422363 -8.8306884765625\\
113.625 -8.83137512207031\\
113.637496948242 -8.83206176757813\\
113.650001525879 -8.83274841308594\\
113.662498474121 -8.83343505859375\\
113.674995422363 -8.83412170410156\\
113.6875 -8.83480834960938\\
113.699996948242 -8.83549499511719\\
113.712501525879 -8.836181640625\\
113.724998474121 -8.83686828613281\\
113.737495422363 -8.83755493164063\\
113.75 -8.83824157714844\\
113.762496948242 -8.83892822265625\\
113.775001525879 -8.83961486816406\\
113.787498474121 -8.84030151367188\\
113.799995422363 -8.84098815917969\\
113.8125 -8.84167385101318\\
113.824996948242 -8.84235858917236\\
113.837501525879 -8.84304428100586\\
113.849998474121 -8.84372901916504\\
113.862495422363 -8.84441089630127\\
113.875 -8.84508800506592\\
113.887496948242 -8.84576511383057\\
113.900001525879 -8.84644222259521\\
113.912498474121 -8.84711933135986\\
113.924995422363 -8.84779453277588\\
113.9375 -8.84847068786621\\
113.949996948242 -8.84914588928223\\
113.962501525879 -8.84982204437256\\
113.974998474121 -8.85049724578857\\
113.987495422363 -8.85117053985596\\
114 -8.85183811187744\\
114.012496948242 -8.85250568389893\\
114.025001525879 -8.85317325592041\\
114.037498474121 -8.85384082794189\\
114.049995422363 -8.85450839996338\\
114.0625 -8.85517597198486\\
114.074996948242 -8.85584163665771\\
114.087501525879 -8.85650825500488\\
114.099998474121 -8.85717391967773\\
114.112495422363 -8.85783958435059\\
114.125 -8.85850238800049\\
114.137496948242 -8.85916042327881\\
114.150001525879 -8.85981845855713\\
114.162498474121 -8.86047649383545\\
114.174995422363 -8.86113452911377\\
114.1875 -8.86179256439209\\
114.199996948242 -8.86244869232178\\
114.212501525879 -8.86310577392578\\
114.224998474121 -8.86376190185547\\
114.237495422363 -8.86441230773926\\
114.25 -8.86506080627441\\
114.262496948242 -8.86570930480957\\
114.275001525879 -8.86635780334473\\
114.287498474121 -8.86700630187988\\
114.299995422363 -8.86765480041504\\
114.3125 -8.8683032989502\\
114.324996948242 -8.86895084381104\\
114.337501525879 -8.86959838867188\\
114.349998474121 -8.8702449798584\\
114.362495422363 -8.87089157104492\\
114.375 -8.87153911590576\\
114.387496948242 -8.87218570709229\\
114.400001525879 -8.87283325195313\\
114.412498474121 -8.87347984313965\\
114.424995422363 -8.87412643432617\\
114.4375 -8.87477493286133\\
114.449996948242 -8.87542343139648\\
114.462501525879 -8.87607192993164\\
114.474998474121 -8.8767204284668\\
114.487495422363 -8.87736892700195\\
114.5 -8.87801742553711\\
114.512496948242 -8.87866592407227\\
114.525001525879 -8.87931442260742\\
114.537498474121 -8.87996292114258\\
114.549995422363 -8.88061141967773\\
114.5625 -8.88125991821289\\
114.574996948242 -8.88190841674805\\
114.587501525879 -8.8825569152832\\
114.599998474121 -8.88320541381836\\
114.612495422363 -8.8838586807251\\
114.625 -8.8845157623291\\
114.637496948242 -8.88517189025879\\
114.650001525879 -8.88582992553711\\
114.662498474121 -8.88648796081543\\
114.674995422363 -8.88714599609375\\
114.6875 -8.88780403137207\\
114.699996948242 -8.88846683502197\\
114.712501525879 -8.88913345336914\\
114.724998474121 -8.88979911804199\\
114.737495422363 -8.89046478271484\\
114.75 -8.89113140106201\\
114.762496948242 -8.89179706573486\\
114.775001525879 -8.89246368408203\\
114.787498474121 -8.89313125610352\\
114.799995422363 -8.893798828125\\
114.8125 -8.89446640014648\\
114.824996948242 -8.89513397216797\\
114.837501525879 -8.89580154418945\\
114.849998474121 -8.89646911621094\\
114.862495422363 -8.89713668823242\\
114.875 -8.89780426025391\\
114.887496948242 -8.89847183227539\\
114.900001525879 -8.89913940429688\\
114.912498474121 -8.89980697631836\\
114.924995422363 -8.90047454833984\\
114.9375 -8.90114212036133\\
114.949996948242 -8.90180778503418\\
114.962501525879 -8.90247440338135\\
114.974998474121 -8.9031400680542\\
114.987495422363 -8.90380573272705\\
115 -8.90446376800537\\
115.012496948242 -8.90512180328369\\
115.025001525879 -8.90577983856201\\
115.037498474121 -8.90643787384033\\
115.049995422363 -8.90709590911865\\
115.0625 -8.90775394439697\\
115.074996948242 -8.90841007232666\\
115.087501525879 -8.90906715393066\\
115.099998474121 -8.90972232818604\\
115.112495422363 -8.91037082672119\\
115.125 -8.91101932525635\\
115.137496948242 -8.9116678237915\\
115.150001525879 -8.91231632232666\\
115.162498474121 -8.91296482086182\\
115.174995422363 -8.91361236572266\\
115.1875 -8.9142599105835\\
115.199996948242 -8.91490650177002\\
115.212501525879 -8.91555404663086\\
115.224998474121 -8.91619300842285\\
115.237495422363 -8.91683197021484\\
115.25 -8.91747093200684\\
115.262496948242 -8.91810989379883\\
115.275001525879 -8.91874885559082\\
115.287498474121 -8.91938781738281\\
115.299995422363 -8.92002487182617\\
115.3125 -8.92066287994385\\
115.324996948242 -8.92129993438721\\
115.337501525879 -8.92193603515625\\
115.349998474121 -8.92256546020508\\
115.362495422363 -8.92319488525391\\
115.375 -8.92382431030273\\
115.387496948242 -8.92445373535156\\
115.400001525879 -8.92508316040039\\
115.412498474121 -8.92571067810059\\
115.424995422363 -8.92633819580078\\
115.4375 -8.92696666717529\\
115.449996948242 -8.92759418487549\\
115.462501525879 -8.92822265625\\
115.474998474121 -8.9288501739502\\
115.487495422363 -8.92947769165039\\
115.5 -8.93010330200195\\
115.512496948242 -8.93072319030762\\
115.525001525879 -8.93134307861328\\
115.537498474121 -8.93196296691895\\
115.549995422363 -8.93258285522461\\
115.5625 -8.93320274353027\\
115.574996948242 -8.93382263183594\\
115.587501525879 -8.9344425201416\\
115.599998474121 -8.93506240844727\\
115.612495422363 -8.93568229675293\\
115.625 -8.93630790710449\\
115.637496948242 -8.93693542480469\\
115.650001525879 -8.9375638961792\\
115.662498474121 -8.93819141387939\\
115.674995422363 -8.93881893157959\\
115.6875 -8.93944835662842\\
115.699996948242 -8.94007778167725\\
115.712501525879 -8.94070720672607\\
115.724998474121 -8.9413366317749\\
115.737495422363 -8.94196605682373\\
115.75 -8.94259548187256\\
115.762496948242 -8.94322490692139\\
115.775001525879 -8.94386100769043\\
115.787498474121 -8.94449806213379\\
115.799995422363 -8.94513511657715\\
115.8125 -8.94577407836914\\
115.824996948242 -8.94641304016113\\
115.837501525879 -8.94705200195313\\
115.849998474121 -8.94769096374512\\
115.862495422363 -8.94832992553711\\
115.875 -8.94897174835205\\
115.887496948242 -8.94961833953857\\
115.900001525879 -8.95026588439941\\
115.912498474121 -8.95091247558594\\
115.924995422363 -8.95155906677246\\
115.9375 -8.9522066116333\\
115.949996948242 -8.95285511016846\\
115.962501525879 -8.95350360870361\\
115.974998474121 -8.95415210723877\\
115.987495422363 -8.95480060577393\\
116 -8.95544910430908\\
116.012496948242 -8.95609760284424\\
116.025001525879 -8.95674610137939\\
116.037498474121 -8.95739364624023\\
116.049995422363 -8.95804023742676\\
116.0625 -8.9586877822876\\
116.074996948242 -8.95933437347412\\
116.087501525879 -8.95998191833496\\
116.099998474121 -8.96062278747559\\
116.112495422363 -8.96126174926758\\
116.125 -8.96190071105957\\
116.137496948242 -8.96253967285156\\
116.150001525879 -8.96317863464355\\
116.162498474121 -8.96381759643555\\
116.174995422363 -8.96445465087891\\
116.1875 -8.96509265899658\\
116.199996948242 -8.96572971343994\\
116.212501525879 -8.96636009216309\\
116.224998474121 -8.96698951721191\\
116.237495422363 -8.96761894226074\\
116.25 -8.96824836730957\\
116.262496948242 -8.9688777923584\\
116.275001525879 -8.96950721740723\\
116.287498474121 -8.97013664245605\\
116.299995422363 -8.97076511383057\\
116.3125 -8.97139358520508\\
116.324996948242 -8.97202110290527\\
116.337501525879 -8.97264957427979\\
116.349998474121 -8.97327613830566\\
116.362495422363 -8.97389602661133\\
116.375 -8.97451591491699\\
116.387496948242 -8.97513580322266\\
116.400001525879 -8.97575569152832\\
116.412498474121 -8.97637462615967\\
116.424995422363 -8.9769926071167\\
116.4375 -8.97761154174805\\
116.449996948242 -8.97822856903076\\
116.462501525879 -8.97883892059326\\
116.474998474121 -8.97944927215576\\
116.487495422363 -8.98005962371826\\
116.5 -8.98066997528076\\
116.512496948242 -8.98128032684326\\
116.525001525879 -8.98189067840576\\
116.537498474121 -8.98250007629395\\
116.549995422363 -8.98310852050781\\
116.5625 -8.983717918396\\
116.574996948242 -8.98432064056396\\
116.587501525879 -8.9849214553833\\
116.599998474121 -8.98552227020264\\
116.612495422363 -8.98612308502197\\
116.625 -8.98672389984131\\
116.637496948242 -8.98732471466064\\
116.650001525879 -8.98792552947998\\
116.662498474121 -8.98852634429932\\
116.674995422363 -8.98912715911865\\
116.6875 -8.98972797393799\\
116.699996948242 -8.99032878875732\\
116.712501525879 -8.99092960357666\\
116.724998474121 -8.991530418396\\
116.737495422363 -8.99213123321533\\
116.75 -8.99273204803467\\
116.762496948242 -8.993332862854\\
116.775001525879 -8.99393367767334\\
116.787498474121 -8.99453449249268\\
116.799995422363 -8.99513912200928\\
116.8125 -8.99574851989746\\
116.824996948242 -8.99635696411133\\
116.837501525879 -8.99696636199951\\
116.849998474121 -8.9975757598877\\
116.862495422363 -8.9981861114502\\
116.875 -8.9987964630127\\
116.887496948242 -8.9994068145752\\
116.900001525879 -9.0000171661377\\
116.912498474121 -9.0006275177002\\
116.924995422363 -9.00124359130859\\
116.9375 -9.00186252593994\\
116.949996948242 -9.00248146057129\\
116.962501525879 -9.00310134887695\\
116.974998474121 -9.00372123718262\\
116.987495422363 -9.00434112548828\\
117 -9.00496101379395\\
117.012496948242 -9.00558090209961\\
117.025001525879 -9.00620269775391\\
117.037498474121 -9.0068302154541\\
117.049995422363 -9.0074577331543\\
117.0625 -9.00808620452881\\
117.074996948242 -9.008713722229\\
117.087501525879 -9.00934314727783\\
117.099998474121 -9.00997257232666\\
117.112495422363 -9.01060199737549\\
117.125 -9.01123142242432\\
117.137496948242 -9.01186084747314\\
117.150001525879 -9.01249027252197\\
117.162498474121 -9.0131196975708\\
117.174995422363 -9.01374912261963\\
117.1875 -9.01437854766846\\
117.199996948242 -9.01500606536865\\
117.212501525879 -9.01563453674316\\
117.224998474121 -9.01626205444336\\
117.237495422363 -9.01688957214355\\
117.25 -9.01751804351807\\
117.262496948242 -9.01814460754395\\
117.275001525879 -9.01876449584961\\
117.287498474121 -9.01938438415527\\
117.299995422363 -9.02000427246094\\
117.3125 -9.0206241607666\\
117.324996948242 -9.02124309539795\\
117.337501525879 -9.0218620300293\\
117.349998474121 -9.02248001098633\\
117.362495422363 -9.02309322357178\\
117.375 -9.02370357513428\\
117.387496948242 -9.02431392669678\\
117.400001525879 -9.02492427825928\\
117.412498474121 -9.02553462982178\\
117.424995422363 -9.02614498138428\\
117.4375 -9.02675533294678\\
117.449996948242 -9.02736473083496\\
117.462501525879 -9.02797412872314\\
117.474998474121 -9.02858257293701\\
117.487495422363 -9.02919101715088\\
117.5 -9.02980041503906\\
117.512496948242 -9.03040218353271\\
117.525001525879 -9.03100299835205\\
117.537498474121 -9.03160381317139\\
117.549995422363 -9.03220462799072\\
117.5625 -9.03280544281006\\
117.574996948242 -9.03340435028076\\
117.587501525879 -9.03400421142578\\
117.599998474121 -9.03460311889648\\
117.612495422363 -9.03520202636719\\
117.625 -9.03580188751221\\
117.637496948242 -9.03639602661133\\
117.650001525879 -9.0369873046875\\
117.662498474121 -9.03757858276367\\
117.674995422363 -9.03816986083984\\
117.6875 -9.03876113891602\\
117.699996948242 -9.03935241699219\\
117.712501525879 -9.03994369506836\\
117.724998474121 -9.04053497314453\\
117.737495422363 -9.0411262512207\\
117.75 -9.04171752929688\\
117.762496948242 -9.04230880737305\\
117.775001525879 -9.04290008544922\\
117.787498474121 -9.04349136352539\\
117.799995422363 -9.04408073425293\\
117.8125 -9.04467105865479\\
117.824996948242 -9.04526042938232\\
117.837501525879 -9.04585075378418\\
117.849998474121 -9.04644012451172\\
117.862495422363 -9.04702949523926\\
117.875 -9.04761981964111\\
117.887496948242 -9.04820919036865\\
117.900001525879 -9.04879951477051\\
117.912498474121 -9.04938507080078\\
117.924995422363 -9.04996681213379\\
117.9375 -9.0505542755127\\
117.949996948242 -9.05114364624023\\
117.962501525879 -9.05173397064209\\
117.974998474121 -9.05232334136963\\
117.987495422363 -9.05291271209717\\
118 -9.05350303649902\\
118.012496948242 -9.05409336090088\\
118.025001525879 -9.05468463897705\\
118.037498474121 -9.05527591705322\\
118.049995422363 -9.05586719512939\\
118.0625 -9.05645847320557\\
118.074996948242 -9.05704975128174\\
118.087501525879 -9.05764102935791\\
118.099998474121 -9.05823707580566\\
118.112495422363 -9.05883598327637\\
118.125 -9.05943584442139\\
118.137496948242 -9.06003475189209\\
118.150001525879 -9.06063461303711\\
118.162498474121 -9.06123447418213\\
118.174995422363 -9.06183528900146\\
118.1875 -9.0624361038208\\
118.199996948242 -9.06303691864014\\
118.212501525879 -9.06363773345947\\
118.224998474121 -9.06423854827881\\
118.237495422363 -9.06483936309814\\
118.25 -9.06544017791748\\
118.262496948242 -9.06604862213135\\
118.275001525879 -9.06665802001953\\
118.287498474121 -9.0672664642334\\
118.299995422363 -9.06787490844727\\
118.3125 -9.06848430633545\\
118.324996948242 -9.06909275054932\\
118.337501525879 -9.0697021484375\\
118.349998474121 -9.07031059265137\\
118.362495422363 -9.07091903686523\\
118.375 -9.07152843475342\\
118.387496948242 -9.07213687896729\\
118.400001525879 -9.07274627685547\\
118.412498474121 -9.07335472106934\\
118.424995422363 -9.07395553588867\\
118.4375 -9.07455635070801\\
118.449996948242 -9.07515716552734\\
118.462501525879 -9.07575798034668\\
118.474998474121 -9.07635879516602\\
118.487495422363 -9.07695960998535\\
118.5 -9.07756042480469\\
118.512496948242 -9.07816123962402\\
118.525001525879 -9.07876110076904\\
118.537498474121 -9.07936000823975\\
118.549995422363 -9.07995891571045\\
118.5625 -9.08055114746094\\
118.574996948242 -9.08114242553711\\
118.587501525879 -9.08173370361328\\
118.599998474121 -9.08232498168945\\
118.612495422363 -9.08291625976563\\
118.625 -9.08350658416748\\
118.637496948242 -9.08409595489502\\
118.650001525879 -9.08468627929688\\
118.662498474121 -9.08527278900146\\
118.674995422363 -9.08585453033447\\
118.6875 -9.08643627166748\\
118.699996948242 -9.08701801300049\\
118.712501525879 -9.0875997543335\\
118.724998474121 -9.0881814956665\\
118.737495422363 -9.08876323699951\\
118.75 -9.0893440246582\\
118.762496948242 -9.08992385864258\\
118.775001525879 -9.09050464630127\\
118.787498474121 -9.09108448028564\\
118.799995422363 -9.09166049957275\\
118.8125 -9.0922327041626\\
118.824996948242 -9.09280490875244\\
118.837501525879 -9.09337711334229\\
118.849998474121 -9.09394931793213\\
118.862495422363 -9.09452056884766\\
118.875 -9.09509181976318\\
118.887496948242 -9.09566211700439\\
118.900001525879 -9.09623336791992\\
118.912498474121 -9.09680366516113\\
118.924995422363 -9.09737396240234\\
118.9375 -9.09794521331787\\
118.949996948242 -9.09851551055908\\
118.962501525879 -9.09908676147461\\
118.974998474121 -9.09965705871582\\
118.987495422363 -9.10022735595703\\
119 -9.10079860687256\\
119.012496948242 -9.10136890411377\\
119.025001525879 -9.1019401550293\\
119.037498474121 -9.10251045227051\\
119.049995422363 -9.10308074951172\\
119.0625 -9.10365200042725\\
119.074996948242 -9.10422229766846\\
119.087501525879 -9.10479354858398\\
119.099998474121 -9.1053638458252\\
119.112495422363 -9.10593509674072\\
119.125 -9.10650730133057\\
119.137496948242 -9.10707950592041\\
119.150001525879 -9.10765171051025\\
119.162498474121 -9.1082239151001\\
119.174995422363 -9.10879993438721\\
119.1875 -9.1093807220459\\
119.199996948242 -9.10996055603027\\
119.212501525879 -9.11054134368896\\
119.224998474121 -9.11112117767334\\
119.237495422363 -9.11170101165771\\
119.25 -9.11228275299072\\
119.262496948242 -9.11286449432373\\
119.275001525879 -9.11344623565674\\
119.287498474121 -9.11402797698975\\
119.299995422363 -9.11460971832275\\
119.3125 -9.11519145965576\\
119.324996948242 -9.11577320098877\\
119.337501525879 -9.11635494232178\\
119.349998474121 -9.11693668365479\\
119.362495422363 -9.11752414703369\\
119.375 -9.11811447143555\\
119.387496948242 -9.11870384216309\\
119.400001525879 -9.11929416656494\\
119.412498474121 -9.11988353729248\\
119.424995422363 -9.12047290802002\\
119.4375 -9.12106323242188\\
119.449996948242 -9.12165260314941\\
119.462501525879 -9.12224292755127\\
119.474998474121 -9.12283229827881\\
119.487495422363 -9.12342166900635\\
119.5 -9.1240119934082\\
119.512496948242 -9.12460136413574\\
119.525001525879 -9.1251916885376\\
119.537498474121 -9.12577438354492\\
119.549995422363 -9.12635612487793\\
119.5625 -9.12693786621094\\
119.574996948242 -9.12751960754395\\
119.587501525879 -9.12810134887695\\
119.599998474121 -9.12868309020996\\
119.612495422363 -9.12926483154297\\
119.625 -9.12984657287598\\
119.637496948242 -9.13042640686035\\
119.650001525879 -9.13100719451904\\
119.662498474121 -9.13158702850342\\
119.674995422363 -9.13216304779053\\
119.6875 -9.13273525238037\\
119.699996948242 -9.13330745697021\\
119.712501525879 -9.13387966156006\\
119.724998474121 -9.1344518661499\\
119.737495422363 -9.13502311706543\\
119.75 -9.13559436798096\\
119.762496948242 -9.13616466522217\\
119.775001525879 -9.1367359161377\\
119.787498474121 -9.13730239868164\\
119.799995422363 -9.13786506652832\\
119.8125 -9.138427734375\\
119.824996948242 -9.13899040222168\\
119.837501525879 -9.13955307006836\\
119.849998474121 -9.14011573791504\\
119.862495422363 -9.1406774520874\\
119.875 -9.14123916625977\\
119.887496948242 -9.14179992675781\\
119.900001525879 -9.14236164093018\\
119.912498474121 -9.14292240142822\\
119.924995422363 -9.14348316192627\\
119.9375 -9.1440372467041\\
119.949996948242 -9.14459037780762\\
119.962501525879 -9.14514350891113\\
119.974998474121 -9.14569664001465\\
119.987495422363 -9.14624977111816\\
120 -9.14680290222168\\
120.012496948242 -9.1473560333252\\
120.025001525879 -9.14790916442871\\
120.037498474121 -9.14846229553223\\
120.049995422363 -9.14901542663574\\
120.0625 -9.14956855773926\\
120.074996948242 -9.15012168884277\\
120.087501525879 -9.15067481994629\\
120.099998474121 -9.1512279510498\\
120.112495422363 -9.15178108215332\\
120.125 -9.15233421325684\\
120.137496948242 -9.15288734436035\\
120.150001525879 -9.15344047546387\\
120.162498474121 -9.15399360656738\\
120.174995422363 -9.1545467376709\\
120.1875 -9.15509986877441\\
120.199996948242 -9.15565299987793\\
120.212501525879 -9.15620613098145\\
120.224998474121 -9.15676403045654\\
120.237495422363 -9.15732574462891\\
120.25 -9.15788841247559\\
120.262496948242 -9.15845108032227\\
120.275001525879 -9.15901851654053\\
120.287498474121 -9.15958976745605\\
120.299995422363 -9.1601619720459\\
120.3125 -9.16073417663574\\
120.324996948242 -9.16130638122559\\
120.337501525879 -9.16187858581543\\
120.349998474121 -9.16244888305664\\
120.362495422363 -9.16301918029785\\
120.375 -9.1635913848877\\
120.387496948242 -9.16416358947754\\
120.400001525879 -9.16473579406738\\
120.412498474121 -9.16530799865723\\
120.424995422363 -9.16588020324707\\
120.4375 -9.16645240783691\\
120.449996948242 -9.16702461242676\\
120.462501525879 -9.1675968170166\\
120.474998474121 -9.16816902160645\\
120.487495422363 -9.16874122619629\\
120.5 -9.16931343078613\\
120.512496948242 -9.16988563537598\\
120.525001525879 -9.17045783996582\\
120.537498474121 -9.17103004455566\\
120.549995422363 -9.17160224914551\\
120.5625 -9.17217636108398\\
120.574996948242 -9.17275619506836\\
120.587501525879 -9.17333698272705\\
120.599998474121 -9.17391681671143\\
120.612495422363 -9.1744966506958\\
120.625 -9.17507553100586\\
120.637496948242 -9.1756477355957\\
120.650001525879 -9.17621994018555\\
120.662498474121 -9.17679214477539\\
120.674995422363 -9.17736434936523\\
120.6875 -9.17793655395508\\
120.699996948242 -9.17850875854492\\
120.712501525879 -9.17908096313477\\
120.724998474121 -9.17965316772461\\
120.737495422363 -9.18022441864014\\
120.75 -9.18079662322998\\
120.762496948242 -9.18136692047119\\
120.775001525879 -9.18193817138672\\
120.787498474121 -9.18250846862793\\
120.799995422363 -9.18307781219482\\
120.8125 -9.1836404800415\\
120.824996948242 -9.18420314788818\\
120.837501525879 -9.18476581573486\\
120.849998474121 -9.18532848358154\\
120.862495422363 -9.18589115142822\\
120.875 -9.1864538192749\\
120.887496948242 -9.18701457977295\\
120.900001525879 -9.18757629394531\\
120.912498474121 -9.18813705444336\\
120.924995422363 -9.18869781494141\\
120.9375 -9.18925285339355\\
120.949996948242 -9.18980598449707\\
120.962501525879 -9.19035911560059\\
120.974998474121 -9.1909122467041\\
120.987495422363 -9.19146537780762\\
121 -9.19201850891113\\
121.012496948242 -9.19257164001465\\
121.025001525879 -9.19312477111816\\
121.037498474121 -9.19367694854736\\
121.049995422363 -9.19422817230225\\
121.0625 -9.19478034973145\\
121.074996948242 -9.19533157348633\\
121.087501525879 -9.19588375091553\\
121.099998474121 -9.19643497467041\\
121.112495422363 -9.19698619842529\\
121.125 -9.19753837585449\\
121.137496948242 -9.19808959960938\\
121.150001525879 -9.19864177703857\\
121.162498474121 -9.19919300079346\\
121.174995422363 -9.19974422454834\\
121.1875 -9.20029640197754\\
121.199996948242 -9.20084762573242\\
121.212501525879 -9.20139980316162\\
121.224998474121 -9.2019510269165\\
121.237495422363 -9.20250225067139\\
121.25 -9.20305442810059\\
121.262496948242 -9.20360565185547\\
121.275001525879 -9.20415782928467\\
121.287498474121 -9.20470905303955\\
121.299995422363 -9.20526027679443\\
121.3125 -9.20581245422363\\
121.324996948242 -9.20636367797852\\
121.337501525879 -9.20691680908203\\
121.349998474121 -9.20746898651123\\
121.362495422363 -9.20802021026611\\
121.375 -9.20857238769531\\
121.387496948242 -9.20912551879883\\
121.400001525879 -9.20967864990234\\
121.412498474121 -9.21023178100586\\
121.424995422363 -9.21078491210938\\
121.4375 -9.21133804321289\\
121.449996948242 -9.21189117431641\\
121.462501525879 -9.21244430541992\\
121.474998474121 -9.21299743652344\\
121.487495422363 -9.21355152130127\\
121.5 -9.21411323547363\\
121.512496948242 -9.21467399597168\\
121.525001525879 -9.21523571014404\\
121.537498474121 -9.21579647064209\\
121.549995422363 -9.21635723114014\\
121.5625 -9.2169189453125\\
121.574996948242 -9.21747970581055\\
121.587501525879 -9.21804141998291\\
121.599998474121 -9.21860218048096\\
121.612495422363 -9.219162940979\\
121.625 -9.21972465515137\\
121.637496948242 -9.22028255462646\\
121.650001525879 -9.22083568572998\\
121.662498474121 -9.2213888168335\\
121.674995422363 -9.22194194793701\\
121.6875 -9.22249507904053\\
121.699996948242 -9.22304821014404\\
121.712501525879 -9.22360134124756\\
121.724998474121 -9.22415351867676\\
121.737495422363 -9.22470474243164\\
121.75 -9.22525691986084\\
121.762496948242 -9.22580432891846\\
121.775001525879 -9.22634792327881\\
121.787498474121 -9.22689151763916\\
121.799995422363 -9.22743511199951\\
121.8125 -9.22797775268555\\
121.824996948242 -9.22851943969727\\
121.837501525879 -9.2290620803833\\
121.849998474121 -9.22959804534912\\
121.862495422363 -9.23013210296631\\
121.875 -9.2306661605835\\
121.887496948242 -9.23120021820068\\
121.900001525879 -9.23173332214355\\
121.912498474121 -9.23226070404053\\
121.924995422363 -9.23278522491455\\
121.9375 -9.23330974578857\\
121.949996948242 -9.2338342666626\\
121.962501525879 -9.23435878753662\\
121.974998474121 -9.23488330841064\\
121.987495422363 -9.23540782928467\\
122 -9.23593139648438\\
122.012496948242 -9.23645401000977\\
122.025001525879 -9.23697566986084\\
122.037498474121 -9.2374906539917\\
122.049995422363 -9.23800563812256\\
122.0625 -9.23852062225342\\
122.074996948242 -9.23903465270996\\
122.087501525879 -9.2395486831665\\
122.099998474121 -9.24006175994873\\
122.112495422363 -9.24057483673096\\
122.125 -9.24108028411865\\
122.137496948242 -9.24158573150635\\
122.150001525879 -9.24209117889404\\
122.162498474121 -9.24259662628174\\
122.174995422363 -9.24310207366943\\
122.1875 -9.24360752105713\\
122.199996948242 -9.24411296844482\\
122.212501525879 -9.24461841583252\\
122.224998474121 -9.2451229095459\\
122.237495422363 -9.24562644958496\\
122.25 -9.24613094329834\\
122.262496948242 -9.2466344833374\\
122.275001525879 -9.24713897705078\\
122.287498474121 -9.24764251708984\\
122.299995422363 -9.24814605712891\\
122.3125 -9.24864959716797\\
122.324996948242 -9.24914741516113\\
122.337501525879 -9.24964427947998\\
122.349998474121 -9.25014781951904\\
122.362495422363 -9.25064945220947\\
122.375 -9.251145362854\\
122.387496948242 -9.25164127349854\\
122.400001525879 -9.2521390914917\\
122.412498474121 -9.25264263153076\\
122.424995422363 -9.25314617156982\\
122.4375 -9.2536506652832\\
122.449996948242 -9.25415420532227\\
122.462501525879 -9.25465106964111\\
122.474998474121 -9.25514698028564\\
122.487495422363 -9.25564670562744\\
122.5 -9.25615119934082\\
122.512496948242 -9.25665473937988\\
122.525001525879 -9.25715923309326\\
122.537498474121 -9.25766277313232\\
122.549995422363 -9.25816631317139\\
122.5625 -9.25866794586182\\
122.574996948242 -9.25916385650635\\
122.587501525879 -9.25966835021973\\
122.599998474121 -9.26016521453857\\
122.612495422363 -9.26066112518311\\
122.625 -9.26115703582764\\
122.637496948242 -9.26165294647217\\
122.650001525879 -9.2621488571167\\
122.662498474121 -9.26264476776123\\
122.674995422363 -9.26314067840576\\
122.6875 -9.26363658905029\\
122.699996948242 -9.26413249969482\\
122.712501525879 -9.26462841033936\\
122.724998474121 -9.26512241363525\\
122.737495422363 -9.26561641693115\\
122.75 -9.26610660552979\\
122.762496948242 -9.26659297943115\\
122.775001525879 -9.26707935333252\\
122.787498474121 -9.26756572723389\\
122.799995422363 -9.26805210113525\\
122.8125 -9.2685375213623\\
122.824996948242 -9.26902198791504\\
122.837501525879 -9.26950740814209\\
122.849998474121 -9.26998996734619\\
122.862495422363 -9.27046680450439\\
122.875 -9.2709436416626\\
122.887496948242 -9.2714204788208\\
122.900001525879 -9.271897315979\\
122.912498474121 -9.27237415313721\\
122.924995422363 -9.27285003662109\\
122.9375 -9.27332592010498\\
122.949996948242 -9.27379989624023\\
122.962501525879 -9.27426719665527\\
122.974998474121 -9.27473449707031\\
122.987495422363 -9.27520179748535\\
123 -9.27566909790039\\
123.012496948242 -9.27613544464111\\
123.025001525879 -9.2765998840332\\
123.037498474121 -9.27705764770508\\
123.049995422363 -9.27751541137695\\
123.0625 -9.27797317504883\\
123.074996948242 -9.2784309387207\\
123.087501525879 -9.27888870239258\\
123.099998474121 -9.27934455871582\\
123.112495422363 -9.27980041503906\\
123.125 -9.28025341033936\\
123.137496948242 -9.28070163726807\\
123.150001525879 -9.28114986419678\\
123.162498474121 -9.28159809112549\\
123.174995422363 -9.2820463180542\\
123.1875 -9.28249454498291\\
123.199996948242 -9.2829418182373\\
123.212501525879 -9.2833890914917\\
123.224998474121 -9.28383541107178\\
123.237495422363 -9.28427410125732\\
123.25 -9.28471279144287\\
123.262496948242 -9.28515148162842\\
123.275001525879 -9.28559017181396\\
123.287498474121 -9.28602886199951\\
123.299995422363 -9.28646755218506\\
123.3125 -9.28690624237061\\
123.324996948242 -9.28734493255615\\
123.337501525879 -9.2877836227417\\
123.349998474121 -9.28822231292725\\
123.362495422363 -9.28866100311279\\
123.375 -9.28909969329834\\
123.387496948242 -9.28953838348389\\
123.400001525879 -9.28997707366943\\
123.412498474121 -9.29041576385498\\
123.424995422363 -9.29085445404053\\
123.4375 -9.29129314422607\\
123.449996948242 -9.29173183441162\\
123.462501525879 -9.29217052459717\\
123.474998474121 -9.29260921478271\\
123.487495422363 -9.29304790496826\\
123.5 -9.29348659515381\\
123.512496948242 -9.29392528533936\\
123.525001525879 -9.2943639755249\\
123.537498474121 -9.29480266571045\\
123.549995422363 -9.29524230957031\\
123.5625 -9.29568672180176\\
123.574996948242 -9.2961254119873\\
123.587501525879 -9.29656410217285\\
123.599998474121 -9.2970027923584\\
123.612495422363 -9.29744148254395\\
123.625 -9.29788017272949\\
123.637496948242 -9.29831886291504\\
123.650001525879 -9.29875755310059\\
123.662498474121 -9.29919624328613\\
123.674995422363 -9.29963493347168\\
123.6875 -9.30007362365723\\
123.699996948242 -9.30051231384277\\
123.712501525879 -9.30095100402832\\
123.724998474121 -9.30138969421387\\
123.737495422363 -9.30182838439941\\
123.75 -9.30226707458496\\
123.762496948242 -9.30270576477051\\
123.775001525879 -9.30314350128174\\
123.787498474121 -9.30358028411865\\
123.799995422363 -9.30401706695557\\
123.8125 -9.3044548034668\\
123.824996948242 -9.30488586425781\\
123.837501525879 -9.3053150177002\\
123.849998474121 -9.30574417114258\\
123.862495422363 -9.30617332458496\\
123.875 -9.30660152435303\\
123.887496948242 -9.3070240020752\\
123.900001525879 -9.30744361877441\\
123.912498474121 -9.30786323547363\\
123.924995422363 -9.30828285217285\\
123.9375 -9.30870246887207\\
123.949996948242 -9.30912208557129\\
123.962501525879 -9.30954074859619\\
123.974998474121 -9.30995845794678\\
123.987495422363 -9.3103723526001\\
124 -9.31078243255615\\
124.012496948242 -9.31119251251221\\
124.025001525879 -9.31160259246826\\
124.037498474121 -9.31201171875\\
124.049995422363 -9.31241989135742\\
124.0625 -9.31282043457031\\
124.074996948242 -9.3132209777832\\
124.087501525879 -9.31362152099609\\
124.099998474121 -9.31402206420898\\
124.112495422363 -9.31442260742188\\
124.125 -9.31482315063477\\
124.137496948242 -9.31522369384766\\
124.150001525879 -9.31562423706055\\
124.162498474121 -9.31602668762207\\
124.174995422363 -9.31643486022949\\
124.1875 -9.31684398651123\\
124.199996948242 -9.31724452972412\\
124.212501525879 -9.31764507293701\\
124.224998474121 -9.31804466247559\\
124.237495422363 -9.31843566894531\\
124.25 -9.31882667541504\\
124.262496948242 -9.31921768188477\\
124.275001525879 -9.31960773468018\\
124.287498474121 -9.31999683380127\\
124.299995422363 -9.32037830352783\\
124.3125 -9.32075977325439\\
124.324996948242 -9.32114124298096\\
124.337501525879 -9.32152271270752\\
124.349998474121 -9.32190418243408\\
124.362495422363 -9.32228565216064\\
124.375 -9.32266712188721\\
124.387496948242 -9.32304859161377\\
124.400001525879 -9.32343006134033\\
124.412498474121 -9.32381153106689\\
124.424995422363 -9.32419300079346\\
124.4375 -9.32457447052002\\
124.449996948242 -9.32495594024658\\
124.462501525879 -9.32533740997314\\
124.474998474121 -9.32571887969971\\
124.487495422363 -9.32610034942627\\
124.5 -9.32648277282715\\
124.512496948242 -9.32687187194824\\
124.525001525879 -9.32726192474365\\
124.537498474121 -9.32765102386475\\
124.549995422363 -9.32804203033447\\
124.5625 -9.3284330368042\\
124.574996948242 -9.32882404327393\\
124.587501525879 -9.32921504974365\\
124.599998474121 -9.32960605621338\\
124.612495422363 -9.32999706268311\\
124.625 -9.33039093017578\\
124.637496948242 -9.33078956604004\\
124.650001525879 -9.33118915557861\\
124.662498474121 -9.33158779144287\\
124.674995422363 -9.33198738098145\\
124.6875 -9.33238792419434\\
124.699996948242 -9.33278846740723\\
124.712501525879 -9.33318901062012\\
124.724998474121 -9.33358955383301\\
124.737495422363 -9.3339900970459\\
124.75 -9.33439064025879\\
124.762496948242 -9.33479118347168\\
124.775001525879 -9.33519172668457\\
124.787498474121 -9.33559036254883\\
124.799995422363 -9.3359842300415\\
124.8125 -9.33637523651123\\
124.824996948242 -9.33676624298096\\
124.837501525879 -9.33715724945068\\
124.849998474121 -9.33754825592041\\
124.862495422363 -9.33793830871582\\
124.875 -9.33832836151123\\
124.887496948242 -9.33871269226074\\
124.900001525879 -9.3390941619873\\
124.912498474121 -9.33947563171387\\
124.924995422363 -9.33985614776611\\
124.9375 -9.34023475646973\\
124.949996948242 -9.34060668945313\\
124.962501525879 -9.34097862243652\\
124.974998474121 -9.34135055541992\\
124.987495422363 -9.34172058105469\\
125 -9.3420877456665\\
125.012496948242 -9.34245014190674\\
125.025001525879 -9.34281253814697\\
125.037498474121 -9.34317493438721\\
125.049995422363 -9.34353733062744\\
125.0625 -9.34389877319336\\
125.074996948242 -9.34425354003906\\
125.087501525879 -9.34460639953613\\
125.099998474121 -9.3449592590332\\
125.112495422363 -9.34531211853027\\
125.125 -9.34566497802734\\
125.137496948242 -9.34601783752441\\
125.150001525879 -9.34637069702148\\
125.162498474121 -9.34672355651855\\
125.174995422363 -9.34707546234131\\
125.1875 -9.34742450714111\\
125.199996948242 -9.34776782989502\\
125.212501525879 -9.34811115264893\\
125.224998474121 -9.34845447540283\\
125.237495422363 -9.34879779815674\\
125.25 -9.34914112091064\\
125.262496948242 -9.34948444366455\\
125.275001525879 -9.34982776641846\\
125.287498474121 -9.35017108917236\\
125.299995422363 -9.35051441192627\\
125.3125 -9.35085773468018\\
125.324996948242 -9.35120105743408\\
125.337501525879 -9.35154438018799\\
125.349998474121 -9.35188770294189\\
125.362495422363 -9.3522310256958\\
125.375 -9.35257434844971\\
125.387496948242 -9.35291767120361\\
125.400001525879 -9.35326099395752\\
125.412498474121 -9.35360431671143\\
125.424995422363 -9.35394763946533\\
125.4375 -9.35429096221924\\
125.449996948242 -9.35463428497314\\
125.462501525879 -9.35497760772705\\
125.474998474121 -9.35532093048096\\
125.487495422363 -9.35566425323486\\
125.5 -9.35600757598877\\
125.512496948242 -9.35635089874268\\
125.525001525879 -9.35669422149658\\
125.537498474121 -9.35703754425049\\
125.549995422363 -9.35738849639893\\
125.5625 -9.35774040222168\\
125.574996948242 -9.35809135437012\\
125.587501525879 -9.35844326019287\\
125.599998474121 -9.35879516601563\\
125.612495422363 -9.3591480255127\\
125.625 -9.35950088500977\\
125.637496948242 -9.35985374450684\\
125.650001525879 -9.36020660400391\\
125.662498474121 -9.36055946350098\\
125.674995422363 -9.36091232299805\\
125.6875 -9.36126518249512\\
125.699996948242 -9.36161804199219\\
125.712501525879 -9.36197090148926\\
125.724998474121 -9.36232376098633\\
125.737495422363 -9.3626766204834\\
125.75 -9.36302947998047\\
125.762496948242 -9.36338233947754\\
125.775001525879 -9.36373519897461\\
125.787498474121 -9.36408805847168\\
125.799995422363 -9.36444091796875\\
125.8125 -9.36479377746582\\
125.824996948242 -9.36514568328857\\
125.837501525879 -9.36549854278564\\
125.849998474121 -9.36584949493408\\
125.862495422363 -9.36619853973389\\
125.875 -9.36654186248779\\
125.887496948242 -9.3668851852417\\
125.900001525879 -9.36722850799561\\
125.912498474121 -9.36757183074951\\
125.924995422363 -9.3679141998291\\
125.9375 -9.36825275421143\\
125.949996948242 -9.36858654022217\\
125.962501525879 -9.36892032623291\\
125.974998474121 -9.36925411224365\\
125.987495422363 -9.36958694458008\\
126 -9.36991596221924\\
126.012496948242 -9.37024021148682\\
126.025001525879 -9.37056446075439\\
126.037498474121 -9.37088871002197\\
126.049995422363 -9.37121200561523\\
126.0625 -9.3715353012085\\
126.074996948242 -9.37185287475586\\
126.087501525879 -9.37216758728027\\
126.099998474121 -9.37248229980469\\
126.112495422363 -9.3727970123291\\
126.125 -9.3731107711792\\
126.137496948242 -9.37342166900635\\
126.150001525879 -9.3737268447876\\
126.162498474121 -9.37403202056885\\
126.174995422363 -9.3743371963501\\
126.1875 -9.37464237213135\\
126.199996948242 -9.3749475479126\\
126.212501525879 -9.37525272369385\\
126.224998474121 -9.3755578994751\\
126.237495422363 -9.37586307525635\\
126.25 -9.3761682510376\\
126.262496948242 -9.37647342681885\\
126.275001525879 -9.3767786026001\\
126.287498474121 -9.37708187103271\\
126.299995422363 -9.37738609313965\\
126.3125 -9.3776912689209\\
126.324996948242 -9.37799644470215\\
126.337501525879 -9.3783016204834\\
126.349998474121 -9.37860679626465\\
126.362495422363 -9.3789119720459\\
126.375 -9.37921714782715\\
126.387496948242 -9.3795223236084\\
126.400001525879 -9.37982749938965\\
126.412498474121 -9.3801326751709\\
126.424995422363 -9.38043785095215\\
126.4375 -9.3807430267334\\
126.449996948242 -9.3810510635376\\
126.462501525879 -9.3813648223877\\
126.474998474121 -9.38167953491211\\
126.487495422363 -9.38199424743652\\
126.5 -9.38230895996094\\
126.512496948242 -9.38262367248535\\
126.525001525879 -9.38293838500977\\
126.537498474121 -9.38325309753418\\
126.549995422363 -9.38356781005859\\
126.5625 -9.38388252258301\\
126.574996948242 -9.38420295715332\\
126.587501525879 -9.38452625274658\\
126.599998474121 -9.38484859466553\\
126.612495422363 -9.38517093658447\\
126.625 -9.3854866027832\\
126.637496948242 -9.38580131530762\\
126.650001525879 -9.38611602783203\\
126.662498474121 -9.38643074035645\\
126.674995422363 -9.38674545288086\\
126.6875 -9.38706016540527\\
126.699996948242 -9.38737487792969\\
126.712501525879 -9.3876895904541\\
126.724998474121 -9.38800430297852\\
126.737495422363 -9.38831901550293\\
126.75 -9.38863372802734\\
126.762496948242 -9.38894653320313\\
126.775001525879 -9.38925266265869\\
126.787498474121 -9.38955783843994\\
126.799995422363 -9.38986301422119\\
126.8125 -9.39016819000244\\
126.824996948242 -9.39047336578369\\
126.837501525879 -9.39077854156494\\
126.849998474121 -9.39108276367188\\
126.862495422363 -9.39137935638428\\
126.875 -9.39167499542236\\
126.887496948242 -9.39197063446045\\
126.900001525879 -9.39226627349854\\
126.912498474121 -9.39256191253662\\
126.924995422363 -9.39285469055176\\
126.9375 -9.39314079284668\\
126.949996948242 -9.3934268951416\\
126.962501525879 -9.39371299743652\\
126.974998474121 -9.39399909973145\\
126.987495422363 -9.39428520202637\\
127 -9.39457130432129\\
127.012496948242 -9.39485740661621\\
127.025001525879 -9.39514350891113\\
127.037498474121 -9.39542961120605\\
127.049995422363 -9.39571571350098\\
127.0625 -9.3960018157959\\
127.074996948242 -9.39628791809082\\
127.087501525879 -9.39657402038574\\
127.099998474121 -9.39685916900635\\
127.112495422363 -9.39713573455811\\
127.125 -9.39741230010986\\
127.137496948242 -9.39768886566162\\
127.150001525879 -9.39796543121338\\
127.162498474121 -9.39824199676514\\
127.174995422363 -9.39851665496826\\
127.1875 -9.3987865447998\\
127.199996948242 -9.3990535736084\\
127.212501525879 -9.39932060241699\\
127.224998474121 -9.39958763122559\\
127.237495422363 -9.39985466003418\\
127.25 -9.40012168884277\\
127.262496948242 -9.40038871765137\\
127.275001525879 -9.40065574645996\\
127.287498474121 -9.40092277526855\\
127.299995422363 -9.40118980407715\\
127.3125 -9.40145683288574\\
127.324996948242 -9.40172386169434\\
127.337501525879 -9.40199089050293\\
127.349998474121 -9.40226459503174\\
127.362495422363 -9.40254020690918\\
127.375 -9.40281677246094\\
127.387496948242 -9.4030933380127\\
127.400001525879 -9.40336990356445\\
127.412498474121 -9.40364646911621\\
127.424995422363 -9.40392303466797\\
127.4375 -9.40419960021973\\
127.449996948242 -9.40448188781738\\
127.462501525879 -9.4047679901123\\
127.474998474121 -9.40505409240723\\
127.487495422363 -9.40534019470215\\
127.5 -9.40562629699707\\
127.512496948242 -9.40591239929199\\
127.525001525879 -9.40619850158691\\
127.537498474121 -9.40648651123047\\
127.549995422363 -9.40678024291992\\
127.5625 -9.40707588195801\\
127.574996948242 -9.40737152099609\\
127.587501525879 -9.40766716003418\\
127.599998474121 -9.40796279907227\\
127.612495422363 -9.40825843811035\\
127.625 -9.40855407714844\\
127.637496948242 -9.40884971618652\\
127.650001525879 -9.40914535522461\\
127.662498474121 -9.40944004058838\\
127.674995422363 -9.40973377227783\\
127.6875 -9.41002559661865\\
127.699996948242 -9.41031169891357\\
127.712501525879 -9.4105978012085\\
127.724998474121 -9.41088390350342\\
127.737495422363 -9.41117000579834\\
127.75 -9.41145610809326\\
127.762496948242 -9.41174221038818\\
127.775001525879 -9.41202831268311\\
127.787498474121 -9.41231441497803\\
127.799995422363 -9.41259860992432\\
127.8125 -9.41288375854492\\
127.824996948242 -9.41316318511963\\
127.837501525879 -9.41343975067139\\
127.849998474121 -9.41371631622314\\
127.862495422363 -9.4139928817749\\
127.875 -9.41426849365234\\
127.887496948242 -9.41453552246094\\
127.900001525879 -9.41480255126953\\
127.912498474121 -9.41506958007813\\
127.924995422363 -9.4153356552124\\
127.9375 -9.41560077667236\\
127.949996948242 -9.41585826873779\\
127.962501525879 -9.41611576080322\\
127.974998474121 -9.41637325286865\\
127.987495422363 -9.41663074493408\\
128 -9.4168872833252\\
128.012496948242 -9.41714096069336\\
128.024993896484 -9.41738891601563\\
128.037506103516 -9.41763687133789\\
128.050003051758 -9.41788482666016\\
128.0625 -9.41813278198242\\
128.074996948242 -9.41837978363037\\
128.087493896484 -9.41862678527832\\
128.100006103516 -9.41887474060059\\
128.112503051758 -9.41912269592285\\
128.125 -9.41937065124512\\
128.137496948242 -9.41961860656738\\
128.149993896484 -9.41986560821533\\
128.162506103516 -9.4201135635376\\
128.175003051758 -9.42036056518555\\
128.1875 -9.42060852050781\\
128.199996948242 -9.42085647583008\\
128.212493896484 -9.42110443115234\\
128.225006103516 -9.42135238647461\\
128.237503051758 -9.42160034179688\\
128.25 -9.42184829711914\\
128.262496948242 -9.42209625244141\\
128.274993896484 -9.42234420776367\\
128.287506103516 -9.42259216308594\\
128.300003051758 -9.42284679412842\\
128.3125 -9.42310333251953\\
128.324996948242 -9.42336082458496\\
128.337493896484 -9.42361831665039\\
128.350006103516 -9.42387962341309\\
128.362503051758 -9.42414569854736\\
128.375 -9.42441272735596\\
128.387496948242 -9.42468643188477\\
128.399993896484 -9.42496299743652\\
128.412506103516 -9.42524814605713\\
128.425003051758 -9.42553424835205\\
128.4375 -9.42581081390381\\
128.449996948242 -9.42608737945557\\
128.462493896484 -9.42636394500732\\
128.475006103516 -9.42663860321045\\
128.487503051758 -9.42690563201904\\
128.5 -9.42717266082764\\
128.512496948242 -9.42743587493896\\
128.524993896484 -9.42769336700439\\
128.537506103516 -9.42795085906982\\
128.550003051758 -9.42820739746094\\
128.5625 -9.42846488952637\\
128.574996948242 -9.4287223815918\\
128.587493896484 -9.42897987365723\\
128.600006103516 -9.42923736572266\\
128.612503051758 -9.42949485778809\\
128.625 -9.42975234985352\\
128.637496948242 -9.43000888824463\\
128.649993896484 -9.43026542663574\\
128.662506103516 -9.43052196502686\\
128.675003051758 -9.43076992034912\\
128.6875 -9.43101787567139\\
128.699996948242 -9.43126583099365\\
128.712493896484 -9.43151378631592\\
128.725006103516 -9.43176174163818\\
128.737503051758 -9.43200969696045\\
128.75 -9.43225765228271\\
128.762496948242 -9.43250560760498\\
128.774993896484 -9.43275356292725\\
128.787506103516 -9.43300151824951\\
128.800003051758 -9.43324947357178\\
128.8125 -9.43349742889404\\
128.824996948242 -9.43374538421631\\
128.837493896484 -9.43399333953857\\
128.850006103516 -9.43424129486084\\
128.862503051758 -9.43448925018311\\
128.875 -9.43473720550537\\
128.887496948242 -9.43498420715332\\
128.899993896484 -9.43523120880127\\
128.912506103516 -9.43547916412354\\
128.925003051758 -9.43572616577148\\
128.9375 -9.43597316741943\\
128.949996948242 -9.43622016906738\\
128.962493896484 -9.43646717071533\\
128.975006103516 -9.4367151260376\\
128.987503051758 -9.43696212768555\\
129 -9.4372091293335\\
129.012496948242 -9.43745613098145\\
129.024993896484 -9.43770313262939\\
129.037506103516 -9.43795108795166\\
129.050003051758 -9.43819808959961\\
129.0625 -9.43844509124756\\
129.074996948242 -9.43869209289551\\
129.087493896484 -9.43893909454346\\
129.100006103516 -9.43918704986572\\
129.112503051758 -9.43943405151367\\
129.125 -9.43968105316162\\
129.137496948242 -9.43992805480957\\
129.149993896484 -9.44017505645752\\
129.162506103516 -9.44042301177979\\
129.175003051758 -9.44067096710205\\
129.1875 -9.44091892242432\\
129.199996948242 -9.44116687774658\\
129.212493896484 -9.44141483306885\\
129.225006103516 -9.44166278839111\\
129.237503051758 -9.44191074371338\\
129.25 -9.44215869903564\\
129.262496948242 -9.44240665435791\\
129.274993896484 -9.44265460968018\\
129.287506103516 -9.44290256500244\\
129.300003051758 -9.44315052032471\\
129.3125 -9.44339847564697\\
129.324996948242 -9.44365215301514\\
129.337493896484 -9.44390869140625\\
129.350006103516 -9.44416618347168\\
129.362503051758 -9.44442272186279\\
129.375 -9.44467926025391\\
129.387496948242 -9.44493579864502\\
129.399993896484 -9.44519233703613\\
129.412506103516 -9.44544982910156\\
129.425003051758 -9.44570636749268\\
129.4375 -9.44596290588379\\
129.449996948242 -9.4462194442749\\
129.462493896484 -9.44647598266602\\
129.475006103516 -9.44673347473145\\
129.487503051758 -9.44699001312256\\
129.5 -9.44724655151367\\
129.512496948242 -9.44749736785889\\
129.524993896484 -9.44774532318115\\
129.537506103516 -9.44799327850342\\
129.550003051758 -9.44824123382568\\
129.5625 -9.44848918914795\\
129.574996948242 -9.44873714447021\\
129.587493896484 -9.44898509979248\\
129.600006103516 -9.44923305511475\\
129.612503051758 -9.44948101043701\\
129.625 -9.44972896575928\\
129.637496948242 -9.44997692108154\\
129.649993896484 -9.45022392272949\\
129.662506103516 -9.45047187805176\\
129.675003051758 -9.45071887969971\\
129.6875 -9.45096397399902\\
129.699996948242 -9.45120239257813\\
129.712493896484 -9.45144081115723\\
129.725006103516 -9.45167922973633\\
129.737503051758 -9.45191764831543\\
129.75 -9.45215606689453\\
129.762496948242 -9.45239448547363\\
129.774993896484 -9.45263290405273\\
129.787506103516 -9.45287132263184\\
129.800003051758 -9.45310974121094\\
129.8125 -9.45334815979004\\
129.824996948242 -9.45358657836914\\
129.837493896484 -9.45382499694824\\
129.850006103516 -9.45406341552734\\
129.862503051758 -9.45430183410645\\
129.875 -9.45454025268555\\
129.887496948242 -9.45477867126465\\
129.899993896484 -9.45501708984375\\
129.912506103516 -9.45525550842285\\
129.925003051758 -9.45549392700195\\
129.9375 -9.45573234558105\\
129.949996948242 -9.45597076416016\\
129.962493896484 -9.45620918273926\\
129.975006103516 -9.45644760131836\\
129.987503051758 -9.45668601989746\\
130 -9.45692825317383\\
130.012496948242 -9.45717525482178\\
130.024993896484 -9.45742225646973\\
130.037506103516 -9.45767021179199\\
130.050003051758 -9.45791816711426\\
130.0625 -9.45816612243652\\
130.074996948242 -9.45841407775879\\
130.087493896484 -9.45866203308105\\
130.100006103516 -9.45890998840332\\
130.112503051758 -9.45915794372559\\
130.125 -9.45940589904785\\
130.137496948242 -9.45965385437012\\
130.149993896484 -9.45990180969238\\
130.162506103516 -9.46014976501465\\
130.175003051758 -9.46039772033691\\
130.1875 -9.46064567565918\\
130.199996948242 -9.46089363098145\\
130.212493896484 -9.46114158630371\\
130.225006103516 -9.46138954162598\\
130.237503051758 -9.46163749694824\\
130.25 -9.46188545227051\\
130.262496948242 -9.46213340759277\\
130.274993896484 -9.46238136291504\\
130.287506103516 -9.4626293182373\\
130.300003051758 -9.46287727355957\\
130.3125 -9.46312522888184\\
130.324996948242 -9.4633731842041\\
130.337493896484 -9.46362113952637\\
130.350006103516 -9.46386909484863\\
130.362503051758 -9.4641170501709\\
130.375 -9.46436500549316\\
130.387496948242 -9.46461200714111\\
130.399993896484 -9.46485900878906\\
130.412506103516 -9.46510696411133\\
130.425003051758 -9.46535396575928\\
130.4375 -9.46560096740723\\
130.449996948242 -9.46584033966064\\
130.462493896484 -9.46607875823975\\
130.475006103516 -9.46631717681885\\
130.487503051758 -9.46655559539795\\
130.5 -9.46679401397705\\
130.512496948242 -9.46703243255615\\
130.524993896484 -9.46727085113525\\
130.537506103516 -9.46750926971436\\
130.550003051758 -9.46774768829346\\
130.5625 -9.46798610687256\\
130.574996948242 -9.46822452545166\\
130.587493896484 -9.46846294403076\\
130.600006103516 -9.46870136260986\\
130.612503051758 -9.46893978118896\\
130.625 -9.46917724609375\\
130.637496948242 -9.46941471099854\\
130.649993896484 -9.46965312957764\\
130.662506103516 -9.46989154815674\\
130.675003051758 -9.47012996673584\\
130.6875 -9.47036838531494\\
130.699996948242 -9.47060680389404\\
130.712493896484 -9.47084522247314\\
130.725006103516 -9.47108364105225\\
130.737503051758 -9.47132205963135\\
130.75 -9.47156047821045\\
130.762496948242 -9.47179889678955\\
130.774993896484 -9.47203731536865\\
130.787506103516 -9.47228050231934\\
130.800003051758 -9.47252750396729\\
130.8125 -9.47277450561523\\
130.824996948242 -9.4730224609375\\
130.837493896484 -9.47327041625977\\
130.850006103516 -9.47351837158203\\
130.862503051758 -9.4737663269043\\
130.875 -9.47401428222656\\
130.887496948242 -9.47426223754883\\
130.899993896484 -9.47451019287109\\
130.912506103516 -9.47476673126221\\
130.925003051758 -9.47502422332764\\
130.9375 -9.47528743743896\\
130.949996948242 -9.47555446624756\\
130.962493896484 -9.475830078125\\
130.975006103516 -9.47610664367676\\
130.987503051758 -9.47638320922852\\
131 -9.4766674041748\\
131.012496948242 -9.47695350646973\\
131.024993896484 -9.47723960876465\\
131.037506103516 -9.47753143310547\\
131.050003051758 -9.47782611846924\\
131.0625 -9.47812175750732\\
131.074996948242 -9.47841739654541\\
131.087493896484 -9.47870922088623\\
131.100006103516 -9.47898769378662\\
131.112503051758 -9.47926425933838\\
131.125 -9.4795389175415\\
131.137496948242 -9.47980880737305\\
131.149993896484 -9.48007392883301\\
131.162506103516 -9.48034954071045\\
131.175003051758 -9.48062610626221\\
131.1875 -9.48090267181396\\
131.199996948242 -9.48118591308594\\
131.212493896484 -9.48147201538086\\
131.225006103516 -9.48176097869873\\
131.237503051758 -9.48205852508545\\
131.25 -9.48237037658691\\
131.262496948242 -9.48270702362061\\
131.274993896484 -9.48307991027832\\
131.287506103516 -9.48353862762451\\
131.300003051758 -9.48427581787109\\
131.3125 -9.48574447631836\\
131.324996948242 -9.48917579650879\\
131.337493896484 -9.49731540679932\\
131.350006103516 -9.52929496765137\\
131.362503051758 -9.73299980163574\\
131.375 -9.94408416748047\\
131.387496948242 -10.1473350524902\\
131.399993896484 -10.3405857086182\\
131.412506103516 -10.5232954025269\\
131.425003051758 -10.6943368911743\\
131.4375 -10.853401184082\\
131.449996948242 -11.0001640319824\\
131.462493896484 -11.1341924667358\\
131.475006103516 -11.2550792694092\\
131.487503051758 -11.3657703399658\\
131.5 -11.4738626480103\\
131.512496948242 -11.5816030502319\\
131.524993896484 -11.6884250640869\\
131.537506103516 -11.7917747497559\\
131.550003051758 -11.8891544342041\\
131.5625 -11.9791202545166\\
131.574996948242 -12.0597705841064\\
131.587493896484 -12.1295118331909\\
131.600006103516 -12.1852750778198\\
131.612503051758 -12.2270555496216\\
131.625 -12.2628650665283\\
131.637496948242 -12.3403387069702\\
131.649993896484 -12.4933271408081\\
131.662506103516 -12.6530742645264\\
131.675003051758 -12.8003168106079\\
131.687484741211 -12.9424924850464\\
131.699981689453 -13.0766534805298\\
131.712478637695 -13.2030563354492\\
131.724975585938 -13.3033418655396\\
131.736328125 -13.3253211975098\\
131.748825073242 -13.32985496521\\
131.761322021484 -13.3328771591187\\
131.773834228516 -13.3353672027588\\
131.786331176758 -13.3376636505127\\
131.798828125 -13.3398466110229\\
131.811325073242 -13.3419876098633\\
131.823822021484 -13.3441247940063\\
131.836334228516 -13.3462190628052\\
131.848831176758 -13.3482599258423\\
131.861328125 -13.3502569198608\\
131.873825073242 -13.3522386550903\\
131.886322021484 -13.3542928695679\\
131.898834228516 -13.356463432312\\
131.911331176758 -13.3586931228638\\
131.923828125 -13.3662252426147\\
131.936325073242 -13.4101858139038\\
131.948822021484 -13.7156505584717\\
131.961334228516 -13.958475112915\\
131.973831176758 -14.0450782775879\\
131.986328125 -14.0717926025391\\
131.998825073242 -14.0964250564575\\
132.011322021484 -14.1503629684448\\
132.023834228516 -14.3407983779907\\
132.036331176758 -14.5521993637085\\
132.048828125 -14.7485418319702\\
132.061325073242 -14.9100866317749\\
132.073822021484 -15.0253648757935\\
132.086334228516 -15.0971879959106\\
132.098831176758 -15.132116317749\\
132.111328125 -15.1445598602295\\
132.123825073242 -15.1477947235107\\
132.136322021484 -15.1535043716431\\
132.148834228516 -15.1754837036133\\
132.161331176758 -15.2064981460571\\
132.173828125 -15.2849740982056\\
132.186325073242 -15.6356210708618\\
132.198822021484 -15.9864683151245\\
132.211334228516 -16.2633647918701\\
132.223831176758 -16.4494400024414\\
132.236328125 -16.5467586517334\\
132.248825073242 -16.5771903991699\\
132.261322021484 -16.5804672241211\\
132.273834228516 -16.5812225341797\\
132.286331176758 -16.5816631317139\\
132.298828125 -16.5823593139648\\
132.311325073242 -16.5842838287354\\
132.323822021484 -16.603816986084\\
132.336334228516 -16.6394004821777\\
132.348831176758 -16.8575649261475\\
132.361328125 -17.2376499176025\\
132.373825073242 -17.5165519714355\\
132.386322021484 -17.6665573120117\\
132.398834228516 -17.7033710479736\\
132.411331176758 -17.7051811218262\\
132.423828125 -17.7059078216553\\
132.436325073242 -17.7063655853271\\
132.448822021484 -17.7067413330078\\
132.461334228516 -17.707103729248\\
132.473831176758 -17.7074890136719\\
132.486328125 -17.7082138061523\\
132.498825073242 -17.7136516571045\\
132.511322021484 -17.7415828704834\\
132.523834228516 -17.8835945129395\\
132.536331176758 -18.3207244873047\\
132.548828125 -18.6715850830078\\
132.561325073242 -18.9095153808594\\
132.573822021484 -19.0465183258057\\
132.586334228516 -19.0778331756592\\
132.598831176758 -19.079553604126\\
132.611328125 -19.0803890228271\\
132.623825073242 -19.0810050964355\\
132.636322021484 -19.0815372467041\\
132.648834228516 -19.0820178985596\\
132.661331176758 -19.0825119018555\\
132.673828125 -19.0832214355469\\
132.684478759766 -19.0840015411377\\
132.696990966797 -19.0855865478516\\
132.709487915039 -19.0907497406006\\
132.721984863281 -19.1115112304688\\
132.733261108398 -19.161169052124\\
132.737869262695 -19.226390838623\\
132.738525390625 -19.2522716522217\\
132.739181518555 -19.2779026031494\\
132.739868164063 -19.3046722412109\\
132.74055480957 -19.3306579589844\\
132.741271972656 -19.3572101593018\\
132.742004394531 -19.3834495544434\\
132.742767333984 -19.4099273681641\\
132.743560791016 -19.4365234375\\
132.744384765625 -19.4631652832031\\
132.745239257813 -19.489595413208\\
132.746154785156 -19.5169830322266\\
132.747100830078 -19.5439224243164\\
132.748092651367 -19.5709953308105\\
132.749145507813 -19.5981788635254\\
132.750274658203 -19.6258068084717\\
132.751480102539 -19.6531658172607\\
132.752792358398 -19.6808681488037\\
132.754211425781 -19.7079734802246\\
132.755844116211 -19.7358264923096\\
132.757751464844 -19.7630672454834\\
132.760284423828 -19.7906074523926\\
132.765197753906 -19.818754196167\\
132.777450561523 -19.8294544219971\\
132.789947509766 -19.8307571411133\\
132.802444458008 -19.8319892883301\\
132.81494140625 -19.8332252502441\\
132.827438354492 -19.834342956543\\
132.839950561523 -19.8354587554932\\
132.852447509766 -19.8364944458008\\
132.864944458008 -19.8373603820801\\
132.87712097168 -19.8379821777344\\
132.884429931641 -19.8382186889648\\
132.896926879883 -19.8383712768555\\
132.909423828125 -19.8382034301758\\
132.921936035156 -19.8378257751465\\
132.934432983398 -19.8374118804932\\
132.946929931641 -19.8370990753174\\
132.959426879883 -19.8369579315186\\
132.971923828125 -19.8369483947754\\
132.984436035156 -19.8370189666748\\
132.996932983398 -19.8371429443359\\
133.009429931641 -19.8373222351074\\
133.021926879883 -19.8375816345215\\
133.034423828125 -19.8379325866699\\
133.046936035156 -19.8383407592773\\
133.059432983398 -19.8387908935547\\
133.071929931641 -19.8392543792725\\
133.084426879883 -19.8397159576416\\
133.096923828125 -19.8401641845703\\
133.109436035156 -19.8405895233154\\
133.121932983398 -19.8409824371338\\
133.134429931641 -19.841344833374\\
133.146926879883 -19.841667175293\\
133.159423828125 -19.841947555542\\
133.171936035156 -19.842191696167\\
133.184432983398 -19.8424091339111\\
133.196929931641 -19.8426055908203\\
133.209426879883 -19.8427963256836\\
133.221923828125 -19.8429870605469\\
133.234436035156 -19.8431777954102\\
133.246932983398 -19.8433685302734\\
133.259429931641 -19.8435459136963\\
133.271926879883 -19.843713760376\\
133.284423828125 -19.8438663482666\\
133.296936035156 -19.8440055847168\\
133.309432983398 -19.8441333770752\\
133.321929931641 -19.8442478179932\\
133.334426879883 -19.8443565368652\\
133.346923828125 -19.8444519042969\\
133.359436035156 -19.8445472717285\\
133.371932983398 -19.8446426391602\\
133.384429931641 -19.8447494506836\\
133.396926879883 -19.8448638916016\\
133.409423828125 -19.8449878692627\\
133.421936035156 -19.845121383667\\
133.434432983398 -19.8452548980713\\
133.446929931641 -19.8454074859619\\
133.459426879883 -19.8455600738525\\
133.471923828125 -19.8457126617432\\
133.484436035156 -19.8458786010742\\
133.496932983398 -19.8460502624512\\
133.509429931641 -19.8462219238281\\
133.521926879883 -19.8463935852051\\
133.534423828125 -19.846565246582\\
133.546936035156 -19.846736907959\\
133.559432983398 -19.8469085693359\\
133.571929931641 -19.8470649719238\\
133.584426879883 -19.8472175598145\\
133.596923828125 -19.8473682403564\\
133.609436035156 -19.8475017547607\\
133.621932983398 -19.847635269165\\
133.634429931641 -19.8477668762207\\
133.646926879883 -19.8478813171387\\
133.659423828125 -19.8479957580566\\
133.671936035156 -19.8481101989746\\
133.684432983398 -19.8482246398926\\
133.696929931641 -19.8483238220215\\
133.709426879883 -19.8484191894531\\
133.721923828125 -19.8485145568848\\
133.734436035156 -19.8486289978027\\
133.746932983398 -19.8487434387207\\
133.759429931641 -19.8488578796387\\
133.771926879883 -19.8489875793457\\
133.784423828125 -19.84912109375\\
133.796936035156 -19.8492546081543\\
133.809432983398 -19.8493938446045\\
133.821929931641 -19.8495464324951\\
133.834426879883 -19.8496990203857\\
133.846923828125 -19.8498516082764\\
133.859436035156 -19.850004196167\\
133.871932983398 -19.8501567840576\\
133.884429931641 -19.8503093719482\\
133.896926879883 -19.8504619598389\\
133.909423828125 -19.8506145477295\\
133.921936035156 -19.8507671356201\\
133.934432983398 -19.8509197235107\\
133.946929931641 -19.8510665893555\\
133.959426879883 -19.8512001037598\\
133.971923828125 -19.8513355255127\\
133.984436035156 -19.8514881134033\\
133.996932983398 -19.8516407012939\\
134.009429931641 -19.8517780303955\\
134.021926879883 -19.8519115447998\\
134.034423828125 -19.8520317077637\\
134.046936035156 -19.8521461486816\\
134.059432983398 -19.8522605895996\\
134.071929931641 -19.8523750305176\\
134.084426879883 -19.8524894714355\\
134.096923828125 -19.8526000976563\\
134.109436035156 -19.8526954650879\\
134.121932983398 -19.8527908325195\\
134.134429931641 -19.8528861999512\\
134.146926879883 -19.8529815673828\\
134.159423828125 -19.8530750274658\\
134.171936035156 -19.8531723022461\\
134.184432983398 -19.8532676696777\\
134.196929931641 -19.8533630371094\\
134.209426879883 -19.853458404541\\
134.221923828125 -19.8535537719727\\
134.234436035156 -19.8536491394043\\
134.246932983398 -19.8537483215332\\
134.259429931641 -19.8538436889648\\
134.271926879883 -19.8539180755615\\
134.284423828125 -19.8540115356445\\
134.296936035156 -19.8541069030762\\
134.309432983398 -19.8541851043701\\
134.321929931641 -19.8542881011963\\
134.334426879883 -19.8543910980225\\
134.346923828125 -19.8545036315918\\
134.359436035156 -19.8545989990234\\
134.371932983398 -19.8547134399414\\
134.384429931641 -19.8548278808594\\
134.396926879883 -19.8549270629883\\
134.409423828125 -19.8550415039063\\
134.421936035156 -19.8551559448242\\
134.434432983398 -19.8552703857422\\
134.446929931641 -19.8553848266602\\
134.459426879883 -19.8555164337158\\
134.471923828125 -19.8556499481201\\
134.484436035156 -19.8557834625244\\
134.496932983398 -19.8559169769287\\
134.509429931641 -19.856050491333\\
134.521926879883 -19.8561840057373\\
134.534423828125 -19.8563175201416\\
134.546936035156 -19.8564510345459\\
134.559432983398 -19.8565845489502\\
134.571929931641 -19.8567180633545\\
134.584426879883 -19.8568515777588\\
134.596923828125 -19.8569850921631\\
134.609436035156 -19.8571300506592\\
134.621932983398 -19.8572826385498\\
134.634429931641 -19.8574352264404\\
134.646926879883 -19.8575878143311\\
134.659423828125 -19.8577404022217\\
134.671936035156 -19.8578929901123\\
134.684432983398 -19.8580474853516\\
134.696929931641 -19.8582172393799\\
134.709426879883 -19.8583889007568\\
134.721923828125 -19.8585605621338\\
134.734436035156 -19.8587322235107\\
134.746932983398 -19.8589038848877\\
134.759429931641 -19.8590755462646\\
134.771926879883 -19.8592548370361\\
134.784423828125 -19.8594455718994\\
134.796936035156 -19.8596363067627\\
134.809432983398 -19.8598251342773\\
134.821929931641 -19.8600044250488\\
134.834426879883 -19.8601951599121\\
134.846923828125 -19.8603610992432\\
134.859436035156 -19.8605785369873\\
134.871932983398 -19.8607940673828\\
134.884429931641 -19.8609848022461\\
134.896926879883 -19.8611755371094\\
134.909423828125 -19.8613719940186\\
134.921936035156 -19.8615646362305\\
134.934432983398 -19.8617553710938\\
134.946929931641 -19.8619651794434\\
134.959426879883 -19.8621711730957\\
134.971923828125 -19.8623905181885\\
134.984436035156 -19.8626003265381\\
134.996932983398 -19.8628101348877\\
135.009429931641 -19.8630199432373\\
135.021926879883 -19.8632354736328\\
135.034423828125 -19.8634452819824\\
135.046936035156 -19.863655090332\\
135.059432983398 -19.8638687133789\\
135.071929931641 -19.8640899658203\\
135.084426879883 -19.8642997741699\\
135.096923828125 -19.8645095825195\\
135.109436035156 -19.8647193908691\\
135.121932983398 -19.8649291992188\\
135.134429931641 -19.8651390075684\\
135.146926879883 -19.865348815918\\
135.159423828125 -19.8655586242676\\
135.171936035156 -19.8657684326172\\
135.184432983398 -19.8659782409668\\
135.196929931641 -19.8661880493164\\
135.209426879883 -19.866397857666\\
135.221923828125 -19.8666229248047\\
135.234436035156 -19.8668518066406\\
135.246932983398 -19.8670806884766\\
135.259429931641 -19.8673229217529\\
135.271926879883 -19.8675689697266\\
135.284423828125 -19.867805480957\\
135.296936035156 -19.868034362793\\
135.309432983398 -19.8682632446289\\
135.321929931641 -19.8684921264648\\
135.334426879883 -19.8687210083008\\
135.346923828125 -19.8689498901367\\
135.359436035156 -19.8691787719727\\
135.371932983398 -19.8693885803223\\
135.384429931641 -19.8695983886719\\
135.396926879883 -19.8698081970215\\
135.409423828125 -19.8700180053711\\
135.421936035156 -19.8702278137207\\
135.434432983398 -19.870418548584\\
135.446929931641 -19.8706092834473\\
135.459426879883 -19.8708000183105\\
135.471923828125 -19.8709907531738\\
135.484436035156 -19.8711814880371\\
135.496932983398 -19.8713722229004\\
135.509429931641 -19.8715629577637\\
135.521926879883 -19.8717422485352\\
135.534423828125 -19.8719139099121\\
135.546936035156 -19.8720855712891\\
135.559432983398 -19.872257232666\\
135.571929931641 -19.872428894043\\
135.584426879883 -19.8726005554199\\
135.596923828125 -19.8727645874023\\
135.609436035156 -19.872917175293\\
135.621932983398 -19.8730697631836\\
135.634429931641 -19.8732147216797\\
135.646926879883 -19.8733863830566\\
135.659423828125 -19.8735294342041\\
135.671936035156 -19.8736820220947\\
135.684432983398 -19.873836517334\\
135.696929931641 -19.8739891052246\\
135.709426879883 -19.8741493225098\\
135.721923828125 -19.8743190765381\\
135.734436035156 -19.8744831085205\\
135.746932983398 -19.8746547698975\\
135.759429931641 -19.8748264312744\\
135.771926879883 -19.8750171661377\\
135.784423828125 -19.875207901001\\
135.796936035156 -19.8753986358643\\
135.809432983398 -19.8755893707275\\
135.821929931641 -19.8757801055908\\
135.834426879883 -19.8759708404541\\
135.846923828125 -19.8761692047119\\
135.859436035156 -19.8763656616211\\
135.871932983398 -19.8765735626221\\
135.884429931641 -19.8767833709717\\
135.896926879883 -19.8769931793213\\
135.909423828125 -19.8772029876709\\
135.921936035156 -19.8774127960205\\
135.934432983398 -19.8776245117188\\
135.946929931641 -19.8778533935547\\
135.959426879883 -19.8780822753906\\
135.971923828125 -19.8783111572266\\
135.984436035156 -19.8785400390625\\
135.996932983398 -19.8787689208984\\
136.009429931641 -19.8790035247803\\
136.021926879883 -19.8792514801025\\
136.034423828125 -19.8794994354248\\
136.046936035156 -19.8797473907471\\
136.059432983398 -19.8799953460693\\
136.071929931641 -19.8802433013916\\
136.084426879883 -19.8804912567139\\
136.096923828125 -19.8807392120361\\
136.109436035156 -19.8809871673584\\
136.121932983398 -19.8812522888184\\
136.134429931641 -19.881519317627\\
136.146926879883 -19.8817768096924\\
136.159423828125 -19.8820152282715\\
136.171936035156 -19.8822441101074\\
136.184432983398 -19.882474899292\\
136.196929931641 -19.8827152252197\\
136.209426879883 -19.8829765319824\\
136.221923828125 -19.8832244873047\\
136.234436035156 -19.8834533691406\\
136.246932983398 -19.8836822509766\\
136.259429931641 -19.8839111328125\\
136.271926879883 -19.8841457366943\\
136.284423828125 -19.8843936920166\\
136.296936035156 -19.8846416473389\\
136.309432983398 -19.8848896026611\\
136.321929931641 -19.8851375579834\\
136.334426879883 -19.8853969573975\\
136.346923828125 -19.8856639862061\\
136.359436035156 -19.8859310150146\\
136.371932983398 -19.886209487915\\
136.384429931641 -19.88649559021\\
136.396926879883 -19.8867816925049\\
136.409423828125 -19.8870677947998\\
136.421936035156 -19.8873538970947\\
136.434432983398 -19.8876552581787\\
136.446929931641 -19.88796043396\\
136.459426879883 -19.8882656097412\\
136.471923828125 -19.8885707855225\\
136.484436035156 -19.8888759613037\\
136.496932983398 -19.889181137085\\
136.509429931641 -19.8894863128662\\
136.521926879883 -19.8897914886475\\
136.534423828125 -19.8900966644287\\
136.546936035156 -19.8904151916504\\
136.559432983398 -19.8907375335693\\
136.571929931641 -19.8910655975342\\
136.584426879883 -19.8913516998291\\
136.596923828125 -19.8916187286377\\
136.609436035156 -19.8918933868408\\
136.621932983398 -19.8921909332275\\
136.634429931641 -19.8924980163574\\
136.646926879883 -19.892822265625\\
136.659423828125 -19.8931465148926\\
136.671936035156 -19.8934841156006\\
136.684432983398 -19.8938255310059\\
136.696929931641 -19.8941516876221\\
136.709426879883 -19.894495010376\\
136.721923828125 -19.8948669433594\\
136.734436035156 -19.8952484130859\\
136.746932983398 -19.8956298828125\\
136.759429931641 -19.8960113525391\\
136.771926879883 -19.896390914917\\
136.784423828125 -19.8967533111572\\
136.796936035156 -19.8971157073975\\
136.809432983398 -19.8974761962891\\
136.821929931641 -19.8978233337402\\
136.834426879883 -19.8981666564941\\
136.846923828125 -19.898509979248\\
136.859436035156 -19.898868560791\\
136.871932983398 -19.8992290496826\\
136.884429931641 -19.8996105194092\\
136.896926879883 -19.9000110626221\\
136.909423828125 -19.900411605835\\
136.921936035156 -19.9007930755615\\
136.934432983398 -19.9011325836182\\
136.946929931641 -19.9014472961426\\
136.959426879883 -19.9017467498779\\
136.971923828125 -19.9019966125488\\
136.984436035156 -19.9022216796875\\
136.996932983398 -19.9024314880371\\
137.009429931641 -19.9026412963867\\
137.021926879883 -19.9028511047363\\
137.034423828125 -19.9030609130859\\
137.046936035156 -19.9032859802246\\
137.059432983398 -19.9035148620605\\
137.071929931641 -19.9037437438965\\
137.084426879883 -19.9039726257324\\
137.096923828125 -19.9042110443115\\
137.109436035156 -19.9044589996338\\
137.121932983398 -19.9047069549561\\
137.134429931641 -19.9049549102783\\
137.146926879883 -19.9052028656006\\
137.159423828125 -19.9054584503174\\
137.171936035156 -19.9057140350342\\
137.184432983398 -19.9059619903564\\
137.196929931641 -19.9062099456787\\
137.209426879883 -19.9064502716064\\
137.221923828125 -19.9066696166992\\
137.234436035156 -19.9068832397461\\
137.246932983398 -19.9070796966553\\
137.259429931641 -19.9072704315186\\
137.271926879883 -19.9074764251709\\
137.284423828125 -19.9076728820801\\
137.296936035156 -19.9078636169434\\
137.309432983398 -19.9080543518066\\
137.321929931641 -19.9082450866699\\
137.334426879883 -19.9084358215332\\
137.346923828125 -19.908618927002\\
137.359436035156 -19.9087905883789\\
137.371932983398 -19.9089469909668\\
137.384429931641 -19.909086227417\\
137.396926879883 -19.9092197418213\\
137.409423828125 -19.9093418121338\\
137.421936035156 -19.9094562530518\\
137.434432983398 -19.9095706939697\\
137.446929931641 -19.9096851348877\\
137.459426879883 -19.9097995758057\\
137.471923828125 -19.9099140167236\\
137.484436035156 -19.9100284576416\\
137.496932983398 -19.9101428985596\\
137.509429931641 -19.9102668762207\\
137.521926879883 -19.910400390625\\
137.534423828125 -19.9105339050293\\
137.546936035156 -19.9106845855713\\
137.559432983398 -19.9108371734619\\
137.571929931641 -19.9110012054443\\
137.584426879883 -19.9111728668213\\
137.596923828125 -19.9113445281982\\
137.609436035156 -19.9115161895752\\
137.621932983398 -19.9116878509521\\
137.634429931641 -19.9118595123291\\
137.646926879883 -19.9120311737061\\
137.659423828125 -19.912202835083\\
137.671936035156 -19.91237449646\\
137.684432983398 -19.9125461578369\\
137.696929931641 -19.9127254486084\\
137.709426879883 -19.9129161834717\\
137.721923828125 -19.913106918335\\
137.734436035156 -19.9132862091064\\
137.746932983398 -19.9134883880615\\
137.759429931641 -19.9136791229248\\
137.771926879883 -19.9138698577881\\
137.784423828125 -19.9140605926514\\
137.796936035156 -19.9142627716064\\
137.809432983398 -19.9144630432129\\
137.821929931641 -19.9146537780762\\
137.834426879883 -19.9148445129395\\
137.846923828125 -19.9150352478027\\
137.859436035156 -19.9152336120605\\
137.871932983398 -19.9154262542725\\
137.884429931641 -19.9156169891357\\
137.896926879883 -19.915807723999\\
137.909423828125 -19.9159984588623\\
137.921936035156 -19.9161891937256\\
137.934432983398 -19.9163799285889\\
137.946929931641 -19.9165592193604\\
137.959426879883 -19.9167308807373\\
137.971923828125 -19.9169025421143\\
137.984436035156 -19.9170742034912\\
137.996932983398 -19.9172630310059\\
138.009429931641 -19.9174652099609\\
138.021926879883 -19.9176826477051\\
138.034423828125 -19.9179153442383\\
138.046936035156 -19.9181652069092\\
138.059432983398 -19.9184322357178\\
138.071929931641 -19.9187183380127\\
138.084426879883 -19.9190196990967\\
138.096923828125 -19.9193382263184\\
138.109436035156 -19.9196739196777\\
138.121932983398 -19.9200267791748\\
138.134429931641 -19.9203948974609\\
138.146926879883 -19.9207763671875\\
138.159423828125 -19.9211750030518\\
138.171936035156 -19.9215755462646\\
138.184432983398 -19.9219760894775\\
138.196929931641 -19.9223937988281\\
138.209426879883 -19.9228191375732\\
138.221923828125 -19.9232597351074\\
138.234436035156 -19.9237174987793\\
138.246932983398 -19.9241924285889\\
138.259429931641 -19.924674987793\\
138.271926879883 -19.9251708984375\\
138.284423828125 -19.9256706237793\\
138.296936035156 -19.9261856079102\\
138.309432983398 -19.9267158508301\\
138.321929931641 -19.9272575378418\\
138.334426879883 -19.9278144836426\\
138.346923828125 -19.9283866882324\\
138.359436035156 -19.9289684295654\\
138.371932983398 -19.9295558929443\\
138.384429931641 -19.9301280975342\\
138.396926879883 -19.930700302124\\
138.409423828125 -19.9312725067139\\
138.421936035156 -19.9318447113037\\
138.434432983398 -19.9324283599854\\
138.446929931641 -19.9330177307129\\
138.459426879883 -19.9336090087891\\
138.471923828125 -19.934211730957\\
138.484436035156 -19.9348220825195\\
138.496932983398 -19.9354438781738\\
138.509429931641 -19.9360733032227\\
138.521926879883 -19.9367122650146\\
138.534423828125 -19.9373607635498\\
138.546936035156 -19.9380130767822\\
138.559432983398 -19.9386787414551\\
138.571929931641 -19.9393577575684\\
138.584426879883 -19.9400539398193\\
138.596923828125 -19.9407653808594\\
138.609436035156 -19.9414901733398\\
138.621932983398 -19.9422168731689\\
138.634429931641 -19.9429588317871\\
138.646926879883 -19.9437026977539\\
138.659423828125 -19.9444541931152\\
138.671936035156 -19.9452171325684\\
138.684432983398 -19.9459781646729\\
138.696929931641 -19.9467449188232\\
138.709426879883 -19.9475250244141\\
138.721923828125 -19.9483051300049\\
138.734436035156 -19.9490871429443\\
138.746932983398 -19.9498691558838\\
138.759429931641 -19.9506664276123\\
138.771926879883 -19.9514656066895\\
138.784423828125 -19.9522666931152\\
138.796936035156 -19.953067779541\\
138.809432983398 -19.9538688659668\\
138.821929931641 -19.9546794891357\\
138.834426879883 -19.9554977416992\\
138.846923828125 -19.9563159942627\\
138.859420776367 -19.9571285247803\\
138.871917724609 -19.9579582214355\\
138.884429931641 -19.958797454834\\
138.896926879883 -19.9596290588379\\
138.909423828125 -19.9604530334473\\
138.921920776367 -19.9612922668457\\
138.934417724609 -19.9621295928955\\
138.946929931641 -19.9629688262939\\
138.959426879883 -19.9638061523438\\
138.971923828125 -19.9646434783936\\
138.984420776367 -19.965482711792\\
138.996917724609 -19.9663219451904\\
139.009429931641 -19.9671611785889\\
139.021926879883 -19.9679985046387\\
139.034423828125 -19.9688358306885\\
139.046920776367 -19.9696731567383\\
139.059417724609 -19.9705104827881\\
139.071929931641 -19.9713497161865\\
139.084426879883 -19.9721870422363\\
139.096923828125 -19.9730243682861\\
139.109420776367 -19.9738616943359\\
139.121917724609 -19.9746990203857\\
139.134429931641 -19.9755382537842\\
139.146926879883 -19.976375579834\\
139.159423828125 -19.9772129058838\\
139.171920776367 -19.978048324585\\
139.184417724609 -19.9788684844971\\
139.196929931641 -19.9796886444092\\
139.209426879883 -19.9805088043213\\
139.221923828125 -19.9813289642334\\
139.234420776367 -19.9821472167969\\
139.246917724609 -19.9829654693604\\
139.259429931641 -19.9837856292725\\
139.271926879883 -19.9845867156982\\
139.284423828125 -19.985387802124\\
139.296920776367 -19.9861888885498\\
139.309417724609 -19.986988067627\\
139.321929931641 -19.9877891540527\\
139.334426879883 -19.9885711669922\\
139.346923828125 -19.9893531799316\\
139.359420776367 -19.9901332855225\\
139.371917724609 -19.9909133911133\\
139.384429931641 -19.9916839599609\\
139.396926879883 -19.9924468994141\\
139.409423828125 -19.9932098388672\\
139.421920776367 -19.9939708709717\\
139.434417724609 -19.9947280883789\\
139.446929931641 -19.9954719543457\\
139.459426879883 -19.9962139129639\\
139.471923828125 -19.9969577789307\\
139.484420776367 -19.9976902008057\\
139.496917724609 -19.9984149932861\\
139.509429931641 -19.9991397857666\\
139.521926879883 -19.9998569488525\\
139.534423828125 -20.0005626678467\\
139.546920776367 -20.0012664794922\\
139.559417724609 -20.0019664764404\\
139.571929931641 -20.0026531219482\\
139.584426879883 -20.0033397674561\\
139.596923828125 -20.004020690918\\
139.609420776367 -20.0046882629395\\
139.621917724609 -20.0053539276123\\
139.634429931641 -20.0060176849365\\
139.646926879883 -20.0066661834717\\
139.659423828125 -20.0073146820068\\
139.671920776367 -20.0079593658447\\
139.684417724609 -20.0085887908936\\
139.696929931641 -20.0092182159424\\
139.709426879883 -20.0098457336426\\
139.721923828125 -20.0104579925537\\
139.734420776367 -20.0110683441162\\
139.746917724609 -20.0116767883301\\
139.759429931641 -20.012279510498\\
139.771926879883 -20.0128707885742\\
139.784423828125 -20.0134620666504\\
139.796920776367 -20.0140514373779\\
139.809417724609 -20.0146255493164\\
139.821929931641 -20.0151977539063\\
139.834426879883 -20.0157699584961\\
139.846923828125 -20.0163402557373\\
139.859420776367 -20.0168952941895\\
139.871917724609 -20.017448425293\\
139.884429931641 -20.0180015563965\\
139.896926879883 -20.0185508728027\\
139.909423828125 -20.0190849304199\\
139.921920776367 -20.0196189880371\\
139.934417724609 -20.0201511383057\\
139.946929931641 -20.0206851959229\\
139.959426879883 -20.0212059020996\\
139.971923828125 -20.0217208862305\\
139.984420776367 -20.0222358703613\\
139.996917724609 -20.0227489471436\\
140.009429931641 -20.0232582092285\\
140.021926879883 -20.023754119873\\
140.034423828125 -20.0242500305176\\
140.046920776367 -20.0247459411621\\
140.059417724609 -20.025239944458\\
140.071929931641 -20.0257358551025\\
140.084426879883 -20.0262145996094\\
140.096923828125 -20.0266914367676\\
140.109420776367 -20.0271682739258\\
140.121917724609 -20.027645111084\\
140.134429931641 -20.0281181335449\\
140.146926879883 -20.0285758972168\\
140.159423828125 -20.0290336608887\\
140.171920776367 -20.0294914245605\\
140.184417724609 -20.0299491882324\\
140.196929931641 -20.0304069519043\\
140.209426879883 -20.0308628082275\\
140.221923828125 -20.0313186645508\\
140.234420776367 -20.0317573547363\\
140.246917724609 -20.0321960449219\\
140.259429931641 -20.0326347351074\\
140.271926879883 -20.0330505371094\\
140.284423828125 -20.0334644317627\\
140.296920776367 -20.0339126586914\\
140.309417724609 -20.0343856811523\\
140.321929931641 -20.0348625183105\\
140.334426879883 -20.0353393554688\\
140.346923828125 -20.0358142852783\\
140.359420776367 -20.0362720489502\\
140.371917724609 -20.0367298126221\\
140.384429931641 -20.0371875762939\\
140.396926879883 -20.0376453399658\\
140.409423828125 -20.0381011962891\\
140.421920776367 -20.0385570526123\\
140.434417724609 -20.0390129089355\\
140.446929931641 -20.0394706726074\\
140.459426879883 -20.0399265289307\\
140.471923828125 -20.0403652191162\\
140.484420776367 -20.0408039093018\\
140.496917724609 -20.0412406921387\\
140.509429931641 -20.0416603088379\\
140.521926879883 -20.0420799255371\\
140.534423828125 -20.0424995422363\\
140.546920776367 -20.0429191589355\\
140.559417724609 -20.0433368682861\\
140.571929931641 -20.0437450408936\\
140.584426879883 -20.0441455841064\\
140.596923828125 -20.0445461273193\\
140.609420776367 -20.0449371337891\\
140.621917724609 -20.0453186035156\\
140.634429931641 -20.0457000732422\\
140.646926879883 -20.0460815429688\\
140.659423828125 -20.0464477539063\\
140.671920776367 -20.0468101501465\\
140.684417724609 -20.0471725463867\\
140.696929931641 -20.047534942627\\
140.709426879883 -20.0478782653809\\
140.721923828125 -20.0482215881348\\
140.734420776367 -20.0485649108887\\
140.746917724609 -20.0488948822021\\
140.759429931641 -20.0492191314697\\
140.771926879883 -20.0495338439941\\
140.784423828125 -20.0498275756836\\
140.796920776367 -20.0500907897949\\
140.809417724609 -20.0503273010254\\
140.821929931641 -20.0505180358887\\
140.834426879883 -20.0506706237793\\
140.846923828125 -20.0507946014404\\
140.859420776367 -20.0508995056152\\
140.871917724609 -20.0510082244873\\
140.884429931641 -20.0511493682861\\
140.896926879883 -20.0513134002686\\
140.909423828125 -20.0515079498291\\
140.921920776367 -20.0517177581787\\
140.934417724609 -20.0519275665283\\
140.946929931641 -20.0521373748779\\
140.959426879883 -20.0523471832275\\
140.971923828125 -20.0525493621826\\
140.984420776367 -20.0527400970459\\
140.996917724609 -20.0529308319092\\
141.009429931641 -20.0531158447266\\
141.021926879883 -20.0532875061035\\
141.034423828125 -20.0534591674805\\
141.046920776367 -20.0536117553711\\
141.059417724609 -20.0537643432617\\
141.071929931641 -20.0539169311523\\
141.084426879883 -20.0540752410889\\
141.096923828125 -20.0542469024658\\
141.109420776367 -20.0544185638428\\
141.121917724609 -20.0545902252197\\
141.134429931641 -20.0547618865967\\
141.146926879883 -20.0549354553223\\
141.159423828125 -20.0551242828369\\
141.171920776367 -20.0553131103516\\
141.184417724609 -20.0555019378662\\
141.196929931641 -20.0556926727295\\
141.209426879883 -20.0558815002441\\
141.221923828125 -20.0560722351074\\
141.234420776367 -20.0562629699707\\
141.246917724609 -20.056453704834\\
141.259429931641 -20.0566444396973\\
141.271926879883 -20.0568351745605\\
141.284423828125 -20.0570259094238\\
141.296920776367 -20.0572166442871\\
141.309417724609 -20.0574073791504\\
141.321929931641 -20.0575981140137\\
141.334426879883 -20.057788848877\\
141.346923828125 -20.0579795837402\\
141.359420776367 -20.0581703186035\\
141.371917724609 -20.0583610534668\\
141.384429931641 -20.0585460662842\\
141.396926879883 -20.0587005615234\\
141.409423828125 -20.0588035583496\\
141.421920776367 -20.0588798522949\\
141.434417724609 -20.0589561462402\\
141.446929931641 -20.0590324401855\\
141.459426879883 -20.0591259002686\\
141.471923828125 -20.0592212677002\\
141.484420776367 -20.0593166351318\\
141.496917724609 -20.0594120025635\\
141.509429931641 -20.0595073699951\\
141.521926879883 -20.0596160888672\\
141.534423828125 -20.0597305297852\\
141.546920776367 -20.0598449707031\\
141.559417724609 -20.0599613189697\\
141.571929931641 -20.0600929260254\\
141.584426879883 -20.0602264404297\\
141.596923828125 -20.060359954834\\
141.609420776367 -20.0604934692383\\
141.621917724609 -20.0606269836426\\
141.634429931641 -20.0607929229736\\
141.646926879883 -20.0610008239746\\
141.659423828125 -20.0612239837646\\
141.671920776367 -20.0614337921143\\
141.684417724609 -20.0616474151611\\
141.696929931641 -20.0618762969971\\
141.709426879883 -20.062105178833\\
141.721923828125 -20.0623531341553\\
141.734420776367 -20.0626068115234\\
141.746917724609 -20.062873840332\\
141.759429931641 -20.0631561279297\\
141.771926879883 -20.0634536743164\\
141.784423828125 -20.0637798309326\\
141.796920776367 -20.0641479492188\\
141.809417724609 -20.0645713806152\\
141.821929931641 -20.0650577545166\\
141.834426879883 -20.0656070709229\\
141.846923828125 -20.0662231445313\\
141.859420776367 -20.0669040679932\\
141.871917724609 -20.0676441192627\\
141.884429931641 -20.0684432983398\\
141.896926879883 -20.0692882537842\\
141.909423828125 -20.070161819458\\
141.921920776367 -20.0710334777832\\
141.934417724609 -20.0718650817871\\
141.946929931641 -20.0725955963135\\
141.959426879883 -20.0732040405273\\
141.971923828125 -20.0737247467041\\
141.984420776367 -20.074197769165\\
141.996917724609 -20.0746574401855\\
142.009429931641 -20.0751342773438\\
142.021926879883 -20.0756282806396\\
142.034423828125 -20.0761413574219\\
142.046920776367 -20.0766754150391\\
142.059417724609 -20.0772304534912\\
142.071929931641 -20.077808380127\\
142.084426879883 -20.0784111022949\\
142.096923828125 -20.0790481567383\\
142.109420776367 -20.0797157287598\\
142.121917724609 -20.0804138183594\\
142.134429931641 -20.0811386108398\\
142.146926879883 -20.0818843841553\\
142.159423828125 -20.0826473236084\\
142.171920776367 -20.083423614502\\
142.184417724609 -20.0842037200928\\
142.196929931641 -20.0849781036377\\
142.209426879883 -20.0857048034668\\
142.221923828125 -20.0863952636719\\
142.234420776367 -20.087085723877\\
142.246917724609 -20.0877933502197\\
142.259429931641 -20.0885219573975\\
142.271926879883 -20.0892677307129\\
142.284423828125 -20.0900287628174\\
142.296920776367 -20.0908088684082\\
142.309417724609 -20.0916023254395\\
142.321929931641 -20.0924053192139\\
142.334426879883 -20.0932235717773\\
142.346923828125 -20.0940475463867\\
142.359420776367 -20.0948848724365\\
142.371917724609 -20.0957317352295\\
142.384429931641 -20.0965900421143\\
142.396926879883 -20.0974655151367\\
142.409423828125 -20.0983428955078\\
142.421920776367 -20.0992259979248\\
142.434417724609 -20.1001110076904\\
142.446929931641 -20.100980758667\\
142.459426879883 -20.1017742156982\\
142.471923828125 -20.1024036407471\\
142.484420776367 -20.1029186248779\\
142.496917724609 -20.1034164428711\\
142.509429931641 -20.1039390563965\\
142.521926879883 -20.1044902801514\\
142.534423828125 -20.105073928833\\
142.546920776367 -20.10569190979\\
142.559417724609 -20.1063385009766\\
142.571929931641 -20.1070137023926\\
142.584426879883 -20.1077117919922\\
142.596923828125 -20.108434677124\\
142.609420776367 -20.1091861724854\\
142.621917724609 -20.1099510192871\\
142.634429931641 -20.1107330322266\\
142.646926879883 -20.1115245819092\\
142.659423828125 -20.1123237609863\\
142.671920776367 -20.1131248474121\\
142.684417724609 -20.1139259338379\\
142.696929931641 -20.114725112915\\
142.709426879883 -20.1154842376709\\
142.721923828125 -20.1160984039307\\
142.734420776367 -20.1165008544922\\
142.746917724609 -20.1168022155762\\
142.759429931641 -20.1170883178711\\
142.771926879883 -20.1173706054688\\
142.784423828125 -20.1176319122314\\
142.796920776367 -20.1178607940674\\
142.809417724609 -20.1180458068848\\
142.821929931641 -20.1182041168213\\
142.834426879883 -20.1183567047119\\
142.846923828125 -20.1185092926025\\
142.859420776367 -20.1186618804932\\
142.871917724609 -20.1188087463379\\
142.884429931641 -20.1189422607422\\
142.896926879883 -20.1190567016602\\
142.909423828125 -20.1191558837891\\
142.921920776367 -20.1192512512207\\
142.934417724609 -20.1193466186523\\
142.946929931641 -20.119441986084\\
142.959426879883 -20.1195373535156\\
142.971923828125 -20.1196327209473\\
142.984420776367 -20.1197280883789\\
142.996917724609 -20.1198234558105\\
143.009429931641 -20.1199188232422\\
143.021926879883 -20.1200141906738\\
143.034423828125 -20.1201229095459\\
143.046920776367 -20.1202373504639\\
143.059417724609 -20.1203517913818\\
143.071929931641 -20.1204662322998\\
143.084426879883 -20.1205806732178\\
143.096923828125 -20.1206951141357\\
143.109420776367 -20.1208095550537\\
143.121917724609 -20.1209239959717\\
143.134429931641 -20.1210384368896\\
143.146926879883 -20.1211681365967\\
143.159423828125 -20.121301651001\\
143.171920776367 -20.1214408874512\\
143.184417724609 -20.1215934753418\\
143.196929931641 -20.1217555999756\\
143.209426879883 -20.1219272613525\\
143.221923828125 -20.1221122741699\\
143.234420776367 -20.1223049163818\\
143.246917724609 -20.1225147247314\\
143.259429931641 -20.122730255127\\
143.271926879883 -20.1229591369629\\
143.284423828125 -20.1231937408447\\
143.296920776367 -20.123441696167\\
143.309417724609 -20.1237010955811\\
143.321929931641 -20.1239681243896\\
143.334426879883 -20.1242523193359\\
143.346923828125 -20.1245403289795\\
143.359420776367 -20.1248455047607\\
143.371917724609 -20.1251525878906\\
143.384429931641 -20.1254768371582\\
143.396926879883 -20.1258010864258\\
143.409423828125 -20.1261367797852\\
143.421920776367 -20.1264801025391\\
143.434417724609 -20.1268253326416\\
143.446929931641 -20.1271877288818\\
143.459426879883 -20.1275501251221\\
143.471923828125 -20.1279144287109\\
143.484420776367 -20.1282939910889\\
143.496917724609 -20.1286754608154\\
143.509429931641 -20.129056930542\\
143.521926879883 -20.1294422149658\\
143.534423828125 -20.1298408508301\\
143.546920776367 -20.130241394043\\
143.559417724609 -20.1306419372559\\
143.571929931641 -20.1310424804688\\
143.584426879883 -20.1314430236816\\
143.596923828125 -20.1318435668945\\
143.609420776367 -20.1322536468506\\
143.621917724609 -20.1326713562012\\
143.634429931641 -20.1330909729004\\
143.646926879883 -20.133508682251\\
143.659423828125 -20.1339282989502\\
143.671920776367 -20.1343479156494\\
143.684417724609 -20.1347675323486\\
143.696929931641 -20.1351871490479\\
143.709426879883 -20.1356067657471\\
143.721923828125 -20.1360263824463\\
143.734420776367 -20.1364459991455\\
143.746917724609 -20.1368656158447\\
143.759429931641 -20.1372852325439\\
143.771926879883 -20.1377029418945\\
143.784423828125 -20.1381206512451\\
143.796920776367 -20.1385383605957\\
143.809417724609 -20.1389560699463\\
143.821929931641 -20.1393756866455\\
143.834426879883 -20.1397933959961\\
143.846923828125 -20.1402111053467\\
143.859420776367 -20.1406288146973\\
143.871917724609 -20.1410465240479\\
143.884429931641 -20.1414566040039\\
143.896926879883 -20.1418590545654\\
143.909423828125 -20.1422595977783\\
143.921920776367 -20.1426601409912\\
143.934417724609 -20.1430606842041\\
143.946929931641 -20.143461227417\\
143.959426879883 -20.1438617706299\\
143.971923828125 -20.1442623138428\\
143.984420776367 -20.1446628570557\\
143.996917724609 -20.1450634002686\\
144.009429931641 -20.1454639434814\\
144.021926879883 -20.1458644866943\\
144.034423828125 -20.1462650299072\\
144.046920776367 -20.1466655731201\\
144.059417724609 -20.147066116333\\
144.071929931641 -20.1474666595459\\
144.084426879883 -20.1478672027588\\
144.096923828125 -20.1482677459717\\
144.109420776367 -20.1486663818359\\
144.121917724609 -20.1490650177002\\
144.134429931641 -20.1494655609131\\
144.146926879883 -20.1498565673828\\
144.159423828125 -20.1502380371094\\
144.171920776367 -20.1506195068359\\
144.184417724609 -20.1510009765625\\
144.196929931641 -20.1513824462891\\
144.209426879883 -20.1517639160156\\
144.221923828125 -20.1521453857422\\
144.234420776367 -20.1525268554688\\
144.246917724609 -20.1529083251953\\
144.259429931641 -20.1532897949219\\
144.271926879883 -20.1536712646484\\
144.284423828125 -20.154052734375\\
144.296920776367 -20.1544342041016\\
144.309417724609 -20.1548156738281\\
144.321929931641 -20.1551971435547\\
144.334426879883 -20.1555786132813\\
144.346923828125 -20.1559600830078\\
144.359420776367 -20.1563415527344\\
144.371917724609 -20.1567230224609\\
144.384429931641 -20.1571044921875\\
144.396926879883 -20.1574859619141\\
144.409423828125 -20.1578674316406\\
144.421920776367 -20.1582489013672\\
144.434417724609 -20.1586303710938\\
144.446929931641 -20.1590118408203\\
144.459426879883 -20.1593933105469\\
144.471923828125 -20.1597747802734\\
144.484420776367 -20.16015625\\
144.496917724609 -20.1605377197266\\
144.509429931641 -20.1609191894531\\
144.521926879883 -20.1613006591797\\
144.534423828125 -20.1616802215576\\
144.546920776367 -20.1620597839355\\
144.559417724609 -20.1624393463135\\
144.571929931641 -20.1628093719482\\
144.584426879883 -20.1631717681885\\
144.596923828125 -20.1635341644287\\
144.609420776367 -20.1638965606689\\
144.621917724609 -20.1642589569092\\
144.634429931641 -20.1646213531494\\
144.646926879883 -20.1649837493896\\
144.659423828125 -20.1653461456299\\
144.671920776367 -20.1657085418701\\
144.684417724609 -20.1660709381104\\
144.696929931641 -20.1664333343506\\
144.709426879883 -20.1667938232422\\
144.721923828125 -20.1671504974365\\
144.734420776367 -20.1674938201904\\
144.746917724609 -20.1678371429443\\
144.759429931641 -20.1681804656982\\
144.771926879883 -20.1685237884521\\
144.784423828125 -20.1688671112061\\
144.796920776367 -20.16921043396\\
144.809417724609 -20.1695537567139\\
144.821929931641 -20.1698970794678\\
144.834426879883 -20.170238494873\\
144.846923828125 -20.1705684661865\\
144.859420776367 -20.1708927154541\\
144.871917724609 -20.1712169647217\\
144.884429931641 -20.1715412139893\\
144.896926879883 -20.1718654632568\\
144.909423828125 -20.1721897125244\\
144.921920776367 -20.172513961792\\
144.934417724609 -20.1728363037109\\
144.946929931641 -20.1731605529785\\
144.959426879883 -20.173469543457\\
144.971923828125 -20.1737747192383\\
144.984420776367 -20.1740798950195\\
144.996917724609 -20.1743850708008\\
145.009429931641 -20.174690246582\\
145.021926879883 -20.1749954223633\\
145.034423828125 -20.1753005981445\\
145.046920776367 -20.1756057739258\\
145.059417724609 -20.175910949707\\
145.071929931641 -20.1762161254883\\
145.084426879883 -20.1765213012695\\
145.096923828125 -20.1768264770508\\
145.109420776367 -20.177131652832\\
145.121917724609 -20.1774349212646\\
145.134429931641 -20.1777400970459\\
145.146926879883 -20.1780433654785\\
145.159423828125 -20.1783313751221\\
145.171920776367 -20.178617477417\\
145.184417724609 -20.1789035797119\\
145.196929931641 -20.1791896820068\\
145.209426879883 -20.1794757843018\\
145.221923828125 -20.1797618865967\\
145.234420776367 -20.1800479888916\\
145.246917724609 -20.1803340911865\\
145.259429931641 -20.1806201934814\\
145.271926879883 -20.1809062957764\\
145.284423828125 -20.1811904907227\\
145.296920776367 -20.1814651489258\\
145.309417724609 -20.1817321777344\\
145.321929931641 -20.181999206543\\
145.334426879883 -20.1822662353516\\
145.346923828125 -20.1825332641602\\
145.359420776367 -20.1828002929688\\
145.371917724609 -20.1830673217773\\
145.384429931641 -20.1833343505859\\
145.396926879883 -20.1836013793945\\
145.409423828125 -20.1838665008545\\
145.421920776367 -20.184118270874\\
145.434417724609 -20.1843662261963\\
145.446929931641 -20.1846141815186\\
145.459426879883 -20.1848621368408\\
145.471923828125 -20.1851100921631\\
145.484420776367 -20.1853580474854\\
145.496917724609 -20.1856060028076\\
145.509429931641 -20.1858539581299\\
145.521926879883 -20.1861019134521\\
145.534423828125 -20.1863479614258\\
145.546920776367 -20.1865787506104\\
145.559417724609 -20.1868076324463\\
145.571929931641 -20.1870365142822\\
145.584426879883 -20.1872653961182\\
145.596923828125 -20.1874942779541\\
145.609420776367 -20.18772315979\\
145.621917724609 -20.187952041626\\
145.634429931641 -20.1881809234619\\
145.646926879883 -20.1884098052979\\
145.659423828125 -20.1886386871338\\
145.671920776367 -20.1888656616211\\
145.684417724609 -20.1890926361084\\
145.696929931641 -20.1893100738525\\
145.709426879883 -20.1895198822021\\
145.721923828125 -20.1897296905518\\
145.734420776367 -20.1899394989014\\
145.746917724609 -20.190149307251\\
145.759429931641 -20.1903591156006\\
145.771926879883 -20.1905689239502\\
145.784423828125 -20.1907787322998\\
145.796920776367 -20.1909885406494\\
145.809417724609 -20.191198348999\\
145.821929931641 -20.1914081573486\\
145.834426879883 -20.1916179656982\\
145.846923828125 -20.1918277740479\\
145.859420776367 -20.1920375823975\\
145.871917724609 -20.1922473907471\\
145.884429931641 -20.1924571990967\\
145.896926879883 -20.1926651000977\\
145.909423828125 -20.1928730010986\\
145.921920776367 -20.1930809020996\\
145.934417724609 -20.1932754516602\\
145.946929931641 -20.1934661865234\\
145.959426879883 -20.1936569213867\\
145.971923828125 -20.19384765625\\
145.984420776367 -20.1940383911133\\
145.996917724609 -20.1942291259766\\
146.009429931641 -20.1944198608398\\
146.021926879883 -20.1946105957031\\
146.034423828125 -20.1948013305664\\
146.046920776367 -20.1949920654297\\
146.059417724609 -20.195182800293\\
146.071929931641 -20.1953735351563\\
146.084426879883 -20.1955642700195\\
146.096923828125 -20.1957550048828\\
146.109420776367 -20.1959457397461\\
146.121917724609 -20.1961345672607\\
146.134429931641 -20.196325302124\\
146.146926879883 -20.1964988708496\\
146.159423828125 -20.1966705322266\\
146.171920776367 -20.1968421936035\\
146.184417724609 -20.1970138549805\\
146.196929931641 -20.1971855163574\\
146.209426879883 -20.1973571777344\\
146.221923828125 -20.1975288391113\\
146.234420776367 -20.1977005004883\\
146.246917724609 -20.1978721618652\\
146.259429931641 -20.1980438232422\\
146.271926879883 -20.1982154846191\\
146.284423828125 -20.1983852386475\\
146.296920776367 -20.1985416412354\\
146.309417724609 -20.198694229126\\
146.321929931641 -20.1988468170166\\
146.334426879883 -20.1989994049072\\
146.346923828125 -20.1991519927979\\
146.359420776367 -20.1993045806885\\
146.371917724609 -20.1994571685791\\
146.384429931641 -20.1996097564697\\
146.396926879883 -20.1997623443604\\
146.409423828125 -20.199914932251\\
146.421920776367 -20.2000675201416\\
146.434417724609 -20.2002182006836\\
146.446929931641 -20.200366973877\\
146.459426879883 -20.2005004882813\\
146.471923828125 -20.2006340026855\\
146.484420776367 -20.2007675170898\\
146.496917724609 -20.2009010314941\\
146.509429931641 -20.2010345458984\\
146.521926879883 -20.2011680603027\\
146.534423828125 -20.201301574707\\
146.546920776367 -20.2014350891113\\
146.559417724609 -20.2015686035156\\
146.571929931641 -20.2017021179199\\
146.584426879883 -20.2018356323242\\
146.596923828125 -20.2019691467285\\
146.609420776367 -20.2021026611328\\
146.621917724609 -20.2022361755371\\
146.634429931641 -20.2023696899414\\
146.646926879883 -20.202486038208\\
146.659423828125 -20.202600479126\\
146.671920776367 -20.2027149200439\\
146.684417724609 -20.2028293609619\\
146.696929931641 -20.2029438018799\\
146.709426879883 -20.2030582427979\\
146.721923828125 -20.2031726837158\\
146.734420776367 -20.2032871246338\\
146.746917724609 -20.2034015655518\\
146.759429931641 -20.2035160064697\\
146.771926879883 -20.2036304473877\\
146.784423828125 -20.2037448883057\\
146.796920776367 -20.2038593292236\\
146.809417724609 -20.2039737701416\\
146.821929931641 -20.2040882110596\\
146.834426879883 -20.204195022583\\
146.846923828125 -20.2042903900146\\
146.859420776367 -20.2043857574463\\
146.871917724609 -20.2044811248779\\
146.884429931641 -20.2045764923096\\
146.896926879883 -20.2046718597412\\
146.909423828125 -20.2047672271729\\
146.921920776367 -20.2048625946045\\
146.934417724609 -20.2049579620361\\
146.946929931641 -20.2050533294678\\
146.959426879883 -20.2051486968994\\
146.971923828125 -20.2052440643311\\
146.984420776367 -20.2053394317627\\
146.996917724609 -20.2054347991943\\
147.009429931641 -20.205530166626\\
147.021926879883 -20.2056255340576\\
147.034423828125 -20.2057209014893\\
147.046920776367 -20.2058162689209\\
147.059417724609 -20.2059020996094\\
147.071929931641 -20.2059783935547\\
147.084426879883 -20.2060546875\\
147.096923828125 -20.2061309814453\\
147.109420776367 -20.2062072753906\\
147.121917724609 -20.2062835693359\\
147.134429931641 -20.2063598632813\\
147.146926879883 -20.2064361572266\\
147.159423828125 -20.2065124511719\\
147.171920776367 -20.2065887451172\\
147.184417724609 -20.2066650390625\\
147.196929931641 -20.2067413330078\\
147.209426879883 -20.2068290710449\\
147.221923828125 -20.2069129943848\\
147.234420776367 -20.2069892883301\\
147.246917724609 -20.2070655822754\\
147.259429931641 -20.2071418762207\\
147.271926879883 -20.207218170166\\
147.284423828125 -20.2072944641113\\
147.296920776367 -20.2073707580566\\
147.309417724609 -20.207447052002\\
147.321929931641 -20.2075233459473\\
147.334426879883 -20.2075996398926\\
147.346923828125 -20.2076759338379\\
147.359420776367 -20.2077522277832\\
147.371917724609 -20.2078285217285\\
147.384429931641 -20.2079048156738\\
147.396926879883 -20.2079811096191\\
147.409423828125 -20.2080574035645\\
147.421920776367 -20.2081336975098\\
147.434417724609 -20.2082099914551\\
147.446929931641 -20.2082862854004\\
147.459426879883 -20.2083625793457\\
147.471923828125 -20.208438873291\\
147.484420776367 -20.2085151672363\\
147.496917724609 -20.2085914611816\\
147.509429931641 -20.2086639404297\\
147.521926879883 -20.2087268829346\\
147.534423828125 -20.2088012695313\\
147.546920776367 -20.2088775634766\\
147.559417724609 -20.2089538574219\\
147.571929931641 -20.2090301513672\\
147.584426879883 -20.2091064453125\\
147.596923828125 -20.2091827392578\\
147.609420776367 -20.2092590332031\\
147.621917724609 -20.2093353271484\\
147.634429931641 -20.2094116210938\\
147.646926879883 -20.2094879150391\\
147.659423828125 -20.2095737457275\\
147.671920776367 -20.2096691131592\\
147.684417724609 -20.2097644805908\\
147.696929931641 -20.2098789215088\\
147.709426879883 -20.2099933624268\\
147.721923828125 -20.2101078033447\\
147.734420776367 -20.2102222442627\\
147.746917724609 -20.2103233337402\\
147.759429931641 -20.2104148864746\\
147.771926879883 -20.2104911804199\\
147.784423828125 -20.2105674743652\\
147.796920776367 -20.2106246948242\\
147.809417724609 -20.2106819152832\\
147.821929931641 -20.2107391357422\\
147.834426879883 -20.2108001708984\\
147.846923828125 -20.2108764648438\\
147.859420776367 -20.2109622955322\\
147.871917724609 -20.2110633850098\\
147.884429931641 -20.2111778259277\\
147.896926879883 -20.2113075256348\\
147.909423828125 -20.211446762085\\
147.921920776367 -20.2115993499756\\
147.934417724609 -20.2117710113525\\
147.946929931641 -20.2119560241699\\
147.959426879883 -20.2121467590332\\
147.971923828125 -20.2123241424561\\
147.984420776367 -20.2124767303467\\
147.996917724609 -20.2125988006592\\
148.009429931641 -20.2126884460449\\
148.021926879883 -20.2127647399902\\
148.034423828125 -20.2128410339355\\
148.046920776367 -20.2129173278809\\
148.059417724609 -20.213005065918\\
148.071929931641 -20.2131080627441\\
148.084426879883 -20.2132244110107\\
148.096923828125 -20.213357925415\\
148.109420776367 -20.2134971618652\\
148.121917724609 -20.2136497497559\\
148.134429931641 -20.2138023376465\\
148.146926879883 -20.213960647583\\
148.159423828125 -20.21413230896\\
148.171920776367 -20.2143077850342\\
148.184417724609 -20.2144985198975\\
148.196929931641 -20.2147006988525\\
148.209426879883 -20.2149181365967\\
148.221923828125 -20.2151489257813\\
148.234420776367 -20.2153968811035\\
148.246917724609 -20.2156448364258\\
148.259429931641 -20.215892791748\\
148.271926879883 -20.2161388397217\\
148.284423828125 -20.2163677215576\\
148.296920776367 -20.216588973999\\
148.309417724609 -20.2167987823486\\
148.321929931641 -20.2170085906982\\
148.334426879883 -20.2172012329102\\
148.346923828125 -20.2173919677734\\
148.359420776367 -20.2175827026367\\
148.371917724609 -20.2177734375\\
148.384429931641 -20.2179641723633\\
148.396926879883 -20.2181549072266\\
148.409423828125 -20.2183456420898\\
148.421920776367 -20.2185363769531\\
148.434417724609 -20.2187271118164\\
148.446929931641 -20.2189178466797\\
148.459426879883 -20.219108581543\\
148.471923828125 -20.2192993164063\\
148.484420776367 -20.2194900512695\\
};
\addplot [
color=mycolor1,
solid
]
table[row sep=crcr]{
148.484420776367 -20.2194900512695\\
148.496917724609 -20.2196807861328\\
148.509429931641 -20.2198715209961\\
148.521926879883 -20.2200489044189\\
148.534423828125 -20.2202205657959\\
148.546920776367 -20.2203922271729\\
148.559417724609 -20.2205638885498\\
148.571929931641 -20.2207450866699\\
148.584426879883 -20.2209339141846\\
148.596923828125 -20.2211246490479\\
148.609420776367 -20.2213153839111\\
148.621917724609 -20.2215061187744\\
148.634429931641 -20.2216968536377\\
148.646926879883 -20.221887588501\\
148.659423828125 -20.2220802307129\\
148.671920776367 -20.2222900390625\\
148.684417724609 -20.2224998474121\\
148.696929931641 -20.2227115631104\\
148.709426879883 -20.2229385375977\\
148.721923828125 -20.2231769561768\\
148.734420776367 -20.223424911499\\
148.746917724609 -20.2236919403076\\
148.759429931641 -20.2239589691162\\
148.771926879883 -20.2242183685303\\
148.784423828125 -20.2244663238525\\
148.796920776367 -20.2247180938721\\
148.809417724609 -20.2249851226807\\
148.821929931641 -20.2252521514893\\
148.834426879883 -20.2255191802979\\
148.846923828125 -20.225793838501\\
148.859420776367 -20.2260799407959\\
148.871917724609 -20.2263851165771\\
148.884429931641 -20.2267093658447\\
148.896926879883 -20.2270565032959\\
148.909423828125 -20.2274303436279\\
148.921920776367 -20.2278327941895\\
148.934417724609 -20.2282619476318\\
148.946929931641 -20.2287101745605\\
148.959426879883 -20.2291679382324\\
148.971923828125 -20.2296161651611\\
148.984420776367 -20.2300262451172\\
148.996917724609 -20.2303886413574\\
149.009429931641 -20.2307090759277\\
149.021926879883 -20.2310009002686\\
149.034423828125 -20.231273651123\\
149.046920776367 -20.2315368652344\\
149.059417724609 -20.2317848205566\\
149.071929931641 -20.2320327758789\\
149.084426879883 -20.2322807312012\\
149.096923828125 -20.2325267791748\\
149.109420776367 -20.2327709197998\\
149.121917724609 -20.2330169677734\\
149.134429931641 -20.2332649230957\\
149.146926879883 -20.233512878418\\
149.159423828125 -20.2337608337402\\
149.171920776367 -20.2339916229248\\
149.184417724609 -20.2342071533203\\
149.196929931641 -20.2344055175781\\
149.209426879883 -20.2345962524414\\
149.221923828125 -20.2347869873047\\
149.234420776367 -20.234977722168\\
149.246917724609 -20.2351684570313\\
149.259429931641 -20.2353591918945\\
149.271926879883 -20.2355499267578\\
149.284423828125 -20.2357406616211\\
149.296920776367 -20.2359313964844\\
149.309417724609 -20.2361221313477\\
149.321929931641 -20.2363052368164\\
149.334426879883 -20.2364730834961\\
149.346923828125 -20.2366256713867\\
149.359420776367 -20.2367782592773\\
149.371917724609 -20.236930847168\\
149.384429931641 -20.2370948791504\\
149.396926879883 -20.2372665405273\\
149.409423828125 -20.2374382019043\\
149.421920776367 -20.2376098632813\\
149.434417724609 -20.2377815246582\\
149.446929931641 -20.2379531860352\\
149.459426879883 -20.2381153106689\\
149.471923828125 -20.2382678985596\\
149.484420776367 -20.2384204864502\\
149.496917724609 -20.2385730743408\\
149.509429931641 -20.2387256622314\\
149.521926879883 -20.2388935089111\\
149.534423828125 -20.2390651702881\\
149.546920776367 -20.239236831665\\
149.559417724609 -20.2394218444824\\
149.571929931641 -20.2396125793457\\
149.584426879883 -20.239803314209\\
149.596923828125 -20.2399940490723\\
149.609420776367 -20.2401847839355\\
149.621917724609 -20.2403755187988\\
149.634429931641 -20.2405662536621\\
149.646926879883 -20.2407569885254\\
149.659423828125 -20.2409477233887\\
149.671920776367 -20.241138458252\\
149.684417724609 -20.2413291931152\\
149.696929931641 -20.2415199279785\\
149.709426879883 -20.2417106628418\\
149.721923828125 -20.2419013977051\\
149.734420776367 -20.2420921325684\\
149.746917724609 -20.2422828674316\\
149.759429931641 -20.2424736022949\\
149.771926879883 -20.2426643371582\\
149.784423828125 -20.2428550720215\\
149.796920776367 -20.2430324554443\\
149.809417724609 -20.2431869506836\\
149.821929931641 -20.243314743042\\
149.834426879883 -20.24342918396\\
149.846923828125 -20.2435436248779\\
149.859420776367 -20.2436580657959\\
149.871917724609 -20.2437839508057\\
149.884429931641 -20.24391746521\\
149.896926879883 -20.2440509796143\\
149.909423828125 -20.2441844940186\\
149.921920776367 -20.2443180084229\\
149.934417724609 -20.2444515228271\\
149.946929931641 -20.2445850372314\\
149.959426879883 -20.2447185516357\\
149.971923828125 -20.24485206604\\
149.984420776367 -20.2449855804443\\
149.996917724609 -20.2451190948486\\
};
\addlegendentry{$\dtmax{} = \expneg{1.25}{3}{[s]}$};

\end{axis}
\end{tikzpicture}%
\caption{Condensation in upper head as calculated by the linear solver.}
\label{fig:refillGammaLin}
\end{figure}

First, the linear solution for the four different \dtmax{} is shown in \fig{fig:refillGammaLin}.
This shows the linear solver's behavior.
There is drastic variability in the condensation in the upper head.
However, note that the two smallest \dtmax{} appear to have reached a timestep-size insensitive solution.
Or, more precisely, there is less variability in this quantity over the last two simulations than between any previous two.

\begin{figure}[h!tb]
\centering
% This file was created by matlab2tikz v0.4.3.
% Copyright (c) 2008--2013, Nico Schlömer <nico.schloemer@gmail.com>
% All rights reserved.
% 
\tikzsetnextfilename{plots/refillGammaNln_pdf}
%
% defining custom colors
\definecolor{mycolor1}{rgb}{0,0.75,0.75}%
%
\begin{tikzpicture}

\begin{axis}[%
width=\mytikzpicwidth,
height=\mytikzpicheight,
scale only axis,
xmin=0,
xmax=150,
xlabel={Time $[\text{s}]$},
ymin=-30,
ymax=2,
ylabel={$\Gamma\, [\,\lbm{}\,]$},
legend style={at={(0.03,0.03)},anchor=south west,draw=black,fill=white,legend cell align=left}
]
\addplot [
color=blue,
solid
]
table[row sep=crcr]{
0 0\\
1.01079523563385 0\\
2.01079511642456 0\\
3.01079511642456 0\\
4.01079511642456 0\\
5.01079511642456 0\\
6.01079511642456 0\\
7.01079511642456 0\\
8.01079559326172 0\\
9.01079559326172 0\\
10.0107955932617 0\\
11.0107955932617 0\\
12.0107955932617 0\\
13.0107955932617 0\\
14.0107955932617 0\\
15.0107955932617 0\\
16.0107955932617 0\\
17.0107955932617 0\\
18.0107955932617 0\\
19.0107955932617 0\\
20.0107955932617 2.60989463640726e-06\\
21.0314159393311 -0.554439544677734\\
22.0460662841797 -0.647129535675049\\
23.0586357116699 -0.681936979293823\\
24.0713806152344 -0.706454634666443\\
25.096809387207 -0.728805422782898\\
26.1447067260742 -0.755968451499939\\
27.1853427886963 -0.790492534637451\\
28.1853427886963 -0.833746016025543\\
29.1853427886963 -0.889175176620483\\
30.2133598327637 -0.963945090770721\\
31.2133598327637 -1.05793881416321\\
32.24365234375 -1.17301607131958\\
33.2635192871094 -1.2957855463028\\
34.2936706542969 -1.42693030834198\\
35.3107719421387 -1.55557513237\\
36.3702278137207 -1.68389368057251\\
37.3802223205566 -1.80427730083466\\
38.4273223876953 -1.92830228805542\\
39.4817199707031 -2.05623698234558\\
40.5510749816895 -2.18928122520447\\
41.635326385498 -2.32601308822632\\
42.635326385498 -2.45070242881775\\
43.635326385498 -2.5736825466156\\
44.635326385498 -2.6958019733429\\
45.635326385498 -2.81917333602905\\
46.6425666809082 -2.94626617431641\\
47.6696014404297 -3.07866954803467\\
48.6947135925293 -3.19338321685791\\
49.7208023071289 -3.29190802574158\\
50.748161315918 -3.38221049308777\\
51.7829132080078 -3.46999287605286\\
52.8166389465332 -3.55597949028015\\
53.8467292785645 -3.64115071296692\\
54.8848724365234 -3.72512197494507\\
55.9106521606445 -3.80846357345581\\
56.9505271911621 -3.89277958869934\\
57.9935493469238 -3.97816252708435\\
59.0321197509766 -4.0638542175293\\
60.0685691833496 -4.15210056304932\\
61.1056594848633 -4.24027872085571\\
62.1273498535156 -4.32637882232666\\
63.1542701721191 -4.41491031646729\\
64.1740188598633 -4.50362205505371\\
65.1998062133789 -4.59477996826172\\
66.2371139526367 -4.68892765045166\\
67.2613906860352 -4.78248834609985\\
68.2861785888672 -4.87776470184326\\
69.3181381225586 -4.97287321090698\\
70.3109436035156 -5.06574964523315\\
71.3348541259766 -5.16253995895386\\
72.3680801391602 -5.25950813293457\\
73.3897323608398 -5.35609006881714\\
74.4266510009766 -5.45360994338989\\
75.4613265991211 -5.55025863647461\\
76.5046539306641 -5.64799070358276\\
77.5217590332031 -5.74406671524048\\
78.5217590332031 -5.84062004089355\\
79.5217590332031 -5.93739891052246\\
80.5217590332031 -6.0351128578186\\
81.5217590332031 -6.13259124755859\\
82.5279541015625 -6.22989177703857\\
83.5347366333008 -6.32651662826538\\
84.5347366333008 -6.42251062393188\\
85.5347366333008 -6.51837635040283\\
86.5347366333008 -6.6148362159729\\
87.5347366333008 -6.71155309677124\\
88.5347366333008 -6.80806827545166\\
89.5347366333008 -6.90390729904175\\
90.5347366333008 -6.999755859375\\
91.5457077026367 -7.09661531448364\\
92.5801544189453 -7.19539308547974\\
93.6313858032227 -7.29456472396851\\
94.6669616699219 -7.38760852813721\\
95.6824951171875 -7.47358226776123\\
96.7062530517578 -7.5568904876709\\
97.7426071166992 -7.63897657394409\\
98.7648620605469 -7.71809530258179\\
99.7931747436523 -7.79650592803955\\
100.079086303711 -7.81750583648682\\
100.179092407227 -7.82474803924561\\
100.279090881348 -7.83205461502075\\
100.379089355469 -7.83944129943848\\
100.47908782959 -7.846923828125\\
100.57852935791 -7.85445308685303\\
100.678527832031 -7.86202526092529\\
100.778526306152 -7.8695240020752\\
100.878532409668 -7.87685775756836\\
100.978530883789 -7.88404035568237\\
101.07852935791 -7.89115142822266\\
101.178527832031 -7.89824295043945\\
101.278526306152 -7.90537023544312\\
101.378532409668 -7.91256141662598\\
101.478530883789 -7.91983604431152\\
101.57852935791 -7.92719888687134\\
101.678527832031 -7.93457651138306\\
101.778526306152 -7.94190692901611\\
101.878532409668 -7.9490795135498\\
101.978530883789 -7.95613670349121\\
102.07852935791 -7.96312475204468\\
102.178527832031 -7.97008752822876\\
102.278526306152 -7.97707080841064\\
102.378532409668 -7.98411273956299\\
102.478530883789 -7.99123287200928\\
102.57852935791 -7.99844551086426\\
102.678527832031 -8.00569534301758\\
102.778526306152 -8.01289367675781\\
102.878532409668 -8.01995658874512\\
102.978530883789 -8.02690601348877\\
103.07852935791 -8.03378391265869\\
103.178527832031 -8.04062271118164\\
103.278526306152 -8.04747772216797\\
103.378532409668 -8.05437088012695\\
103.478530883789 -8.06134510040283\\
103.57852935791 -8.06842708587646\\
103.678527832031 -8.07555294036865\\
103.778526306152 -8.08264350891113\\
103.878532409668 -8.0895881652832\\
103.978530883789 -8.09641456604004\\
104.07852935791 -8.10317039489746\\
104.178527832031 -8.10988426208496\\
104.278526306152 -8.11660099029541\\
104.378532409668 -8.12335872650146\\
104.478530883789 -8.13018989562988\\
104.57852935791 -8.13711452484131\\
104.678527832031 -8.14411449432373\\
104.778526306152 -8.15107154846191\\
104.878532409668 -8.15793323516846\\
104.978530883789 -8.16468524932861\\
105.07852935791 -8.17135906219482\\
105.178527832031 -8.17798805236816\\
105.278526306152 -8.18461418151855\\
105.378532409668 -8.19125080108643\\
105.478530883789 -8.19791603088379\\
105.57852935791 -8.20464992523193\\
105.678527832031 -8.21147727966309\\
105.778526306152 -8.21833419799805\\
105.878532409668 -8.22513866424561\\
105.978530883789 -8.23186492919922\\
106.07852935791 -8.23851299285889\\
106.178527832031 -8.24510955810547\\
106.278526306152 -8.25168037414551\\
106.378532409668 -8.25822734832764\\
106.478530883789 -8.26479244232178\\
106.57852935791 -8.27141189575195\\
106.678527832031 -8.27810764312744\\
106.778526306152 -8.28485202789307\\
106.878532409668 -8.29160690307617\\
106.978530883789 -8.29831695556641\\
107.07852935791 -8.30495929718018\\
107.178527832031 -8.31153964996338\\
107.278526306152 -8.31805610656738\\
107.378532409668 -8.32454776763916\\
107.478530883789 -8.33104801177979\\
107.57852935791 -8.33758163452148\\
107.678527832031 -8.34418296813965\\
107.778526306152 -8.35085201263428\\
107.878532409668 -8.35753536224365\\
107.978530883789 -8.36418151855469\\
108.07852935791 -8.37076473236084\\
108.178527832031 -8.37727928161621\\
108.278526306152 -8.38372707366943\\
108.378532409668 -8.3901309967041\\
108.478530883789 -8.39653301239014\\
108.57852935791 -8.40296649932861\\
108.678527832031 -8.40945243835449\\
108.778526306152 -8.41601085662842\\
108.878532409668 -8.42258358001709\\
108.978530883789 -8.42912483215332\\
109.07852935791 -8.43560409545898\\
109.178527832031 -8.44201755523682\\
109.278526306152 -8.44837188720703\\
109.378532409668 -8.45469284057617\\
109.478530883789 -8.46099853515625\\
109.57852935791 -8.46734809875488\\
109.678527832031 -8.47376441955566\\
109.778526306152 -8.48023891448975\\
109.878532409668 -8.48673057556152\\
109.978530883789 -8.4932107925415\\
110.07852935791 -8.49963188171387\\
110.178527832031 -8.5059986114502\\
110.278526306152 -8.51232624053955\\
110.378532409668 -8.51864051818848\\
110.478530883789 -8.52492046356201\\
110.57852935791 -8.53114032745361\\
110.678527832031 -8.5373649597168\\
110.778526306152 -8.54363822937012\\
110.878532409668 -8.54997539520264\\
110.978530883789 -8.55632495880127\\
111.07852935791 -8.56262111663818\\
111.178527832031 -8.56881999969482\\
111.278526306152 -8.57495880126953\\
111.378532409668 -8.5810432434082\\
111.478530883789 -8.58708095550537\\
111.57852935791 -8.59307193756104\\
111.678527832031 -8.59904766082764\\
111.778526306152 -8.60501384735107\\
111.878532409668 -8.61100196838379\\
111.978530883789 -8.61698246002197\\
112.07852935791 -8.62294673919678\\
112.178527832031 -8.6289005279541\\
112.278526306152 -8.63483619689941\\
112.378532409668 -8.6407356262207\\
112.478530883789 -8.64661312103271\\
112.57852935791 -8.65244674682617\\
112.678527832031 -8.65824317932129\\
112.778526306152 -8.663987159729\\
112.878532409668 -8.66970539093018\\
112.978530883789 -8.67541790008545\\
113.07852935791 -8.68113136291504\\
113.178527832031 -8.68684768676758\\
113.278526306152 -8.69255447387695\\
113.378532409668 -8.69823455810547\\
113.478530883789 -8.70387268066406\\
113.57852935791 -8.70947265625\\
113.678527832031 -8.71502494812012\\
113.778526306152 -8.7205171585083\\
113.878532409668 -8.72598171234131\\
113.978530883789 -8.73145961761475\\
114.07852935791 -8.73695468902588\\
114.178527832031 -8.74246788024902\\
114.278526306152 -8.74800205230713\\
114.378532409668 -8.75353908538818\\
114.478530883789 -8.75905990600586\\
114.57852935791 -8.76454925537109\\
114.678527832031 -8.77001762390137\\
114.778526306152 -8.7754373550415\\
114.878532409668 -8.78080940246582\\
114.978530883789 -8.78613662719727\\
115.07852935791 -8.79143047332764\\
115.178527832031 -8.79672527313232\\
115.278526306152 -8.80203151702881\\
115.378532409668 -8.80735206604004\\
115.478530883789 -8.81268787384033\\
115.57852935791 -8.81802654266357\\
115.678527832031 -8.82334136962891\\
115.778526306152 -8.8286190032959\\
115.878532409668 -8.83385467529297\\
115.978530883789 -8.83904457092285\\
116.07852935791 -8.84419345855713\\
116.178527832031 -8.84930992126465\\
116.278526306152 -8.85442066192627\\
116.378532409668 -8.85953330993652\\
116.478530883789 -8.86465835571289\\
116.57852935791 -8.86982345581055\\
116.678527832031 -8.87501430511475\\
116.778526306152 -8.88018703460693\\
116.878532409668 -8.88533210754395\\
116.978530883789 -8.89045143127441\\
117.07852935791 -8.89553260803223\\
117.178527832031 -8.90055465698242\\
117.278526306152 -8.90553665161133\\
117.378532409668 -8.91048717498779\\
117.478530883789 -8.91543388366699\\
117.57852935791 -8.92037773132324\\
117.678527832031 -8.9253396987915\\
117.778526306152 -8.93033695220947\\
117.878532409668 -8.93534851074219\\
117.978530883789 -8.94034671783447\\
118.07852935791 -8.94532012939453\\
118.178527832031 -8.95026779174805\\
118.278526306152 -8.95516872406006\\
118.378532409668 -8.96000480651855\\
118.478530883789 -8.96479225158691\\
118.57852935791 -8.96954250335693\\
118.678527832031 -8.97427749633789\\
118.778526306152 -8.97901916503906\\
118.878532409668 -8.98378562927246\\
118.978530883789 -8.98857879638672\\
119.07852935791 -8.99337291717529\\
119.178527832031 -8.99815082550049\\
119.278526306152 -9.00289154052734\\
119.378532409668 -9.00757884979248\\
119.478530883789 -9.01219654083252\\
119.57852935791 -9.01675891876221\\
119.678527832031 -9.02130126953125\\
119.778526306152 -9.02582836151123\\
119.878532409668 -9.03035640716553\\
119.978530883789 -9.03490924835205\\
120.07852935791 -9.03948974609375\\
120.178527832031 -9.0440731048584\\
120.278526306152 -9.04862976074219\\
120.378532409668 -9.05314254760742\\
120.478530883789 -9.05760478973389\\
120.57852935791 -9.06202411651611\\
120.678527832031 -9.06641101837158\\
120.778526306152 -9.07074546813965\\
120.878532409668 -9.07503128051758\\
120.978530883789 -9.07930564880371\\
121.07852935791 -9.08359241485596\\
121.178527832031 -9.08791542053223\\
121.278526306152 -9.09226322174072\\
121.378532409668 -9.09659099578857\\
121.478530883789 -9.1008825302124\\
121.57852935791 -9.10511112213135\\
121.678527832031 -9.10926151275635\\
121.778526306152 -9.11336421966553\\
121.878532409668 -9.11742305755615\\
121.978530883789 -9.12147235870361\\
122.07852935791 -9.1255578994751\\
122.178527832031 -9.12963676452637\\
122.278526306152 -9.13373470306396\\
122.378532409668 -9.13783931732178\\
122.478530883789 -9.14194011688232\\
122.57852935791 -9.14600467681885\\
122.678527832031 -9.15002822875977\\
122.778526306152 -9.15400695800781\\
122.878532409668 -9.1579704284668\\
122.978530883789 -9.16195297241211\\
123.07852935791 -9.16592788696289\\
123.178527832031 -9.16989898681641\\
123.278526306152 -9.17387294769287\\
123.378532409668 -9.17784404754639\\
123.478530883789 -9.18181991577148\\
123.57852935791 -9.1857852935791\\
123.678527832031 -9.18972015380859\\
123.778526306152 -9.19362449645996\\
123.878532409668 -9.19749927520752\\
123.978530883789 -9.20133209228516\\
124.07852935791 -9.20513343811035\\
124.178527832031 -9.20889568328857\\
124.278526306152 -9.21263027191162\\
124.378532409668 -9.21635913848877\\
124.478530883789 -9.22000694274902\\
124.57852935791 -9.22358131408691\\
124.678527832031 -9.22709274291992\\
124.778526306152 -9.23057079315186\\
124.878532409668 -9.23404979705811\\
124.978530883789 -9.23754215240479\\
125.07852935791 -9.24098682403564\\
125.178527832031 -9.24435901641846\\
125.278526306152 -9.24765014648438\\
125.378532409668 -9.25085926055908\\
125.478530883789 -9.25402450561523\\
125.57852935791 -9.25716686248779\\
125.678527832031 -9.26028060913086\\
125.778526306152 -9.26337146759033\\
125.878532409668 -9.26642990112305\\
125.978530883789 -9.26946640014648\\
126.07852935791 -9.2724552154541\\
126.178527832031 -9.27536296844482\\
126.278526306152 -9.27820014953613\\
126.378532409668 -9.28098773956299\\
126.478530883789 -9.28374099731445\\
126.57852935791 -9.28649139404297\\
126.678527832031 -9.28924179077148\\
126.778526306152 -9.29197406768799\\
126.878532409668 -9.29467391967773\\
126.972305297852 -9.29714584350586\\
127.047981262207 -9.29910182952881\\
127.147178649902 -9.30161380767822\\
127.211181640625 -9.30320453643799\\
127.289909362793 -9.30514717102051\\
127.37060546875 -9.30715274810791\\
127.470603942871 -9.30963802337646\\
127.570610046387 -9.3120641708374\\
127.670608520508 -9.3144645690918\\
127.770606994629 -9.31684684753418\\
127.87060546875 -9.31921005249023\\
127.970603942871 -9.32155799865723\\
128.070602416992 -9.32389640808105\\
128.170608520508 -9.32624626159668\\
128.269058227539 -9.3285083770752\\
128.35578918457 -9.33045196533203\\
128.437850952148 -9.33225059509277\\
128.520446777344 -9.33404350280762\\
128.608612060547 -9.33594512939453\\
128.707809448242 -9.33809375762939\\
128.807800292969 -9.34027671813965\\
128.907806396484 -9.34247970581055\\
129.0078125 -9.34468078613281\\
129.107803344727 -9.34689140319824\\
129.207809448242 -9.34915733337402\\
129.30696105957 -9.35129547119141\\
129.404495239258 -9.35310554504395\\
129.462509155273 -9.35416126251221\\
129.547668457031 -9.35572528839111\\
129.647674560547 -9.35765838623047\\
129.747665405273 -9.35972785949707\\
129.847671508789 -9.36192607879639\\
129.947677612305 -9.36407947540283\\
130.047668457031 -9.3660945892334\\
130.141616821289 -9.36786365509033\\
130.230224609375 -9.36944580078125\\
130.312683105469 -9.37089729309082\\
130.412170410156 -9.3726978302002\\
130.512176513672 -9.3746395111084\\
130.612182617188 -9.37668418884277\\
130.712173461914 -9.3787260055542\\
130.81217956543 -9.38068199157715\\
130.912170410156 -9.38253974914551\\
131.012176513672 -9.38434219360352\\
131.112182617188 -9.38614654541016\\
131.212173461914 -9.38799571990967\\
131.31217956543 -9.38993644714355\\
131.412170410156 -9.39192390441895\\
131.512176513672 -9.3940486907959\\
131.612182617188 -9.39631938934326\\
131.712173461914 -9.39852905273438\\
131.81217956543 -9.4060230255127\\
131.902801513672 -9.53136730194092\\
131.95573425293 -9.59810829162598\\
131.989608764648 -9.62932300567627\\
132.028182983398 -9.65642356872559\\
132.073120117188 -9.68236064910889\\
132.135833740234 -9.74444961547852\\
132.169692993164 -9.85078144073486\\
132.191146850586 -9.90583992004395\\
132.211013793945 -9.9367208480835\\
132.226058959961 -9.94288730621338\\
132.265762329102 -9.95108985900879\\
132.348342895508 -9.96490097045898\\
132.370986938477 -9.96875190734863\\
132.402877807617 -9.97582721710205\\
132.456558227539 -10.1563739776611\\
132.521469116211 -10.2632255554199\\
132.620666503906 -10.353590965271\\
132.662750244141 -10.4463510513306\\
132.735778808594 -10.5726528167725\\
132.818984985352 -10.6500177383423\\
132.886764526367 -10.7388515472412\\
132.951278686523 -10.7416343688965\\
133.004486083984 -10.7870101928711\\
133.027770996094 -10.8576240539551\\
133.053298950195 -10.912202835083\\
133.098373413086 -10.939977645874\\
133.154907226563 -10.9437284469604\\
133.19775390625 -10.9451961517334\\
133.247360229492 -10.9462442398071\\
133.321166992188 -10.9479646682739\\
133.421173095703 -10.9501037597656\\
133.521179199219 -10.9525327682495\\
133.621170043945 -10.9540119171143\\
133.721176147461 -10.9577798843384\\
133.821166992188 -10.9621419906616\\
133.921173095703 -10.964394569397\\
134.021179199219 -10.9770631790161\\
134.072509765625 -11.0444936752319\\
134.109161376953 -11.0618877410889\\
134.157302856445 -11.0682058334351\\
134.187774658203 -11.0697641372681\\
134.196990966797 -11.0700817108154\\
134.205642700195 -11.0704717636108\\
134.214950561523 -11.0709285736084\\
134.225479125977 -11.0714416503906\\
134.238174438477 -11.0720386505127\\
134.254135131836 -11.0726909637451\\
134.275527954102 -11.0733203887939\\
134.316802978516 -11.0737752914429\\
134.394195556641 -11.0738039016724\\
134.450088500977 -11.0743026733398\\
134.4951171875 -11.0748834609985\\
134.535583496094 -11.0755043029785\\
134.577865600586 -11.0760898590088\\
134.627151489258 -11.0765800476074\\
134.678466796875 -11.0769186019897\\
134.731018066406 -11.0772247314453\\
134.783065795898 -11.0776281356812\\
134.823043823242 -11.0780172348022\\
134.847274780273 -11.0782623291016\\
134.890426635742 -11.0787162780762\\
134.974411010742 -11.0796680450439\\
135.030136108398 -11.0803279876709\\
135.086151123047 -11.0810298919678\\
135.142272949219 -11.0816488265991\\
135.198486328125 -11.0822372436523\\
135.254745483398 -11.0828084945679\\
135.310913085938 -11.08336353302\\
135.366729736328 -11.0838937759399\\
135.422088623047 -11.0843982696533\\
135.477066040039 -11.084885597229\\
135.531814575195 -11.0853700637817\\
135.586502075195 -11.0858612060547\\
135.641265869141 -11.0863752365112\\
135.696090698242 -11.0869188308716\\
135.750961303711 -11.087495803833\\
135.805999755859 -11.0881061553955\\
135.861358642578 -11.0887517929077\\
135.916961669922 -11.0894222259521\\
135.972671508789 -11.0901384353638\\
136.028533935547 -11.0909872055054\\
136.084579467773 -11.0919513702393\\
136.14094543457 -11.09299659729\\
136.198318481445 -11.0940685272217\\
136.256469726563 -11.0951070785522\\
136.315887451172 -11.096116065979\\
136.377899169922 -11.097110748291\\
136.445541381836 -11.0981349945068\\
136.521469116211 -11.0991659164429\\
136.617645263672 -11.1003065109253\\
136.717651367188 -11.1015062332153\\
136.817642211914 -11.1027870178223\\
136.91764831543 -11.1043615341187\\
137.017654418945 -11.1064138412476\\
137.117645263672 -11.1085662841797\\
137.203964233398 -11.110577583313\\
137.299621582031 -11.1130771636963\\
137.362747192383 -11.1148986816406\\
137.436676025391 -11.1171817779541\\
137.536682128906 -11.1203918457031\\
137.636672973633 -11.123423576355\\
137.736679077148 -11.1260375976563\\
137.836685180664 -11.1286277770996\\
137.936676025391 -11.13108253479\\
138.036682128906 -11.1332921981812\\
138.136672973633 -11.1351957321167\\
138.236679077148 -11.1369161605835\\
138.336685180664 -11.1387281417847\\
138.436676025391 -11.1404638290405\\
138.536682128906 -11.141809463501\\
138.636672973633 -11.1433439254761\\
138.736679077148 -11.145411491394\\
138.836685180664 -11.1475830078125\\
138.936676025391 -11.1500816345215\\
139.036682128906 -11.1533994674683\\
139.136672973633 -11.159517288208\\
139.236679077148 -11.1657667160034\\
139.336685180664 -11.1730337142944\\
139.436676025391 -11.1804685592651\\
139.536682128906 -11.1886148452759\\
139.636672973633 -11.1970233917236\\
139.736679077148 -11.2055368423462\\
139.836685180664 -11.2138872146606\\
139.936676025391 -11.2219839096069\\
140.036682128906 -11.2297859191895\\
140.136672973633 -11.2372331619263\\
140.236679077148 -11.2443037033081\\
140.336685180664 -11.2509737014771\\
140.436676025391 -11.2572259902954\\
140.536682128906 -11.2630653381348\\
140.636672973633 -11.2685098648071\\
140.736679077148 -11.27357006073\\
140.836685180664 -11.2782497406006\\
140.936676025391 -11.282546043396\\
141.036682128906 -11.2864742279053\\
141.136672973633 -11.2900638580322\\
141.236679077148 -11.2934341430664\\
141.336685180664 -11.2967920303345\\
141.436676025391 -11.2997331619263\\
141.536682128906 -11.3023109436035\\
141.636672973633 -11.3044681549072\\
141.736679077148 -11.3059520721436\\
141.836685180664 -11.3078708648682\\
141.936676025391 -11.3099117279053\\
142.036682128906 -11.3117723464966\\
142.136672973633 -11.313775062561\\
142.236679077148 -11.3157968521118\\
142.336685180664 -11.3172521591187\\
142.436676025391 -11.3188705444336\\
142.536682128906 -11.3203296661377\\
142.636672973633 -11.3218011856079\\
142.736679077148 -11.3231716156006\\
142.836685180664 -11.3247928619385\\
142.936676025391 -11.3261518478394\\
143.036682128906 -11.3273077011108\\
143.136672973633 -11.3314962387085\\
143.236679077148 -11.3361692428589\\
143.336685180664 -11.3448162078857\\
143.436676025391 -11.354175567627\\
143.536682128906 -11.3634424209595\\
143.636672973633 -11.3720960617065\\
143.736679077148 -11.3802766799927\\
143.836685180664 -11.3893136978149\\
143.936676025391 -11.3974075317383\\
144.036682128906 -11.4031143188477\\
144.136672973633 -11.4063215255737\\
144.236679077148 -11.4117107391357\\
144.336685180664 -11.4180107116699\\
144.436676025391 -11.4248323440552\\
144.536682128906 -11.4319591522217\\
144.636672973633 -11.4386577606201\\
144.736679077148 -11.446418762207\\
144.836685180664 -11.4539089202881\\
144.936676025391 -11.4614896774292\\
145.036682128906 -11.4689979553223\\
145.136672973633 -11.476016998291\\
145.236679077148 -11.4827146530151\\
145.336685180664 -11.4891319274902\\
145.436676025391 -11.4952936172485\\
145.536682128906 -11.5010747909546\\
145.636672973633 -11.5069046020508\\
145.736679077148 -11.5121297836304\\
145.836685180664 -11.5173482894897\\
145.936676025391 -11.5221242904663\\
146.036682128906 -11.5266885757446\\
146.136672973633 -11.5309152603149\\
146.236679077148 -11.5348749160767\\
146.336685180664 -11.5385799407959\\
146.436676025391 -11.5420904159546\\
146.536682128906 -11.5454130172729\\
146.636672973633 -11.5485363006592\\
146.736679077148 -11.5514574050903\\
146.836685180664 -11.5541839599609\\
146.936676025391 -11.5567541122437\\
147.036682128906 -11.5592164993286\\
147.136672973633 -11.5615949630737\\
147.236679077148 -11.5638818740845\\
147.336685180664 -11.5660581588745\\
147.436676025391 -11.5681123733521\\
147.536682128906 -11.5700492858887\\
147.636672973633 -11.5718746185303\\
147.736679077148 -11.5735836029053\\
147.836685180664 -11.5751676559448\\
147.936676025391 -11.5766324996948\\
148.036682128906 -11.5779905319214\\
148.136672973633 -11.5792512893677\\
148.236679077148 -11.5804376602173\\
148.336685180664 -11.5815601348877\\
148.436676025391 -11.5826196670532\\
148.536682128906 -11.5836248397827\\
148.636672973633 -11.584602355957\\
148.736679077148 -11.5855684280396\\
148.836685180664 -11.5865135192871\\
148.936676025391 -11.5874032974243\\
149.036682128906 -11.5882253646851\\
149.136672973633 -11.5889825820923\\
149.236679077148 -11.5896968841553\\
149.336685180664 -11.5903806686401\\
149.436676025391 -11.5910358428955\\
149.536682128906 -11.5916700363159\\
149.636672973633 -11.5922689437866\\
149.736679077148 -11.5928239822388\\
149.836685180664 -11.5934047698975\\
149.936676025391 -11.5940227508545\\
};
\addlegendentry{$\dtmax{} = \expneg{1.0}{2}{[s]}$};

\addplot [
color=green!50!black,
solid
]
table[row sep=crcr]{
0 0\\
1.00115954875946 0\\
2.00115942955017 0\\
3.00115942955017 0\\
4.00115966796875 0\\
5.00115966796875 0\\
6.00115966796875 0\\
7.00115966796875 0\\
8.00115966796875 0\\
9.00115966796875 0\\
10.0011596679688 0\\
11.0011596679688 0\\
12.0011596679688 0\\
13.0011596679688 0\\
14.0011596679688 0\\
15.0011596679688 0\\
16.0011596679688 0\\
17.0011596679688 0\\
18.0011596679688 0\\
19.0011596679688 0\\
20.0011596679688 -1.39027505952072e-07\\
21.0175743103027 -0.554076135158539\\
22.0266017913818 -0.654182016849518\\
23.0407810211182 -0.689319968223572\\
24.0549068450928 -0.713408827781677\\
25.0701732635498 -0.734689116477966\\
26.0701732635498 -0.758184492588043\\
27.0714912414551 -0.788697898387909\\
28.0714912414551 -0.827654302120209\\
29.0714912414551 -0.878591477870941\\
30.0714912414551 -0.945153474807739\\
31.0714912414551 -1.03309631347656\\
32.0714912414551 -1.14432573318481\\
33.0714912414551 -1.26886379718781\\
34.0714912414551 -1.4009827375412\\
35.0014915466309 -1.52410125732422\\
36.0014915466309 -1.6575585603714\\
37.0014915466309 -1.78615081310272\\
38.0014915466309 -1.91073358058929\\
39.0014915466309 -2.03463625907898\\
40.0014915466309 -2.15955448150635\\
41.0014915466309 -2.28620481491089\\
42.0014915466309 -2.41280126571655\\
43.0014915466309 -2.53932166099548\\
44.0014915466309 -2.66495895385742\\
45.0014915466309 -2.79067039489746\\
46.0014915466309 -2.91756772994995\\
47.0096054077148 -3.04498600959778\\
48.0157241821289 -3.16104102134705\\
49.0255546569824 -3.27053642272949\\
50.0435066223145 -3.37452459335327\\
51.0599021911621 -3.4697949886322\\
52.0791015625 -3.56204199790955\\
53.0895309448242 -3.6514310836792\\
54.1090660095215 -3.74032783508301\\
55.121883392334 -3.82844829559326\\
56.1328926086426 -3.91549062728882\\
57.1488304138184 -4.00235652923584\\
58.1584663391113 -4.08906412124634\\
59.1728782653809 -4.17721319198608\\
60.1908111572266 -4.26868104934692\\
61.200553894043 -4.36113691329956\\
62.2218132019043 -4.4537935256958\\
63.2370910644531 -4.54758882522583\\
64.2441635131836 -4.64062309265137\\
65.2565231323242 -4.73538064956665\\
66.2757110595703 -4.83157968521118\\
67.2929611206055 -4.92879247665405\\
68.3008651733398 -5.02587127685547\\
69.3320541381836 -5.12398529052734\\
70.3084030151367 -5.21889352798462\\
71.3284072875977 -5.31419324874878\\
72.3450775146484 -5.41116428375244\\
73.357292175293 -5.5102071762085\\
74.3703765869141 -5.60612630844116\\
75.375129699707 -5.70184755325317\\
76.375129699707 -5.79851627349854\\
77.375129699707 -5.89621925354004\\
78.375129699707 -5.99500751495361\\
79.375129699707 -6.09457063674927\\
80.375129699707 -6.19589996337891\\
81.375129699707 -6.29591703414917\\
82.375129699707 -6.39772272109985\\
83.375129699707 -6.49725914001465\\
84.375129699707 -6.59745168685913\\
85.375129699707 -6.69681310653687\\
86.375129699707 -6.79631090164185\\
87.375129699707 -6.8959527015686\\
88.375129699707 -6.99495315551758\\
89.375129699707 -7.0930871963501\\
90.375129699707 -7.19097232818604\\
91.375129699707 -7.28853511810303\\
92.375129699707 -7.38569736480713\\
93.375129699707 -7.48068571090698\\
94.375129699707 -7.56949377059937\\
95.375129699707 -7.6522479057312\\
96.375129699707 -7.731041431427\\
97.375129699707 -7.80662965774536\\
98.375129699707 -7.87968778610229\\
99.375129699707 -7.95081901550293\\
100.014999389648 -7.99532890319824\\
100.065002441406 -7.99886131286621\\
100.11499786377 -8.0024242401123\\
100.165000915527 -8.0059700012207\\
100.214996337891 -8.00948524475098\\
100.264999389648 -8.01297473907471\\
100.315002441406 -8.01644229888916\\
100.36499786377 -8.0199031829834\\
100.415000915527 -8.02336597442627\\
100.464996337891 -8.02681827545166\\
100.514999389648 -8.03025531768799\\
100.565002441406 -8.03371524810791\\
100.61499786377 -8.0372200012207\\
100.665000915527 -8.04071044921875\\
100.714996337891 -8.04414558410645\\
100.764999389648 -8.04752445220947\\
100.815002441406 -8.05085277557373\\
100.86499786377 -8.05415916442871\\
100.915000915527 -8.05748462677002\\
100.964996337891 -8.06083202362061\\
101.014999389648 -8.06422519683838\\
101.065002441406 -8.06764221191406\\
101.11499786377 -8.07107830047607\\
101.165000915527 -8.07451057434082\\
101.214996337891 -8.0779275894165\\
101.264999389648 -8.08133506774902\\
101.315002441406 -8.08473587036133\\
101.36499786377 -8.08813762664795\\
101.415000915527 -8.09153366088867\\
101.464996337891 -8.09491348266602\\
101.514999389648 -8.098313331604\\
101.565002441406 -8.10173988342285\\
101.61499786377 -8.10521507263184\\
101.665000915527 -8.10867977142334\\
101.714996337891 -8.11207962036133\\
101.764999389648 -8.1154088973999\\
101.815002441406 -8.11869049072266\\
101.86499786377 -8.12193298339844\\
101.915000915527 -8.12515926361084\\
101.964996337891 -8.1284008026123\\
102.014999389648 -8.13167953491211\\
102.065002441406 -8.13498592376709\\
102.11499786377 -8.1383056640625\\
102.165000915527 -8.14163398742676\\
102.214996337891 -8.1449670791626\\
102.264999389648 -8.1483097076416\\
102.315002441406 -8.15165328979492\\
102.36499786377 -8.15499496459961\\
102.415000915527 -8.15833187103271\\
102.464996337891 -8.1616849899292\\
102.514999389648 -8.16505813598633\\
102.565002441406 -8.16841697692871\\
102.61499786377 -8.17182350158691\\
102.665000915527 -8.17522144317627\\
102.714996337891 -8.1785717010498\\
102.764999389648 -8.18186187744141\\
102.815002441406 -8.18509578704834\\
102.86499786377 -8.18828678131104\\
102.915000915527 -8.1914587020874\\
102.964996337891 -8.19463634490967\\
103.014999389648 -8.19783782958984\\
103.065002441406 -8.20106792449951\\
103.11499786377 -8.20431900024414\\
103.165000915527 -8.20758628845215\\
103.214996337891 -8.21086692810059\\
103.264999389648 -8.21416473388672\\
103.315002441406 -8.21747589111328\\
103.36499786377 -8.22080135345459\\
103.415000915527 -8.22413730621338\\
103.464996337891 -8.2274694442749\\
103.514999389648 -8.23079299926758\\
103.565002441406 -8.23415565490723\\
103.61499786377 -8.23752021789551\\
103.665000915527 -8.24085903167725\\
103.714996337891 -8.24415874481201\\
103.764999389648 -8.24740409851074\\
103.815002441406 -8.2506046295166\\
103.86499786377 -8.25377178192139\\
103.915000915527 -8.256911277771\\
103.964996337891 -8.26003932952881\\
104.014999389648 -8.26317119598389\\
104.065002441406 -8.26631832122803\\
104.11499786377 -8.26948642730713\\
104.165000915527 -8.27267169952393\\
104.214996337891 -8.27587223052979\\
104.264999389648 -8.27907848358154\\
104.315002441406 -8.2822961807251\\
104.36499786377 -8.28555202484131\\
104.415000915527 -8.28883743286133\\
104.464996337891 -8.29213428497314\\
104.514999389648 -8.29542827606201\\
104.565002441406 -8.29872417449951\\
104.61499786377 -8.30200386047363\\
104.665000915527 -8.30527496337891\\
104.714996337891 -8.30852127075195\\
104.764999389648 -8.31173229217529\\
104.815002441406 -8.31490612030029\\
104.86499786377 -8.31804847717285\\
104.915000915527 -8.32116317749023\\
104.964996337891 -8.32426071166992\\
105.014999389648 -8.32735252380371\\
105.065002441406 -8.33045196533203\\
105.11499786377 -8.33355808258057\\
105.165000915527 -8.336669921875\\
105.214996337891 -8.33979225158691\\
105.264999389648 -8.34292793273926\\
105.315002441406 -8.34608173370361\\
105.36499786377 -8.34925842285156\\
105.415000915527 -8.35246276855469\\
105.464996337891 -8.35568618774414\\
105.514999389648 -8.35891914367676\\
105.565002441406 -8.36215114593506\\
105.61499786377 -8.36538696289063\\
105.665000915527 -8.36862277984619\\
105.714996337891 -8.37184429168701\\
105.764999389648 -8.37504291534424\\
105.815002441406 -8.37820625305176\\
105.86499786377 -8.38134098052979\\
105.915000915527 -8.38445472717285\\
105.964996337891 -8.38754653930664\\
106.014999389648 -8.3906192779541\\
106.065002441406 -8.39369010925293\\
106.11499786377 -8.39676380157471\\
106.165000915527 -8.3998384475708\\
106.214996337891 -8.40291690826416\\
106.264999389648 -8.40600109100342\\
106.315002441406 -8.40908813476563\\
106.36499786377 -8.41218566894531\\
106.415000915527 -8.41530132293701\\
106.464996337891 -8.41843128204346\\
106.514999389648 -8.4215841293335\\
106.565002441406 -8.42475700378418\\
106.61499786377 -8.42794132232666\\
106.665000915527 -8.4311351776123\\
106.714996337891 -8.4343376159668\\
106.764999389648 -8.43752861022949\\
106.815002441406 -8.44069671630859\\
106.86499786377 -8.44383716583252\\
106.915000915527 -8.44695281982422\\
106.964996337891 -8.45005130767822\\
107.014999389648 -8.45312976837158\\
107.065002441406 -8.45618724822998\\
107.11499786377 -8.45923519134521\\
107.165000915527 -8.46227359771729\\
107.214996337891 -8.46530437469482\\
107.264999389648 -8.46832275390625\\
107.315002441406 -8.47134590148926\\
107.36499786377 -8.47437286376953\\
107.415000915527 -8.4774112701416\\
107.464996337891 -8.48046684265137\\
107.514999389648 -8.4835376739502\\
107.565002441406 -8.4866247177124\\
107.61499786377 -8.48973846435547\\
107.665000915527 -8.49287414550781\\
107.714996337891 -8.49601364135742\\
107.764999389648 -8.49915504455566\\
107.815002441406 -8.50228500366211\\
107.86499786377 -8.50540161132813\\
107.915000915527 -8.50850105285645\\
107.964996337891 -8.51157474517822\\
108.014999389648 -8.51462745666504\\
108.065002441406 -8.51766014099121\\
108.11499786377 -8.52067279815674\\
108.165000915527 -8.52366828918457\\
108.214996337891 -8.52665042877197\\
108.264999389648 -8.52962684631348\\
108.315002441406 -8.53260326385498\\
108.36499786377 -8.53558349609375\\
108.415000915527 -8.53857421875\\
108.464996337891 -8.54158020019531\\
108.514999389648 -8.54460906982422\\
108.565002441406 -8.54765033721924\\
108.61499786377 -8.5507173538208\\
108.665000915527 -8.55381488800049\\
108.714996337891 -8.55693054199219\\
108.764999389648 -8.56005859375\\
108.815002441406 -8.56318187713623\\
108.86499786377 -8.56629276275635\\
108.915000915527 -8.56937599182129\\
108.964996337891 -8.572434425354\\
109.014999389648 -8.57546997070313\\
109.065002441406 -8.57848167419434\\
109.11499786377 -8.58147048950195\\
109.165000915527 -8.58444023132324\\
109.214996337891 -8.58739280700684\\
109.264999389648 -8.59033298492432\\
109.315002441406 -8.59326267242432\\
109.36499786377 -8.59618663787842\\
109.415000915527 -8.59911251068115\\
109.464996337891 -8.60204696655273\\
109.514999389648 -8.60499286651611\\
109.565002441406 -8.6079568862915\\
109.61499786377 -8.61093902587891\\
109.665000915527 -8.61394500732422\\
109.714996337891 -8.61697292327881\\
109.764999389648 -8.62001419067383\\
109.815002441406 -8.62305450439453\\
109.86499786377 -8.62608242034912\\
109.915000915527 -8.62909889221191\\
109.964996337891 -8.63208675384521\\
110.014999389648 -8.63505363464355\\
110.065002441406 -8.63800621032715\\
110.11499786377 -8.64094638824463\\
110.165000915527 -8.64388084411621\\
110.214996337891 -8.64680290222168\\
110.264999389648 -8.64971542358398\\
110.315002441406 -8.65261554718018\\
110.36499786377 -8.65550804138184\\
110.415000915527 -8.65839862823486\\
110.464996337891 -8.66129779815674\\
110.514999389648 -8.66421031951904\\
110.565002441406 -8.66711521148682\\
110.61499786377 -8.67001438140869\\
110.665000915527 -8.6729154586792\\
110.714996337891 -8.67582035064697\\
110.764999389648 -8.67873287200928\\
110.815002441406 -8.68165969848633\\
110.86499786377 -8.68459892272949\\
110.915000915527 -8.68754863739014\\
110.964996337891 -8.69049549102783\\
111.014999389648 -8.69343757629395\\
111.065002441406 -8.69637012481689\\
111.11499786377 -8.69928550720215\\
111.165000915527 -8.70219421386719\\
111.214996337891 -8.70509243011475\\
111.264999389648 -8.70798397064209\\
111.315002441406 -8.71086597442627\\
111.36499786377 -8.7137336730957\\
111.415000915527 -8.71659564971924\\
111.464996337891 -8.71945667266846\\
111.514999389648 -8.72231101989746\\
111.565002441406 -8.72515678405762\\
111.61499786377 -8.72800159454346\\
111.665000915527 -8.73084163665771\\
111.714996337891 -8.73367595672607\\
111.764999389648 -8.73650932312012\\
111.815002441406 -8.73934936523438\\
111.86499786377 -8.74219417572021\\
111.915000915527 -8.74504375457764\\
111.964996337891 -8.74789810180664\\
112.014999389648 -8.75074863433838\\
112.065002441406 -8.7535924911499\\
112.11499786377 -8.75642967224121\\
112.165000915527 -8.75926876068115\\
112.214996337891 -8.76210880279541\\
112.264999389648 -8.76494026184082\\
112.315002441406 -8.7677640914917\\
112.36499786377 -8.77058219909668\\
112.415000915527 -8.77339172363281\\
112.464996337891 -8.77619457244873\\
112.514999389648 -8.77899265289307\\
112.565002441406 -8.78178024291992\\
112.61499786377 -8.78455829620361\\
112.665000915527 -8.78732681274414\\
112.714996337891 -8.79008197784424\\
112.764999389648 -8.79283142089844\\
112.815002441406 -8.79557991027832\\
112.86499786377 -8.79832744598389\\
112.915000915527 -8.80107498168945\\
112.964996337891 -8.80383014678955\\
113.014999389648 -8.80659294128418\\
113.065002441406 -8.80935478210449\\
113.11499786377 -8.81212139129639\\
113.165000915527 -8.81489276885986\\
113.214996337891 -8.81766414642334\\
113.264999389648 -8.82042503356934\\
113.315002441406 -8.82318687438965\\
113.36499786377 -8.82594394683838\\
113.415000915527 -8.82869434356689\\
113.464996337891 -8.83143043518066\\
113.514999389648 -8.83416175842285\\
113.565002441406 -8.83688926696777\\
113.61499786377 -8.83960723876953\\
113.665000915527 -8.84231281280518\\
113.714996337891 -8.84500312805176\\
113.764999389648 -8.84767532348633\\
113.815002441406 -8.85033416748047\\
113.86499786377 -8.85298347473145\\
113.915000915527 -8.85562610626221\\
113.964996337891 -8.85827159881592\\
114.014999389648 -8.86091804504395\\
114.065002441406 -8.86357402801514\\
114.11499786377 -8.86624526977539\\
114.165000915527 -8.86892509460449\\
114.214996337891 -8.87160491943359\\
114.264999389648 -8.87429237365723\\
114.315002441406 -8.87698173522949\\
114.36499786377 -8.87967872619629\\
114.415000915527 -8.8823766708374\\
114.464996337891 -8.8850679397583\\
114.514999389648 -8.8877477645874\\
114.565002441406 -8.89041805267334\\
114.61499786377 -8.8930835723877\\
114.665000915527 -8.89573955535889\\
114.714996337891 -8.8983850479126\\
114.764999389648 -8.90102100372314\\
114.815002441406 -8.90364265441895\\
114.86499786377 -8.90625286102295\\
114.915000915527 -8.90884876251221\\
114.964996337891 -8.91142749786377\\
115.014999389648 -8.91399669647217\\
115.065002441406 -8.91655826568604\\
115.11499786377 -8.9191198348999\\
115.165000915527 -8.92167949676514\\
115.214996337891 -8.92424583435059\\
115.264999389648 -8.92681312561035\\
115.315002441406 -8.92938709259033\\
115.36499786377 -8.93197154998779\\
115.415000915527 -8.9345588684082\\
115.464996337891 -8.93715190887451\\
115.514999389648 -8.93974876403809\\
115.565002441406 -8.94234275817871\\
115.61499786377 -8.94493198394775\\
115.665000915527 -8.9475154876709\\
115.714996337891 -8.95008945465088\\
115.764999389648 -8.95265960693359\\
115.815002441406 -8.95521926879883\\
115.86499786377 -8.95776271820068\\
115.915000915527 -8.96029567718506\\
115.964996337891 -8.96281147003174\\
116.014999389648 -8.9653148651123\\
116.065002441406 -8.96780490875244\\
116.11499786377 -8.97028064727783\\
116.165000915527 -8.97274684906006\\
116.214996337891 -8.9752082824707\\
116.264999389648 -8.97766780853271\\
116.315002441406 -8.98013114929199\\
116.36499786377 -8.9826021194458\\
116.415000915527 -8.98507976531982\\
116.464996337891 -8.9875659942627\\
116.514999389648 -8.990065574646\\
116.565002441406 -8.99257564544678\\
116.61499786377 -8.99509334564209\\
116.665000915527 -8.9976110458374\\
116.714996337891 -9.00013160705566\\
116.764999389648 -9.00265216827393\\
116.815002441406 -9.00516605377197\\
116.86499786377 -9.00766754150391\\
116.915000915527 -9.01015472412109\\
116.964996337891 -9.01263046264648\\
117.014999389648 -9.01509475708008\\
117.065002441406 -9.01754951477051\\
117.11499786377 -9.01998996734619\\
117.165000915527 -9.02241706848145\\
117.214996337891 -9.02483081817627\\
117.264999389648 -9.02723598480225\\
117.315002441406 -9.02963352203369\\
117.36499786377 -9.03202724456787\\
117.415000915527 -9.03442096710205\\
117.464996337891 -9.03680896759033\\
117.514999389648 -9.03920269012451\\
117.565002441406 -9.04160404205322\\
117.61499786377 -9.04400825500488\\
117.665000915527 -9.04642391204834\\
117.714996337891 -9.04885005950928\\
117.764999389648 -9.0512866973877\\
117.815002441406 -9.05372905731201\\
117.86499786377 -9.05617523193359\\
117.915000915527 -9.05861854553223\\
117.964996337891 -9.06105613708496\\
118.014999389648 -9.06348609924316\\
118.065002441406 -9.06590557098389\\
118.11499786377 -9.06831169128418\\
118.165000915527 -9.07070636749268\\
118.214996337891 -9.07309341430664\\
118.264999389648 -9.07547187805176\\
118.315002441406 -9.07783794403076\\
118.36499786377 -9.08018779754639\\
118.415000915527 -9.08252239227295\\
118.464996337891 -9.08484268188477\\
118.514999389648 -9.08715629577637\\
118.565002441406 -9.08946418762207\\
118.61499786377 -9.09177303314209\\
118.665000915527 -9.09408092498779\\
118.714996337891 -9.09639549255371\\
118.764999389648 -9.09871578216553\\
118.815002441406 -9.10104370117188\\
118.86499786377 -9.10338401794434\\
118.915000915527 -9.10573577880859\\
118.964996337891 -9.10809898376465\\
119.014999389648 -9.11046695709229\\
119.065002441406 -9.11283206939697\\
119.11499786377 -9.1151876449585\\
119.165000915527 -9.11753559112549\\
119.214996337891 -9.11987209320068\\
119.264999389648 -9.12219429016113\\
119.315002441406 -9.12450218200684\\
119.36499786377 -9.12679290771484\\
119.415000915527 -9.12906837463379\\
119.464996337891 -9.13133335113525\\
119.514999389648 -9.13358306884766\\
119.565002441406 -9.135817527771\\
119.61499786377 -9.13804340362549\\
119.665000915527 -9.14026927947998\\
119.714996337891 -9.14249134063721\\
119.764999389648 -9.14471912384033\\
119.815002441406 -9.1469554901123\\
119.86499786377 -9.14919376373291\\
119.915000915527 -9.15144443511963\\
119.964996337891 -9.15370941162109\\
120.014999389648 -9.15598201751709\\
120.065002441406 -9.15825366973877\\
120.11499786377 -9.16052341461182\\
120.165000915527 -9.16278743743896\\
120.214996337891 -9.16504573822021\\
120.264999389648 -9.16729831695557\\
120.315002441406 -9.16954231262207\\
120.36499786377 -9.17177486419678\\
120.415000915527 -9.17399120330811\\
120.464996337891 -9.17618560791016\\
120.514999389648 -9.17836666107178\\
120.565002441406 -9.18054103851318\\
120.61499786377 -9.18270778656006\\
120.665000915527 -9.18488311767578\\
120.714996337891 -9.18705368041992\\
120.764999389648 -9.18920993804932\\
120.815002441406 -9.19134998321533\\
120.86499786377 -9.19347667694092\\
120.915000915527 -9.19559383392334\\
120.964996337891 -9.1977071762085\\
121.014999389648 -9.19982624053955\\
121.065002441406 -9.2019567489624\\
121.11499786377 -9.20409107208252\\
121.165000915527 -9.20623779296875\\
121.214996337891 -9.20839691162109\\
121.264999389648 -9.21056652069092\\
121.315002441406 -9.21273899078369\\
121.36499786377 -9.21491050720215\\
121.415000915527 -9.21706962585449\\
121.464996337891 -9.21921634674072\\
121.514999389648 -9.22134685516357\\
121.565002441406 -9.22345638275146\\
121.61499786377 -9.22554588317871\\
121.665000915527 -9.22762107849121\\
121.714996337891 -9.22968482971191\\
121.764999389648 -9.23173236846924\\
121.815002441406 -9.23376655578613\\
121.86499786377 -9.23579597473145\\
121.915000915527 -9.23781871795654\\
121.964996337891 -9.23984146118164\\
122.014999389648 -9.24187564849854\\
122.065002441406 -9.24392318725586\\
122.11499786377 -9.24596691131592\\
122.165000915527 -9.24801158905029\\
122.214996337891 -9.25006198883057\\
122.264999389648 -9.25212669372559\\
122.315002441406 -9.25419807434082\\
122.36499786377 -9.25627136230469\\
122.415000915527 -9.25834560394287\\
122.464996337891 -9.260422706604\\
122.514999389648 -9.26249408721924\\
122.565002441406 -9.26455688476563\\
122.61499786377 -9.26661396026611\\
122.665000915527 -9.26865482330322\\
122.714996337891 -9.27067756652832\\
122.764999389648 -9.27269172668457\\
122.815002441406 -9.27470302581787\\
122.86499786377 -9.27670860290527\\
122.915000915527 -9.27872085571289\\
122.964996337891 -9.28073406219482\\
123.014999389648 -9.28274345397949\\
123.065002441406 -9.28474712371826\\
123.11499786377 -9.28674030303955\\
123.165000915527 -9.28872966766357\\
123.214996337891 -9.29072284698486\\
123.264999389648 -9.29271602630615\\
123.315002441406 -9.29470920562744\\
123.36499786377 -9.29670238494873\\
123.415000915527 -9.29869556427002\\
123.464996337891 -9.30068969726563\\
123.514999389648 -9.30268096923828\\
123.565002441406 -9.30465984344482\\
123.61499786377 -9.30662441253662\\
123.665000915527 -9.30858039855957\\
123.714996337891 -9.31052112579346\\
123.764999389648 -9.31244850158691\\
123.815002441406 -9.31435680389404\\
123.86499786377 -9.31624889373779\\
123.915000915527 -9.31812286376953\\
123.964996337891 -9.31998729705811\\
124.014999389648 -9.32183837890625\\
124.065002441406 -9.3236780166626\\
124.11499786377 -9.32550048828125\\
124.165000915527 -9.32731628417969\\
124.214996337891 -9.32912635803223\\
124.264999389648 -9.33093929290771\\
124.315002441406 -9.33276176452637\\
124.36499786377 -9.33455848693848\\
124.415000915527 -9.33633613586426\\
124.464996337891 -9.33809471130371\\
124.514999389648 -9.33983516693115\\
124.565002441406 -9.34156322479248\\
124.61499786377 -9.34327983856201\\
124.665000915527 -9.34498977661133\\
124.714996337891 -9.34668731689453\\
124.764999389648 -9.3483829498291\\
124.815002441406 -9.35007095336914\\
124.86499786377 -9.35175895690918\\
124.915000915527 -9.35344409942627\\
124.964996337891 -9.35513305664063\\
125.014999389648 -9.35680198669434\\
125.065002441406 -9.35845279693604\\
125.11499786377 -9.36008262634277\\
125.165000915527 -9.36169338226318\\
125.214996337891 -9.36328887939453\\
125.264999389648 -9.36486339569092\\
125.315002441406 -9.36642169952393\\
125.36499786377 -9.36795997619629\\
125.415000915527 -9.3694953918457\\
125.464996337891 -9.37102890014648\\
125.514999389648 -9.37255573272705\\
125.565002441406 -9.3740816116333\\
125.61499786377 -9.37560081481934\\
125.665000915527 -9.37710952758789\\
125.714996337891 -9.37861442565918\\
125.764999389648 -9.38011169433594\\
125.815002441406 -9.38159370422363\\
125.86499786377 -9.38307189941406\\
125.915000915527 -9.38454914093018\\
125.964996337891 -9.38601589202881\\
126.014999389648 -9.38746643066406\\
126.065002441406 -9.38889980316162\\
126.11499786377 -9.39030933380127\\
126.165000915527 -9.39169692993164\\
126.214996337891 -9.39307117462158\\
126.264999389648 -9.39443588256836\\
126.315002441406 -9.39578342437744\\
126.36499786377 -9.39712429046631\\
126.415000915527 -9.39846229553223\\
126.464996337891 -9.39980316162109\\
126.514999389648 -9.40113830566406\\
126.565002441406 -9.40247917175293\\
126.61499786377 -9.40382385253906\\
126.665000915527 -9.40516376495361\\
126.714996337891 -9.40649890899658\\
126.764999389648 -9.40783023834229\\
126.815002441406 -9.40914344787598\\
126.86499786377 -9.41044521331787\\
126.915000915527 -9.41172790527344\\
126.964996337891 -9.41298961639404\\
127.013610839844 -9.41420078277588\\
127.058181762695 -9.41529846191406\\
127.101631164551 -9.41636180877686\\
127.145645141602 -9.4174280166626\\
127.193618774414 -9.41857719421387\\
127.243621826172 -9.41977596282959\\
127.293617248535 -9.42099094390869\\
127.343620300293 -9.42220592498779\\
127.393615722656 -9.42340278625488\\
127.443618774414 -9.42458343505859\\
127.493621826172 -9.42575645446777\\
127.543617248535 -9.42692947387695\\
127.593620300293 -9.4281005859375\\
127.643615722656 -9.4292688369751\\
127.693618774414 -9.43044090270996\\
127.743621826172 -9.43161010742188\\
127.793617248535 -9.43277359008789\\
127.843620300293 -9.43393707275391\\
127.893615722656 -9.43508720397949\\
127.943618774414 -9.43623161315918\\
127.993621826172 -9.43737602233887\\
128.04362487793 -9.43852043151855\\
128.093612670898 -9.43966484069824\\
128.143615722656 -9.4407958984375\\
128.193618774414 -9.44190788269043\\
128.243621826172 -9.44299983978271\\
128.29362487793 -9.44407749176025\\
128.343612670898 -9.44514465332031\\
128.393615722656 -9.44619846343994\\
128.443618774414 -9.44724750518799\\
128.493621826172 -9.44829940795898\\
128.54362487793 -9.44935989379883\\
128.593612670898 -9.45042991638184\\
128.643615722656 -9.45150566101074\\
128.693618774414 -9.45258903503418\\
128.743621826172 -9.45367813110352\\
128.79362487793 -9.45477294921875\\
128.843612670898 -9.45586776733398\\
128.893615722656 -9.45695972442627\\
128.943618774414 -9.45804595947266\\
128.993621826172 -9.45912456512451\\
129.04362487793 -9.46019744873047\\
129.093612670898 -9.46126842498779\\
129.143615722656 -9.46237373352051\\
129.193618774414 -9.463547706604\\
129.243621826172 -9.46472263336182\\
129.29362487793 -9.46578598022461\\
129.343612670898 -9.46674346923828\\
129.391830444336 -9.46763706207275\\
129.434173583984 -9.46841812133789\\
129.484176635742 -9.46934700012207\\
129.5341796875 -9.4702844619751\\
129.584182739258 -9.47123718261719\\
129.634170532227 -9.4722261428833\\
129.684173583984 -9.47327613830566\\
129.734176635742 -9.474365234375\\
129.7841796875 -9.47549247741699\\
129.834182739258 -9.47665596008301\\
129.884170532227 -9.47780799865723\\
129.934173583984 -9.47892570495605\\
129.984176635742 -9.47999382019043\\
130.0341796875 -9.48101711273193\\
130.084182739258 -9.48199367523193\\
130.134170532227 -9.48293590545654\\
130.184173583984 -9.48385715484619\\
130.234176635742 -9.48476791381836\\
130.2841796875 -9.48568153381348\\
130.334182739258 -9.48660278320313\\
130.384170532227 -9.48755264282227\\
130.434173583984 -9.4885425567627\\
130.484176635742 -9.48958301544189\\
130.5341796875 -9.49066638946533\\
130.584182739258 -9.49177074432373\\
130.634170532227 -9.49287223815918\\
130.684173583984 -9.49395656585693\\
130.734176635742 -9.4950122833252\\
130.7841796875 -9.49603462219238\\
130.834182739258 -9.49702453613281\\
130.884170532227 -9.49798679351807\\
130.934173583984 -9.49892807006836\\
130.984176635742 -9.49985980987549\\
131.0341796875 -9.50078868865967\\
131.084182739258 -9.5017204284668\\
131.134170532227 -9.50266647338867\\
131.184173583984 -9.50364303588867\\
131.234176635742 -9.50466060638428\\
131.2841796875 -9.50571155548096\\
131.334182739258 -9.50677585601807\\
131.384170532227 -9.50784015655518\\
131.434173583984 -9.50890731811523\\
131.484176635742 -9.51003360748291\\
131.5341796875 -9.51121139526367\\
131.584182739258 -9.51240730285645\\
131.634170532227 -9.513596534729\\
131.684173583984 -9.51477432250977\\
131.734176635742 -9.51676082611084\\
131.7841796875 -9.55588531494141\\
131.834182739258 -9.6308708190918\\
131.881576538086 -9.69806480407715\\
131.916702270508 -9.74321937561035\\
131.95475769043 -9.78479480743408\\
131.998229980469 -9.82196712493896\\
132.046524047852 -9.85073375701904\\
132.093734741211 -9.98837566375732\\
132.108963012695 -10.032172203064\\
132.132751464844 -10.0857944488525\\
132.145050048828 -10.0983362197876\\
132.164337158203 -10.1050186157227\\
132.208511352539 -10.1137800216675\\
132.258514404297 -10.1223936080933\\
132.284164428711 -10.1272325515747\\
132.320907592773 -10.1794881820679\\
132.365341186523 -10.3108282089233\\
132.389022827148 -10.3444204330444\\
132.436248779297 -10.3806867599487\\
132.486251831055 -10.4090976715088\\
132.535430908203 -10.4503612518311\\
132.567016601563 -10.5397052764893\\
132.616226196289 -10.6298885345459\\
132.666229248047 -10.7031517028809\\
132.716232299805 -10.7350025177002\\
132.766082763672 -10.8622808456421\\
132.81608581543 -10.9083242416382\\
132.86442565918 -10.9106998443604\\
132.903747558594 -10.968056678772\\
132.9326171875 -11.0441455841064\\
132.980010986328 -11.0789833068848\\
133.026809692383 -11.0818004608154\\
133.072555541992 -11.0832262039185\\
133.121200561523 -11.0838212966919\\
133.171188354492 -11.0842905044556\\
133.22119140625 -11.084716796875\\
133.271194458008 -11.0856637954712\\
133.321197509766 -11.0868253707886\\
133.371200561523 -11.0879135131836\\
133.421188354492 -11.0887126922607\\
133.47119140625 -11.0892171859741\\
133.521194458008 -11.0896835327148\\
133.571197509766 -11.090916633606\\
133.621200561523 -11.0924663543701\\
133.671188354492 -11.0942363739014\\
133.72119140625 -11.095552444458\\
133.771194458008 -11.0962743759155\\
133.821197509766 -11.0970630645752\\
133.871200561523 -11.0978212356567\\
133.921188354492 -11.0984840393066\\
133.97119140625 -11.0989141464233\\
134.021194458008 -11.0992498397827\\
134.071197509766 -11.0998373031616\\
134.121200561523 -11.1007070541382\\
134.171188354492 -11.1017217636108\\
134.22119140625 -11.1026382446289\\
134.271194458008 -11.1033353805542\\
134.321197509766 -11.1037683486938\\
134.371200561523 -11.1042604446411\\
134.421188354492 -11.1049938201904\\
134.47119140625 -11.1058540344238\\
134.521194458008 -11.1067504882813\\
134.571197509766 -11.1083335876465\\
134.621200561523 -11.1125078201294\\
134.671188354492 -11.1163377761841\\
134.72119140625 -11.117657661438\\
134.771194458008 -11.1188039779663\\
134.821197509766 -11.1202421188354\\
134.871200561523 -11.1256351470947\\
134.91423034668 -11.1741809844971\\
134.964019775391 -11.1815567016602\\
135.014022827148 -11.1835308074951\\
135.064025878906 -11.1848258972168\\
135.114028930664 -11.185863494873\\
135.164031982422 -11.1868724822998\\
135.214019775391 -11.1878442764282\\
135.264022827148 -11.1887216567993\\
135.314025878906 -11.189525604248\\
135.364028930664 -11.1904106140137\\
135.414031982422 -11.1914196014404\\
135.464019775391 -11.1924905776978\\
135.514022827148 -11.1935586929321\\
135.564025878906 -11.1945915222168\\
135.614028930664 -11.1956348419189\\
135.664031982422 -11.1967191696167\\
135.714019775391 -11.1978950500488\\
135.764022827148 -11.1996879577637\\
135.814025878906 -11.201470375061\\
135.864028930664 -11.2031717300415\\
135.914031982422 -11.2049360275269\\
135.964019775391 -11.2066764831543\\
136.014022827148 -11.2084903717041\\
136.064025878906 -11.2103424072266\\
136.112365722656 -11.211950302124\\
136.162368774414 -11.2136144638062\\
136.212356567383 -11.2154655456543\\
136.262359619141 -11.2171754837036\\
136.312362670898 -11.2190551757813\\
136.362365722656 -11.2211399078369\\
136.412368774414 -11.2228441238403\\
136.462356567383 -11.2245893478394\\
136.512359619141 -11.226900100708\\
136.562362670898 -11.2297458648682\\
136.612365722656 -11.2321701049805\\
136.662368774414 -11.2344446182251\\
136.712356567383 -11.2361373901367\\
136.762359619141 -11.2377214431763\\
136.812362670898 -11.2393283843994\\
136.862365722656 -11.2409725189209\\
136.912368774414 -11.2425737380981\\
136.962356567383 -11.2440462112427\\
137.012359619141 -11.2454891204834\\
137.062362670898 -11.2468776702881\\
137.112365722656 -11.2482748031616\\
137.162368774414 -11.2496910095215\\
137.212356567383 -11.25110912323\\
137.262359619141 -11.2525157928467\\
137.312362670898 -11.2537822723389\\
137.362365722656 -11.2549257278442\\
137.412368774414 -11.2561473846436\\
137.462356567383 -11.2574825286865\\
137.512359619141 -11.2589149475098\\
137.562362670898 -11.2599925994873\\
137.612365722656 -11.2610082626343\\
137.662368774414 -11.2621955871582\\
137.712356567383 -11.2636833190918\\
137.762359619141 -11.2653827667236\\
137.812362670898 -11.2671451568604\\
137.862365722656 -11.2690725326538\\
137.912368774414 -11.2708196640015\\
137.962356567383 -11.2724637985229\\
138.012359619141 -11.2741651535034\\
138.062362670898 -11.2760171890259\\
138.112365722656 -11.278112411499\\
138.162368774414 -11.2803411483765\\
138.212356567383 -11.281907081604\\
138.262359619141 -11.2830686569214\\
138.312362670898 -11.2841968536377\\
138.362365722656 -11.2852268218994\\
138.412368774414 -11.2861433029175\\
138.462356567383 -11.286979675293\\
138.512359619141 -11.28795337677\\
138.562362670898 -11.2889413833618\\
138.612365722656 -11.2900228500366\\
138.662368774414 -11.2913856506348\\
138.712356567383 -11.2930841445923\\
138.762359619141 -11.2950839996338\\
138.812362670898 -11.297324180603\\
138.862365722656 -11.2997903823853\\
138.912368774414 -11.3024044036865\\
138.962356567383 -11.3051958084106\\
139.012359619141 -11.3081483840942\\
139.062362670898 -11.3112344741821\\
139.112365722656 -11.3144454956055\\
139.162368774414 -11.3177528381348\\
139.212356567383 -11.321138381958\\
139.262359619141 -11.3246021270752\\
139.312362670898 -11.328106880188\\
139.362365722656 -11.3316192626953\\
139.412368774414 -11.335132598877\\
139.462356567383 -11.3386268615723\\
139.512359619141 -11.3420877456665\\
139.562362670898 -11.3455018997192\\
139.612365722656 -11.3488483428955\\
139.662368774414 -11.3521184921265\\
139.712356567383 -11.3553037643433\\
139.762359619141 -11.3584012985229\\
139.812362670898 -11.3614139556885\\
139.862365722656 -11.3643417358398\\
139.912368774414 -11.3671894073486\\
139.962356567383 -11.3699617385864\\
140.012359619141 -11.3726654052734\\
140.062362670898 -11.3753080368042\\
140.112365722656 -11.377890586853\\
140.162368774414 -11.380410194397\\
140.212356567383 -11.3828725814819\\
140.262359619141 -11.3852853775024\\
140.312362670898 -11.3876495361328\\
140.362365722656 -11.3899641036987\\
140.412368774414 -11.3922233581543\\
140.462356567383 -11.394419670105\\
140.512359619141 -11.3965482711792\\
140.562362670898 -11.3985996246338\\
140.612365722656 -11.4005613327026\\
140.662368774414 -11.4024505615234\\
140.712356567383 -11.4042749404907\\
140.762359619141 -11.406042098999\\
140.812362670898 -11.4077138900757\\
140.862365722656 -11.4093570709229\\
140.912368774414 -11.4109287261963\\
140.962356567383 -11.4124898910522\\
141.012359619141 -11.4139232635498\\
141.062362670898 -11.4152669906616\\
141.112365722656 -11.4165172576904\\
141.162368774414 -11.417688369751\\
141.212356567383 -11.4187803268433\\
141.262359619141 -11.4197883605957\\
141.312362670898 -11.4206447601318\\
141.362365722656 -11.4213838577271\\
141.412368774414 -11.422158241272\\
141.462356567383 -11.4231958389282\\
141.512359619141 -11.4244251251221\\
141.562362670898 -11.4253673553467\\
141.612365722656 -11.4262132644653\\
141.662368774414 -11.4271116256714\\
141.712356567383 -11.4280281066895\\
141.762359619141 -11.4288921356201\\
141.812362670898 -11.4297618865967\\
141.862365722656 -11.4305067062378\\
141.912368774414 -11.4312000274658\\
141.962356567383 -11.432056427002\\
142.012359619141 -11.4331092834473\\
142.062362670898 -11.4344282150269\\
142.112365722656 -11.4355173110962\\
142.162368774414 -11.4365549087524\\
142.212356567383 -11.4375782012939\\
142.262359619141 -11.4386072158813\\
142.312362670898 -11.4396228790283\\
142.362365722656 -11.440616607666\\
142.412368774414 -11.4415597915649\\
142.462356567383 -11.4422540664673\\
142.512359619141 -11.4429359436035\\
142.562362670898 -11.4435815811157\\
142.612365722656 -11.4441814422607\\
142.662368774414 -11.4447336196899\\
142.712356567383 -11.4453859329224\\
142.762359619141 -11.4462594985962\\
142.812362670898 -11.4472541809082\\
142.862365722656 -11.4482936859131\\
142.912368774414 -11.4492435455322\\
142.962356567383 -11.4501686096191\\
143.012359619141 -11.4512281417847\\
143.062362670898 -11.4524078369141\\
143.112365722656 -11.4536180496216\\
143.162368774414 -11.4547986984253\\
143.212356567383 -11.4561061859131\\
143.262359619141 -11.457576751709\\
143.312362670898 -11.4592227935791\\
143.362365722656 -11.4608297348022\\
143.412368774414 -11.4620056152344\\
143.462356567383 -11.4633102416992\\
143.512359619141 -11.4647159576416\\
143.562362670898 -11.4666137695313\\
143.612365722656 -11.4685564041138\\
143.662368774414 -11.470142364502\\
143.712356567383 -11.4722509384155\\
143.762359619141 -11.4747552871704\\
143.812362670898 -11.476731300354\\
143.862365722656 -11.4772748947144\\
143.912368774414 -11.4779119491577\\
143.962356567383 -11.4785957336426\\
144.012359619141 -11.4794874191284\\
144.062362670898 -11.4805583953857\\
144.112365722656 -11.4818515777588\\
144.162368774414 -11.4832277297974\\
144.212356567383 -11.4846439361572\\
144.262359619141 -11.4860916137695\\
144.312362670898 -11.4877052307129\\
144.362365722656 -11.4893922805786\\
144.412368774414 -11.4915704727173\\
144.462356567383 -11.4934692382813\\
144.512359619141 -11.4954996109009\\
144.562362670898 -11.4980878829956\\
144.612365722656 -11.5010871887207\\
144.662368774414 -11.5036592483521\\
144.712356567383 -11.5059642791748\\
144.762359619141 -11.5087461471558\\
144.812362670898 -11.5115737915039\\
144.862365722656 -11.5138330459595\\
144.912368774414 -11.5164623260498\\
144.962356567383 -11.5193920135498\\
145.012359619141 -11.5220327377319\\
145.062362670898 -11.5244302749634\\
145.112365722656 -11.527081489563\\
145.162368774414 -11.5295505523682\\
145.212356567383 -11.5321922302246\\
145.262359619141 -11.5350103378296\\
145.312362670898 -11.5376329421997\\
145.362365722656 -11.5402326583862\\
145.412368774414 -11.5429592132568\\
145.462356567383 -11.545615196228\\
145.512359619141 -11.5481290817261\\
145.562362670898 -11.5506153106689\\
145.612365722656 -11.5530681610107\\
145.662368774414 -11.5554599761963\\
145.712356567383 -11.5578060150146\\
145.762359619141 -11.5600986480713\\
145.812362670898 -11.5623445510864\\
145.862365722656 -11.5645351409912\\
145.912368774414 -11.5666694641113\\
145.962356567383 -11.5687475204468\\
146.012359619141 -11.5707674026489\\
146.062362670898 -11.5727338790894\\
146.112365722656 -11.5746488571167\\
146.162368774414 -11.5765190124512\\
146.212356567383 -11.5783433914185\\
146.262359619141 -11.5801258087158\\
146.312362670898 -11.5818681716919\\
146.362365722656 -11.5835723876953\\
146.412368774414 -11.5852422714233\\
146.462356567383 -11.5868740081787\\
146.512359619141 -11.5884666442871\\
146.562362670898 -11.5900230407715\\
146.612365722656 -11.5915441513062\\
146.662368774414 -11.5930299758911\\
146.712356567383 -11.5944776535034\\
146.762359619141 -11.5958890914917\\
146.812362670898 -11.5972633361816\\
146.862365722656 -11.5985984802246\\
146.912368774414 -11.5998935699463\\
146.962356567383 -11.6011409759521\\
147.012359619141 -11.6023416519165\\
147.062362670898 -11.603497505188\\
147.112365722656 -11.604606628418\\
147.162368774414 -11.6056728363037\\
147.212356567383 -11.6066980361938\\
147.262359619141 -11.607684135437\\
147.312362670898 -11.6086330413818\\
147.362365722656 -11.6095504760742\\
147.412368774414 -11.6104307174683\\
147.462356567383 -11.6112823486328\\
147.512359619141 -11.6121006011963\\
147.562362670898 -11.6128911972046\\
147.612365722656 -11.6136541366577\\
147.662368774414 -11.6143922805786\\
147.712356567383 -11.6151065826416\\
147.762359619141 -11.6158008575439\\
147.812362670898 -11.6164779663086\\
147.862365722656 -11.6171398162842\\
147.912368774414 -11.6177864074707\\
147.962356567383 -11.6184186935425\\
148.012359619141 -11.6190347671509\\
148.062362670898 -11.6196393966675\\
148.112365722656 -11.6202306747437\\
148.162368774414 -11.6208114624023\\
148.212356567383 -11.6213827133179\\
148.262359619141 -11.6219444274902\\
148.312362670898 -11.6224985122681\\
148.362365722656 -11.6230516433716\\
148.412368774414 -11.6236047744751\\
148.462356567383 -11.6241579055786\\
148.512359619141 -11.6247110366821\\
148.562362670898 -11.6252603530884\\
148.612365722656 -11.6258039474487\\
148.662368774414 -11.6263399124146\\
148.712356567383 -11.6268682479858\\
148.762359619141 -11.6273880004883\\
148.812362670898 -11.6278953552246\\
148.862365722656 -11.6283884048462\\
148.912368774414 -11.6288747787476\\
148.962356567383 -11.6293601989746\\
149.012359619141 -11.629846572876\\
149.062362670898 -11.6303329467773\\
149.112365722656 -11.6308164596558\\
149.162368774414 -11.6312942504883\\
149.212356567383 -11.6317672729492\\
149.262359619141 -11.6322336196899\\
149.312362670898 -11.6326923370361\\
149.362365722656 -11.6331491470337\\
149.412368774414 -11.6336002349854\\
149.462356567383 -11.6340484619141\\
149.512359619141 -11.6344966888428\\
149.562362670898 -11.6349449157715\\
149.612365722656 -11.6353874206543\\
149.662368774414 -11.6358203887939\\
149.712356567383 -11.6362380981445\\
149.762359619141 -11.6366443634033\\
149.812362670898 -11.6370401382446\\
149.862365722656 -11.6374254226685\\
149.912368774414 -11.6378011703491\\
149.962356567383 -11.6381664276123\\
};
\addlegendentry{$\dtmax{} = \expneg{5.0}{3}{[s]}$};

\addplot [
color=red,
solid
]
table[row sep=crcr]{
0 0\\
1.00060153007507 0\\
2.00060153007507 0\\
3.00060153007507 0\\
4.00060129165649 0\\
5.00060129165649 0\\
6.00060129165649 0\\
7.00060129165649 0\\
8.00060176849365 0\\
9.00060176849365 0\\
10.0006017684937 0\\
11.0006017684937 0\\
12.0006017684937 0\\
13.0006017684937 0\\
14.0006017684937 0\\
15.0006017684937 0\\
16.0006008148193 0\\
17.0006008148193 0\\
18.0006008148193 0\\
19.0006008148193 0\\
20.0006008148193 -5.83601945436385e-07\\
21.0022106170654 -0.554080188274384\\
22.0102825164795 -0.65607762336731\\
23.0160064697266 -0.691244065761566\\
24.0294227600098 -0.715729773044586\\
25.0317840576172 -0.738530099391937\\
26.0317840576172 -0.765316307544708\\
27.0317840576172 -0.798945963382721\\
28.0317840576172 -0.841111838817596\\
29.0317840576172 -0.895805835723877\\
30.0317840576172 -0.96610701084137\\
31.0317840576172 -1.05752277374268\\
32.0317840576172 -1.17191779613495\\
33.0317840576172 -1.29796457290649\\
34.0317840576172 -1.42988932132721\\
35.0017852783203 -1.55805087089539\\
36.0017852783203 -1.69097173213959\\
37.0017852783203 -1.82042157649994\\
38.0017852783203 -1.94572138786316\\
39.0017852783203 -2.06812834739685\\
40.0017852783203 -2.18989062309265\\
41.0017852783203 -2.31067252159119\\
42.0017852783203 -2.43037271499634\\
43.0017852783203 -2.55125641822815\\
44.0017852783203 -2.67461609840393\\
45.0017852783203 -2.8007984161377\\
46.0017852783203 -2.92492389678955\\
47.0017852783203 -3.04374742507935\\
48.0017852783203 -3.16315960884094\\
49.0017852783203 -3.28099989891052\\
50.0017852783203 -3.39454007148743\\
51.0017852783203 -3.50127220153809\\
52.0017852783203 -3.59822511672974\\
53.0017852783203 -3.68881320953369\\
54.0017852783203 -3.77682948112488\\
55.0017852783203 -3.86231446266174\\
56.0017852783203 -3.94669413566589\\
57.0017852783203 -4.03067827224731\\
58.0017852783203 -4.11482858657837\\
59.0017852783203 -4.19993925094604\\
60.0017852783203 -4.28682231903076\\
61.0017852783203 -4.37742328643799\\
62.0017852783203 -4.46653318405151\\
63.0017852783203 -4.55715703964233\\
64.0017852783203 -4.64753103256226\\
65.0017852783203 -4.73807001113892\\
66.0017852783203 -4.82947444915771\\
67.0017852783203 -4.92185735702515\\
68.0017852783203 -5.01526641845703\\
69.0017852783203 -5.10920095443726\\
70.0017852783203 -5.20432233810425\\
71.0017852783203 -5.29801273345947\\
72.0017852783203 -5.39327764511108\\
73.0017852783203 -5.48588943481445\\
74.0017852783203 -5.57854986190796\\
75.0017852783203 -5.67219877243042\\
76.0017852783203 -5.76673650741577\\
77.0017852783203 -5.86810159683228\\
78.0017852783203 -5.9605450630188\\
79.0017852783203 -6.05299615859985\\
80.0017852783203 -6.14794683456421\\
81.0017852783203 -6.24326229095459\\
82.0017852783203 -6.33908414840698\\
83.0017852783203 -6.4347767829895\\
84.0017852783203 -6.5310492515564\\
85.0017852783203 -6.62827682495117\\
86.0017852783203 -6.72480726242065\\
87.0017852783203 -6.819899559021\\
88.0017852783203 -6.91330051422119\\
89.0017852783203 -7.00475978851318\\
90.0017852783203 -7.09481000900269\\
91.0017852783203 -7.18459892272949\\
92.0017852783203 -7.27474927902222\\
93.0017852783203 -7.36418151855469\\
94.0017852783203 -7.45005130767822\\
95.0017852783203 -7.53138160705566\\
96.0017852783203 -7.6088662147522\\
97.0017852783203 -7.68305540084839\\
98.0017852783203 -7.75423336029053\\
99.0017852783203 -7.82320261001587\\
100 -7.89037561416626\\
100.023208618164 -7.89194917678833\\
100.048210144043 -7.89364099502563\\
100.073204040527 -7.89532709121704\\
100.098205566406 -7.89700174331665\\
100.123207092285 -7.89865827560425\\
100.148208618164 -7.90029621124268\\
100.173210144043 -7.90191888809204\\
100.198204040527 -7.90352821350098\\
100.223205566406 -7.90513134002686\\
100.248207092285 -7.90673208236694\\
100.273208618164 -7.90833806991577\\
100.298210144043 -7.90995454788208\\
100.323204040527 -7.91158485412598\\
100.348205566406 -7.9132251739502\\
100.373207092285 -7.91487646102905\\
100.398208618164 -7.9165301322937\\
100.423210144043 -7.91818809509277\\
100.448204040527 -7.91984415054321\\
100.473205566406 -7.92150974273682\\
100.498207092285 -7.92318153381348\\
100.523208618164 -7.92485857009888\\
100.548210144043 -7.92653560638428\\
100.573204040527 -7.92821216583252\\
100.598205566406 -7.9298849105835\\
100.623207092285 -7.93155288696289\\
100.648208618164 -7.93321704864502\\
100.673210144043 -7.93487691879272\\
100.698204040527 -7.93653345108032\\
100.723205566406 -7.93818616867065\\
100.748207092285 -7.93983459472656\\
100.773208618164 -7.94148302078247\\
100.798210144043 -7.94312858581543\\
100.823204040527 -7.94476699829102\\
100.848205566406 -7.94641065597534\\
100.873207092285 -7.94807291030884\\
100.898208618164 -7.94973230361938\\
100.923210144043 -7.95138597488403\\
100.948204040527 -7.95303106307983\\
100.973205566406 -7.95468044281006\\
100.998207092285 -7.95633125305176\\
101.023208618164 -7.95799875259399\\
101.048210144043 -7.95967578887939\\
101.073204040527 -7.96134662628174\\
101.098205566406 -7.96301031112671\\
101.123207092285 -7.96466255187988\\
101.148208618164 -7.96630048751831\\
101.173210144043 -7.96792364120483\\
101.198204040527 -7.9695348739624\\
101.223205566406 -7.97113370895386\\
101.248207092285 -7.97272109985352\\
101.273208618164 -7.97430086135864\\
101.298210144043 -7.97587776184082\\
101.323204040527 -7.97745656967163\\
101.348205566406 -7.97904348373413\\
101.373207092285 -7.98063850402832\\
101.398208618164 -7.9822473526001\\
101.423210144043 -7.98387908935547\\
101.448204040527 -7.98553371429443\\
101.473205566406 -7.9872031211853\\
101.498207092285 -7.98888301849365\\
101.523208618164 -7.99057340621948\\
101.548210144043 -7.99226188659668\\
101.573204040527 -7.99394607543945\\
101.598205566406 -7.9956259727478\\
101.623207092285 -7.99730062484741\\
101.648208618164 -7.9989709854126\\
101.673210144043 -8.00063800811768\\
101.698204040527 -8.00229930877686\\
101.723205566406 -8.00395584106445\\
101.748207092285 -8.00560760498047\\
101.773208618164 -8.00725555419922\\
101.798210144043 -8.00890350341797\\
101.823204040527 -8.01055335998535\\
101.848205566406 -8.01220035552979\\
101.873207092285 -8.01384735107422\\
101.898208618164 -8.01549053192139\\
101.923210144043 -8.01712703704834\\
101.948204040527 -8.01876640319824\\
101.973205566406 -8.02041530609131\\
101.998207092285 -8.02205467224121\\
102.023208618164 -8.02370834350586\\
102.048210144043 -8.02537631988525\\
102.073204040527 -8.02704048156738\\
102.098205566406 -8.02869892120361\\
102.123207092285 -8.03034496307373\\
102.148208618164 -8.03197956085205\\
102.173210144043 -8.03360176086426\\
102.198204040527 -8.03520679473877\\
102.223205566406 -8.0368013381958\\
102.248207092285 -8.03838539123535\\
102.273208618164 -8.03995800018311\\
102.298210144043 -8.04152297973633\\
102.323204040527 -8.04308414459229\\
102.348205566406 -8.04464626312256\\
102.373207092285 -8.04621315002441\\
102.398208618164 -8.04778671264648\\
102.423210144043 -8.04937076568604\\
102.448204040527 -8.05096435546875\\
102.473205566406 -8.052565574646\\
102.498207092285 -8.05417442321777\\
102.523208618164 -8.05578422546387\\
102.548210144043 -8.05739593505859\\
102.573204040527 -8.05901145935059\\
102.598205566406 -8.06062984466553\\
102.623207092285 -8.06224918365479\\
102.648208618164 -8.06387042999268\\
102.673210144043 -8.06549167633057\\
102.698204040527 -8.06711483001709\\
102.723205566406 -8.06874179840088\\
102.748207092285 -8.07036972045898\\
102.773208618164 -8.07199764251709\\
102.798210144043 -8.0736255645752\\
102.823204040527 -8.07525253295898\\
102.848205566406 -8.07687664031982\\
102.873207092285 -8.07851314544678\\
102.898208618164 -8.08016204833984\\
102.923210144043 -8.08181190490723\\
102.948204040527 -8.08345413208008\\
102.973205566406 -8.08508968353271\\
102.998207092285 -8.08671951293945\\
103.023208618164 -8.08836555480957\\
103.048210144043 -8.09002017974854\\
103.073204040527 -8.09166812896729\\
103.098205566406 -8.09330749511719\\
103.123207092285 -8.09494113922119\\
103.148208618164 -8.09656047821045\\
103.173210144043 -8.09816551208496\\
103.198204040527 -8.09975719451904\\
103.223205566406 -8.1013355255127\\
103.248207092285 -8.10290241241455\\
103.273208618164 -8.10445785522461\\
103.298210144043 -8.1060037612915\\
103.323204040527 -8.1075439453125\\
103.348205566406 -8.10907745361328\\
103.373207092285 -8.11061096191406\\
103.398208618164 -8.11214637756348\\
103.423210144043 -8.11368846893311\\
103.448204040527 -8.11523628234863\\
103.473205566406 -8.11679077148438\\
103.498207092285 -8.11835384368896\\
103.523208618164 -8.11992168426514\\
103.548210144043 -8.12149333953857\\
103.573204040527 -8.12306690216064\\
103.598205566406 -8.12464046478271\\
103.623207092285 -8.12621879577637\\
103.648208618164 -8.12779903411865\\
103.673210144043 -8.12938213348389\\
103.698204040527 -8.13096618652344\\
103.723205566406 -8.13255596160889\\
103.748207092285 -8.13414764404297\\
103.773208618164 -8.13574028015137\\
103.798210144043 -8.13733577728271\\
103.823204040527 -8.13893413543701\\
103.848205566406 -8.14053344726563\\
103.873207092285 -8.14213275909424\\
103.898208618164 -8.14373111724854\\
103.923210144043 -8.14532279968262\\
103.948204040527 -8.14690685272217\\
103.973205566406 -8.1485013961792\\
103.998207092285 -8.15010929107666\\
104.023208618164 -8.15171241760254\\
104.048210144043 -8.1533203125\\
104.073204040527 -8.15492248535156\\
104.098205566406 -8.15652179718018\\
104.123207092285 -8.15811443328857\\
104.148208618164 -8.15969753265381\\
104.173210144043 -8.16127014160156\\
104.198204040527 -8.16282939910889\\
104.223205566406 -8.16437911987305\\
104.248207092285 -8.16591739654541\\
104.273208618164 -8.16744422912598\\
104.298210144043 -8.16896152496338\\
104.323204040527 -8.17047214508057\\
104.348205566406 -8.17197799682617\\
104.373207092285 -8.17348194122314\\
104.398208618164 -8.17498588562012\\
104.423210144043 -8.17648983001709\\
104.448204040527 -8.17799663543701\\
104.473205566406 -8.17950534820557\\
104.498207092285 -8.1810188293457\\
104.523208618164 -8.18253517150879\\
104.548210144043 -8.18405818939209\\
104.573204040527 -8.18558406829834\\
104.598205566406 -8.18711471557617\\
104.623207092285 -8.18864822387695\\
104.648208618164 -8.19018363952637\\
104.673210144043 -8.19172286987305\\
104.698204040527 -8.19326400756836\\
104.723205566406 -8.19480609893799\\
104.748207092285 -8.19634819030762\\
104.773208618164 -8.1978931427002\\
104.798210144043 -8.19944763183594\\
104.823204040527 -8.20101070404053\\
104.848205566406 -8.20258235931396\\
104.873207092285 -8.20415592193604\\
104.898208618164 -8.205735206604\\
104.923210144043 -8.20730972290039\\
104.948204040527 -8.20888233184814\\
104.973205566406 -8.21045589447021\\
104.998207092285 -8.21203804016113\\
105.023208618164 -8.21361827850342\\
105.048210144043 -8.21518993377686\\
105.073204040527 -8.21675968170166\\
105.098205566406 -8.21832752227783\\
105.123207092285 -8.21988868713379\\
105.148208618164 -8.22144222259521\\
105.173210144043 -8.22298336029053\\
105.198204040527 -8.22451591491699\\
105.223205566406 -8.22604179382324\\
105.248207092285 -8.22756195068359\\
105.273208618164 -8.22907733917236\\
105.298210144043 -8.23058891296387\\
105.323204040527 -8.23209571838379\\
105.348205566406 -8.23360061645508\\
105.373207092285 -8.2351016998291\\
105.398208618164 -8.23659801483154\\
105.423210144043 -8.23808860778809\\
105.448204040527 -8.23957538604736\\
105.473205566406 -8.24106025695801\\
105.498207092285 -8.24254512786865\\
105.523208618164 -8.2440299987793\\
105.548210144043 -8.24551486968994\\
105.573204040527 -8.24700260162354\\
105.598205566406 -8.24849224090576\\
105.623207092285 -8.24998664855957\\
105.648208618164 -8.25148391723633\\
105.673210144043 -8.25298118591309\\
105.698204040527 -8.25447940826416\\
105.723205566406 -8.25598335266113\\
105.748207092285 -8.25748729705811\\
105.773208618164 -8.25899410247803\\
105.798210144043 -8.26050662994385\\
105.823204040527 -8.26202201843262\\
105.848205566406 -8.26354789733887\\
105.873207092285 -8.26508140563965\\
105.898208618164 -8.2666187286377\\
105.923210144043 -8.26816082000732\\
105.948204040527 -8.26970291137695\\
105.973205566406 -8.27124691009521\\
105.998207092285 -8.27278900146484\\
106.023208618164 -8.27433586120605\\
106.048210144043 -8.2758903503418\\
106.073204040527 -8.27744293212891\\
106.098205566406 -8.27899169921875\\
106.123207092285 -8.28053379058838\\
106.148208618164 -8.28207492828369\\
106.173210144043 -8.28360939025879\\
106.198204040527 -8.28513431549072\\
106.223205566406 -8.28665161132813\\
106.248207092285 -8.28816509246826\\
106.273208618164 -8.2896728515625\\
106.298210144043 -8.29117870330811\\
106.323204040527 -8.29267978668213\\
106.348205566406 -8.29417705535889\\
106.373207092285 -8.29567337036133\\
106.398208618164 -8.29716777801514\\
106.423210144043 -8.29865550994873\\
106.448204040527 -8.30014133453369\\
106.473205566406 -8.30162620544434\\
106.498207092285 -8.30310440063477\\
106.523208618164 -8.30457973480225\\
106.548210144043 -8.30605125427246\\
106.573204040527 -8.30751991271973\\
106.598205566406 -8.30898666381836\\
106.623207092285 -8.31045246124268\\
106.648208618164 -8.31192016601563\\
106.673210144043 -8.31338882446289\\
106.698204040527 -8.31485748291016\\
106.723205566406 -8.31632614135742\\
106.748207092285 -8.31779956817627\\
106.773208618164 -8.31927490234375\\
106.798210144043 -8.32075023651123\\
106.823204040527 -8.32222461700439\\
106.848205566406 -8.32370185852051\\
106.873207092285 -8.32518005371094\\
106.898208618164 -8.32665920257568\\
106.923210144043 -8.32814407348633\\
106.948204040527 -8.32962894439697\\
106.973205566406 -8.33111667633057\\
106.998207092285 -8.33260917663574\\
107.023208618164 -8.3341064453125\\
107.048210144043 -8.33560752868652\\
107.073204040527 -8.33711338043213\\
107.098205566406 -8.33862686157227\\
107.123207092285 -8.34014415740967\\
107.148208618164 -8.34166717529297\\
107.173210144043 -8.34319305419922\\
107.198204040527 -8.34471893310547\\
107.223205566406 -8.34624481201172\\
107.248207092285 -8.34777069091797\\
107.273208618164 -8.3492956161499\\
107.298210144043 -8.35081481933594\\
107.323204040527 -8.35232830047607\\
107.348205566406 -8.35383319854736\\
107.373207092285 -8.35533237457275\\
107.398208618164 -8.35682964324951\\
107.423210144043 -8.35832405090332\\
107.448204040527 -8.35981559753418\\
107.473205566406 -8.36130237579346\\
107.498207092285 -8.3627872467041\\
107.523208618164 -8.36426639556885\\
107.548210144043 -8.36574363708496\\
107.573204040527 -8.36721801757813\\
107.598205566406 -8.36869335174561\\
107.623207092285 -8.3701639175415\\
107.648208618164 -8.37163162231445\\
107.673210144043 -8.37309741973877\\
107.698204040527 -8.37456226348877\\
107.723205566406 -8.37602806091309\\
107.748207092285 -8.37749195098877\\
107.773208618164 -8.37895011901855\\
107.798210144043 -8.38040447235107\\
107.823204040527 -8.38185405731201\\
107.848205566406 -8.38330364227295\\
107.873207092285 -8.38475036621094\\
107.898208618164 -8.38619709014893\\
107.923210144043 -8.3876428604126\\
107.948204040527 -8.38908290863037\\
107.973205566406 -8.39052295684814\\
107.998207092285 -8.39196300506592\\
108.023208618164 -8.39340400695801\\
108.048210144043 -8.39484786987305\\
108.073204040527 -8.39628982543945\\
108.098205566406 -8.39773654937744\\
108.123207092285 -8.39918613433838\\
108.148208618164 -8.40064144134521\\
108.173210144043 -8.402099609375\\
108.198204040527 -8.40356254577637\\
108.223205566406 -8.40503120422363\\
108.248207092285 -8.40650653839111\\
108.273208618164 -8.40798473358154\\
108.298210144043 -8.40946674346924\\
108.323204040527 -8.4109525680542\\
108.348205566406 -8.41244029998779\\
108.373207092285 -8.41393089294434\\
108.398208618164 -8.41541862487793\\
108.423210144043 -8.41690635681152\\
108.448204040527 -8.41839408874512\\
108.473205566406 -8.41988468170166\\
108.498207092285 -8.42137241363525\\
108.523208618164 -8.42285919189453\\
108.548210144043 -8.42434215545654\\
108.573204040527 -8.42582035064697\\
108.598205566406 -8.42729663848877\\
108.623207092285 -8.4287691116333\\
108.648208618164 -8.43023777008057\\
108.673210144043 -8.4317045211792\\
108.698204040527 -8.4331693649292\\
108.723205566406 -8.4346342086792\\
108.748207092285 -8.4360933303833\\
108.773208618164 -8.43755054473877\\
108.798210144043 -8.43900680541992\\
108.823204040527 -8.44046211242676\\
108.848205566406 -8.4419116973877\\
108.873207092285 -8.44336128234863\\
108.898208618164 -8.44480800628662\\
108.923210144043 -8.44624996185303\\
108.948204040527 -8.44768714904785\\
108.973205566406 -8.44912433624268\\
108.998207092285 -8.45055675506592\\
109.023208618164 -8.45198631286621\\
109.048210144043 -8.45341396331787\\
109.073204040527 -8.45484066009521\\
109.098205566406 -8.45626831054688\\
109.123207092285 -8.45769596099854\\
109.148208618164 -8.4591236114502\\
109.173210144043 -8.46055126190186\\
109.198204040527 -8.46197986602783\\
109.223205566406 -8.46341037750244\\
109.248207092285 -8.46484756469727\\
109.273208618164 -8.46628570556641\\
109.298210144043 -8.46772575378418\\
109.323204040527 -8.46917152404785\\
109.348205566406 -8.47062110900879\\
109.373207092285 -8.47207546234131\\
109.398208618164 -8.47353458404541\\
109.423210144043 -8.47500038146973\\
109.448204040527 -8.47646903991699\\
109.473205566406 -8.47793769836426\\
109.498207092285 -8.47941303253174\\
109.523208618164 -8.48088836669922\\
109.548210144043 -8.4823637008667\\
109.573204040527 -8.48383808135986\\
109.598205566406 -8.48531341552734\\
109.623207092285 -8.48678207397461\\
109.648208618164 -8.48824787139893\\
109.673210144043 -8.48970699310303\\
109.698204040527 -8.4911642074585\\
109.723205566406 -8.4926233291626\\
109.748207092285 -8.49407958984375\\
109.773208618164 -8.49553394317627\\
109.798210144043 -8.49698352813721\\
109.823204040527 -8.49842929840088\\
109.848205566406 -8.49987888336182\\
109.873207092285 -8.5013256072998\\
109.898208618164 -8.50276851654053\\
109.923210144043 -8.5042085647583\\
109.948204040527 -8.50564575195313\\
109.973205566406 -8.50708103179932\\
109.998207092285 -8.50851154327393\\
110.023208618164 -8.50994205474854\\
110.048210144043 -8.5113697052002\\
110.073204040527 -8.51279640197754\\
110.098205566406 -8.51421737670898\\
110.123207092285 -8.51563835144043\\
110.148208618164 -8.51705837249756\\
110.173210144043 -8.51847648620605\\
110.198204040527 -8.51989364624023\\
110.223205566406 -8.52131175994873\\
110.248207092285 -8.52272987365723\\
110.273208618164 -8.52414798736572\\
110.298210144043 -8.52556800842285\\
110.323204040527 -8.5269889831543\\
110.348205566406 -8.52840995788574\\
110.373207092285 -8.5298376083374\\
110.398208618164 -8.53126525878906\\
110.423210144043 -8.53269290924072\\
110.448204040527 -8.53412342071533\\
110.473205566406 -8.53555774688721\\
110.498207092285 -8.5369930267334\\
110.523208618164 -8.53843116760254\\
110.548210144043 -8.53987216949463\\
110.573204040527 -8.5413179397583\\
110.598205566406 -8.54276466369629\\
110.623207092285 -8.54421234130859\\
110.648208618164 -8.54566192626953\\
110.673210144043 -8.54711151123047\\
110.698204040527 -8.54856014251709\\
110.723205566406 -8.55000782012939\\
110.748207092285 -8.5514554977417\\
110.773208618164 -8.55289840698242\\
110.798210144043 -8.55433750152588\\
110.823204040527 -8.55577182769775\\
110.848205566406 -8.55720233917236\\
110.873207092285 -8.55863285064697\\
110.898208618164 -8.56006336212158\\
110.923210144043 -8.56149387359619\\
110.948204040527 -8.56292152404785\\
110.973205566406 -8.56434917449951\\
110.998207092285 -8.56577205657959\\
111.023208618164 -8.56719303131104\\
111.048210144043 -8.56861400604248\\
111.073204040527 -8.57003498077393\\
111.098205566406 -8.57145595550537\\
111.123207092285 -8.57287693023682\\
111.148208618164 -8.57429599761963\\
111.173210144043 -8.57571411132813\\
111.198204040527 -8.5771312713623\\
111.223205566406 -8.5785493850708\\
111.248207092285 -8.57996273040771\\
111.273208618164 -8.581374168396\\
111.298210144043 -8.58278560638428\\
111.323204040527 -8.58419704437256\\
111.348205566406 -8.58560848236084\\
111.373207092285 -8.58701992034912\\
111.398208618164 -8.58842754364014\\
111.423210144043 -8.58982944488525\\
111.448204040527 -8.59122562408447\\
111.473205566406 -8.59261608123779\\
111.498207092285 -8.5940055847168\\
111.523208618164 -8.5953893661499\\
111.548210144043 -8.59677219390869\\
111.573204040527 -8.59815502166748\\
111.598205566406 -8.59953784942627\\
111.623207092285 -8.60092067718506\\
111.648208618164 -8.60231018066406\\
111.673210144043 -8.60370254516602\\
111.698204040527 -8.60509777069092\\
111.723205566406 -8.60649871826172\\
111.748207092285 -8.60790348052979\\
111.773208618164 -8.60931396484375\\
111.798210144043 -8.61072635650635\\
111.823204040527 -8.61213779449463\\
111.848205566406 -8.61355590820313\\
111.873207092285 -8.61496925354004\\
111.898208618164 -8.61638164520264\\
111.923210144043 -8.61779975891113\\
111.948204040527 -8.61921691894531\\
111.973205566406 -8.62063121795654\\
111.998207092285 -8.62204265594482\\
112.023208618164 -8.62345409393311\\
112.048210144043 -8.62486553192139\\
112.073204040527 -8.62627601623535\\
112.098205566406 -8.62768459320068\\
112.123207092285 -8.62908935546875\\
112.148208618164 -8.63049125671387\\
112.173210144043 -8.63189220428467\\
112.198204040527 -8.63329315185547\\
112.223205566406 -8.63469219207764\\
112.248207092285 -8.63608837127686\\
112.273208618164 -8.63748073577881\\
112.298210144043 -8.63887214660645\\
112.323204040527 -8.64026069641113\\
112.348205566406 -8.64164638519287\\
112.373207092285 -8.64302921295166\\
112.398208618164 -8.64441204071045\\
112.423210144043 -8.64579200744629\\
112.448204040527 -8.64716720581055\\
112.473205566406 -8.64854049682617\\
112.498207092285 -8.6499137878418\\
112.523208618164 -8.65128612518311\\
112.548210144043 -8.6526575088501\\
112.573204040527 -8.65402889251709\\
112.598205566406 -8.65539932250977\\
112.623207092285 -8.65676975250244\\
112.648208618164 -8.65814018249512\\
112.673210144043 -8.65950775146484\\
112.698204040527 -8.6608715057373\\
112.723205566406 -8.66223239898682\\
112.748207092285 -8.66359329223633\\
112.773208618164 -8.66495418548584\\
112.798210144043 -8.6663179397583\\
112.823204040527 -8.66767978668213\\
112.848205566406 -8.66904354095459\\
112.873207092285 -8.67040729522705\\
112.898208618164 -8.67177104949951\\
112.923210144043 -8.67313957214355\\
112.948204040527 -8.67450904846191\\
112.973205566406 -8.67588233947754\\
112.998207092285 -8.67725563049316\\
113.023208618164 -8.67862892150879\\
113.048210144043 -8.68000221252441\\
113.073204040527 -8.68137836456299\\
113.098205566406 -8.68275928497314\\
113.123207092285 -8.68414211273193\\
113.148208618164 -8.68552398681641\\
113.173210144043 -8.68690395355225\\
113.198204040527 -8.68828296661377\\
113.223205566406 -8.68966197967529\\
113.248207092285 -8.69103527069092\\
113.273208618164 -8.69240856170654\\
113.298210144043 -8.69377899169922\\
113.323204040527 -8.69514274597168\\
113.348205566406 -8.69650650024414\\
113.373207092285 -8.6978702545166\\
113.398208618164 -8.69923210144043\\
113.423210144043 -8.70059204101563\\
113.448204040527 -8.70194625854492\\
113.473205566406 -8.7032995223999\\
113.498207092285 -8.70465087890625\\
113.523208618164 -8.70600128173828\\
113.548210144043 -8.70734596252441\\
113.573204040527 -8.70869064331055\\
113.598205566406 -8.71003437042236\\
113.623207092285 -8.71137619018555\\
113.648208618164 -8.71271228790283\\
113.673210144043 -8.7140474319458\\
113.698204040527 -8.71538066864014\\
113.723205566406 -8.71671199798584\\
113.748207092285 -8.71803665161133\\
113.773208618164 -8.71935939788818\\
113.798210144043 -8.72067642211914\\
113.823204040527 -8.72199249267578\\
113.848205566406 -8.72330856323242\\
113.873207092285 -8.72462368011475\\
113.898208618164 -8.72593688964844\\
113.923210144043 -8.72725200653076\\
113.948204040527 -8.7285680770874\\
113.973205566406 -8.72988414764404\\
113.998207092285 -8.73120021820068\\
114.023208618164 -8.73251628875732\\
114.048210144043 -8.73383522033691\\
114.073204040527 -8.73515796661377\\
114.098205566406 -8.73648357391357\\
114.123207092285 -8.73780918121338\\
114.148208618164 -8.73913669586182\\
114.173210144043 -8.74046897888184\\
114.198204040527 -8.74180030822754\\
114.223205566406 -8.74313545227051\\
114.248207092285 -8.74447059631348\\
114.273208618164 -8.74580574035645\\
114.298210144043 -8.74714660644531\\
114.323204040527 -8.74848747253418\\
114.348205566406 -8.74982261657715\\
114.373207092285 -8.75115776062012\\
114.398208618164 -8.75249290466309\\
114.423210144043 -8.75382804870605\\
114.448204040527 -8.75516319274902\\
114.473205566406 -8.75649547576904\\
114.498207092285 -8.75782775878906\\
114.523208618164 -8.75915431976318\\
114.548210144043 -8.76047992706299\\
114.573204040527 -8.76180553436279\\
114.598205566406 -8.76312923431396\\
114.623207092285 -8.7644510269165\\
114.648208618164 -8.76576709747314\\
114.673210144043 -8.76708316802979\\
114.698204040527 -8.76839542388916\\
114.723205566406 -8.76970481872559\\
114.748207092285 -8.77101135253906\\
114.773208618164 -8.77231407165527\\
114.798210144043 -8.77361106872559\\
114.823204040527 -8.77490520477295\\
114.848205566406 -8.77619552612305\\
114.873207092285 -8.77748203277588\\
114.898208618164 -8.77876567840576\\
114.923210144043 -8.78004360198975\\
114.948204040527 -8.78132152557373\\
114.973205566406 -8.78259658813477\\
114.998207092285 -8.78386878967285\\
115.023208618164 -8.78513717651367\\
115.048210144043 -8.78640556335449\\
115.073204040527 -8.78767395019531\\
115.098205566406 -8.78894329071045\\
115.123207092285 -8.7902193069458\\
115.148208618164 -8.79149723052979\\
115.173210144043 -8.79277515411377\\
115.198204040527 -8.79405689239502\\
115.223205566406 -8.79534149169922\\
115.248207092285 -8.79662609100342\\
115.273208618164 -8.79791164398193\\
115.298210144043 -8.79919910430908\\
115.323204040527 -8.80049228668213\\
115.348205566406 -8.80178737640381\\
115.373207092285 -8.80308437347412\\
115.398208618164 -8.80438232421875\\
115.423210144043 -8.80568599700928\\
115.448204040527 -8.80699157714844\\
115.473205566406 -8.80829811096191\\
115.498207092285 -8.80960464477539\\
115.523208618164 -8.81091117858887\\
115.548210144043 -8.81221771240234\\
115.573204040527 -8.81352424621582\\
115.598205566406 -8.8148307800293\\
115.623207092285 -8.81613445281982\\
115.648208618164 -8.81743812561035\\
115.673210144043 -8.81874179840088\\
115.698204040527 -8.82004070281982\\
115.723205566406 -8.82133769989014\\
115.748207092285 -8.82263374328613\\
115.773208618164 -8.8239278793335\\
115.798210144043 -8.82521533966064\\
115.823204040527 -8.82650089263916\\
115.848205566406 -8.82778167724609\\
115.873207092285 -8.82905960083008\\
115.898208618164 -8.83033561706543\\
115.923210144043 -8.83160495758057\\
115.948204040527 -8.83287239074707\\
115.973205566406 -8.83413791656494\\
115.998207092285 -8.83539867401123\\
116.023208618164 -8.83665752410889\\
116.048210144043 -8.83791351318359\\
116.073204040527 -8.83916473388672\\
116.098205566406 -8.84041404724121\\
116.123207092285 -8.84166145324707\\
116.148208618164 -8.84290790557861\\
116.173210144043 -8.84415054321289\\
116.198204040527 -8.84539031982422\\
116.223205566406 -8.84663009643555\\
116.248207092285 -8.84786987304688\\
116.273208618164 -8.8491096496582\\
116.298210144043 -8.85034942626953\\
116.323204040527 -8.85158920288086\\
116.348205566406 -8.85283374786377\\
116.373207092285 -8.85408210754395\\
116.398208618164 -8.85533142089844\\
116.423210144043 -8.85658740997314\\
116.448204040527 -8.8578462600708\\
116.473205566406 -8.85910511016846\\
116.498207092285 -8.8603687286377\\
116.523208618164 -8.86163711547852\\
116.548210144043 -8.86290550231934\\
116.573204040527 -8.86417961120605\\
116.598205566406 -8.86545753479004\\
116.623207092285 -8.86673545837402\\
116.648208618164 -8.86801338195801\\
116.673210144043 -8.86929130554199\\
116.698204040527 -8.87056922912598\\
116.723205566406 -8.87184715270996\\
116.748207092285 -8.87312507629395\\
116.773208618164 -8.87440299987793\\
116.798210144043 -8.87567901611328\\
116.823204040527 -8.87695121765137\\
116.848205566406 -8.87821960449219\\
116.873207092285 -8.87948608398438\\
116.898208618164 -8.88074970245361\\
116.923210144043 -8.88200855255127\\
116.948204040527 -8.88326263427734\\
116.973205566406 -8.88451099395752\\
116.998207092285 -8.88575744628906\\
117.023208618164 -8.88700294494629\\
117.048210144043 -8.88824272155762\\
117.073204040527 -8.88948059082031\\
117.098205566406 -8.89071273803711\\
117.123207092285 -8.89193916320801\\
117.148208618164 -8.89315986633301\\
117.173210144043 -8.89437961578369\\
117.198204040527 -8.89559650421143\\
117.223205566406 -8.89680767059326\\
117.248207092285 -8.8980188369751\\
117.273208618164 -8.89922904968262\\
117.298210144043 -8.9004373550415\\
117.323204040527 -8.90164470672607\\
117.348205566406 -8.90284824371338\\
117.373207092285 -8.90404987335205\\
117.398208618164 -8.90525436401367\\
117.423210144043 -8.90646266937256\\
117.448204040527 -8.90767192840576\\
117.473205566406 -8.9088830947876\\
117.498207092285 -8.91009426116943\\
117.523208618164 -8.91130924224854\\
117.548210144043 -8.9125280380249\\
117.573204040527 -8.91374969482422\\
117.598205566406 -8.91497898101807\\
117.623207092285 -8.91620922088623\\
117.648208618164 -8.91744422912598\\
117.673210144043 -8.91868305206299\\
117.698204040527 -8.91992282867432\\
117.723205566406 -8.92116260528564\\
117.748207092285 -8.92240905761719\\
117.773208618164 -8.92365550994873\\
117.798210144043 -8.92490386962891\\
117.823204040527 -8.92615032196045\\
117.848205566406 -8.92739677429199\\
117.873207092285 -8.92864322662354\\
117.898208618164 -8.92988586425781\\
117.923210144043 -8.93112564086914\\
117.948204040527 -8.93236541748047\\
117.973205566406 -8.9336051940918\\
117.998207092285 -8.93484306335449\\
118.023208618164 -8.93607616424561\\
118.048210144043 -8.93730640411377\\
118.073204040527 -8.93853378295898\\
118.098205566406 -8.93975448608398\\
118.123207092285 -8.94097328186035\\
118.148208618164 -8.94218921661377\\
118.173210144043 -8.94340038299561\\
118.198204040527 -8.94460964202881\\
118.223205566406 -8.94581413269043\\
118.248207092285 -8.9470157623291\\
118.273208618164 -8.94821643829346\\
118.298210144043 -8.94941520690918\\
118.323204040527 -8.95061492919922\\
118.348205566406 -8.95181369781494\\
118.373207092285 -8.95300579071045\\
118.398208618164 -8.95419502258301\\
118.423210144043 -8.95537757873535\\
118.448204040527 -8.95655632019043\\
118.473205566406 -8.95773220062256\\
118.498207092285 -8.95890522003174\\
118.523208618164 -8.96007823944092\\
118.548210144043 -8.9612512588501\\
118.573204040527 -8.96242427825928\\
118.598205566406 -8.96359729766846\\
118.623207092285 -8.96477031707764\\
118.648208618164 -8.96594333648682\\
118.673210144043 -8.96712207794189\\
118.698204040527 -8.96830272674561\\
118.723205566406 -8.96948528289795\\
118.748207092285 -8.97067070007324\\
118.773208618164 -8.97186088562012\\
118.798210144043 -8.97305297851563\\
118.823204040527 -8.97424507141113\\
118.848205566406 -8.97543907165527\\
118.873207092285 -8.97663879394531\\
118.898208618164 -8.97784042358398\\
118.923210144043 -8.97904205322266\\
118.948204040527 -8.98024368286133\\
118.973205566406 -8.9814453125\\
118.998207092285 -8.98265361785889\\
119.023208618164 -8.98386192321777\\
119.048210144043 -8.98506450653076\\
119.073204040527 -8.98626613616943\\
119.098205566406 -8.98746585845947\\
119.123207092285 -8.98866176605225\\
119.148208618164 -8.98985385894775\\
119.173210144043 -8.99104499816895\\
119.198204040527 -8.99223136901855\\
119.223205566406 -8.9934139251709\\
119.248207092285 -8.99459362030029\\
119.273208618164 -8.99576759338379\\
119.298210144043 -8.99693965911865\\
119.323204040527 -8.99810600280762\\
119.348205566406 -8.99926948547363\\
119.373207092285 -9.00043106079102\\
119.398208618164 -9.00158596038818\\
119.423210144043 -9.00273895263672\\
119.448204040527 -9.00388526916504\\
119.473205566406 -9.00502967834473\\
119.498207092285 -9.00617408752441\\
119.523208618164 -9.00731658935547\\
119.548210144043 -9.00845146179199\\
119.573204040527 -9.00958633422852\\
119.598205566406 -9.01071834564209\\
119.623207092285 -9.01185035705566\\
119.648208618164 -9.01298236846924\\
119.673210144043 -9.01411437988281\\
119.698204040527 -9.01524925231934\\
119.723205566406 -9.01638412475586\\
119.748207092285 -9.01751899719238\\
119.773208618164 -9.01865386962891\\
119.798210144043 -9.01979351043701\\
119.823204040527 -9.02093505859375\\
119.848205566406 -9.02207946777344\\
119.873207092285 -9.02322673797607\\
119.898208618164 -9.02437782287598\\
119.923210144043 -9.02553081512451\\
119.948204040527 -9.02668476104736\\
119.973205566406 -9.02783870697021\\
119.998207092285 -9.02899837493896\\
120.023208618164 -9.03015899658203\\
120.048210144043 -9.0313196182251\\
120.073204040527 -9.03247928619385\\
120.098205566406 -9.03363990783691\\
120.123207092285 -9.03480052947998\\
120.148208618164 -9.03596115112305\\
120.173210144043 -9.03711700439453\\
120.198204040527 -9.03827095031738\\
120.223205566406 -9.0394229888916\\
120.248207092285 -9.0405740737915\\
120.273208618164 -9.04171943664551\\
120.298210144043 -9.0428638458252\\
120.323204040527 -9.04400444030762\\
120.348205566406 -9.04514217376709\\
120.373207092285 -9.04627704620361\\
120.398208618164 -9.04740905761719\\
120.423210144043 -9.04853630065918\\
120.448204040527 -9.04966163635254\\
120.473205566406 -9.05078601837158\\
120.498207092285 -9.05190849304199\\
120.523208618164 -9.05302429199219\\
120.548210144043 -9.05413818359375\\
120.573204040527 -9.05524730682373\\
120.598205566406 -9.05635356903076\\
120.623207092285 -9.05745601654053\\
120.648208618164 -9.05855274200439\\
120.673210144043 -9.05964946746826\\
120.698204040527 -9.06074333190918\\
120.723205566406 -9.0618371963501\\
120.748207092285 -9.06293106079102\\
120.773208618164 -9.06402778625488\\
120.798210144043 -9.06513118743896\\
120.823204040527 -9.066237449646\\
120.848205566406 -9.06734180450439\\
120.873207092285 -9.06844520568848\\
120.898208618164 -9.06954860687256\\
120.923210144043 -9.07064628601074\\
120.948204040527 -9.07174873352051\\
120.973205566406 -9.07285404205322\\
120.998207092285 -9.07396030426025\\
121.023208618164 -9.0750675201416\\
121.048210144043 -9.07618045806885\\
121.073204040527 -9.07729530334473\\
121.098205566406 -9.07841110229492\\
121.123207092285 -9.07952690124512\\
121.148208618164 -9.08064460754395\\
121.173210144043 -9.08176040649414\\
121.198204040527 -9.08287811279297\\
121.223205566406 -9.08399391174316\\
121.248207092285 -9.08511638641357\\
121.273208618164 -9.08623886108398\\
121.298210144043 -9.08736133575439\\
121.323204040527 -9.08848285675049\\
121.348205566406 -9.0896053314209\\
121.373207092285 -9.09072303771973\\
121.398208618164 -9.09183883666992\\
121.423210144043 -9.0929536819458\\
121.448204040527 -9.0940637588501\\
121.473205566406 -9.09517002105713\\
121.498207092285 -9.09627628326416\\
121.523208618164 -9.09738159179688\\
121.548210144043 -9.09848117828369\\
121.573204040527 -9.09957790374756\\
121.598205566406 -9.10067367553711\\
121.623207092285 -9.10176658630371\\
121.648208618164 -9.10285377502441\\
121.673210144043 -9.10394096374512\\
121.698204040527 -9.10502433776855\\
121.723205566406 -9.10610866546631\\
121.748207092285 -9.10718631744385\\
121.773208618164 -9.10826396942139\\
121.798210144043 -9.10933780670166\\
121.823204040527 -9.11040592193604\\
121.848205566406 -9.11147403717041\\
121.873207092285 -9.11254215240479\\
121.898208618164 -9.11361026763916\\
121.923210144043 -9.11467838287354\\
121.948204040527 -9.11574459075928\\
121.973205566406 -9.1168098449707\\
121.998207092285 -9.11787509918213\\
122.023208618164 -9.11894035339355\\
122.048210144043 -9.1200065612793\\
122.073204040527 -9.12107276916504\\
122.098205566406 -9.12214088439941\\
122.123207092285 -9.12321186065674\\
122.148208618164 -9.12428283691406\\
122.173210144043 -9.12534999847412\\
122.198204040527 -9.12641429901123\\
122.223205566406 -9.12747955322266\\
122.248207092285 -9.12854480743408\\
122.273208618164 -9.12961006164551\\
122.298210144043 -9.13066959381104\\
122.323204040527 -9.13172817230225\\
122.348205566406 -9.13278579711914\\
122.373207092285 -9.1338415145874\\
122.398208618164 -9.13489151000977\\
122.423210144043 -9.13594055175781\\
122.448204040527 -9.13698959350586\\
122.473205566406 -9.13803768157959\\
122.498207092285 -9.13908100128174\\
122.523208618164 -9.1401195526123\\
122.548210144043 -9.14115047454834\\
122.573204040527 -9.14217948913574\\
122.598205566406 -9.1432056427002\\
122.623207092285 -9.14422512054443\\
122.648208618164 -9.14523792266846\\
122.673210144043 -9.14624881744385\\
122.698204040527 -9.14725971221924\\
122.723205566406 -9.14826488494873\\
122.748207092285 -9.14926433563232\\
122.773208618164 -9.1502571105957\\
122.798210144043 -9.15124893188477\\
122.823204040527 -9.15223598480225\\
122.848205566406 -9.15321826934814\\
122.873207092285 -9.15420341491699\\
122.898208618164 -9.15519237518311\\
122.923210144043 -9.156174659729\\
122.948204040527 -9.1571569442749\\
122.973205566406 -9.1581392288208\\
122.998207092285 -9.15912055969238\\
123.023208618164 -9.16009330749512\\
123.048210144043 -9.16106605529785\\
123.073204040527 -9.16203880310059\\
123.098205566406 -9.16300964355469\\
123.123207092285 -9.16397380828857\\
123.148208618164 -9.16493701934814\\
123.173210144043 -9.16590023040771\\
123.198204040527 -9.16686153411865\\
123.223205566406 -9.16781520843506\\
123.248207092285 -9.16876888275146\\
123.273208618164 -9.16972064971924\\
123.298210144043 -9.17067050933838\\
123.323204040527 -9.17161464691162\\
123.348205566406 -9.17255878448486\\
123.373207092285 -9.17350196838379\\
123.398208618164 -9.17444038391113\\
123.423210144043 -9.17537498474121\\
123.448204040527 -9.17630672454834\\
123.473205566406 -9.17723846435547\\
123.498207092285 -9.17816638946533\\
123.523208618164 -9.17909049987793\\
123.548210144043 -9.18001174926758\\
123.573204040527 -9.18092727661133\\
123.598205566406 -9.18184280395508\\
123.623207092285 -9.18275165557861\\
123.648208618164 -9.1836576461792\\
123.673210144043 -9.18456363677979\\
123.698204040527 -9.18546581268311\\
123.723205566406 -9.18636131286621\\
123.748207092285 -9.18725490570068\\
123.773208618164 -9.18814468383789\\
123.798210144043 -9.18903160095215\\
123.823204040527 -9.18991565704346\\
123.848205566406 -9.19079303741455\\
123.873207092285 -9.19166946411133\\
123.898208618164 -9.19254398345947\\
123.923210144043 -9.19341659545898\\
123.948204040527 -9.19428443908691\\
123.973205566406 -9.19515037536621\\
123.998207092285 -9.19601345062256\\
124.023208618164 -9.19687175750732\\
124.048210144043 -9.19773006439209\\
124.073204040527 -9.19858646392822\\
124.098205566406 -9.19944190979004\\
124.123207092285 -9.20029735565186\\
124.148208618164 -9.20114707946777\\
124.173210144043 -9.20199584960938\\
124.198204040527 -9.20284271240234\\
124.223205566406 -9.203688621521\\
124.248207092285 -9.20453453063965\\
124.273208618164 -9.20537567138672\\
124.298210144043 -9.20621681213379\\
124.323204040527 -9.20706367492676\\
124.348205566406 -9.20791244506836\\
124.373207092285 -9.20875835418701\\
124.398208618164 -9.20959758758545\\
124.423210144043 -9.21042919158936\\
124.448204040527 -9.211256980896\\
124.473205566406 -9.21207714080811\\
124.498207092285 -9.21289443969727\\
124.523208618164 -9.21370506286621\\
124.548210144043 -9.21450805664063\\
124.573204040527 -9.21530914306641\\
124.598205566406 -9.21610832214355\\
124.623207092285 -9.21690082550049\\
124.648208618164 -9.21769237518311\\
124.673210144043 -9.21847915649414\\
124.698204040527 -9.21926116943359\\
124.723205566406 -9.22004318237305\\
124.748207092285 -9.2208251953125\\
124.773208618164 -9.22160720825195\\
124.798210144043 -9.22238922119141\\
124.823204040527 -9.22317123413086\\
124.848205566406 -9.22395324707031\\
124.873207092285 -9.22473526000977\\
124.898208618164 -9.22551727294922\\
124.923210144043 -9.22630310058594\\
124.948204040527 -9.22709274291992\\
124.973205566406 -9.22788429260254\\
124.998207092285 -9.22867584228516\\
125.023208618164 -9.22946643829346\\
125.048210144043 -9.23025131225586\\
125.073204040527 -9.231032371521\\
125.098205566406 -9.23180961608887\\
125.123207092285 -9.23258209228516\\
125.148208618164 -9.23335456848145\\
125.173210144043 -9.2341194152832\\
125.198204040527 -9.23488235473633\\
125.223205566406 -9.23563861846924\\
125.248207092285 -9.2363920211792\\
125.273208618164 -9.23713874816895\\
125.298210144043 -9.23788166046143\\
125.323204040527 -9.23861789703369\\
125.348205566406 -9.23935222625732\\
125.373207092285 -9.24007987976074\\
125.398208618164 -9.24080467224121\\
125.423210144043 -9.24152660369873\\
125.448204040527 -9.24224376678467\\
125.473205566406 -9.24295997619629\\
125.498207092285 -9.24368095397949\\
125.523208618164 -9.2443962097168\\
125.548210144043 -9.2451114654541\\
125.573204040527 -9.24582672119141\\
125.598205566406 -9.24654197692871\\
125.623207092285 -9.24725723266602\\
125.648208618164 -9.24797248840332\\
125.673210144043 -9.24868583679199\\
125.698204040527 -9.24939727783203\\
125.723205566406 -9.25010776519775\\
125.748207092285 -9.25081348419189\\
125.773208618164 -9.25151920318604\\
125.798210144043 -9.25222492218018\\
125.823204040527 -9.25293064117432\\
125.848205566406 -9.25363636016846\\
125.873207092285 -9.2543420791626\\
125.898208618164 -9.25504779815674\\
125.923210144043 -9.25575351715088\\
125.948204040527 -9.25645923614502\\
125.973205566406 -9.25716114044189\\
125.998207092285 -9.25785732269287\\
126.023208618164 -9.25855350494385\\
126.048210144043 -9.25924491882324\\
126.073204040527 -9.25993156433105\\
126.098205566406 -9.26061630249023\\
126.123207092285 -9.26129341125488\\
126.148208618164 -9.26197052001953\\
126.173210144043 -9.26263999938965\\
126.198204040527 -9.26330661773682\\
126.223205566406 -9.26396560668945\\
126.248207092285 -9.26462364196777\\
126.273208618164 -9.26528167724609\\
126.298210144043 -9.26593208312988\\
126.323204040527 -9.26658058166504\\
126.348205566406 -9.2672290802002\\
126.373207092285 -9.26787757873535\\
126.398208618164 -9.26852035522461\\
126.423210144043 -9.2691593170166\\
126.448204040527 -9.26979827880859\\
126.473205566406 -9.27043724060059\\
126.498207092285 -9.27107620239258\\
126.523208618164 -9.27171516418457\\
126.548210144043 -9.27235412597656\\
126.573204040527 -9.27299308776855\\
126.598205566406 -9.27363204956055\\
126.623207092285 -9.27427101135254\\
126.648208618164 -9.27490997314453\\
126.673210144043 -9.27554893493652\\
126.698204040527 -9.27618789672852\\
126.723205566406 -9.27682685852051\\
126.748207092285 -9.2774658203125\\
126.773208618164 -9.27810382843018\\
126.798210144043 -9.27873992919922\\
126.823204040527 -9.27936935424805\\
126.848205566406 -9.27999877929688\\
126.873207092285 -9.2806282043457\\
126.898208618164 -9.28125381469727\\
126.923210144043 -9.28187370300293\\
126.948204040527 -9.28249359130859\\
126.973205566406 -9.28310680389404\\
126.998207092285 -9.28371715545654\\
127.023208618164 -9.28432083129883\\
127.048210144043 -9.28492164611816\\
127.073204040527 -9.2855224609375\\
127.098205566406 -9.28611946105957\\
127.123207092285 -9.28671073913574\\
127.148208618164 -9.28730201721191\\
127.173210144043 -9.28789329528809\\
127.198204040527 -9.28848266601563\\
127.223205566406 -9.28907108306885\\
127.248207092285 -9.28966045379639\\
127.273208618164 -9.29025173187256\\
127.298210144043 -9.29084205627441\\
127.323204040527 -9.29142379760742\\
127.348205566406 -9.29200077056885\\
127.373207092285 -9.29257297515869\\
127.398208618164 -9.29314517974854\\
127.423210144043 -9.29371547698975\\
127.448204040527 -9.29427814483643\\
127.473205566406 -9.29484081268311\\
127.498207092285 -9.29540348052979\\
127.523208618164 -9.29596614837646\\
127.548210144043 -9.29652881622314\\
127.573204040527 -9.29709339141846\\
127.598205566406 -9.29765701293945\\
127.623207092285 -9.29821968078613\\
127.648208618164 -9.29878616333008\\
127.673210144043 -9.29935836791992\\
127.698204040527 -9.29993057250977\\
127.723205566406 -9.30050277709961\\
127.748207092285 -9.30107498168945\\
127.773208618164 -9.3016471862793\\
127.798210144043 -9.30221939086914\\
127.823204040527 -9.30279159545898\\
127.848205566406 -9.30336380004883\\
127.873207092285 -9.30393600463867\\
127.898208618164 -9.30450820922852\\
127.923210144043 -9.30507850646973\\
127.948204040527 -9.3056468963623\\
127.973205566406 -9.30621433258057\\
127.998207092285 -9.30677700042725\\
128.023208618164 -9.30733966827393\\
128.048202514648 -9.30790233612061\\
128.073211669922 -9.30846500396729\\
128.098205566406 -9.30902767181396\\
128.12321472168 -9.30959033966064\\
128.148208618164 -9.31014537811279\\
128.173202514648 -9.31069469451904\\
128.198211669922 -9.31123733520508\\
128.223205566406 -9.31177139282227\\
128.24821472168 -9.3123025894165\\
128.273208618164 -9.31282711029053\\
128.298202514648 -9.31335163116455\\
128.323211669922 -9.31387519836426\\
128.348205566406 -9.31439304351807\\
128.37321472168 -9.31490802764893\\
128.398208618164 -9.31542301177979\\
128.423202514648 -9.31593799591064\\
128.448211669922 -9.3164529800415\\
128.473205566406 -9.31696796417236\\
128.49821472168 -9.31748294830322\\
128.523208618164 -9.31800270080566\\
128.548202514648 -9.31852626800537\\
128.573211669922 -9.31905078887939\\
128.598205566406 -9.31957626342773\\
128.62321472168 -9.32010936737061\\
128.648208618164 -9.32064247131348\\
128.673202514648 -9.32117652893066\\
128.698211669922 -9.32171535491943\\
128.723205566406 -9.32225704193115\\
128.74821472168 -9.32279968261719\\
128.773208618164 -9.32334136962891\\
128.798202514648 -9.32388496398926\\
128.823211669922 -9.32442855834961\\
128.848205566406 -9.32497024536133\\
128.87321472168 -9.32551288604736\\
128.898208618164 -9.32605171203613\\
128.923202514648 -9.32658576965332\\
128.948211669922 -9.32711982727051\\
128.973205566406 -9.3276538848877\\
128.99821472168 -9.32818794250488\\
129.023208618164 -9.32872200012207\\
129.048202514648 -9.32925605773926\\
129.073211669922 -9.32979583740234\\
129.098205566406 -9.33034133911133\\
129.12321472168 -9.33090019226074\\
129.148208618164 -9.33147144317627\\
129.173202514648 -9.3320484161377\\
129.198211669922 -9.33261299133301\\
129.223205566406 -9.33314990997314\\
129.24821472168 -9.33365345001221\\
129.273208618164 -9.33412837982178\\
129.298202514648 -9.3345890045166\\
129.323211669922 -9.33504390716553\\
129.348205566406 -9.33549213409424\\
129.37321472168 -9.33594036102295\\
129.398208618164 -9.33638858795166\\
129.423202514648 -9.33683967590332\\
129.448211669922 -9.3372974395752\\
129.473205566406 -9.33776092529297\\
129.49821472168 -9.33822822570801\\
129.523208618164 -9.33869552612305\\
129.548202514648 -9.33916568756104\\
129.573211669922 -9.33964252471924\\
129.598205566406 -9.34012699127197\\
129.62321472168 -9.34062004089355\\
129.648208618164 -9.34112739562988\\
129.673202514648 -9.34165954589844\\
129.698211669922 -9.34220695495605\\
129.723205566406 -9.34276103973389\\
129.74821472168 -9.34332847595215\\
129.773208618164 -9.34391212463379\\
129.798202514648 -9.34450817108154\\
129.823211669922 -9.34509944915771\\
129.848205566406 -9.34568023681641\\
129.87321472168 -9.34624671936035\\
129.898208618164 -9.3467960357666\\
129.923202514648 -9.34732818603516\\
129.948211669922 -9.34784603118896\\
129.973205566406 -9.34834861755371\\
129.99821472168 -9.34883785247803\\
130.023208618164 -9.34931373596191\\
130.048202514648 -9.34977722167969\\
130.073211669922 -9.3502311706543\\
130.098205566406 -9.35067844390869\\
130.12321472168 -9.35111904144287\\
130.148208618164 -9.35155773162842\\
130.173202514648 -9.35199642181396\\
130.198211669922 -9.35243511199951\\
130.223205566406 -9.35287380218506\\
130.24821472168 -9.35331916809082\\
130.273208618164 -9.35376644134521\\
130.298202514648 -9.35421562194824\\
130.323211669922 -9.35467338562012\\
130.348205566406 -9.35514068603516\\
130.37321472168 -9.35562038421631\\
130.398208618164 -9.35611057281494\\
130.423202514648 -9.35661220550537\\
130.448211669922 -9.35712146759033\\
130.473205566406 -9.35764312744141\\
130.49821472168 -9.35817718505859\\
130.523208618164 -9.35871505737305\\
130.548202514648 -9.35925960540771\\
130.573211669922 -9.35981273651123\\
130.598205566406 -9.36036491394043\\
130.62321472168 -9.3609094619751\\
130.648208618164 -9.36144828796387\\
130.673202514648 -9.36197566986084\\
130.698211669922 -9.36249256134033\\
130.723205566406 -9.36299705505371\\
130.74821472168 -9.36349201202393\\
130.773208618164 -9.36397552490234\\
130.798202514648 -9.36445045471191\\
130.823211669922 -9.36491775512695\\
130.848205566406 -9.36537647247314\\
130.87321472168 -9.3658332824707\\
130.898208618164 -9.36628150939941\\
130.923202514648 -9.36672973632813\\
130.948211669922 -9.36717796325684\\
130.973205566406 -9.36762619018555\\
130.99821472168 -9.36807346343994\\
131.023208618164 -9.36852169036865\\
131.048202514648 -9.36897087097168\\
131.073211669922 -9.36942863464355\\
131.098205566406 -9.3698902130127\\
131.12321472168 -9.37035942077637\\
131.148208618164 -9.37083530426025\\
131.173202514648 -9.3713207244873\\
131.198211669922 -9.37181663513184\\
131.223205566406 -9.37232208251953\\
131.24821472168 -9.37283706665039\\
131.273208618164 -9.37335872650146\\
131.298202514648 -9.37388324737549\\
131.323211669922 -9.37440776824951\\
131.348205566406 -9.37493228912354\\
131.37321472168 -9.37546348571777\\
131.398208618164 -9.37599086761475\\
131.423202514648 -9.37653350830078\\
131.448211669922 -9.37709140777588\\
131.473205566406 -9.37766265869141\\
131.49821472168 -9.3782434463501\\
131.523208618164 -9.37882137298584\\
131.548202514648 -9.3793888092041\\
131.573211669922 -9.37993240356445\\
131.598205566406 -9.38046932220459\\
131.62321472168 -9.38101577758789\\
131.648208618164 -9.38156795501709\\
131.673202514648 -9.38212776184082\\
131.698211669922 -9.3826904296875\\
131.723205566406 -9.38325881958008\\
131.74821472168 -9.3838415145874\\
131.773208618164 -9.38450908660889\\
131.798202514648 -9.38744926452637\\
131.823211669922 -9.40704250335693\\
131.848205566406 -9.44504833221436\\
131.87321472168 -9.48420429229736\\
131.898208618164 -9.52077007293701\\
131.923202514648 -9.55350780487061\\
131.948211669922 -9.58154678344727\\
131.973205566406 -9.60583400726318\\
131.99821472168 -9.62870216369629\\
132.023208618164 -9.64956951141357\\
132.048202514648 -9.66819477081299\\
132.073211669922 -9.68378353118896\\
132.098205566406 -9.6957893371582\\
132.12321472168 -9.73848915100098\\
132.148208618164 -9.81881237030029\\
132.164413452148 -9.86783123016357\\
132.18798828125 -9.92541599273682\\
132.202728271484 -9.94437694549561\\
132.21891784668 -9.95090961456299\\
132.243927001953 -9.95661354064941\\
132.268920898438 -9.96133136749268\\
132.293914794922 -9.96580600738525\\
132.318923950195 -9.97011947631836\\
132.34358215332 -9.97430896759033\\
132.363616943359 -9.97822093963623\\
132.388610839844 -10.0029802322388\\
132.413604736328 -10.103346824646\\
132.438613891602 -10.1618671417236\\
132.463607788086 -10.1993131637573\\
132.488616943359 -10.2292003631592\\
132.513610839844 -10.2564649581909\\
132.538604736328 -10.2809200286865\\
132.563613891602 -10.301947593689\\
132.588607788086 -10.3192853927612\\
132.613616943359 -10.3710556030273\\
132.638610839844 -10.4162521362305\\
132.663604736328 -10.4687252044678\\
132.688613891602 -10.5225076675415\\
132.713607788086 -10.5651741027832\\
132.738616943359 -10.5906076431274\\
132.763610839844 -10.5980281829834\\
132.788604736328 -10.6292219161987\\
132.813613891602 -10.7004356384277\\
132.838607788086 -10.7524681091309\\
132.863616943359 -10.7679014205933\\
132.888610839844 -10.7699012756348\\
132.913604736328 -10.7706670761108\\
132.938613891602 -10.771523475647\\
132.963607788086 -10.8110904693604\\
132.988616943359 -10.8846225738525\\
133.013610839844 -10.9308624267578\\
133.038604736328 -10.9423418045044\\
133.063613891602 -10.9444055557251\\
133.088607788086 -10.945704460144\\
133.113616943359 -10.9466609954834\\
133.138610839844 -10.9474296569824\\
133.163604736328 -10.9477033615112\\
133.188613891602 -10.9482164382935\\
133.213607788086 -10.9489421844482\\
133.238616943359 -10.9494552612305\\
133.263610839844 -10.9498119354248\\
133.288604736328 -10.9501523971558\\
133.313613891602 -10.9505453109741\\
133.338607788086 -10.9510164260864\\
133.363616943359 -10.9515218734741\\
133.388610839844 -10.9520025253296\\
133.413604736328 -10.9524459838867\\
133.438613891602 -10.9528484344482\\
133.463607788086 -10.9532051086426\\
133.488616943359 -10.9535055160522\\
133.513610839844 -10.9537410736084\\
133.538604736328 -10.9539489746094\\
133.563613891602 -10.9541444778442\\
133.588607788086 -10.9543609619141\\
133.613616943359 -10.9545927047729\\
133.638610839844 -10.9548435211182\\
133.663604736328 -10.9551544189453\\
133.688613891602 -10.9555768966675\\
133.713607788086 -10.9560298919678\\
133.738616943359 -10.9564552307129\\
133.763610839844 -10.9567766189575\\
133.788604736328 -10.9570198059082\\
133.813613891602 -10.9572296142578\\
133.838607788086 -10.9574165344238\\
133.863616943359 -10.9576091766357\\
133.888610839844 -10.9578084945679\\
133.913604736328 -10.9579944610596\\
133.938613891602 -10.958197593689\\
133.963607788086 -10.9584970474243\\
133.988616943359 -10.9588975906372\\
134.013610839844 -10.959379196167\\
134.038604736328 -10.959755897522\\
134.063613891602 -10.9601259231567\\
134.088607788086 -10.9605922698975\\
134.113616943359 -10.9611787796021\\
134.138610839844 -10.9619359970093\\
134.163604736328 -10.9627752304077\\
134.188613891602 -10.9636745452881\\
134.213607788086 -10.9649248123169\\
134.238616943359 -10.9688949584961\\
134.263610839844 -10.9874515533447\\
134.288604736328 -11.0324831008911\\
134.313613891602 -11.0433292388916\\
134.338607788086 -11.04616355896\\
134.363616943359 -11.0483427047729\\
134.388610839844 -11.0500068664551\\
134.413604736328 -11.0513153076172\\
134.438613891602 -11.0523815155029\\
134.463607788086 -11.0531253814697\\
134.488616943359 -11.0536098480225\\
134.513610839844 -11.0541839599609\\
134.538604736328 -11.0548486709595\\
134.563613891602 -11.0554151535034\\
134.588607788086 -11.0559129714966\\
134.613616943359 -11.0564212799072\\
134.638610839844 -11.0569181442261\\
134.663604736328 -11.0573892593384\\
134.688613891602 -11.0578260421753\\
134.713607788086 -11.058274269104\\
134.738616943359 -11.0587768554688\\
134.763610839844 -11.0592880249023\\
134.788604736328 -11.0597648620605\\
134.813613891602 -11.0602321624756\\
134.838607788086 -11.0607089996338\\
134.863616943359 -11.0611686706543\\
134.888610839844 -11.0616073608398\\
134.913604736328 -11.0620069503784\\
134.938613891602 -11.0623731613159\\
134.963607788086 -11.0626935958862\\
134.988616943359 -11.0629739761353\\
135.013610839844 -11.0632371902466\\
135.038604736328 -11.063494682312\\
135.063613891602 -11.0637788772583\\
135.088607788086 -11.064076423645\\
135.113616943359 -11.0644035339355\\
135.138610839844 -11.0647563934326\\
135.163604736328 -11.0651264190674\\
135.188613891602 -11.0655107498169\\
135.213607788086 -11.0659093856812\\
135.238616943359 -11.0663242340088\\
135.263610839844 -11.0667533874512\\
135.288604736328 -11.0671968460083\\
135.313613891602 -11.0676488876343\\
135.338607788086 -11.0681104660034\\
135.363616943359 -11.0685806274414\\
135.388610839844 -11.0690546035767\\
135.413604736328 -11.0695266723633\\
135.438613891602 -11.0699968338013\\
135.463607788086 -11.0704641342163\\
135.488616943359 -11.0709362030029\\
135.513610839844 -11.0714092254639\\
135.538604736328 -11.0718755722046\\
135.563613891602 -11.0723333358765\\
135.588607788086 -11.0727815628052\\
135.613616943359 -11.0732202529907\\
135.638610839844 -11.0736455917358\\
135.663604736328 -11.0740575790405\\
135.688613891602 -11.0745029449463\\
135.713607788086 -11.0749521255493\\
135.738616943359 -11.0753660202026\\
135.763610839844 -11.0757417678833\\
135.788604736328 -11.0760908126831\\
135.813613891602 -11.0764303207397\\
135.838607788086 -11.0767850875854\\
135.863616943359 -11.0772171020508\\
135.888610839844 -11.0777454376221\\
135.913604736328 -11.0782499313354\\
135.938613891602 -11.0787363052368\\
135.963607788086 -11.0792121887207\\
135.988616943359 -11.0796680450439\\
136.013610839844 -11.0802097320557\\
136.038604736328 -11.0809669494629\\
136.063613891602 -11.0817842483521\\
136.088607788086 -11.0826063156128\\
136.113616943359 -11.083441734314\\
136.138610839844 -11.0842990875244\\
136.163604736328 -11.08518409729\\
136.188613891602 -11.0861024856567\\
136.213607788086 -11.0870628356934\\
136.238616943359 -11.0880556106567\\
136.263610839844 -11.0889501571655\\
136.288604736328 -11.089879989624\\
136.313613891602 -11.0908823013306\\
136.338607788086 -11.0919523239136\\
136.363616943359 -11.093092918396\\
136.388610839844 -11.0942649841309\\
136.413604736328 -11.0954351425171\\
136.438613891602 -11.0965852737427\\
136.463607788086 -11.0977067947388\\
136.488616943359 -11.0987854003906\\
136.513610839844 -11.0999250411987\\
136.538604736328 -11.1010665893555\\
136.563613891602 -11.1020421981812\\
136.588607788086 -11.1028528213501\\
136.613616943359 -11.103627204895\\
136.638610839844 -11.1043939590454\\
136.663604736328 -11.105170249939\\
136.688613891602 -11.1059465408325\\
136.713607788086 -11.10671043396\\
136.738616943359 -11.1074733734131\\
136.763610839844 -11.108247756958\\
136.788604736328 -11.1090211868286\\
136.813613891602 -11.1097860336304\\
136.838607788086 -11.1105709075928\\
136.863616943359 -11.1113653182983\\
136.888610839844 -11.1121835708618\\
136.913604736328 -11.1129741668701\\
136.938613891602 -11.1137027740479\\
136.963607788086 -11.1144180297852\\
136.988616943359 -11.1151332855225\\
137.013610839844 -11.1158533096313\\
137.038604736328 -11.1165685653687\\
137.063613891602 -11.117283821106\\
137.088607788086 -11.117995262146\\
137.113616943359 -11.1186828613281\\
137.138610839844 -11.1193046569824\\
137.163604736328 -11.1198806762695\\
137.188613891602 -11.1204528808594\\
137.213607788086 -11.1210241317749\\
137.238616943359 -11.1215896606445\\
137.263610839844 -11.1221628189087\\
137.288604736328 -11.1227626800537\\
137.313613891602 -11.1233949661255\\
137.338607788086 -11.1240406036377\\
137.363616943359 -11.1246891021729\\
137.388610839844 -11.1253423690796\\
137.413604736328 -11.1260108947754\\
137.438613891602 -11.1267099380493\\
137.463607788086 -11.1274423599243\\
137.488616943359 -11.1282072067261\\
137.513610839844 -11.1289939880371\\
137.538604736328 -11.1296997070313\\
137.563613891602 -11.1302328109741\\
137.588607788086 -11.1307497024536\\
137.613616943359 -11.1313104629517\\
137.638610839844 -11.1319437026978\\
137.663604736328 -11.1326856613159\\
137.688613891602 -11.1335830688477\\
137.713607788086 -11.1346817016602\\
137.738616943359 -11.1359987258911\\
137.763610839844 -11.1374797821045\\
137.788604736328 -11.1389970779419\\
137.813613891602 -11.1403560638428\\
137.838607788086 -11.1416387557983\\
137.863616943359 -11.1430540084839\\
137.888610839844 -11.1446399688721\\
137.913604736328 -11.1463031768799\\
137.938613891602 -11.1477460861206\\
137.963607788086 -11.1489763259888\\
137.988616943359 -11.1502466201782\\
138.013610839844 -11.1514778137207\\
138.038604736328 -11.1525945663452\\
138.063613891602 -11.1535968780518\\
138.088607788086 -11.1545305252075\\
138.113616943359 -11.1554327011108\\
138.138610839844 -11.1563110351563\\
138.163604736328 -11.1571598052979\\
138.188613891602 -11.1579647064209\\
138.213607788086 -11.1587734222412\\
138.238616943359 -11.1596260070801\\
138.263610839844 -11.1605367660522\\
138.288604736328 -11.161527633667\\
138.313613891602 -11.1625776290894\\
138.338607788086 -11.1637325286865\\
138.363616943359 -11.1649827957153\\
138.388610839844 -11.1663694381714\\
138.413604736328 -11.1678438186646\\
138.438613891602 -11.1694574356079\\
138.463607788086 -11.1712579727173\\
138.488616943359 -11.1731023788452\\
138.513610839844 -11.1749629974365\\
138.538604736328 -11.1769771575928\\
138.563613891602 -11.1791410446167\\
138.588607788086 -11.1813297271729\\
138.613616943359 -11.1834354400635\\
138.638610839844 -11.1856079101563\\
138.663604736328 -11.1878814697266\\
138.688613891602 -11.1902170181274\\
138.713607788086 -11.1925001144409\\
138.738616943359 -11.1946926116943\\
138.763610839844 -11.1969003677368\\
138.788604736328 -11.1991348266602\\
138.813613891602 -11.2013893127441\\
138.838607788086 -11.2036809921265\\
138.863616943359 -11.2060108184814\\
138.888610839844 -11.2082529067993\\
138.913604736328 -11.2104511260986\\
138.938613891602 -11.2126531600952\\
138.963607788086 -11.2148542404175\\
138.988616943359 -11.2170562744141\\
139.013610839844 -11.2192392349243\\
139.038604736328 -11.221360206604\\
139.063613891602 -11.2234582901001\\
139.088607788086 -11.2255544662476\\
139.113616943359 -11.227650642395\\
139.138610839844 -11.2297458648682\\
139.163604736328 -11.231822013855\\
139.188613891602 -11.2338752746582\\
139.213607788086 -11.2359142303467\\
139.238616943359 -11.2379322052002\\
139.263610839844 -11.2399301528931\\
139.288604736328 -11.2419118881226\\
139.313613891602 -11.243878364563\\
139.338607788086 -11.2458276748657\\
139.363616943359 -11.247766494751\\
139.388610839844 -11.2496871948242\\
139.413604736328 -11.2515869140625\\
139.438613891602 -11.2534675598145\\
139.463607788086 -11.2553272247314\\
139.488616943359 -11.2571687698364\\
139.513610839844 -11.2589902877808\\
139.538604736328 -11.2607946395874\\
139.563613891602 -11.2625789642334\\
139.588607788086 -11.2643404006958\\
139.613616943359 -11.2660827636719\\
139.638610839844 -11.2677974700928\\
139.663604736328 -11.2694911956787\\
139.688613891602 -11.2711658477783\\
139.713607788086 -11.2728147506714\\
139.738616943359 -11.2744398117065\\
139.763610839844 -11.2760400772095\\
139.788604736328 -11.2776165008545\\
139.813613891602 -11.2791681289673\\
139.838607788086 -11.2806949615479\\
139.863616943359 -11.2821998596191\\
139.888610839844 -11.2836847305298\\
139.913604736328 -11.2851505279541\\
139.938613891602 -11.2865962982178\\
139.963607788086 -11.2880220413208\\
139.988616943359 -11.2894268035889\\
140.013610839844 -11.290810585022\\
140.038604736328 -11.2921752929688\\
140.063613891602 -11.2935199737549\\
140.088607788086 -11.294846534729\\
140.113616943359 -11.2961540222168\\
140.138610839844 -11.2974433898926\\
140.163604736328 -11.298716545105\\
140.188613891602 -11.2999725341797\\
140.213607788086 -11.301212310791\\
140.238616943359 -11.3024368286133\\
140.263610839844 -11.3036422729492\\
140.288604736328 -11.3048295974731\\
140.313613891602 -11.3060026168823\\
140.338607788086 -11.3071584701538\\
140.363616943359 -11.3083000183105\\
140.388610839844 -11.3094263076782\\
140.413604736328 -11.3105401992798\\
140.438613891602 -11.3116416931152\\
140.463607788086 -11.3127298355103\\
140.488616943359 -11.3138055801392\\
140.513610839844 -11.3148622512817\\
140.538604736328 -11.3159036636353\\
140.563613891602 -11.316933631897\\
140.588607788086 -11.3179531097412\\
140.613616943359 -11.318962097168\\
140.638610839844 -11.3199605941772\\
140.663604736328 -11.3209457397461\\
140.688613891602 -11.3219137191772\\
140.713607788086 -11.3228616714478\\
140.738616943359 -11.3237867355347\\
140.763610839844 -11.3246755599976\\
140.788604736328 -11.3254957199097\\
140.813613891602 -11.3263702392578\\
140.838607788086 -11.3272075653076\\
140.863616943359 -11.3279943466187\\
140.888610839844 -11.3287439346313\\
140.913604736328 -11.3294668197632\\
140.938613891602 -11.3301668167114\\
140.963607788086 -11.3308401107788\\
140.988616943359 -11.331485748291\\
141.013610839844 -11.3321104049683\\
141.038604736328 -11.3327169418335\\
141.063613891602 -11.3333044052124\\
141.088607788086 -11.3338737487793\\
141.113616943359 -11.3344230651855\\
141.138610839844 -11.3349533081055\\
141.163604736328 -11.3354635238647\\
141.188613891602 -11.3359441757202\\
141.213607788086 -11.3363552093506\\
141.238616943359 -11.3367099761963\\
141.263610839844 -11.3370141983032\\
141.288604736328 -11.3372955322266\\
141.313613891602 -11.3376226425171\\
141.338607788086 -11.337984085083\\
141.363616943359 -11.3383674621582\\
141.388610839844 -11.3387727737427\\
141.413604736328 -11.3392009735107\\
141.438613891602 -11.339653968811\\
141.463607788086 -11.340145111084\\
141.488616943359 -11.3406820297241\\
141.513610839844 -11.341251373291\\
141.538604736328 -11.3418340682983\\
141.563613891602 -11.3423748016357\\
141.588607788086 -11.3428287506104\\
141.613616943359 -11.3432121276855\\
141.638610839844 -11.3435745239258\\
141.663604736328 -11.3439521789551\\
141.688613891602 -11.3443622589111\\
141.713607788086 -11.3448104858398\\
141.738616943359 -11.3452920913696\\
141.763610839844 -11.3458003997803\\
141.788604736328 -11.3463115692139\\
141.813613891602 -11.3467292785645\\
141.838607788086 -11.347020149231\\
141.863616943359 -11.347373008728\\
141.888610839844 -11.3477802276611\\
141.913604736328 -11.3482027053833\\
141.938613891602 -11.3486318588257\\
141.963607788086 -11.349048614502\\
141.988616943359 -11.3494462966919\\
142.013610839844 -11.3497838973999\\
142.038604736328 -11.3500843048096\\
142.063613891602 -11.3504228591919\\
142.088607788086 -11.350793838501\\
142.113616943359 -11.3511438369751\\
142.138610839844 -11.3514680862427\\
142.163604736328 -11.3518209457397\\
142.188613891602 -11.3522109985352\\
142.213607788086 -11.3525991439819\\
142.238616943359 -11.352991104126\\
142.263610839844 -11.3534002304077\\
142.288604736328 -11.3538160324097\\
142.313613891602 -11.3542394638062\\
142.338607788086 -11.3546743392944\\
142.363616943359 -11.3551177978516\\
142.388610839844 -11.3555679321289\\
142.413604736328 -11.356029510498\\
142.438613891602 -11.3564968109131\\
142.463607788086 -11.3569631576538\\
142.488616943359 -11.3574209213257\\
142.513610839844 -11.3578748703003\\
142.538604736328 -11.3583011627197\\
142.563613891602 -11.3585634231567\\
142.588607788086 -11.3586959838867\\
142.613616943359 -11.3588714599609\\
142.638610839844 -11.3591480255127\\
142.663604736328 -11.3594627380371\\
142.688613891602 -11.3597707748413\\
142.713607788086 -11.3600673675537\\
142.738616943359 -11.3603591918945\\
142.763610839844 -11.3606538772583\\
142.788604736328 -11.3609647750854\\
142.813613891602 -11.3613080978394\\
142.838607788086 -11.361686706543\\
142.863616943359 -11.3620796203613\\
142.888610839844 -11.3624897003174\\
142.913604736328 -11.3629159927368\\
142.938613891602 -11.3633451461792\\
142.963607788086 -11.3637771606445\\
142.988616943359 -11.3642416000366\\
143.013610839844 -11.3647928237915\\
143.038604736328 -11.365626335144\\
143.063613891602 -11.3668432235718\\
143.088607788086 -11.3682565689087\\
143.113616943359 -11.3696613311768\\
143.138610839844 -11.3710823059082\\
143.163604736328 -11.3726148605347\\
143.188613891602 -11.3742666244507\\
143.213607788086 -11.376012802124\\
143.238616943359 -11.3778305053711\\
143.263610839844 -11.3796052932739\\
143.288604736328 -11.380970954895\\
143.313613891602 -11.3820581436157\\
143.338607788086 -11.3831853866577\\
143.363616943359 -11.3843193054199\\
143.388610839844 -11.3854141235352\\
143.413604736328 -11.3864641189575\\
143.438613891602 -11.3874769210815\\
143.463607788086 -11.3884687423706\\
143.488616943359 -11.3894834518433\\
143.513610839844 -11.3907070159912\\
143.538604736328 -11.3925657272339\\
143.563613891602 -11.3950052261353\\
143.588607788086 -11.3975849151611\\
143.613616943359 -11.3999061584473\\
143.638610839844 -11.4012546539307\\
143.663604736328 -11.4022941589355\\
143.688613891602 -11.4034852981567\\
143.713607788086 -11.4048585891724\\
143.738616943359 -11.4063720703125\\
143.763610839844 -11.4079513549805\\
143.788604736328 -11.4095315933228\\
143.813613891602 -11.4111404418945\\
143.838607788086 -11.4127731323242\\
143.863616943359 -11.4144353866577\\
143.888610839844 -11.4161262512207\\
143.913604736328 -11.4178352355957\\
143.938613891602 -11.4195699691772\\
143.963607788086 -11.4214200973511\\
143.988616943359 -11.4233207702637\\
144.013610839844 -11.4251728057861\\
144.038604736328 -11.426944732666\\
144.063613891602 -11.4286823272705\\
144.088607788086 -11.4303855895996\\
144.113616943359 -11.4320430755615\\
144.138610839844 -11.433614730835\\
144.163604736328 -11.4350576400757\\
144.188613891602 -11.436427116394\\
144.213607788086 -11.4377479553223\\
144.238616943359 -11.4390392303467\\
144.263610839844 -11.4403076171875\\
144.288604736328 -11.4415531158447\\
144.313613891602 -11.4427928924561\\
144.338607788086 -11.4440317153931\\
144.363616943359 -11.4452495574951\\
144.388610839844 -11.4464473724365\\
144.413604736328 -11.4476289749146\\
144.438613891602 -11.4487676620483\\
144.463607788086 -11.4496288299561\\
144.488616943359 -11.4498634338379\\
144.513610839844 -11.4500904083252\\
144.538604736328 -11.4504518508911\\
144.563613891602 -11.450891494751\\
144.588607788086 -11.4513568878174\\
144.613616943359 -11.4518337249756\\
144.638610839844 -11.4523105621338\\
144.663604736328 -11.4527797698975\\
144.688613891602 -11.4532251358032\\
144.713607788086 -11.4536428451538\\
144.738616943359 -11.4540643692017\\
144.763610839844 -11.4545345306396\\
144.788604736328 -11.4550428390503\\
144.813613891602 -11.4555902481079\\
144.838607788086 -11.4562826156616\\
144.863616943359 -11.457293510437\\
144.888610839844 -11.4584178924561\\
144.913604736328 -11.4596328735352\\
144.938613891602 -11.4609336853027\\
144.963607788086 -11.4623098373413\\
144.988616943359 -11.4637508392334\\
145.013610839844 -11.4652280807495\\
145.038604736328 -11.4666013717651\\
145.063613891602 -11.4679431915283\\
145.088607788086 -11.4693460464478\\
145.113616943359 -11.4708108901978\\
145.138610839844 -11.4723262786865\\
145.163604736328 -11.4738988876343\\
145.188613891602 -11.475528717041\\
145.213607788086 -11.477165222168\\
145.238616943359 -11.4785985946655\\
145.263610839844 -11.4799814224243\\
145.288604736328 -11.4814596176147\\
145.313613891602 -11.4830341339111\\
145.338607788086 -11.4846677780151\\
145.363616943359 -11.4863433837891\\
145.388610839844 -11.4880380630493\\
145.413604736328 -11.4897403717041\\
145.438613891602 -11.4914255142212\\
145.463607788086 -11.492974281311\\
145.488616943359 -11.494402885437\\
145.513610839844 -11.4958696365356\\
145.538604736328 -11.4973764419556\\
145.563613891602 -11.4988632202148\\
145.588607788086 -11.5002527236938\\
145.613616943359 -11.5015726089478\\
145.638610839844 -11.5029315948486\\
145.663604736328 -11.504373550415\\
145.688613891602 -11.5057582855225\\
145.713607788086 -11.5070972442627\\
145.738616943359 -11.5084667205811\\
145.763610839844 -11.5098609924316\\
145.788604736328 -11.5112590789795\\
145.813613891602 -11.5126333236694\\
145.838607788086 -11.5139665603638\\
145.863616943359 -11.5152616500854\\
145.888610839844 -11.5164632797241\\
145.913604736328 -11.5175924301147\\
145.938613891602 -11.5187244415283\\
145.963607788086 -11.5198574066162\\
145.988616943359 -11.5209827423096\\
146.013610839844 -11.5220899581909\\
146.038604736328 -11.5231475830078\\
146.063613891602 -11.5241413116455\\
146.088607788086 -11.5251235961914\\
146.113616943359 -11.5261058807373\\
146.138610839844 -11.5270843505859\\
146.163604736328 -11.5280437469482\\
146.188613891602 -11.5289602279663\\
146.213607788086 -11.5298643112183\\
146.238616943359 -11.5307645797729\\
146.263610839844 -11.531644821167\\
146.288604736328 -11.5325107574463\\
146.313613891602 -11.5333690643311\\
146.338607788086 -11.5342092514038\\
146.363616943359 -11.5350351333618\\
146.388610839844 -11.5358505249023\\
146.413604736328 -11.5366535186768\\
146.438613891602 -11.5374402999878\\
146.463607788086 -11.5382127761841\\
146.488616943359 -11.5389699935913\\
146.513610839844 -11.5397090911865\\
146.538604736328 -11.5404281616211\\
146.563613891602 -11.5411310195923\\
146.588607788086 -11.5418176651001\\
146.613616943359 -11.5424890518188\\
146.638610839844 -11.5431423187256\\
146.663604736328 -11.5437831878662\\
146.688613891602 -11.5444135665894\\
146.713607788086 -11.5450325012207\\
146.738616943359 -11.5456428527832\\
146.763610839844 -11.5462427139282\\
146.788604736328 -11.5468339920044\\
146.813613891602 -11.547417640686\\
146.838607788086 -11.5479927062988\\
146.863616943359 -11.5485591888428\\
146.888610839844 -11.5491170883179\\
146.913604736328 -11.5496644973755\\
146.938613891602 -11.5502014160156\\
146.963607788086 -11.550726890564\\
146.988616943359 -11.5512409210205\\
147.013610839844 -11.5517425537109\\
147.038604736328 -11.5522308349609\\
147.063613891602 -11.5527076721191\\
147.088607788086 -11.5531721115112\\
147.113616943359 -11.5536251068115\\
147.138610839844 -11.554069519043\\
147.163604736328 -11.5545034408569\\
147.188613891602 -11.554931640625\\
147.213607788086 -11.5553512573242\\
147.238616943359 -11.5557661056519\\
147.263610839844 -11.5561761856079\\
147.288604736328 -11.5565853118896\\
147.313613891602 -11.5569944381714\\
147.338607788086 -11.5574026107788\\
147.363616943359 -11.5578117370605\\
147.388610839844 -11.5582141876221\\
147.413604736328 -11.5586156845093\\
147.438613891602 -11.5590162277222\\
147.463607788086 -11.5594167709351\\
147.488616943359 -11.5598173141479\\
147.513610839844 -11.5602178573608\\
147.538604736328 -11.5606155395508\\
147.563613891602 -11.5610065460205\\
147.588607788086 -11.5613965988159\\
147.613616943359 -11.5617790222168\\
147.638610839844 -11.562159538269\\
147.663604736328 -11.5625314712524\\
147.688613891602 -11.5628995895386\\
147.713607788086 -11.5632619857788\\
147.738616943359 -11.5636177062988\\
147.763610839844 -11.5639696121216\\
147.788604736328 -11.5643129348755\\
147.813613891602 -11.5646543502808\\
147.838607788086 -11.5649881362915\\
147.863616943359 -11.5653219223022\\
147.888610839844 -11.5656461715698\\
147.913604736328 -11.5659704208374\\
147.938613891602 -11.5662899017334\\
147.963607788086 -11.5666046142578\\
147.988616943359 -11.5669193267822\\
148.013610839844 -11.5672273635864\\
148.038604736328 -11.5675325393677\\
148.063613891602 -11.5678329467773\\
148.088607788086 -11.5681285858154\\
148.113616943359 -11.5684194564819\\
148.138610839844 -11.5687055587769\\
148.163604736328 -11.5689849853516\\
148.188613891602 -11.569260597229\\
148.213607788086 -11.5695276260376\\
148.238616943359 -11.5697898864746\\
148.263610839844 -11.57004737854\\
148.288604736328 -11.5702991485596\\
148.313613891602 -11.5705471038818\\
148.338607788086 -11.5707950592041\\
148.363616943359 -11.5710430145264\\
148.388610839844 -11.5712909698486\\
148.413604736328 -11.5715389251709\\
148.438613891602 -11.5717868804932\\
148.463607788086 -11.5720348358154\\
148.488616943359 -11.5722827911377\\
148.513610839844 -11.5725317001343\\
148.538604736328 -11.5727882385254\\
148.563613891602 -11.5730457305908\\
148.588607788086 -11.5733032226563\\
148.613616943359 -11.5735607147217\\
148.638610839844 -11.5738182067871\\
148.663604736328 -11.5740756988525\\
148.688613891602 -11.574333190918\\
148.713607788086 -11.5745906829834\\
148.738616943359 -11.5748462677002\\
148.763610839844 -11.5750942230225\\
148.788604736328 -11.5753421783447\\
148.813613891602 -11.575590133667\\
148.838607788086 -11.5758285522461\\
148.863616943359 -11.5760669708252\\
148.888610839844 -11.57630443573\\
148.913604736328 -11.5765352249146\\
148.938613891602 -11.5767641067505\\
148.963607788086 -11.5769929885864\\
148.988616943359 -11.5772218704224\\
149.013610839844 -11.5774507522583\\
149.038604736328 -11.5776796340942\\
149.063613891602 -11.5779085159302\\
149.088607788086 -11.5781373977661\\
149.113616943359 -11.5783662796021\\
149.138610839844 -11.578595161438\\
149.163604736328 -11.5788240432739\\
149.188613891602 -11.5790529251099\\
149.213607788086 -11.5792818069458\\
149.238616943359 -11.5795106887817\\
149.263610839844 -11.5797395706177\\
149.288604736328 -11.5799674987793\\
149.313613891602 -11.5801868438721\\
149.338607788086 -11.5804061889648\\
149.363616943359 -11.5806217193604\\
149.388610839844 -11.58083152771\\
149.413604736328 -11.5810403823853\\
149.438613891602 -11.581241607666\\
149.463607788086 -11.5814418792725\\
149.488616943359 -11.5816345214844\\
149.513610839844 -11.5818252563477\\
149.538604736328 -11.5820140838623\\
149.563613891602 -11.5821952819824\\
149.588607788086 -11.5823764801025\\
149.613616943359 -11.5825538635254\\
149.638610839844 -11.5827255249023\\
149.663604736328 -11.5828971862793\\
149.688613891602 -11.5830659866333\\
149.713607788086 -11.5832281112671\\
149.738616943359 -11.5833902359009\\
149.763610839844 -11.5835523605347\\
149.788604736328 -11.5837135314941\\
149.813613891602 -11.5838670730591\\
149.838607788086 -11.5840196609497\\
149.863616943359 -11.5841722488403\\
149.888610839844 -11.584324836731\\
149.913604736328 -11.5844764709473\\
149.938613891602 -11.5846223831177\\
149.963607788086 -11.5847654342651\\
149.988616943359 -11.5849084854126\\
};
\addlegendentry{$\dtmax{} = \expneg{2.5}{3}{[s]}$};

\addplot [
color=mycolor1,
solid,
forget plot
]
table[row sep=crcr]{
0 0\\
1.00055420398712 0\\
2.00055408477783 0\\
3.00055408477783 0\\
4.00055408477783 0\\
5.00055408477783 0\\
6.00055408477783 0\\
7.00055408477783 0\\
8.00055408477783 0\\
9.00055408477783 0\\
10.0005540847778 0\\
11.0005540847778 0\\
12.0005540847778 0\\
13.0005540847778 0\\
14.0005540847778 0\\
15.0005540847778 0\\
16.0005550384521 0\\
17.0005550384521 0\\
18.0005550384521 0\\
19.0005550384521 -1.12699849452724e-08\\
20.0005550384521 -8.20270543044899e-07\\
21.0005550384521 -0.548887193202972\\
22.0005550384521 -0.650461733341217\\
23.0006408691406 -0.685949206352234\\
24.0006408691406 -0.711122274398804\\
25.0006408691406 -0.732963085174561\\
26.0006408691406 -0.758409261703491\\
27.0006408691406 -0.789991915225983\\
28.0006408691406 -0.830740511417389\\
29.0006408691406 -0.883774220943451\\
30.0006408691406 -0.953247547149658\\
31.0006408691406 -1.04505157470703\\
32.0006408691406 -1.15915536880493\\
33.0006408691406 -1.28872978687286\\
34.0006408691406 -1.42769694328308\\
35.0006408691406 -1.57231330871582\\
36.0006408691406 -1.71494042873383\\
37.0006408691406 -1.85206854343414\\
38.0006408691406 -1.98236501216888\\
39.0006408691406 -2.11141300201416\\
40.0006408691406 -2.24019575119019\\
41.0006408691406 -2.36968302726746\\
42.0006408691406 -2.49782872200012\\
43.0006408691406 -2.62717247009277\\
44.0006408691406 -2.75741338729858\\
45.0006408691406 -2.88978886604309\\
46.0006408691406 -3.01370620727539\\
47.0006408691406 -3.13049674034119\\
48.0006408691406 -3.24518799781799\\
49.0006408691406 -3.36080193519592\\
50.0006408691406 -3.47648024559021\\
51.0006408691406 -3.58767008781433\\
52.0006408691406 -3.68480253219604\\
53.0006408691406 -3.77820944786072\\
54.0006408691406 -3.86728072166443\\
55.0006408691406 -3.95484352111816\\
56.0006408691406 -4.04140949249268\\
57.0006408691406 -4.12770700454712\\
58.0006408691406 -4.21496963500977\\
59.0006408691406 -4.30377340316772\\
60.0006408691406 -4.39482402801514\\
61.0006408691406 -4.4878511428833\\
62.0006408691406 -4.58062553405762\\
63.0006408691406 -4.67277336120605\\
64.0006408691406 -4.76538896560669\\
65.0006408691406 -4.86030578613281\\
66.0006408691406 -4.95492744445801\\
67.0006408691406 -5.04932069778442\\
68.0006408691406 -5.143723487854\\
69.0006408691406 -5.23998260498047\\
70.0006408691406 -5.33474731445313\\
71.0006408691406 -5.42863368988037\\
72.0006408691406 -5.52386283874512\\
73.0006408691406 -5.62032604217529\\
74.0006408691406 -5.71525430679321\\
75.0006408691406 -5.81080961227417\\
76.0006408691406 -5.90614891052246\\
77.0006408691406 -6.00279569625854\\
78.0006408691406 -6.09974431991577\\
79.0006408691406 -6.19847059249878\\
80.0006408691406 -6.29669904708862\\
81.0006408691406 -6.39477920532227\\
82.0006408691406 -6.49177551269531\\
83.0006408691406 -6.58825635910034\\
84.0006408691406 -6.68462133407593\\
85.0006408691406 -6.78049755096436\\
86.0006408691406 -6.87669372558594\\
87.0006408691406 -6.97214651107788\\
88.0006408691406 -7.06656503677368\\
89.0006408691406 -7.16002702713013\\
90.0006408691406 -7.25314378738403\\
91.0006408691406 -7.34696102142334\\
92.0006408691406 -7.43944406509399\\
93.0006408691406 -7.52564764022827\\
94.0006408691406 -7.60593414306641\\
95.0006408691406 -7.6819109916687\\
96.0006408691406 -7.75452470779419\\
97.0006408691406 -7.82453298568726\\
98.0006408691406 -7.89245939254761\\
99.0006408691406 -7.95899629592896\\
100 -8.02488422393799\\
100.012496948242 -8.02570247650146\\
100.025001525879 -8.02652168273926\\
100.037498474121 -8.02733993530273\\
100.050003051758 -8.02815914154053\\
100.0625 -8.028977394104\\
100.074996948242 -8.02979564666748\\
100.087501525879 -8.03061485290527\\
100.099998474121 -8.03143310546875\\
100.112503051758 -8.03225231170654\\
100.125 -8.03307056427002\\
100.137496948242 -8.0338888168335\\
100.150001525879 -8.03470802307129\\
100.162498474121 -8.03552341461182\\
100.175003051758 -8.03633403778076\\
100.1875 -8.03714275360107\\
100.199996948242 -8.03795909881592\\
100.212501525879 -8.03877925872803\\
100.224998474121 -8.03959846496582\\
100.237503051758 -8.04041290283203\\
100.25 -8.04122734069824\\
100.262496948242 -8.04204750061035\\
100.275001525879 -8.04286670684814\\
100.287498474121 -8.04368591308594\\
100.300003051758 -8.04451370239258\\
100.3125 -8.04534339904785\\
100.324996948242 -8.04617309570313\\
100.337501525879 -8.0470027923584\\
100.349998474121 -8.04783248901367\\
100.362503051758 -8.04866218566895\\
100.375 -8.04948997497559\\
100.387496948242 -8.05031681060791\\
100.400001525879 -8.05113697052002\\
100.412498474121 -8.0519552230835\\
100.425003051758 -8.05276870727539\\
100.4375 -8.05357837677002\\
100.449996948242 -8.05438613891602\\
100.462501525879 -8.0551872253418\\
100.474998474121 -8.05598735809326\\
100.487503051758 -8.05678749084473\\
100.5 -8.05758190155029\\
100.512496948242 -8.05837345123291\\
100.525001525879 -8.05916500091553\\
100.537498474121 -8.05995655059814\\
100.550003051758 -8.06074810028076\\
100.5625 -8.06153964996338\\
100.574996948242 -8.062331199646\\
100.587501525879 -8.06312942504883\\
100.599998474121 -8.06392955780029\\
100.612503051758 -8.06473541259766\\
100.625 -8.06554508209229\\
100.637496948242 -8.06636238098145\\
100.650001525879 -8.06718635559082\\
100.662498474121 -8.06801605224609\\
100.675003051758 -8.06885433197021\\
100.6875 -8.06969833374023\\
100.699996948242 -8.0705394744873\\
100.712501525879 -8.07138824462891\\
100.724998474121 -8.07224273681641\\
100.737503051758 -8.07309150695801\\
100.75 -8.07394027709961\\
100.762496948242 -8.07478904724121\\
100.775001525879 -8.0756368637085\\
100.787498474121 -8.07648372650146\\
100.800003051758 -8.07733154296875\\
100.8125 -8.07817840576172\\
100.824996948242 -8.07902526855469\\
100.837501525879 -8.07987308502197\\
100.849998474121 -8.08071994781494\\
100.862503051758 -8.08156776428223\\
100.875 -8.08240699768066\\
100.887496948242 -8.0832462310791\\
100.900001525879 -8.08408546447754\\
100.912498474121 -8.08492279052734\\
100.925003051758 -8.08576107025146\\
100.9375 -8.086594581604\\
100.949996948242 -8.08742427825928\\
100.962501525879 -8.08825397491455\\
100.974998474121 -8.08908176422119\\
100.987503051758 -8.08991050720215\\
101 -8.09073448181152\\
101.012496948242 -8.09155464172363\\
101.025001525879 -8.09237480163574\\
101.037498474121 -8.09319400787354\\
101.050003051758 -8.09401321411133\\
101.0625 -8.0948314666748\\
101.074996948242 -8.09564590454102\\
101.087501525879 -8.09645652770996\\
101.099998474121 -8.09726715087891\\
101.112503051758 -8.09807682037354\\
101.125 -8.09888553619385\\
101.137496948242 -8.09970283508301\\
101.150001525879 -8.10052299499512\\
101.162498474121 -8.10134315490723\\
101.175003051758 -8.10216331481934\\
101.1875 -8.10298347473145\\
101.199996948242 -8.10380268096924\\
101.212501525879 -8.10462188720703\\
101.224998474121 -8.10543918609619\\
101.237503051758 -8.10624980926514\\
101.25 -8.10706806182861\\
101.262496948242 -8.10788822174072\\
101.275001525879 -8.10870838165283\\
101.287498474121 -8.10952949523926\\
101.300003051758 -8.11036205291748\\
101.3125 -8.11119937896729\\
101.324996948242 -8.11203861236572\\
101.337501525879 -8.11287784576416\\
101.349998474121 -8.11371517181396\\
101.362503051758 -8.11455345153809\\
101.375 -8.11538791656494\\
101.387496948242 -8.11621761322021\\
101.400001525879 -8.11704635620117\\
101.412498474121 -8.11787223815918\\
101.425003051758 -8.11869239807129\\
101.4375 -8.11950874328613\\
101.449996948242 -8.12031936645508\\
101.462501525879 -8.12112903594971\\
101.474998474121 -8.12193489074707\\
101.487503051758 -8.12273597717285\\
101.5 -8.12353610992432\\
101.512496948242 -8.12433433532715\\
101.525001525879 -8.12512588500977\\
101.537498474121 -8.12591743469238\\
101.550003051758 -8.126708984375\\
101.5625 -8.12750053405762\\
101.574996948242 -8.12829208374023\\
101.587501525879 -8.12908363342285\\
101.599998474121 -8.12987518310547\\
101.612503051758 -8.13067150115967\\
101.625 -8.13147068023682\\
101.637496948242 -8.1322717666626\\
101.650001525879 -8.13307952880859\\
101.662498474121 -8.13388919830322\\
101.675003051758 -8.1347017288208\\
101.6875 -8.13551998138428\\
101.699996948242 -8.13633251190186\\
101.712501525879 -8.13715171813965\\
101.724998474121 -8.13797187805176\\
101.737503051758 -8.13879203796387\\
101.75 -8.13961219787598\\
101.762496948242 -8.14043140411377\\
101.775001525879 -8.14125156402588\\
101.787498474121 -8.14207172393799\\
101.800003051758 -8.1428918838501\\
101.8125 -8.14371204376221\\
101.824996948242 -8.14453220367432\\
101.837501525879 -8.14535236358643\\
101.849998474121 -8.14617252349854\\
101.862503051758 -8.14699268341064\\
101.875 -8.14781284332275\\
101.887496948242 -8.14863300323486\\
101.900001525879 -8.14945316314697\\
101.912498474121 -8.15027236938477\\
101.925003051758 -8.15109157562256\\
101.9375 -8.15191173553467\\
101.949996948242 -8.15273189544678\\
101.962501525879 -8.15355205535889\\
101.974998474121 -8.154372215271\\
101.987503051758 -8.15519237518311\\
102 -8.15601253509521\\
102.012496948242 -8.15683269500732\\
102.025001525879 -8.15765285491943\\
102.037498474121 -8.15847301483154\\
102.050003051758 -8.15929317474365\\
102.0625 -8.16011333465576\\
102.074996948242 -8.16093349456787\\
102.087501525879 -8.16176223754883\\
102.099998474121 -8.16259002685547\\
102.112503051758 -8.16341876983643\\
102.125 -8.16424655914307\\
102.137496948242 -8.16507434844971\\
102.150001525879 -8.16590309143066\\
102.162498474121 -8.1667308807373\\
102.175003051758 -8.16755962371826\\
102.1875 -8.1683874130249\\
102.199996948242 -8.16921520233154\\
102.212501525879 -8.17004013061523\\
102.224998474121 -8.17086029052734\\
102.237503051758 -8.17167949676514\\
102.25 -8.17250347137451\\
102.262496948242 -8.17333126068115\\
102.275001525879 -8.17416000366211\\
102.287498474121 -8.17498111724854\\
102.300003051758 -8.17580986022949\\
102.3125 -8.17664432525635\\
102.324996948242 -8.17748260498047\\
102.337501525879 -8.17832183837891\\
102.349998474121 -8.17916011810303\\
102.362503051758 -8.17999839782715\\
102.375 -8.18083572387695\\
102.387496948242 -8.18166828155518\\
102.400001525879 -8.18249797821045\\
102.412498474121 -8.18332576751709\\
102.425003051758 -8.1841459274292\\
102.4375 -8.18496513366699\\
102.449996948242 -8.1857795715332\\
102.462501525879 -8.18659019470215\\
102.474998474121 -8.18739891052246\\
102.487503051758 -8.18820285797119\\
102.5 -8.18900299072266\\
102.512496948242 -8.18980121612549\\
102.525001525879 -8.19059276580811\\
102.537498474121 -8.19138431549072\\
102.550003051758 -8.19217586517334\\
102.5625 -8.19296550750732\\
102.574996948242 -8.19375514984131\\
102.587501525879 -8.19454002380371\\
102.599998474121 -8.19532299041748\\
102.612503051758 -8.19611358642578\\
102.625 -8.19690322875977\\
102.637496948242 -8.19769382476807\\
102.650001525879 -8.19848537445068\\
102.662498474121 -8.1992826461792\\
102.675003051758 -8.20008277893066\\
102.6875 -8.20088386535645\\
102.699996948242 -8.20168495178223\\
102.712501525879 -8.20248603820801\\
102.724998474121 -8.203293800354\\
102.737503051758 -8.20409965515137\\
102.75 -8.20490837097168\\
102.762496948242 -8.20571708679199\\
102.775001525879 -8.20652675628662\\
102.787498474121 -8.20733547210693\\
102.800003051758 -8.20814609527588\\
102.8125 -8.20895671844482\\
102.824996948242 -8.20976734161377\\
102.837501525879 -8.21057796478271\\
102.849998474121 -8.21138858795166\\
102.862503051758 -8.21219921112061\\
102.875 -8.21300983428955\\
102.887496948242 -8.2138204574585\\
102.900001525879 -8.21463108062744\\
102.912498474121 -8.21544170379639\\
102.925003051758 -8.21625232696533\\
102.9375 -8.21706962585449\\
102.949996948242 -8.21788787841797\\
102.962501525879 -8.21870708465576\\
102.974998474121 -8.21952533721924\\
102.987503051758 -8.22034454345703\\
103 -8.22116470336914\\
103.012496948242 -8.22198486328125\\
103.025001525879 -8.22280502319336\\
103.037498474121 -8.22362518310547\\
103.050003051758 -8.22444534301758\\
103.0625 -8.22526550292969\\
103.074996948242 -8.22609043121338\\
103.087501525879 -8.22691917419434\\
103.099998474121 -8.22774410247803\\
103.112503051758 -8.22856426239014\\
103.125 -8.22939205169678\\
103.137496948242 -8.23021984100342\\
103.150001525879 -8.23104858398438\\
103.162498474121 -8.23187637329102\\
103.175003051758 -8.23270511627197\\
103.1875 -8.23353290557861\\
103.199996948242 -8.23436069488525\\
103.212501525879 -8.23518943786621\\
103.224998474121 -8.23601722717285\\
103.237503051758 -8.23683738708496\\
103.25 -8.23765754699707\\
103.262496948242 -8.23847675323486\\
103.275001525879 -8.23929595947266\\
103.287498474121 -8.24010848999023\\
103.300003051758 -8.24091911315918\\
103.3125 -8.24173545837402\\
103.324996948242 -8.24255561828613\\
103.337501525879 -8.24337577819824\\
103.349998474121 -8.24419593811035\\
103.362503051758 -8.24501609802246\\
103.375 -8.24583625793457\\
103.387496948242 -8.24665641784668\\
103.400001525879 -8.24747562408447\\
103.412498474121 -8.24829387664795\\
103.425003051758 -8.24911117553711\\
103.4375 -8.24992179870605\\
103.449996948242 -8.25073146820068\\
103.462501525879 -8.25154113769531\\
103.474998474121 -8.25234317779541\\
103.487503051758 -8.25314426422119\\
103.5 -8.25394344329834\\
103.512496948242 -8.25473594665527\\
103.525001525879 -8.25552749633789\\
103.537498474121 -8.25631713867188\\
103.550003051758 -8.25710296630859\\
103.5625 -8.25788497924805\\
103.574996948242 -8.25866603851318\\
103.587501525879 -8.25944709777832\\
103.599998474121 -8.26022720336914\\
103.612503051758 -8.26100540161133\\
103.625 -8.26177787780762\\
103.637496948242 -8.26255035400391\\
103.650001525879 -8.2633228302002\\
103.662498474121 -8.26409530639648\\
103.675003051758 -8.2648754119873\\
103.6875 -8.26565551757813\\
103.699996948242 -8.26643562316895\\
103.712501525879 -8.26721668243408\\
103.724998474121 -8.26799774169922\\
103.737503051758 -8.26877975463867\\
103.75 -8.26956176757813\\
103.762496948242 -8.27034759521484\\
103.775001525879 -8.27113819122314\\
103.787498474121 -8.27192783355713\\
103.800003051758 -8.27271938323975\\
103.8125 -8.27351093292236\\
103.824996948242 -8.27430248260498\\
103.837501525879 -8.2750940322876\\
103.849998474121 -8.27588558197021\\
103.862503051758 -8.27667713165283\\
103.875 -8.27746868133545\\
103.887496948242 -8.27826023101807\\
103.900001525879 -8.27905178070068\\
103.912498474121 -8.2798433303833\\
103.925003051758 -8.28064060211182\\
103.9375 -8.28143501281738\\
103.949996948242 -8.28223419189453\\
103.962501525879 -8.283034324646\\
103.974998474121 -8.28383350372314\\
103.987503051758 -8.28463363647461\\
104 -8.28543281555176\\
104.012496948242 -8.28623199462891\\
104.025001525879 -8.28703212738037\\
104.037498474121 -8.28783130645752\\
104.050003051758 -8.28863143920898\\
104.0625 -8.28943061828613\\
104.074996948242 -8.29022979736328\\
104.087501525879 -8.29102993011475\\
104.099998474121 -8.29182910919189\\
104.112503051758 -8.29262924194336\\
104.125 -8.29342937469482\\
104.137496948242 -8.29423046112061\\
104.150001525879 -8.29503154754639\\
104.162498474121 -8.29583168029785\\
104.175003051758 -8.29663181304932\\
104.1875 -8.2974328994751\\
104.199996948242 -8.29823398590088\\
104.212501525879 -8.29903507232666\\
104.224998474121 -8.29983520507813\\
104.237503051758 -8.30063533782959\\
104.25 -8.30143451690674\\
104.262496948242 -8.30223369598389\\
104.275001525879 -8.30303382873535\\
104.287498474121 -8.30383586883545\\
104.300003051758 -8.30464553833008\\
104.3125 -8.30545616149902\\
104.324996948242 -8.30626583099365\\
104.337501525879 -8.30707550048828\\
104.349998474121 -8.30788421630859\\
104.362503051758 -8.30869388580322\\
104.375 -8.30950260162354\\
104.387496948242 -8.31031131744385\\
104.400001525879 -8.31112098693848\\
104.412498474121 -8.31192970275879\\
104.425003051758 -8.31273555755615\\
104.4375 -8.31353664398193\\
104.449996948242 -8.31433773040771\\
104.462501525879 -8.31513786315918\\
104.474998474121 -8.31593704223633\\
104.487503051758 -8.31673145294189\\
104.5 -8.31752300262451\\
104.512496948242 -8.3183126449585\\
104.525001525879 -8.3191032409668\\
104.537498474121 -8.31988620758057\\
104.550003051758 -8.32066822052002\\
104.5625 -8.32144927978516\\
104.574996948242 -8.32222938537598\\
104.587501525879 -8.3230037689209\\
104.599998474121 -8.32377624511719\\
104.612503051758 -8.32454872131348\\
104.625 -8.32531929016113\\
104.637496948242 -8.32608985900879\\
104.650001525879 -8.32686138153076\\
104.662498474121 -8.32763195037842\\
104.675003051758 -8.32839679718018\\
104.6875 -8.3291597366333\\
104.699996948242 -8.32992267608643\\
104.712501525879 -8.33068561553955\\
104.724998474121 -8.33144855499268\\
104.737503051758 -8.3322114944458\\
104.75 -8.33297443389893\\
104.762496948242 -8.33374404907227\\
104.775001525879 -8.33451557159424\\
104.787498474121 -8.33528614044189\\
104.800003051758 -8.33605766296387\\
104.8125 -8.33682823181152\\
104.824996948242 -8.33759880065918\\
104.837501525879 -8.33837032318115\\
104.849998474121 -8.33914089202881\\
104.862503051758 -8.33991241455078\\
104.875 -8.34068298339844\\
104.887496948242 -8.34145355224609\\
104.900001525879 -8.34222507476807\\
104.912498474121 -8.34299564361572\\
104.925003051758 -8.3437671661377\\
104.9375 -8.34453773498535\\
104.949996948242 -8.34530830383301\\
104.962501525879 -8.34607982635498\\
104.974998474121 -8.34685039520264\\
104.987503051758 -8.34762191772461\\
105 -8.34839344024658\\
105.012496948242 -8.34916591644287\\
105.025001525879 -8.34993839263916\\
105.037498474121 -8.35071086883545\\
105.050003051758 -8.35148334503174\\
105.0625 -8.35225582122803\\
105.074996948242 -8.35303020477295\\
105.087501525879 -8.35381126403809\\
105.099998474121 -8.35459136962891\\
105.112503051758 -8.35537242889404\\
105.125 -8.35615253448486\\
105.137496948242 -8.35693454742432\\
105.150001525879 -8.35771656036377\\
105.162498474121 -8.35849857330322\\
105.175003051758 -8.35928058624268\\
105.1875 -8.36007022857666\\
105.199996948242 -8.36085987091064\\
105.212501525879 -8.36165046691895\\
105.224998474121 -8.36244010925293\\
105.237503051758 -8.36323165893555\\
105.25 -8.36402320861816\\
105.262496948242 -8.36481475830078\\
105.275001525879 -8.3656063079834\\
105.287498474121 -8.36639785766602\\
105.300003051758 -8.36718845367432\\
105.3125 -8.36797904968262\\
105.324996948242 -8.36877250671387\\
105.337501525879 -8.36957263946533\\
105.349998474121 -8.37037181854248\\
105.362503051758 -8.37116432189941\\
105.375 -8.37195587158203\\
105.387496948242 -8.37274742126465\\
105.400001525879 -8.37353801727295\\
105.412498474121 -8.37432765960693\\
105.425003051758 -8.37511825561523\\
105.4375 -8.37590789794922\\
105.449996948242 -8.3766975402832\\
105.462501525879 -8.3774881362915\\
105.474998474121 -8.37827301025391\\
105.487503051758 -8.37905502319336\\
105.5 -8.37983703613281\\
105.512496948242 -8.38061809539795\\
105.525001525879 -8.38139915466309\\
105.537498474121 -8.38217735290527\\
105.550003051758 -8.38294982910156\\
105.5625 -8.38372135162354\\
105.574996948242 -8.38449192047119\\
105.587501525879 -8.38525581359863\\
105.599998474121 -8.38601875305176\\
105.612503051758 -8.38678169250488\\
105.625 -8.38754367828369\\
105.637496948242 -8.38830471038818\\
105.650001525879 -8.38906288146973\\
105.662498474121 -8.38981628417969\\
105.675003051758 -8.39056968688965\\
105.6875 -8.39132213592529\\
105.699996948242 -8.39207363128662\\
105.712501525879 -8.39282608032227\\
105.724998474121 -8.39357757568359\\
105.737503051758 -8.39433002471924\\
105.75 -8.39508152008057\\
105.762496948242 -8.39582824707031\\
105.775001525879 -8.39657211303711\\
105.787498474121 -8.39731597900391\\
105.800003051758 -8.3980598449707\\
105.8125 -8.3988037109375\\
105.824996948242 -8.3995475769043\\
105.837501525879 -8.40029144287109\\
105.849998474121 -8.40103530883789\\
105.862503051758 -8.40177917480469\\
105.875 -8.40252304077148\\
105.887496948242 -8.40326690673828\\
105.900001525879 -8.40401077270508\\
105.912498474121 -8.40475463867188\\
105.925003051758 -8.40549850463867\\
105.9375 -8.40624237060547\\
105.949996948242 -8.40698623657227\\
105.962501525879 -8.40773010253906\\
105.974998474121 -8.40847969055176\\
105.987503051758 -8.4092321395874\\
106 -8.40998363494873\\
106.012496948242 -8.41073513031006\\
106.025001525879 -8.4114875793457\\
106.037498474121 -8.41223907470703\\
106.050003051758 -8.41299152374268\\
106.0625 -8.413743019104\\
106.074996948242 -8.41449642181396\\
106.087501525879 -8.41524982452393\\
106.099998474121 -8.41600322723389\\
106.112503051758 -8.41675662994385\\
106.125 -8.41751766204834\\
106.137496948242 -8.41827869415283\\
106.150001525879 -8.41904067993164\\
106.162498474121 -8.41980171203613\\
106.175003051758 -8.42056369781494\\
106.1875 -8.42132472991943\\
106.199996948242 -8.42208766937256\\
106.212501525879 -8.42285060882568\\
106.224998474121 -8.42362022399902\\
106.237503051758 -8.424391746521\\
106.25 -8.42516422271729\\
106.262496948242 -8.42593669891357\\
106.275001525879 -8.42671394348145\\
106.287498474121 -8.42749404907227\\
106.300003051758 -8.4282751083374\\
106.3125 -8.42905521392822\\
106.324996948242 -8.42983531951904\\
106.337501525879 -8.43061637878418\\
106.349998474121 -8.43139743804932\\
106.362503051758 -8.43217849731445\\
106.375 -8.43295860290527\\
106.387496948242 -8.43373870849609\\
106.400001525879 -8.43451976776123\\
106.412498474121 -8.43529987335205\\
106.425003051758 -8.43608093261719\\
106.4375 -8.43686103820801\\
106.449996948242 -8.43764114379883\\
106.462501525879 -8.43842220306396\\
106.474998474121 -8.4392032623291\\
106.487503051758 -8.43998527526855\\
106.5 -8.44076728820801\\
106.512496948242 -8.44154739379883\\
106.525001525879 -8.44232654571533\\
106.537498474121 -8.44309902191162\\
106.550003051758 -8.44387149810791\\
106.5625 -8.44464206695557\\
106.574996948242 -8.44540786743164\\
106.587501525879 -8.44617080688477\\
106.599998474121 -8.44693279266357\\
106.612503051758 -8.44769477844238\\
106.625 -8.44845008850098\\
106.637496948242 -8.44920349121094\\
106.650001525879 -8.44995594024658\\
106.662498474121 -8.45070743560791\\
106.675003051758 -8.45145893096924\\
106.6875 -8.45220279693604\\
106.699996948242 -8.45294666290283\\
106.712501525879 -8.45369052886963\\
106.724998474121 -8.45443439483643\\
106.737503051758 -8.45517730712891\\
106.75 -8.45591926574707\\
106.762496948242 -8.45665836334229\\
106.775001525879 -8.45739269256592\\
106.787498474121 -8.45812702178955\\
106.800003051758 -8.45886135101318\\
106.8125 -8.45959568023682\\
106.824996948242 -8.46032810211182\\
106.837501525879 -8.46106147766113\\
106.849998474121 -8.46179389953613\\
106.862503051758 -8.46252727508545\\
106.875 -8.46325969696045\\
106.887496948242 -8.46399211883545\\
106.900001525879 -8.46472549438477\\
106.912498474121 -8.46545791625977\\
106.925003051758 -8.4661922454834\\
106.9375 -8.4669246673584\\
106.949996948242 -8.46765804290771\\
106.962501525879 -8.46839237213135\\
106.974998474121 -8.46912670135498\\
106.987503051758 -8.46986103057861\\
107 -8.47059535980225\\
107.012496948242 -8.47132968902588\\
107.025001525879 -8.47206401824951\\
107.037498474121 -8.47279834747314\\
107.050003051758 -8.47353267669678\\
107.0625 -8.47426700592041\\
107.074996948242 -8.47500133514404\\
107.087501525879 -8.47573566436768\\
107.099998474121 -8.47646999359131\\
107.112503051758 -8.47720432281494\\
107.125 -8.47794055938721\\
107.137496948242 -8.47868251800537\\
107.150001525879 -8.47942543029785\\
107.162498474121 -8.48016834259033\\
107.175003051758 -8.48091220855713\\
107.1875 -8.48165607452393\\
107.199996948242 -8.48240566253662\\
107.212501525879 -8.48315811157227\\
107.224998474121 -8.48390960693359\\
107.237503051758 -8.48466300964355\\
107.25 -8.48541641235352\\
107.262496948242 -8.48616981506348\\
107.275001525879 -8.48692321777344\\
107.287498474121 -8.48768043518066\\
107.300003051758 -8.48844242095947\\
107.3125 -8.48920345306396\\
107.324996948242 -8.48996448516846\\
107.337501525879 -8.49072647094727\\
107.349998474121 -8.49148750305176\\
107.362503051758 -8.49225044250488\\
107.375 -8.49301338195801\\
107.387496948242 -8.49378299713135\\
107.400001525879 -8.49455451965332\\
107.412498474121 -8.49532508850098\\
107.425003051758 -8.49609184265137\\
107.4375 -8.49685478210449\\
107.449996948242 -8.49761962890625\\
107.462501525879 -8.49839115142822\\
107.474998474121 -8.49916172027588\\
107.487503051758 -8.49993419647217\\
107.5 -8.50070667266846\\
107.512496948242 -8.50147914886475\\
107.525001525879 -8.50225162506104\\
107.537498474121 -8.50302314758301\\
107.550003051758 -8.50379467010498\\
107.5625 -8.50456237792969\\
107.574996948242 -8.50532531738281\\
107.587501525879 -8.50608825683594\\
107.599998474121 -8.50685024261475\\
107.612503051758 -8.50761222839355\\
107.625 -8.50837230682373\\
107.637496948242 -8.50912570953369\\
107.650001525879 -8.50987911224365\\
107.662498474121 -8.51063060760498\\
107.675003051758 -8.51138305664063\\
107.6875 -8.51212978363037\\
107.699996948242 -8.51287364959717\\
107.712501525879 -8.51361751556396\\
107.724998474121 -8.51435947418213\\
107.737503051758 -8.51510143280029\\
107.75 -8.51583576202393\\
107.762496948242 -8.51657009124756\\
107.775001525879 -8.51730442047119\\
107.787498474121 -8.51803684234619\\
107.800003051758 -8.51877021789551\\
107.8125 -8.51950263977051\\
107.824996948242 -8.52023506164551\\
107.837501525879 -8.52095985412598\\
107.849998474121 -8.52168464660645\\
107.862503051758 -8.52240943908691\\
107.875 -8.52313423156738\\
107.887496948242 -8.52385902404785\\
107.900001525879 -8.52458381652832\\
107.912498474121 -8.52530860900879\\
107.925003051758 -8.52603340148926\\
107.9375 -8.52675819396973\\
107.949996948242 -8.5274829864502\\
107.962501525879 -8.52820777893066\\
107.974998474121 -8.52893257141113\\
107.987503051758 -8.5296573638916\\
108 -8.53038215637207\\
108.012496948242 -8.5311107635498\\
108.025001525879 -8.53184413909912\\
108.037498474121 -8.53257656097412\\
108.050003051758 -8.53330993652344\\
108.0625 -8.53404235839844\\
108.074996948242 -8.53477478027344\\
108.087501525879 -8.53550815582275\\
108.099998474121 -8.53624057769775\\
108.112503051758 -8.53697395324707\\
108.125 -8.53770637512207\\
108.137496948242 -8.53843879699707\\
108.150001525879 -8.53917217254639\\
108.162498474121 -8.53990459442139\\
108.175003051758 -8.5406379699707\\
108.1875 -8.54137229919434\\
108.199996948242 -8.54210662841797\\
108.212501525879 -8.5428409576416\\
108.224998474121 -8.54357528686523\\
108.237503051758 -8.54430866241455\\
108.25 -8.54504108428955\\
108.262496948242 -8.54577350616455\\
108.275001525879 -8.54650783538818\\
108.287498474121 -8.54724311828613\\
108.300003051758 -8.54798603057861\\
108.3125 -8.54873371124268\\
108.324996948242 -8.549485206604\\
108.337501525879 -8.55023860931396\\
108.349998474121 -8.55099201202393\\
108.362503051758 -8.55174541473389\\
108.375 -8.55249881744385\\
108.387496948242 -8.55325222015381\\
108.400001525879 -8.55400562286377\\
108.412498474121 -8.55475902557373\\
108.425003051758 -8.55551815032959\\
108.4375 -8.5562801361084\\
108.449996948242 -8.55704307556152\\
108.462501525879 -8.55780601501465\\
108.474998474121 -8.55856895446777\\
108.487503051758 -8.5593318939209\\
108.5 -8.56009769439697\\
108.512496948242 -8.56086826324463\\
108.525001525879 -8.5616397857666\\
108.537498474121 -8.56241035461426\\
108.550003051758 -8.56318187713623\\
108.5625 -8.56394863128662\\
108.574996948242 -8.56471157073975\\
108.587501525879 -8.56547355651855\\
108.599998474121 -8.56623077392578\\
108.612503051758 -8.56698417663574\\
108.625 -8.56773662567139\\
108.637496948242 -8.56848812103271\\
108.650001525879 -8.56923294067383\\
108.662498474121 -8.56997680664063\\
108.675003051758 -8.57072067260742\\
108.6875 -8.57146453857422\\
108.699996948242 -8.57220840454102\\
108.712501525879 -8.5729513168335\\
108.724998474121 -8.57369327545166\\
108.737503051758 -8.57443618774414\\
108.75 -8.57517719268799\\
108.762496948242 -8.57591152191162\\
108.775001525879 -8.57664585113525\\
108.787498474121 -8.57738018035889\\
108.800003051758 -8.5781135559082\\
108.8125 -8.5788459777832\\
108.824996948242 -8.5795783996582\\
108.837501525879 -8.5803050994873\\
108.849998474121 -8.58102989196777\\
108.862503051758 -8.58175468444824\\
108.875 -8.58247947692871\\
108.887496948242 -8.58320236206055\\
108.900001525879 -8.5839262008667\\
108.912498474121 -8.58464908599854\\
108.925003051758 -8.58537292480469\\
108.9375 -8.58609580993652\\
108.949996948242 -8.58681869506836\\
108.962501525879 -8.58754253387451\\
108.974998474121 -8.58826541900635\\
108.987503051758 -8.5889892578125\\
109 -8.58971214294434\\
109.012496948242 -8.59043502807617\\
109.025001525879 -8.59115886688232\\
109.037498474121 -8.59187793731689\\
109.050003051758 -8.5925931930542\\
109.0625 -8.59331512451172\\
109.074996948242 -8.59403800964355\\
109.087501525879 -8.59476184844971\\
109.099998474121 -8.59548473358154\\
109.112503051758 -8.5962085723877\\
109.125 -8.59693145751953\\
109.137496948242 -8.59765434265137\\
109.150001525879 -8.59837818145752\\
109.162498474121 -8.59910106658936\\
109.175003051758 -8.59982490539551\\
109.1875 -8.60054969787598\\
109.199996948242 -8.60127449035645\\
109.212501525879 -8.60199928283691\\
109.224998474121 -8.60272407531738\\
109.237503051758 -8.60344791412354\\
109.25 -8.604172706604\\
109.262496948242 -8.60489749908447\\
109.275001525879 -8.60562229156494\\
109.287498474121 -8.60634708404541\\
109.300003051758 -8.60707187652588\\
109.3125 -8.60779666900635\\
109.324996948242 -8.60852146148682\\
109.337501525879 -8.60924625396729\\
109.349998474121 -8.60997104644775\\
109.362503051758 -8.61069583892822\\
109.375 -8.61142826080322\\
109.387496948242 -8.6121654510498\\
109.400001525879 -8.6129093170166\\
109.412498474121 -8.6136531829834\\
109.425003051758 -8.6143970489502\\
109.4375 -8.61514377593994\\
109.449996948242 -8.61589527130127\\
109.462501525879 -8.61664772033691\\
109.474998474121 -8.61739921569824\\
109.487503051758 -8.61815166473389\\
109.5 -8.61890316009521\\
109.512496948242 -8.61965656280518\\
109.525001525879 -8.62040996551514\\
109.537498474121 -8.6211633682251\\
109.550003051758 -8.62191581726074\\
109.5625 -8.62266731262207\\
109.574996948242 -8.62342071533203\\
109.587501525879 -8.62417411804199\\
109.599998474121 -8.62492752075195\\
109.612503051758 -8.62568092346191\\
109.625 -8.62643337249756\\
109.637496948242 -8.62718486785889\\
109.650001525879 -8.62793731689453\\
109.662498474121 -8.62868881225586\\
109.675003051758 -8.6294412612915\\
109.6875 -8.63019275665283\\
109.699996948242 -8.63094043731689\\
109.712501525879 -8.63168430328369\\
109.724998474121 -8.63242816925049\\
109.737503051758 -8.63317203521729\\
109.75 -8.63391494750977\\
109.762496948242 -8.63465690612793\\
109.775001525879 -8.63539791107178\\
109.787498474121 -8.63613224029541\\
109.800003051758 -8.63686656951904\\
109.8125 -8.63760089874268\\
109.824996948242 -8.63833427429199\\
109.837501525879 -8.63906764984131\\
109.849998474121 -8.63980007171631\\
109.862503051758 -8.64052867889404\\
109.875 -8.64125347137451\\
109.887496948242 -8.64197826385498\\
109.900001525879 -8.64270305633545\\
109.912498474121 -8.64342594146729\\
109.925003051758 -8.64414978027344\\
109.9375 -8.64487266540527\\
109.949996948242 -8.64559268951416\\
109.962501525879 -8.64630794525146\\
109.974998474121 -8.64702320098877\\
109.987503051758 -8.64773845672607\\
110 -8.64845371246338\\
110.012496948242 -8.64916801452637\\
110.025001525879 -8.64988231658936\\
110.037498474121 -8.65059566497803\\
110.050003051758 -8.65130996704102\\
110.0625 -8.65202331542969\\
110.074996948242 -8.65273571014404\\
110.087501525879 -8.65344142913818\\
110.099998474121 -8.65414714813232\\
110.112503051758 -8.65485286712646\\
110.125 -8.65555858612061\\
110.137496948242 -8.65626430511475\\
110.150001525879 -8.65697002410889\\
110.162498474121 -8.65767574310303\\
110.175003051758 -8.65838146209717\\
110.1875 -8.65908718109131\\
110.199996948242 -8.65979290008545\\
110.212501525879 -8.66049861907959\\
110.224998474121 -8.66120433807373\\
110.237503051758 -8.66191005706787\\
110.25 -8.66261577606201\\
110.262496948242 -8.66332149505615\\
110.275001525879 -8.66402721405029\\
110.287498474121 -8.66473293304443\\
110.300003051758 -8.66543865203857\\
110.3125 -8.66615009307861\\
110.324996948242 -8.66686344146729\\
110.337501525879 -8.66757774353027\\
110.349998474121 -8.66829204559326\\
110.362503051758 -8.66900730133057\\
110.375 -8.66972255706787\\
110.387496948242 -8.67043781280518\\
110.400001525879 -8.67115306854248\\
110.412498474121 -8.67187404632568\\
110.425003051758 -8.67259788513184\\
110.4375 -8.67332077026367\\
110.449996948242 -8.67404365539551\\
110.462501525879 -8.67476749420166\\
110.474998474121 -8.67549228668213\\
110.487503051758 -8.6762170791626\\
110.5 -8.67694187164307\\
110.512496948242 -8.67766666412354\\
110.525001525879 -8.67839241027832\\
110.537498474121 -8.67912483215332\\
110.550003051758 -8.67985820770264\\
110.5625 -8.68059253692627\\
110.574996948242 -8.6813268661499\\
110.587501525879 -8.68206119537354\\
110.599998474121 -8.68279552459717\\
110.612503051758 -8.6835298538208\\
110.625 -8.68426418304443\\
110.637496948242 -8.68499851226807\\
110.650001525879 -8.6857328414917\\
110.662498474121 -8.68646717071533\\
110.675003051758 -8.68720054626465\\
110.6875 -8.68793296813965\\
110.699996948242 -8.68866539001465\\
110.712501525879 -8.68939971923828\\
110.724998474121 -8.69013404846191\\
110.737503051758 -8.69086837768555\\
110.75 -8.69160270690918\\
110.762496948242 -8.69233703613281\\
110.775001525879 -8.69307327270508\\
110.787498474121 -8.69381523132324\\
110.800003051758 -8.69455814361572\\
110.8125 -8.69530010223389\\
110.824996948242 -8.69604206085205\\
110.837501525879 -8.69677925109863\\
110.849998474121 -8.69751358032227\\
110.862503051758 -8.6982479095459\\
110.875 -8.69898223876953\\
110.887496948242 -8.69971656799316\\
110.900001525879 -8.70044994354248\\
110.912498474121 -8.70118236541748\\
110.925003051758 -8.7019157409668\\
110.9375 -8.7026424407959\\
110.949996948242 -8.70336723327637\\
110.962501525879 -8.70409202575684\\
110.974998474121 -8.7048168182373\\
110.987503051758 -8.70554161071777\\
111 -8.70626449584961\\
111.012496948242 -8.70698738098145\\
111.025001525879 -8.70770740509033\\
111.037498474121 -8.70842266082764\\
111.050003051758 -8.70913791656494\\
111.0625 -8.70985317230225\\
111.074996948242 -8.71056747436523\\
111.087501525879 -8.71128177642822\\
111.099998474121 -8.71199512481689\\
111.112503051758 -8.71270751953125\\
111.125 -8.71341323852539\\
111.137496948242 -8.71411895751953\\
111.150001525879 -8.71482467651367\\
111.162498474121 -8.71553039550781\\
111.175003051758 -8.71623611450195\\
111.1875 -8.71693992614746\\
111.199996948242 -8.71764373779297\\
111.212501525879 -8.71834850311279\\
111.224998474121 -8.7190523147583\\
111.237503051758 -8.71975708007813\\
111.25 -8.72046089172363\\
111.262496948242 -8.72116470336914\\
111.275001525879 -8.72186946868896\\
111.287498474121 -8.72257328033447\\
111.300003051758 -8.7232780456543\\
111.3125 -8.7239818572998\\
111.324996948242 -8.72468566894531\\
111.337501525879 -8.72539043426514\\
111.349998474121 -8.72609424591064\\
111.362503051758 -8.72679901123047\\
111.375 -8.72750282287598\\
111.387496948242 -8.72819900512695\\
111.400001525879 -8.72889518737793\\
111.412498474121 -8.72959136962891\\
111.425003051758 -8.73028755187988\\
111.4375 -8.73098373413086\\
111.449996948242 -8.73167991638184\\
111.462501525879 -8.73237609863281\\
111.474998474121 -8.73307228088379\\
111.487503051758 -8.73376846313477\\
111.5 -8.73446464538574\\
111.512496948242 -8.73516082763672\\
111.525001525879 -8.7358570098877\\
111.537498474121 -8.73655319213867\\
111.550003051758 -8.73724937438965\\
111.5625 -8.73794555664063\\
111.574996948242 -8.73864269256592\\
111.587501525879 -8.73934745788574\\
111.599998474121 -8.74005126953125\\
111.612503051758 -8.74075603485107\\
111.625 -8.7414608001709\\
111.637496948242 -8.74216651916504\\
111.650001525879 -8.74287605285645\\
111.662498474121 -8.74358940124512\\
111.675003051758 -8.74430370330811\\
111.6875 -8.74501895904541\\
111.699996948242 -8.74573421478271\\
111.712501525879 -8.74644947052002\\
111.724998474121 -8.74716472625732\\
111.737503051758 -8.74787998199463\\
111.75 -8.7485990524292\\
111.762496948242 -8.74932193756104\\
111.775001525879 -8.75004577636719\\
111.787498474121 -8.75076866149902\\
111.800003051758 -8.75149250030518\\
111.8125 -8.75221538543701\\
111.824996948242 -8.75294017791748\\
111.837501525879 -8.75366497039795\\
111.849998474121 -8.75439643859863\\
111.862503051758 -8.75512981414795\\
111.875 -8.75586223602295\\
111.887496948242 -8.75659465789795\\
111.900001525879 -8.75732135772705\\
111.912498474121 -8.75804615020752\\
111.925003051758 -8.75877094268799\\
111.9375 -8.75949573516846\\
111.949996948242 -8.76022148132324\\
111.962501525879 -8.76095390319824\\
111.974998474121 -8.76167869567871\\
111.987503051758 -8.76240348815918\\
112 -8.76312828063965\\
112.012496948242 -8.76385307312012\\
112.025001525879 -8.76457786560059\\
112.037498474121 -8.76530265808105\\
112.050003051758 -8.76602649688721\\
112.0625 -8.76674938201904\\
112.074996948242 -8.76746940612793\\
112.087501525879 -8.76818466186523\\
112.099998474121 -8.76889991760254\\
112.112503051758 -8.76961517333984\\
112.125 -8.77032852172852\\
112.137496948242 -8.77104187011719\\
112.150001525879 -8.77175617218018\\
112.162498474121 -8.77246189117432\\
112.175003051758 -8.77316761016846\\
112.1875 -8.7738733291626\\
112.199996948242 -8.77457809448242\\
112.212501525879 -8.77528285980225\\
112.224998474121 -8.77598667144775\\
112.237503051758 -8.77668571472168\\
112.25 -8.77738189697266\\
112.262496948242 -8.77807807922363\\
112.275001525879 -8.77877426147461\\
112.287498474121 -8.77947044372559\\
112.300003051758 -8.78016567230225\\
112.3125 -8.78085994720459\\
112.324996948242 -8.78155422210693\\
112.337501525879 -8.78224945068359\\
112.349998474121 -8.78293609619141\\
112.362503051758 -8.78362274169922\\
112.375 -8.78430938720703\\
112.387496948242 -8.78499412536621\\
112.400001525879 -8.78567981719971\\
112.412498474121 -8.78636455535889\\
112.425003051758 -8.78705024719238\\
112.4375 -8.78773212432861\\
112.449996948242 -8.78840923309326\\
112.462501525879 -8.78908634185791\\
112.474998474121 -8.78976345062256\\
112.487503051758 -8.79044055938721\\
112.5 -8.79111766815186\\
112.512496948242 -8.79179382324219\\
112.525001525879 -8.79246997833252\\
112.537498474121 -8.79314517974854\\
112.550003051758 -8.79382133483887\\
112.5625 -8.79449653625488\\
112.574996948242 -8.7951717376709\\
112.587501525879 -8.79584789276123\\
112.599998474121 -8.79652404785156\\
112.612503051758 -8.79720020294189\\
112.625 -8.79787540435791\\
112.637496948242 -8.79855060577393\\
112.650001525879 -8.79922771453857\\
112.662498474121 -8.79990482330322\\
112.675003051758 -8.80058193206787\\
112.6875 -8.80125904083252\\
112.699996948242 -8.80193614959717\\
112.712501525879 -8.80261325836182\\
112.724998474121 -8.80329036712646\\
112.737503051758 -8.80396747589111\\
112.75 -8.80464458465576\\
112.762496948242 -8.80532169342041\\
112.775001525879 -8.80600643157959\\
112.787498474121 -8.80669116973877\\
112.800003051758 -8.80737781524658\\
112.8125 -8.80806446075439\\
112.824996948242 -8.80875110626221\\
112.837501525879 -8.80944633483887\\
112.849998474121 -8.81014060974121\\
112.862503051758 -8.81083583831787\\
112.875 -8.81153011322021\\
112.887496948242 -8.81222438812256\\
112.900001525879 -8.81292057037354\\
112.912498474121 -8.81361675262451\\
112.925003051758 -8.81431293487549\\
112.9375 -8.81500911712646\\
112.949996948242 -8.81570529937744\\
112.962501525879 -8.81641006469727\\
112.974998474121 -8.81711578369141\\
112.987503051758 -8.81782150268555\\
113 -8.81852722167969\\
113.012496948242 -8.81923484802246\\
113.025001525879 -8.81994915008545\\
113.037498474121 -8.82066249847412\\
113.050003051758 -8.82137012481689\\
113.0625 -8.82207584381104\\
113.074996948242 -8.82278156280518\\
113.087501525879 -8.82349300384521\\
113.099998474121 -8.82420635223389\\
113.112503051758 -8.82492065429688\\
113.125 -8.82563495635986\\
113.137496948242 -8.82635021209717\\
113.150001525879 -8.82706546783447\\
113.162498474121 -8.82778072357178\\
113.175003051758 -8.82849502563477\\
113.1875 -8.82920837402344\\
113.199996948242 -8.82991695404053\\
113.212501525879 -8.83062267303467\\
113.224998474121 -8.83132743835449\\
113.237503051758 -8.83203220367432\\
113.25 -8.83273029327393\\
113.262496948242 -8.8334264755249\\
113.275001525879 -8.83412265777588\\
113.287498474121 -8.83481788635254\\
113.300003051758 -8.8355131149292\\
113.3125 -8.83620452880859\\
113.324996948242 -8.83689117431641\\
113.337501525879 -8.83757781982422\\
113.349998474121 -8.83826351165771\\
113.362503051758 -8.83894920349121\\
113.375 -8.83963108062744\\
113.387496948242 -8.84030818939209\\
113.400001525879 -8.84098529815674\\
113.412498474121 -8.84166145324707\\
113.425003051758 -8.8423376083374\\
113.4375 -8.84301280975342\\
113.449996948242 -8.84368133544922\\
113.462501525879 -8.8443489074707\\
113.474998474121 -8.84501647949219\\
113.487503051758 -8.84568405151367\\
113.5 -8.84635066986084\\
113.512496948242 -8.84701633453369\\
113.525001525879 -8.84768295288086\\
113.537498474121 -8.84834861755371\\
113.550003051758 -8.84901523590088\\
113.5625 -8.84968090057373\\
113.574996948242 -8.85033893585205\\
113.587501525879 -8.85099697113037\\
113.599998474121 -8.85165500640869\\
113.612503051758 -8.85231304168701\\
113.625 -8.85297107696533\\
113.637496948242 -8.85362911224365\\
113.650001525879 -8.85428714752197\\
113.662498474121 -8.85494518280029\\
113.675003051758 -8.85560321807861\\
113.6875 -8.8562593460083\\
113.699996948242 -8.85691547393799\\
113.712501525879 -8.85757255554199\\
113.724998474121 -8.85822868347168\\
113.737503051758 -8.85888576507568\\
113.75 -8.85954189300537\\
113.762496948242 -8.86019802093506\\
113.775001525879 -8.86085605621338\\
113.787498474121 -8.8615140914917\\
113.800003051758 -8.86217212677002\\
113.8125 -8.86283016204834\\
113.824996948242 -8.86348819732666\\
113.837501525879 -8.86414623260498\\
113.849998474121 -8.8648042678833\\
113.862503051758 -8.86546230316162\\
113.875 -8.86612033843994\\
113.887496948242 -8.86677837371826\\
113.900001525879 -8.8674430847168\\
113.912498474121 -8.86810874938965\\
113.925003051758 -8.86877536773682\\
113.9375 -8.8694429397583\\
113.949996948242 -8.87011051177979\\
113.962501525879 -8.87077808380127\\
113.974998474121 -8.87144565582275\\
113.987503051758 -8.87211608886719\\
114 -8.8727912902832\\
114.012496948242 -8.87346649169922\\
114.025001525879 -8.87414264678955\\
114.037498474121 -8.8748197555542\\
114.050003051758 -8.87549686431885\\
114.0625 -8.8761739730835\\
114.074996948242 -8.87685108184814\\
114.087501525879 -8.87753105163574\\
114.099998474121 -8.87821578979492\\
114.112503051758 -8.87890243530273\\
114.125 -8.87958908081055\\
114.137496948242 -8.88027572631836\\
114.150001525879 -8.88096237182617\\
114.162498474121 -8.88164901733398\\
114.175003051758 -8.88234043121338\\
114.1875 -8.88303470611572\\
114.199996948242 -8.88372898101807\\
114.212501525879 -8.88441562652588\\
114.224998474121 -8.88510227203369\\
114.237503051758 -8.88579559326172\\
114.25 -8.88648986816406\\
114.262496948242 -8.88718414306641\\
114.275001525879 -8.88787937164307\\
114.287498474121 -8.88857460021973\\
114.300003051758 -8.8892707824707\\
114.3125 -8.88996696472168\\
114.324996948242 -8.89066314697266\\
114.337501525879 -8.89135837554932\\
114.349998474121 -8.89205265045166\\
114.362503051758 -8.89274406433105\\
114.375 -8.89343070983887\\
114.387496948242 -8.89411640167236\\
114.400001525879 -8.89480209350586\\
114.412498474121 -8.89548301696777\\
114.425003051758 -8.89616012573242\\
114.4375 -8.89683723449707\\
114.449996948242 -8.89751243591309\\
114.462501525879 -8.89818286895752\\
114.474998474121 -8.898850440979\\
114.487503051758 -8.89951801300049\\
114.5 -8.90018463134766\\
114.512496948242 -8.90085029602051\\
114.525001525879 -8.90150833129883\\
114.537498474121 -8.90216636657715\\
114.550003051758 -8.90282440185547\\
114.5625 -8.90348148345947\\
114.574996948242 -8.90413761138916\\
114.587501525879 -8.90478897094727\\
114.599998474121 -8.90543746948242\\
114.612503051758 -8.90608596801758\\
114.625 -8.90673351287842\\
114.637496948242 -8.90738010406494\\
114.650001525879 -8.90802764892578\\
114.662498474121 -8.90867042541504\\
114.675003051758 -8.90930938720703\\
114.6875 -8.90994834899902\\
114.699996948242 -8.91058731079102\\
114.712501525879 -8.91122627258301\\
114.724998474121 -8.91186428070068\\
114.737503051758 -8.91250228881836\\
114.75 -8.91313934326172\\
114.762496948242 -8.91377639770508\\
114.775001525879 -8.91440582275391\\
114.787498474121 -8.91503524780273\\
114.800003051758 -8.91566467285156\\
114.8125 -8.91629409790039\\
114.824996948242 -8.91692352294922\\
114.837501525879 -8.91755294799805\\
114.849998474121 -8.91818237304688\\
114.862503051758 -8.91881942749023\\
114.875 -8.91945648193359\\
114.887496948242 -8.92009353637695\\
114.900001525879 -8.92073154449463\\
114.912498474121 -8.92136859893799\\
114.925003051758 -8.92200660705566\\
114.9375 -8.92264366149902\\
114.949996948242 -8.9232816696167\\
114.962501525879 -8.92392063140869\\
114.974998474121 -8.92455959320068\\
114.987503051758 -8.92519855499268\\
115 -8.92583751678467\\
115.012496948242 -8.92647647857666\\
115.025001525879 -8.92711544036865\\
115.037498474121 -8.92775535583496\\
115.050003051758 -8.9284029006958\\
115.0625 -8.92904949188232\\
115.074996948242 -8.92969608306885\\
115.087501525879 -8.930344581604\\
115.099998474121 -8.93099308013916\\
115.112503051758 -8.93164157867432\\
115.125 -8.93229007720947\\
115.137496948242 -8.93294143676758\\
115.150001525879 -8.93359851837158\\
115.162498474121 -8.93425559997559\\
115.175003051758 -8.93491363525391\\
115.1875 -8.93557167053223\\
115.199996948242 -8.93623542785645\\
115.212501525879 -8.93690204620361\\
115.224998474121 -8.93756866455078\\
115.237503051758 -8.93823623657227\\
115.25 -8.93890380859375\\
115.262496948242 -8.93957138061523\\
115.275001525879 -8.94024658203125\\
115.287498474121 -8.94092178344727\\
115.300003051758 -8.9415979385376\\
115.3125 -8.94227313995361\\
115.324996948242 -8.94294834136963\\
115.337501525879 -8.94362449645996\\
115.349998474121 -8.94429969787598\\
115.362503051758 -8.94497585296631\\
115.375 -8.94565200805664\\
115.387496948242 -8.94632911682129\\
115.400001525879 -8.94700622558594\\
115.412498474121 -8.94768333435059\\
115.425003051758 -8.94836044311523\\
115.4375 -8.94903755187988\\
115.449996948242 -8.94971466064453\\
115.462501525879 -8.95039176940918\\
115.474998474121 -8.95106792449951\\
115.487503051758 -8.95174407958984\\
115.5 -8.95241165161133\\
115.512496948242 -8.95307922363281\\
115.525001525879 -8.95374584197998\\
115.537498474121 -8.95440864562988\\
115.550003051758 -8.9550666809082\\
115.5625 -8.95572376251221\\
115.574996948242 -8.95637989044189\\
115.587501525879 -8.95702934265137\\
115.599998474121 -8.95767784118652\\
115.612503051758 -8.95832633972168\\
115.625 -8.9589729309082\\
115.637496948242 -8.95961666107178\\
115.650001525879 -8.96025562286377\\
115.662498474121 -8.96089458465576\\
115.675003051758 -8.96153259277344\\
115.6875 -8.9621696472168\\
115.699996948242 -8.96280097961426\\
115.712501525879 -8.96343040466309\\
115.724998474121 -8.96405982971191\\
115.737503051758 -8.96468925476074\\
115.75 -8.96531677246094\\
115.762496948242 -8.96594429016113\\
115.775001525879 -8.96656799316406\\
115.787498474121 -8.96718788146973\\
115.800003051758 -8.96780776977539\\
115.8125 -8.96842765808105\\
115.824996948242 -8.96904754638672\\
115.837501525879 -8.96966648101807\\
115.849998474121 -8.9702844619751\\
115.862503051758 -8.97090339660645\\
115.875 -8.97152137756348\\
115.887496948242 -8.97213363647461\\
115.900001525879 -8.97274398803711\\
115.912498474121 -8.97335910797119\\
115.925003051758 -8.97397804260254\\
115.9375 -8.97459602355957\\
115.949996948242 -8.9752140045166\\
115.962501525879 -8.97583293914795\\
115.974998474121 -8.97645092010498\\
115.987503051758 -8.97706985473633\\
116 -8.97768783569336\\
116.012496948242 -8.97830581665039\\
116.025001525879 -8.97892475128174\\
116.037498474121 -8.97954273223877\\
116.050003051758 -8.98016166687012\\
116.0625 -8.98077964782715\\
116.074996948242 -8.9813985824585\\
116.087501525879 -8.98201847076416\\
116.099998474121 -8.98263835906982\\
116.112503051758 -8.98325824737549\\
116.125 -8.98387813568115\\
116.137496948242 -8.98449802398682\\
116.150001525879 -8.98511791229248\\
116.162498474121 -8.98573780059814\\
116.175003051758 -8.98635768890381\\
116.1875 -8.98697757720947\\
116.199996948242 -8.98759746551514\\
116.212501525879 -8.98822402954102\\
116.224998474121 -8.98885154724121\\
116.237503051758 -8.98948001861572\\
116.25 -8.99010944366455\\
116.262496948242 -8.99073886871338\\
116.275001525879 -8.99136829376221\\
116.287498474121 -8.99200534820557\\
116.300003051758 -8.99264335632324\\
116.3125 -8.99328136444092\\
116.324996948242 -8.99392032623291\\
116.337501525879 -8.9945592880249\\
116.349998474121 -8.99519920349121\\
116.362503051758 -8.99584674835205\\
116.375 -8.99649333953857\\
116.387496948242 -8.99714088439941\\
116.400001525879 -8.99778938293457\\
116.412498474121 -8.99843788146973\\
116.425003051758 -8.99908638000488\\
116.4375 -8.99973487854004\\
116.449996948242 -9.0003833770752\\
116.462501525879 -9.00103187561035\\
116.474998474121 -9.00168037414551\\
116.487503051758 -9.00232887268066\\
116.5 -9.00298023223877\\
116.512496948242 -9.00363636016846\\
116.525001525879 -9.00429439544678\\
116.537498474121 -9.0049524307251\\
116.550003051758 -9.00561046600342\\
116.5625 -9.00626754760742\\
116.574996948242 -9.00692558288574\\
116.587501525879 -9.00758361816406\\
116.599998474121 -9.00824165344238\\
116.612503051758 -9.0088996887207\\
116.625 -9.00955581665039\\
116.637496948242 -9.01021194458008\\
116.650001525879 -9.01086330413818\\
116.662498474121 -9.01151180267334\\
116.675003051758 -9.0121603012085\\
116.6875 -9.01280689239502\\
116.699996948242 -9.01344966888428\\
116.712501525879 -9.01408863067627\\
116.724998474121 -9.01472759246826\\
116.737503051758 -9.01536560058594\\
116.75 -9.0159969329834\\
116.762496948242 -9.01662635803223\\
116.775001525879 -9.01725482940674\\
116.787498474121 -9.0178804397583\\
116.800003051758 -9.01850032806396\\
116.8125 -9.01912021636963\\
116.824996948242 -9.01973819732666\\
116.837501525879 -9.02035713195801\\
116.849998474121 -9.02097129821777\\
116.862503051758 -9.02158164978027\\
116.875 -9.02219200134277\\
116.887496948242 -9.02280139923096\\
116.900001525879 -9.02341079711914\\
116.912498474121 -9.02401638031006\\
116.925003051758 -9.02461719512939\\
116.9375 -9.02521800994873\\
116.949996948242 -9.02581882476807\\
116.962501525879 -9.0264196395874\\
116.974998474121 -9.02702045440674\\
116.987503051758 -9.02762126922607\\
117 -9.02822208404541\\
117.012496948242 -9.02882099151611\\
117.025001525879 -9.02942085266113\\
117.037498474121 -9.03001976013184\\
117.050003051758 -9.03061962127686\\
117.0625 -9.03121376037598\\
117.074996948242 -9.03180503845215\\
117.087501525879 -9.03239631652832\\
117.099998474121 -9.03298759460449\\
117.112503051758 -9.03357887268066\\
117.125 -9.03416919708252\\
117.137496948242 -9.03475856781006\\
117.150001525879 -9.03534889221191\\
117.162498474121 -9.03593826293945\\
117.175003051758 -9.03652858734131\\
117.1875 -9.03711795806885\\
117.199996948242 -9.03770732879639\\
117.212501525879 -9.03829765319824\\
117.224998474121 -9.03888893127441\\
117.237503051758 -9.03948020935059\\
117.25 -9.04007148742676\\
117.262496948242 -9.04066276550293\\
117.275001525879 -9.0412540435791\\
117.287498474121 -9.04184532165527\\
117.300003051758 -9.04243659973145\\
117.3125 -9.04302787780762\\
117.324996948242 -9.04362392425537\\
117.337501525879 -9.04422378540039\\
117.349998474121 -9.04482269287109\\
117.362503051758 -9.04542255401611\\
117.375 -9.04602241516113\\
117.387496948242 -9.04662322998047\\
117.400001525879 -9.0472240447998\\
117.412498474121 -9.04782485961914\\
117.425003051758 -9.04842567443848\\
117.4375 -9.04903411865234\\
117.449996948242 -9.04964256286621\\
117.462501525879 -9.05025291442871\\
117.474998474121 -9.05086326599121\\
117.487503051758 -9.05147457122803\\
117.5 -9.05209255218506\\
117.512496948242 -9.05271148681641\\
117.525001525879 -9.05333137512207\\
117.537498474121 -9.05395126342773\\
117.550003051758 -9.0545711517334\\
117.5625 -9.05519104003906\\
117.574996948242 -9.05581188201904\\
117.587501525879 -9.05644035339355\\
117.599998474121 -9.05706787109375\\
117.612503051758 -9.05769634246826\\
117.625 -9.05832386016846\\
117.637496948242 -9.05895137786865\\
117.650001525879 -9.05957984924316\\
117.662498474121 -9.06020927429199\\
117.675003051758 -9.06083965301514\\
117.6875 -9.0614767074585\\
117.699996948242 -9.06211376190186\\
117.712501525879 -9.06275177001953\\
117.724998474121 -9.06338882446289\\
117.737503051758 -9.06402778625488\\
117.75 -9.06466674804688\\
117.762496948242 -9.06530570983887\\
117.775001525879 -9.06594467163086\\
117.787498474121 -9.06658267974854\\
117.800003051758 -9.06722068786621\\
117.8125 -9.06785774230957\\
117.824996948242 -9.06849479675293\\
117.837501525879 -9.06912422180176\\
117.849998474121 -9.06975364685059\\
117.862503051758 -9.0703821182251\\
117.875 -9.07100963592529\\
117.887496948242 -9.07162952423096\\
117.900001525879 -9.07224941253662\\
117.912498474121 -9.07286834716797\\
117.925003051758 -9.07348728179932\\
117.9375 -9.07409763336182\\
117.949996948242 -9.07470798492432\\
117.962501525879 -9.0753173828125\\
117.974998474121 -9.0759220123291\\
117.987503051758 -9.07652282714844\\
118 -9.07712268829346\\
118.012496948242 -9.07772159576416\\
118.025001525879 -9.07831573486328\\
118.037498474121 -9.07890701293945\\
118.050003051758 -9.07949829101563\\
118.0625 -9.08008766174316\\
118.074996948242 -9.08067607879639\\
118.087501525879 -9.08125782012939\\
118.099998474121 -9.0818395614624\\
118.112503051758 -9.08242130279541\\
118.125 -9.0830020904541\\
118.137496948242 -9.08358192443848\\
118.150001525879 -9.08415508270264\\
118.162498474121 -9.08472728729248\\
118.175003051758 -9.08529949188232\\
118.1875 -9.08587169647217\\
118.199996948242 -9.08644199371338\\
118.212501525879 -9.08701324462891\\
118.224998474121 -9.08758354187012\\
118.237503051758 -9.08815097808838\\
118.25 -9.08871364593506\\
118.262496948242 -9.08927631378174\\
118.275001525879 -9.08983898162842\\
118.287498474121 -9.0904016494751\\
118.300003051758 -9.09097003936768\\
118.3125 -9.09154033660889\\
118.324996948242 -9.0921106338501\\
118.337501525879 -9.09268188476563\\
118.349998474121 -9.09325313568115\\
118.362503051758 -9.093825340271\\
118.375 -9.09439754486084\\
118.387496948242 -9.09496974945068\\
118.400001525879 -9.09554195404053\\
118.412498474121 -9.09611415863037\\
118.425003051758 -9.09668636322021\\
118.4375 -9.09725856781006\\
118.449996948242 -9.0978307723999\\
118.462501525879 -9.09840297698975\\
118.474998474121 -9.09897518157959\\
118.487503051758 -9.0995512008667\\
118.5 -9.10013103485107\\
118.512496948242 -9.10071086883545\\
118.525001525879 -9.10129165649414\\
118.537498474121 -9.10187149047852\\
118.550003051758 -9.10245323181152\\
118.5625 -9.10303497314453\\
118.574996948242 -9.10361671447754\\
118.587501525879 -9.10419845581055\\
118.599998474121 -9.10478019714355\\
118.612503051758 -9.10536193847656\\
118.625 -9.10594367980957\\
118.637496948242 -9.10652542114258\\
118.650001525879 -9.10710716247559\\
118.662498474121 -9.10768890380859\\
118.675003051758 -9.1082706451416\\
118.6875 -9.10885143280029\\
118.699996948242 -9.10943126678467\\
118.712501525879 -9.11001205444336\\
118.724998474121 -9.11059188842773\\
118.737503051758 -9.11116886138916\\
118.75 -9.11174964904785\\
118.762496948242 -9.11233520507813\\
118.775001525879 -9.11292552947998\\
118.787498474121 -9.11351680755615\\
118.800003051758 -9.11410808563232\\
118.8125 -9.1146993637085\\
118.824996948242 -9.11529445648193\\
118.837501525879 -9.11589431762695\\
118.849998474121 -9.11649322509766\\
118.862503051758 -9.11709308624268\\
118.875 -9.11769199371338\\
118.887496948242 -9.11829280853271\\
118.900001525879 -9.11889362335205\\
118.912498474121 -9.11949443817139\\
118.925003051758 -9.12009716033936\\
118.9375 -9.12069797515869\\
118.949996948242 -9.12129878997803\\
118.962501525879 -9.12189960479736\\
118.974998474121 -9.12249851226807\\
118.987503051758 -9.12309837341309\\
119 -9.12368965148926\\
119.012496948242 -9.12428092956543\\
119.025001525879 -9.1248722076416\\
119.037498474121 -9.12546062469482\\
119.050003051758 -9.12604236602783\\
119.0625 -9.12662410736084\\
119.074996948242 -9.12720394134521\\
119.087501525879 -9.12778091430664\\
119.099998474121 -9.12835311889648\\
119.112503051758 -9.12892532348633\\
119.125 -9.12949657440186\\
119.137496948242 -9.13006496429443\\
119.150001525879 -9.13062763214111\\
119.162498474121 -9.13119029998779\\
119.175003051758 -9.13175201416016\\
119.1875 -9.13230991363525\\
119.199996948242 -9.13286304473877\\
119.212501525879 -9.13341617584229\\
119.224998474121 -9.13396835327148\\
119.237503051758 -9.13451862335205\\
119.25 -9.1350622177124\\
119.262496948242 -9.13560581207275\\
119.275001525879 -9.13614845275879\\
119.287498474121 -9.13669013977051\\
119.300003051758 -9.13722515106201\\
119.3125 -9.1377592086792\\
119.324996948242 -9.13829326629639\\
119.337501525879 -9.13882732391357\\
119.349998474121 -9.13935947418213\\
119.362503051758 -9.139892578125\\
119.375 -9.14042472839355\\
119.387496948242 -9.14095115661621\\
119.400001525879 -9.14147853851318\\
119.412498474121 -9.14200305938721\\
119.425003051758 -9.14252758026123\\
119.4375 -9.14305210113525\\
119.449996948242 -9.14357662200928\\
119.462501525879 -9.14410972595215\\
119.474998474121 -9.1446418762207\\
119.487503051758 -9.14517498016357\\
119.5 -9.14570713043213\\
119.512496948242 -9.14623928070068\\
119.525001525879 -9.14677238464355\\
119.537498474121 -9.14730453491211\\
119.550003051758 -9.14783763885498\\
119.5625 -9.14836978912354\\
119.574996948242 -9.14890289306641\\
119.587501525879 -9.14943599700928\\
119.599998474121 -9.14996814727783\\
119.612503051758 -9.1505012512207\\
119.625 -9.15103530883789\\
119.637496948242 -9.15156936645508\\
119.650001525879 -9.15210342407227\\
119.662498474121 -9.15263748168945\\
119.675003051758 -9.15317153930664\\
119.6875 -9.15370559692383\\
119.699996948242 -9.15423965454102\\
119.712501525879 -9.1547737121582\\
119.724998474121 -9.15530776977539\\
119.737503051758 -9.15584182739258\\
119.75 -9.15637588500977\\
119.762496948242 -9.15690994262695\\
119.775001525879 -9.15744400024414\\
119.787498474121 -9.15797805786133\\
119.800003051758 -9.15851211547852\\
119.8125 -9.1590461730957\\
119.824996948242 -9.15957927703857\\
119.837501525879 -9.16011238098145\\
119.849998474121 -9.16064643859863\\
119.862503051758 -9.16118621826172\\
119.875 -9.16172981262207\\
119.887496948242 -9.16227340698242\\
119.900001525879 -9.16281700134277\\
119.912498474121 -9.16336727142334\\
119.925003051758 -9.16391944885254\\
119.9375 -9.16447257995605\\
119.949996948242 -9.16502571105957\\
119.962501525879 -9.16557788848877\\
119.974998474121 -9.16613006591797\\
119.987503051758 -9.16668319702148\\
120 -9.167236328125\\
120.012496948242 -9.16778945922852\\
120.025001525879 -9.16834259033203\\
120.037498474121 -9.16889572143555\\
120.050003051758 -9.16944885253906\\
120.0625 -9.17000198364258\\
120.074996948242 -9.17055416107178\\
120.087501525879 -9.17110252380371\\
120.099998474121 -9.17164611816406\\
120.112503051758 -9.1721887588501\\
120.125 -9.1727237701416\\
120.137496948242 -9.17325687408447\\
120.150001525879 -9.17378997802734\\
120.162498474121 -9.17431449890137\\
120.175003051758 -9.17483901977539\\
120.1875 -9.1753625869751\\
120.199996948242 -9.17588043212891\\
120.212501525879 -9.17639541625977\\
120.224998474121 -9.17690944671631\\
120.237503051758 -9.17742156982422\\
120.25 -9.17792701721191\\
120.262496948242 -9.17843246459961\\
120.275001525879 -9.17893695831299\\
120.287498474121 -9.17943572998047\\
120.300003051758 -9.179931640625\\
120.3125 -9.18042755126953\\
120.324996948242 -9.18092155456543\\
120.337501525879 -9.1814136505127\\
120.349998474121 -9.18190002441406\\
120.362503051758 -9.18238639831543\\
120.375 -9.1828727722168\\
120.387496948242 -9.18335723876953\\
120.400001525879 -9.18384265899658\\
120.412498474121 -9.18432712554932\\
120.425003051758 -9.18480491638184\\
120.4375 -9.18528175354004\\
120.449996948242 -9.18575859069824\\
120.462501525879 -9.18623542785645\\
120.474998474121 -9.18671226501465\\
120.487503051758 -9.18718910217285\\
120.5 -9.18766593933105\\
120.512496948242 -9.18814277648926\\
120.525001525879 -9.18861961364746\\
120.537498474121 -9.18909645080566\\
120.550003051758 -9.18957328796387\\
120.5625 -9.19005012512207\\
120.574996948242 -9.19052696228027\\
120.587501525879 -9.19100379943848\\
120.599998474121 -9.19148063659668\\
120.612503051758 -9.19195747375488\\
120.625 -9.19243431091309\\
120.637496948242 -9.19291114807129\\
120.650001525879 -9.19338798522949\\
120.662498474121 -9.1938648223877\\
120.675003051758 -9.1943416595459\\
120.6875 -9.1948184967041\\
120.699996948242 -9.1952953338623\\
120.712501525879 -9.19577789306641\\
120.724998474121 -9.19626235961914\\
120.737503051758 -9.19674873352051\\
120.75 -9.19723510742188\\
120.762496948242 -9.19772148132324\\
120.775001525879 -9.19820785522461\\
120.787498474121 -9.19869422912598\\
120.800003051758 -9.19918918609619\\
120.8125 -9.19968414306641\\
120.824996948242 -9.20018005371094\\
120.837501525879 -9.20067596435547\\
120.849998474121 -9.201171875\\
120.862503051758 -9.20166778564453\\
120.875 -9.20216369628906\\
120.887496948242 -9.20266056060791\\
120.900001525879 -9.20316505432129\\
120.912498474121 -9.20366859436035\\
120.925003051758 -9.20417308807373\\
120.9375 -9.20467758178711\\
120.949996948242 -9.2051830291748\\
120.962501525879 -9.2056884765625\\
120.974998474121 -9.2061939239502\\
120.987503051758 -9.20669937133789\\
121 -9.20720481872559\\
121.012496948242 -9.20771026611328\\
121.025001525879 -9.20821571350098\\
121.037498474121 -9.20872116088867\\
121.050003051758 -9.20922660827637\\
121.0625 -9.20973014831543\\
121.074996948242 -9.21023082733154\\
121.087501525879 -9.21072673797607\\
121.099998474121 -9.21122264862061\\
121.112503051758 -9.21171760559082\\
121.125 -9.21220684051514\\
121.137496948242 -9.2126932144165\\
121.150001525879 -9.21317958831787\\
121.162498474121 -9.21366119384766\\
121.175003051758 -9.21413803100586\\
121.1875 -9.21461486816406\\
121.199996948242 -9.21508884429932\\
121.212501525879 -9.21555614471436\\
121.224998474121 -9.21602344512939\\
121.237503051758 -9.21649074554443\\
121.25 -9.21695423126221\\
121.262496948242 -9.21741199493408\\
121.275001525879 -9.21786975860596\\
121.287498474121 -9.21832656860352\\
121.300003051758 -9.21877861022949\\
121.3125 -9.2192268371582\\
121.324996948242 -9.21967506408691\\
121.337501525879 -9.22012329101563\\
121.349998474121 -9.2205696105957\\
121.362503051758 -9.2210168838501\\
121.375 -9.22145557403564\\
121.387496948242 -9.22189426422119\\
121.400001525879 -9.22233772277832\\
121.412498474121 -9.2227840423584\\
121.425003051758 -9.22323131561279\\
121.4375 -9.22367095947266\\
121.449996948242 -9.2241096496582\\
121.462501525879 -9.22454833984375\\
121.474998474121 -9.2249870300293\\
121.487503051758 -9.22542572021484\\
121.5 -9.22586441040039\\
121.512496948242 -9.22630310058594\\
121.525001525879 -9.22674179077148\\
121.537498474121 -9.22718048095703\\
121.550003051758 -9.22761917114258\\
121.5625 -9.22805786132813\\
121.574996948242 -9.22849655151367\\
121.587501525879 -9.22893524169922\\
121.599998474121 -9.22937393188477\\
121.612503051758 -9.22981262207031\\
121.625 -9.23025035858154\\
121.637496948242 -9.23068714141846\\
121.650001525879 -9.23111820220947\\
121.662498474121 -9.23155403137207\\
121.675003051758 -9.2319917678833\\
121.6875 -9.23242855072021\\
121.699996948242 -9.23286724090576\\
121.712501525879 -9.23330593109131\\
121.724998474121 -9.23374462127686\\
121.737503051758 -9.2341833114624\\
121.75 -9.23462200164795\\
121.762496948242 -9.2350606918335\\
121.775001525879 -9.23550415039063\\
121.787498474121 -9.2359504699707\\
121.800003051758 -9.23639869689941\\
121.8125 -9.23684692382813\\
121.824996948242 -9.23729515075684\\
121.837501525879 -9.23774337768555\\
121.849998474121 -9.23819160461426\\
121.862503051758 -9.2386474609375\\
121.875 -9.23910331726074\\
121.887496948242 -9.23956108093262\\
121.900001525879 -9.24001884460449\\
121.912498474121 -9.24047660827637\\
121.925003051758 -9.24093437194824\\
121.9375 -9.24139213562012\\
121.949996948242 -9.24185752868652\\
121.962501525879 -9.24232387542725\\
121.974998474121 -9.24278926849365\\
121.987503051758 -9.24325656890869\\
122 -9.24372386932373\\
122.012496948242 -9.24419116973877\\
122.025001525879 -9.24465847015381\\
122.037498474121 -9.24512386322021\\
122.050003051758 -9.24558639526367\\
122.0625 -9.24604415893555\\
122.074996948242 -9.24650192260742\\
122.087501525879 -9.2469596862793\\
122.099998474121 -9.24741744995117\\
122.112503051758 -9.24787521362305\\
122.125 -9.24833202362061\\
122.137496948242 -9.24878787994385\\
122.150001525879 -9.24923610687256\\
122.162498474121 -9.24968433380127\\
122.175003051758 -9.25013256072998\\
122.1875 -9.25057888031006\\
122.199996948242 -9.25101947784424\\
122.212501525879 -9.25145816802979\\
122.224998474121 -9.25189685821533\\
122.237503051758 -9.25233459472656\\
122.25 -9.25276756286621\\
122.262496948242 -9.25319671630859\\
122.275001525879 -9.25362586975098\\
122.287498474121 -9.25405406951904\\
122.300003051758 -9.25447559356689\\
122.3125 -9.25489521026611\\
122.324996948242 -9.25531482696533\\
122.337501525879 -9.25573348999023\\
122.349998474121 -9.2561502456665\\
122.362503051758 -9.25656032562256\\
122.375 -9.25697040557861\\
122.387496948242 -9.25738048553467\\
122.400001525879 -9.25779056549072\\
122.412498474121 -9.25820064544678\\
122.425003051758 -9.25860977172852\\
122.4375 -9.25901794433594\\
122.449996948242 -9.25942134857178\\
122.462501525879 -9.25982189178467\\
122.474998474121 -9.26022243499756\\
122.487503051758 -9.26062297821045\\
122.5 -9.26102256774902\\
122.512496948242 -9.26142024993896\\
122.525001525879 -9.26181125640869\\
122.537498474121 -9.26220226287842\\
122.550003051758 -9.26259326934814\\
122.5625 -9.26298427581787\\
122.574996948242 -9.2633752822876\\
122.587501525879 -9.26376628875732\\
122.599998474121 -9.26415729522705\\
122.612503051758 -9.26454830169678\\
122.625 -9.2649393081665\\
122.637496948242 -9.26533031463623\\
122.650001525879 -9.26572132110596\\
122.662498474121 -9.26611232757568\\
122.675003051758 -9.26650524139404\\
122.6875 -9.26689720153809\\
122.699996948242 -9.26729202270508\\
122.712501525879 -9.26769161224365\\
122.724998474121 -9.26809120178223\\
122.737503051758 -9.26849174499512\\
122.75 -9.26889228820801\\
122.762496948242 -9.2692928314209\\
122.775001525879 -9.26969432830811\\
122.787498474121 -9.27010250091553\\
122.800003051758 -9.27051162719727\\
122.8125 -9.27092170715332\\
122.824996948242 -9.27133178710938\\
122.837501525879 -9.27174186706543\\
122.849998474121 -9.2721586227417\\
122.862503051758 -9.27257823944092\\
122.875 -9.27299785614014\\
122.887496948242 -9.27341747283936\\
122.900001525879 -9.27383708953857\\
122.912498474121 -9.27425670623779\\
122.925003051758 -9.27468204498291\\
122.9375 -9.27510929107666\\
122.949996948242 -9.27553653717041\\
122.962501525879 -9.27596473693848\\
122.974998474121 -9.27639293670654\\
122.987503051758 -9.27682209014893\\
123 -9.27724933624268\\
123.012496948242 -9.27767372131348\\
123.025001525879 -9.2780933380127\\
123.037498474121 -9.27851295471191\\
123.050003051758 -9.27893257141113\\
123.0625 -9.27935218811035\\
123.074996948242 -9.27977180480957\\
123.087501525879 -9.28019142150879\\
123.099998474121 -9.28061008453369\\
123.112503051758 -9.28102684020996\\
123.125 -9.28143692016602\\
123.137496948242 -9.28184700012207\\
123.150001525879 -9.28225612640381\\
123.162498474121 -9.28265762329102\\
123.175003051758 -9.28305816650391\\
123.1875 -9.2834587097168\\
123.199996948242 -9.28385734558105\\
123.212501525879 -9.28424835205078\\
123.224998474121 -9.28463935852051\\
123.237503051758 -9.28503036499023\\
123.25 -9.2854175567627\\
123.262496948242 -9.28579902648926\\
123.275001525879 -9.28618049621582\\
123.287498474121 -9.28656196594238\\
123.300003051758 -9.28693962097168\\
123.3125 -9.28731155395508\\
123.324996948242 -9.28768348693848\\
123.337501525879 -9.28805541992188\\
123.349998474121 -9.28842544555664\\
123.362503051758 -9.28879070281982\\
123.375 -9.28915309906006\\
123.387496948242 -9.28951549530029\\
123.400001525879 -9.28987789154053\\
123.412498474121 -9.29024028778076\\
123.425003051758 -9.290602684021\\
123.4375 -9.29096508026123\\
123.449996948242 -9.29132747650146\\
123.462501525879 -9.2916898727417\\
123.474998474121 -9.29205226898193\\
123.487503051758 -9.29241466522217\\
123.5 -9.29277610778809\\
123.512496948242 -9.29313087463379\\
123.525001525879 -9.29348373413086\\
123.537498474121 -9.29383659362793\\
123.550003051758 -9.294189453125\\
123.5625 -9.29454231262207\\
123.574996948242 -9.29489517211914\\
123.587501525879 -9.29524803161621\\
123.599998474121 -9.29560089111328\\
123.612503051758 -9.29595375061035\\
123.625 -9.29630661010742\\
123.637496948242 -9.29665946960449\\
123.650001525879 -9.29701232910156\\
123.662498474121 -9.29736518859863\\
123.675003051758 -9.2977180480957\\
123.6875 -9.29807662963867\\
123.699996948242 -9.29843902587891\\
123.712501525879 -9.29880142211914\\
123.724998474121 -9.29916381835938\\
123.737503051758 -9.29952812194824\\
123.75 -9.29989910125732\\
123.762496948242 -9.30027103424072\\
123.775001525879 -9.30064296722412\\
123.787498474121 -9.30101490020752\\
123.800003051758 -9.30139446258545\\
123.8125 -9.3017749786377\\
123.824996948242 -9.30215644836426\\
123.837501525879 -9.30253791809082\\
123.849998474121 -9.30291938781738\\
123.862503051758 -9.30330181121826\\
123.875 -9.30369091033936\\
123.887496948242 -9.30408096313477\\
123.900001525879 -9.30447196960449\\
123.912498474121 -9.30486297607422\\
123.925003051758 -9.30525398254395\\
123.9375 -9.30564498901367\\
123.949996948242 -9.3060359954834\\
123.962501525879 -9.30642700195313\\
123.974998474121 -9.30681610107422\\
123.987503051758 -9.30720615386963\\
124 -9.30759143829346\\
124.012496948242 -9.30797290802002\\
124.025001525879 -9.30835437774658\\
124.037498474121 -9.30873489379883\\
124.050003051758 -9.30911064147949\\
124.0625 -9.30948257446289\\
124.074996948242 -9.30985450744629\\
124.087501525879 -9.31022644042969\\
124.099998474121 -9.31059551239014\\
124.112503051758 -9.31095790863037\\
124.125 -9.31132030487061\\
124.137496948242 -9.31168174743652\\
124.150001525879 -9.31203842163086\\
124.162498474121 -9.31239128112793\\
124.175003051758 -9.312744140625\\
124.1875 -9.31309413909912\\
124.199996948242 -9.31343746185303\\
124.212501525879 -9.31378078460693\\
124.224998474121 -9.31412410736084\\
124.237503051758 -9.31446552276611\\
124.25 -9.31479930877686\\
124.262496948242 -9.3151330947876\\
124.275001525879 -9.31546688079834\\
124.287498474121 -9.31580066680908\\
124.300003051758 -9.31613349914551\\
124.3125 -9.31646537780762\\
124.324996948242 -9.31679058074951\\
124.337501525879 -9.31711483001709\\
124.349998474121 -9.31743907928467\\
124.362503051758 -9.31776332855225\\
124.375 -9.31808757781982\\
124.387496948242 -9.3184118270874\\
124.400001525879 -9.31873607635498\\
124.412498474121 -9.31906032562256\\
124.425003051758 -9.31938457489014\\
124.4375 -9.31970882415771\\
124.449996948242 -9.32003307342529\\
124.462501525879 -9.32035636901855\\
124.474998474121 -9.3206787109375\\
124.487503051758 -9.32099342346191\\
124.5 -9.32131385803223\\
124.512496948242 -9.32163619995117\\
124.525001525879 -9.32195949554443\\
124.537498474121 -9.32228183746338\\
124.550003051758 -9.32260513305664\\
124.5625 -9.32292747497559\\
124.574996948242 -9.32324981689453\\
124.587501525879 -9.32357311248779\\
124.599998474121 -9.32389736175537\\
124.612503051758 -9.32422161102295\\
124.625 -9.32454586029053\\
124.637496948242 -9.32487010955811\\
124.650001525879 -9.32519435882568\\
124.662498474121 -9.32552146911621\\
124.675003051758 -9.32585430145264\\
124.6875 -9.32618618011475\\
124.699996948242 -9.32651901245117\\
124.712501525879 -9.32685279846191\\
124.724998474121 -9.32718658447266\\
124.737503051758 -9.3275203704834\\
124.75 -9.32785415649414\\
124.762496948242 -9.32819080352783\\
124.775001525879 -9.32853317260742\\
124.787498474121 -9.32887649536133\\
124.800003051758 -9.32921981811523\\
124.8125 -9.32956314086914\\
124.824996948242 -9.32990646362305\\
124.837501525879 -9.33024978637695\\
124.849998474121 -9.33059310913086\\
124.862503051758 -9.33093643188477\\
124.875 -9.33127975463867\\
124.887496948242 -9.33162307739258\\
124.900001525879 -9.33196830749512\\
124.912498474121 -9.33231925964355\\
124.925003051758 -9.33267116546631\\
124.9375 -9.33301830291748\\
124.949996948242 -9.33336162567139\\
124.962501525879 -9.33370494842529\\
124.974998474121 -9.3340482711792\\
124.987503051758 -9.33439159393311\\
125 -9.33473491668701\\
125.012496948242 -9.33507823944092\\
125.025001525879 -9.33541774749756\\
125.037498474121 -9.3357515335083\\
125.050003051758 -9.33608531951904\\
125.0625 -9.33641910552979\\
125.074996948242 -9.33675289154053\\
125.087501525879 -9.33708667755127\\
125.099998474121 -9.3374195098877\\
125.112503051758 -9.33774757385254\\
125.125 -9.33807182312012\\
125.137496948242 -9.3383960723877\\
125.150001525879 -9.33872032165527\\
125.162498474121 -9.33903980255127\\
125.175003051758 -9.33935451507568\\
125.1875 -9.3396692276001\\
125.199996948242 -9.3399829864502\\
125.212501525879 -9.34029483795166\\
125.224998474121 -9.34060001373291\\
125.237503051758 -9.34090518951416\\
125.25 -9.34121036529541\\
125.262496948242 -9.34151554107666\\
125.275001525879 -9.34181976318359\\
125.287498474121 -9.34211540222168\\
125.300003051758 -9.34241104125977\\
125.3125 -9.34270668029785\\
125.324996948242 -9.34300231933594\\
125.337501525879 -9.34329700469971\\
125.349998474121 -9.34359073638916\\
125.362503051758 -9.34388065338135\\
125.375 -9.34416675567627\\
125.387496948242 -9.34445285797119\\
125.400001525879 -9.34473896026611\\
125.412498474121 -9.34502506256104\\
125.425003051758 -9.34531116485596\\
125.4375 -9.34559726715088\\
125.449996948242 -9.3458833694458\\
125.462501525879 -9.34616947174072\\
125.474998474121 -9.34645557403564\\
125.487503051758 -9.34674167633057\\
125.5 -9.34702777862549\\
125.512496948242 -9.34731388092041\\
125.525001525879 -9.34759998321533\\
125.537498474121 -9.34788703918457\\
125.550003051758 -9.34818172454834\\
125.5625 -9.34847640991211\\
125.574996948242 -9.3487720489502\\
125.587501525879 -9.34906768798828\\
125.599998474121 -9.34936332702637\\
125.612503051758 -9.34965896606445\\
125.625 -9.34995937347412\\
125.637496948242 -9.35026454925537\\
125.650001525879 -9.35056972503662\\
125.662498474121 -9.35087490081787\\
125.675003051758 -9.35118007659912\\
125.6875 -9.35148525238037\\
125.699996948242 -9.35179042816162\\
125.712501525879 -9.35210227966309\\
125.724998474121 -9.35241603851318\\
125.737503051758 -9.3527307510376\\
125.75 -9.35304546356201\\
125.762496948242 -9.35336017608643\\
125.775001525879 -9.35367488861084\\
125.787498474121 -9.35398960113525\\
125.800003051758 -9.35430431365967\\
125.8125 -9.35461902618408\\
125.824996948242 -9.3549337387085\\
125.837501525879 -9.35524845123291\\
125.849998474121 -9.35556316375732\\
125.862503051758 -9.35587787628174\\
125.875 -9.35619258880615\\
125.887496948242 -9.35650730133057\\
125.900001525879 -9.35682201385498\\
125.912498474121 -9.35713672637939\\
125.925003051758 -9.35745143890381\\
125.9375 -9.35776138305664\\
125.949996948242 -9.35806655883789\\
125.962501525879 -9.35837173461914\\
125.974998474121 -9.35867691040039\\
125.987503051758 -9.35898208618164\\
126 -9.35928630828857\\
126.012496948242 -9.35958290100098\\
126.025001525879 -9.35987854003906\\
126.037498474121 -9.36017417907715\\
126.050003051758 -9.36046886444092\\
126.0625 -9.36075496673584\\
126.074996948242 -9.36104106903076\\
126.087501525879 -9.36132717132568\\
126.099998474121 -9.36161041259766\\
126.112503051758 -9.36188697814941\\
126.125 -9.36216354370117\\
126.137496948242 -9.36244010925293\\
126.150001525879 -9.36271667480469\\
126.162498474121 -9.36299324035645\\
126.175003051758 -9.3632698059082\\
126.1875 -9.36354637145996\\
126.199996948242 -9.36382293701172\\
126.212501525879 -9.36409950256348\\
126.224998474121 -9.36437606811523\\
126.237503051758 -9.36465263366699\\
126.25 -9.3649206161499\\
126.262496948242 -9.3651876449585\\
126.275001525879 -9.36545467376709\\
126.287498474121 -9.36572170257568\\
126.300003051758 -9.36598777770996\\
126.3125 -9.36624526977539\\
126.324996948242 -9.36650276184082\\
126.337501525879 -9.36676025390625\\
126.349998474121 -9.36701774597168\\
126.362503051758 -9.36727523803711\\
126.375 -9.36753273010254\\
126.387496948242 -9.36779022216797\\
126.400001525879 -9.3680477142334\\
126.412498474121 -9.36830520629883\\
126.425003051758 -9.36856269836426\\
126.4375 -9.36882019042969\\
126.449996948242 -9.36907768249512\\
126.462501525879 -9.36933517456055\\
126.474998474121 -9.36959266662598\\
126.487503051758 -9.36985015869141\\
126.5 -9.37011528015137\\
126.512496948242 -9.37038230895996\\
126.525001525879 -9.37064933776855\\
126.537498474121 -9.37091636657715\\
126.550003051758 -9.37118816375732\\
126.5625 -9.37146377563477\\
126.574996948242 -9.37174034118652\\
126.587501525879 -9.37201690673828\\
126.599998474121 -9.37229347229004\\
126.612503051758 -9.37257862091064\\
126.625 -9.37286376953125\\
126.637496948242 -9.37314987182617\\
126.650001525879 -9.37343597412109\\
126.662498474121 -9.37372207641602\\
126.675003051758 -9.37401008605957\\
126.6875 -9.37430381774902\\
126.699996948242 -9.37459754943848\\
126.712501525879 -9.37489223480225\\
126.724998474121 -9.37518787384033\\
126.737503051758 -9.3754825592041\\
126.75 -9.37577629089355\\
126.762496948242 -9.37607002258301\\
126.775001525879 -9.37636470794678\\
126.787498474121 -9.37665843963623\\
126.800003051758 -9.376953125\\
126.8125 -9.37724685668945\\
126.824996948242 -9.37753391265869\\
126.837501525879 -9.37782001495361\\
126.849998474121 -9.37810611724854\\
126.862503051758 -9.37839221954346\\
126.875 -9.37867832183838\\
126.887496948242 -9.3789644241333\\
126.900001525879 -9.37924957275391\\
126.912498474121 -9.37952899932861\\
126.925003051758 -9.37980556488037\\
126.9375 -9.38008213043213\\
126.949996948242 -9.38035869598389\\
126.962501525879 -9.38063526153564\\
126.974998474121 -9.3809118270874\\
126.987503051758 -9.38118648529053\\
127 -9.38145351409912\\
127.012496948242 -9.38172054290771\\
127.025001525879 -9.38198757171631\\
127.037498474121 -9.38225078582764\\
127.050003051758 -9.38250827789307\\
127.0625 -9.3827657699585\\
127.074996948242 -9.38302326202393\\
127.087501525879 -9.38327980041504\\
127.099998474121 -9.3835277557373\\
127.112503051758 -9.38377571105957\\
127.125 -9.38402366638184\\
127.137496948242 -9.3842716217041\\
127.150001525879 -9.38451957702637\\
127.162498474121 -9.38476753234863\\
127.175003051758 -9.3850154876709\\
127.1875 -9.38526344299316\\
127.199996948242 -9.38551044464111\\
127.212501525879 -9.38575744628906\\
127.224998474121 -9.38599872589111\\
127.237503051758 -9.38623714447021\\
127.25 -9.38647556304932\\
127.262496948242 -9.38671398162842\\
127.275001525879 -9.38695240020752\\
127.287498474121 -9.38719177246094\\
127.300003051758 -9.38743877410889\\
127.3125 -9.38768577575684\\
127.324996948242 -9.3879337310791\\
127.337501525879 -9.38819026947021\\
127.349998474121 -9.38846588134766\\
127.362503051758 -9.38877391815186\\
127.375 -9.38912105560303\\
127.387496948242 -9.38947296142578\\
127.400001525879 -9.38979625701904\\
127.412498474121 -9.39009380340576\\
127.425003051758 -9.39037418365479\\
127.4375 -9.39064121246338\\
127.449996948242 -9.39089584350586\\
127.462501525879 -9.39113998413086\\
127.474998474121 -9.39137840270996\\
127.487503051758 -9.3916130065918\\
127.5 -9.39184188842773\\
127.512496948242 -9.39207077026367\\
127.525001525879 -9.39229965209961\\
127.537498474121 -9.39252853393555\\
127.550003051758 -9.39275741577148\\
127.5625 -9.39298152923584\\
127.574996948242 -9.39320087432861\\
127.587501525879 -9.39342021942139\\
127.599998474121 -9.39363956451416\\
127.612503051758 -9.39385890960693\\
127.625 -9.39407825469971\\
127.637496948242 -9.39430141448975\\
127.650001525879 -9.39452743530273\\
127.662498474121 -9.39475440979004\\
127.675003051758 -9.39498233795166\\
127.6875 -9.3952112197876\\
127.699996948242 -9.39544010162354\\
127.712501525879 -9.39566898345947\\
127.724998474121 -9.39589786529541\\
127.737503051758 -9.39613342285156\\
127.75 -9.39637184143066\\
127.762496948242 -9.39661026000977\\
127.775001525879 -9.3968563079834\\
127.787498474121 -9.39710426330566\\
127.800003051758 -9.39735507965088\\
127.8125 -9.39761161804199\\
127.824996948242 -9.39786911010742\\
127.837501525879 -9.39813232421875\\
127.849998474121 -9.39839935302734\\
127.862503051758 -9.39866828918457\\
127.875 -9.39894390106201\\
127.887496948242 -9.39922046661377\\
127.900001525879 -9.39949703216553\\
127.912498474121 -9.39977359771729\\
127.925003051758 -9.40005874633789\\
127.9375 -9.4003438949585\\
127.949996948242 -9.40062999725342\\
127.962501525879 -9.40091609954834\\
127.974998474121 -9.40120029449463\\
127.987503051758 -9.40148067474365\\
128 -9.40175724029541\\
128.012496948242 -9.40203380584717\\
128.024993896484 -9.40230941772461\\
128.037506103516 -9.4025764465332\\
128.050003051758 -9.4028434753418\\
128.0625 -9.40310955047607\\
128.074996948242 -9.40337181091309\\
128.087493896484 -9.40362930297852\\
128.100006103516 -9.40388679504395\\
128.112503051758 -9.40414237976074\\
128.125 -9.40439033508301\\
128.137496948242 -9.40463829040527\\
128.149993896484 -9.40488529205322\\
128.162506103516 -9.40512466430664\\
128.175003051758 -9.40536308288574\\
128.1875 -9.40560150146484\\
128.199996948242 -9.40583801269531\\
128.212493896484 -9.40606689453125\\
128.225006103516 -9.40629577636719\\
128.237503051758 -9.40652465820313\\
128.25 -9.40675258636475\\
128.262496948242 -9.40697383880615\\
128.274993896484 -9.40719318389893\\
128.287506103516 -9.4074125289917\\
128.300003051758 -9.40763187408447\\
128.3125 -9.40785121917725\\
128.324996948242 -9.40807056427002\\
128.337493896484 -9.40828990936279\\
128.350006103516 -9.40850925445557\\
128.362503051758 -9.40872859954834\\
128.375 -9.40894794464111\\
128.387496948242 -9.40916728973389\\
128.399993896484 -9.40938663482666\\
128.412506103516 -9.40960597991943\\
128.425003051758 -9.40982532501221\\
128.4375 -9.41004467010498\\
128.449996948242 -9.41026401519775\\
128.462493896484 -9.41048336029053\\
128.475006103516 -9.4107027053833\\
128.487503051758 -9.41092205047607\\
128.5 -9.41114139556885\\
128.512496948242 -9.41136360168457\\
128.524993896484 -9.41159152984619\\
128.537506103516 -9.41182041168213\\
128.550003051758 -9.41204929351807\\
128.5625 -9.412278175354\\
128.574996948242 -9.41251468658447\\
128.587493896484 -9.41275310516357\\
128.600006103516 -9.41299152374268\\
128.612503051758 -9.41322994232178\\
128.625 -9.41347312927246\\
128.637496948242 -9.41372013092041\\
128.649993896484 -9.41396808624268\\
128.662506103516 -9.41421604156494\\
128.675003051758 -9.41446399688721\\
128.6875 -9.41471195220947\\
128.699996948242 -9.41496753692627\\
128.712493896484 -9.41522407531738\\
128.725006103516 -9.41548156738281\\
128.737503051758 -9.41573810577393\\
128.75 -9.41599464416504\\
128.762496948242 -9.41625118255615\\
128.774993896484 -9.41650772094727\\
128.787506103516 -9.4167652130127\\
128.800003051758 -9.41702175140381\\
128.8125 -9.41727447509766\\
128.824996948242 -9.41752243041992\\
128.837493896484 -9.41777038574219\\
128.850006103516 -9.41801834106445\\
128.862503051758 -9.41826629638672\\
128.875 -9.41851425170898\\
128.887496948242 -9.41876125335693\\
128.899993896484 -9.41900730133057\\
128.912506103516 -9.41924571990967\\
128.925003051758 -9.41948413848877\\
128.9375 -9.41972255706787\\
128.949996948242 -9.41996002197266\\
128.962493896484 -9.42019557952881\\
128.975006103516 -9.42042446136475\\
128.987503051758 -9.42065334320068\\
129 -9.42088222503662\\
129.012496948242 -9.42111110687256\\
129.024993896484 -9.42133903503418\\
129.037506103516 -9.42156791687012\\
129.050003051758 -9.42178821563721\\
129.0625 -9.42200756072998\\
129.074996948242 -9.42222690582275\\
129.087493896484 -9.42244625091553\\
129.100006103516 -9.4226655960083\\
129.112503051758 -9.42288494110107\\
129.125 -9.42310428619385\\
129.137496948242 -9.42332363128662\\
129.149993896484 -9.42354297637939\\
129.162506103516 -9.42376232147217\\
129.175003051758 -9.42398071289063\\
129.1875 -9.42419910430908\\
129.199996948242 -9.42441749572754\\
129.212493896484 -9.424635887146\\
129.225006103516 -9.42485523223877\\
129.237503051758 -9.42507362365723\\
129.25 -9.42529201507568\\
129.262496948242 -9.42551040649414\\
129.274993896484 -9.4257287979126\\
129.287506103516 -9.42594814300537\\
129.300003051758 -9.42616748809814\\
129.3125 -9.42638683319092\\
129.324996948242 -9.42660617828369\\
129.337493896484 -9.42682552337646\\
129.350006103516 -9.42704486846924\\
129.362503051758 -9.42726421356201\\
129.375 -9.427490234375\\
129.387496948242 -9.42771816253662\\
129.399993896484 -9.42794704437256\\
129.412506103516 -9.4281759262085\\
129.425003051758 -9.42840480804443\\
129.4375 -9.42863845825195\\
129.449996948242 -9.42887592315674\\
129.462493896484 -9.42911434173584\\
129.475006103516 -9.42935276031494\\
129.487503051758 -9.42959117889404\\
129.5 -9.42982959747314\\
129.512496948242 -9.43006801605225\\
129.524993896484 -9.43030643463135\\
129.537506103516 -9.4305477142334\\
129.550003051758 -9.43079471588135\\
129.5625 -9.4310417175293\\
129.574996948242 -9.43128871917725\\
129.587493896484 -9.4315357208252\\
129.600006103516 -9.43178367614746\\
129.612503051758 -9.4320240020752\\
129.625 -9.4322624206543\\
129.637496948242 -9.43250179290771\\
129.649993896484 -9.43274974822998\\
129.662506103516 -9.43300342559814\\
129.675003051758 -9.43326091766357\\
129.6875 -9.43352699279785\\
129.699996948242 -9.43379402160645\\
129.712493896484 -9.43406772613525\\
129.725006103516 -9.43434429168701\\
129.737503051758 -9.43462085723877\\
129.75 -9.43490409851074\\
129.762496948242 -9.43518829345703\\
129.774993896484 -9.43546485900879\\
129.787506103516 -9.43574142456055\\
129.800003051758 -9.43601322174072\\
129.8125 -9.43628025054932\\
129.824996948242 -9.43654632568359\\
129.837493896484 -9.43681240081787\\
129.850006103516 -9.43707942962646\\
129.862503051758 -9.43734550476074\\
129.875 -9.43761253356934\\
129.887496948242 -9.43787956237793\\
129.899993896484 -9.43814659118652\\
129.912506103516 -9.43841361999512\\
129.925003051758 -9.43868923187256\\
129.9375 -9.43896484375\\
129.949996948242 -9.43924140930176\\
129.962493896484 -9.43952751159668\\
129.975006103516 -9.43983268737793\\
129.987503051758 -9.44020748138428\\
130 -9.44131755828857\\
130.012496948242 -9.4452543258667\\
130.024993896484 -9.45332050323486\\
130.037506103516 -9.46887683868408\\
130.050003051758 -9.48755931854248\\
130.0625 -9.50713062286377\\
130.074996948242 -9.52669620513916\\
130.087493896484 -9.54593086242676\\
130.100006103516 -9.56468677520752\\
130.112503051758 -9.58277225494385\\
130.125 -9.60006427764893\\
130.137496948242 -9.61643028259277\\
130.149993896484 -9.63185977935791\\
130.162506103516 -9.64658546447754\\
130.175003051758 -9.6607551574707\\
130.1875 -9.67442035675049\\
130.199996948242 -9.68744945526123\\
130.212493896484 -9.69965744018555\\
130.225006103516 -9.7110538482666\\
130.237503051758 -9.72163963317871\\
130.25 -9.73137187957764\\
130.262496948242 -9.74019241333008\\
130.274993896484 -9.7480640411377\\
130.287506103516 -9.75498294830322\\
130.300003051758 -9.76103496551514\\
130.3125 -9.78083992004395\\
130.324996948242 -9.82388401031494\\
130.337493896484 -9.86302375793457\\
130.350006103516 -9.90105819702148\\
130.362014770508 -9.9348669052124\\
130.37451171875 -9.96505451202393\\
130.387008666992 -9.98692512512207\\
130.397430419922 -9.99517631530762\\
130.409927368164 -9.99990940093994\\
130.422424316406 -10.0031766891479\\
130.434921264648 -10.0058917999268\\
130.44743347168 -10.0084018707275\\
130.459930419922 -10.0108432769775\\
130.472427368164 -10.0131826400757\\
130.484924316406 -10.0154628753662\\
130.497421264648 -10.0177040100098\\
130.50993347168 -10.0199193954468\\
130.522430419922 -10.0222311019897\\
130.534927368164 -10.0249576568604\\
130.547424316406 -10.0283508300781\\
130.559921264648 -10.0549802780151\\
130.57243347168 -10.1039972305298\\
130.584930419922 -10.1488647460938\\
130.597427368164 -10.1759538650513\\
130.609924316406 -10.2004556655884\\
130.622421264648 -10.22034740448\\
130.63493347168 -10.2338609695435\\
130.647430419922 -10.2439317703247\\
130.659927368164 -10.2525396347046\\
130.672424316406 -10.2604207992554\\
130.684921264648 -10.2677717208862\\
130.69743347168 -10.2746706008911\\
130.709930419922 -10.2811365127563\\
130.722427368164 -10.2871799468994\\
130.734924316406 -10.2928066253662\\
130.747421264648 -10.2980451583862\\
130.75993347168 -10.3040552139282\\
130.772430419922 -10.3322629928589\\
130.784927368164 -10.3760032653809\\
130.797424316406 -10.4144821166992\\
130.809921264648 -10.4364366531372\\
130.82243347168 -10.4570589065552\\
130.834930419922 -10.4777526855469\\
130.847427368164 -10.4983434677124\\
130.859924316406 -10.5183048248291\\
130.872421264648 -10.5369863510132\\
130.88493347168 -10.5539331436157\\
130.897430419922 -10.5690231323242\\
130.909927368164 -10.5816955566406\\
130.922424316406 -10.5910882949829\\
130.934921264648 -10.5971193313599\\
130.94743347168 -10.6069555282593\\
130.959930419922 -10.629921913147\\
130.972427368164 -10.6610517501831\\
130.984924316406 -10.6956014633179\\
130.997421264648 -10.7291898727417\\
131.00993347168 -10.7579145431519\\
131.022430419922 -10.778416633606\\
131.034927368164 -10.7904291152954\\
131.047424316406 -10.7962169647217\\
131.059921264648 -10.7980356216431\\
131.07243347168 -10.7986660003662\\
131.084930419922 -10.7990531921387\\
131.097427368164 -10.799524307251\\
131.109924316406 -10.8067111968994\\
131.122421264648 -10.8378257751465\\
131.13493347168 -10.8744926452637\\
131.147430419922 -10.9058837890625\\
131.159927368164 -10.92737865448\\
131.172424316406 -10.9372749328613\\
131.184921264648 -10.9400434494019\\
131.19743347168 -10.941089630127\\
131.209930419922 -10.9418621063232\\
131.222427368164 -10.9425172805786\\
131.234924316406 -10.943097114563\\
131.247421264648 -10.94358253479\\
131.25993347168 -10.9437055587769\\
131.272430419922 -10.9481544494629\\
131.284927368164 -10.9899616241455\\
131.291900634766 -11.0288286209106\\
131.297744750977 -11.0589714050293\\
131.304916381836 -11.0849914550781\\
131.315460205078 -11.1012134552002\\
131.32795715332 -11.1040782928467\\
131.340454101563 -11.1056127548218\\
131.352951049805 -11.1071929931641\\
131.365447998047 -11.1087369918823\\
131.377960205078 -11.1100378036499\\
131.39045715332 -11.1112604141235\\
131.402954101563 -11.1123428344727\\
131.415451049805 -11.1131772994995\\
131.427947998047 -11.1137437820435\\
131.438369750977 -11.1140041351318\\
131.450881958008 -11.1140546798706\\
131.46337890625 -11.1138439178467\\
131.475875854492 -11.1134557723999\\
131.488372802734 -11.1131258010864\\
131.500869750977 -11.1129751205444\\
131.513381958008 -11.1129379272461\\
131.52587890625 -11.1129789352417\\
131.538375854492 -11.1130876541138\\
131.550872802734 -11.1132774353027\\
131.563369750977 -11.1135501861572\\
131.575881958008 -11.1139249801636\\
131.58837890625 -11.1143779754639\\
131.600875854492 -11.1148624420166\\
131.613372802734 -11.1153545379639\\
131.625869750977 -11.1158304214478\\
131.638381958008 -11.1162843704224\\
131.65087890625 -11.1167163848877\\
131.663375854492 -11.1171293258667\\
131.675872802734 -11.1175289154053\\
131.688369750977 -11.1179122924805\\
131.700881958008 -11.1182727813721\\
131.71337890625 -11.1186017990112\\
131.725875854492 -11.1188955307007\\
131.738372802734 -11.1191625595093\\
131.750869750977 -11.1194067001343\\
131.763381958008 -11.1196279525757\\
131.77587890625 -11.1198215484619\\
131.788375854492 -11.1199893951416\\
131.800872802734 -11.120135307312\\
131.813369750977 -11.120265007019\\
131.825881958008 -11.1203889846802\\
131.83837890625 -11.1205158233643\\
131.850875854492 -11.1206541061401\\
131.863372802734 -11.1208086013794\\
131.875869750977 -11.120979309082\\
131.888381958008 -11.121166229248\\
131.90087890625 -11.1213636398315\\
131.913375854492 -11.121563911438\\
131.925872802734 -11.1217641830444\\
131.938369750977 -11.1219606399536\\
131.950881958008 -11.1221475601196\\
131.96337890625 -11.1223230361938\\
131.975875854492 -11.1224870681763\\
131.988372802734 -11.1226367950439\\
132.000869750977 -11.1227769851685\\
132.013381958008 -11.1229104995728\\
132.02587890625 -11.1230459213257\\
132.038375854492 -11.1231889724731\\
132.050872802734 -11.1233367919922\\
132.063369750977 -11.1234893798828\\
132.075881958008 -11.1236419677734\\
132.08837890625 -11.1237945556641\\
132.100875854492 -11.1239471435547\\
132.113372802734 -11.1240911483765\\
132.125869750977 -11.1242265701294\\
132.138381958008 -11.1243600845337\\
132.15087890625 -11.1244926452637\\
132.163375854492 -11.1246252059937\\
132.175872802734 -11.1247520446777\\
132.188369750977 -11.1248760223389\\
132.200881958008 -11.1249980926514\\
132.21337890625 -11.1251125335693\\
132.225875854492 -11.1252174377441\\
132.238372802734 -11.1253156661987\\
132.250869750977 -11.1254110336304\\
132.263381958008 -11.125506401062\\
132.27587890625 -11.1256017684937\\
132.288375854492 -11.1257028579712\\
132.300872802734 -11.125807762146\\
132.313369750977 -11.1259126663208\\
132.325881958008 -11.1260175704956\\
132.33837890625 -11.1261291503906\\
132.350875854492 -11.1262435913086\\
132.363372802734 -11.1263618469238\\
132.375869750977 -11.1264848709106\\
132.388381958008 -11.1266012191772\\
132.40087890625 -11.1267156600952\\
132.413375854492 -11.1268301010132\\
132.425872802734 -11.1269454956055\\
132.438369750977 -11.1270694732666\\
132.450881958008 -11.1272010803223\\
132.46337890625 -11.1273345947266\\
132.475875854492 -11.1274681091309\\
132.488372802734 -11.1276016235352\\
132.500869750977 -11.1277313232422\\
132.513381958008 -11.1278648376465\\
132.52587890625 -11.1279973983765\\
132.538375854492 -11.1281309127808\\
132.550872802734 -11.1282682418823\\
132.563369750977 -11.1284112930298\\
132.575881958008 -11.1285543441772\\
132.58837890625 -11.1286888122559\\
132.600875854492 -11.1288146972656\\
132.613372802734 -11.1289386749268\\
132.625869750977 -11.1290626525879\\
132.638381958008 -11.1291952133179\\
132.65087890625 -11.129319190979\\
132.663375854492 -11.1294431686401\\
132.675872802734 -11.1295623779297\\
132.688369750977 -11.1296768188477\\
132.700881958008 -11.1297998428345\\
132.71337890625 -11.1299381256104\\
132.725875854492 -11.1300811767578\\
132.738372802734 -11.1302165985107\\
132.750869750977 -11.1303491592407\\
132.763381958008 -11.1304740905762\\
132.774932861328 -11.1305894851685\\
132.78742980957 -11.1307249069214\\
132.799942016602 -11.1308679580688\\
132.812438964844 -11.131007194519\\
132.824935913086 -11.1311340332031\\
132.837432861328 -11.1312494277954\\
132.84992980957 -11.1313457489014\\
132.862442016602 -11.131477355957\\
132.874938964844 -11.1316080093384\\
132.887435913086 -11.1317319869995\\
132.899932861328 -11.1318559646606\\
132.91242980957 -11.1319808959961\\
132.924942016602 -11.1321182250977\\
132.937438964844 -11.1322612762451\\
132.949935913086 -11.1323986053467\\
132.962432861328 -11.132532119751\\
132.97492980957 -11.1326656341553\\
132.987442016602 -11.1327991485596\\
132.999938964844 -11.1329326629639\\
133.012435913086 -11.1330680847168\\
133.024932861328 -11.1332111358643\\
133.03742980957 -11.1333541870117\\
133.049942016602 -11.133508682251\\
133.062438964844 -11.1336679458618\\
133.074935913086 -11.1338396072388\\
133.087432861328 -11.1340055465698\\
133.09992980957 -11.1341667175293\\
133.112442016602 -11.1343193054199\\
133.124938964844 -11.1344814300537\\
133.137435913086 -11.1346426010132\\
133.149932861328 -11.134804725647\\
133.16242980957 -11.1349668502808\\
133.174942016602 -11.1351385116577\\
133.187438964844 -11.1353101730347\\
133.199935913086 -11.1354818344116\\
133.212432861328 -11.1356534957886\\
133.22492980957 -11.1358251571655\\
133.237442016602 -11.1359968185425\\
133.249938964844 -11.1361742019653\\
133.262435913086 -11.1363544464111\\
133.274932861328 -11.1365356445313\\
133.28742980957 -11.1367168426514\\
133.299942016602 -11.1368980407715\\
133.312438964844 -11.1370792388916\\
133.324935913086 -11.1372604370117\\
133.337432861328 -11.1374483108521\\
133.34992980957 -11.1376390457153\\
133.362442016602 -11.1378297805786\\
133.374938964844 -11.1380205154419\\
133.387435913086 -11.1382112503052\\
133.399932861328 -11.1384019851685\\
133.41242980957 -11.1385927200317\\
133.424942016602 -11.1387853622437\\
133.437438964844 -11.1389827728271\\
133.449935913086 -11.1391735076904\\
133.462432861328 -11.1393642425537\\
133.47492980957 -11.1395606994629\\
133.487442016602 -11.1397609710693\\
133.499938964844 -11.1399612426758\\
133.512435913086 -11.1401615142822\\
133.524932861328 -11.1403617858887\\
133.53742980957 -11.1405620574951\\
133.549942016602 -11.1407632827759\\
133.562438964844 -11.1409730911255\\
133.574935913086 -11.1411733627319\\
133.587432861328 -11.1413812637329\\
133.59992980957 -11.1415910720825\\
133.612442016602 -11.1418008804321\\
133.624938964844 -11.1420068740845\\
133.637435913086 -11.1422166824341\\
133.649932861328 -11.1424264907837\\
133.66242980957 -11.1426362991333\\
133.674942016602 -11.1428461074829\\
133.687438964844 -11.1430559158325\\
133.699935913086 -11.1432657241821\\
133.712432861328 -11.143482208252\\
133.72492980957 -11.1437015533447\\
133.737442016602 -11.1439208984375\\
133.749938964844 -11.1441402435303\\
133.762435913086 -11.1443653106689\\
133.774932861328 -11.1445932388306\\
133.78742980957 -11.1448221206665\\
133.799942016602 -11.1450510025024\\
133.812438964844 -11.1452798843384\\
133.824935913086 -11.1455135345459\\
133.837432861328 -11.145751953125\\
133.84992980957 -11.1459827423096\\
133.862442016602 -11.1462116241455\\
133.874938964844 -11.1464405059814\\
133.887435913086 -11.1466693878174\\
133.899932861328 -11.1468963623047\\
133.91242980957 -11.1471328735352\\
133.924942016602 -11.1473617553711\\
133.937438964844 -11.147590637207\\
133.949935913086 -11.147819519043\\
133.962432861328 -11.1480464935303\\
133.97492980957 -11.148265838623\\
133.987442016602 -11.1484851837158\\
133.999938964844 -11.1486978530884\\
134.012435913086 -11.148907661438\\
134.024932861328 -11.1491165161133\\
134.03742980957 -11.1493234634399\\
134.049942016602 -11.1495237350464\\
134.062438964844 -11.1497240066528\\
134.074935913086 -11.1499242782593\\
134.087432861328 -11.1501245498657\\
134.09992980957 -11.1503238677979\\
134.112442016602 -11.1505165100098\\
134.124938964844 -11.150707244873\\
134.137435913086 -11.1508979797363\\
134.149932861328 -11.1510887145996\\
134.16242980957 -11.1512784957886\\
134.174942016602 -11.1514692306519\\
134.187438964844 -11.1516542434692\\
134.199935913086 -11.1518354415894\\
134.212432861328 -11.1520166397095\\
134.22492980957 -11.1521978378296\\
134.237442016602 -11.1523790359497\\
134.249938964844 -11.1525602340698\\
134.262435913086 -11.1527414321899\\
134.274932861328 -11.1529226303101\\
134.28742980957 -11.1531066894531\\
134.299942016602 -11.1532974243164\\
134.312438964844 -11.1534881591797\\
134.324935913086 -11.153678894043\\
134.337432861328 -11.1538743972778\\
134.34992980957 -11.1540746688843\\
134.362442016602 -11.1542749404907\\
134.374938964844 -11.1544752120972\\
134.387435913086 -11.1546802520752\\
134.399932861328 -11.1548900604248\\
134.41242980957 -11.1550998687744\\
134.424942016602 -11.155309677124\\
134.437438964844 -11.1555223464966\\
134.449935913086 -11.155740737915\\
134.462432861328 -11.1559600830078\\
134.47492980957 -11.1561794281006\\
134.487442016602 -11.1563987731934\\
134.499938964844 -11.1566181182861\\
134.512435913086 -11.1568422317505\\
134.524932861328 -11.1570701599121\\
134.53742980957 -11.157299041748\\
134.549942016602 -11.157527923584\\
134.562438964844 -11.1577568054199\\
134.574935913086 -11.1579856872559\\
134.587432861328 -11.1582155227661\\
134.59992980957 -11.1584539413452\\
134.612442016602 -11.1586942672729\\
134.624938964844 -11.1589412689209\\
134.637435913086 -11.1591892242432\\
134.649932861328 -11.1594438552856\\
134.66242980957 -11.159704208374\\
134.674942016602 -11.1599645614624\\
134.687438964844 -11.160231590271\\
134.699935913086 -11.1605072021484\\
134.712432861328 -11.16077709198\\
134.72492980957 -11.1610336303711\\
134.737442016602 -11.1612815856934\\
134.749938964844 -11.1615362167358\\
134.762435913086 -11.1617889404297\\
134.774932861328 -11.1620578765869\\
134.78742980957 -11.1623296737671\\
134.799942016602 -11.1626091003418\\
134.812438964844 -11.1628751754761\\
134.824935913086 -11.1631326675415\\
134.837432861328 -11.1633834838867\\
134.84992980957 -11.163631439209\\
134.862442016602 -11.1638793945313\\
134.874938964844 -11.1641273498535\\
134.887435913086 -11.1643753051758\\
134.899932861328 -11.164623260498\\
134.91242980957 -11.1648712158203\\
134.924942016602 -11.1651191711426\\
134.937438964844 -11.1653671264648\\
134.949935913086 -11.1656150817871\\
134.962432861328 -11.1658697128296\\
134.97492980957 -11.166127204895\\
134.987442016602 -11.166392326355\\
134.999938964844 -11.1666593551636\\
135.012435913086 -11.166934967041\\
135.024932861328 -11.1672163009644\\
135.03742980957 -11.1675033569336\\
135.049942016602 -11.1677989959717\\
135.062438964844 -11.1680994033813\\
135.074935913086 -11.1684045791626\\
135.087432861328 -11.1687116622925\\
135.09992980957 -11.1690263748169\\
135.112442016602 -11.1693410873413\\
135.124938964844 -11.1696586608887\\
135.137435913086 -11.1699819564819\\
135.149932861328 -11.1703062057495\\
135.16242980957 -11.17063331604\\
135.174942016602 -11.1709671020508\\
135.187438964844 -11.1712999343872\\
135.199935913086 -11.1716337203979\\
135.212432861328 -11.1719751358032\\
135.22492980957 -11.1723184585571\\
135.237442016602 -11.172661781311\\
135.249938964844 -11.1730117797852\\
135.262435913086 -11.1733636856079\\
135.274932861328 -11.173716545105\\
135.28742980957 -11.1740741729736\\
135.299942016602 -11.1744365692139\\
135.312438964844 -11.1748018264771\\
135.324935913086 -11.1751728057861\\
135.337432861328 -11.1755447387695\\
135.34992980957 -11.1759185791016\\
135.362442016602 -11.1762990951538\\
135.374938964844 -11.1766557693481\\
135.387435913086 -11.1770315170288\\
135.399932861328 -11.1774301528931\\
135.41242980957 -11.1778402328491\\
135.424942016602 -11.1782531738281\\
135.437438964844 -11.1786613464355\\
135.449935913086 -11.1790580749512\\
135.462432861328 -11.1794452667236\\
135.47492980957 -11.1798248291016\\
135.487442016602 -11.180196762085\\
135.499938964844 -11.180567741394\\
135.512435913086 -11.1809387207031\\
135.524932861328 -11.1813097000122\\
135.53742980957 -11.1816806793213\\
135.549942016602 -11.1820526123047\\
135.562438964844 -11.1824188232422\\
135.574935913086 -11.1827774047852\\
135.587432861328 -11.1831226348877\\
135.59992980957 -11.1834783554077\\
135.612442016602 -11.1838293075562\\
135.624938964844 -11.1841335296631\\
135.637435913086 -11.1844139099121\\
135.649932861328 -11.1846837997437\\
135.66242980957 -11.1849508285522\\
135.674942016602 -11.1852149963379\\
135.687438964844 -11.1854867935181\\
135.699935913086 -11.1857719421387\\
135.712432861328 -11.1860656738281\\
135.72492980957 -11.1863603591919\\
135.737442016602 -11.1866464614868\\
135.749938964844 -11.1869316101074\\
135.762435913086 -11.1872081756592\\
135.774932861328 -11.1874837875366\\
135.78742980957 -11.1877536773682\\
135.799942016602 -11.1880207061768\\
135.812438964844 -11.1882877349854\\
135.824935913086 -11.188549041748\\
135.837432861328 -11.1888065338135\\
135.84992980957 -11.1890630722046\\
135.862442016602 -11.1893148422241\\
135.874938964844 -11.1895589828491\\
135.887435913086 -11.1897964477539\\
135.899932861328 -11.1900339126587\\
135.91242980957 -11.1902713775635\\
135.924942016602 -11.1905002593994\\
135.937438964844 -11.1907291412354\\
135.949935913086 -11.1909580230713\\
135.962432861328 -11.1911859512329\\
135.97492980957 -11.1914129257202\\
135.987442016602 -11.191632270813\\
135.999938964844 -11.1918516159058\\
136.012435913086 -11.1920709609985\\
136.024932861328 -11.192289352417\\
136.03742980957 -11.1925010681152\\
136.049942016602 -11.1927108764648\\
136.062438964844 -11.1929206848145\\
136.074935913086 -11.1931276321411\\
136.087432861328 -11.1933279037476\\
136.09992980957 -11.1935195922852\\
136.112442016602 -11.1937017440796\\
136.124938964844 -11.1938724517822\\
136.137435913086 -11.1940317153931\\
136.149932861328 -11.1941804885864\\
136.16242980957 -11.1943225860596\\
136.174942016602 -11.1944618225098\\
136.187438964844 -11.1946039199829\\
136.199935913086 -11.1947469711304\\
136.212432861328 -11.1948900222778\\
136.22492980957 -11.1950330734253\\
136.237442016602 -11.1951847076416\\
136.249938964844 -11.1953372955322\\
136.262435913086 -11.1954898834229\\
136.274932861328 -11.1956462860107\\
136.28742980957 -11.1958084106445\\
136.299942016602 -11.1959705352783\\
136.312438964844 -11.196141242981\\
136.324935913086 -11.1963129043579\\
136.337432861328 -11.1964912414551\\
136.34992980957 -11.1966724395752\\
136.362442016602 -11.1968536376953\\
136.374938964844 -11.1970348358154\\
136.387435913086 -11.1972150802612\\
136.399932861328 -11.1973962783813\\
136.41242980957 -11.1975774765015\\
136.424942016602 -11.1977596282959\\
136.437438964844 -11.1979494094849\\
136.449935913086 -11.1981401443481\\
136.462432861328 -11.1983308792114\\
136.47492980957 -11.1985216140747\\
136.487442016602 -11.1987161636353\\
136.499938964844 -11.1989154815674\\
136.512435913086 -11.1991147994995\\
136.524932861328 -11.1993141174316\\
136.53742980957 -11.1995134353638\\
136.549942016602 -11.1997137069702\\
136.562438964844 -11.1999139785767\\
136.574935913086 -11.2001142501831\\
136.587432861328 -11.2003221511841\\
136.59992980957 -11.2005319595337\\
136.612442016602 -11.2007436752319\\
136.624938964844 -11.2009620666504\\
136.637435913086 -11.2011814117432\\
136.649932861328 -11.2014045715332\\
136.66242980957 -11.2016334533691\\
136.674942016602 -11.2018623352051\\
136.687438964844 -11.202094078064\\
136.699935913086 -11.2023305892944\\
136.712432861328 -11.2025690078735\\
136.72492980957 -11.2028341293335\\
136.737442016602 -11.2031393051147\\
136.749938964844 -11.2034912109375\\
136.762435913086 -11.2038908004761\\
136.774932861328 -11.2043266296387\\
136.78742980957 -11.2047882080078\\
136.799942016602 -11.2052774429321\\
136.812438964844 -11.2057933807373\\
136.824935913086 -11.2063341140747\\
136.837432861328 -11.2068910598755\\
136.84992980957 -11.2074594497681\\
136.862442016602 -11.2080383300781\\
136.874938964844 -11.2086296081543\\
136.887435913086 -11.2092351913452\\
136.899932861328 -11.2098541259766\\
136.91242980957 -11.210485458374\\
136.924942016602 -11.211124420166\\
136.937438964844 -11.2117595672607\\
136.949935913086 -11.2123794555664\\
136.962432861328 -11.2129983901978\\
136.97492980957 -11.2136287689209\\
136.987442016602 -11.2142715454102\\
136.999938964844 -11.2149267196655\\
137.012435913086 -11.2155923843384\\
137.024932861328 -11.2162714004517\\
137.03742980957 -11.2169599533081\\
137.049942016602 -11.2176599502563\\
137.062438964844 -11.2183685302734\\
137.074935913086 -11.219087600708\\
137.087432861328 -11.2198152542114\\
137.09992980957 -11.220552444458\\
137.112442016602 -11.2213010787964\\
137.124938964844 -11.2220592498779\\
137.137435913086 -11.2228269577026\\
137.149932861328 -11.2236051559448\\
137.16242980957 -11.2243938446045\\
137.174942016602 -11.2251920700073\\
137.187438964844 -11.2259969711304\\
137.199935913086 -11.2268075942993\\
137.212432861328 -11.2276268005371\\
137.22492980957 -11.2284526824951\\
137.237442016602 -11.2292823791504\\
137.249938964844 -11.2301197052002\\
137.262435913086 -11.2309579849243\\
137.274932861328 -11.231803894043\\
137.28742980957 -11.2326517105103\\
137.299942016602 -11.2335042953491\\
137.312438964844 -11.2343616485596\\
137.324935913086 -11.2352237701416\\
137.337432861328 -11.2360906600952\\
137.34992980957 -11.2369594573975\\
137.362442016602 -11.2378368377686\\
137.374938964844 -11.2387142181396\\
137.387435913086 -11.2396001815796\\
137.399932861328 -11.2404851913452\\
137.41242980957 -11.2413711547852\\
137.424942016602 -11.242262840271\\
137.437438964844 -11.2431545257568\\
137.449935913086 -11.2440490722656\\
137.462432861328 -11.2449541091919\\
137.47492980957 -11.2458505630493\\
137.487442016602 -11.2467498779297\\
137.499938964844 -11.2476472854614\\
137.512435913086 -11.2485475540161\\
137.524932861328 -11.2494487762451\\
137.53742980957 -11.2503471374512\\
137.549942016602 -11.2512464523315\\
137.562438964844 -11.2521419525146\\
137.574935913086 -11.2530374526978\\
137.587432861328 -11.2539329528809\\
137.59992980957 -11.254828453064\\
137.612442016602 -11.2557249069214\\
137.624938964844 -11.2566204071045\\
137.637435913086 -11.2575092315674\\
137.649932861328 -11.2583951950073\\
137.66242980957 -11.2592811584473\\
137.674942016602 -11.2601680755615\\
137.687438964844 -11.2610521316528\\
137.699935913086 -11.2619285583496\\
137.712432861328 -11.2628049850464\\
137.72492980957 -11.2636814117432\\
137.737442016602 -11.2645530700684\\
137.749938964844 -11.265419960022\\
137.762435913086 -11.2662858963013\\
137.774932861328 -11.2671432495117\\
137.78742980957 -11.2680006027222\\
137.799942016602 -11.2688550949097\\
137.812438964844 -11.269702911377\\
137.824935913086 -11.2705478668213\\
137.837432861328 -11.2713861465454\\
137.84992980957 -11.2722215652466\\
137.862442016602 -11.2730512619019\\
137.874938964844 -11.2738761901855\\
137.887435913086 -11.2746953964233\\
137.899932861328 -11.2755088806152\\
137.91242980957 -11.2763185501099\\
137.924942016602 -11.27712059021\\
137.937438964844 -11.2779197692871\\
137.949935913086 -11.2787103652954\\
137.962432861328 -11.2794961929321\\
137.97492980957 -11.2802772521973\\
137.987442016602 -11.2810497283936\\
137.999938964844 -11.2818183898926\\
138.012435913086 -11.2825803756714\\
138.024932861328 -11.2833337783813\\
138.03742980957 -11.2840824127197\\
138.049942016602 -11.2848253250122\\
138.062438964844 -11.2855587005615\\
138.074935913086 -11.2862854003906\\
138.087432861328 -11.2870054244995\\
138.09992980957 -11.2877178192139\\
138.112442016602 -11.2884225845337\\
138.124938964844 -11.2891178131104\\
138.137435913086 -11.2898044586182\\
138.149932861328 -11.2904834747314\\
138.16242980957 -11.2911558151245\\
138.174942016602 -11.291820526123\\
138.187438964844 -11.2924766540527\\
138.199935913086 -11.2931241989136\\
138.212432861328 -11.2937641143799\\
138.22492980957 -11.2943964004517\\
138.237442016602 -11.2950229644775\\
138.249938964844 -11.2956418991089\\
138.262435913086 -11.2962512969971\\
138.274932861328 -11.296854019165\\
138.28742980957 -11.2974500656128\\
138.299942016602 -11.298041343689\\
138.312438964844 -11.298623085022\\
138.324935913086 -11.2991991043091\\
138.337432861328 -11.299768447876\\
138.34992980957 -11.3003301620483\\
138.362442016602 -11.3008861541748\\
138.374938964844 -11.3014364242554\\
138.387435913086 -11.3019790649414\\
138.399932861328 -11.3025169372559\\
138.41242980957 -11.3030500411987\\
138.424942016602 -11.3035745620728\\
138.437438964844 -11.3040933609009\\
138.449935913086 -11.3046073913574\\
138.462432861328 -11.3051156997681\\
138.47492980957 -11.3056182861328\\
138.487442016602 -11.3061141967773\\
138.499938964844 -11.3066024780273\\
138.512435913086 -11.3070878982544\\
138.524932861328 -11.3075647354126\\
138.53742980957 -11.3080396652222\\
138.549942016602 -11.3085069656372\\
138.562438964844 -11.3089685440063\\
138.574935913086 -11.3094253540039\\
138.587432861328 -11.3098773956299\\
138.59992980957 -11.3103246688843\\
138.612442016602 -11.3107643127441\\
138.624938964844 -11.3112020492554\\
138.637435913086 -11.3116331100464\\
138.649932861328 -11.3120613098145\\
138.66242980957 -11.3124895095825\\
138.674942016602 -11.3129110336304\\
138.687438964844 -11.3133001327515\\
138.699935913086 -11.3136978149414\\
138.712432861328 -11.3141202926636\\
138.72492980957 -11.3145599365234\\
138.737442016602 -11.3150081634521\\
138.749938964844 -11.315447807312\\
138.762435913086 -11.315881729126\\
138.774932861328 -11.316309928894\\
138.78742980957 -11.3167285919189\\
138.799942016602 -11.3171482086182\\
138.812438964844 -11.3175582885742\\
138.824935913086 -11.317967414856\\
138.837432861328 -11.3183746337891\\
138.84992980957 -11.3187742233276\\
138.862442016602 -11.3191747665405\\
138.874938964844 -11.3195657730103\\
138.887435913086 -11.3199501037598\\
138.899932861328 -11.3203296661377\\
138.91242980957 -11.3207015991211\\
138.924942016602 -11.3210735321045\\
138.937438964844 -11.3214416503906\\
138.949935913086 -11.3218040466309\\
138.962432861328 -11.3221654891968\\
138.97492980957 -11.3225193023682\\
138.987442016602 -11.3228721618652\\
138.999938964844 -11.3232154846191\\
139.012435913086 -11.3235559463501\\
139.024932861328 -11.3238897323608\\
139.03742980957 -11.3242225646973\\
139.049942016602 -11.3245468139648\\
139.062438964844 -11.3248701095581\\
139.074935913086 -11.3251848220825\\
139.087432861328 -11.3254976272583\\
139.09992980957 -11.3258018493652\\
139.112442016602 -11.3261013031006\\
139.124938964844 -11.3263902664185\\
139.137435913086 -11.3266592025757\\
139.149932861328 -11.3269071578979\\
139.16242980957 -11.3271245956421\\
139.174942016602 -11.3273048400879\\
139.187438964844 -11.3274602890015\\
139.199935913086 -11.327600479126\\
139.212432861328 -11.3277339935303\\
139.22492980957 -11.3278675079346\\
139.237442016602 -11.3280096054077\\
139.249938964844 -11.3281698226929\\
139.262435913086 -11.3283529281616\\
139.274932861328 -11.3285541534424\\
139.28742980957 -11.3287696838379\\
139.299942016602 -11.328989982605\\
139.312438964844 -11.3292179107666\\
139.324935913086 -11.3294467926025\\
139.337432861328 -11.3296756744385\\
139.34992980957 -11.3299036026001\\
139.362442016602 -11.330132484436\\
139.374938964844 -11.3303604125977\\
139.387435913086 -11.3305883407593\\
139.399932861328 -11.3308172225952\\
139.41242980957 -11.3310461044312\\
139.424942016602 -11.3312749862671\\
139.437438964844 -11.331503868103\\
139.449935913086 -11.331732749939\\
139.462432861328 -11.331958770752\\
139.47492980957 -11.3321771621704\\
139.487442016602 -11.3323907852173\\
139.499938964844 -11.3326005935669\\
139.512435913086 -11.3328018188477\\
139.524932861328 -11.3330011367798\\
139.53742980957 -11.3331956863403\\
139.549942016602 -11.3333864212036\\
139.562438964844 -11.3335771560669\\
139.574935913086 -11.3337678909302\\
139.587432861328 -11.3339605331421\\
139.59992980957 -11.3341608047485\\
139.612442016602 -11.3343629837036\\
139.624938964844 -11.3345718383789\\
139.637435913086 -11.3347826004028\\
139.649932861328 -11.3350009918213\\
139.66242980957 -11.3352203369141\\
139.674942016602 -11.3354396820068\\
139.687438964844 -11.3356523513794\\
139.699935913086 -11.3358402252197\\
139.712432861328 -11.335973739624\\
139.72492980957 -11.3360719680786\\
139.737442016602 -11.3361806869507\\
139.749938964844 -11.3363075256348\\
139.762435913086 -11.3364477157593\\
139.774932861328 -11.3366022109985\\
139.78742980957 -11.3367681503296\\
139.799942016602 -11.3369398117065\\
139.812438964844 -11.3371114730835\\
139.824935913086 -11.3372831344604\\
139.837432861328 -11.3374509811401\\
139.84992980957 -11.3376131057739\\
139.862442016602 -11.3377752304077\\
139.874938964844 -11.3379373550415\\
139.887435913086 -11.3381042480469\\
139.899932861328 -11.3382759094238\\
139.91242980957 -11.3384475708008\\
139.924942016602 -11.3386192321777\\
139.937438964844 -11.3387985229492\\
139.949935913086 -11.3389749526978\\
139.962432861328 -11.3391561508179\\
139.97492980957 -11.339337348938\\
139.987442016602 -11.3395185470581\\
139.999938964844 -11.3396997451782\\
140.012435913086 -11.3398809432983\\
140.024932861328 -11.3400621414185\\
140.03742980957 -11.3402433395386\\
140.049942016602 -11.3404264450073\\
140.062438964844 -11.3406076431274\\
140.074935913086 -11.3407917022705\\
140.087432861328 -11.3409824371338\\
140.09992980957 -11.3411750793457\\
140.112442016602 -11.3413753509521\\
140.124938964844 -11.3415756225586\\
140.137435913086 -11.341775894165\\
140.149932861328 -11.3419809341431\\
140.16242980957 -11.3421907424927\\
140.174942016602 -11.3424062728882\\
140.187438964844 -11.3426265716553\\
140.199935913086 -11.3428554534912\\
140.212432861328 -11.3430910110474\\
140.22492980957 -11.3433351516724\\
140.237442016602 -11.3435888290405\\
140.249938964844 -11.3438529968262\\
140.262435913086 -11.3441286087036\\
140.274932861328 -11.3444175720215\\
140.28742980957 -11.3447179794312\\
140.299942016602 -11.3450298309326\\
140.312438964844 -11.3453512191772\\
140.324935913086 -11.3456830978394\\
140.337432861328 -11.3460264205933\\
140.34992980957 -11.346378326416\\
140.362442016602 -11.3467388153076\\
140.374938964844 -11.3471117019653\\
140.387435913086 -11.3475046157837\\
140.399932861328 -11.3479337692261\\
140.41242980957 -11.3484153747559\\
140.424942016602 -11.3489675521851\\
140.437438964844 -11.3496036529541\\
140.449935913086 -11.3503389358521\\
140.462432861328 -11.3511724472046\\
140.47492980957 -11.3521099090576\\
140.487442016602 -11.3531455993652\\
140.499938964844 -11.3542766571045\\
140.512435913086 -11.3554954528809\\
140.524932861328 -11.3567876815796\\
140.53742980957 -11.3581371307373\\
140.549942016602 -11.3595170974731\\
140.562438964844 -11.3608131408691\\
140.574935913086 -11.3618631362915\\
140.587432861328 -11.362696647644\\
140.59992980957 -11.3634624481201\\
140.612442016602 -11.3642530441284\\
140.624938964844 -11.3650894165039\\
140.637435913086 -11.3659639358521\\
140.649932861328 -11.366868019104\\
140.66242980957 -11.3677940368652\\
140.674942016602 -11.3687572479248\\
140.687438964844 -11.3697662353516\\
140.699935913086 -11.3708209991455\\
140.712432861328 -11.371919631958\\
140.72492980957 -11.3730602264404\\
140.737442016602 -11.3742389678955\\
140.749938964844 -11.3754510879517\\
140.762435913086 -11.3766899108887\\
140.774932861328 -11.3779563903809\\
140.78742980957 -11.3792428970337\\
140.799942016602 -11.3805465698242\\
140.812438964844 -11.3818588256836\\
140.824935913086 -11.3831806182861\\
140.837432861328 -11.3845071792603\\
140.84992980957 -11.3858413696289\\
140.862442016602 -11.3871812820435\\
140.874938964844 -11.388524055481\\
140.887435913086 -11.3898773193359\\
140.899932861328 -11.3912630081177\\
140.91242980957 -11.3926753997803\\
140.924942016602 -11.3941040039063\\
140.937438964844 -11.3955450057983\\
140.949935913086 -11.3970108032227\\
140.962432861328 -11.3984813690186\\
140.97492980957 -11.3999223709106\\
140.987442016602 -11.4013195037842\\
140.999938964844 -11.4026670455933\\
141.012435913086 -11.4039812088013\\
141.024932861328 -11.4052734375\\
141.03742980957 -11.406548500061\\
141.049942016602 -11.4078168869019\\
141.062438964844 -11.4090747833252\\
141.074935913086 -11.4103412628174\\
141.087432861328 -11.4116096496582\\
141.09992980957 -11.4128866195679\\
141.112442016602 -11.4141693115234\\
141.124938964844 -11.415454864502\\
141.137435913086 -11.4167423248291\\
141.149932861328 -11.4180374145508\\
141.16242980957 -11.4193325042725\\
141.174942016602 -11.4206295013428\\
141.187438964844 -11.4219341278076\\
141.199935913086 -11.4232387542725\\
141.212432861328 -11.4245433807373\\
141.22492980957 -11.4258480072021\\
141.237442016602 -11.4271488189697\\
141.249938964844 -11.4284448623657\\
141.262435913086 -11.4297380447388\\
141.274932861328 -11.4310235977173\\
141.28742980957 -11.4323034286499\\
141.299942016602 -11.433575630188\\
141.312438964844 -11.43483543396\\
141.324935913086 -11.4360818862915\\
141.337432861328 -11.4373111724854\\
141.34992980957 -11.4385080337524\\
141.362442016602 -11.4396657943726\\
141.374938964844 -11.4407873153687\\
141.387435913086 -11.4418916702271\\
141.399932861328 -11.4429845809937\\
141.41242980957 -11.4440689086914\\
141.424942016602 -11.4451465606689\\
141.437438964844 -11.4462194442749\\
141.449935913086 -11.447286605835\\
141.462432861328 -11.448353767395\\
141.47492980957 -11.4494209289551\\
141.487442016602 -11.4504890441895\\
141.499938964844 -11.4515562057495\\
141.512435913086 -11.4526233673096\\
141.524932861328 -11.4536905288696\\
141.53742980957 -11.4547576904297\\
141.549942016602 -11.4558258056641\\
141.562438964844 -11.4568929672241\\
141.574935913086 -11.4579591751099\\
141.587432861328 -11.4590253829956\\
141.59992980957 -11.4600868225098\\
141.612442016602 -11.4611520767212\\
141.624938964844 -11.4622144699097\\
141.637435913086 -11.4632720947266\\
141.649932861328 -11.4643201828003\\
141.66242980957 -11.4653387069702\\
141.674942016602 -11.4662818908691\\
141.687438964844 -11.4670610427856\\
141.699935913086 -11.4676990509033\\
141.712432861328 -11.468204498291\\
141.72492980957 -11.4685754776001\\
141.737442016602 -11.4688768386841\\
141.749938964844 -11.4691486358643\\
141.762435913086 -11.4694147109985\\
141.774932861328 -11.4696931838989\\
141.78742980957 -11.4699974060059\\
141.799942016602 -11.4703302383423\\
141.812438964844 -11.4706916809082\\
141.824935913086 -11.4710788726807\\
141.837432861328 -11.471489906311\\
141.84992980957 -11.4719257354736\\
141.862442016602 -11.4723844528198\\
141.874938964844 -11.4728660583496\\
141.887435913086 -11.4733800888062\\
141.899932861328 -11.4739360809326\\
141.91242980957 -11.4745225906372\\
141.924942016602 -11.4751224517822\\
141.937438964844 -11.4757308959961\\
141.949935913086 -11.4763412475586\\
141.962432861328 -11.4769601821899\\
141.97492980957 -11.4775791168213\\
141.987442016602 -11.478199005127\\
141.999938964844 -11.4788274765015\\
142.012435913086 -11.4794645309448\\
142.024932861328 -11.4801025390625\\
142.03742980957 -11.4807386398315\\
142.049942016602 -11.4813575744629\\
142.062438964844 -11.4819469451904\\
142.074935913086 -11.4825229644775\\
142.087432861328 -11.4830942153931\\
142.09992980957 -11.4836664199829\\
142.112442016602 -11.4842386245728\\
142.124938964844 -11.4848098754883\\
142.137435913086 -11.4853811264038\\
142.149932861328 -11.4859523773193\\
142.16242980957 -11.4865236282349\\
142.174942016602 -11.4870901107788\\
142.187438964844 -11.4876518249512\\
142.199935913086 -11.4882135391235\\
142.212432861328 -11.4887752532959\\
142.22492980957 -11.4893369674683\\
142.237442016602 -11.4898996353149\\
142.249938964844 -11.490460395813\\
142.262435913086 -11.4910125732422\\
142.274932861328 -11.4915533065796\\
142.28742980957 -11.4920778274536\\
142.299942016602 -11.4925928115845\\
142.312438964844 -11.493106842041\\
142.324935913086 -11.4936265945435\\
142.337432861328 -11.4941501617432\\
142.34992980957 -11.4946746826172\\
142.362442016602 -11.4952087402344\\
142.374938964844 -11.4957418441772\\
142.387435913086 -11.4962759017944\\
142.399932861328 -11.4968175888062\\
142.41242980957 -11.497350692749\\
142.424942016602 -11.4978847503662\\
142.437438964844 -11.4984178543091\\
142.449935913086 -11.4989538192749\\
142.462432861328 -11.4994964599609\\
142.47492980957 -11.500039100647\\
142.487442016602 -11.5005826950073\\
142.499938964844 -11.5011262893677\\
142.512435913086 -11.501669883728\\
142.524932861328 -11.5022134780884\\
142.53742980957 -11.5027561187744\\
142.549942016602 -11.5032968521118\\
142.562438964844 -11.5038261413574\\
142.574935913086 -11.5043497085571\\
142.587432861328 -11.5048742294312\\
142.59992980957 -11.5053987503052\\
142.612442016602 -11.5059299468994\\
142.624938964844 -11.5064630508423\\
142.637435913086 -11.5069961547852\\
142.649932861328 -11.507529258728\\
142.66242980957 -11.5080623626709\\
142.674942016602 -11.5085935592651\\
142.687438964844 -11.5091180801392\\
142.699935913086 -11.5096387863159\\
142.712432861328 -11.5101490020752\\
142.72492980957 -11.510648727417\\
142.737442016602 -11.5111446380615\\
142.749938964844 -11.5116395950317\\
142.762435913086 -11.512134552002\\
142.774932861328 -11.5126295089722\\
142.78742980957 -11.5131244659424\\
142.799942016602 -11.5136184692383\\
142.812438964844 -11.514102935791\\
142.824935913086 -11.5145797729492\\
142.837432861328 -11.5150556564331\\
142.84992980957 -11.515531539917\\
142.862442016602 -11.5160083770752\\
142.874938964844 -11.5164766311646\\
142.887435913086 -11.5169429779053\\
142.899932861328 -11.517409324646\\
142.91242980957 -11.5178756713867\\
142.924942016602 -11.5183391571045\\
142.937438964844 -11.5187969207764\\
142.949935913086 -11.5192546844482\\
142.962432861328 -11.5197114944458\\
142.97492980957 -11.5201683044434\\
142.987442016602 -11.5206260681152\\
142.999938964844 -11.5210752487183\\
143.012435913086 -11.521523475647\\
143.024932861328 -11.5219717025757\\
143.03742980957 -11.5224199295044\\
143.049942016602 -11.5228681564331\\
143.062438964844 -11.5233163833618\\
143.074935913086 -11.5237636566162\\
143.087432861328 -11.5242109298706\\
143.09992980957 -11.5246562957764\\
143.112442016602 -11.5251045227051\\
143.124938964844 -11.5255460739136\\
143.137435913086 -11.5259847640991\\
143.149932861328 -11.5264234542847\\
143.16242980957 -11.5268611907959\\
143.174942016602 -11.5272998809814\\
143.187438964844 -11.5277309417725\\
143.199935913086 -11.5281600952148\\
143.212432861328 -11.5285882949829\\
143.22492980957 -11.529013633728\\
143.237442016602 -11.5294332504272\\
143.249938964844 -11.5298519134521\\
143.262435913086 -11.5302667617798\\
143.274932861328 -11.5306758880615\\
143.28742980957 -11.5310850143433\\
143.299942016602 -11.5314855575562\\
143.312438964844 -11.5318851470947\\
143.324935913086 -11.5322799682617\\
143.337432861328 -11.5326700210571\\
143.34992980957 -11.5330600738525\\
143.362442016602 -11.5334415435791\\
143.374938964844 -11.5338230133057\\
143.387435913086 -11.5342035293579\\
143.399932861328 -11.5345821380615\\
143.41242980957 -11.5349540710449\\
143.424942016602 -11.5353260040283\\
143.437438964844 -11.5356969833374\\
143.449935913086 -11.5360641479492\\
143.462432861328 -11.5364265441895\\
143.47492980957 -11.5367879867554\\
143.487442016602 -11.5371503829956\\
143.499938964844 -11.5375118255615\\
143.512435913086 -11.5378732681274\\
143.524932861328 -11.5382261276245\\
143.53742980957 -11.5385789871216\\
143.549942016602 -11.5389318466187\\
143.562438964844 -11.5392847061157\\
143.574935913086 -11.5396366119385\\
143.587432861328 -11.5399885177612\\
143.59992980957 -11.540340423584\\
143.612442016602 -11.5406837463379\\
143.624938964844 -11.5410270690918\\
143.637435913086 -11.5413703918457\\
143.649932861328 -11.5417137145996\\
143.66242980957 -11.5420570373535\\
143.674942016602 -11.5424003601074\\
143.687438964844 -11.5427436828613\\
143.699935913086 -11.5430870056152\\
143.712432861328 -11.5434293746948\\
143.72492980957 -11.5437717437744\\
143.737442016602 -11.5441150665283\\
143.749938964844 -11.5444574356079\\
143.762435913086 -11.5447912216187\\
143.774932861328 -11.5451250076294\\
143.78742980957 -11.5454587936401\\
143.799942016602 -11.5457925796509\\
143.812438964844 -11.5461254119873\\
143.824935913086 -11.5464496612549\\
143.837432861328 -11.5467739105225\\
143.84992980957 -11.5470972061157\\
143.862442016602 -11.5474214553833\\
143.874938964844 -11.5477361679077\\
143.887435913086 -11.5480508804321\\
143.899932861328 -11.5483646392822\\
143.91242980957 -11.5486764907837\\
143.924942016602 -11.5489816665649\\
143.937438964844 -11.5492868423462\\
143.949935913086 -11.5495910644531\\
143.962432861328 -11.5498914718628\\
143.97492980957 -11.5501871109009\\
143.987442016602 -11.550482749939\\
143.999938964844 -11.5507774353027\\
144.012435913086 -11.5510683059692\\
144.024932861328 -11.5513544082642\\
144.03742980957 -11.5516405105591\\
144.049942016602 -11.551926612854\\
144.062438964844 -11.5522127151489\\
144.074935913086 -11.5524978637695\\
144.087432861328 -11.5527830123901\\
144.09992980957 -11.5530681610107\\
144.112442016602 -11.5533542633057\\
144.124938964844 -11.5536327362061\\
144.137435913086 -11.5539093017578\\
144.149932861328 -11.5541858673096\\
144.16242980957 -11.5544691085815\\
144.174942016602 -11.5547456741333\\
144.187438964844 -11.5550298690796\\
144.199935913086 -11.5553150177002\\
144.212432861328 -11.5556001663208\\
144.22492980957 -11.5558853149414\\
144.237442016602 -11.5561714172363\\
144.249938964844 -11.5564565658569\\
144.262435913086 -11.5567417144775\\
144.274932861328 -11.5570268630981\\
144.28742980957 -11.5573120117188\\
144.299942016602 -11.5575981140137\\
144.312438964844 -11.5578832626343\\
144.324935913086 -11.5581684112549\\
144.337432861328 -11.5584535598755\\
144.34992980957 -11.5587387084961\\
144.362442016602 -11.559024810791\\
144.374938964844 -11.5593013763428\\
144.387435913086 -11.5595779418945\\
144.399932861328 -11.5598545074463\\
144.41242980957 -11.5601301193237\\
144.424942016602 -11.5604028701782\\
144.437438964844 -11.5606698989868\\
144.449935913086 -11.5609369277954\\
144.462432861328 -11.5612030029297\\
144.47492980957 -11.5614604949951\\
144.487442016602 -11.5617179870605\\
144.499938964844 -11.5619745254517\\
144.512435913086 -11.5622253417969\\
144.524932861328 -11.5624732971191\\
144.53742980957 -11.5627202987671\\
144.549942016602 -11.5629682540894\\
144.562438964844 -11.5632076263428\\
144.574935913086 -11.5634460449219\\
144.587432861328 -11.563684463501\\
144.59992980957 -11.5639219284058\\
144.612442016602 -11.5641603469849\\
144.624938964844 -11.5643920898438\\
144.637435913086 -11.5646209716797\\
144.649932861328 -11.5648498535156\\
144.66242980957 -11.5650787353516\\
144.674942016602 -11.5653076171875\\
144.687438964844 -11.5655364990234\\
144.699935913086 -11.5657653808594\\
144.712432861328 -11.565993309021\\
144.72492980957 -11.5662212371826\\
144.737442016602 -11.5664501190186\\
144.749938964844 -11.5666761398315\\
144.762435913086 -11.5668954849243\\
144.774932861328 -11.5671148300171\\
144.78742980957 -11.5673341751099\\
144.799942016602 -11.5675535202026\\
144.812438964844 -11.5677728652954\\
144.824935913086 -11.5679922103882\\
144.837432861328 -11.568211555481\\
144.84992980957 -11.5684299468994\\
144.862442016602 -11.5686492919922\\
144.874938964844 -11.5688610076904\\
144.887435913086 -11.56907081604\\
144.899932861328 -11.5692806243896\\
144.91242980957 -11.5694904327393\\
144.924942016602 -11.5697002410889\\
144.937438964844 -11.5699071884155\\
144.949935913086 -11.570107460022\\
144.962432861328 -11.5703077316284\\
144.97492980957 -11.5705080032349\\
144.987442016602 -11.5707082748413\\
144.999938964844 -11.5709075927734\\
145.012435913086 -11.571102142334\\
145.024932861328 -11.5712928771973\\
145.03742980957 -11.5714836120605\\
145.049942016602 -11.5716743469238\\
145.062438964844 -11.5718650817871\\
145.074935913086 -11.5720558166504\\
145.087432861328 -11.5722455978394\\
145.09992980957 -11.5724306106567\\
145.112442016602 -11.5726118087769\\
145.124938964844 -11.572793006897\\
145.137435913086 -11.5729742050171\\
145.149932861328 -11.5731554031372\\
145.16242980957 -11.5733366012573\\
145.174942016602 -11.5735177993774\\
145.187438964844 -11.5736970901489\\
145.199935913086 -11.5738687515259\\
145.212432861328 -11.5740404129028\\
145.22492980957 -11.5742120742798\\
145.237442016602 -11.5743837356567\\
145.249938964844 -11.5745553970337\\
145.262435913086 -11.5747270584106\\
145.274932861328 -11.5748977661133\\
145.28742980957 -11.5750665664673\\
145.299942016602 -11.5752286911011\\
145.312438964844 -11.5753908157349\\
145.324935913086 -11.5755529403687\\
145.337432861328 -11.5757150650024\\
145.34992980957 -11.5758771896362\\
145.362442016602 -11.57603931427\\
145.374938964844 -11.5761976242065\\
145.387435913086 -11.5763502120972\\
145.399932861328 -11.5765027999878\\
145.41242980957 -11.5766553878784\\
145.424942016602 -11.576807975769\\
145.437438964844 -11.5769605636597\\
145.449935913086 -11.577112197876\\
145.462432861328 -11.5772590637207\\
145.47492980957 -11.5774021148682\\
145.487442016602 -11.5775451660156\\
145.499938964844 -11.5776882171631\\
145.512435913086 -11.5778312683105\\
145.524932861328 -11.577974319458\\
145.53742980957 -11.5781164169312\\
145.549942016602 -11.5782585144043\\
145.562438964844 -11.5783920288086\\
145.574935913086 -11.5785255432129\\
145.587432861328 -11.5786590576172\\
145.59992980957 -11.5787925720215\\
145.612442016602 -11.5789194107056\\
145.624938964844 -11.5790376663208\\
145.637435913086 -11.5791616439819\\
145.649932861328 -11.5792856216431\\
145.66242980957 -11.5794095993042\\
145.674942016602 -11.5795288085938\\
145.687438964844 -11.5796432495117\\
145.699935913086 -11.5797576904297\\
145.712432861328 -11.5798711776733\\
145.72492980957 -11.579981803894\\
145.737442016602 -11.5800867080688\\
145.749938964844 -11.5801916122437\\
145.762435913086 -11.5802965164185\\
145.774932861328 -11.5804014205933\\
145.78742980957 -11.5805063247681\\
145.799942016602 -11.5806112289429\\
145.812438964844 -11.5807123184204\\
145.824935913086 -11.5808076858521\\
145.837432861328 -11.5809030532837\\
145.84992980957 -11.5809926986694\\
145.862442016602 -11.5810785293579\\
145.874938964844 -11.5811643600464\\
145.887435913086 -11.5812501907349\\
145.899932861328 -11.5813360214233\\
145.91242980957 -11.5814218521118\\
145.924942016602 -11.5815076828003\\
145.937438964844 -11.5815935134888\\
145.949935913086 -11.5816793441772\\
145.962432861328 -11.5817651748657\\
145.97492980957 -11.5818500518799\\
145.987442016602 -11.5819358825684\\
145.999938964844 -11.5820150375366\\
146.012435913086 -11.5820913314819\\
146.024932861328 -11.5821676254272\\
146.03742980957 -11.5822439193726\\
146.049942016602 -11.5823202133179\\
146.062438964844 -11.5823965072632\\
146.074935913086 -11.5824728012085\\
146.087432861328 -11.5825490951538\\
146.09992980957 -11.5826263427734\\
146.112442016602 -11.5827121734619\\
146.124938964844 -11.5828046798706\\
146.137435913086 -11.5829010009766\\
146.149932861328 -11.5830059051514\\
146.16242980957 -11.5831108093262\\
146.174942016602 -11.5832090377808\\
146.187438964844 -11.5833024978638\\
146.199935913086 -11.5833883285522\\
146.212432861328 -11.5834732055664\\
146.22492980957 -11.5835590362549\\
146.237442016602 -11.5836448669434\\
146.249938964844 -11.5837306976318\\
146.262435913086 -11.5838241577148\\
146.274932861328 -11.5839214324951\\
146.28742980957 -11.5840291976929\\
146.299942016602 -11.5841464996338\\
146.312438964844 -11.5842733383179\\
146.324935913086 -11.5844097137451\\
146.337432861328 -11.5845546722412\\
146.34992980957 -11.5847072601318\\
146.362442016602 -11.5848608016968\\
146.374938964844 -11.585015296936\\
146.387435913086 -11.5851678848267\\
146.399932861328 -11.5853109359741\\
146.41242980957 -11.5854444503784\\
146.424942016602 -11.5855779647827\\
146.437438964844 -11.5857105255127\\
146.449935913086 -11.585844039917\\
146.462432861328 -11.5859775543213\\
146.47492980957 -11.5861177444458\\
146.487442016602 -11.5862607955933\\
146.499938964844 -11.5864105224609\\
146.512435913086 -11.5865650177002\\
146.524932861328 -11.586727142334\\
146.53742980957 -11.5868968963623\\
146.549942016602 -11.5870723724365\\
146.562438964844 -11.5872535705566\\
146.574935913086 -11.5874347686768\\
146.587432861328 -11.5876092910767\\
146.59992980957 -11.5877637863159\\
146.612442016602 -11.5879068374634\\
146.624938964844 -11.5880498886108\\
146.637435913086 -11.5882005691528\\
146.649932861328 -11.5883588790894\\
146.66242980957 -11.5885210037231\\
146.674942016602 -11.5886907577515\\
146.687438964844 -11.5888624191284\\
146.699935913086 -11.5890340805054\\
146.712432861328 -11.5892143249512\\
146.72492980957 -11.5893955230713\\
146.737442016602 -11.5895767211914\\
146.749938964844 -11.5897569656372\\
146.762435913086 -11.589937210083\\
146.774932861328 -11.5901174545288\\
146.78742980957 -11.5902948379517\\
146.799942016602 -11.5904664993286\\
146.812438964844 -11.5906381607056\\
146.824935913086 -11.5908098220825\\
146.837432861328 -11.5909814834595\\
146.84992980957 -11.5911521911621\\
146.862442016602 -11.5913219451904\\
146.874938964844 -11.5914840698242\\
146.887435913086 -11.591646194458\\
146.899932861328 -11.5918083190918\\
146.91242980957 -11.5919694900513\\
146.924942016602 -11.5921316146851\\
146.937438964844 -11.5922927856445\\
146.949935913086 -11.5924549102783\\
146.962432861328 -11.5926170349121\\
146.97492980957 -11.5927791595459\\
146.987442016602 -11.5929412841797\\
146.999938964844 -11.5931034088135\\
147.012435913086 -11.5932645797729\\
147.024932861328 -11.5934162139893\\
147.03742980957 -11.5935649871826\\
147.049942016602 -11.5937232971191\\
147.062438964844 -11.5938930511475\\
147.074935913086 -11.5940732955933\\
147.087432861328 -11.5942544937134\\
147.09992980957 -11.5944356918335\\
147.112442016602 -11.5946168899536\\
147.124938964844 -11.5947980880737\\
147.137435913086 -11.5949792861938\\
147.149932861328 -11.595160484314\\
147.16242980957 -11.5953416824341\\
147.174942016602 -11.5955228805542\\
147.187438964844 -11.5957040786743\\
147.199935913086 -11.5958852767944\\
147.212432861328 -11.5960702896118\\
147.22492980957 -11.5962610244751\\
147.237442016602 -11.5964517593384\\
147.249938964844 -11.5966482162476\\
147.262435913086 -11.596848487854\\
147.274932861328 -11.597053527832\\
147.28742980957 -11.5972633361816\\
147.299942016602 -11.5974760055542\\
147.312438964844 -11.5976943969727\\
147.324935913086 -11.5979137420654\\
147.337432861328 -11.5981330871582\\
147.34992980957 -11.598352432251\\
147.362442016602 -11.5985717773438\\
147.374938964844 -11.5987911224365\\
147.387435913086 -11.5990104675293\\
147.399932861328 -11.5992269515991\\
147.41242980957 -11.5994367599487\\
147.424942016602 -11.5996465682983\\
147.437438964844 -11.5998554229736\\
147.449935913086 -11.6000566482544\\
147.462432861328 -11.6002569198608\\
147.47492980957 -11.6004571914673\\
147.487442016602 -11.6006536483765\\
147.499938964844 -11.6008443832397\\
147.512435913086 -11.6010236740112\\
147.524932861328 -11.6011791229248\\
147.53742980957 -11.6013221740723\\
147.549942016602 -11.6014699935913\\
147.562438964844 -11.6016254425049\\
147.574935913086 -11.6017913818359\\
147.587432861328 -11.6019668579102\\
147.59992980957 -11.6021480560303\\
147.612442016602 -11.6023292541504\\
147.624938964844 -11.602502822876\\
147.637435913086 -11.6026744842529\\
147.649932861328 -11.6028451919556\\
147.66242980957 -11.6030168533325\\
147.674942016602 -11.6031942367554\\
147.687438964844 -11.6033744812012\\
147.699935913086 -11.6035480499268\\
147.712432861328 -11.6037197113037\\
147.72492980957 -11.6038913726807\\
147.737442016602 -11.6040687561035\\
147.749938964844 -11.6042499542236\\
147.762435913086 -11.6044311523438\\
147.774932861328 -11.6046123504639\\
147.78742980957 -11.604793548584\\
147.799942016602 -11.6049833297729\\
147.812438964844 -11.6051731109619\\
147.824935913086 -11.6053628921509\\
147.837432861328 -11.6055536270142\\
147.84992980957 -11.6057443618774\\
147.862442016602 -11.6059350967407\\
147.874938964844 -11.606125831604\\
147.887435913086 -11.6063165664673\\
147.899932861328 -11.6065073013306\\
147.91242980957 -11.6066980361938\\
147.924942016602 -11.6068887710571\\
147.937438964844 -11.6070785522461\\
147.949935913086 -11.6072683334351\\
147.962432861328 -11.607458114624\\
147.97492980957 -11.6076431274414\\
147.987442016602 -11.6078233718872\\
147.999938964844 -11.6079874038696\\
148.012435913086 -11.608115196228\\
148.024932861328 -11.6082162857056\\
148.03742980957 -11.6083116531372\\
148.049942016602 -11.6084156036377\\
148.062438964844 -11.6085243225098\\
148.074935913086 -11.6086387634277\\
148.087432861328 -11.6087532043457\\
148.09992980957 -11.6088743209839\\
148.112442016602 -11.608998298645\\
148.124938964844 -11.6091222763062\\
148.137435913086 -11.6092510223389\\
148.149932861328 -11.6093845367432\\
148.16242980957 -11.6095180511475\\
148.174942016602 -11.6096611022949\\
148.187438964844 -11.6098041534424\\
148.199935913086 -11.609956741333\\
148.212432861328 -11.6101112365723\\
148.22492980957 -11.6102733612061\\
148.237442016602 -11.6104383468628\\
148.249938964844 -11.6106100082397\\
148.262435913086 -11.6107816696167\\
148.274932861328 -11.6109609603882\\
148.28742980957 -11.6111421585083\\
148.299942016602 -11.6113233566284\\
148.312438964844 -11.6115083694458\\
148.324935913086 -11.6116952896118\\
148.337432861328 -11.611873626709\\
148.34992980957 -11.6120452880859\\
148.362442016602 -11.6122169494629\\
148.374938964844 -11.6123886108398\\
148.387435913086 -11.6125507354736\\
148.399932861328 -11.6127023696899\\
148.41242980957 -11.6128339767456\\
148.424942016602 -11.6129579544067\\
148.437438964844 -11.6130809783936\\
148.449935913086 -11.6132049560547\\
148.462432861328 -11.6133289337158\\
148.47492980957 -11.613452911377\\
148.487442016602 -11.6135816574097\\
148.499938964844 -11.613715171814\\
148.512435913086 -11.6138486862183\\
148.524932861328 -11.6139822006226\\
148.53742980957 -11.6141233444214\\
148.549942016602 -11.6142663955688\\
148.562438964844 -11.6144094467163\\
148.574935913086 -11.614559173584\\
148.587432861328 -11.6147117614746\\
148.59992980957 -11.6148643493652\\
148.612442016602 -11.615026473999\\
148.624938964844 -11.6151895523071\\
148.637435913086 -11.6153602600098\\
148.649932861328 -11.6155319213867\\
148.66242980957 -11.6157035827637\\
148.674942016602 -11.6158828735352\\
148.687438964844 -11.6160774230957\\
148.699935913086 -11.6162996292114\\
148.712432861328 -11.61656665802\\
148.72492980957 -11.6168832778931\\
};
\addplot [
color=mycolor1,
solid
]
table[row sep=crcr]{
148.72492980957 -11.6168832778931\\
148.737442016602 -11.6172552108765\\
148.749938964844 -11.6176815032959\\
148.762435913086 -11.6181612014771\\
148.774932861328 -11.6186895370483\\
148.78742980957 -11.6192588806152\\
148.799942016602 -11.6198568344116\\
148.812438964844 -11.6204452514648\\
148.824935913086 -11.62096118927\\
148.837432861328 -11.6213493347168\\
148.84992980957 -11.621636390686\\
148.862442016602 -11.6218748092651\\
148.874938964844 -11.6220941543579\\
148.887435913086 -11.6223049163818\\
148.899932861328 -11.6225242614746\\
148.91242980957 -11.6227569580078\\
148.924942016602 -11.6230068206787\\
148.937438964844 -11.6232786178589\\
148.949935913086 -11.623574256897\\
148.962432861328 -11.6238965988159\\
148.97492980957 -11.6242418289185\\
148.987442016602 -11.6246109008789\\
148.999938964844 -11.6249990463257\\
149.012435913086 -11.6254062652588\\
149.024932861328 -11.6258344650269\\
149.03742980957 -11.6262874603271\\
149.049942016602 -11.626766204834\\
149.062438964844 -11.6272659301758\\
149.074935913086 -11.6277875900269\\
149.087432861328 -11.6283302307129\\
149.09992980957 -11.6288938522339\\
149.112442016602 -11.6294689178467\\
149.124938964844 -11.6300420761108\\
149.137435913086 -11.6305742263794\\
149.149932861328 -11.6309995651245\\
149.16242980957 -11.6313505172729\\
149.174942016602 -11.631685256958\\
149.187438964844 -11.6320323944092\\
149.199935913086 -11.6323976516724\\
149.212432861328 -11.6327800750732\\
149.22492980957 -11.6331777572632\\
149.237442016602 -11.6335926055908\\
149.249938964844 -11.6340188980103\\
149.262435913086 -11.6344575881958\\
149.274932861328 -11.6349077224731\\
149.28742980957 -11.6353693008423\\
149.299942016602 -11.6358413696289\\
149.312438964844 -11.6363229751587\\
149.324935913086 -11.636812210083\\
149.337432861328 -11.6373081207275\\
149.34992980957 -11.6378107070923\\
149.362442016602 -11.6383113861084\\
149.374938964844 -11.6388072967529\\
149.387435913086 -11.6393022537231\\
149.399932861328 -11.6397848129272\\
149.41242980957 -11.6402263641357\\
149.424942016602 -11.6406059265137\\
149.437438964844 -11.6409587860107\\
149.449935913086 -11.6413164138794\\
149.462432861328 -11.6416912078857\\
149.47492980957 -11.6420879364014\\
149.487442016602 -11.6425075531006\\
149.499938964844 -11.6429462432861\\
149.512435913086 -11.6433944702148\\
149.524932861328 -11.6438074111938\\
149.53742980957 -11.6440839767456\\
149.549942016602 -11.6442708969116\\
149.562438964844 -11.6444883346558\\
149.574935913086 -11.6447305679321\\
149.587432861328 -11.6449928283691\\
149.59992980957 -11.6452722549438\\
149.612442016602 -11.6455659866333\\
149.624938964844 -11.6458759307861\\
149.637435913086 -11.6462030410767\\
149.649932861328 -11.6465501785278\\
149.66242980957 -11.6469192504883\\
149.674942016602 -11.6473121643066\\
149.687438964844 -11.6477289199829\\
149.699935913086 -11.6481628417969\\
149.712432861328 -11.6486082077026\\
149.72492980957 -11.6490430831909\\
149.737442016602 -11.6494464874268\\
149.749938964844 -11.6498117446899\\
149.762435913086 -11.6501445770264\\
149.774932861328 -11.6504354476929\\
149.78742980957 -11.6506872177124\\
149.799942016602 -11.6509256362915\\
149.812438964844 -11.6511564254761\\
149.824935913086 -11.6513862609863\\
149.837432861328 -11.6516199111938\\
149.84992980957 -11.6518487930298\\
149.862442016602 -11.6520776748657\\
149.874938964844 -11.6523065567017\\
149.887435913086 -11.6525354385376\\
149.899932861328 -11.6527643203735\\
149.91242980957 -11.6530017852783\\
149.924942016602 -11.6532459259033\\
149.937438964844 -11.6534957885742\\
149.949935913086 -11.6537532806396\\
149.962432861328 -11.654013633728\\
149.97492980957 -11.6542797088623\\
149.987442016602 -11.6545429229736\\
149.99934387207 -11.6547880172729\\
};
\addlegendentry{$\dtmax{} = \expneg{1.25}{3}{[s]}$};

\end{axis}
\end{tikzpicture}%
\caption{Condensation in upper head as calculated by the nonlinear solver.}
\label{fig:refillGammaNln}
\end{figure}

Next, \fig{fig:refillGammaNln} contains the condensation for the four different \dtmax{}.
The axes of this graph are the same as those in \fig{fig:refillGammaLin}.
Note the dramatic decrease both the magnitude and variability of the quantity as a result of using the nonlinear solver.
This results argues that the nonlinear solver is capable of producing solutions that are not only  more consistent than those produced by the linear solver but also different in value.

\begin{figure}[h!tb]
\centering
% This file was created by matlab2tikz v0.4.3.
% Copyright (c) 2008--2013, Nico Schlömer <nico.schloemer@gmail.com>
% All rights reserved.
% 
\tikzsetnextfilename{plots/refillGammaDom_eps}
%
% defining custom colors
\definecolor{mycolor1}{rgb}{0,0.75,0.75}%
%
\begin{tikzpicture}

\begin{axis}[%
width=\mytikzpicwidth,
height=\mytikzpicheight,
scale only axis,
xmin=0,
xmax=150,
xlabel={Time $[\text{s}]$},
ymin=-30,
ymax=2,
ylabel={$\Gamma \,[\,\lbm{}\,]$},
legend style={at={(0.03,0.03)},anchor=south west,draw=black,fill=white,legend cell align=left}
]
\addplot [
color=blue,
solid
]
table[row sep=crcr]{
0 0\\
1.02020955085754 0\\
2.02020955085754 0\\
3.02020955085754 0\\
4.02020978927612 0\\
5.02020978927612 0\\
6.02020978927612 0\\
7.02020978927612 0\\
8.02020931243896 0\\
9.02020931243896 0\\
10.020209312439 0\\
11.020209312439 0\\
12.020209312439 0\\
13.020209312439 0\\
14.020209312439 0\\
15.020209312439 0\\
16.0202102661133 0\\
17.0202102661133 0\\
18.0202102661133 0\\
19.0202102661133 0\\
20.0202102661133 7.86967746080336e-07\\
21.0372467041016 -0.559368312358856\\
22.0499591827393 -0.654244422912598\\
23.0610542297363 -0.688821375370026\\
24.0718193054199 -0.712441563606262\\
25.092472076416 -0.73548686504364\\
26.1183681488037 -0.766193687915802\\
27.1576538085938 -0.8132404088974\\
28.1733055114746 -0.873833954334259\\
29.1733245849609 -0.948797821998596\\
30.2167072296143 -1.05374681949615\\
31.2297992706299 -1.17880618572235\\
32.2609901428223 -1.32520854473114\\
33.2609901428223 -1.4756646156311\\
34.2609901428223 -1.62950968742371\\
35.2009925842285 -1.77639412879944\\
36.2009925842285 -1.93185329437256\\
37.229076385498 -2.08523011207581\\
38.2595138549805 -2.23744487762451\\
39.3087348937988 -2.39393544197083\\
40.3428039550781 -2.547203540802\\
41.3930206298828 -2.70191931724548\\
42.3930206298828 -2.8459620475769\\
43.3930206298828 -2.98291277885437\\
44.3930206298828 -3.12144470214844\\
45.4188117980957 -3.27311515808105\\
46.4643478393555 -3.42284321784973\\
47.5020332336426 -3.56293082237244\\
48.529109954834 -3.69608664512634\\
49.5564041137695 -3.82822871208191\\
50.6047248840332 -3.96237206459045\\
51.636043548584 -4.08508729934692\\
52.6689109802246 -4.20111227035522\\
53.7050323486328 -4.31172466278076\\
54.760066986084 -4.42609453201294\\
55.7928657531738 -4.53814601898193\\
56.8185920715332 -4.6520209312439\\
57.8270454406738 -4.76798915863037\\
58.8301467895508 -4.8863377571106\\
59.8399772644043 -5.00525951385498\\
60.8548774719238 -5.12281465530396\\
61.8615226745605 -5.23837995529175\\
62.8769264221191 -5.3571515083313\\
63.8769264221191 -5.47415018081665\\
64.8928070068359 -5.59333038330078\\
65.9300231933594 -5.72198581695557\\
66.9683990478516 -5.84428548812866\\
68.001579284668 -5.96295642852783\\
69.0236587524414 -6.0666356086731\\
70.0005722045898 -6.159499168396\\
71.0294036865234 -6.2600040435791\\
72.0606994628906 -6.3646821975708\\
73.0947799682617 -6.47059917449951\\
74.1375885009766 -6.57564306259155\\
75.1573333740234 -6.68160581588745\\
76.1802139282227 -6.7931809425354\\
77.2074661254883 -6.90487957000732\\
78.2074661254883 -7.00946998596191\\
79.2074661254883 -7.11572599411011\\
80.2074661254883 -7.2213978767395\\
81.2074661254883 -7.32678842544556\\
82.2074661254883 -7.43029308319092\\
83.2074661254883 -7.53550720214844\\
84.2074661254883 -7.64155864715576\\
85.2074661254883 -7.74731254577637\\
86.2074661254883 -7.85052156448364\\
87.2074661254883 -7.95134687423706\\
88.2074661254883 -8.05166435241699\\
89.2074661254883 -8.15128421783447\\
90.2074661254883 -8.25059413909912\\
91.2074661254883 -8.35227203369141\\
92.2377548217773 -8.46111297607422\\
93.2607727050781 -8.56967449188232\\
94.2871475219727 -8.67463970184326\\
95.3266677856445 -8.77704906463623\\
96.3612213134766 -8.87549591064453\\
97.3994293212891 -8.97043895721436\\
98.4278793334961 -9.06012439727783\\
99.4573516845703 -9.1471700668335\\
100.040000915527 -9.19574356079102\\
100.139999389648 -9.20410537719727\\
100.23999786377 -9.2122106552124\\
100.339996337891 -9.22017383575439\\
100.440002441406 -9.22809219360352\\
100.540000915527 -9.23605442047119\\
100.639999389648 -9.24409866333008\\
100.73999786377 -9.25224494934082\\
100.839996337891 -9.26047706604004\\
100.940002441406 -9.26875591278076\\
101.040000915527 -9.27700519561768\\
101.139999389648 -9.2851095199585\\
101.23999786377 -9.29299545288086\\
101.339996337891 -9.30078029632568\\
101.440002441406 -9.30854320526123\\
101.540000915527 -9.31634998321533\\
101.639999389648 -9.32424354553223\\
101.73999786377 -9.33223628997803\\
101.839996337891 -9.34029102325439\\
101.940002441406 -9.34837055206299\\
102.040000915527 -9.3563985824585\\
102.139999389648 -9.36427307128906\\
102.23999786377 -9.37197113037109\\
102.339996337891 -9.37960815429688\\
102.440002441406 -9.38722705841064\\
102.540000915527 -9.39488697052002\\
102.639999389648 -9.40262699127197\\
102.73999786377 -9.41045188903809\\
102.839996337891 -9.41830825805664\\
102.940002441406 -9.42617130279541\\
103.040000915527 -9.4339656829834\\
103.139999389648 -9.44161033630371\\
103.23999786377 -9.44911098480225\\
103.339996337891 -9.45656204223633\\
103.440002441406 -9.46400737762451\\
103.540000915527 -9.47147846221924\\
103.639999389648 -9.47901725769043\\
103.73999786377 -9.4866247177124\\
103.839996337891 -9.49427032470703\\
103.940002441406 -9.50191783905029\\
104.040000915527 -9.50949096679688\\
104.139999389648 -9.51692771911621\\
104.23999786377 -9.52424716949463\\
104.339996337891 -9.53151035308838\\
104.440002441406 -9.5387601852417\\
104.540000915527 -9.54603576660156\\
104.639999389648 -9.55336380004883\\
104.73999786377 -9.56074047088623\\
104.839996337891 -9.56813144683838\\
104.940002441406 -9.57550144195557\\
105.040000915527 -9.58280849456787\\
105.139999389648 -9.59001731872559\\
105.23999786377 -9.59712505340576\\
105.339996337891 -9.60419464111328\\
105.440002441406 -9.61124134063721\\
105.540000915527 -9.61829662322998\\
105.639999389648 -9.62538719177246\\
105.73999786377 -9.63251209259033\\
105.839996337891 -9.63963317871094\\
105.940002441406 -9.64673805236816\\
106.040000915527 -9.65379905700684\\
106.139999389648 -9.66078186035156\\
106.23999786377 -9.66768646240234\\
106.339996337891 -9.67454624176025\\
106.440002441406 -9.68138408660889\\
106.540000915527 -9.68821907043457\\
106.639999389648 -9.69507122039795\\
106.73999786377 -9.70195770263672\\
106.839996337891 -9.70884418487549\\
106.940002441406 -9.71571731567383\\
107.040000915527 -9.72257328033447\\
107.139999389648 -9.72938060760498\\
107.23999786377 -9.73612022399902\\
107.339996337891 -9.74279880523682\\
107.440002441406 -9.74945449829102\\
107.540000915527 -9.75609970092773\\
107.639999389648 -9.76276302337646\\
107.73999786377 -9.76943874359131\\
107.839996337891 -9.776123046875\\
107.940002441406 -9.78278064727783\\
108.040000915527 -9.78942680358887\\
108.139999389648 -9.79604339599609\\
108.23999786377 -9.80263233184814\\
108.339996337891 -9.80917930603027\\
108.440002441406 -9.81568431854248\\
108.540000915527 -9.82217788696289\\
108.639999389648 -9.82865142822266\\
108.73999786377 -9.83512878417969\\
108.839996337891 -9.84160423278809\\
108.940002441406 -9.84809494018555\\
109.040000915527 -9.85457801818848\\
109.139999389648 -9.8610372543335\\
109.23999786377 -9.86747074127197\\
109.339996337891 -9.87388515472412\\
109.440002441406 -9.8802604675293\\
109.540000915527 -9.88660335540771\\
109.639999389648 -9.89292049407959\\
109.73999786377 -9.89923667907715\\
109.839996337891 -9.90555286407471\\
109.940002441406 -9.91187858581543\\
110.040000915527 -9.91819953918457\\
110.139999389648 -9.92450904846191\\
110.23999786377 -9.93080425262451\\
110.339996337891 -9.93707752227783\\
110.440002441406 -9.94333744049072\\
110.540000915527 -9.94957160949707\\
110.639999389648 -9.95578861236572\\
110.73999786377 -9.96199893951416\\
110.839996337891 -9.96820640563965\\
110.940002441406 -9.9744119644165\\
111.040000915527 -9.98060035705566\\
111.139999389648 -9.98677062988281\\
111.23999786377 -9.99290657043457\\
111.339996337891 -9.99901676177979\\
111.440002441406 -10.0051317214966\\
111.540000915527 -10.0112676620483\\
111.639999389648 -10.0174322128296\\
111.73999786377 -10.0235939025879\\
111.839996337891 -10.0297422409058\\
111.940002441406 -10.0358810424805\\
112.040000915527 -10.0419950485229\\
112.139999389648 -10.0480928421021\\
112.23999786377 -10.0541725158691\\
112.339996337891 -10.0602512359619\\
112.440002441406 -10.0663290023804\\
112.540000915527 -10.0724039077759\\
112.639999389648 -10.0784749984741\\
112.73999786377 -10.0845336914063\\
112.839996337891 -10.0905876159668\\
112.940002441406 -10.0966386795044\\
113.040000915527 -10.1026754379272\\
113.139999389648 -10.108681678772\\
113.23999786377 -10.1146650314331\\
113.339996337891 -10.1206226348877\\
113.440002441406 -10.1265535354614\\
113.540000915527 -10.1324596405029\\
113.639999389648 -10.1383447647095\\
113.73999786377 -10.1442308425903\\
113.839996337891 -10.1501026153564\\
113.940002441406 -10.1559638977051\\
114.040000915527 -10.1618223190308\\
114.139999389648 -10.1676769256592\\
114.23999786377 -10.1735153198242\\
114.339996337891 -10.1793241500854\\
114.440002441406 -10.1851119995117\\
114.540000915527 -10.1908779144287\\
114.639999389648 -10.1966075897217\\
114.73999786377 -10.2022914886475\\
114.839996337891 -10.2079448699951\\
114.940002441406 -10.2135734558105\\
115.040000915527 -10.2191944122314\\
115.139999389648 -10.2248182296753\\
115.23999786377 -10.2304277420044\\
115.339996337891 -10.2360277175903\\
115.440002441406 -10.2415962219238\\
115.540000915527 -10.2471446990967\\
115.639999389648 -10.2526769638062\\
115.73999786377 -10.2581796646118\\
115.839996337891 -10.2636575698853\\
115.940002441406 -10.2691125869751\\
116.040000915527 -10.2745428085327\\
116.139999389648 -10.2799625396729\\
116.23999786377 -10.2853765487671\\
116.339996337891 -10.2907915115356\\
116.440002441406 -10.2961826324463\\
116.540000915527 -10.3015489578247\\
116.639999389648 -10.3069019317627\\
116.73999786377 -10.3122291564941\\
116.839996337891 -10.3175315856934\\
116.940002441406 -10.3228120803833\\
117.040000915527 -10.328052520752\\
117.139999389648 -10.3332614898682\\
117.23999786377 -10.3384370803833\\
117.339996337891 -10.3436040878296\\
117.440002441406 -10.3487634658813\\
117.540000915527 -10.3539047241211\\
117.639999389648 -10.3590450286865\\
117.73999786377 -10.3641691207886\\
117.839996337891 -10.3692626953125\\
117.940002441406 -10.3743343353271\\
118.040000915527 -10.3793792724609\\
118.139999389648 -10.3843927383423\\
118.23999786377 -10.38938331604\\
118.339996337891 -10.3943319320679\\
118.440002441406 -10.3992376327515\\
118.540000915527 -10.4041185379028\\
118.639999389648 -10.4089775085449\\
118.73999786377 -10.4138288497925\\
118.839996337891 -10.418664932251\\
118.940002441406 -10.4234952926636\\
119.040000915527 -10.4283008575439\\
119.139999389648 -10.4330997467041\\
119.23999786377 -10.4378643035889\\
119.339996337891 -10.4426078796387\\
119.440002441406 -10.4473133087158\\
119.540000915527 -10.451979637146\\
119.639999389648 -10.456615447998\\
119.73999786377 -10.4612369537354\\
119.839996337891 -10.4658451080322\\
119.940002441406 -10.4704370498657\\
120.040000915527 -10.475025177002\\
120.139999389648 -10.4796018600464\\
120.23999786377 -10.4841718673706\\
120.339996337891 -10.4887142181396\\
120.440002441406 -10.4932403564453\\
120.540000915527 -10.497745513916\\
120.639999389648 -10.5022430419922\\
120.73999786377 -10.5066833496094\\
120.839996337891 -10.5110836029053\\
120.940002441406 -10.5154638290405\\
121.040000915527 -10.5198059082031\\
121.139999389648 -10.5241355895996\\
121.23999786377 -10.5284690856934\\
121.339996337891 -10.5328226089478\\
121.440002441406 -10.5372037887573\\
121.540000915527 -10.5415935516357\\
121.639999389648 -10.5459613800049\\
121.73999786377 -10.5503187179565\\
121.839996337891 -10.5546388626099\\
121.940002441406 -10.5589094161987\\
122.040000915527 -10.5631351470947\\
122.139999389648 -10.5673513412476\\
122.23999786377 -10.5715074539185\\
122.339996337891 -10.5756425857544\\
122.440002441406 -10.5797510147095\\
122.540000915527 -10.5838785171509\\
122.639999389648 -10.5880184173584\\
122.73999786377 -10.5921649932861\\
122.839996337891 -10.5962934494019\\
122.940002441406 -10.6003885269165\\
123.040000915527 -10.6044673919678\\
123.139999389648 -10.6085147857666\\
123.23999786377 -10.6124954223633\\
123.339996337891 -10.6163778305054\\
123.440002441406 -10.620189666748\\
123.540000915527 -10.6239500045776\\
123.639999389648 -10.6276884078979\\
123.73999786377 -10.631402015686\\
123.839996337891 -10.6350898742676\\
123.940002441406 -10.6387481689453\\
124.040000915527 -10.6423559188843\\
124.139999389648 -10.6459074020386\\
124.23999786377 -10.6494112014771\\
124.339996337891 -10.652886390686\\
124.440002441406 -10.6563482284546\\
124.540000915527 -10.659704208374\\
124.639999389648 -10.6629953384399\\
124.73999786377 -10.6662693023682\\
124.839996337891 -10.6695489883423\\
124.940002441406 -10.6728563308716\\
125.040000915527 -10.6761493682861\\
125.139999389648 -10.6793794631958\\
125.23999786377 -10.6825370788574\\
125.339996337891 -10.685604095459\\
125.440002441406 -10.6886138916016\\
125.540000915527 -10.6916007995605\\
125.639999389648 -10.6945638656616\\
125.73999786377 -10.6975193023682\\
125.839996337891 -10.7004690170288\\
125.940002441406 -10.7034158706665\\
126.040000915527 -10.7063665390015\\
126.139999389648 -10.709267616272\\
126.23999786377 -10.7120933532715\\
126.339996337891 -10.7148361206055\\
126.440002441406 -10.7175359725952\\
126.537605285645 -10.7201509475708\\
126.637603759766 -10.7228498458862\\
126.736190795898 -10.7255373001099\\
126.83618927002 -10.7282476425171\\
126.926094055176 -10.7306661605835\\
127.01570892334 -10.7330379486084\\
127.099319458008 -10.735203742981\\
127.149597167969 -10.7364892959595\\
127.202972412109 -10.7378425598145\\
127.265632629395 -10.7394418716431\\
127.359680175781 -10.7418584823608\\
127.455467224121 -10.7442712783813\\
127.540840148926 -10.7464151382446\\
127.633682250977 -10.7487468719482\\
127.733253479004 -10.7512798309326\\
127.833251953125 -10.7538480758667\\
127.933250427246 -10.7564182281494\\
128.027877807617 -10.7588558197021\\
128.127777099609 -10.7614450454712\\
128.208786010742 -10.7635040283203\\
128.296493530273 -10.7656869888306\\
128.369720458984 -10.7674884796143\\
128.465789794922 -10.7698354721069\\
128.553833007813 -10.7719945907593\\
128.651824951172 -10.7744407653809\\
128.744003295898 -10.7767868041992\\
128.837112426758 -10.7791862487793\\
128.927124023438 -10.781530380249\\
129.025665283203 -10.7841062545776\\
129.125671386719 -10.7865657806396\\
129.225540161133 -10.7887945175171\\
129.287475585938 -10.790168762207\\
129.385833740234 -10.7924137115479\\
129.48583984375 -10.7948341369629\\
129.585830688477 -10.7973918914795\\
129.685836791992 -10.7999629974365\\
129.785842895508 -10.8024282455444\\
129.885833740234 -10.8047695159912\\
129.980926513672 -10.8069181442261\\
130.073272705078 -10.808967590332\\
130.164947509766 -10.8110275268555\\
130.264938354492 -10.8133783340454\\
130.364944458008 -10.8158254623413\\
130.464935302734 -10.8183069229126\\
130.56494140625 -10.8207206726074\\
130.664947509766 -10.8230390548706\\
130.764938354492 -10.8252687454224\\
130.864944458008 -10.8274402618408\\
130.964935302734 -10.8296384811401\\
131.06494140625 -10.8318996429443\\
131.164947509766 -10.8342018127441\\
131.264938354492 -10.8365468978882\\
131.364944458008 -10.8389625549316\\
131.464935302734 -10.8415327072144\\
131.56494140625 -10.8445777893066\\
131.664733886719 -10.9342517852783\\
131.731979370117 -11.0308637619019\\
131.767120361328 -11.0737104415894\\
131.805328369141 -11.1110153198242\\
131.848602294922 -11.1452045440674\\
131.895553588867 -11.1711139678955\\
131.944778442383 -11.2567892074585\\
131.959899902344 -11.2986879348755\\
131.984527587891 -11.3471794128418\\
131.996688842773 -11.3543510437012\\
132.030822753906 -11.3629474639893\\
132.103256225586 -11.3735942840576\\
132.133438110352 -11.3774032592773\\
132.169555664063 -11.3977251052856\\
132.21142578125 -11.5591373443604\\
132.245681762695 -11.6290349960327\\
132.283920288086 -11.6708106994629\\
132.328720092773 -11.7090034484863\\
132.373626708984 -11.7390556335449\\
132.429122924805 -11.8133869171143\\
132.488189697266 -11.9517602920532\\
132.541427612305 -12.0276536941528\\
132.580184936523 -12.0359678268433\\
132.625335693359 -12.0486087799072\\
132.666885375977 -12.1640892028809\\
132.717391967773 -12.217227935791\\
132.754379272461 -12.2195768356323\\
132.78759765625 -12.2205657958984\\
132.823303222656 -12.2213544845581\\
132.879913330078 -12.2246894836426\\
132.978958129883 -12.250373840332\\
133.07275390625 -12.2519931793213\\
133.122467041016 -12.2526388168335\\
133.178573608398 -12.2534780502319\\
133.276428222656 -12.2576351165771\\
133.355438232422 -12.3195190429688\\
133.400604248047 -12.3205080032349\\
133.447311401367 -12.3163738250732\\
133.475555419922 -12.3176040649414\\
133.507720947266 -12.3185892105103\\
133.5478515625 -12.3197708129883\\
133.601211547852 -12.3205862045288\\
133.653167724609 -12.3211269378662\\
133.746200561523 -12.3232011795044\\
133.806793212891 -12.3248729705811\\
133.863128662109 -12.3261041641235\\
133.92724609375 -12.3317842483521\\
134.026596069336 -12.3808765411377\\
134.09538269043 -12.3904542922974\\
134.17707824707 -12.3945846557617\\
134.275115966797 -12.3966236114502\\
134.375122070313 -12.3990392684937\\
134.475128173828 -12.4174671173096\\
134.534240722656 -12.4416055679321\\
134.598251342773 -12.4442539215088\\
134.695205688477 -12.4443845748901\\
134.795211791992 -12.4400253295898\\
134.895202636719 -12.4349317550659\\
134.994369506836 -12.4361324310303\\
135.094360351563 -12.4375696182251\\
135.194366455078 -12.4374694824219\\
135.294357299805 -12.4356698989868\\
135.39436340332 -12.4356918334961\\
135.494064331055 -12.4373140335083\\
135.59407043457 -12.4398460388184\\
135.666229248047 -12.4412107467651\\
135.723327636719 -12.4412097930908\\
135.818481445313 -12.4403285980225\\
135.918487548828 -12.4420547485352\\
136.018493652344 -12.4440975189209\\
136.11848449707 -12.4462966918945\\
136.218490600586 -12.4497032165527\\
136.297134399414 -12.451849937439\\
136.35041809082 -12.4538526535034\\
136.441253662109 -12.4573707580566\\
136.541244506836 -12.4610986709595\\
136.641250610352 -12.4647283554077\\
136.741241455078 -12.4682388305664\\
136.841247558594 -12.4718379974365\\
136.941253662109 -12.475305557251\\
137.025588989258 -12.477915763855\\
137.0654296875 -12.4791316986084\\
137.122131347656 -12.4808530807495\\
137.218658447266 -12.4835233688354\\
137.318649291992 -12.4866533279419\\
137.418655395508 -12.4903039932251\\
137.518646240234 -12.4944105148315\\
137.61865234375 -12.498348236084\\
137.718658447266 -12.5002794265747\\
137.818649291992 -12.5041675567627\\
137.918655395508 -12.5096740722656\\
138.018646240234 -12.5171356201172\\
138.11865234375 -12.5271224975586\\
138.218658447266 -12.5403690338135\\
138.318649291992 -12.5514526367188\\
138.418655395508 -12.5610437393188\\
138.518646240234 -12.5661029815674\\
138.61865234375 -12.5712776184082\\
138.718658447266 -12.5777359008789\\
138.818649291992 -12.5853357315063\\
138.918655395508 -12.5925760269165\\
139.018646240234 -12.597861289978\\
139.11865234375 -12.6013269424438\\
139.218658447266 -12.6062440872192\\
139.318649291992 -12.6116857528687\\
139.418655395508 -12.6169309616089\\
139.518646240234 -12.6219596862793\\
139.61865234375 -12.6268033981323\\
139.718658447266 -12.6314029693604\\
139.818649291992 -12.6357135772705\\
139.918655395508 -12.6395292282104\\
140.018646240234 -12.6430349349976\\
140.11865234375 -12.6464605331421\\
140.218658447266 -12.649941444397\\
140.318649291992 -12.6532611846924\\
140.418655395508 -12.6562767028809\\
140.518646240234 -12.6587219238281\\
140.61865234375 -12.6605930328369\\
140.718658447266 -12.6621236801147\\
140.818649291992 -12.6635761260986\\
140.918655395508 -12.6648111343384\\
141.018646240234 -12.6660671234131\\
141.11865234375 -12.6676378250122\\
141.218658447266 -12.6689939498901\\
141.318649291992 -12.6700305938721\\
141.418655395508 -12.6711511611938\\
141.518646240234 -12.6723918914795\\
141.61865234375 -12.6733293533325\\
141.718658447266 -12.6743392944336\\
141.818649291992 -12.6753349304199\\
141.918655395508 -12.6765155792236\\
142.018646240234 -12.6777944564819\\
142.11865234375 -12.6792268753052\\
142.218658447266 -12.681321144104\\
142.318649291992 -12.6833114624023\\
142.418655395508 -12.6862630844116\\
142.518646240234 -12.6905555725098\\
142.61865234375 -12.6979207992554\\
142.718658447266 -12.7037725448608\\
142.818649291992 -12.7098016738892\\
142.918655395508 -12.7140684127808\\
143.018646240234 -12.71555519104\\
143.11865234375 -12.7175760269165\\
143.218658447266 -12.7205791473389\\
143.318649291992 -12.7242755889893\\
143.418655395508 -12.7288551330566\\
143.518646240234 -12.7362871170044\\
143.61865234375 -12.7435503005981\\
143.718658447266 -12.7482585906982\\
143.818649291992 -12.7513294219971\\
143.918655395508 -12.7546129226685\\
144.018646240234 -12.7587480545044\\
144.11865234375 -12.7633104324341\\
144.218658447266 -12.7666072845459\\
144.318649291992 -12.7694988250732\\
144.418655395508 -12.7726182937622\\
144.518646240234 -12.7759675979614\\
144.61865234375 -12.7800674438477\\
144.718658447266 -12.7837572097778\\
144.818649291992 -12.7877244949341\\
144.918655395508 -12.7910146713257\\
145.018646240234 -12.7940330505371\\
145.11865234375 -12.7969074249268\\
145.218658447266 -12.7995901107788\\
145.318649291992 -12.8022155761719\\
145.418655395508 -12.8048105239868\\
145.518646240234 -12.8073596954346\\
145.61865234375 -12.8098421096802\\
145.718658447266 -12.81223487854\\
145.818649291992 -12.8145179748535\\
145.918655395508 -12.8166933059692\\
146.018646240234 -12.8187952041626\\
146.11865234375 -12.8208627700806\\
146.218658447266 -12.8229122161865\\
146.318649291992 -12.8249263763428\\
146.418655395508 -12.8268957138062\\
146.518646240234 -12.8289070129395\\
146.61865234375 -12.8309392929077\\
146.718658447266 -12.8329000473022\\
146.818649291992 -12.8347492218018\\
146.918655395508 -12.8364744186401\\
147.018646240234 -12.8381652832031\\
147.11865234375 -12.8398685455322\\
147.218658447266 -12.8415489196777\\
147.318649291992 -12.8431215286255\\
147.418655395508 -12.8445596694946\\
147.518646240234 -12.8458728790283\\
147.61865234375 -12.8470659255981\\
147.718658447266 -12.8481502532959\\
147.818649291992 -12.8491439819336\\
147.918655395508 -12.8500890731812\\
148.018646240234 -12.8510007858276\\
148.11865234375 -12.8519353866577\\
148.218658447266 -12.8527603149414\\
148.318649291992 -12.8535757064819\\
148.418655395508 -12.8543891906738\\
148.518646240234 -12.8551912307739\\
148.61865234375 -12.8559799194336\\
148.718658447266 -12.8567724227905\\
148.818649291992 -12.8575439453125\\
148.918655395508 -12.8583078384399\\
149.018646240234 -12.8590517044067\\
149.11865234375 -12.859842300415\\
149.218658447266 -12.8606462478638\\
149.318649291992 -12.8614406585693\\
149.418655395508 -12.8622245788574\\
149.518646240234 -12.8630123138428\\
149.61865234375 -12.8638868331909\\
149.718658447266 -12.8647546768188\\
149.818649291992 -12.8657732009888\\
149.918655395508 -12.8665752410889\\
};
\addlegendentry{$\dtmax{} = \expneg{1.0}{2}{[s]}$};

\addplot [
color=green!50!black,
solid
]
table[row sep=crcr]{
0 0\\
1.00300097465515 0\\
2.00300097465515 0\\
3.00300097465515 0\\
4.00300073623657 0\\
5.00300073623657 0\\
6.00300073623657 0\\
7.00300073623657 0\\
8.00300121307373 0\\
9.00300121307373 0\\
10.0030012130737 0\\
11.0030012130737 0\\
12.0030012130737 0\\
13.0030012130737 0\\
14.0030012130737 0\\
15.0030012130737 0\\
16.0030002593994 0\\
17.0030002593994 0\\
18.0030002593994 0\\
19.0030002593994 0\\
20.0030002593994 -3.417011100737e-07\\
21.0194816589355 -0.569060564041138\\
22.0289421081543 -0.670745551586151\\
23.036771774292 -0.705836176872253\\
24.0486106872559 -0.730356454849243\\
25.0589027404785 -0.753448486328125\\
26.0589027404785 -0.782096445560455\\
27.0589027404785 -0.82364821434021\\
28.0589027404785 -0.876264631748199\\
29.0589027404785 -0.940972685813904\\
30.0589027404785 -1.02216899394989\\
31.0589027404785 -1.1257883310318\\
32.0589027404785 -1.24973857402802\\
33.0589027404785 -1.38369655609131\\
34.0589027404785 -1.52617084980011\\
35.0039024353027 -1.66391599178314\\
36.0039024353027 -1.80588161945343\\
37.0039024353027 -1.94258010387421\\
38.0039024353027 -2.07640790939331\\
39.0039024353027 -2.20951390266418\\
40.0039024353027 -2.34655570983887\\
41.0039024353027 -2.48744440078735\\
42.0039024353027 -2.62828326225281\\
43.0039024353027 -2.76549339294434\\
44.0039024353027 -2.90192365646362\\
45.0039024353027 -3.03193783760071\\
46.0039024353027 -3.15506172180176\\
47.0039024353027 -3.2746148109436\\
48.0039024353027 -3.39407086372375\\
49.0039024353027 -3.51518988609314\\
50.0039024353027 -3.63633871078491\\
51.0086364746094 -3.75771021842957\\
52.0148582458496 -3.87331962585449\\
53.0196990966797 -3.982741355896\\
54.0196990966797 -4.08828020095825\\
55.0196990966797 -4.19320917129517\\
56.0196990966797 -4.29833650588989\\
57.0196990966797 -4.40371084213257\\
58.0196990966797 -4.51086378097534\\
59.0196990966797 -4.62621402740479\\
60.0196990966797 -4.7335033416748\\
61.0196990966797 -4.84630537033081\\
62.0196990966797 -4.96164703369141\\
63.0196990966797 -5.07026195526123\\
64.0196990966797 -5.17992258071899\\
65.0254669189453 -5.28915977478027\\
66.0254669189453 -5.39239740371704\\
67.0254669189453 -5.50276708602905\\
68.0254669189453 -5.61604309082031\\
69.0358047485352 -5.74078845977783\\
70.0009002685547 -5.84997653961182\\
71.0059280395508 -5.94570302963257\\
72.0116119384766 -6.04024314880371\\
73.0187759399414 -6.13846826553345\\
74.0207061767578 -6.24007511138916\\
75.0229797363281 -6.34649515151978\\
76.0229797363281 -6.45292854309082\\
77.0229797363281 -6.5610294342041\\
78.0229797363281 -6.67180728912354\\
79.0229797363281 -6.78302335739136\\
80.0229797363281 -6.89381790161133\\
81.0229797363281 -7.00454950332642\\
82.0229797363281 -7.11685848236084\\
83.0229797363281 -7.23056173324585\\
84.0229797363281 -7.34429168701172\\
85.0229797363281 -7.4562840461731\\
86.0229797363281 -7.56781053543091\\
87.0229797363281 -7.67761564254761\\
88.0229797363281 -7.78563165664673\\
89.0229797363281 -7.89115953445435\\
90.0229797363281 -7.99593877792358\\
91.0229797363281 -8.10017681121826\\
92.0229797363281 -8.20646095275879\\
93.0229797363281 -8.31391525268555\\
94.0229797363281 -8.42231273651123\\
95.0229797363281 -8.53012466430664\\
96.0229797363281 -8.63573455810547\\
97.0229797363281 -8.73472690582275\\
98.0229797363281 -8.82545280456543\\
99.0229797363281 -8.90894889831543\\
100 -8.98528575897217\\
100.047782897949 -8.98895359039307\\
100.097785949707 -8.99277591705322\\
100.14778137207 -8.99660205841064\\
100.197784423828 -9.00039577484131\\
100.247779846191 -9.00413608551025\\
100.297782897949 -9.00780868530273\\
100.347785949707 -9.0114278793335\\
100.39778137207 -9.01502513885498\\
100.447784423828 -9.01861190795898\\
100.497779846191 -9.02219867706299\\
100.547782897949 -9.02579021453857\\
100.597785949707 -9.02940654754639\\
100.64778137207 -9.03304767608643\\
100.697784423828 -9.03669261932373\\
100.747779846191 -9.04032516479492\\
100.797782897949 -9.04393482208252\\
100.847785949707 -9.04750823974609\\
100.89778137207 -9.05104351043701\\
100.947784423828 -9.05454254150391\\
100.997779846191 -9.05804347991943\\
101.047782897949 -9.06155681610107\\
101.097785949707 -9.06505966186523\\
101.14778137207 -9.06854152679443\\
101.197784423828 -9.07198905944824\\
101.247779846191 -9.07538795471191\\
101.297782897949 -9.07873916625977\\
101.347785949707 -9.0820369720459\\
101.39778137207 -9.08529376983643\\
101.447784423828 -9.08851909637451\\
101.497779846191 -9.09172058105469\\
101.547782897949 -9.0949125289917\\
101.597785949707 -9.09810638427734\\
101.64778137207 -9.10130023956299\\
101.697784423828 -9.10450458526611\\
101.747779846191 -9.10770225524902\\
101.797782897949 -9.11088848114014\\
101.847785949707 -9.1140718460083\\
101.89778137207 -9.11728954315186\\
101.947784423828 -9.120530128479\\
101.997779846191 -9.12377643585205\\
102.047782897949 -9.1270227432251\\
102.097785949707 -9.13027858734131\\
102.14778137207 -9.13351154327393\\
102.197784423828 -9.13673400878906\\
102.247779846191 -9.13993167877197\\
102.297782897949 -9.14309597015381\\
102.347785949707 -9.14621448516846\\
102.39778137207 -9.14929485321045\\
102.447784423828 -9.15235424041748\\
102.497779846191 -9.15539073944092\\
102.547782897949 -9.15842151641846\\
102.597785949707 -9.16144561767578\\
102.64778137207 -9.16446590423584\\
102.697784423828 -9.16750144958496\\
102.747779846191 -9.17054653167725\\
102.797782897949 -9.17360210418701\\
102.847785949707 -9.17667579650879\\
102.89778137207 -9.17976760864258\\
102.947784423828 -9.18287086486816\\
102.997779846191 -9.18600082397461\\
103.047782897949 -9.18915176391602\\
103.097785949707 -9.19229507446289\\
103.14778137207 -9.19542980194092\\
103.197784423828 -9.19856452941895\\
103.247779846191 -9.20168113708496\\
103.297782897949 -9.20476722717285\\
103.347785949707 -9.20780754089355\\
103.39778137207 -9.21080207824707\\
103.447784423828 -9.21376991271973\\
103.497779846191 -9.21671390533447\\
103.547782897949 -9.21964263916016\\
103.597785949707 -9.22256374359131\\
103.64778137207 -9.22548675537109\\
103.697784423828 -9.22841835021973\\
103.747779846191 -9.23135757446289\\
103.797782897949 -9.23430824279785\\
103.847785949707 -9.23728179931641\\
103.89778137207 -9.24028015136719\\
103.947784423828 -9.24330043792725\\
103.997779846191 -9.24634075164795\\
104.047782897949 -9.24939250946045\\
104.097785949707 -9.25245666503906\\
104.14778137207 -9.25553703308105\\
104.197784423828 -9.2586088180542\\
104.247779846191 -9.26167774200439\\
104.297782897949 -9.26471138000488\\
104.347785949707 -9.26770877838135\\
104.39778137207 -9.27066707611084\\
104.447784423828 -9.27359771728516\\
104.497779846191 -9.27650356292725\\
104.547782897949 -9.27939701080322\\
104.597785949707 -9.28227615356445\\
104.64778137207 -9.28514862060547\\
104.697784423828 -9.28803157806396\\
104.747779846191 -9.29092216491699\\
104.797782897949 -9.29382228851318\\
104.847785949707 -9.29674053192139\\
104.89778137207 -9.29967975616455\\
104.947784423828 -9.30264186859131\\
104.997779846191 -9.30562973022461\\
105.047782897949 -9.30864715576172\\
105.097785949707 -9.31169509887695\\
105.14778137207 -9.31474781036377\\
105.197784423828 -9.31780433654785\\
105.247779846191 -9.3208646774292\\
105.297782897949 -9.32390308380127\\
105.347785949707 -9.3269100189209\\
105.39778137207 -9.32988262176514\\
105.447784423828 -9.33282661437988\\
105.497779846191 -9.33576202392578\\
105.547782897949 -9.3386926651001\\
105.597785949707 -9.34161853790283\\
105.64778137207 -9.34453392028809\\
105.697784423828 -9.34744644165039\\
105.747779846191 -9.35035514831543\\
105.797782897949 -9.35327243804932\\
105.847785949707 -9.35620021820068\\
105.89778137207 -9.35914039611816\\
105.947784423828 -9.36209964752197\\
105.997779846191 -9.36508274078369\\
106.047782897949 -9.36809158325195\\
106.097785949707 -9.37112808227539\\
106.14778137207 -9.37420272827148\\
106.197784423828 -9.37730026245117\\
106.247779846191 -9.38040637969971\\
106.297782897949 -9.38351440429688\\
106.347785949707 -9.38660430908203\\
106.39778137207 -9.3896656036377\\
106.447784423828 -9.3927116394043\\
106.497779846191 -9.39575004577637\\
106.547782897949 -9.39877510070801\\
106.597785949707 -9.40179252624512\\
106.64778137207 -9.40480422973633\\
106.697784423828 -9.40781402587891\\
106.747779846191 -9.41081809997559\\
106.797782897949 -9.41382122039795\\
106.847785949707 -9.4168176651001\\
106.89778137207 -9.4198169708252\\
106.947784423828 -9.42282867431641\\
106.997779846191 -9.42585182189941\\
107.047782897949 -9.42889022827148\\
107.097785949707 -9.43194389343262\\
107.14778137207 -9.43501472473145\\
107.197784423828 -9.4381046295166\\
107.247779846191 -9.44121265411377\\
107.297782897949 -9.44433116912842\\
107.347785949707 -9.44745635986328\\
107.39778137207 -9.45058631896973\\
107.447784423828 -9.45371341705322\\
107.497779846191 -9.45683002471924\\
107.547782897949 -9.45993232727051\\
107.597785949707 -9.4630184173584\\
107.64778137207 -9.46609687805176\\
107.697784423828 -9.46916961669922\\
107.747779846191 -9.4722318649292\\
107.797782897949 -9.47529220581055\\
107.847785949707 -9.47834587097168\\
107.89778137207 -9.48139381408691\\
107.947784423828 -9.48443794250488\\
107.997779846191 -9.48749446868896\\
108.047782897949 -9.49055957794189\\
108.097785949707 -9.49362468719482\\
108.14778137207 -9.49669361114502\\
108.197784423828 -9.49976539611816\\
108.247779846191 -9.50284862518311\\
108.297782897949 -9.50593566894531\\
108.347785949707 -9.50902462005615\\
108.39778137207 -9.51211738586426\\
108.447784423828 -9.51521396636963\\
108.497779846191 -9.51831531524658\\
108.547782897949 -9.52142333984375\\
108.597785949707 -9.52453517913818\\
108.64778137207 -9.52764701843262\\
108.697784423828 -9.5307559967041\\
108.747779846191 -9.53385829925537\\
108.797782897949 -9.53695487976074\\
108.847785949707 -9.54004287719727\\
108.89778137207 -9.54311847686768\\
108.947784423828 -9.5461893081665\\
108.997779846191 -9.54925537109375\\
109.047782897949 -9.55231857299805\\
109.097785949707 -9.55537986755371\\
109.14778137207 -9.55844020843506\\
109.197784423828 -9.56150436401367\\
109.247779846191 -9.56456756591797\\
109.297782897949 -9.56762886047363\\
109.347785949707 -9.5706901550293\\
109.39778137207 -9.57375812530518\\
109.447784423828 -9.57683372497559\\
109.497779846191 -9.57991409301758\\
109.547782897949 -9.5829906463623\\
109.597785949707 -9.5860652923584\\
109.64778137207 -9.58914375305176\\
109.697784423828 -9.59221839904785\\
109.747779846191 -9.59529209136963\\
109.797782897949 -9.59836387634277\\
109.847785949707 -9.60143756866455\\
109.89778137207 -9.60451126098633\\
109.947784423828 -9.60758590698242\\
109.997779846191 -9.61065769195557\\
110.047782897949 -9.61372566223145\\
110.097785949707 -9.61678695678711\\
110.14778137207 -9.61984825134277\\
110.197784423828 -9.62290287017822\\
110.247779846191 -9.62595462799072\\
110.297782897949 -9.62900257110596\\
110.347785949707 -9.63204956054688\\
110.39778137207 -9.63510131835938\\
110.447784423828 -9.63814926147461\\
110.497779846191 -9.64119148254395\\
110.547782897949 -9.64423561096191\\
110.597785949707 -9.6472864151001\\
110.64778137207 -9.65034103393555\\
110.697784423828 -9.65339756011963\\
110.747779846191 -9.65645217895508\\
110.797782897949 -9.65950393676758\\
110.847785949707 -9.66255855560303\\
110.89778137207 -9.66561126708984\\
110.947784423828 -9.66866302490234\\
110.997779846191 -9.67171382904053\\
111.047782897949 -9.67475986480713\\
111.097785949707 -9.67780494689941\\
111.14778137207 -9.68084716796875\\
111.197784423828 -9.68388843536377\\
111.247779846191 -9.68692111968994\\
111.297782897949 -9.68994808197021\\
111.347785949707 -9.69297409057617\\
111.39778137207 -9.69599533081055\\
111.447784423828 -9.69901275634766\\
111.497779846191 -9.70202827453613\\
111.547782897949 -9.70504188537598\\
111.597785949707 -9.70804977416992\\
111.64778137207 -9.71106147766113\\
111.697784423828 -9.71407508850098\\
111.747779846191 -9.71708869934082\\
111.797782897949 -9.7200984954834\\
111.847785949707 -9.72310256958008\\
111.89778137207 -9.72610187530518\\
111.947784423828 -9.72909641265869\\
111.997779846191 -9.73208522796631\\
112.047782897949 -9.73507308959961\\
112.097785949707 -9.73805999755859\\
112.14778137207 -9.74104785919189\\
112.197784423828 -9.74403667449951\\
112.247779846191 -9.74702739715576\\
112.297782897949 -9.75000667572021\\
112.347785949707 -9.75297832489014\\
112.39778137207 -9.75593852996826\\
112.447784423828 -9.7588939666748\\
112.497779846191 -9.76184272766113\\
112.547782897949 -9.76479721069336\\
112.597785949707 -9.76775455474854\\
112.64778137207 -9.77070903778076\\
112.697784423828 -9.77365684509277\\
112.747779846191 -9.77659702301025\\
112.797782897949 -9.77954006195068\\
112.847785949707 -9.78248119354248\\
112.89778137207 -9.78542137145996\\
112.947784423828 -9.78836250305176\\
112.997779846191 -9.79130268096924\\
113.047782897949 -9.79424381256104\\
113.097785949707 -9.7971830368042\\
113.14778137207 -9.80011749267578\\
113.197784423828 -9.80304527282715\\
113.247779846191 -9.80596733093262\\
113.297782897949 -9.80888748168945\\
113.347785949707 -9.81180572509766\\
113.39778137207 -9.81471824645996\\
113.447784423828 -9.8176212310791\\
113.497779846191 -9.82051467895508\\
113.547782897949 -9.82339859008789\\
113.597785949707 -9.8262767791748\\
113.64778137207 -9.8291482925415\\
113.697784423828 -9.83201503753662\\
113.747779846191 -9.834885597229\\
113.797782897949 -9.83775043487549\\
113.847785949707 -9.84061527252197\\
113.89778137207 -9.84347915649414\\
113.947784423828 -9.84634399414063\\
113.997779846191 -9.84921073913574\\
114.047782897949 -9.85208225250244\\
114.097785949707 -9.85495662689209\\
114.14778137207 -9.85782718658447\\
114.197784423828 -9.86069202423096\\
114.247779846191 -9.86355209350586\\
114.297782897949 -9.86640739440918\\
114.347785949707 -9.86925888061523\\
114.39778137207 -9.87210273742676\\
114.447784423828 -9.87494087219238\\
114.497779846191 -9.87777137756348\\
114.547782897949 -9.88059425354004\\
114.597785949707 -9.88340950012207\\
114.64778137207 -9.88621234893799\\
114.697784423828 -9.88900852203369\\
114.747779846191 -9.89179611206055\\
114.797782897949 -9.89457607269287\\
114.847785949707 -9.89734840393066\\
114.89778137207 -9.90011119842529\\
114.947784423828 -9.90286922454834\\
114.997779846191 -9.90562534332275\\
115.047782897949 -9.90838146209717\\
115.097785949707 -9.91113948822021\\
115.14778137207 -9.91390323638916\\
115.197784423828 -9.91666984558105\\
115.247779846191 -9.91944408416748\\
115.297782897949 -9.92222499847412\\
115.347785949707 -9.9250020980835\\
115.39778137207 -9.9277868270874\\
115.447784423828 -9.93057441711426\\
115.497779846191 -9.93335914611816\\
115.547782897949 -9.9361400604248\\
115.597785949707 -9.93891048431396\\
115.64778137207 -9.94167041778564\\
115.697784423828 -9.94442176818848\\
115.747779846191 -9.94716358184814\\
115.797782897949 -9.9498929977417\\
115.847785949707 -9.95261192321777\\
115.89778137207 -9.95531940460205\\
115.947784423828 -9.95801639556885\\
115.997779846191 -9.96070194244385\\
116.047782897949 -9.96337604522705\\
116.097785949707 -9.9660472869873\\
116.14778137207 -9.96870899200439\\
116.197784423828 -9.97137260437012\\
116.247779846191 -9.97404003143311\\
116.297782897949 -9.97671413421631\\
116.347785949707 -9.97940063476563\\
116.39778137207 -9.98209762573242\\
116.447784423828 -9.98481369018555\\
116.497779846191 -9.98752593994141\\
116.547782897949 -9.99023818969727\\
116.597785949707 -9.99295520782471\\
116.64778137207 -9.99567127227783\\
116.697784423828 -9.99838256835938\\
116.747779846191 -10.0010871887207\\
116.797782897949 -10.0037841796875\\
116.847785949707 -10.0064725875854\\
116.89778137207 -10.0091419219971\\
116.947784423828 -10.0117998123169\\
116.997779846191 -10.0144500732422\\
117.047782897949 -10.0170888900757\\
117.097785949707 -10.0197191238403\\
117.14778137207 -10.0223331451416\\
117.197784423828 -10.0249366760254\\
117.247779846191 -10.0275325775146\\
117.297782897949 -10.0301265716553\\
117.347785949707 -10.0327291488647\\
117.39778137207 -10.0353355407715\\
117.447784423828 -10.0379467010498\\
117.497779846191 -10.0405607223511\\
117.547782897949 -10.0431861877441\\
117.597785949707 -10.0458250045776\\
117.64778137207 -10.0484800338745\\
117.697784423828 -10.0511474609375\\
117.747779846191 -10.0538177490234\\
117.797782897949 -10.0564880371094\\
117.847785949707 -10.0591535568237\\
117.89778137207 -10.0618124008179\\
117.947784423828 -10.0644626617432\\
117.997779846191 -10.067099571228\\
118.047782897949 -10.0697250366211\\
118.097785949707 -10.0723352432251\\
118.14778137207 -10.0749311447144\\
118.197784423828 -10.0775098800659\\
118.247779846191 -10.0800771713257\\
118.297782897949 -10.0826349258423\\
118.347785949707 -10.08518409729\\
118.39778137207 -10.0877294540405\\
118.447784423828 -10.0902700424194\\
118.497779846191 -10.0928106307983\\
118.547782897949 -10.0953493118286\\
118.597785949707 -10.0978975296021\\
118.64778137207 -10.1004552841187\\
118.697784423828 -10.1030282974243\\
118.747779846191 -10.1056146621704\\
118.797782897949 -10.1082143783569\\
118.847785949707 -10.110818862915\\
118.89778137207 -10.1134223937988\\
118.947784423828 -10.1160230636597\\
118.997779846191 -10.1186180114746\\
119.047782897949 -10.1212024688721\\
119.097785949707 -10.1237754821777\\
119.14778137207 -10.1263360977173\\
119.197784423828 -10.1288909912109\\
119.247779846191 -10.1314315795898\\
119.297782897949 -10.1339550018311\\
119.347785949707 -10.1364555358887\\
119.39778137207 -10.1389417648315\\
119.447784423828 -10.1414136886597\\
119.497779846191 -10.1438703536987\\
119.547782897949 -10.1463212966919\\
119.597785949707 -10.1487693786621\\
119.64778137207 -10.1512107849121\\
119.697784423828 -10.1536598205566\\
119.747779846191 -10.1561126708984\\
119.797782897949 -10.1585731506348\\
119.847785949707 -10.1610488891602\\
119.89778137207 -10.1635398864746\\
119.947784423828 -10.1660442352295\\
119.997779846191 -10.168553352356\\
120.047782897949 -10.1710662841797\\
120.097785949707 -10.1735744476318\\
120.14778137207 -10.1760787963867\\
120.197784423828 -10.1785669326782\\
120.247779846191 -10.181037902832\\
120.297782897949 -10.1834926605225\\
120.347785949707 -10.1859245300293\\
120.39778137207 -10.1883411407471\\
120.447784423828 -10.1907329559326\\
120.497779846191 -10.1931037902832\\
120.547782897949 -10.1954555511475\\
120.597785949707 -10.1977863311768\\
120.64778137207 -10.200098991394\\
120.697784423828 -10.2024011611938\\
120.747779846191 -10.2046995162964\\
120.797782897949 -10.2069931030273\\
120.847785949707 -10.209300994873\\
120.89778137207 -10.2116327285767\\
120.947784423828 -10.2139520645142\\
120.997779846191 -10.2162494659424\\
121.047782897949 -10.2185478210449\\
121.097785949707 -10.2208490371704\\
121.14778137207 -10.2231531143188\\
121.197784423828 -10.2254590988159\\
121.247779846191 -10.227765083313\\
121.297782897949 -10.2300672531128\\
121.347785949707 -10.2323570251465\\
121.39778137207 -10.2346258163452\\
121.447784423828 -10.2368812561035\\
121.497779846191 -10.2391090393066\\
121.547782897949 -10.2413129806519\\
121.597785949707 -10.2434911727905\\
121.64778137207 -10.2456493377686\\
121.697784423828 -10.2477903366089\\
121.747779846191 -10.2499170303345\\
121.797782897949 -10.2520370483398\\
121.847785949707 -10.2541475296021\\
121.89778137207 -10.2562551498413\\
121.947784423828 -10.2583675384521\\
121.997779846191 -10.2604856491089\\
122.047782897949 -10.2625970840454\\
122.097785949707 -10.2647104263306\\
122.14778137207 -10.2668294906616\\
122.197784423828 -10.2689485549927\\
122.247779846191 -10.2710590362549\\
122.297782897949 -10.273157119751\\
122.347785949707 -10.2752456665039\\
122.39778137207 -10.2773218154907\\
122.447784423828 -10.2793760299683\\
122.497779846191 -10.2814121246338\\
122.547782897949 -10.2834300994873\\
122.597785949707 -10.2854242324829\\
122.64778137207 -10.2874021530151\\
122.697784423828 -10.2893619537354\\
122.747779846191 -10.2913074493408\\
122.797782897949 -10.293249130249\\
122.847785949707 -10.2951803207397\\
122.89778137207 -10.2971134185791\\
122.947784423828 -10.299054145813\\
122.997779846191 -10.300989151001\\
123.047782897949 -10.3029165267944\\
123.097785949707 -10.3048429489136\\
123.14778137207 -10.3067674636841\\
123.197784423828 -10.3086833953857\\
123.247779846191 -10.3105926513672\\
123.297782897949 -10.3125038146973\\
123.347785949707 -10.3144035339355\\
123.39778137207 -10.3162937164307\\
123.447784423828 -10.3181676864624\\
123.497779846191 -10.3200235366821\\
123.547782897949 -10.3218593597412\\
123.597785949707 -10.3236722946167\\
123.64778137207 -10.3254623413086\\
123.697784423828 -10.3272371292114\\
123.747779846191 -10.3289976119995\\
123.797782897949 -10.3307495117188\\
123.847785949707 -10.3324871063232\\
123.89778137207 -10.3342227935791\\
123.947784423828 -10.335958480835\\
123.997779846191 -10.3376951217651\\
124.047782897949 -10.339430809021\\
124.097785949707 -10.3411703109741\\
124.14778137207 -10.3428983688354\\
124.197784423828 -10.344633102417\\
124.247779846191 -10.3463687896729\\
124.297782897949 -10.3480987548828\\
124.347785949707 -10.3498191833496\\
124.39778137207 -10.3515243530273\\
124.447784423828 -10.3532161712646\\
124.497779846191 -10.3548879623413\\
124.547782897949 -10.3565406799316\\
124.597785949707 -10.35817527771\\
124.64778137207 -10.3598136901855\\
124.697784423828 -10.3614368438721\\
124.747779846191 -10.3630199432373\\
124.797782897949 -10.3645706176758\\
124.847785949707 -10.3661022186279\\
124.89778137207 -10.3676280975342\\
124.947784423828 -10.3691539764404\\
124.997779846191 -10.3706922531128\\
125.047782897949 -10.3722372055054\\
125.097785949707 -10.3737907409668\\
125.14778137207 -10.3753681182861\\
125.197784423828 -10.3769664764404\\
125.247779846191 -10.3785591125488\\
125.297782897949 -10.3801431655884\\
125.347785949707 -10.3817090988159\\
125.39778137207 -10.3832550048828\\
125.447784423828 -10.3847742080688\\
125.497779846191 -10.3862743377686\\
125.547782897949 -10.3877506256104\\
125.597785949707 -10.3892040252686\\
125.64778137207 -10.3906373977661\\
125.697784423828 -10.3920602798462\\
125.747779846191 -10.3934736251831\\
125.797782897949 -10.3948755264282\\
125.847785949707 -10.396276473999\\
125.89778137207 -10.3976774215698\\
125.947784423828 -10.3990879058838\\
125.997779846191 -10.4004983901978\\
126.047782897949 -10.401909828186\\
126.097785949707 -10.4033269882202\\
126.14778137207 -10.4047479629517\\
126.197784423828 -10.4061765670776\\
126.247779846191 -10.4076070785522\\
126.297782897949 -10.4090270996094\\
126.347785949707 -10.4104328155518\\
126.39778137207 -10.4118232727051\\
126.447784423828 -10.4132032394409\\
126.497779846191 -10.4145641326904\\
126.547782897949 -10.4159069061279\\
126.597785949707 -10.4172267913818\\
126.64778137207 -10.4185333251953\\
126.697784423828 -10.4198350906372\\
126.747779846191 -10.4211359024048\\
126.797782897949 -10.4224395751953\\
126.847785949707 -10.4237384796143\\
126.89778137207 -10.4250402450562\\
126.947784423828 -10.4263467788696\\
126.997779846191 -10.427659034729\\
127.047782897949 -10.4289798736572\\
127.097785949707 -10.4302968978882\\
127.14778137207 -10.4316091537476\\
127.197784423828 -10.432915687561\\
127.247779846191 -10.4342174530029\\
127.297782897949 -10.435507774353\\
127.347785949707 -10.4367895126343\\
127.39778137207 -10.4380588531494\\
127.447784423828 -10.4393148422241\\
127.497779846191 -10.4405641555786\\
127.547782897949 -10.4418163299561\\
127.597785949707 -10.4430503845215\\
127.64778137207 -10.4442625045776\\
127.697784423828 -10.445463180542\\
127.747779846191 -10.4466552734375\\
127.797782897949 -10.447847366333\\
127.847785949707 -10.4490451812744\\
127.89778137207 -10.4502544403076\\
127.947784423828 -10.451473236084\\
127.997779846191 -10.4527053833008\\
128.047790527344 -10.4539470672607\\
128.097778320313 -10.4551944732666\\
128.14778137207 -10.4564437866211\\
128.197784423828 -10.4576930999756\\
128.247787475586 -10.4589424133301\\
128.297790527344 -10.4601831436157\\
128.347778320313 -10.4614105224609\\
128.39778137207 -10.4626140594482\\
128.447784423828 -10.4637928009033\\
128.497787475586 -10.4649534225464\\
128.547790527344 -10.4660978317261\\
128.597778320313 -10.4672317504883\\
128.64778137207 -10.4683647155762\\
128.697784423828 -10.4694995880127\\
128.747787475586 -10.4706363677979\\
128.797790527344 -10.4717874526978\\
128.847778320313 -10.472957611084\\
128.89778137207 -10.4741506576538\\
128.947784423828 -10.4753856658936\\
128.997787475586 -10.4766521453857\\
129.047790527344 -10.4779090881348\\
129.097778320313 -10.4791374206543\\
129.14778137207 -10.4803295135498\\
129.197784423828 -10.4814720153809\\
129.247787475586 -10.4825782775879\\
129.297790527344 -10.4836616516113\\
129.347778320313 -10.4847316741943\\
129.396820068359 -10.4857702255249\\
129.445129394531 -10.4867973327637\\
129.4951171875 -10.4878711700439\\
129.545120239258 -10.488977432251\\
129.595123291016 -10.4901285171509\\
129.645126342773 -10.4913311004639\\
129.695129394531 -10.4925651550293\\
129.7451171875 -10.4938116073608\\
129.795120239258 -10.4950494766235\\
129.845123291016 -10.496265411377\\
129.895126342773 -10.4974498748779\\
129.945129394531 -10.4986066818237\\
129.9951171875 -10.4997301101685\\
130.045120239258 -10.5008296966553\\
130.095123291016 -10.5019083023071\\
130.145126342773 -10.5029764175415\\
130.195129394531 -10.5040435791016\\
130.2451171875 -10.5051116943359\\
130.295120239258 -10.5061874389648\\
130.345123291016 -10.5072832107544\\
130.395126342773 -10.5084037780762\\
130.445129394531 -10.5095529556274\\
130.4951171875 -10.5107221603394\\
130.545120239258 -10.5119047164917\\
130.595123291016 -10.5130834579468\\
130.645126342773 -10.5142507553101\\
130.695129394531 -10.51540184021\\
130.7451171875 -10.516529083252\\
130.795120239258 -10.517632484436\\
130.845123291016 -10.5187139511108\\
130.895126342773 -10.5197772979736\\
130.945129394531 -10.5208292007446\\
130.9951171875 -10.5218839645386\\
131.045120239258 -10.5229425430298\\
131.095123291016 -10.5240106582642\\
131.145126342773 -10.525089263916\\
131.195129394531 -10.526180267334\\
131.2451171875 -10.5272836685181\\
131.295120239258 -10.5284032821655\\
131.345123291016 -10.5295343399048\\
131.395126342773 -10.5306692123413\\
131.445129394531 -10.5318002700806\\
131.4951171875 -10.5329437255859\\
131.545120239258 -10.5341329574585\\
131.595123291016 -10.5353698730469\\
131.645126342773 -10.5366287231445\\
131.695129394531 -10.537878036499\\
131.7451171875 -10.539023399353\\
131.795120239258 -10.5404367446899\\
131.845123291016 -10.5448427200317\\
131.883605957031 -10.5759630203247\\
131.907867431641 -10.6081495285034\\
131.955932617188 -10.6759595870972\\
131.998687744141 -10.7356910705566\\
132.027740478516 -10.7726793289185\\
132.060134887695 -10.8074741363525\\
132.098220825195 -10.8401165008545\\
132.142501831055 -10.868109703064\\
132.180099487305 -10.9276294708252\\
132.208694458008 -11.0122537612915\\
132.226898193359 -11.0612421035767\\
132.250457763672 -11.1031475067139\\
132.262130737305 -11.1088371276855\\
132.293197631836 -11.1162624359131\\
132.343200683594 -11.1236801147461\\
132.367538452148 -11.125922203064\\
132.394805908203 -11.1287603378296\\
132.427749633789 -11.1369314193726\\
132.455307006836 -11.2032985687256\\
132.4775390625 -11.291111946106\\
132.510101318359 -11.3613805770874\\
132.539855957031 -11.3855295181274\\
132.581039428711 -11.4063873291016\\
132.621719360352 -11.4241008758545\\
132.670135498047 -11.4411325454712\\
132.718933105469 -11.5394735336304\\
132.757522583008 -11.6598749160767\\
132.804626464844 -11.7524890899658\\
132.845245361328 -11.7907543182373\\
132.881790161133 -11.797158241272\\
132.925720214844 -11.8251447677612\\
132.969039916992 -11.9385948181152\\
133.016632080078 -11.9630136489868\\
133.052139282227 -11.9643211364746\\
133.084838867188 -11.9651136398315\\
133.121841430664 -11.9686079025269\\
133.154052734375 -12.0509767532349\\
133.18034362793 -12.1171522140503\\
133.218246459961 -12.1578302383423\\
133.248672485352 -12.1618375778198\\
133.277740478516 -12.1634454727173\\
133.309906005859 -12.1647138595581\\
133.356002807617 -12.1659736633301\\
133.406005859375 -12.1679515838623\\
133.456008911133 -12.1734085083008\\
133.506011962891 -12.175877571106\\
133.555999755859 -12.1773099899292\\
133.604598999023 -12.1781911849976\\
133.652908325195 -12.1786155700684\\
133.702911376953 -12.1793022155762\\
133.752914428711 -12.1801385879517\\
133.802917480469 -12.1808376312256\\
133.852920532227 -12.1814632415771\\
133.902908325195 -12.1819639205933\\
133.952911376953 -12.1838836669922\\
134.002914428711 -12.1876382827759\\
134.052917480469 -12.1906270980835\\
134.102920532227 -12.1926403045654\\
134.152908325195 -12.193377494812\\
134.202911376953 -12.1939010620117\\
134.252914428711 -12.194842338562\\
134.302917480469 -12.1965808868408\\
134.352920532227 -12.1982135772705\\
134.402908325195 -12.199481010437\\
134.452911376953 -12.2007369995117\\
134.502914428711 -12.2021465301514\\
134.552917480469 -12.204363822937\\
134.593536376953 -12.2445831298828\\
134.628036499023 -12.2654151916504\\
134.675704956055 -12.2683115005493\\
134.725708007813 -12.270299911499\\
134.77571105957 -12.2716827392578\\
134.825714111328 -12.2723693847656\\
134.875717163086 -12.2724199295044\\
134.925704956055 -12.2718124389648\\
134.975708007813 -12.2729749679565\\
135.02571105957 -12.2742624282837\\
135.075714111328 -12.2754030227661\\
135.125717163086 -12.2762880325317\\
135.175704956055 -12.2766790390015\\
135.225708007813 -12.277117729187\\
135.27571105957 -12.2778930664063\\
135.325714111328 -12.2788982391357\\
135.375717163086 -12.2799549102783\\
135.425704956055 -12.2809629440308\\
135.475708007813 -12.281928062439\\
135.52571105957 -12.2828941345215\\
135.575714111328 -12.2844266891479\\
135.625717163086 -12.2860593795776\\
135.675704956055 -12.2878036499023\\
135.725708007813 -12.2895259857178\\
135.77571105957 -12.2912569046021\\
135.825714111328 -12.29274559021\\
135.875717163086 -12.2942028045654\\
135.925704956055 -12.2953367233276\\
135.975708007813 -12.2961130142212\\
136.02571105957 -12.2968988418579\\
136.075714111328 -12.2982168197632\\
136.125717163086 -12.2994842529297\\
136.175704956055 -12.3007345199585\\
136.225708007813 -12.3021717071533\\
136.27571105957 -12.3040132522583\\
136.322280883789 -12.3057537078857\\
136.370239257813 -12.3073253631592\\
136.42024230957 -12.3091459274292\\
136.470245361328 -12.3112592697144\\
136.520248413086 -12.3135604858398\\
136.570251464844 -12.3159160614014\\
136.620239257813 -12.3180027008057\\
136.67024230957 -12.3197574615479\\
136.720245361328 -12.3214454650879\\
136.770248413086 -12.3231248855591\\
136.820251464844 -12.3248167037964\\
136.870239257813 -12.3265037536621\\
136.92024230957 -12.3281803131104\\
136.970245361328 -12.3298349380493\\
137.020248413086 -12.3314371109009\\
137.070251464844 -12.3329877853394\\
137.120239257813 -12.3344745635986\\
137.17024230957 -12.3358869552612\\
137.220245361328 -12.3371114730835\\
137.270248413086 -12.3383512496948\\
137.320251464844 -12.3397245407104\\
137.370239257813 -12.3413391113281\\
137.42024230957 -12.3432445526123\\
137.470245361328 -12.3452568054199\\
137.520248413086 -12.3468561172485\\
137.570251464844 -12.34850025177\\
137.620239257813 -12.3505077362061\\
137.67024230957 -12.352897644043\\
137.720245361328 -12.3552083969116\\
137.770248413086 -12.3575029373169\\
137.820251464844 -12.3593502044678\\
137.870239257813 -12.3609952926636\\
137.92024230957 -12.3627586364746\\
137.970245361328 -12.3645219802856\\
138.020248413086 -12.3662252426147\\
138.070251464844 -12.367618560791\\
138.120239257813 -12.3690881729126\\
138.17024230957 -12.3708915710449\\
138.220245361328 -12.3732109069824\\
138.270248413086 -12.3760404586792\\
138.320251464844 -12.3788642883301\\
138.370239257813 -12.3821315765381\\
138.42024230957 -12.3854885101318\\
138.470245361328 -12.3889570236206\\
138.520248413086 -12.3927125930786\\
138.570251464844 -12.3965749740601\\
138.620239257813 -12.4000253677368\\
138.67024230957 -12.4035444259644\\
138.720245361328 -12.4072046279907\\
138.770248413086 -12.4106063842773\\
138.820251464844 -12.4138946533203\\
138.870239257813 -12.4174022674561\\
138.92024230957 -12.4207992553711\\
138.970245361328 -12.4242057800293\\
139.020248413086 -12.4277896881104\\
139.070251464844 -12.4313430786133\\
139.120239257813 -12.4350261688232\\
139.17024230957 -12.4388265609741\\
139.220245361328 -12.4425849914551\\
139.270248413086 -12.4463500976563\\
139.320251464844 -12.4500665664673\\
139.370239257813 -12.4537734985352\\
139.42024230957 -12.4574012756348\\
139.470245361328 -12.4609231948853\\
139.520248413086 -12.4643745422363\\
139.570251464844 -12.4677829742432\\
139.620239257813 -12.4710788726807\\
139.67024230957 -12.4742956161499\\
139.720245361328 -12.4774341583252\\
139.770248413086 -12.4804792404175\\
139.820251464844 -12.4834432601929\\
139.870239257813 -12.4863414764404\\
139.92024230957 -12.4891786575317\\
139.970245361328 -12.4919519424438\\
140.020248413086 -12.4946737289429\\
140.070251464844 -12.4973487854004\\
140.120239257813 -12.4999704360962\\
140.17024230957 -12.5025482177734\\
140.220245361328 -12.5050792694092\\
140.270248413086 -12.5075492858887\\
140.320251464844 -12.5099544525146\\
140.370239257813 -12.5122709274292\\
140.42024230957 -12.5145092010498\\
140.470245361328 -12.5166845321655\\
140.520248413086 -12.5188016891479\\
140.570251464844 -12.5208044052124\\
140.620239257813 -12.5228509902954\\
140.67024230957 -12.5248250961304\\
140.720245361328 -12.5267515182495\\
140.770248413086 -12.528603553772\\
140.820251464844 -12.53040599823\\
140.870239257813 -12.5321788787842\\
140.92024230957 -12.5339117050171\\
140.970245361328 -12.5355806350708\\
141.020248413086 -12.5371494293213\\
141.070251464844 -12.5386371612549\\
141.120239257813 -12.5400581359863\\
141.17024230957 -12.5414190292358\\
141.220245361328 -12.5427045822144\\
141.270248413086 -12.5437564849854\\
141.320251464844 -12.5444993972778\\
141.370239257813 -12.5451602935791\\
141.42024230957 -12.5459098815918\\
141.470245361328 -12.546778678894\\
141.520248413086 -12.547890663147\\
141.570251464844 -12.5494165420532\\
141.620239257813 -12.550838470459\\
141.67024230957 -12.5517845153809\\
141.720245361328 -12.5526275634766\\
141.770248413086 -12.5535163879395\\
141.820251464844 -12.5544509887695\\
141.870239257813 -12.555326461792\\
141.92024230957 -12.5558586120605\\
141.970245361328 -12.5565357208252\\
142.020248413086 -12.5572147369385\\
142.070251464844 -12.5578269958496\\
142.120239257813 -12.558403968811\\
142.17024230957 -12.5590400695801\\
142.220245361328 -12.5596513748169\\
142.270248413086 -12.5603132247925\\
142.320251464844 -12.5609302520752\\
142.370239257813 -12.5615978240967\\
142.42024230957 -12.562180519104\\
142.470245361328 -12.5626573562622\\
142.520248413086 -12.5632228851318\\
142.570251464844 -12.5638904571533\\
142.620239257813 -12.5644550323486\\
142.67024230957 -12.5651216506958\\
142.720245361328 -12.5659914016724\\
142.770248413086 -12.5669298171997\\
142.820251464844 -12.568042755127\\
142.870239257813 -12.5699148178101\\
142.92024230957 -12.570972442627\\
142.970245361328 -12.5715293884277\\
143.020248413086 -12.5722036361694\\
143.070251464844 -12.5731630325317\\
143.120239257813 -12.5742378234863\\
143.17024230957 -12.5749950408936\\
143.220245361328 -12.5761909484863\\
143.270248413086 -12.5780344009399\\
143.320251464844 -12.5804357528687\\
143.370239257813 -12.5833311080933\\
143.42024230957 -12.5863094329834\\
143.470245361328 -12.5884580612183\\
143.520248413086 -12.5907611846924\\
143.570251464844 -12.593337059021\\
143.620239257813 -12.5963525772095\\
143.67024230957 -12.5997867584229\\
143.720245361328 -12.6033554077148\\
143.770248413086 -12.6061105728149\\
143.820251464844 -12.6066226959229\\
143.870239257813 -12.6078777313232\\
143.92024230957 -12.6098899841309\\
143.970245361328 -12.6125555038452\\
144.020248413086 -12.6153926849365\\
144.070251464844 -12.6171255111694\\
144.120239257813 -12.6185903549194\\
144.17024230957 -12.620719909668\\
144.220245361328 -12.6233577728271\\
144.270248413086 -12.6262521743774\\
144.320251464844 -12.6288051605225\\
144.370239257813 -12.6312313079834\\
144.42024230957 -12.6340112686157\\
144.470245361328 -12.636999130249\\
144.520248413086 -12.6398143768311\\
144.570251464844 -12.6423149108887\\
144.620239257813 -12.6450080871582\\
144.67024230957 -12.6478967666626\\
144.720245361328 -12.6506900787354\\
144.770248413086 -12.6531419754028\\
144.820251464844 -12.6558017730713\\
144.870239257813 -12.6585960388184\\
144.92024230957 -12.6613054275513\\
144.970245361328 -12.6638793945313\\
145.020248413086 -12.6665258407593\\
145.070251464844 -12.669093132019\\
145.120239257813 -12.6716413497925\\
145.17024230957 -12.6741714477539\\
145.220245361328 -12.676682472229\\
145.270248413086 -12.6791677474976\\
145.320251464844 -12.6816110610962\\
145.370239257813 -12.6840286254883\\
145.42024230957 -12.6864099502563\\
145.470245361328 -12.6887407302856\\
145.520248413086 -12.6910209655762\\
145.570251464844 -12.6932535171509\\
145.620239257813 -12.6954317092896\\
145.67024230957 -12.6975536346436\\
145.720245361328 -12.6996212005615\\
145.770248413086 -12.7016324996948\\
145.820251464844 -12.7035913467407\\
145.870239257813 -12.7054977416992\\
145.92024230957 -12.7073612213135\\
145.970245361328 -12.7091827392578\\
146.020248413086 -12.7109670639038\\
146.070251464844 -12.7127132415771\\
146.120239257813 -12.7144212722778\\
146.17024230957 -12.7160921096802\\
146.220245361328 -12.7177257537842\\
146.270248413086 -12.7193202972412\\
146.320251464844 -12.7208786010742\\
146.370239257813 -12.7224006652832\\
146.42024230957 -12.7238845825195\\
146.470245361328 -12.7253389358521\\
146.520248413086 -12.7267618179321\\
146.570251464844 -12.7281503677368\\
146.620239257813 -12.7295036315918\\
146.67024230957 -12.73082447052\\
146.720245361328 -12.7321071624756\\
146.770248413086 -12.7333583831787\\
146.820251464844 -12.7345733642578\\
146.870239257813 -12.7357540130615\\
146.92024230957 -12.7369108200073\\
146.970245361328 -12.7380466461182\\
147.020248413086 -12.7391605377197\\
147.070251464844 -12.7402544021606\\
147.120239257813 -12.7413244247437\\
147.17024230957 -12.7423706054688\\
147.220245361328 -12.7433881759644\\
147.270248413086 -12.7443780899048\\
147.320251464844 -12.7453355789185\\
147.370239257813 -12.7462549209595\\
147.42024230957 -12.7471351623535\\
147.470245361328 -12.7479848861694\\
147.520248413086 -12.7488088607788\\
147.570251464844 -12.7496089935303\\
147.620239257813 -12.7503852844238\\
147.67024230957 -12.7511405944824\\
147.720245361328 -12.7518739700317\\
147.770248413086 -12.7525873184204\\
147.820251464844 -12.7532815933228\\
147.870239257813 -12.7539548873901\\
147.92024230957 -12.7546119689941\\
147.970245361328 -12.7552509307861\\
148.020248413086 -12.7558746337891\\
148.070251464844 -12.7564830780029\\
148.120239257813 -12.757080078125\\
148.17024230957 -12.7576665878296\\
148.220245361328 -12.7582406997681\\
148.270248413086 -12.7588033676147\\
148.320251464844 -12.7593536376953\\
148.370239257813 -12.7598896026611\\
148.42024230957 -12.7604112625122\\
148.470245361328 -12.7609205245972\\
148.520248413086 -12.7614135742188\\
148.570251464844 -12.7618932723999\\
148.620239257813 -12.7623615264893\\
148.67024230957 -12.7628211975098\\
148.720245361328 -12.7632789611816\\
148.770248413086 -12.7637300491333\\
148.820251464844 -12.764178276062\\
148.870239257813 -12.7646198272705\\
148.92024230957 -12.7650585174561\\
148.970245361328 -12.7654905319214\\
149.020248413086 -12.7659120559692\\
149.070251464844 -12.7663192749023\\
149.120239257813 -12.7667112350464\\
149.17024230957 -12.76708984375\\
149.220245361328 -12.7674570083618\\
149.270248413086 -12.7678079605103\\
149.320251464844 -12.7681407928467\\
149.370239257813 -12.7684383392334\\
149.42024230957 -12.7687215805054\\
149.470245361328 -12.7690029144287\\
149.520248413086 -12.7692985534668\\
149.570251464844 -12.7695846557617\\
149.620239257813 -12.7698707580566\\
149.67024230957 -12.7701511383057\\
149.720245361328 -12.7704277038574\\
149.770248413086 -12.7707118988037\\
149.820251464844 -12.7709980010986\\
149.870239257813 -12.7712669372559\\
149.92024230957 -12.7715244293213\\
149.970245361328 -12.7717914581299\\
};
\addlegendentry{$\dtmax{} = \expneg{5.0}{3}{[s]}$};

\addplot [
color=red,
solid
]
table[row sep=crcr]{
0 0\\
1.00109052658081 0\\
2.00109052658081 0\\
3.00109052658081 0\\
4.00109052658081 0\\
5.00109052658081 0\\
6.00109052658081 0\\
7.00109052658081 0\\
8.00109100341797 0\\
9.00109100341797 0\\
10.001091003418 0\\
11.001091003418 0\\
12.001091003418 0\\
13.001091003418 0\\
14.001091003418 0\\
15.001091003418 0\\
16.001091003418 0\\
17.001091003418 0\\
18.001091003418 0\\
19.001091003418 0\\
20.001091003418 -6.65302991365024e-07\\
21.0026092529297 -0.557377398014069\\
22.0118007659912 -0.666934370994568\\
23.0227470397949 -0.702873289585114\\
24.0352840423584 -0.728023946285248\\
25.0375232696533 -0.751092374324799\\
26.0375232696533 -0.77983957529068\\
27.0375232696533 -0.81791889667511\\
28.0375232696533 -0.864790439605713\\
29.0375232696533 -0.923611283302307\\
30.0375232696533 -0.998440742492676\\
31.0375232696533 -1.09425795078278\\
32.0375213623047 -1.21107649803162\\
33.0375213623047 -1.33789312839508\\
34.0375213623047 -1.46984040737152\\
35.0000228881836 -1.59977650642395\\
36.0000228881836 -1.73227119445801\\
37.0000228881836 -1.86194396018982\\
38.0000228881836 -1.98691952228546\\
39.0000228881836 -2.11113858222961\\
40.0000228881836 -2.23601961135864\\
41.0000228881836 -2.36051321029663\\
42.0000228881836 -2.48407363891602\\
43.0000228881836 -2.60807824134827\\
44.0000228881836 -2.73510217666626\\
45.0000228881836 -2.86274552345276\\
46.0000228881836 -2.98958373069763\\
47.0000228881836 -3.11151146888733\\
48.0000228881836 -3.22838759422302\\
49.0000228881836 -3.34346032142639\\
50.0000228881836 -3.45747780799866\\
51.0000228881836 -3.56964945793152\\
52.0000228881836 -3.68006587028503\\
53.0000228881836 -3.78468751907349\\
54.0000228881836 -3.88312911987305\\
55.0000228881836 -3.98231792449951\\
56.0000228881836 -4.07981634140015\\
57.0000228881836 -4.17745923995972\\
58.0000228881836 -4.27527332305908\\
59.0000228881836 -4.37253761291504\\
60.0000228881836 -4.47092294692993\\
61.0000228881836 -4.57194328308105\\
62.0000228881836 -4.67724657058716\\
63.0000228881836 -4.78537893295288\\
64.0000228881836 -4.89603233337402\\
65.0000228881836 -5.01360273361206\\
66.0000228881836 -5.12178182601929\\
67.0000228881836 -5.21634721755981\\
68.0000228881836 -5.31139612197876\\
69.0000228881836 -5.40564918518066\\
70.0000228881836 -5.50232315063477\\
71.0000228881836 -5.60091924667358\\
72.0000228881836 -5.69982433319092\\
73.0000228881836 -5.79918813705444\\
74.0000228881836 -5.90180158615112\\
75.0000228881836 -6.00276231765747\\
76.0000228881836 -6.1054892539978\\
77.0000228881836 -6.21007537841797\\
78.0000228881836 -6.32072830200195\\
79.0000228881836 -6.43704986572266\\
80.0000228881836 -6.55582714080811\\
81.0000228881836 -6.6712851524353\\
82.0000228881836 -6.77889680862427\\
83.0000228881836 -6.88384246826172\\
84.0000228881836 -6.99023532867432\\
85.0000228881836 -7.09360456466675\\
86.0000228881836 -7.19576930999756\\
87.0000228881836 -7.29415559768677\\
88.0000228881836 -7.39023876190186\\
89.0000228881836 -7.48512125015259\\
90.0000228881836 -7.58003664016724\\
91.0000228881836 -7.67383098602295\\
92.0000228881836 -7.76947546005249\\
93.0000228881836 -7.86358737945557\\
94.0000228881836 -7.95499849319458\\
95.0000228881836 -8.04429721832275\\
96.0000228881836 -8.13333225250244\\
97.0000228881836 -8.22021770477295\\
98.0000228881836 -8.30066108703613\\
99.0000228881836 -8.37530136108398\\
100 -8.44563102722168\\
100.025001525879 -8.44741249084473\\
100.050003051758 -8.44918251037598\\
100.074996948242 -8.45093059539795\\
100.099998474121 -8.45266532897949\\
100.125 -8.45439052581787\\
100.150001525879 -8.4561071395874\\
100.175003051758 -8.45782089233398\\
100.199996948242 -8.45952892303467\\
100.224998474121 -8.46123504638672\\
100.25 -8.46293640136719\\
100.275001525879 -8.46463012695313\\
100.300003051758 -8.46631717681885\\
100.324996948242 -8.46799278259277\\
100.349998474121 -8.4696569442749\\
100.375 -8.47133350372314\\
100.400001525879 -8.47299957275391\\
100.425003051758 -8.47466659545898\\
100.449996948242 -8.47634506225586\\
100.474998474121 -8.47803115844727\\
100.5 -8.47970867156982\\
100.525001525879 -8.48137092590332\\
100.550003051758 -8.48301315307617\\
100.574996948242 -8.4846363067627\\
100.599998474121 -8.48624610900879\\
100.625 -8.48784637451172\\
100.650001525879 -8.48943710327148\\
100.675003051758 -8.49102020263672\\
100.699996948242 -8.49260330200195\\
100.724998474121 -8.4941987991333\\
100.75 -8.49581241607666\\
100.775001525879 -8.49744701385498\\
100.800003051758 -8.499098777771\\
100.824996948242 -8.50076389312744\\
100.849998474121 -8.50244617462158\\
100.875 -8.50413799285889\\
100.900001525879 -8.50584888458252\\
100.925003051758 -8.50757694244385\\
100.949996948242 -8.50931739807129\\
100.974998474121 -8.51106643676758\\
101 -8.51282119750977\\
101.025001525879 -8.51457500457764\\
101.050003051758 -8.51632404327393\\
101.074996948242 -8.51806640625\\
101.099998474121 -8.51979923248291\\
101.125 -8.52151393890381\\
101.150001525879 -8.52320384979248\\
101.175003051758 -8.52487373352051\\
101.199996948242 -8.52653121948242\\
101.224998474121 -8.52817916870117\\
101.25 -8.52981853485107\\
101.275001525879 -8.53145599365234\\
101.300003051758 -8.53309154510498\\
101.324996948242 -8.53472137451172\\
101.349998474121 -8.53634548187256\\
101.375 -8.5379581451416\\
101.400001525879 -8.53958034515381\\
101.425003051758 -8.5412015914917\\
101.449996948242 -8.54281616210938\\
101.474998474121 -8.54443454742432\\
101.5 -8.54605960845947\\
101.525001525879 -8.54770565032959\\
101.550003051758 -8.54934501647949\\
101.574996948242 -8.55097103118896\\
101.599998474121 -8.55258464813232\\
101.625 -8.55418395996094\\
101.650001525879 -8.55576801300049\\
101.675003051758 -8.55734062194824\\
101.699996948242 -8.55890464782715\\
101.724998474121 -8.56045818328857\\
101.75 -8.56200218200684\\
101.775001525879 -8.5635461807251\\
101.800003051758 -8.56509685516357\\
101.824996948242 -8.56665802001953\\
101.849998474121 -8.56824493408203\\
101.875 -8.56986808776855\\
101.900001525879 -8.57151889801025\\
101.925003051758 -8.57318305969238\\
101.949996948242 -8.57485198974609\\
101.974998474121 -8.57652759552002\\
102 -8.57820701599121\\
102.025001525879 -8.5798921585083\\
102.050003051758 -8.58157920837402\\
102.074996948242 -8.58326721191406\\
102.099998474121 -8.5849552154541\\
102.125 -8.58664035797119\\
102.150001525879 -8.58831977844238\\
102.175003051758 -8.58999443054199\\
102.199996948242 -8.59166145324707\\
102.224998474121 -8.59331798553467\\
102.25 -8.5949649810791\\
102.275001525879 -8.59660339355469\\
102.300003051758 -8.59823417663574\\
102.324996948242 -8.59986305236816\\
102.349998474121 -8.60149002075195\\
102.375 -8.60310840606689\\
102.400001525879 -8.60471820831299\\
102.425003051758 -8.60635089874268\\
102.449996948242 -8.60799026489258\\
102.474998474121 -8.60962390899658\\
102.5 -8.61127090454102\\
102.525001525879 -8.61290836334229\\
102.550003051758 -8.61455821990967\\
102.574996948242 -8.61622714996338\\
102.599998474121 -8.61789417266846\\
102.625 -8.61955738067627\\
102.650001525879 -8.62120723724365\\
102.675003051758 -8.62284374237061\\
102.699996948242 -8.62446975708008\\
102.724998474121 -8.62608528137207\\
102.75 -8.62769508361816\\
102.775001525879 -8.62929916381836\\
102.800003051758 -8.63089179992676\\
102.824996948242 -8.63247966766357\\
102.849998474121 -8.63406085968018\\
102.875 -8.63564205169678\\
102.900001525879 -8.63722705841064\\
102.925003051758 -8.63882064819336\\
102.949996948242 -8.64043140411377\\
102.974998474121 -8.64206027984619\\
103 -8.64370250701904\\
103.025001525879 -8.64535045623779\\
103.050003051758 -8.64700984954834\\
103.074996948242 -8.6486759185791\\
103.099998474121 -8.65034484863281\\
103.125 -8.65201950073242\\
103.150001525879 -8.65369415283203\\
103.175003051758 -8.65536880493164\\
103.199996948242 -8.65703964233398\\
103.224998474121 -8.65870761871338\\
103.25 -8.66037273406982\\
103.275001525879 -8.66203880310059\\
103.300003051758 -8.66370487213135\\
103.324996948242 -8.66537380218506\\
103.349998474121 -8.66704273223877\\
103.375 -8.66871547698975\\
103.400001525879 -8.67039108276367\\
103.425003051758 -8.6720666885376\\
103.449996948242 -8.67374134063721\\
103.474998474121 -8.67541885375977\\
103.5 -8.67709445953369\\
103.525001525879 -8.67876720428467\\
103.550003051758 -8.68043422698975\\
103.574996948242 -8.68211555480957\\
103.599998474121 -8.68378829956055\\
103.625 -8.68545913696289\\
103.650001525879 -8.6871509552002\\
103.675003051758 -8.68884658813477\\
103.699996948242 -8.69053554534912\\
103.724998474121 -8.69221115112305\\
103.75 -8.69387340545654\\
103.775001525879 -8.69552421569824\\
103.800003051758 -8.69716644287109\\
103.824996948242 -8.69879627227783\\
103.849998474121 -8.70041561126709\\
103.875 -8.70202159881592\\
103.900001525879 -8.70361423492432\\
103.925003051758 -8.7052001953125\\
103.949996948242 -8.70678234100342\\
103.974998474121 -8.7083625793457\\
104 -8.70994281768799\\
104.025001525879 -8.71152782440186\\
104.050003051758 -8.71312141418457\\
104.074996948242 -8.71472454071045\\
104.099998474121 -8.71634197235107\\
104.125 -8.71796607971191\\
104.150001525879 -8.7195987701416\\
104.175003051758 -8.72123432159424\\
104.199996948242 -8.72286701202393\\
104.224998474121 -8.72450637817383\\
104.25 -8.726149559021\\
104.275001525879 -8.72779655456543\\
104.300003051758 -8.7294454574585\\
104.324996948242 -8.73109912872314\\
104.349998474121 -8.73275566101074\\
104.375 -8.73441219329834\\
104.400001525879 -8.73606300354004\\
104.425003051758 -8.73771190643311\\
104.449996948242 -8.73935604095459\\
104.474998474121 -8.74099636077881\\
104.5 -8.74264812469482\\
104.525001525879 -8.74430656433105\\
104.550003051758 -8.7459659576416\\
104.574996948242 -8.74762344360352\\
104.599998474121 -8.74927997589111\\
104.625 -8.75093555450439\\
104.650001525879 -8.75258445739746\\
104.675003051758 -8.75422668457031\\
104.699996948242 -8.755859375\\
104.724998474121 -8.75748825073242\\
104.75 -8.75912570953369\\
104.775001525879 -8.76076412200928\\
104.800003051758 -8.76240253448486\\
104.824996948242 -8.76403522491455\\
104.849998474121 -8.76566028594971\\
104.875 -8.76727676391602\\
104.900001525879 -8.76888179779053\\
104.925003051758 -8.77047348022461\\
104.949996948242 -8.77205562591553\\
104.974998474121 -8.77363204956055\\
105 -8.77519989013672\\
105.025001525879 -8.77676010131836\\
105.050003051758 -8.77831268310547\\
105.074996948242 -8.77985858917236\\
105.099998474121 -8.78140068054199\\
105.125 -8.78293800354004\\
105.150001525879 -8.78447341918945\\
105.175003051758 -8.78600883483887\\
105.199996948242 -8.78754901885986\\
105.224998474121 -8.78909111022949\\
105.25 -8.79063701629639\\
105.275001525879 -8.79219245910645\\
105.300003051758 -8.79375457763672\\
105.324996948242 -8.79532051086426\\
105.349998474121 -8.79689121246338\\
105.375 -8.79846477508545\\
105.400001525879 -8.80004024505615\\
105.425003051758 -8.80162048339844\\
105.449996948242 -8.80319976806641\\
105.474998474121 -8.80478000640869\\
105.5 -8.80636024475098\\
105.525001525879 -8.80794048309326\\
105.550003051758 -8.80952072143555\\
105.574996948242 -8.81109809875488\\
105.599998474121 -8.81267166137695\\
105.625 -8.81424522399902\\
105.650001525879 -8.81581878662109\\
105.675003051758 -8.81739234924316\\
105.699996948242 -8.81896305084229\\
105.724998474121 -8.82052898406982\\
105.75 -8.82209873199463\\
105.775001525879 -8.82367420196533\\
105.800003051758 -8.8252592086792\\
105.824996948242 -8.8268404006958\\
105.849998474121 -8.82841396331787\\
105.875 -8.82998752593994\\
105.900001525879 -8.83155345916748\\
105.925003051758 -8.83311080932617\\
105.949996948242 -8.83466529846191\\
105.974998474121 -8.83621692657471\\
106 -8.83776187896729\\
106.025001525879 -8.83930110931396\\
106.050003051758 -8.8408317565918\\
106.074996948242 -8.84235286712646\\
106.099998474121 -8.84386730194092\\
106.125 -8.84537410736084\\
106.150001525879 -8.84687995910645\\
106.175003051758 -8.84837818145752\\
106.199996948242 -8.84987258911133\\
106.224998474121 -8.85135936737061\\
106.25 -8.85284328460693\\
106.275001525879 -8.85432147979736\\
106.300003051758 -8.85579967498779\\
106.324996948242 -8.85727405548096\\
106.349998474121 -8.85874938964844\\
106.375 -8.86022472381592\\
106.400001525879 -8.8617000579834\\
106.425003051758 -8.86317825317383\\
106.449996948242 -8.86465644836426\\
106.474998474121 -8.86613750457764\\
106.5 -8.86762237548828\\
106.525001525879 -8.86910724639893\\
106.550003051758 -8.87059497833252\\
106.574996948242 -8.87208271026611\\
106.599998474121 -8.87357044219971\\
106.625 -8.8750581741333\\
106.650001525879 -8.87654590606689\\
106.675003051758 -8.87803649902344\\
106.699996948242 -8.87953186035156\\
106.724998474121 -8.88103294372559\\
106.75 -8.88254261016846\\
106.775001525879 -8.88405609130859\\
106.800003051758 -8.88557052612305\\
106.824996948242 -8.88708686828613\\
106.849998474121 -8.88860893249512\\
106.875 -8.89013195037842\\
106.900001525879 -8.89165496826172\\
106.925003051758 -8.89317226409912\\
106.949996948242 -8.89468860626221\\
106.974998474121 -8.89620876312256\\
107 -8.89772605895996\\
107.025001525879 -8.8992395401001\\
107.050003051758 -8.90074920654297\\
107.074996948242 -8.90225028991699\\
107.099998474121 -8.90374565124512\\
107.125 -8.90523338317871\\
107.150001525879 -8.90671634674072\\
107.175003051758 -8.90819358825684\\
107.199996948242 -8.90966510772705\\
107.224998474121 -8.91113185882568\\
107.25 -8.91259288787842\\
107.275001525879 -8.91405200958252\\
107.300003051758 -8.91550922393799\\
107.324996948242 -8.91696071624756\\
107.349998474121 -8.91840744018555\\
107.375 -8.919846534729\\
107.400001525879 -8.92127895355225\\
107.425003051758 -8.92270755767822\\
107.449996948242 -8.92413425445557\\
107.474998474121 -8.92556190490723\\
107.5 -8.92698955535889\\
107.525001525879 -8.92841720581055\\
107.550003051758 -8.92984485626221\\
107.574996948242 -8.93127155303955\\
107.599998474121 -8.93269920349121\\
107.625 -8.93412399291992\\
107.650001525879 -8.93554496765137\\
107.675003051758 -8.93696784973145\\
107.699996948242 -8.93839454650879\\
107.724998474121 -8.9398250579834\\
107.75 -8.94125556945801\\
107.775001525879 -8.94268321990967\\
107.800003051758 -8.94411087036133\\
107.824996948242 -8.94553852081299\\
107.849998474121 -8.94697093963623\\
107.875 -8.94840908050537\\
107.900001525879 -8.94985198974609\\
107.925003051758 -8.95130252838135\\
107.949996948242 -8.95276069641113\\
107.974998474121 -8.95421981811523\\
108 -8.95568370819092\\
108.025001525879 -8.95715045928955\\
108.050003051758 -8.95861911773682\\
108.074996948242 -8.96009254455566\\
108.099998474121 -8.96157073974609\\
108.125 -8.96304893493652\\
108.150001525879 -8.96452713012695\\
108.175003051758 -8.96600532531738\\
108.199996948242 -8.96748065948486\\
108.224998474121 -8.96894931793213\\
108.25 -8.97041320800781\\
108.275001525879 -8.9718713760376\\
108.300003051758 -8.97332382202148\\
108.324996948242 -8.97477054595947\\
108.349998474121 -8.97621059417725\\
108.375 -8.97764587402344\\
108.400001525879 -8.97907543182373\\
108.425003051758 -8.98050117492676\\
108.449996948242 -8.98192119598389\\
108.474998474121 -8.98333644866943\\
108.5 -8.98474788665771\\
108.525001525879 -8.986159324646\\
108.550003051758 -8.98756790161133\\
108.574996948242 -8.98897075653076\\
108.599998474121 -8.99037170410156\\
108.625 -8.99177074432373\\
108.650001525879 -8.99316310882568\\
108.675003051758 -8.99455547332764\\
108.699996948242 -8.99594402313232\\
108.724998474121 -8.99733352661133\\
108.75 -8.99872303009033\\
108.775001525879 -9.00010776519775\\
108.800003051758 -9.00149059295654\\
108.824996948242 -9.00287342071533\\
108.849998474121 -9.00425720214844\\
108.875 -9.00564670562744\\
108.900001525879 -9.00703716278076\\
108.925003051758 -9.00842952728271\\
108.949996948242 -9.00982189178467\\
108.974998474121 -9.01121711730957\\
109 -9.01261615753174\\
109.025001525879 -9.01401996612549\\
109.050003051758 -9.01543045043945\\
109.074996948242 -9.01684951782227\\
109.099998474121 -9.01827716827393\\
109.125 -9.01970767974854\\
109.150001525879 -9.02114200592041\\
109.175003051758 -9.02258014678955\\
109.199996948242 -9.02402019500732\\
109.224998474121 -9.02546882629395\\
109.25 -9.02691841125488\\
109.275001525879 -9.02836799621582\\
109.300003051758 -9.02981758117676\\
109.324996948242 -9.03126621246338\\
109.349998474121 -9.03271293640137\\
109.375 -9.03415298461914\\
109.400001525879 -9.03558826446533\\
109.425003051758 -9.03701877593994\\
109.449996948242 -9.03844165802002\\
109.474998474121 -9.03986167907715\\
109.5 -9.04127883911133\\
109.525001525879 -9.04269027709961\\
109.550003051758 -9.04409503936768\\
109.574996948242 -9.04549407958984\\
109.599998474121 -9.04688739776611\\
109.625 -9.04827785491943\\
109.650001525879 -9.04966354370117\\
109.675003051758 -9.05104541778564\\
109.699996948242 -9.05242156982422\\
109.724998474121 -9.05379486083984\\
109.75 -9.05516529083252\\
109.775001525879 -9.0565357208252\\
109.800003051758 -9.05790138244629\\
109.824996948242 -9.05926513671875\\
109.849998474121 -9.06062889099121\\
109.875 -9.06199264526367\\
109.900001525879 -9.06335639953613\\
109.925003051758 -9.06471729278564\\
109.949996948242 -9.06608104705811\\
109.974998474121 -9.06744480133057\\
110 -9.06881332397461\\
110.025001525879 -9.0701847076416\\
110.050003051758 -9.07155799865723\\
110.074996948242 -9.07293701171875\\
110.099998474121 -9.07431888580322\\
110.125 -9.07570552825928\\
110.150001525879 -9.07709884643555\\
110.175003051758 -9.07849788665771\\
110.199996948242 -9.07989978790283\\
110.224998474121 -9.08130645751953\\
110.25 -9.08271789550781\\
110.275001525879 -9.08413696289063\\
110.300003051758 -9.08556175231934\\
110.324996948242 -9.08699131011963\\
110.349998474121 -9.08842182159424\\
110.375 -9.08985233306885\\
110.400001525879 -9.09128284454346\\
110.425003051758 -9.09271049499512\\
110.449996948242 -9.09413719177246\\
110.474998474121 -9.09555816650391\\
110.5 -9.09697818756104\\
110.525001525879 -9.09839630126953\\
110.550003051758 -9.09981060028076\\
110.574996948242 -9.10122203826904\\
110.599998474121 -9.10263061523438\\
110.625 -9.10403251647949\\
110.650001525879 -9.10543060302734\\
110.675003051758 -9.1068229675293\\
110.699996948242 -9.10820960998535\\
110.724998474121 -9.10959148406982\\
110.75 -9.11097145080566\\
110.775001525879 -9.11234474182129\\
110.800003051758 -9.11371612548828\\
110.824996948242 -9.11508178710938\\
110.849998474121 -9.11644458770752\\
110.875 -9.11780548095703\\
110.900001525879 -9.11916255950928\\
110.925003051758 -9.12051677703857\\
110.949996948242 -9.12187099456787\\
110.974998474121 -9.12322521209717\\
111 -9.12457942962646\\
111.025001525879 -9.12593841552734\\
111.050003051758 -9.12729930877686\\
111.074996948242 -9.12865924835205\\
111.099998474121 -9.13002109527588\\
111.125 -9.13138484954834\\
111.150001525879 -9.13275146484375\\
111.175003051758 -9.13412284851074\\
111.199996948242 -9.13549613952637\\
111.224998474121 -9.13687229156494\\
111.25 -9.138258934021\\
111.275001525879 -9.13965702056885\\
111.300003051758 -9.14105606079102\\
111.324996948242 -9.14245414733887\\
111.349998474121 -9.14385318756104\\
111.375 -9.1452522277832\\
111.400001525879 -9.14665126800537\\
111.425003051758 -9.14805030822754\\
111.449996948242 -9.14944839477539\\
111.474998474121 -9.15084838867188\\
111.5 -9.15224742889404\\
111.525001525879 -9.15364646911621\\
111.550003051758 -9.15504550933838\\
111.574996948242 -9.15644359588623\\
111.599998474121 -9.1578426361084\\
111.625 -9.15924167633057\\
111.650001525879 -9.16063499450684\\
111.675003051758 -9.16202735900879\\
111.699996948242 -9.16341686248779\\
111.724998474121 -9.1648063659668\\
111.75 -9.16619110107422\\
111.775001525879 -9.16757392883301\\
111.800003051758 -9.16895580291748\\
111.824996948242 -9.17033386230469\\
111.849998474121 -9.17170715332031\\
111.875 -9.17307949066162\\
111.900001525879 -9.1744499206543\\
111.925003051758 -9.17582035064697\\
111.949996948242 -9.17718505859375\\
111.974998474121 -9.17854881286621\\
112 -9.17991161346436\\
112.025001525879 -9.18127250671387\\
112.050003051758 -9.18263149261475\\
112.074996948242 -9.18398571014404\\
112.099998474121 -9.18533897399902\\
112.125 -9.18669033050537\\
112.150001525879 -9.1880350112915\\
112.175003051758 -9.18937873840332\\
112.199996948242 -9.19071865081787\\
112.224998474121 -9.19205951690674\\
112.25 -9.19339847564697\\
112.275001525879 -9.19473361968994\\
112.300003051758 -9.19606876373291\\
112.324996948242 -9.19740390777588\\
112.349998474121 -9.19874572753906\\
112.375 -9.20008754730225\\
112.400001525879 -9.20143222808838\\
112.425003051758 -9.20278167724609\\
112.449996948242 -9.20413494110107\\
112.474998474121 -9.20549201965332\\
112.5 -9.20685291290283\\
112.525001525879 -9.20821666717529\\
112.550003051758 -9.20958518981934\\
112.574996948242 -9.21096134185791\\
112.599998474121 -9.21234130859375\\
112.625 -9.21372318267822\\
112.650001525879 -9.21510601043701\\
112.675003051758 -9.2164888381958\\
112.699996948242 -9.21787166595459\\
112.724998474121 -9.21925449371338\\
112.75 -9.22064018249512\\
112.775001525879 -9.22202968597412\\
112.800003051758 -9.22341728210449\\
112.824996948242 -9.22479915618896\\
112.849998474121 -9.2261791229248\\
112.875 -9.22755336761475\\
112.900001525879 -9.22892570495605\\
112.925003051758 -9.23029041290283\\
112.949996948242 -9.23165225982666\\
112.974998474121 -9.23301029205322\\
113 -9.23436450958252\\
113.025001525879 -9.23571681976318\\
113.050003051758 -9.23706531524658\\
113.074996948242 -9.23840999603271\\
113.099998474121 -9.23975276947021\\
113.125 -9.2410945892334\\
113.150001525879 -9.24243640899658\\
113.175003051758 -9.24377822875977\\
113.199996948242 -9.24511432647705\\
113.224998474121 -9.24644947052002\\
113.25 -9.24778461456299\\
113.275001525879 -9.24911689758301\\
113.300003051758 -9.25044536590576\\
113.324996948242 -9.25177097320557\\
113.349998474121 -9.25310134887695\\
113.375 -9.25442695617676\\
113.400001525879 -9.25575256347656\\
113.425003051758 -9.25707817077637\\
113.449996948242 -9.25840377807617\\
113.474998474121 -9.25972938537598\\
113.5 -9.26105499267578\\
113.525001525879 -9.26237964630127\\
113.550003051758 -9.26370429992676\\
113.574996948242 -9.26502990722656\\
113.599998474121 -9.266357421875\\
113.625 -9.26768589019775\\
113.650001525879 -9.26901817321777\\
113.675003051758 -9.27035331726074\\
113.699996948242 -9.27169227600098\\
113.724998474121 -9.27303695678711\\
113.75 -9.27438640594482\\
113.775001525879 -9.27574062347412\\
113.800003051758 -9.27709484100342\\
113.824996948242 -9.27844905853271\\
113.849998474121 -9.27980327606201\\
113.875 -9.28116130828857\\
113.900001525879 -9.28252220153809\\
113.925003051758 -9.28387928009033\\
113.949996948242 -9.28523349761963\\
113.974998474121 -9.28658771514893\\
114 -9.28794193267822\\
114.025001525879 -9.28929424285889\\
114.050003051758 -9.29064464569092\\
114.074996948242 -9.29198932647705\\
114.099998474121 -9.29332828521729\\
114.125 -9.29466247558594\\
114.150001525879 -9.29599285125732\\
114.175003051758 -9.29731845855713\\
114.199996948242 -9.29863834381104\\
114.224998474121 -9.29995441436768\\
114.25 -9.30126762390137\\
114.275001525879 -9.30257511138916\\
114.300003051758 -9.30388164520264\\
114.324996948242 -9.30518627166748\\
114.349998474121 -9.30648994445801\\
114.375 -9.30778694152832\\
114.400001525879 -9.30908393859863\\
114.425003051758 -9.310378074646\\
114.449996948242 -9.31166744232178\\
114.474998474121 -9.31295394897461\\
114.5 -9.31423854827881\\
114.525001525879 -9.31551647186279\\
114.550003051758 -9.31679344177246\\
114.574996948242 -9.31807041168213\\
114.599998474121 -9.31934833526611\\
114.625 -9.32062435150146\\
114.650001525879 -9.3218994140625\\
114.675003051758 -9.32317543029785\\
114.699996948242 -9.32445335388184\\
114.724998474121 -9.32573127746582\\
114.75 -9.32700729370117\\
114.775001525879 -9.32828521728516\\
114.800003051758 -9.32956600189209\\
114.824996948242 -9.33084964752197\\
114.849998474121 -9.33213710784912\\
114.875 -9.33343315124512\\
114.900001525879 -9.3347339630127\\
114.925003051758 -9.33603954315186\\
114.949996948242 -9.33734703063965\\
114.974998474121 -9.33865451812744\\
115 -9.33996391296387\\
115.025001525879 -9.34127616882324\\
115.050003051758 -9.34258937835693\\
115.074996948242 -9.34390354156494\\
115.099998474121 -9.34521961212158\\
115.125 -9.34653568267822\\
115.150001525879 -9.34785175323486\\
115.175003051758 -9.34916591644287\\
115.199996948242 -9.35047245025635\\
115.224998474121 -9.35177707672119\\
115.25 -9.3530797958374\\
115.275001525879 -9.35437679290771\\
115.300003051758 -9.35566902160645\\
115.324996948242 -9.35695457458496\\
115.349998474121 -9.35823631286621\\
115.375 -9.3595142364502\\
115.400001525879 -9.3607873916626\\
115.425003051758 -9.36205577850342\\
115.449996948242 -9.36332225799561\\
115.474998474121 -9.36458778381348\\
115.5 -9.36584663391113\\
115.525001525879 -9.36710453033447\\
115.550003051758 -9.36835670471191\\
115.574996948242 -9.36960601806641\\
115.599998474121 -9.37085247039795\\
115.625 -9.37209224700928\\
115.650001525879 -9.37333011627197\\
115.675003051758 -9.37456321716309\\
115.699996948242 -9.37579345703125\\
115.724998474121 -9.37702178955078\\
115.75 -9.378249168396\\
115.775001525879 -9.37947177886963\\
115.800003051758 -9.38069248199463\\
115.824996948242 -9.38191890716553\\
115.849998474121 -9.38314723968506\\
115.875 -9.38437747955322\\
115.900001525879 -9.38560485839844\\
115.925003051758 -9.38683414459229\\
115.949996948242 -9.38806438446045\\
115.974998474121 -9.3892993927002\\
116 -9.39053916931152\\
116.025001525879 -9.39178466796875\\
116.050003051758 -9.39303302764893\\
116.074996948242 -9.39428615570068\\
116.099998474121 -9.39554500579834\\
116.125 -9.39680576324463\\
116.150001525879 -9.39807224273682\\
116.175003051758 -9.39934062957764\\
116.199996948242 -9.40061092376709\\
116.224998474121 -9.40188598632813\\
116.25 -9.40316104888916\\
116.275001525879 -9.40443706512451\\
116.300003051758 -9.40571308135986\\
116.324996948242 -9.40698337554932\\
116.349998474121 -9.40825176239014\\
116.375 -9.40952014923096\\
116.400001525879 -9.41078758239746\\
116.425003051758 -9.41204929351807\\
116.449996948242 -9.41330718994141\\
116.474998474121 -9.41455841064453\\
116.5 -9.41580581665039\\
116.525001525879 -9.41704750061035\\
116.550003051758 -9.41828536987305\\
116.574996948242 -9.41951751708984\\
116.599998474121 -9.42074680328369\\
116.625 -9.42197227478027\\
116.650001525879 -9.42319297790527\\
116.675003051758 -9.42441272735596\\
116.699996948242 -9.42562484741211\\
116.724998474121 -9.42683410644531\\
116.75 -9.42803859710693\\
116.775001525879 -9.42924022674561\\
116.800003051758 -9.43043899536133\\
116.824996948242 -9.43163108825684\\
116.849998474121 -9.43282318115234\\
116.875 -9.43401336669922\\
116.900001525879 -9.43520259857178\\
116.925003051758 -9.43639183044434\\
116.949996948242 -9.43758010864258\\
116.974998474121 -9.43877124786377\\
117 -9.43996334075928\\
117.025001525879 -9.44115543365479\\
117.050003051758 -9.44234943389893\\
117.074996948242 -9.44354820251465\\
117.099998474121 -9.44474983215332\\
117.125 -9.44595527648926\\
117.150001525879 -9.44716644287109\\
117.175003051758 -9.44838237762451\\
117.199996948242 -9.44960308074951\\
117.224998474121 -9.45083045959473\\
117.25 -9.45205974578857\\
117.275001525879 -9.45329284667969\\
117.300003051758 -9.4545316696167\\
117.324996948242 -9.45577144622803\\
117.349998474121 -9.45701122283936\\
117.375 -9.4582576751709\\
117.400001525879 -9.45950412750244\\
117.425003051758 -9.46075057983398\\
117.449996948242 -9.46199321746826\\
117.474998474121 -9.46323299407959\\
117.5 -9.46447277069092\\
117.525001525879 -9.46571063995361\\
117.550003051758 -9.46694374084473\\
117.574996948242 -9.46817207336426\\
117.599998474121 -9.46939468383789\\
117.625 -9.47061347961426\\
117.650001525879 -9.47182655334473\\
117.675003051758 -9.47303771972656\\
117.699996948242 -9.47424507141113\\
117.724998474121 -9.47544765472412\\
117.75 -9.47664356231689\\
117.775001525879 -9.47783374786377\\
117.800003051758 -9.47901630401611\\
117.824996948242 -9.48019695281982\\
117.849998474121 -9.48137664794922\\
117.875 -9.48255062103271\\
117.900001525879 -9.48372268676758\\
117.925003051758 -9.48489284515381\\
117.949996948242 -9.48605632781982\\
117.974998474121 -9.48722553253174\\
118 -9.48839569091797\\
118.025001525879 -9.4895658493042\\
118.050003051758 -9.49073600769043\\
118.074996948242 -9.49190330505371\\
118.099998474121 -9.49306678771973\\
118.125 -9.49423027038574\\
118.150001525879 -9.49539375305176\\
118.175003051758 -9.49655723571777\\
118.199996948242 -9.49772453308105\\
118.224998474121 -9.49889469146729\\
118.25 -9.50006771087646\\
118.275001525879 -9.50124645233154\\
118.300003051758 -9.50242805480957\\
118.324996948242 -9.50361061096191\\
118.349998474121 -9.50479602813721\\
118.375 -9.50598812103271\\
118.400001525879 -9.50718021392822\\
118.425003051758 -9.50837230682373\\
118.449996948242 -9.50956726074219\\
118.474998474121 -9.51076602935791\\
118.5 -9.51196765899658\\
118.525001525879 -9.51317024230957\\
118.550003051758 -9.51437854766846\\
118.574996948242 -9.51558017730713\\
118.599998474121 -9.5167818069458\\
118.625 -9.51798343658447\\
118.650001525879 -9.51918315887451\\
118.675003051758 -9.52037525177002\\
118.699996948242 -9.52156734466553\\
118.724998474121 -9.52275657653809\\
118.75 -9.52394199371338\\
118.775001525879 -9.52512454986572\\
118.800003051758 -9.52629852294922\\
118.824996948242 -9.52746963500977\\
118.849998474121 -9.52863502502441\\
118.875 -9.52979373931885\\
118.900001525879 -9.5309476852417\\
118.925003051758 -9.53209972381592\\
118.949996948242 -9.53324413299561\\
118.974998474121 -9.53438758850098\\
119 -9.53552627563477\\
119.025001525879 -9.53666114807129\\
119.050003051758 -9.5377950668335\\
119.074996948242 -9.53892707824707\\
119.099998474121 -9.54005336761475\\
119.125 -9.54117870330811\\
119.150001525879 -9.54230308532715\\
119.175003051758 -9.54342555999756\\
119.199996948242 -9.54454708099365\\
119.224998474121 -9.54566287994385\\
119.25 -9.54677867889404\\
119.275001525879 -9.54789924621582\\
119.300003051758 -9.54902172088623\\
119.324996948242 -9.55014705657959\\
119.349998474121 -9.55127239227295\\
119.375 -9.55240440368652\\
119.400001525879 -9.55353736877441\\
119.425003051758 -9.55467224121094\\
119.449996948242 -9.55581188201904\\
119.474998474121 -9.5569543838501\\
119.5 -9.55810165405273\\
119.525001525879 -9.55925559997559\\
119.550003051758 -9.56040954589844\\
119.574996948242 -9.56156730651855\\
119.599998474121 -9.56272888183594\\
119.625 -9.56389236450195\\
119.650001525879 -9.56505584716797\\
119.675003051758 -9.5662260055542\\
119.699996948242 -9.56739521026611\\
119.724998474121 -9.56856536865234\\
119.75 -9.56973266601563\\
119.775001525879 -9.57089900970459\\
119.800003051758 -9.5720682144165\\
119.824996948242 -9.57323169708252\\
119.849998474121 -9.57439422607422\\
119.875 -9.5755500793457\\
119.900001525879 -9.57670307159424\\
119.925003051758 -9.57785320281982\\
119.949996948242 -9.57899761199951\\
119.974998474121 -9.58013820648193\\
120 -9.58127307891846\\
120.025001525879 -9.58240413665771\\
120.050003051758 -9.58352947235107\\
120.074996948242 -9.58465385437012\\
120.099998474121 -9.58577632904053\\
120.125 -9.58689212799072\\
120.150001525879 -9.5880069732666\\
120.175003051758 -9.5891170501709\\
120.199996948242 -9.5902214050293\\
120.224998474121 -9.59132099151611\\
120.25 -9.59241771697998\\
120.275001525879 -9.59351253509521\\
120.300003051758 -9.59460639953613\\
120.324996948242 -9.59569358825684\\
120.349998474121 -9.59678077697754\\
120.375 -9.59786796569824\\
120.400001525879 -9.59895515441895\\
120.425003051758 -9.60004234313965\\
120.449996948242 -9.60113143920898\\
120.474998474121 -9.60222721099854\\
120.5 -9.6033296585083\\
120.525001525879 -9.6044340133667\\
120.550003051758 -9.60553073883057\\
120.574996948242 -9.60662746429443\\
120.599998474121 -9.6077241897583\\
120.625 -9.60882186889648\\
120.650001525879 -9.60992527008057\\
120.675003051758 -9.61102867126465\\
120.699996948242 -9.61213397979736\\
120.724998474121 -9.61324024200439\\
120.75 -9.61434650421143\\
120.775001525879 -9.61545944213867\\
120.800003051758 -9.61657524108887\\
120.824996948242 -9.61769104003906\\
120.849998474121 -9.61880683898926\\
120.875 -9.61992263793945\\
120.900001525879 -9.62103843688965\\
120.925003051758 -9.62215423583984\\
120.949996948242 -9.62327003479004\\
120.974998474121 -9.62438488006592\\
121 -9.62549686431885\\
121.025001525879 -9.62660312652588\\
121.050003051758 -9.62770366668701\\
121.074996948242 -9.62879943847656\\
121.099998474121 -9.62989139556885\\
121.125 -9.63097763061523\\
121.150001525879 -9.63205718994141\\
121.175003051758 -9.633131980896\\
121.199996948242 -9.63419914245605\\
121.224998474121 -9.63526153564453\\
121.25 -9.63631916046143\\
121.275001525879 -9.63737297058105\\
121.300003051758 -9.6384220123291\\
121.324996948242 -9.6394681930542\\
121.349998474121 -9.64050769805908\\
121.375 -9.64154720306396\\
121.400001525879 -9.64258575439453\\
121.425003051758 -9.6436185836792\\
121.449996948242 -9.64464855194092\\
121.474998474121 -9.64567852020264\\
121.5 -9.64670848846436\\
121.525001525879 -9.64773845672607\\
121.550003051758 -9.64876842498779\\
121.574996948242 -9.64979839324951\\
121.599998474121 -9.65082836151123\\
121.625 -9.65186500549316\\
121.650001525879 -9.6528959274292\\
121.675003051758 -9.65392971038818\\
121.699996948242 -9.65496730804443\\
121.724998474121 -9.656005859375\\
121.75 -9.65704536437988\\
121.775001525879 -9.6580867767334\\
121.800003051758 -9.6591329574585\\
121.824996948242 -9.66018009185791\\
121.849998474121 -9.66122817993164\\
121.875 -9.66227436065674\\
121.900001525879 -9.66331481933594\\
121.925003051758 -9.66435432434082\\
121.949996948242 -9.66539192199707\\
121.974998474121 -9.66642570495605\\
122 -9.66745567321777\\
122.025001525879 -9.66847896575928\\
122.050003051758 -9.66949939727783\\
122.074996948242 -9.67051696777344\\
122.099998474121 -9.67152690887451\\
122.125 -9.67252922058105\\
122.150001525879 -9.67352390289307\\
122.175003051758 -9.67451477050781\\
122.199996948242 -9.67549896240234\\
122.224998474121 -9.67648029327393\\
122.25 -9.67745780944824\\
122.275001525879 -9.67842960357666\\
122.300003051758 -9.67939567565918\\
122.324996948242 -9.68035888671875\\
122.349998474121 -9.681321144104\\
122.375 -9.68227767944336\\
122.400001525879 -9.68323135375977\\
122.425003051758 -9.68418502807617\\
122.449996948242 -9.68514060974121\\
122.474998474121 -9.68609428405762\\
122.5 -9.68704795837402\\
122.525001525879 -9.68800163269043\\
122.550003051758 -9.68895530700684\\
122.574996948242 -9.68990898132324\\
122.599998474121 -9.69086265563965\\
122.625 -9.69181632995605\\
122.650001525879 -9.69277000427246\\
122.675003051758 -9.69372367858887\\
122.699996948242 -9.69467735290527\\
122.724998474121 -9.69563102722168\\
122.75 -9.69658470153809\\
122.775001525879 -9.69753837585449\\
122.800003051758 -9.6984920501709\\
122.824996948242 -9.6994457244873\\
122.849998474121 -9.70039939880371\\
122.875 -9.70135307312012\\
122.900001525879 -9.70230674743652\\
122.925003051758 -9.70325946807861\\
122.949996948242 -9.70420932769775\\
122.974998474121 -9.70515441894531\\
123 -9.70609760284424\\
123.025001525879 -9.70703411102295\\
123.050003051758 -9.70796871185303\\
123.074996948242 -9.70889854431152\\
123.099998474121 -9.70983028411865\\
123.125 -9.71075630187988\\
123.150001525879 -9.71168041229248\\
123.175003051758 -9.71259593963623\\
123.199996948242 -9.71350479125977\\
123.224998474121 -9.71440505981445\\
123.25 -9.71530151367188\\
123.275001525879 -9.71619319915771\\
123.300003051758 -9.71708011627197\\
123.324996948242 -9.71796226501465\\
123.349998474121 -9.71883964538574\\
123.375 -9.71971702575684\\
123.400001525879 -9.72059059143066\\
123.425003051758 -9.72145843505859\\
123.449996948242 -9.72232627868652\\
123.474998474121 -9.72319412231445\\
123.5 -9.72406196594238\\
123.525001525879 -9.72492790222168\\
123.550003051758 -9.72579288482666\\
123.574996948242 -9.72665691375732\\
123.599998474121 -9.72752475738525\\
123.625 -9.72839260101318\\
123.650001525879 -9.72925758361816\\
123.675003051758 -9.73012256622314\\
123.699996948242 -9.73099040985107\\
123.724998474121 -9.731858253479\\
123.75 -9.73272609710693\\
123.775001525879 -9.73359394073486\\
123.800003051758 -9.73446178436279\\
123.824996948242 -9.73533153533936\\
123.849998474121 -9.7362060546875\\
123.875 -9.73708057403564\\
123.900001525879 -9.73795127868652\\
123.925003051758 -9.73881912231445\\
123.949996948242 -9.73968696594238\\
123.974998474121 -9.74055290222168\\
124 -9.74141311645508\\
124.025001525879 -9.74227142333984\\
124.050003051758 -9.74312973022461\\
124.074996948242 -9.74398040771484\\
124.099998474121 -9.74482822418213\\
124.125 -9.74566841125488\\
124.150001525879 -9.74650096893311\\
124.175003051758 -9.74733066558838\\
124.199996948242 -9.74815273284912\\
124.224998474121 -9.74897193908691\\
124.25 -9.74978160858154\\
124.275001525879 -9.75058269500732\\
124.300003051758 -9.75138187408447\\
124.324996948242 -9.75218105316162\\
124.349998474121 -9.7529878616333\\
124.375 -9.75378894805908\\
124.400001525879 -9.75457763671875\\
124.425003051758 -9.7553596496582\\
124.449996948242 -9.75613307952881\\
124.474998474121 -9.75690364837646\\
124.5 -9.75766849517822\\
124.525001525879 -9.75843143463135\\
124.550003051758 -9.75919437408447\\
124.574996948242 -9.75995349884033\\
124.599998474121 -9.76071548461914\\
124.625 -9.76147842407227\\
124.650001525879 -9.76224231719971\\
124.675003051758 -9.76301383972168\\
124.699996948242 -9.76378631591797\\
124.724998474121 -9.76456165313721\\
124.75 -9.76534175872803\\
124.775001525879 -9.76612567901611\\
124.800003051758 -9.76691627502441\\
124.824996948242 -9.76770973205566\\
124.849998474121 -9.76851081848145\\
124.875 -9.76931190490723\\
124.900001525879 -9.77011299133301\\
124.925003051758 -9.77091407775879\\
124.949996948242 -9.77171516418457\\
124.974998474121 -9.77251243591309\\
125 -9.77330303192139\\
125.025001525879 -9.77408504486084\\
125.050003051758 -9.77486038208008\\
125.074996948242 -9.77563190460205\\
125.099998474121 -9.77639579772949\\
125.125 -9.77715492248535\\
125.150001525879 -9.77790355682373\\
125.175003051758 -9.77864646911621\\
125.199996948242 -9.77938079833984\\
125.224998474121 -9.78011131286621\\
125.25 -9.78083515167236\\
125.275001525879 -9.78154850006104\\
125.300003051758 -9.78225421905518\\
125.324996948242 -9.78295993804932\\
125.349998474121 -9.78366374969482\\
125.375 -9.78436374664307\\
125.400001525879 -9.78505992889404\\
125.425003051758 -9.78575611114502\\
125.449996948242 -9.786452293396\\
125.474998474121 -9.78714847564697\\
125.5 -9.78784465789795\\
125.525001525879 -9.78854084014893\\
125.550003051758 -9.7892370223999\\
125.574996948242 -9.78993320465088\\
125.599998474121 -9.79062938690186\\
125.625 -9.79132556915283\\
125.650001525879 -9.79202175140381\\
125.675003051758 -9.79271793365479\\
125.699996948242 -9.79341411590576\\
125.724998474121 -9.79411602020264\\
125.75 -9.79482173919678\\
125.775001525879 -9.79552745819092\\
125.800003051758 -9.79623699188232\\
125.824996948242 -9.79695224761963\\
125.849998474121 -9.79766750335693\\
125.875 -9.79838275909424\\
125.900001525879 -9.79909801483154\\
125.925003051758 -9.79981327056885\\
125.949996948242 -9.80052852630615\\
125.974998474121 -9.80124378204346\\
126 -9.80195903778076\\
126.025001525879 -9.80267429351807\\
126.050003051758 -9.80338287353516\\
126.074996948242 -9.80408573150635\\
126.099998474121 -9.80478096008301\\
126.125 -9.80546855926514\\
126.150001525879 -9.80615329742432\\
126.175003051758 -9.80683040618896\\
126.199996948242 -9.80749797821045\\
126.224998474121 -9.8081579208374\\
126.25 -9.80881023406982\\
126.275001525879 -9.80945873260498\\
126.300003051758 -9.81010341644287\\
126.324996948242 -9.81074237823486\\
126.349998474121 -9.81138134002686\\
126.375 -9.81201839447021\\
126.400001525879 -9.81264877319336\\
126.425003051758 -9.81327819824219\\
126.449996948242 -9.81390953063965\\
126.474998474121 -9.81454753875732\\
126.5 -9.81518650054932\\
126.525001525879 -9.81582546234131\\
126.550003051758 -9.8164644241333\\
126.574996948242 -9.81710910797119\\
126.599998474121 -9.81775760650635\\
126.625 -9.8184061050415\\
126.650001525879 -9.81905937194824\\
126.675003051758 -9.81971740722656\\
126.699996948242 -9.82037544250488\\
126.724998474121 -9.82104015350342\\
126.75 -9.8217077255249\\
126.775001525879 -9.82237529754639\\
126.800003051758 -9.82304286956787\\
126.824996948242 -9.82371044158936\\
126.849998474121 -9.82437801361084\\
126.875 -9.82504558563232\\
126.900001525879 -9.82571220397949\\
126.925003051758 -9.82637214660645\\
126.949996948242 -9.82702922821045\\
126.974998474121 -9.82767963409424\\
127 -9.82832622528076\\
127.025001525879 -9.82896518707275\\
127.050003051758 -9.82959651947021\\
127.074996948242 -9.83022022247314\\
127.099998474121 -9.83083820343018\\
127.125 -9.83144855499268\\
127.150001525879 -9.83205509185791\\
127.175003051758 -9.83265590667725\\
127.199996948242 -9.8332576751709\\
127.224998474121 -9.8338680267334\\
127.25 -9.8344783782959\\
127.275001525879 -9.83508110046387\\
127.300003051758 -9.83567523956299\\
127.324996948242 -9.83626365661621\\
127.349998474121 -9.8368444442749\\
127.375 -9.83741664886475\\
127.400001525879 -9.83798885345459\\
127.425003051758 -9.83856105804443\\
127.449996948242 -9.83913326263428\\
127.474998474121 -9.83970642089844\\
127.5 -9.84028816223145\\
127.525001525879 -9.84086990356445\\
127.550003051758 -9.84146022796631\\
127.574996948242 -9.84205150604248\\
127.599998474121 -9.84265041351318\\
127.625 -9.84325504302979\\
127.650001525879 -9.84386539459229\\
127.675003051758 -9.84448337554932\\
127.699996948242 -9.84510326385498\\
127.724998474121 -9.84573173522949\\
127.75 -9.84636116027832\\
127.775001525879 -9.84699058532715\\
127.800003051758 -9.84762001037598\\
127.824996948242 -9.8482494354248\\
127.849998474121 -9.84887886047363\\
127.875 -9.84950733184814\\
127.900001525879 -9.85013008117676\\
127.925003051758 -9.85074996948242\\
127.949996948242 -9.85136985778809\\
127.974998474121 -9.85198974609375\\
128 -9.85260772705078\\
128.024993896484 -9.85321712493896\\
128.050003051758 -9.85381698608398\\
128.074996948242 -9.85440731048584\\
128.100006103516 -9.85499000549316\\
128.125 -9.85556793212891\\
128.149993896484 -9.85613918304443\\
128.175003051758 -9.85670471191406\\
128.199996948242 -9.85726642608643\\
128.225006103516 -9.85782432556152\\
128.25 -9.85837745666504\\
128.274993896484 -9.85893058776855\\
128.300003051758 -9.85948371887207\\
128.324996948242 -9.8600435256958\\
128.350006103516 -9.86060523986816\\
128.375 -9.86116790771484\\
128.399993896484 -9.86173057556152\\
128.425003051758 -9.86229801177979\\
128.449996948242 -9.86287212371826\\
128.475006103516 -9.86346435546875\\
128.5 -9.86409378051758\\
128.524993896484 -9.86477565765381\\
128.550003051758 -9.86546516418457\\
128.574996948242 -9.86611557006836\\
128.600006103516 -9.86673831939697\\
128.625 -9.86733818054199\\
128.649993896484 -9.8679141998291\\
128.675003051758 -9.86847114562988\\
128.699996948242 -9.86902236938477\\
128.725006103516 -9.86956977844238\\
128.75 -9.87011337280273\\
128.774993896484 -9.87065696716309\\
128.800003051758 -9.87120056152344\\
128.824996948242 -9.87174224853516\\
128.850006103516 -9.87227630615234\\
128.875 -9.87281036376953\\
128.899993896484 -9.87334442138672\\
128.925003051758 -9.87387847900391\\
128.949996948242 -9.87442111968994\\
128.975006103516 -9.87496566772461\\
129 -9.87551784515381\\
129.024993896484 -9.87607955932617\\
129.050003051758 -9.8766508102417\\
129.074996948242 -9.87723064422607\\
129.100006103516 -9.87781715393066\\
129.125 -9.87841129302979\\
129.149993896484 -9.87901210784912\\
129.175003051758 -9.87961673736572\\
129.199996948242 -9.88022613525391\\
129.225006103516 -9.88083648681641\\
129.25 -9.88144683837891\\
129.274993896484 -9.88205528259277\\
129.300003051758 -9.88265895843506\\
129.324996948242 -9.88325500488281\\
129.350006103516 -9.88384342193604\\
129.375 -9.88442516326904\\
129.399993896484 -9.88499927520752\\
129.425003051758 -9.88556671142578\\
129.449996948242 -9.88612747192383\\
129.475006103516 -9.88668060302734\\
129.5 -9.8872241973877\\
129.524993896484 -9.88776588439941\\
129.550003051758 -9.8882999420166\\
129.574996948242 -9.88883209228516\\
129.600006103516 -9.88935852050781\\
129.625 -9.88988304138184\\
129.649993896484 -9.89040756225586\\
129.675003051758 -9.89093208312988\\
129.699996948242 -9.89145660400391\\
129.725006103516 -9.89198112487793\\
129.75 -9.8925085067749\\
129.774993896484 -9.89304065704346\\
129.800003051758 -9.89357471466064\\
129.824996948242 -9.89411067962646\\
129.850006103516 -9.89465427398682\\
129.875 -9.89520359039307\\
129.899993896484 -9.89575672149658\\
129.925003051758 -9.89631843566895\\
129.949996948242 -9.89688110351563\\
129.975006103516 -9.89744663238525\\
130 -9.89801692962646\\
130.024993896484 -9.89858722686768\\
130.050003051758 -9.89915943145752\\
130.074996948242 -9.89973068237305\\
130.100006103516 -9.90030193328857\\
130.125 -9.90086555480957\\
130.149993896484 -9.90142726898193\\
130.175003051758 -9.90198516845703\\
130.199996948242 -9.90253829956055\\
130.225006103516 -9.90309047698975\\
130.25 -9.9036340713501\\
130.274993896484 -9.90417575836182\\
130.300003051758 -9.904709815979\\
130.324996948242 -9.90524291992188\\
130.350006103516 -9.90577220916748\\
130.375 -9.9062967300415\\
130.399993896484 -9.90681934356689\\
130.425003051758 -9.9073429107666\\
130.449996948242 -9.90785884857178\\
130.475006103516 -9.90837383270264\\
130.5 -9.9088888168335\\
130.524993896484 -9.90940380096436\\
130.550003051758 -9.90992736816406\\
130.574996948242 -9.91045093536377\\
130.600006103516 -9.91097545623779\\
130.625 -9.91149997711182\\
130.649993896484 -9.91203022003174\\
130.675003051758 -9.91256332397461\\
130.699996948242 -9.9130973815918\\
130.725006103516 -9.91363334655762\\
130.75 -9.91417598724365\\
130.774993896484 -9.914719581604\\
130.800003051758 -9.91526985168457\\
130.824996948242 -9.91582298278809\\
130.850006103516 -9.9163761138916\\
130.875 -9.91692638397217\\
130.899993896484 -9.91748809814453\\
130.925003051758 -9.91807270050049\\
130.949996948242 -9.91867446899414\\
130.975006103516 -9.91928577423096\\
131 -9.91990375518799\\
131.024993896484 -9.9205207824707\\
131.050003051758 -9.92112922668457\\
131.074996948242 -9.92172908782959\\
131.100006103516 -9.92233085632324\\
131.125 -9.92293548583984\\
131.149993896484 -9.92354583740234\\
131.175003051758 -9.92415618896484\\
131.199996948242 -9.92477226257324\\
131.225006103516 -9.92539310455322\\
131.25 -9.92603302001953\\
131.274993896484 -9.92679786682129\\
131.300003051758 -9.93117618560791\\
131.324996948242 -9.95172119140625\\
131.350006103516 -9.99007225036621\\
131.375 -10.0289335250854\\
131.399993896484 -10.0647621154785\\
131.425003051758 -10.0963945388794\\
131.449996948242 -10.1230974197388\\
131.475006103516 -10.1453065872192\\
131.5 -10.167275428772\\
131.524993896484 -10.1884517669678\\
131.550003051758 -10.2073812484741\\
131.574996948242 -10.2229318618774\\
131.600006103516 -10.2349081039429\\
131.625 -10.2927732467651\\
131.649993896484 -10.3708457946777\\
131.665878295898 -10.4170684814453\\
131.690155029297 -10.4702186584473\\
131.701950073242 -10.4807300567627\\
131.722640991211 -10.4876890182495\\
131.747650146484 -10.4930067062378\\
131.772644042969 -10.4974031448364\\
131.797653198242 -10.5009098052979\\
131.822647094727 -10.5041437149048\\
131.845489501953 -10.5071487426758\\
131.868835449219 -10.5103435516357\\
131.893829345703 -10.5412073135376\\
131.918823242188 -10.6203126907349\\
131.943832397461 -10.72727394104\\
131.968826293945 -10.7647228240967\\
131.993835449219 -10.7903547286987\\
132.018829345703 -10.8129768371582\\
132.043823242188 -10.8330726623535\\
132.068832397461 -10.8503360748291\\
132.093826293945 -10.8645915985107\\
132.118835449219 -10.8856315612793\\
132.143829345703 -10.9501085281372\\
132.168823242188 -11.0041055679321\\
132.193832397461 -11.066686630249\\
132.218826293945 -11.1172790527344\\
132.243835449219 -11.1466274261475\\
132.268829345703 -11.1574668884277\\
132.293823242188 -11.1592664718628\\
132.318832397461 -11.1603174209595\\
132.343826293945 -11.1786031723022\\
132.368835449219 -11.2530555725098\\
132.393829345703 -11.3132562637329\\
132.418823242188 -11.3412551879883\\
132.443832397461 -11.3461227416992\\
132.468826293945 -11.3472814559937\\
132.493835449219 -11.3480091094971\\
132.518829345703 -11.3486080169678\\
132.543823242188 -11.3491945266724\\
132.568832397461 -11.3515462875366\\
132.593826293945 -11.4028072357178\\
132.617416381836 -11.4745178222656\\
132.641052246094 -11.5257320404053\\
132.666046142578 -11.5546035766602\\
132.691040039063 -11.5637636184692\\
132.716049194336 -11.5659046173096\\
132.74104309082 -11.5672760009766\\
132.766052246094 -11.5683212280273\\
132.791046142578 -11.5690822601318\\
132.816040039063 -11.5697612762451\\
132.841049194336 -11.5702257156372\\
132.86604309082 -11.5704755783081\\
132.891052246094 -11.570987701416\\
132.916046142578 -11.5716857910156\\
132.941040039063 -11.5726022720337\\
132.966049194336 -11.5734844207764\\
132.99104309082 -11.5742635726929\\
133.016052246094 -11.5749864578247\\
133.041046142578 -11.5757064819336\\
133.066040039063 -11.5764217376709\\
133.091049194336 -11.5771169662476\\
133.11604309082 -11.577787399292\\
133.141052246094 -11.5784215927124\\
133.166046142578 -11.5790548324585\\
133.191040039063 -11.5796937942505\\
133.216049194336 -11.5803289413452\\
133.24104309082 -11.5809392929077\\
133.266052246094 -11.5815095901489\\
133.291046142578 -11.5820112228394\\
133.316040039063 -11.5823078155518\\
133.341049194336 -11.5825242996216\\
133.36604309082 -11.5827398300171\\
133.391052246094 -11.5829725265503\\
133.416046142578 -11.5833072662354\\
133.441040039063 -11.5838460922241\\
133.466049194336 -11.584511756897\\
133.49104309082 -11.5852355957031\\
133.516052246094 -11.5861310958862\\
133.541046142578 -11.5874395370483\\
133.566040039063 -11.5892953872681\\
133.591049194336 -11.5913915634155\\
133.61604309082 -11.5935001373291\\
133.641052246094 -11.5948066711426\\
133.666046142578 -11.5960521697998\\
133.691040039063 -11.5974321365356\\
133.716049194336 -11.5992774963379\\
133.74104309082 -11.6028461456299\\
133.766052246094 -11.6139392852783\\
133.791046142578 -11.6456909179688\\
133.816040039063 -11.6538763046265\\
133.841049194336 -11.6553430557251\\
133.86604309082 -11.6563510894775\\
133.891052246094 -11.657190322876\\
133.916046142578 -11.6580209732056\\
133.941040039063 -11.6581363677979\\
133.966049194336 -11.6574029922485\\
133.99104309082 -11.6562700271606\\
134.016052246094 -11.6549377441406\\
134.041046142578 -11.6536827087402\\
134.066040039063 -11.6528911590576\\
134.091049194336 -11.6531438827515\\
134.11604309082 -11.6539335250854\\
134.141052246094 -11.6546640396118\\
134.166046142578 -11.6553201675415\\
134.191040039063 -11.655930519104\\
134.216049194336 -11.6565227508545\\
134.24104309082 -11.6571102142334\\
134.266052246094 -11.6574583053589\\
134.291046142578 -11.6573400497437\\
134.316040039063 -11.6569356918335\\
134.341049194336 -11.6563835144043\\
134.36604309082 -11.655818939209\\
134.391052246094 -11.6554956436157\\
134.416046142578 -11.6555585861206\\
134.441040039063 -11.6556987762451\\
134.466049194336 -11.6557579040527\\
134.49104309082 -11.6558542251587\\
134.516052246094 -11.6560001373291\\
134.541046142578 -11.6561155319214\\
134.566040039063 -11.6565361022949\\
134.591049194336 -11.6571226119995\\
134.61604309082 -11.6577281951904\\
134.641052246094 -11.6583442687988\\
134.666046142578 -11.658935546875\\
134.691040039063 -11.6594944000244\\
134.716049194336 -11.6600255966187\\
134.74104309082 -11.6605386734009\\
134.766052246094 -11.6610441207886\\
134.791046142578 -11.6615409851074\\
134.816040039063 -11.6619100570679\\
134.841049194336 -11.6619510650635\\
134.86604309082 -11.662015914917\\
134.891052246094 -11.6621713638306\\
134.916046142578 -11.662446975708\\
134.941040039063 -11.66286277771\\
134.966049194336 -11.6634321212769\\
134.99104309082 -11.6641330718994\\
135.016052246094 -11.664927482605\\
135.041046142578 -11.6658201217651\\
135.066040039063 -11.6667413711548\\
135.091049194336 -11.6674013137817\\
135.11604309082 -11.6679277420044\\
135.141052246094 -11.6684856414795\\
135.166046142578 -11.6690664291382\\
135.191040039063 -11.6696653366089\\
135.216049194336 -11.6703424453735\\
135.24104309082 -11.6710567474365\\
135.266052246094 -11.6717882156372\\
135.291046142578 -11.672477722168\\
135.316040039063 -11.6730127334595\\
135.341049194336 -11.6734600067139\\
135.36604309082 -11.6738586425781\\
135.391052246094 -11.6742248535156\\
135.416046142578 -11.6745853424072\\
135.441040039063 -11.6750383377075\\
135.466049194336 -11.6756153106689\\
135.49104309082 -11.6763324737549\\
135.516052246094 -11.6773099899292\\
135.541046142578 -11.6783256530762\\
135.566040039063 -11.6793365478516\\
135.591049194336 -11.6803722381592\\
135.61604309082 -11.6814556121826\\
135.641052246094 -11.6825866699219\\
135.666046142578 -11.6837587356567\\
135.691040039063 -11.6849679946899\\
135.716049194336 -11.686185836792\\
135.74104309082 -11.6872291564941\\
135.766052246094 -11.6882429122925\\
135.791046142578 -11.68932056427\\
135.816040039063 -11.6905040740967\\
135.841049194336 -11.6918420791626\\
135.86604309082 -11.6933307647705\\
135.891052246094 -11.6948089599609\\
135.916046142578 -11.6961803436279\\
135.941040039063 -11.6976766586304\\
135.966049194336 -11.6992235183716\\
135.99104309082 -11.7005491256714\\
136.016052246094 -11.7017011642456\\
136.041046142578 -11.7027196884155\\
136.066040039063 -11.7036561965942\\
136.091049194336 -11.7045593261719\\
136.11604309082 -11.7054224014282\\
136.141052246094 -11.7062664031982\\
136.166046142578 -11.7071208953857\\
136.191040039063 -11.7079772949219\\
136.216049194336 -11.7088298797607\\
136.24104309082 -11.7096710205078\\
136.266052246094 -11.7105188369751\\
136.291046142578 -11.711386680603\\
136.316040039063 -11.7122573852539\\
136.341049194336 -11.7131252288818\\
136.36604309082 -11.7140417098999\\
136.391052246094 -11.7149200439453\\
136.416046142578 -11.7157182693481\\
136.441040039063 -11.7164611816406\\
136.466049194336 -11.7171621322632\\
136.49104309082 -11.7178373336792\\
136.516052246094 -11.7185077667236\\
136.541046142578 -11.7191495895386\\
136.566040039063 -11.719780921936\\
136.591049194336 -11.7204332351685\\
136.61604309082 -11.7211112976074\\
136.641052246094 -11.7218065261841\\
136.666046142578 -11.7225036621094\\
136.691040039063 -11.7232046127319\\
136.716049194336 -11.7239112854004\\
136.74104309082 -11.7245836257935\\
136.766052246094 -11.7252216339111\\
136.791046142578 -11.7258710861206\\
136.816040039063 -11.7265357971191\\
136.841049194336 -11.7272119522095\\
136.86604309082 -11.727915763855\\
136.891052246094 -11.7286624908447\\
136.916046142578 -11.7294483184814\\
136.941040039063 -11.7302398681641\\
136.966049194336 -11.7309703826904\\
136.99104309082 -11.7316389083862\\
137.016052246094 -11.7322826385498\\
137.041046142578 -11.7328987121582\\
137.066040039063 -11.733491897583\\
137.091049194336 -11.7341270446777\\
137.11604309082 -11.7348737716675\\
137.141052246094 -11.7357263565063\\
137.166046142578 -11.7366571426392\\
137.191040039063 -11.737645149231\\
137.216049194336 -11.7386417388916\\
137.24104309082 -11.7395601272583\\
137.266052246094 -11.7404308319092\\
137.291046142578 -11.741307258606\\
137.316040039063 -11.7421932220459\\
137.341049194336 -11.7430820465088\\
137.36604309082 -11.7439794540405\\
137.391052246094 -11.7449235916138\\
137.416046142578 -11.7458963394165\\
137.441040039063 -11.746898651123\\
137.466049194336 -11.747935295105\\
137.49104309082 -11.7489833831787\\
137.516052246094 -11.7499961853027\\
137.541046142578 -11.7509098052979\\
137.566040039063 -11.7517290115356\\
137.591049194336 -11.7524900436401\\
137.61604309082 -11.753231048584\\
137.641052246094 -11.7539720535278\\
137.666046142578 -11.7547178268433\\
137.691040039063 -11.7554740905762\\
137.716049194336 -11.7562379837036\\
137.74104309082 -11.7570095062256\\
137.766052246094 -11.7577896118164\\
137.791046142578 -11.7585945129395\\
137.816040039063 -11.759464263916\\
137.841049194336 -11.7603769302368\\
137.86604309082 -11.7613182067871\\
137.891052246094 -11.7622890472412\\
137.916046142578 -11.7632970809937\\
137.941040039063 -11.7643547058105\\
137.966049194336 -11.7654647827148\\
137.99104309082 -11.7666320800781\\
138.016052246094 -11.7678642272949\\
138.040451049805 -11.7691326141357\\
138.049133300781 -11.7696113586426\\
138.050704956055 -11.7697076797485\\
138.052780151367 -11.7698488235474\\
138.070770263672 -11.7707462310791\\
138.095764160156 -11.7721366882324\\
138.120758056641 -11.7736148834229\\
138.145767211914 -11.7751131057739\\
138.170761108398 -11.7766370773315\\
138.195770263672 -11.7781887054443\\
138.220764160156 -11.7797632217407\\
138.245758056641 -11.7813606262207\\
138.270767211914 -11.7829875946045\\
138.295761108398 -11.7846374511719\\
138.320770263672 -11.7863101959229\\
138.345764160156 -11.7880010604858\\
138.370758056641 -11.7897071838379\\
138.395767211914 -11.7914333343506\\
138.420761108398 -11.7931795120239\\
138.445770263672 -11.7949457168579\\
138.470764160156 -11.7967443466187\\
138.495758056641 -11.7985754013062\\
138.520767211914 -11.80042552948\\
138.545761108398 -11.8022842407227\\
138.570770263672 -11.8041524887085\\
138.595764160156 -11.8060216903687\\
138.620758056641 -11.8078947067261\\
138.645767211914 -11.8097724914551\\
138.670761108398 -11.8116493225098\\
138.695770263672 -11.8135271072388\\
138.720764160156 -11.8154039382935\\
138.745758056641 -11.8172807693481\\
138.770767211914 -11.8191585540771\\
138.795761108398 -11.8210277557373\\
138.820770263672 -11.8228902816772\\
138.845764160156 -11.824746131897\\
138.870758056641 -11.8265895843506\\
138.895767211914 -11.8284225463867\\
138.920761108398 -11.8302392959595\\
138.945770263672 -11.8320398330688\\
138.970764160156 -11.8338212966919\\
138.995758056641 -11.8355836868286\\
139.020767211914 -11.837329864502\\
139.045761108398 -11.839054107666\\
139.070770263672 -11.8407621383667\\
139.095764160156 -11.8424463272095\\
139.120758056641 -11.8441104888916\\
139.145767211914 -11.8457546234131\\
139.170761108398 -11.847375869751\\
139.195770263672 -11.8489770889282\\
139.220764160156 -11.8505563735962\\
139.245758056641 -11.8521146774292\\
139.270767211914 -11.8536520004272\\
139.295761108398 -11.855167388916\\
139.320770263672 -11.8566646575928\\
139.345764160156 -11.8581409454346\\
139.370758056641 -11.85959815979\\
139.395767211914 -11.8610363006592\\
139.420761108398 -11.8624544143677\\
139.445770263672 -11.8638525009155\\
139.470764160156 -11.8652305603027\\
139.495758056641 -11.8665895462036\\
139.520767211914 -11.8679323196411\\
139.545761108398 -11.8692579269409\\
139.570770263672 -11.8705682754517\\
139.595764160156 -11.8718605041504\\
139.620758056641 -11.8731355667114\\
139.645767211914 -11.874397277832\\
139.670761108398 -11.8756446838379\\
139.695770263672 -11.8768768310547\\
139.720764160156 -11.8780899047852\\
139.745758056641 -11.8792905807495\\
139.770767211914 -11.8804779052734\\
139.795761108398 -11.8816528320313\\
139.820770263672 -11.8828115463257\\
139.845764160156 -11.8839530944824\\
139.870758056641 -11.8850688934326\\
139.895767211914 -11.8861608505249\\
139.920761108398 -11.8872270584106\\
139.945770263672 -11.8882684707642\\
139.970764160156 -11.8892440795898\\
139.995758056641 -11.890251159668\\
140.020767211914 -11.8912410736084\\
140.045761108398 -11.8921785354614\\
140.070770263672 -11.8930749893188\\
140.095764160156 -11.8939418792725\\
140.120758056641 -11.8947811126709\\
140.145767211914 -11.8955955505371\\
140.170761108398 -11.8963861465454\\
140.195770263672 -11.8971529006958\\
140.220764160156 -11.8978986740112\\
140.245758056641 -11.898624420166\\
140.270767211914 -11.8993272781372\\
140.295761108398 -11.9000005722046\\
140.320770263672 -11.9006280899048\\
140.345764160156 -11.9011993408203\\
140.370758056641 -11.9017810821533\\
140.395767211914 -11.9022855758667\\
140.420761108398 -11.9027414321899\\
140.445770263672 -11.9032497406006\\
140.470764160156 -11.9038534164429\\
140.495758056641 -11.9046144485474\\
140.520767211914 -11.9054861068726\\
140.545761108398 -11.9062519073486\\
140.570770263672 -11.906834602356\\
140.595764160156 -11.9073638916016\\
140.620758056641 -11.9078798294067\\
140.645767211914 -11.9083852767944\\
140.670761108398 -11.9089021682739\\
140.695770263672 -11.909460067749\\
140.720764160156 -11.9100456237793\\
140.745758056641 -11.9106025695801\\
140.770767211914 -11.9110984802246\\
140.795761108398 -11.9115447998047\\
140.820770263672 -11.9119729995728\\
140.845764160156 -11.9124011993408\\
140.870758056641 -11.9128141403198\\
140.895767211914 -11.9131097793579\\
140.920761108398 -11.9133920669556\\
140.945770263672 -11.9137496948242\\
140.970764160156 -11.9141368865967\\
140.995758056641 -11.9145374298096\\
141.020767211914 -11.9149703979492\\
141.045761108398 -11.9154176712036\\
141.070770263672 -11.9158763885498\\
141.095764160156 -11.9163017272949\\
141.120758056641 -11.9166631698608\\
141.145767211914 -11.9170732498169\\
141.170761108398 -11.9175300598145\\
141.195770263672 -11.9180011749268\\
141.220764160156 -11.9184770584106\\
141.245758056641 -11.9189605712891\\
141.270767211914 -11.919451713562\\
141.295761108398 -11.9199476242065\\
141.320770263672 -11.920449256897\\
141.345764160156 -11.920952796936\\
141.370758056641 -11.9214572906494\\
141.395767211914 -11.9219627380371\\
141.420761108398 -11.922474861145\\
141.445770263672 -11.9229764938354\\
141.470764160156 -11.9234037399292\\
141.495758056641 -11.9237422943115\\
141.520767211914 -11.9241065979004\\
141.545761108398 -11.9244928359985\\
141.570770263672 -11.924877166748\\
141.595764160156 -11.9252471923828\\
141.620758056641 -11.9255895614624\\
141.645767211914 -11.9258899688721\\
141.670761108398 -11.9261465072632\\
141.695770263672 -11.9263906478882\\
141.720764160156 -11.9266424179077\\
141.745758056641 -11.9269151687622\\
141.770767211914 -11.927225112915\\
141.795761108398 -11.9276180267334\\
141.820770263672 -11.9280185699463\\
141.845764160156 -11.9284191131592\\
141.870758056641 -11.9288263320923\\
141.895767211914 -11.9292449951172\\
141.920761108398 -11.929666519165\\
141.945770263672 -11.9300565719604\\
141.970764160156 -11.9303913116455\\
141.995758056641 -11.9307613372803\\
142.020767211914 -11.931245803833\\
142.045761108398 -11.9318180084229\\
142.070770263672 -11.9323930740356\\
142.095764160156 -11.9329509735107\\
142.120758056641 -11.9335222244263\\
142.145767211914 -11.9341259002686\\
142.170761108398 -11.9347772598267\\
142.195770263672 -11.9355249404907\\
142.220764160156 -11.9364023208618\\
142.245758056641 -11.9373960494995\\
142.270767211914 -11.938401222229\\
142.295761108398 -11.9394798278809\\
142.320770263672 -11.9407205581665\\
142.345764160156 -11.9421463012695\\
142.370758056641 -11.9436140060425\\
142.395767211914 -11.9447479248047\\
142.420761108398 -11.9456663131714\\
142.445770263672 -11.9466228485107\\
142.470764160156 -11.9476528167725\\
142.495758056641 -11.9487457275391\\
142.520767211914 -11.9499044418335\\
142.545761108398 -11.9511423110962\\
142.570770263672 -11.952446937561\\
142.595764160156 -11.9537792205811\\
142.620758056641 -11.9550628662109\\
142.645767211914 -11.9560489654541\\
142.670761108398 -11.9569721221924\\
142.695770263672 -11.9580612182617\\
142.720764160156 -11.9593915939331\\
142.745758056641 -11.9610586166382\\
142.770767211914 -11.9630031585693\\
142.795761108398 -11.9650382995605\\
142.820770263672 -11.9671173095703\\
142.845764160156 -11.9692106246948\\
142.870758056641 -11.9712963104248\\
142.895767211914 -11.9730997085571\\
142.920761108398 -11.9741325378418\\
142.945770263672 -11.9750738143921\\
142.970764160156 -11.9761800765991\\
142.995758056641 -11.977484703064\\
143.020767211914 -11.9789819717407\\
143.045761108398 -11.9806795120239\\
143.070770263672 -11.9821891784668\\
143.095764160156 -11.9828643798828\\
143.120758056641 -11.9831838607788\\
143.145767211914 -11.983437538147\\
143.170761108398 -11.9837293624878\\
143.195770263672 -11.9841356277466\\
143.220764160156 -11.9846029281616\\
143.245758056641 -11.9851350784302\\
143.270767211914 -11.9857521057129\\
143.295761108398 -11.9864358901978\\
143.320770263672 -11.9871511459351\\
143.345764160156 -11.9878854751587\\
143.370758056641 -11.9884309768677\\
143.395767211914 -11.9888582229614\\
143.420761108398 -11.989444732666\\
143.445770263672 -11.9902000427246\\
143.470764160156 -11.9911680221558\\
143.495758056641 -11.9923810958862\\
143.520767211914 -11.9937124252319\\
143.545761108398 -11.9949016571045\\
143.570770263672 -11.9959306716919\\
143.595764160156 -11.9970550537109\\
143.620758056641 -11.9982957839966\\
143.645767211914 -11.9996433258057\\
143.670761108398 -12.0010843276978\\
143.695770263672 -12.0025939941406\\
143.720764160156 -12.0040636062622\\
143.745758056641 -12.0053691864014\\
143.770767211914 -12.0066986083984\\
143.795761108398 -12.0081300735474\\
143.820770263672 -12.009654045105\\
143.845764160156 -12.0112371444702\\
143.870758056641 -12.0128440856934\\
143.895767211914 -12.0144367218018\\
143.920761108398 -12.0159149169922\\
143.945770263672 -12.017297744751\\
143.970764160156 -12.018726348877\\
143.995758056641 -12.0202283859253\\
144.020767211914 -12.0217952728271\\
144.045761108398 -12.0233917236328\\
144.070770263672 -12.024923324585\\
144.095764160156 -12.0263833999634\\
144.120758056641 -12.0278596878052\\
144.145767211914 -12.0293626785278\\
144.170761108398 -12.0308904647827\\
144.195770263672 -12.0324401855469\\
144.220764160156 -12.0340013504028\\
144.245758056641 -12.0355167388916\\
144.270767211914 -12.0369462966919\\
144.295761108398 -12.0383672714233\\
144.320770263672 -12.0397996902466\\
144.345764160156 -12.041241645813\\
144.370758056641 -12.0426940917969\\
144.395767211914 -12.0441522598267\\
144.420761108398 -12.0456027984619\\
144.445770263672 -12.0470027923584\\
144.470764160156 -12.0483493804932\\
144.495758056641 -12.0496921539307\\
144.520767211914 -12.0510311126709\\
144.545761108398 -12.0523529052734\\
144.570770263672 -12.0536422729492\\
144.595764160156 -12.0549211502075\\
144.620758056641 -12.0561866760254\\
144.645767211914 -12.057430267334\\
144.670761108398 -12.0586547851563\\
144.695770263672 -12.0598659515381\\
144.720764160156 -12.0610609054565\\
144.745758056641 -12.0622396469116\\
144.770767211914 -12.063404083252\\
144.795761108398 -12.0645513534546\\
144.820770263672 -12.0656833648682\\
144.845764160156 -12.066801071167\\
144.870758056641 -12.0679044723511\\
144.895767211914 -12.0689964294434\\
144.920761108398 -12.0700721740723\\
144.945770263672 -12.0711374282837\\
144.970764160156 -12.0721883773804\\
144.995758056641 -12.0732250213623\\
145.020767211914 -12.0742502212524\\
145.045761108398 -12.0752611160278\\
145.070770263672 -12.0762605667114\\
145.095764160156 -12.0772476196289\\
145.120758056641 -12.078221321106\\
145.145767211914 -12.0791797637939\\
145.170761108398 -12.0801258087158\\
145.195770263672 -12.0810594558716\\
145.220764160156 -12.0819816589355\\
145.245758056641 -12.0828895568848\\
145.270767211914 -12.0837860107422\\
145.295761108398 -12.0846710205078\\
145.320770263672 -12.0855436325073\\
145.345764160156 -12.0864057540894\\
145.370758056641 -12.0872564315796\\
145.395767211914 -12.0880994796753\\
145.420761108398 -12.0889310836792\\
145.445770263672 -12.0897541046143\\
145.470764160156 -12.0905685424805\\
145.495758056641 -12.0913743972778\\
145.520767211914 -12.092170715332\\
145.545761108398 -12.0929584503174\\
145.570770263672 -12.0937376022339\\
145.595764160156 -12.0945072174072\\
145.620758056641 -12.0952682495117\\
145.645767211914 -12.0960168838501\\
145.670761108398 -12.096755027771\\
145.695770263672 -12.0974836349487\\
145.720764160156 -12.0981969833374\\
145.745758056641 -12.0988988876343\\
145.770767211914 -12.0995893478394\\
145.795761108398 -12.1002674102783\\
145.820770263672 -12.1009340286255\\
145.845764160156 -12.1015892028809\\
145.870758056641 -12.1022319793701\\
145.895767211914 -12.1028652191162\\
145.920761108398 -12.1034879684448\\
145.945770263672 -12.1041011810303\\
145.970764160156 -12.1047039031982\\
145.995758056641 -12.1052980422974\\
146.020767211914 -12.105884552002\\
146.045761108398 -12.106463432312\\
146.070770263672 -12.1070356369019\\
146.095764160156 -12.1075983047485\\
146.120758056641 -12.108154296875\\
146.145767211914 -12.1087026596069\\
146.170761108398 -12.1092443466187\\
146.195770263672 -12.1097784042358\\
146.220764160156 -12.1103019714355\\
146.245758056641 -12.1108169555664\\
146.270767211914 -12.1113252639771\\
146.295761108398 -12.1118240356445\\
146.320770263672 -12.1123142242432\\
146.345764160156 -12.1127977371216\\
146.370758056641 -12.1132726669312\\
146.395767211914 -12.1137399673462\\
146.420761108398 -12.1141977310181\\
146.445770263672 -12.114649772644\\
146.470764160156 -12.1150970458984\\
146.495758056641 -12.1155376434326\\
146.520767211914 -12.1159763336182\\
146.545761108398 -12.1164083480835\\
146.570770263672 -12.1168375015259\\
146.595764160156 -12.1172637939453\\
146.620758056641 -12.1176834106445\\
146.645767211914 -12.1181030273438\\
146.670761108398 -12.1185140609741\\
146.695770263672 -12.1189241409302\\
146.720764160156 -12.1193256378174\\
146.745758056641 -12.119725227356\\
146.770767211914 -12.1201162338257\\
146.795761108398 -12.1205024719238\\
146.820770263672 -12.1208839416504\\
146.845764160156 -12.1212568283081\\
146.870758056641 -12.1216268539429\\
146.895767211914 -12.1219892501831\\
146.920761108398 -12.1223478317261\\
146.945770263672 -12.1227006912231\\
146.970764160156 -12.1230516433716\\
146.995758056641 -12.1233949661255\\
147.020767211914 -12.1237382888794\\
147.045761108398 -12.124080657959\\
147.070770263672 -12.124415397644\\
147.095764160156 -12.1247491836548\\
147.120758056641 -12.1250829696655\\
147.145767211914 -12.1254167556763\\
147.170761108398 -12.1257495880127\\
147.195770263672 -12.1260747909546\\
147.220764160156 -12.1263990402222\\
147.245758056641 -12.1267223358154\\
147.270767211914 -12.1270399093628\\
147.295761108398 -12.1273546218872\\
147.320770263672 -12.1276636123657\\
147.345764160156 -12.127968788147\\
147.370758056641 -12.1282663345337\\
147.395767211914 -12.1285591125488\\
147.420761108398 -12.1288452148438\\
147.445770263672 -12.1291255950928\\
147.470764160156 -12.1294021606445\\
147.495758056641 -12.1296720504761\\
147.520767211914 -12.1299390792847\\
147.545761108398 -12.1302042007446\\
147.570770263672 -12.1304616928101\\
147.595764160156 -12.1307191848755\\
147.620758056641 -12.1309766769409\\
147.645767211914 -12.131233215332\\
147.670761108398 -12.1314821243286\\
147.695770263672 -12.1317300796509\\
147.720764160156 -12.1319780349731\\
147.745758056641 -12.1322259902954\\
147.770767211914 -12.132472038269\\
147.795761108398 -12.1327104568481\\
147.820770263672 -12.1329488754272\\
147.845764160156 -12.1331872940063\\
147.870758056641 -12.1334180831909\\
147.895767211914 -12.1336469650269\\
147.920761108398 -12.1338720321655\\
147.945770263672 -12.1340913772583\\
147.970764160156 -12.1343097686768\\
147.995758056641 -12.1345195770264\\
148.020767211914 -12.134729385376\\
148.045761108398 -12.1349363327026\\
148.070770263672 -12.1351366043091\\
148.095764160156 -12.1353368759155\\
148.120758056641 -12.1355314254761\\
148.145767211914 -12.1357221603394\\
148.170761108398 -12.1359128952026\\
148.195770263672 -12.13609790802\\
148.220764160156 -12.1362791061401\\
148.245758056641 -12.1364593505859\\
148.270767211914 -12.1366338729858\\
148.295761108398 -12.1368055343628\\
148.320770263672 -12.1369771957397\\
148.345764160156 -12.1371488571167\\
148.370758056641 -12.1373147964478\\
148.395767211914 -12.1374769210815\\
148.420761108398 -12.1376390457153\\
148.445770263672 -12.1378011703491\\
148.470764160156 -12.1379566192627\\
148.495758056641 -12.1381092071533\\
148.520767211914 -12.1382617950439\\
148.545761108398 -12.1384105682373\\
148.570770263672 -12.1385536193848\\
148.595764160156 -12.1386966705322\\
148.620758056641 -12.1388311386108\\
148.645767211914 -12.1389608383179\\
148.670761108398 -12.1390943527222\\
148.695770263672 -12.1392221450806\\
148.720764160156 -12.1393365859985\\
148.745758056641 -12.1394510269165\\
148.770767211914 -12.1395692825317\\
148.795761108398 -12.1396932601929\\
148.820770263672 -12.139817237854\\
148.845764160156 -12.1399412155151\\
148.870758056641 -12.1400585174561\\
148.895767211914 -12.140172958374\\
148.920761108398 -12.140287399292\\
148.945770263672 -12.14040184021\\
148.970764160156 -12.1405162811279\\
148.995758056641 -12.1406307220459\\
149.020767211914 -12.140754699707\\
149.045761108398 -12.1408834457397\\
149.070770263672 -12.1410360336304\\
149.095764160156 -12.1412086486816\\
149.120758056641 -12.1413860321045\\
149.145767211914 -12.141544342041\\
149.170761108398 -12.1416788101196\\
149.195770263672 -12.1417999267578\\
149.220764160156 -12.1419105529785\\
149.245758056641 -12.142035484314\\
149.270767211914 -12.1421976089478\\
149.295761108398 -12.1424074172974\\
149.320770263672 -12.1426639556885\\
149.345764160156 -12.1429500579834\\
149.370758056641 -12.1432161331177\\
149.395767211914 -12.1434297561646\\
149.420761108398 -12.1436223983765\\
149.445770263672 -12.1438302993774\\
149.470764160156 -12.1440649032593\\
149.495758056641 -12.1443300247192\\
149.520767211914 -12.1446285247803\\
149.545761108398 -12.1449632644653\\
149.570770263672 -12.1453304290771\\
149.595764160156 -12.1457853317261\\
149.620758056641 -12.1463470458984\\
149.645767211914 -12.1469287872314\\
149.670761108398 -12.1475019454956\\
149.695770263672 -12.1480541229248\\
149.720764160156 -12.1485776901245\\
149.745758056641 -12.1490802764893\\
149.770767211914 -12.1495580673218\\
149.795761108398 -12.1500053405762\\
149.820770263672 -12.1504230499268\\
149.845764160156 -12.1508131027222\\
149.870758056641 -12.1511754989624\\
149.895767211914 -12.1515121459961\\
149.920761108398 -12.1518201828003\\
149.945770263672 -12.1521005630493\\
149.970764160156 -12.1523485183716\\
149.995758056641 -12.1526021957397\\
};
\addlegendentry{$\dtmax{} = \expneg{2.5}{3}{[s]}$};

\addplot [
color=mycolor1,
solid,
forget plot
]
table[row sep=crcr]{
0 0\\
1.00008881092072 0\\
2.00008893013 0\\
3.00008893013 0\\
4.00008869171143 0\\
5.00008869171143 0\\
6.00008869171143 0\\
7.00008869171143 0\\
8.00008869171143 0\\
9.00008869171143 0\\
10.0000886917114 0\\
11.0000886917114 0\\
12.0000886917114 0\\
13.0000886917114 0\\
14.0000886917114 0\\
15.0000886917114 0\\
16.0000896453857 0\\
17.0000896453857 0\\
18.0000896453857 0\\
19.0000896453857 -5.42099911626792e-08\\
20.0000896453857 -7.63053549235337e-07\\
21.0000896453857 -0.551618158817291\\
22.0000896453857 -0.656140208244324\\
23.0017566680908 -0.691597878932953\\
24.0017566680908 -0.715859234333038\\
25.0017566680908 -0.738068580627441\\
26.0017566680908 -0.764752209186554\\
27.0017566680908 -0.798926472663879\\
28.0017566680908 -0.842189133167267\\
29.0017566680908 -0.898345768451691\\
30.0017566680908 -0.970405757427216\\
31.0017566680908 -1.06439089775085\\
32.0017547607422 -1.17922711372375\\
33.0017547607422 -1.30489134788513\\
34.0017547607422 -1.4387491941452\\
35.0005073547363 -1.57699608802795\\
36.0005073547363 -1.71048009395599\\
37.0005073547363 -1.84191560745239\\
38.0005073547363 -1.96824610233307\\
39.0005073547363 -2.09402847290039\\
40.0005073547363 -2.21991896629334\\
41.0005073547363 -2.34715604782104\\
42.0005073547363 -2.47329926490784\\
43.0005073547363 -2.60114669799805\\
44.0005073547363 -2.73019337654114\\
45.0005073547363 -2.85645318031311\\
46.0005073547363 -2.97148895263672\\
47.0005073547363 -3.08250021934509\\
48.0005073547363 -3.19829392433167\\
49.0005073547363 -3.31483817100525\\
50.0005073547363 -3.42455887794495\\
51.0005073547363 -3.51806473731995\\
52.0005073547363 -3.60566067695618\\
53.0005073547363 -3.69125723838806\\
54.0005073547363 -3.77427434921265\\
55.0005073547363 -3.85652613639832\\
56.0005073547363 -3.93895959854126\\
57.0005073547363 -4.02236557006836\\
58.0005073547363 -4.10651063919067\\
59.0005073547363 -4.19364452362061\\
60.0005073547363 -4.2822437286377\\
61.0005073547363 -4.37170267105103\\
62.0005073547363 -4.45887565612793\\
63.0005073547363 -4.54623556137085\\
64.0005035400391 -4.63466596603394\\
65.0005035400391 -4.72404623031616\\
66.0005035400391 -4.81467723846436\\
67.0005035400391 -4.90633296966553\\
68.0005035400391 -4.99867296218872\\
69.0005035400391 -5.09061336517334\\
70.0005035400391 -5.18516254425049\\
71.0005035400391 -5.27842712402344\\
72.0005035400391 -5.37239599227905\\
73.0005035400391 -5.46684122085571\\
74.0005035400391 -5.56025075912476\\
75.0005035400391 -5.65363645553589\\
76.0005035400391 -5.75251817703247\\
77.0005035400391 -5.84858512878418\\
78.0005035400391 -5.94299268722534\\
79.0005035400391 -6.04181098937988\\
80.0005035400391 -6.14476108551025\\
81.0005035400391 -6.24708700180054\\
82.0005035400391 -6.34651947021484\\
83.0005035400391 -6.44603824615479\\
84.0005035400391 -6.54329633712769\\
85.0005035400391 -6.63979578018188\\
86.0005035400391 -6.73466348648071\\
87.0005035400391 -6.82779455184937\\
88.0005035400391 -6.91971921920776\\
89.0005035400391 -7.01054859161377\\
90.0005035400391 -7.10042953491211\\
91.0005035400391 -7.190016746521\\
92.0005035400391 -7.2794451713562\\
93.0005035400391 -7.36826276779175\\
94.0005035400391 -7.45264768600464\\
95.0005035400391 -7.5315055847168\\
96.0005035400391 -7.60586309432983\\
97.0005035400391 -7.67699146270752\\
98.0005035400391 -7.74572706222534\\
99.0005035400391 -7.81286764144897\\
100 -7.87911987304688\\
100.012496948242 -7.87993860244751\\
100.025001525879 -7.88075351715088\\
100.037498474121 -7.88156795501709\\
100.050003051758 -7.8823823928833\\
100.0625 -7.88319206237793\\
100.074996948242 -7.88399791717529\\
100.087501525879 -7.88480854034424\\
100.099998474121 -7.88562726974487\\
100.112503051758 -7.88644409179688\\
100.125 -7.88725662231445\\
100.137496948242 -7.88806390762329\\
100.150001525879 -7.88887310028076\\
100.162498474121 -7.88968276977539\\
100.175003051758 -7.89049100875854\\
100.1875 -7.89129877090454\\
100.199996948242 -7.89211416244507\\
100.212501525879 -7.89293384552002\\
100.224998474121 -7.89375305175781\\
100.237503051758 -7.89457273483276\\
100.25 -7.89539194107056\\
100.262496948242 -7.89620780944824\\
100.275001525879 -7.89702272415161\\
100.287498474121 -7.89783239364624\\
100.300003051758 -7.89863777160645\\
100.3125 -7.89943695068359\\
100.324996948242 -7.90023231506348\\
100.337501525879 -7.90102386474609\\
100.349998474121 -7.90181159973145\\
100.362503051758 -7.90259552001953\\
100.375 -7.90337514877319\\
100.387496948242 -7.90415143966675\\
100.400001525879 -7.90492820739746\\
100.412498474121 -7.90570259094238\\
100.425003051758 -7.9064793586731\\
100.4375 -7.90725564956665\\
100.449996948242 -7.90803194046021\\
100.462501525879 -7.90881061553955\\
100.474998474121 -7.909592628479\\
100.487503051758 -7.9103798866272\\
100.5 -7.91117382049561\\
100.512496948242 -7.91197681427002\\
100.525001525879 -7.9127893447876\\
100.537498474121 -7.91361045837402\\
100.550003051758 -7.91443824768066\\
100.5625 -7.91527032852173\\
100.574996948242 -7.91610383987427\\
100.587501525879 -7.91693782806396\\
100.599998474121 -7.91777372360229\\
100.612503051758 -7.91861343383789\\
100.625 -7.91945314407349\\
100.637496948242 -7.92029619216919\\
100.650001525879 -7.92113971710205\\
100.662498474121 -7.92198276519775\\
100.675003051758 -7.92282629013062\\
100.6875 -7.92367315292358\\
100.699996948242 -7.92452049255371\\
100.712501525879 -7.92536878585815\\
100.724998474121 -7.92621660232544\\
100.737503051758 -7.92706489562988\\
100.75 -7.92791271209717\\
100.762496948242 -7.92875957489014\\
100.775001525879 -7.929603099823\\
100.787498474121 -7.93044376373291\\
100.800003051758 -7.93128252029419\\
100.8125 -7.93211841583252\\
100.824996948242 -7.93295097351074\\
100.837501525879 -7.93378019332886\\
100.849998474121 -7.93460607528687\\
100.862503051758 -7.9354305267334\\
100.875 -7.93625164031982\\
100.887496948242 -7.93707036972046\\
100.900001525879 -7.93788528442383\\
100.912498474121 -7.93869733810425\\
100.925003051758 -7.93950748443604\\
100.9375 -7.94031715393066\\
100.949996948242 -7.94112300872803\\
100.962501525879 -7.94192838668823\\
100.974998474121 -7.94273042678833\\
100.987503051758 -7.94353151321411\\
101 -7.94433927536011\\
101.012496948242 -7.94515323638916\\
101.025001525879 -7.94596815109253\\
101.037498474121 -7.94678258895874\\
101.050003051758 -7.94759464263916\\
101.0625 -7.94840431213379\\
101.074996948242 -7.94921398162842\\
101.087501525879 -7.95002412796021\\
101.099998474121 -7.95083045959473\\
101.112503051758 -7.95163536071777\\
101.125 -7.95243549346924\\
101.137496948242 -7.95324230194092\\
101.150001525879 -7.95405244827271\\
101.162498474121 -7.95485877990723\\
101.175003051758 -7.95566511154175\\
101.1875 -7.95647954940796\\
101.199996948242 -7.9573016166687\\
101.212501525879 -7.95812606811523\\
101.224998474121 -7.95895004272461\\
101.237503051758 -7.95977449417114\\
101.25 -7.96059370040894\\
101.262496948242 -7.96141147613525\\
101.275001525879 -7.96222639083862\\
101.287498474121 -7.96303844451904\\
101.300003051758 -7.96384716033936\\
101.3125 -7.96464967727661\\
101.324996948242 -7.96544647216797\\
101.337501525879 -7.9662389755249\\
101.349998474121 -7.96702718734741\\
101.362503051758 -7.96781206130981\\
101.375 -7.96859216690063\\
101.387496948242 -7.96936845779419\\
101.400001525879 -7.97014045715332\\
101.412498474121 -7.97091197967529\\
101.425003051758 -7.97168397903442\\
101.4375 -7.9724555015564\\
101.449996948242 -7.97322702407837\\
101.462501525879 -7.9739990234375\\
101.474998474121 -7.97477054595947\\
101.487503051758 -7.97554445266724\\
101.5 -7.97632074356079\\
101.512496948242 -7.97710132598877\\
101.525001525879 -7.97788619995117\\
101.537498474121 -7.97867441177368\\
101.550003051758 -7.97946786880493\\
101.5625 -7.98026561737061\\
101.574996948242 -7.98106575012207\\
101.587501525879 -7.98186635971069\\
101.599998474121 -7.98267078399658\\
101.612503051758 -7.9834771156311\\
101.625 -7.98428678512573\\
101.637496948242 -7.98509454727173\\
101.650001525879 -7.98589992523193\\
101.662498474121 -7.98670482635498\\
101.675003051758 -7.98751020431519\\
101.6875 -7.98831510543823\\
101.699996948242 -7.98912000656128\\
101.712501525879 -7.98992538452148\\
101.724998474121 -7.99072980880737\\
101.737503051758 -7.991530418396\\
101.75 -7.99233055114746\\
101.762496948242 -7.99313068389893\\
101.775001525879 -7.99393129348755\\
101.787498474121 -7.99473142623901\\
101.800003051758 -7.99552726745605\\
101.8125 -7.99632263183594\\
101.824996948242 -7.99711799621582\\
101.837501525879 -7.99791240692139\\
101.849998474121 -7.99870300292969\\
101.862503051758 -7.99949407577515\\
101.875 -8.00028514862061\\
101.887496948242 -8.00107574462891\\
101.900001525879 -8.00186634063721\\
101.912498474121 -8.00265598297119\\
101.925003051758 -8.00344657897949\\
101.9375 -8.00423622131348\\
101.949996948242 -8.00502586364746\\
101.962501525879 -8.00581645965576\\
101.974998474121 -8.00660610198975\\
101.987503051758 -8.00739669799805\\
102 -8.00818634033203\\
102.012496948242 -8.00897598266602\\
102.025001525879 -8.00976657867432\\
102.037498474121 -8.0105562210083\\
102.050003051758 -8.0113468170166\\
102.0625 -8.01213645935059\\
102.074996948242 -8.01292610168457\\
102.087501525879 -8.01371669769287\\
102.099998474121 -8.01450634002686\\
102.112503051758 -8.01529693603516\\
102.125 -8.01607990264893\\
102.137496948242 -8.01686191558838\\
102.150001525879 -8.01764392852783\\
102.162498474121 -8.0184326171875\\
102.175003051758 -8.0192232131958\\
102.1875 -8.02000522613525\\
102.199996948242 -8.02079200744629\\
102.212501525879 -8.02158737182617\\
102.224998474121 -8.02238750457764\\
102.237503051758 -8.02318859100342\\
102.25 -8.0239896774292\\
102.262496948242 -8.02479076385498\\
102.275001525879 -8.02559185028076\\
102.287498474121 -8.02639102935791\\
102.300003051758 -8.02718448638916\\
102.3125 -8.02797508239746\\
102.324996948242 -8.02876281738281\\
102.337501525879 -8.02954483032227\\
102.349998474121 -8.03032302856445\\
102.362503051758 -8.03109550476074\\
102.375 -8.03186702728271\\
102.387496948242 -8.03263664245605\\
102.400001525879 -8.03339958190918\\
102.412498474121 -8.03416156768799\\
102.425003051758 -8.0349235534668\\
102.4375 -8.03567886352539\\
102.449996948242 -8.03643226623535\\
102.462501525879 -8.037184715271\\
102.474998474121 -8.03793621063232\\
102.487503051758 -8.03868865966797\\
102.5 -8.03944206237793\\
102.512496948242 -8.04019546508789\\
102.525001525879 -8.04095363616943\\
102.537498474121 -8.04171562194824\\
102.550003051758 -8.04247856140137\\
102.5625 -8.04324913024902\\
102.574996948242 -8.044020652771\\
102.587501525879 -8.04479312896729\\
102.599998474121 -8.04556560516357\\
102.612503051758 -8.04634475708008\\
102.625 -8.0471248626709\\
102.637496948242 -8.04790592193604\\
102.650001525879 -8.04868793487549\\
102.662498474121 -8.04946994781494\\
102.675003051758 -8.05025196075439\\
102.6875 -8.05103397369385\\
102.699996948242 -8.05181789398193\\
102.712501525879 -8.05260848999023\\
102.724998474121 -8.05339813232422\\
102.737503051758 -8.05418872833252\\
102.75 -8.0549783706665\\
102.762496948242 -8.05576801300049\\
102.775001525879 -8.05655860900879\\
102.787498474121 -8.05735015869141\\
102.800003051758 -8.05814170837402\\
102.8125 -8.05893325805664\\
102.824996948242 -8.05972480773926\\
102.837501525879 -8.06051635742188\\
102.849998474121 -8.06130790710449\\
102.862503051758 -8.06210231781006\\
102.875 -8.06290149688721\\
102.887496948242 -8.06370067596436\\
102.900001525879 -8.06450080871582\\
102.912498474121 -8.06529998779297\\
102.925003051758 -8.06610107421875\\
102.9375 -8.06690216064453\\
102.949996948242 -8.06770324707031\\
102.962501525879 -8.06850433349609\\
102.974998474121 -8.06930541992188\\
102.987503051758 -8.07010650634766\\
103 -8.07091522216797\\
103.012496948242 -8.07172393798828\\
103.025001525879 -8.07253360748291\\
103.037498474121 -8.07334232330322\\
103.050003051758 -8.07415199279785\\
103.0625 -8.07496070861816\\
103.074996948242 -8.07576942443848\\
103.087501525879 -8.07657909393311\\
103.099998474121 -8.07738780975342\\
103.112503051758 -8.07819747924805\\
103.125 -8.07900619506836\\
103.137496948242 -8.07981395721436\\
103.150001525879 -8.08061504364014\\
103.162498474121 -8.08141613006592\\
103.175003051758 -8.08221626281738\\
103.1875 -8.08300971984863\\
103.199996948242 -8.08380126953125\\
103.212501525879 -8.08459949493408\\
103.224998474121 -8.08540058135986\\
103.237503051758 -8.08620834350586\\
103.25 -8.08701705932617\\
103.262496948242 -8.08782577514648\\
103.275001525879 -8.08863544464111\\
103.287498474121 -8.08944129943848\\
103.300003051758 -8.09024238586426\\
103.3125 -8.09104251861572\\
103.324996948242 -8.09183883666992\\
103.337501525879 -8.09263038635254\\
103.349998474121 -8.09342002868652\\
103.362503051758 -8.09420394897461\\
103.375 -8.09498500823975\\
103.387496948242 -8.09576034545898\\
103.400001525879 -8.09653282165527\\
103.412498474121 -8.09730339050293\\
103.425003051758 -8.098069190979\\
103.4375 -8.09883213043213\\
103.449996948242 -8.09959316253662\\
103.462501525879 -8.10035514831543\\
103.474998474121 -8.10110950469971\\
103.487503051758 -8.10186290740967\\
103.5 -8.10261535644531\\
103.512496948242 -8.10336685180664\\
103.525001525879 -8.10411930084229\\
103.537498474121 -8.10487079620361\\
103.550003051758 -8.10562324523926\\
103.5625 -8.10637474060059\\
103.574996948242 -8.10712814331055\\
103.587501525879 -8.10788154602051\\
103.599998474121 -8.10863494873047\\
103.612503051758 -8.10938930511475\\
103.625 -8.11015033721924\\
103.637496948242 -8.11091136932373\\
103.650001525879 -8.11167430877686\\
103.662498474121 -8.11243724822998\\
103.675003051758 -8.11320018768311\\
103.6875 -8.11396312713623\\
103.699996948242 -8.11472606658936\\
103.712501525879 -8.11548900604248\\
103.724998474121 -8.11625194549561\\
103.737503051758 -8.11701488494873\\
103.75 -8.11777782440186\\
103.762496948242 -8.11853981018066\\
103.775001525879 -8.11930179595947\\
103.787498474121 -8.12006282806396\\
103.800003051758 -8.12082481384277\\
103.8125 -8.12158584594727\\
103.824996948242 -8.12234687805176\\
103.837501525879 -8.12310886383057\\
103.849998474121 -8.12386989593506\\
103.862503051758 -8.12463188171387\\
103.875 -8.12539005279541\\
103.887496948242 -8.12614345550537\\
103.900001525879 -8.12689685821533\\
103.912498474121 -8.12765026092529\\
103.925003051758 -8.12840366363525\\
103.9375 -8.12915706634521\\
103.949996948242 -8.12991046905518\\
103.962501525879 -8.13066387176514\\
103.974998474121 -8.13141632080078\\
103.987503051758 -8.13216876983643\\
104 -8.13292026519775\\
104.012496948242 -8.13367176055908\\
104.025001525879 -8.13442420959473\\
104.037498474121 -8.13517570495605\\
104.050003051758 -8.1359281539917\\
104.0625 -8.1366720199585\\
104.074996948242 -8.13741588592529\\
104.087501525879 -8.13815975189209\\
104.099998474121 -8.13890266418457\\
104.112503051758 -8.13964557647705\\
104.125 -8.14038848876953\\
104.137496948242 -8.14113235473633\\
104.150001525879 -8.14187908172607\\
104.162498474121 -8.14262771606445\\
104.175003051758 -8.1433801651001\\
104.1875 -8.14413356781006\\
104.199996948242 -8.1448917388916\\
104.212501525879 -8.14565277099609\\
104.224998474121 -8.14640617370605\\
104.237503051758 -8.14715957641602\\
104.25 -8.14791584014893\\
104.262496948242 -8.14867687225342\\
104.275001525879 -8.14943885803223\\
104.287498474121 -8.15019989013672\\
104.300003051758 -8.15095806121826\\
104.3125 -8.15171146392822\\
104.324996948242 -8.15246295928955\\
104.337501525879 -8.15321159362793\\
104.349998474121 -8.15395545959473\\
104.362503051758 -8.15469932556152\\
104.375 -8.15544128417969\\
104.387496948242 -8.15617752075195\\
104.400001525879 -8.15691184997559\\
104.412498474121 -8.15764427185059\\
104.425003051758 -8.15837478637695\\
104.4375 -8.15909957885742\\
104.449996948242 -8.15982341766357\\
104.462501525879 -8.16054725646973\\
104.474998474121 -8.16127014160156\\
104.487503051758 -8.16198825836182\\
104.5 -8.16270351409912\\
104.512496948242 -8.16341876983643\\
104.525001525879 -8.16413402557373\\
104.537498474121 -8.1648473739624\\
104.550003051758 -8.16556167602539\\
104.5625 -8.16627502441406\\
104.574996948242 -8.16698837280273\\
104.587501525879 -8.16770267486572\\
104.599998474121 -8.16841602325439\\
104.612503051758 -8.16913032531738\\
104.625 -8.16984367370605\\
104.637496948242 -8.17055892944336\\
104.650001525879 -8.17127418518066\\
104.662498474121 -8.17198944091797\\
104.675003051758 -8.17270469665527\\
104.6875 -8.17341995239258\\
104.699996948242 -8.17414093017578\\
104.712501525879 -8.17486476898193\\
104.724998474121 -8.17558765411377\\
104.737503051758 -8.17631149291992\\
104.75 -8.17703437805176\\
104.762496948242 -8.17775821685791\\
104.775001525879 -8.17848300933838\\
104.787498474121 -8.17920780181885\\
104.800003051758 -8.17993259429932\\
104.8125 -8.18065738677979\\
104.824996948242 -8.18138217926025\\
104.837501525879 -8.18210697174072\\
104.849998474121 -8.18283176422119\\
104.862503051758 -8.18355655670166\\
104.875 -8.18428134918213\\
104.887496948242 -8.1850061416626\\
104.900001525879 -8.18573093414307\\
104.912498474121 -8.18645572662354\\
104.925003051758 -8.187180519104\\
104.9375 -8.18790531158447\\
104.949996948242 -8.18863010406494\\
104.962501525879 -8.18935489654541\\
104.974998474121 -8.19007968902588\\
104.987503051758 -8.19080448150635\\
105 -8.19152927398682\\
105.012496948242 -8.19225406646729\\
105.025001525879 -8.19297885894775\\
105.037498474121 -8.19370460510254\\
105.050003051758 -8.19443798065186\\
105.0625 -8.19517040252686\\
105.074996948242 -8.19590473175049\\
105.087501525879 -8.19663906097412\\
105.099998474121 -8.19737815856934\\
105.112503051758 -8.19812107086182\\
105.125 -8.19886493682861\\
105.137496948242 -8.19960880279541\\
105.150001525879 -8.20035266876221\\
105.162498474121 -8.201096534729\\
105.175003051758 -8.2018404006958\\
105.1875 -8.2025842666626\\
105.199996948242 -8.20332622528076\\
105.212501525879 -8.20406913757324\\
105.224998474121 -8.20481109619141\\
105.237503051758 -8.20555400848389\\
105.25 -8.20629692077637\\
105.262496948242 -8.20704078674316\\
105.275001525879 -8.20778369903564\\
105.287498474121 -8.20852565765381\\
105.300003051758 -8.20926856994629\\
105.3125 -8.21001052856445\\
105.324996948242 -8.21075248718262\\
105.337501525879 -8.21148872375488\\
105.349998474121 -8.21222305297852\\
105.362503051758 -8.21295738220215\\
105.375 -8.21369171142578\\
105.387496948242 -8.21442413330078\\
105.400001525879 -8.21515464782715\\
105.412498474121 -8.21587944030762\\
105.425003051758 -8.21660327911377\\
105.4375 -8.21732616424561\\
105.449996948242 -8.21804428100586\\
105.462501525879 -8.21875953674316\\
105.474998474121 -8.21947479248047\\
105.487503051758 -8.22018909454346\\
105.5 -8.2209005355835\\
105.512496948242 -8.22160625457764\\
105.525001525879 -8.22231197357178\\
105.537498474121 -8.2230167388916\\
105.550003051758 -8.22372150421143\\
105.5625 -8.22442531585693\\
105.574996948242 -8.22512149810791\\
105.587501525879 -8.22581768035889\\
105.599998474121 -8.22651386260986\\
105.612503051758 -8.22721004486084\\
105.625 -8.22790622711182\\
105.637496948242 -8.22860240936279\\
105.650001525879 -8.22929859161377\\
105.662498474121 -8.22999382019043\\
105.675003051758 -8.23068904876709\\
105.6875 -8.23138332366943\\
105.699996948242 -8.23207759857178\\
105.712501525879 -8.23277282714844\\
105.724998474121 -8.23346710205078\\
105.737503051758 -8.23416328430176\\
105.75 -8.23485946655273\\
105.762496948242 -8.23555564880371\\
105.775001525879 -8.23625183105469\\
105.787498474121 -8.23694801330566\\
105.800003051758 -8.23764419555664\\
105.8125 -8.23834037780762\\
105.824996948242 -8.23903656005859\\
105.837501525879 -8.23973274230957\\
105.849998474121 -8.24042892456055\\
105.862503051758 -8.24112510681152\\
105.875 -8.2418212890625\\
105.887496948242 -8.24251747131348\\
105.900001525879 -8.24321365356445\\
105.912498474121 -8.24390983581543\\
105.925003051758 -8.24460601806641\\
105.9375 -8.24530601501465\\
105.949996948242 -8.24600982666016\\
105.962501525879 -8.24671459197998\\
105.974998474121 -8.24741840362549\\
105.987503051758 -8.24812316894531\\
106 -8.24882888793945\\
106.012496948242 -8.24953460693359\\
106.025001525879 -8.25024032592773\\
106.037498474121 -8.25094604492188\\
106.050003051758 -8.25165176391602\\
106.0625 -8.25236034393311\\
106.074996948242 -8.25307369232178\\
106.087501525879 -8.25378799438477\\
106.099998474121 -8.25450134277344\\
106.112503051758 -8.25521564483643\\
106.125 -8.25593090057373\\
106.137496948242 -8.2566499710083\\
106.150001525879 -8.25737380981445\\
106.162498474121 -8.25809860229492\\
106.175003051758 -8.25882339477539\\
106.1875 -8.25954818725586\\
106.199996948242 -8.26027679443359\\
106.212501525879 -8.26100921630859\\
106.224998474121 -8.26174163818359\\
106.237503051758 -8.26247501373291\\
106.25 -8.26320457458496\\
106.262496948242 -8.26392936706543\\
106.275001525879 -8.2646541595459\\
106.287498474121 -8.26537895202637\\
106.300003051758 -8.26610374450684\\
106.3125 -8.26683330535889\\
106.324996948242 -8.26756572723389\\
106.337501525879 -8.26829528808594\\
106.349998474121 -8.26902008056641\\
106.362503051758 -8.26974487304688\\
106.375 -8.27046966552734\\
106.387496948242 -8.2711935043335\\
106.400001525879 -8.27191734313965\\
106.412498474121 -8.27263450622559\\
106.425003051758 -8.27334976196289\\
106.4375 -8.2740650177002\\
106.449996948242 -8.27477836608887\\
106.462501525879 -8.27549076080322\\
106.474998474121 -8.27619647979736\\
106.487503051758 -8.2769021987915\\
106.5 -8.27760601043701\\
106.512496948242 -8.27830505371094\\
106.525001525879 -8.27900123596191\\
106.537498474121 -8.27969741821289\\
106.550003051758 -8.28039264678955\\
106.5625 -8.28108692169189\\
106.574996948242 -8.28178119659424\\
106.587501525879 -8.28246784210205\\
106.599998474121 -8.28315448760986\\
106.612503051758 -8.28384113311768\\
106.625 -8.28452777862549\\
106.637496948242 -8.28521347045898\\
106.650001525879 -8.28589916229248\\
106.662498474121 -8.28658390045166\\
106.675003051758 -8.28726959228516\\
106.6875 -8.2879524230957\\
106.699996948242 -8.28862953186035\\
106.712501525879 -8.289306640625\\
106.724998474121 -8.28998374938965\\
106.737503051758 -8.2906608581543\\
106.75 -8.29133796691895\\
106.762496948242 -8.29201507568359\\
106.775001525879 -8.29269218444824\\
106.787498474121 -8.29336929321289\\
106.800003051758 -8.29404640197754\\
106.8125 -8.29472351074219\\
106.824996948242 -8.29540061950684\\
106.837501525879 -8.29607772827148\\
106.849998474121 -8.29675483703613\\
106.862503051758 -8.29743194580078\\
106.875 -8.29810905456543\\
106.887496948242 -8.29879093170166\\
106.900001525879 -8.29947662353516\\
106.912498474121 -8.30016136169434\\
106.925003051758 -8.30084705352783\\
106.9375 -8.30153179168701\\
106.949996948242 -8.30221652984619\\
106.962501525879 -8.30290222167969\\
106.974998474121 -8.30358695983887\\
106.987503051758 -8.30427265167236\\
107 -8.30495929718018\\
107.012496948242 -8.30564594268799\\
107.025001525879 -8.3063325881958\\
107.037498474121 -8.30702590942383\\
107.050003051758 -8.30772113800049\\
107.0625 -8.30841541290283\\
107.074996948242 -8.30911159515381\\
107.087501525879 -8.30980777740479\\
107.099998474121 -8.31050395965576\\
107.112503051758 -8.31120014190674\\
107.125 -8.31190299987793\\
107.137496948242 -8.31260681152344\\
107.150001525879 -8.31331157684326\\
107.162498474121 -8.31401538848877\\
107.175003051758 -8.31472015380859\\
107.1875 -8.31542587280273\\
107.199996948242 -8.31613349914551\\
107.212501525879 -8.3168478012085\\
107.224998474121 -8.3175630569458\\
107.237503051758 -8.31827831268311\\
107.25 -8.31899356842041\\
107.262496948242 -8.31970882415771\\
107.275001525879 -8.32042407989502\\
107.287498474121 -8.32113933563232\\
107.300003051758 -8.32185459136963\\
107.3125 -8.32256984710693\\
107.324996948242 -8.32328510284424\\
107.337501525879 -8.32400417327881\\
107.349998474121 -8.32472705841064\\
107.362503051758 -8.3254508972168\\
107.375 -8.32617378234863\\
107.387496948242 -8.32689666748047\\
107.400001525879 -8.32762050628662\\
107.412498474121 -8.32834339141846\\
107.425003051758 -8.32906150817871\\
107.4375 -8.32977676391602\\
107.449996948242 -8.33049201965332\\
107.462501525879 -8.33120632171631\\
107.474998474121 -8.33191967010498\\
107.487503051758 -8.33262729644775\\
107.5 -8.33333301544189\\
107.512496948242 -8.33403778076172\\
107.525001525879 -8.33474254608154\\
107.537498474121 -8.33544158935547\\
107.550003051758 -8.33613777160645\\
107.5625 -8.33683395385742\\
107.574996948242 -8.33752918243408\\
107.587501525879 -8.33822441101074\\
107.599998474121 -8.3389139175415\\
107.612503051758 -8.33960056304932\\
107.625 -8.34028720855713\\
107.637496948242 -8.34097194671631\\
107.650001525879 -8.3416576385498\\
107.662498474121 -8.34233951568604\\
107.675003051758 -8.34301662445068\\
107.6875 -8.34369373321533\\
107.699996948242 -8.34437084197998\\
107.712501525879 -8.34504795074463\\
107.724998474121 -8.34572410583496\\
107.737503051758 -8.34640026092529\\
107.75 -8.34707546234131\\
107.762496948242 -8.34775066375732\\
107.775001525879 -8.34842681884766\\
107.787498474121 -8.34910202026367\\
107.800003051758 -8.349778175354\\
107.8125 -8.35045337677002\\
107.824996948242 -8.35112857818604\\
107.837501525879 -8.35180473327637\\
107.849998474121 -8.35247993469238\\
107.862503051758 -8.35315608978271\\
107.875 -8.35383129119873\\
107.887496948242 -8.35450839996338\\
107.900001525879 -8.35518550872803\\
107.912498474121 -8.35586261749268\\
107.925003051758 -8.35653972625732\\
107.9375 -8.35721683502197\\
107.949996948242 -8.35789394378662\\
107.962501525879 -8.35857105255127\\
107.974998474121 -8.35924816131592\\
107.987503051758 -8.35992527008057\\
108 -8.36060237884521\\
108.012496948242 -8.36127948760986\\
108.025001525879 -8.36196231842041\\
108.037498474121 -8.36264705657959\\
108.050003051758 -8.36333274841309\\
108.0625 -8.36401748657227\\
108.074996948242 -8.36470222473145\\
108.087501525879 -8.36538791656494\\
108.099998474121 -8.36607456207275\\
108.112503051758 -8.36676216125488\\
108.125 -8.36745643615723\\
108.137496948242 -8.36815071105957\\
108.150001525879 -8.36884689331055\\
108.162498474121 -8.36954307556152\\
108.175003051758 -8.37024116516113\\
108.1875 -8.37094497680664\\
108.199996948242 -8.37164878845215\\
108.212501525879 -8.37235450744629\\
108.224998474121 -8.37306022644043\\
108.237503051758 -8.37376594543457\\
108.25 -8.37447166442871\\
108.262496948242 -8.37517738342285\\
108.275001525879 -8.37588310241699\\
108.287498474121 -8.37659454345703\\
108.300003051758 -8.37730884552002\\
108.3125 -8.37802410125732\\
108.324996948242 -8.37873935699463\\
108.337501525879 -8.37945461273193\\
108.349998474121 -8.38016986846924\\
108.362503051758 -8.38088512420654\\
108.375 -8.38160228729248\\
108.387496948242 -8.38232517242432\\
108.400001525879 -8.38304901123047\\
108.412498474121 -8.3837718963623\\
108.425003051758 -8.38449573516846\\
108.4375 -8.38521862030029\\
108.449996948242 -8.38593578338623\\
108.462501525879 -8.38665103912354\\
108.474998474121 -8.38736438751221\\
108.487503051758 -8.38807392120361\\
108.5 -8.38877964019775\\
108.512496948242 -8.38948440551758\\
108.525001525879 -8.3901891708374\\
108.537498474121 -8.39089298248291\\
108.550003051758 -8.39159202575684\\
108.5625 -8.39228820800781\\
108.574996948242 -8.39298439025879\\
108.587501525879 -8.39368057250977\\
108.599998474121 -8.39437484741211\\
108.612503051758 -8.39507007598877\\
108.625 -8.39576435089111\\
108.637496948242 -8.39645862579346\\
108.650001525879 -8.39715003967285\\
108.662498474121 -8.39783668518066\\
108.675003051758 -8.39852333068848\\
108.6875 -8.39920997619629\\
108.699996948242 -8.3998966217041\\
108.712501525879 -8.40058326721191\\
108.724998474121 -8.40126800537109\\
108.737503051758 -8.40195369720459\\
108.75 -8.40263843536377\\
108.762496948242 -8.40331745147705\\
108.775001525879 -8.4039945602417\\
108.787498474121 -8.40467166900635\\
108.800003051758 -8.405348777771\\
108.8125 -8.40602588653564\\
108.824996948242 -8.40670299530029\\
108.837501525879 -8.40738010406494\\
108.849998474121 -8.40805721282959\\
108.862503051758 -8.40873432159424\\
108.875 -8.40941047668457\\
108.887496948242 -8.41008567810059\\
108.900001525879 -8.41076183319092\\
108.912498474121 -8.41143703460693\\
108.925003051758 -8.41211318969727\\
108.9375 -8.41278839111328\\
108.949996948242 -8.4134635925293\\
108.962501525879 -8.41413974761963\\
108.974998474121 -8.41481590270996\\
108.987503051758 -8.41549301147461\\
109 -8.41616821289063\\
109.012496948242 -8.41684341430664\\
109.025001525879 -8.41752052307129\\
109.037498474121 -8.41819763183594\\
109.050003051758 -8.41887474060059\\
109.0625 -8.41955184936523\\
109.074996948242 -8.42022895812988\\
109.087501525879 -8.42090606689453\\
109.099998474121 -8.42158794403076\\
109.112503051758 -8.42227363586426\\
109.125 -8.42295837402344\\
109.137496948242 -8.42364406585693\\
109.150001525879 -8.42433071136475\\
109.162498474121 -8.42501735687256\\
109.175003051758 -8.42570400238037\\
109.1875 -8.4263973236084\\
109.199996948242 -8.42709159851074\\
109.212501525879 -8.4277868270874\\
109.224998474121 -8.42848300933838\\
109.237503051758 -8.42917919158936\\
109.25 -8.42987823486328\\
109.262496948242 -8.43058204650879\\
109.275001525879 -8.43128776550293\\
109.287498474121 -8.4319953918457\\
109.300003051758 -8.43270969390869\\
109.3125 -8.43342304229736\\
109.324996948242 -8.43413734436035\\
109.337501525879 -8.43485260009766\\
109.349998474121 -8.43556785583496\\
109.362503051758 -8.43628311157227\\
109.375 -8.43699836730957\\
109.387496948242 -8.43771362304688\\
109.400001525879 -8.43842887878418\\
109.412498474121 -8.43914413452148\\
109.425003051758 -8.43985939025879\\
109.4375 -8.44058132171631\\
109.449996948242 -8.44130420684814\\
109.462501525879 -8.44202709197998\\
109.474998474121 -8.44274425506592\\
109.487503051758 -8.44346618652344\\
109.5 -8.44418907165527\\
109.512496948242 -8.44491195678711\\
109.525001525879 -8.44563293457031\\
109.537498474121 -8.44634819030762\\
109.550003051758 -8.44706344604492\\
109.5625 -8.44777679443359\\
109.574996948242 -8.448486328125\\
109.587501525879 -8.44919204711914\\
109.599998474121 -8.44989776611328\\
109.612503051758 -8.45060348510742\\
109.625 -8.45130920410156\\
109.637496948242 -8.4520149230957\\
109.650001525879 -8.45272064208984\\
109.662498474121 -8.45342540740967\\
109.675003051758 -8.45413017272949\\
109.6875 -8.454833984375\\
109.699996948242 -8.45553779602051\\
109.712501525879 -8.45623397827148\\
109.724998474121 -8.45693016052246\\
109.737503051758 -8.45762634277344\\
109.75 -8.4583215713501\\
109.762496948242 -8.45901584625244\\
109.775001525879 -8.4597110748291\\
109.787498474121 -8.46040534973145\\
109.800003051758 -8.46109580993652\\
109.8125 -8.46178245544434\\
109.824996948242 -8.46246910095215\\
109.837501525879 -8.46315574645996\\
109.849998474121 -8.46384143829346\\
109.862503051758 -8.46452713012695\\
109.875 -8.46521186828613\\
109.887496948242 -8.46589660644531\\
109.900001525879 -8.46657657623291\\
109.912498474121 -8.46725368499756\\
109.925003051758 -8.46793079376221\\
109.9375 -8.46860790252686\\
109.949996948242 -8.4692850112915\\
109.962501525879 -8.46996212005615\\
109.974998474121 -8.4706392288208\\
109.987503051758 -8.47131633758545\\
110 -8.4719934463501\\
110.012496948242 -8.47267055511475\\
110.025001525879 -8.47334766387939\\
110.037498474121 -8.47402477264404\\
110.050003051758 -8.47470188140869\\
110.0625 -8.47537803649902\\
110.074996948242 -8.47605323791504\\
110.087501525879 -8.47672939300537\\
110.099998474121 -8.47740459442139\\
110.112503051758 -8.47808074951172\\
110.125 -8.47875690460205\\
110.137496948242 -8.4794340133667\\
110.150001525879 -8.48011112213135\\
110.162498474121 -8.480788230896\\
110.175003051758 -8.48146533966064\\
110.1875 -8.48214244842529\\
110.199996948242 -8.48282051086426\\
110.212501525879 -8.48350620269775\\
110.224998474121 -8.48419189453125\\
110.237503051758 -8.48487854003906\\
110.25 -8.48556518554688\\
110.262496948242 -8.48625183105469\\
110.275001525879 -8.4869441986084\\
110.287498474121 -8.48763847351074\\
110.300003051758 -8.4883337020874\\
110.3125 -8.48902797698975\\
110.324996948242 -8.48972415924072\\
110.337501525879 -8.4904203414917\\
110.349998474121 -8.49111652374268\\
110.362503051758 -8.49181747436523\\
110.375 -8.49252128601074\\
110.387496948242 -8.49322509765625\\
110.400001525879 -8.49392986297607\\
110.412498474121 -8.49463367462158\\
110.425003051758 -8.49533939361572\\
110.4375 -8.49604511260986\\
110.449996948242 -8.496750831604\\
110.462501525879 -8.49745655059814\\
110.474998474121 -8.49816226959229\\
110.487503051758 -8.49886798858643\\
110.5 -8.49957656860352\\
110.512496948242 -8.50028991699219\\
110.525001525879 -8.50100421905518\\
110.537498474121 -8.50171756744385\\
110.550003051758 -8.50243186950684\\
110.5625 -8.50314521789551\\
110.574996948242 -8.50385856628418\\
110.587501525879 -8.50457286834717\\
110.599998474121 -8.50528621673584\\
110.612503051758 -8.50600051879883\\
110.625 -8.5067138671875\\
110.637496948242 -8.5074291229248\\
110.650001525879 -8.50814437866211\\
110.662498474121 -8.50885963439941\\
110.675003051758 -8.50957489013672\\
110.6875 -8.51029014587402\\
110.699996948242 -8.5110034942627\\
110.712501525879 -8.51171779632568\\
110.724998474121 -8.51243114471436\\
110.737503051758 -8.51314067840576\\
110.75 -8.5138463973999\\
110.762496948242 -8.51455211639404\\
110.775001525879 -8.51525783538818\\
110.787498474121 -8.51596355438232\\
110.800003051758 -8.51666927337646\\
110.8125 -8.51737499237061\\
110.824996948242 -8.51807880401611\\
110.837501525879 -8.51878356933594\\
110.849998474121 -8.51948738098145\\
110.862503051758 -8.52018642425537\\
110.875 -8.52088260650635\\
110.887496948242 -8.52157878875732\\
110.900001525879 -8.5222749710083\\
110.912498474121 -8.52296924591064\\
110.925003051758 -8.5236644744873\\
110.9375 -8.52435874938965\\
110.949996948242 -8.52505302429199\\
110.962501525879 -8.52574443817139\\
110.974998474121 -8.5264310836792\\
110.987503051758 -8.52711772918701\\
111 -8.52780437469482\\
111.012496948242 -8.52849102020264\\
111.025001525879 -8.52917766571045\\
111.037498474121 -8.52986431121826\\
111.050003051758 -8.53055095672607\\
111.0625 -8.53123569488525\\
111.074996948242 -8.53192043304443\\
111.087501525879 -8.53260612487793\\
111.099998474121 -8.53329086303711\\
111.112503051758 -8.53397655487061\\
111.125 -8.53466129302979\\
111.137496948242 -8.53534603118896\\
111.150001525879 -8.53603172302246\\
111.162498474121 -8.53671646118164\\
111.175003051758 -8.53740215301514\\
111.1875 -8.53808689117432\\
111.199996948242 -8.5387716293335\\
111.212501525879 -8.53945732116699\\
111.224998474121 -8.54014205932617\\
111.237503051758 -8.54082775115967\\
111.25 -8.5415096282959\\
111.262496948242 -8.54218673706055\\
111.275001525879 -8.5428638458252\\
111.287498474121 -8.54354095458984\\
111.300003051758 -8.54421806335449\\
111.3125 -8.54489517211914\\
111.324996948242 -8.54557228088379\\
111.337501525879 -8.54624938964844\\
111.349998474121 -8.54692649841309\\
111.362503051758 -8.54760360717773\\
111.375 -8.54828071594238\\
111.387496948242 -8.54895782470703\\
111.400001525879 -8.54963493347168\\
111.412498474121 -8.55031204223633\\
111.425003051758 -8.55098915100098\\
111.4375 -8.55166912078857\\
111.449996948242 -8.55235385894775\\
111.462501525879 -8.55303955078125\\
111.474998474121 -8.55372524261475\\
111.487503051758 -8.55441188812256\\
111.5 -8.55509853363037\\
111.512496948242 -8.55578517913818\\
111.525001525879 -8.556471824646\\
111.537498474121 -8.55716419219971\\
111.550003051758 -8.55785942077637\\
111.5625 -8.55855464935303\\
111.574996948242 -8.559250831604\\
111.587501525879 -8.55995559692383\\
111.599998474121 -8.56066036224365\\
111.612503051758 -8.56136608123779\\
111.625 -8.56207180023193\\
111.637496948242 -8.56277751922607\\
111.650001525879 -8.56348323822021\\
111.662498474121 -8.56418895721436\\
111.675003051758 -8.5648946762085\\
111.6875 -8.56560039520264\\
111.699996948242 -8.56630706787109\\
111.712501525879 -8.56702136993408\\
111.724998474121 -8.56773471832275\\
111.737503051758 -8.56844902038574\\
111.75 -8.56916332244873\\
111.762496948242 -8.56987857818604\\
111.775001525879 -8.57059383392334\\
111.787498474121 -8.57130813598633\\
111.800003051758 -8.57202243804932\\
111.8125 -8.57273578643799\\
111.824996948242 -8.57344913482666\\
111.837501525879 -8.57416343688965\\
111.849998474121 -8.57487106323242\\
111.862503051758 -8.57557678222656\\
111.875 -8.5762825012207\\
111.887496948242 -8.57698822021484\\
111.900001525879 -8.57769393920898\\
111.912498474121 -8.57839965820313\\
111.925003051758 -8.57910537719727\\
111.9375 -8.57980918884277\\
111.949996948242 -8.58051300048828\\
111.962501525879 -8.58121681213379\\
111.974998474121 -8.58191299438477\\
111.987503051758 -8.58260917663574\\
112 -8.58330535888672\\
112.012496948242 -8.58400058746338\\
112.025001525879 -8.58469581604004\\
112.037498474121 -8.58539009094238\\
112.050003051758 -8.58607959747314\\
112.0625 -8.58676624298096\\
112.074996948242 -8.58745288848877\\
112.087501525879 -8.58813953399658\\
112.099998474121 -8.58882617950439\\
112.112503051758 -8.58951187133789\\
112.125 -8.59019660949707\\
112.137496948242 -8.59088134765625\\
112.150001525879 -8.59156703948975\\
112.162498474121 -8.59224700927734\\
112.175003051758 -8.59292411804199\\
112.1875 -8.59360122680664\\
112.199996948242 -8.59427833557129\\
112.212501525879 -8.59495544433594\\
112.224998474121 -8.59563255310059\\
112.237503051758 -8.59630870819092\\
112.25 -8.59698390960693\\
112.262496948242 -8.59765911102295\\
112.275001525879 -8.59833526611328\\
112.287498474121 -8.5990104675293\\
112.300003051758 -8.59968090057373\\
112.3125 -8.60034847259521\\
112.324996948242 -8.6010160446167\\
112.337501525879 -8.60168361663818\\
112.349998474121 -8.60235118865967\\
112.362503051758 -8.60301876068115\\
112.375 -8.60368633270264\\
112.387496948242 -8.60435390472412\\
112.400001525879 -8.60502147674561\\
112.412498474121 -8.60568904876709\\
112.425003051758 -8.60635662078857\\
112.4375 -8.60702419281006\\
112.449996948242 -8.60769176483154\\
112.462501525879 -8.60835933685303\\
112.474998474121 -8.60902690887451\\
112.487503051758 -8.609694480896\\
112.5 -8.61036205291748\\
112.512496948242 -8.61102962493896\\
112.525001525879 -8.61169719696045\\
112.537498474121 -8.61236476898193\\
112.550003051758 -8.61303520202637\\
112.5625 -8.61370277404785\\
112.574996948242 -8.6143741607666\\
112.587501525879 -8.61505031585693\\
112.599998474121 -8.61572551727295\\
112.612503051758 -8.61640167236328\\
112.625 -8.6170768737793\\
112.637496948242 -8.61775398254395\\
112.650001525879 -8.61843109130859\\
112.662498474121 -8.61910820007324\\
112.675003051758 -8.61978530883789\\
112.6875 -8.62046241760254\\
112.699996948242 -8.62113952636719\\
112.712501525879 -8.62181758880615\\
112.724998474121 -8.62250232696533\\
112.737503051758 -8.62318801879883\\
112.75 -8.62387466430664\\
112.762496948242 -8.62456130981445\\
112.775001525879 -8.62525177001953\\
112.787498474121 -8.62594604492188\\
112.800003051758 -8.62664222717285\\
112.8125 -8.62733840942383\\
112.824996948242 -8.6280345916748\\
112.837501525879 -8.62873077392578\\
112.849998474121 -8.62942695617676\\
112.862503051758 -8.63012313842773\\
112.875 -8.63081932067871\\
112.887496948242 -8.63151550292969\\
112.900001525879 -8.63221168518066\\
112.912498474121 -8.63290786743164\\
112.925003051758 -8.63360404968262\\
112.9375 -8.63430023193359\\
112.949996948242 -8.63499641418457\\
112.962501525879 -8.63569641113281\\
112.974998474121 -8.63640022277832\\
112.987503051758 -8.63710498809814\\
113 -8.63780879974365\\
113.012496948242 -8.63851261138916\\
113.025001525879 -8.63921737670898\\
113.037498474121 -8.63991451263428\\
113.050003051758 -8.64061069488525\\
113.0625 -8.64130687713623\\
113.074996948242 -8.64200210571289\\
113.087501525879 -8.64269733428955\\
113.099998474121 -8.64339160919189\\
113.112503051758 -8.64408683776855\\
113.125 -8.64477634429932\\
113.137496948242 -8.64546298980713\\
113.150001525879 -8.64614963531494\\
113.162498474121 -8.64683532714844\\
113.175003051758 -8.64752101898193\\
113.1875 -8.64820575714111\\
113.199996948242 -8.64888477325439\\
113.212501525879 -8.64956188201904\\
113.224998474121 -8.65023899078369\\
113.237503051758 -8.65091609954834\\
113.25 -8.65159130096436\\
113.262496948242 -8.65226650238037\\
113.275001525879 -8.65294075012207\\
113.287498474121 -8.65360832214355\\
113.300003051758 -8.65427589416504\\
113.3125 -8.65494346618652\\
113.324996948242 -8.65561103820801\\
113.337501525879 -8.65627765655518\\
113.349998474121 -8.65694332122803\\
113.362503051758 -8.6576099395752\\
113.375 -8.65827560424805\\
113.387496948242 -8.65893459320068\\
113.400001525879 -8.659592628479\\
113.412498474121 -8.66025066375732\\
113.425003051758 -8.66090869903564\\
113.4375 -8.66156673431396\\
113.449996948242 -8.66222476959229\\
113.462501525879 -8.66288280487061\\
113.474998474121 -8.66353988647461\\
113.487503051758 -8.66419696807861\\
113.5 -8.6648530960083\\
113.512496948242 -8.66550922393799\\
113.525001525879 -8.66616630554199\\
113.537498474121 -8.66682243347168\\
113.550003051758 -8.66747951507568\\
113.5625 -8.66813564300537\\
113.574996948242 -8.66878509521484\\
113.587501525879 -8.66943359375\\
113.599998474121 -8.67008209228516\\
113.612503051758 -8.67073059082031\\
113.625 -8.67137908935547\\
113.637496948242 -8.67202949523926\\
113.650001525879 -8.67268657684326\\
113.662498474121 -8.67334270477295\\
113.675003051758 -8.67399978637695\\
113.6875 -8.67465591430664\\
113.699996948242 -8.67531204223633\\
113.712501525879 -8.67596912384033\\
113.724998474121 -8.67662525177002\\
113.737503051758 -8.67728233337402\\
113.75 -8.67793846130371\\
113.762496948242 -8.6785945892334\\
113.775001525879 -8.67925262451172\\
113.787498474121 -8.67991065979004\\
113.800003051758 -8.68056869506836\\
113.8125 -8.68122673034668\\
113.824996948242 -8.68188571929932\\
113.837501525879 -8.68255233764648\\
113.849998474121 -8.68321800231934\\
113.862503051758 -8.6838846206665\\
113.875 -8.68455219268799\\
113.887496948242 -8.68521976470947\\
113.900001525879 -8.68589115142822\\
113.912498474121 -8.68656635284424\\
113.925003051758 -8.68724250793457\\
113.9375 -8.68791770935059\\
113.949996948242 -8.6885929107666\\
113.962501525879 -8.68926906585693\\
113.974998474121 -8.68994426727295\\
113.987503051758 -8.6906213760376\\
114 -8.69129848480225\\
114.012496948242 -8.69197559356689\\
114.025001525879 -8.69265270233154\\
114.037498474121 -8.69333744049072\\
114.050003051758 -8.69402313232422\\
114.0625 -8.69470977783203\\
114.074996948242 -8.69539642333984\\
114.087501525879 -8.69608306884766\\
114.099998474121 -8.69676971435547\\
114.112503051758 -8.69745635986328\\
114.125 -8.69814300537109\\
114.137496948242 -8.69882869720459\\
114.150001525879 -8.69951438903809\\
114.162498474121 -8.70019912719727\\
114.175003051758 -8.70088481903076\\
114.1875 -8.70156669616699\\
114.199996948242 -8.70224380493164\\
114.212501525879 -8.70292091369629\\
114.224998474121 -8.70359802246094\\
114.237503051758 -8.70427513122559\\
114.25 -8.70495223999023\\
114.262496948242 -8.70562839508057\\
114.275001525879 -8.7063045501709\\
114.287498474121 -8.70697975158691\\
114.300003051758 -8.70764827728271\\
114.3125 -8.7083158493042\\
114.324996948242 -8.70898342132568\\
114.337501525879 -8.70965099334717\\
114.349998474121 -8.71031665802002\\
114.362503051758 -8.71098327636719\\
114.375 -8.71164798736572\\
114.387496948242 -8.71230602264404\\
114.400001525879 -8.71296405792236\\
114.412498474121 -8.71362209320068\\
114.425003051758 -8.714280128479\\
114.4375 -8.71493625640869\\
114.449996948242 -8.71559238433838\\
114.462501525879 -8.71624565124512\\
114.474998474121 -8.71689414978027\\
114.487503051758 -8.71754264831543\\
114.5 -8.71819114685059\\
114.512496948242 -8.71883964538574\\
114.525001525879 -8.7194881439209\\
114.537498474121 -8.72013473510742\\
114.550003051758 -8.72078227996826\\
114.5625 -8.72142887115479\\
114.574996948242 -8.72207546234131\\
114.587501525879 -8.72272205352783\\
114.599998474121 -8.72336101531982\\
114.612503051758 -8.72399997711182\\
114.625 -8.72463893890381\\
114.637496948242 -8.7252779006958\\
114.650001525879 -8.72591686248779\\
114.662498474121 -8.72655582427979\\
114.675003051758 -8.72719478607178\\
114.6875 -8.72783374786377\\
114.699996948242 -8.72847175598145\\
114.712501525879 -8.72910976409912\\
114.724998474121 -8.72974681854248\\
114.737503051758 -8.73038482666016\\
114.75 -8.73102188110352\\
114.762496948242 -8.73165893554688\\
114.775001525879 -8.73229694366455\\
114.787498474121 -8.73293399810791\\
114.800003051758 -8.73357200622559\\
114.8125 -8.73420906066895\\
114.824996948242 -8.7348461151123\\
114.837501525879 -8.73548412322998\\
114.849998474121 -8.73612308502197\\
114.862503051758 -8.73676204681396\\
114.875 -8.73740100860596\\
114.887496948242 -8.73803997039795\\
114.900001525879 -8.73867893218994\\
114.912498474121 -8.73931789398193\\
114.925003051758 -8.73995685577393\\
114.9375 -8.7406005859375\\
114.949996948242 -8.74124717712402\\
114.962501525879 -8.74189472198486\\
114.974998474121 -8.7425422668457\\
114.987503051758 -8.74319076538086\\
115 -8.74383926391602\\
115.012496948242 -8.74448776245117\\
115.025001525879 -8.74514198303223\\
115.037498474121 -8.74579811096191\\
115.050003051758 -8.74645614624023\\
115.0625 -8.74711418151855\\
115.074996948242 -8.74777221679688\\
115.087501525879 -8.74843883514404\\
115.099998474121 -8.74910449981689\\
115.112503051758 -8.74977207183838\\
115.125 -8.75043964385986\\
115.137496948242 -8.7511100769043\\
115.150001525879 -8.75178623199463\\
115.162498474121 -8.75246238708496\\
115.175003051758 -8.75313949584961\\
115.1875 -8.75381660461426\\
115.199996948242 -8.75449371337891\\
115.212501525879 -8.75517082214355\\
115.224998474121 -8.7558479309082\\
115.237503051758 -8.75652503967285\\
115.25 -8.7572021484375\\
115.262496948242 -8.75787925720215\\
115.275001525879 -8.7585563659668\\
115.287498474121 -8.75923347473145\\
115.300003051758 -8.75991058349609\\
115.3125 -8.76058769226074\\
115.324996948242 -8.76126384735107\\
115.337501525879 -8.76194000244141\\
115.349998474121 -8.76261520385742\\
115.362503051758 -8.76328754425049\\
115.375 -8.76395511627197\\
115.387496948242 -8.76462268829346\\
115.400001525879 -8.76529026031494\\
115.412498474121 -8.76595592498779\\
115.425003051758 -8.76662254333496\\
115.4375 -8.76728057861328\\
115.449996948242 -8.7679386138916\\
115.462501525879 -8.76859664916992\\
115.474998474121 -8.76925277709961\\
115.487503051758 -8.76990985870361\\
115.5 -8.77056121826172\\
115.512496948242 -8.77120971679688\\
115.525001525879 -8.77185821533203\\
115.537498474121 -8.77250671386719\\
115.550003051758 -8.77315425872803\\
115.5625 -8.77380084991455\\
115.574996948242 -8.77444553375244\\
115.587501525879 -8.77508449554443\\
115.599998474121 -8.77572345733643\\
115.612503051758 -8.77636241912842\\
115.625 -8.77700138092041\\
115.637496948242 -8.77763843536377\\
115.650001525879 -8.77827644348145\\
115.662498474121 -8.7789134979248\\
115.675003051758 -8.77954483032227\\
115.6875 -8.78017425537109\\
115.699996948242 -8.78080368041992\\
115.712501525879 -8.78143310546875\\
115.724998474121 -8.78206157684326\\
115.737503051758 -8.78269004821777\\
115.75 -8.78331756591797\\
115.762496948242 -8.78394508361816\\
115.775001525879 -8.78456497192383\\
115.787498474121 -8.78518486022949\\
115.800003051758 -8.78580474853516\\
115.8125 -8.78642463684082\\
115.824996948242 -8.78704452514648\\
115.837501525879 -8.78766441345215\\
115.849998474121 -8.78828430175781\\
115.862503051758 -8.78890419006348\\
115.875 -8.78952407836914\\
115.887496948242 -8.7901439666748\\
115.900001525879 -8.79076385498047\\
115.912498474121 -8.79138374328613\\
115.925003051758 -8.7920036315918\\
115.9375 -8.79262351989746\\
115.949996948242 -8.79324626922607\\
115.962501525879 -8.79387474060059\\
115.974998474121 -8.79450225830078\\
115.987503051758 -8.79513072967529\\
116 -8.79575824737549\\
116.012496948242 -8.79638576507568\\
116.025001525879 -8.7970142364502\\
116.037498474121 -8.79764270782471\\
116.050003051758 -8.79827213287354\\
116.0625 -8.79890155792236\\
116.074996948242 -8.79953098297119\\
116.087501525879 -8.80016040802002\\
116.099998474121 -8.80078983306885\\
116.112503051758 -8.80142593383789\\
116.125 -8.80206298828125\\
116.137496948242 -8.80270195007324\\
116.150001525879 -8.80334091186523\\
116.162498474121 -8.80397987365723\\
116.175003051758 -8.80462741851807\\
116.1875 -8.80527496337891\\
116.199996948242 -8.80592346191406\\
116.212501525879 -8.80657196044922\\
116.224998474121 -8.80722045898438\\
116.237503051758 -8.80787372589111\\
116.25 -8.8085298538208\\
116.262496948242 -8.80918788909912\\
116.275001525879 -8.80984592437744\\
116.287498474121 -8.81050395965576\\
116.300003051758 -8.81116199493408\\
116.3125 -8.81182384490967\\
116.324996948242 -8.81248950958252\\
116.337501525879 -8.813157081604\\
116.349998474121 -8.81382465362549\\
116.362503051758 -8.81449222564697\\
116.375 -8.81515979766846\\
116.387496948242 -8.81582736968994\\
116.400001525879 -8.81649494171143\\
116.412498474121 -8.81716156005859\\
116.425003051758 -8.81782817840576\\
116.4375 -8.81849384307861\\
116.449996948242 -8.81915950775146\\
116.462501525879 -8.81982612609863\\
116.474998474121 -8.82048511505127\\
116.487503051758 -8.82114315032959\\
116.5 -8.82180118560791\\
116.512496948242 -8.82245826721191\\
116.525001525879 -8.82311534881592\\
116.537498474121 -8.82377147674561\\
116.550003051758 -8.82442188262939\\
116.5625 -8.82507038116455\\
116.574996948242 -8.82571887969971\\
116.587501525879 -8.82636642456055\\
116.599998474121 -8.82701301574707\\
116.612503051758 -8.82765579223633\\
116.625 -8.82829475402832\\
116.637496948242 -8.82893371582031\\
116.650001525879 -8.8295726776123\\
116.662498474121 -8.83020973205566\\
116.675003051758 -8.83084774017334\\
116.6875 -8.83148002624512\\
116.699996948242 -8.83210945129395\\
116.712501525879 -8.83273887634277\\
116.724998474121 -8.8333683013916\\
116.737503051758 -8.83399677276611\\
116.75 -8.83462429046631\\
116.762496948242 -8.83524703979492\\
116.775001525879 -8.83586692810059\\
116.787498474121 -8.83648681640625\\
116.800003051758 -8.83710670471191\\
116.8125 -8.83772468566895\\
116.824996948242 -8.83834266662598\\
116.837501525879 -8.83896160125732\\
116.849998474121 -8.83957958221436\\
116.862503051758 -8.84019565582275\\
116.875 -8.84080600738525\\
116.887496948242 -8.84141635894775\\
116.900001525879 -8.84202671051025\\
116.912498474121 -8.84263706207275\\
116.925003051758 -8.84324741363525\\
116.9375 -8.84385776519775\\
116.949996948242 -8.84446811676025\\
116.962501525879 -8.84507751464844\\
116.974998474121 -8.8456859588623\\
116.987503051758 -8.84629535675049\\
117 -8.84690380096436\\
117.012496948242 -8.84751224517822\\
117.025001525879 -8.84811687469482\\
117.037498474121 -8.84871768951416\\
117.050003051758 -8.8493185043335\\
117.0625 -8.84991931915283\\
117.074996948242 -8.85052013397217\\
117.087501525879 -8.8511209487915\\
117.099998474121 -8.85172176361084\\
117.112503051758 -8.85232257843018\\
117.125 -8.85292339324951\\
117.137496948242 -8.85352420806885\\
117.150001525879 -8.85412502288818\\
117.162498474121 -8.85472583770752\\
117.175003051758 -8.85532665252686\\
117.1875 -8.85592746734619\\
117.199996948242 -8.85653305053711\\
117.212501525879 -8.85714244842529\\
117.224998474121 -8.85775089263916\\
117.237503051758 -8.85836029052734\\
117.25 -8.85896873474121\\
117.262496948242 -8.85957908630371\\
117.275001525879 -8.86018943786621\\
117.287498474121 -8.86079978942871\\
117.300003051758 -8.86141014099121\\
117.3125 -8.86202049255371\\
117.324996948242 -8.86263561248779\\
117.337501525879 -8.86325454711914\\
117.349998474121 -8.86387252807617\\
117.362503051758 -8.86449241638184\\
117.375 -8.8651123046875\\
117.387496948242 -8.86573219299316\\
117.400001525879 -8.86635971069336\\
117.412498474121 -8.86698722839355\\
117.425003051758 -8.86761569976807\\
117.4375 -8.86824512481689\\
117.449996948242 -8.86887454986572\\
117.462501525879 -8.86950397491455\\
117.474998474121 -8.87013912200928\\
117.487503051758 -8.87077713012695\\
117.5 -8.87141418457031\\
117.512496948242 -8.87205219268799\\
117.525001525879 -8.87269115447998\\
117.537498474121 -8.87333011627197\\
117.550003051758 -8.87396907806396\\
117.5625 -8.87460803985596\\
117.574996948242 -8.87524700164795\\
117.587501525879 -8.87588596343994\\
117.599998474121 -8.87652397155762\\
117.612503051758 -8.87716197967529\\
117.625 -8.87779903411865\\
117.637496948242 -8.87843608856201\\
117.650001525879 -8.87907409667969\\
117.662498474121 -8.87970733642578\\
117.675003051758 -8.88033676147461\\
117.6875 -8.88096618652344\\
117.699996948242 -8.88159561157227\\
117.712501525879 -8.88222503662109\\
117.724998474121 -8.88285255432129\\
117.737503051758 -8.88347911834717\\
117.75 -8.88409900665283\\
117.762496948242 -8.8847188949585\\
117.775001525879 -8.88533878326416\\
117.787498474121 -8.88595676422119\\
117.800003051758 -8.88657474517822\\
117.8125 -8.88718509674072\\
117.824996948242 -8.88779544830322\\
117.837501525879 -8.88840579986572\\
117.849998474121 -8.88901424407959\\
117.862503051758 -8.88961791992188\\
117.875 -8.89021873474121\\
117.887496948242 -8.89081954956055\\
117.900001525879 -8.89142036437988\\
117.912498474121 -8.89201927185059\\
117.925003051758 -8.89261913299561\\
117.9375 -8.89321327209473\\
117.949996948242 -8.8938045501709\\
117.962501525879 -8.89439582824707\\
117.974998474121 -8.89498519897461\\
117.987503051758 -8.89557552337646\\
118 -8.89615917205811\\
118.012496948242 -8.89674091339111\\
118.025001525879 -8.89732265472412\\
118.037498474121 -8.89790439605713\\
118.050003051758 -8.89848518371582\\
118.0625 -8.8990650177002\\
118.074996948242 -8.89964485168457\\
118.087501525879 -8.90021800994873\\
118.099998474121 -8.90079021453857\\
118.112503051758 -8.90136241912842\\
118.125 -8.90193462371826\\
118.137496948242 -8.90250682830811\\
118.150001525879 -8.90307903289795\\
118.162498474121 -8.90365123748779\\
118.175003051758 -8.90422344207764\\
118.1875 -8.90479373931885\\
118.199996948242 -8.90536499023438\\
118.212501525879 -8.9059362411499\\
118.224998474121 -8.90650653839111\\
118.237503051758 -8.90707778930664\\
118.25 -8.90764999389648\\
118.262496948242 -8.90822219848633\\
118.275001525879 -8.90879440307617\\
118.287498474121 -8.90936660766602\\
118.300003051758 -8.90993881225586\\
118.3125 -8.9105110168457\\
118.324996948242 -8.91108322143555\\
118.337501525879 -8.91165542602539\\
118.349998474121 -8.91222763061523\\
118.362503051758 -8.91279983520508\\
118.375 -8.91337871551514\\
118.387496948242 -8.91395854949951\\
118.400001525879 -8.9145393371582\\
118.412498474121 -8.91511917114258\\
118.425003051758 -8.91569995880127\\
118.4375 -8.91627979278564\\
118.449996948242 -8.91686153411865\\
118.462501525879 -8.91744327545166\\
118.474998474121 -8.91802501678467\\
118.487503051758 -8.91860675811768\\
118.5 -8.91918849945068\\
118.512496948242 -8.91977024078369\\
118.525001525879 -8.92035579681396\\
118.537498474121 -8.9209451675415\\
118.550003051758 -8.92153549194336\\
118.5625 -8.9221248626709\\
118.574996948242 -8.92271423339844\\
118.587501525879 -8.92330551147461\\
118.599998474121 -8.92389678955078\\
118.612503051758 -8.92448806762695\\
118.625 -8.92508697509766\\
118.637496948242 -8.92568588256836\\
118.650001525879 -8.9262866973877\\
118.662498474121 -8.92688751220703\\
118.675003051758 -8.92748832702637\\
118.6875 -8.9280891418457\\
118.699996948242 -8.92868995666504\\
118.712501525879 -8.92929363250732\\
118.724998474121 -8.92990207672119\\
118.737503051758 -8.93051147460938\\
118.75 -8.93111991882324\\
118.762496948242 -8.93172836303711\\
118.775001525879 -8.93233776092529\\
118.787498474121 -8.93293952941895\\
118.800003051758 -8.93354034423828\\
118.8125 -8.93414115905762\\
118.824996948242 -8.93474102020264\\
118.837501525879 -8.93534088134766\\
118.849998474121 -8.93593788146973\\
118.862503051758 -8.9365291595459\\
118.875 -8.93712043762207\\
118.887496948242 -8.93771171569824\\
118.900001525879 -8.9383020401001\\
118.912498474121 -8.93889141082764\\
118.925003051758 -8.93947315216064\\
118.9375 -8.94005489349365\\
118.949996948242 -8.94063472747803\\
118.962501525879 -8.94121074676514\\
118.974998474121 -8.94178295135498\\
118.987503051758 -8.94235515594482\\
119 -8.94292640686035\\
119.012496948242 -8.9434928894043\\
119.025001525879 -8.94405555725098\\
119.037498474121 -8.94461822509766\\
119.050003051758 -8.94517993927002\\
119.0625 -8.94573783874512\\
119.074996948242 -8.94629096984863\\
119.087501525879 -8.94684410095215\\
119.099998474121 -8.94739532470703\\
119.112503051758 -8.94794273376465\\
119.125 -8.948486328125\\
119.137496948242 -8.94902992248535\\
119.150001525879 -8.94957256317139\\
119.162498474121 -8.95011425018311\\
119.175003051758 -8.95065021514893\\
119.1875 -8.95118427276611\\
119.199996948242 -8.9517183303833\\
119.212501525879 -8.95225238800049\\
119.224998474121 -8.95278549194336\\
119.237503051758 -8.95331859588623\\
119.25 -8.95385074615479\\
119.262496948242 -8.95438289642334\\
119.275001525879 -8.95491600036621\\
119.287498474121 -8.95544815063477\\
119.300003051758 -8.95598125457764\\
119.3125 -8.95651340484619\\
119.324996948242 -8.95704555511475\\
119.337501525879 -8.95757865905762\\
119.349998474121 -8.95811080932617\\
119.362503051758 -8.95864391326904\\
119.375 -8.9591760635376\\
119.387496948242 -8.95970821380615\\
119.400001525879 -8.96024131774902\\
119.412498474121 -8.96077346801758\\
119.425003051758 -8.96130657196045\\
119.4375 -8.961838722229\\
119.449996948242 -8.96237087249756\\
119.462501525879 -8.96290397644043\\
119.474998474121 -8.9634370803833\\
119.487503051758 -8.96397113800049\\
119.5 -8.96450519561768\\
119.512496948242 -8.96503925323486\\
119.525001525879 -8.96557331085205\\
119.537498474121 -8.96610736846924\\
119.550003051758 -8.96664142608643\\
119.5625 -8.96717548370361\\
119.574996948242 -8.9677095413208\\
119.587501525879 -8.96824359893799\\
119.599998474121 -8.96877765655518\\
119.612503051758 -8.96931171417236\\
119.625 -8.9698486328125\\
119.637496948242 -8.97039031982422\\
119.650001525879 -8.97093296051025\\
119.662498474121 -8.97147464752197\\
119.675003051758 -8.97201728820801\\
119.6875 -8.97255992889404\\
119.699996948242 -8.97310352325439\\
119.712501525879 -8.97364711761475\\
119.724998474121 -8.9741907119751\\
119.737503051758 -8.97473430633545\\
119.75 -8.9752779006958\\
119.762496948242 -8.97582149505615\\
119.775001525879 -8.9763650894165\\
119.787498474121 -8.97690868377686\\
119.800003051758 -8.97745227813721\\
119.8125 -8.97799587249756\\
119.824996948242 -8.97854614257813\\
119.837501525879 -8.97909832000732\\
119.849998474121 -8.97964954376221\\
119.862503051758 -8.98020172119141\\
119.875 -8.98075294494629\\
119.887496948242 -8.9813060760498\\
119.900001525879 -8.98185920715332\\
119.912498474121 -8.98241138458252\\
119.925003051758 -8.98296356201172\\
119.9375 -8.98351383209229\\
119.949996948242 -8.98405742645264\\
119.962501525879 -8.98460102081299\\
119.974998474121 -8.98514461517334\\
119.987503051758 -8.98568820953369\\
120 -8.98622989654541\\
120.012496948242 -8.98677158355713\\
120.025001525879 -8.98731136322021\\
120.037498474121 -8.9878454208374\\
120.050003051758 -8.98837947845459\\
120.0625 -8.98891353607178\\
120.074996948242 -8.98944664001465\\
120.087501525879 -8.98997974395752\\
120.099998474121 -8.99050998687744\\
120.112503051758 -8.99103450775146\\
120.125 -8.99155902862549\\
120.137496948242 -8.99208354949951\\
120.150001525879 -8.99260807037354\\
120.162498474121 -8.99313163757324\\
120.175003051758 -8.99365520477295\\
120.1875 -8.99417304992676\\
120.199996948242 -8.99468803405762\\
120.212501525879 -8.99520301818848\\
120.224998474121 -8.99571800231934\\
120.237503051758 -8.9962329864502\\
120.25 -8.99674606323242\\
120.262496948242 -8.99725341796875\\
120.275001525879 -8.99775886535645\\
120.287498474121 -8.99826335906982\\
120.300003051758 -8.99876689910889\\
120.3125 -8.99926280975342\\
120.324996948242 -8.99975872039795\\
120.337501525879 -9.00025463104248\\
120.349998474121 -9.0007495880127\\
120.362503051758 -9.00124454498291\\
120.375 -9.00173091888428\\
120.387496948242 -9.00221729278564\\
120.400001525879 -9.00270366668701\\
120.412498474121 -9.00319004058838\\
120.425003051758 -9.00367546081543\\
120.4375 -9.00415992736816\\
120.449996948242 -9.00464534759521\\
120.462501525879 -9.00513172149658\\
120.474998474121 -9.00561809539795\\
120.487503051758 -9.00610446929932\\
120.5 -9.00659084320068\\
120.512496948242 -9.00707721710205\\
120.525001525879 -9.00756359100342\\
120.537498474121 -9.00804996490479\\
120.550003051758 -9.00853633880615\\
120.5625 -9.00902271270752\\
120.574996948242 -9.00950908660889\\
120.587501525879 -9.00999546051025\\
120.599998474121 -9.01048469543457\\
120.612503051758 -9.01097679138184\\
120.625 -9.01147079467773\\
120.637496948242 -9.01196479797363\\
120.650001525879 -9.01245212554932\\
120.662498474121 -9.01293849945068\\
120.675003051758 -9.01342487335205\\
120.6875 -9.01391124725342\\
120.699996948242 -9.01439762115479\\
120.712501525879 -9.01488780975342\\
120.724998474121 -9.01538181304932\\
120.737503051758 -9.01587677001953\\
120.75 -9.01637268066406\\
120.762496948242 -9.01686859130859\\
120.775001525879 -9.01736450195313\\
120.787498474121 -9.01786804199219\\
120.800003051758 -9.01837253570557\\
120.8125 -9.01887798309326\\
120.824996948242 -9.01938343048096\\
120.837501525879 -9.01988887786865\\
120.849998474121 -9.02039432525635\\
120.862503051758 -9.02089977264404\\
120.875 -9.02140808105469\\
120.887496948242 -9.02191638946533\\
120.900001525879 -9.02242183685303\\
120.912498474121 -9.02292823791504\\
120.925003051758 -9.02344226837158\\
120.9375 -9.02395725250244\\
120.949996948242 -9.0244722366333\\
120.962501525879 -9.02498722076416\\
120.974998474121 -9.02550220489502\\
120.987503051758 -9.02602577209473\\
121 -9.02654838562012\\
121.012496948242 -9.02707099914551\\
121.025001525879 -9.02759075164795\\
121.037498474121 -9.02810573577881\\
121.050003051758 -9.02862071990967\\
121.0625 -9.02913379669189\\
121.074996948242 -9.02964115142822\\
121.087501525879 -9.03014659881592\\
121.099998474121 -9.0306510925293\\
121.112503051758 -9.03115081787109\\
121.125 -9.03164672851563\\
121.137496948242 -9.03214168548584\\
121.150001525879 -9.03263282775879\\
121.162498474121 -9.03311920166016\\
121.175003051758 -9.03360557556152\\
121.1875 -9.03408908843994\\
121.199996948242 -9.03456592559814\\
121.212501525879 -9.03504276275635\\
121.224998474121 -9.03551959991455\\
121.237503051758 -9.03599548339844\\
121.25 -9.03646373748779\\
121.262496948242 -9.03693103790283\\
121.275001525879 -9.03739833831787\\
121.287498474121 -9.03786373138428\\
121.300003051758 -9.03832817077637\\
121.3125 -9.03878593444824\\
121.324996948242 -9.03924369812012\\
121.337501525879 -9.03970146179199\\
121.349998474121 -9.04015922546387\\
121.362503051758 -9.04061698913574\\
121.375 -9.04107284545898\\
121.387496948242 -9.0415210723877\\
121.400001525879 -9.04196929931641\\
121.412498474121 -9.04241752624512\\
121.425003051758 -9.04286479949951\\
121.4375 -9.04331111907959\\
121.449996948242 -9.04375076293945\\
121.462501525879 -9.044189453125\\
121.474998474121 -9.04462814331055\\
121.487503051758 -9.04506683349609\\
121.5 -9.04550552368164\\
121.512496948242 -9.04594421386719\\
121.525001525879 -9.04638290405273\\
121.537498474121 -9.04682159423828\\
121.550003051758 -9.04726028442383\\
121.5625 -9.04769897460938\\
121.574996948242 -9.04813766479492\\
121.587501525879 -9.04857635498047\\
121.599998474121 -9.04901504516602\\
121.612503051758 -9.04945373535156\\
121.625 -9.04989242553711\\
121.637496948242 -9.05033111572266\\
121.650001525879 -9.0507698059082\\
121.662498474121 -9.05120849609375\\
121.675003051758 -9.0516471862793\\
121.6875 -9.05208587646484\\
121.699996948242 -9.05252456665039\\
121.712501525879 -9.05296325683594\\
121.724998474121 -9.05340194702148\\
121.737503051758 -9.05384063720703\\
121.75 -9.05427932739258\\
121.762496948242 -9.05471801757813\\
121.775001525879 -9.05515670776367\\
121.787498474121 -9.05559539794922\\
121.800003051758 -9.05603408813477\\
121.8125 -9.05647277832031\\
121.824996948242 -9.05691146850586\\
121.837501525879 -9.05735015869141\\
121.849998474121 -9.05778980255127\\
121.862503051758 -9.05823707580566\\
121.875 -9.05868339538574\\
121.887496948242 -9.05913162231445\\
121.900001525879 -9.05957984924316\\
121.912498474121 -9.06002807617188\\
121.925003051758 -9.06047630310059\\
121.9375 -9.0609245300293\\
121.949996948242 -9.06137275695801\\
121.962501525879 -9.06182098388672\\
121.974998474121 -9.06227207183838\\
121.987503051758 -9.06272888183594\\
122 -9.06318473815918\\
122.012496948242 -9.06364250183105\\
122.025001525879 -9.06410026550293\\
122.037498474121 -9.0645580291748\\
122.050003051758 -9.06501579284668\\
122.0625 -9.06547164916992\\
122.074996948242 -9.06592750549316\\
122.087501525879 -9.06638431549072\\
122.099998474121 -9.06684017181396\\
122.112503051758 -9.06729698181152\\
122.125 -9.06774997711182\\
122.137496948242 -9.06819820404053\\
122.150001525879 -9.06864643096924\\
122.162498474121 -9.06909465789795\\
122.175003051758 -9.06954288482666\\
122.1875 -9.06999111175537\\
122.199996948242 -9.07043933868408\\
122.212501525879 -9.07088375091553\\
122.224998474121 -9.07132244110107\\
122.237503051758 -9.07176113128662\\
122.25 -9.07219982147217\\
122.262496948242 -9.07263660430908\\
122.275001525879 -9.07306861877441\\
122.287498474121 -9.0734977722168\\
122.300003051758 -9.07392692565918\\
122.3125 -9.07435607910156\\
122.324996948242 -9.07478427886963\\
122.337501525879 -9.0752124786377\\
122.349998474121 -9.07563495635986\\
122.362503051758 -9.07605457305908\\
122.375 -9.0764741897583\\
122.387496948242 -9.07689380645752\\
122.400001525879 -9.07731246948242\\
122.412498474121 -9.07772731781006\\
122.425003051758 -9.07813739776611\\
122.4375 -9.07854747772217\\
122.449996948242 -9.07895755767822\\
122.462501525879 -9.07936763763428\\
122.474998474121 -9.0797758102417\\
122.487503051758 -9.08018493652344\\
122.5 -9.08058834075928\\
122.512496948242 -9.08098888397217\\
122.525001525879 -9.08138942718506\\
122.537498474121 -9.08178997039795\\
122.550003051758 -9.08219051361084\\
122.5625 -9.08259105682373\\
122.574996948242 -9.08299160003662\\
122.587501525879 -9.0833911895752\\
122.599998474121 -9.08378982543945\\
122.612503051758 -9.08418941497803\\
122.625 -9.08458805084229\\
122.637496948242 -9.08498668670654\\
122.650001525879 -9.08538627624512\\
122.662498474121 -9.08578491210938\\
122.675003051758 -9.08618450164795\\
122.6875 -9.08658313751221\\
122.699996948242 -9.08698177337646\\
122.712501525879 -9.08738136291504\\
122.724998474121 -9.0877799987793\\
122.737503051758 -9.08817958831787\\
122.75 -9.08857822418213\\
122.762496948242 -9.08897686004639\\
122.775001525879 -9.08937740325928\\
122.787498474121 -9.08977794647217\\
122.800003051758 -9.09017848968506\\
122.8125 -9.09057903289795\\
122.824996948242 -9.09097957611084\\
122.837501525879 -9.09138011932373\\
122.849998474121 -9.09178066253662\\
122.862503051758 -9.09218120574951\\
122.875 -9.09258937835693\\
122.887496948242 -9.09299850463867\\
122.900001525879 -9.09340858459473\\
122.912498474121 -9.09381866455078\\
122.925003051758 -9.09422874450684\\
122.9375 -9.09463882446289\\
122.949996948242 -9.09504890441895\\
122.962501525879 -9.09545993804932\\
122.974998474121 -9.0958776473999\\
122.987503051758 -9.0962963104248\\
123 -9.09671401977539\\
123.012496948242 -9.09713172912598\\
123.025001525879 -9.09755039215088\\
123.037498474121 -9.09796810150146\\
123.050003051758 -9.09838676452637\\
123.0625 -9.09880447387695\\
123.074996948242 -9.09922027587891\\
123.087501525879 -9.09963035583496\\
123.099998474121 -9.10004043579102\\
123.112503051758 -9.10045051574707\\
123.125 -9.10086059570313\\
123.137496948242 -9.10127067565918\\
123.150001525879 -9.10168075561523\\
123.162498474121 -9.10209083557129\\
123.175003051758 -9.10250091552734\\
123.1875 -9.10291004180908\\
123.199996948242 -9.1033182144165\\
123.212501525879 -9.10371875762939\\
123.224998474121 -9.10411930084229\\
123.237503051758 -9.10451984405518\\
123.25 -9.10492038726807\\
123.262496948242 -9.10531902313232\\
123.275001525879 -9.10571002960205\\
123.287498474121 -9.10610103607178\\
123.300003051758 -9.1064920425415\\
123.3125 -9.1068811416626\\
123.324996948242 -9.10726547241211\\
123.337501525879 -9.10764694213867\\
123.349998474121 -9.10802841186523\\
123.362503051758 -9.1084098815918\\
123.375 -9.10878849029541\\
123.387496948242 -9.10916042327881\\
123.400001525879 -9.10953235626221\\
123.412498474121 -9.10990428924561\\
123.425003051758 -9.110276222229\\
123.4375 -9.1106481552124\\
123.449996948242 -9.11101818084717\\
123.462501525879 -9.1113862991333\\
123.474998474121 -9.11174869537354\\
123.487503051758 -9.11211109161377\\
123.5 -9.112473487854\\
123.512496948242 -9.11283588409424\\
123.525001525879 -9.11319828033447\\
123.537498474121 -9.11355876922607\\
123.550003051758 -9.11392021179199\\
123.5625 -9.11427307128906\\
123.574996948242 -9.11462783813477\\
123.587501525879 -9.11498928070068\\
123.599998474121 -9.1153507232666\\
123.612503051758 -9.11571311950684\\
123.625 -9.11607551574707\\
123.637496948242 -9.1164379119873\\
123.650001525879 -9.11680030822754\\
123.662498474121 -9.11716175079346\\
123.675003051758 -9.11752319335938\\
123.6875 -9.11788368225098\\
123.699996948242 -9.11824417114258\\
123.712501525879 -9.1186056137085\\
123.724998474121 -9.1189661026001\\
123.737503051758 -9.11932754516602\\
123.75 -9.11968803405762\\
123.762496948242 -9.12004852294922\\
123.775001525879 -9.12040996551514\\
123.787498474121 -9.12077045440674\\
123.800003051758 -9.12113189697266\\
123.8125 -9.12149429321289\\
123.824996948242 -9.12185668945313\\
123.837501525879 -9.12221908569336\\
123.849998474121 -9.12258148193359\\
123.862503051758 -9.12294387817383\\
123.875 -9.12330627441406\\
123.887496948242 -9.1236686706543\\
123.900001525879 -9.12403583526611\\
123.912498474121 -9.12440586090088\\
123.925003051758 -9.12477779388428\\
123.9375 -9.12514972686768\\
123.949996948242 -9.12552165985107\\
123.962501525879 -9.12589359283447\\
123.974998474121 -9.12626552581787\\
123.987503051758 -9.12663745880127\\
124 -9.12700939178467\\
124.012496948242 -9.12738132476807\\
124.025001525879 -9.12775707244873\\
124.037498474121 -9.12813663482666\\
124.050003051758 -9.12851715087891\\
124.0625 -9.12889671325684\\
124.074996948242 -9.12927436828613\\
124.087501525879 -9.12964630126953\\
124.099998474121 -9.13001823425293\\
124.112503051758 -9.13039016723633\\
124.125 -9.13076210021973\\
124.137496948242 -9.13113403320313\\
124.150001525879 -9.13150596618652\\
124.162498474121 -9.13187789916992\\
124.175003051758 -9.13224983215332\\
124.1875 -9.13262176513672\\
124.199996948242 -9.13299369812012\\
124.212501525879 -9.1333646774292\\
124.224998474121 -9.13372707366943\\
124.237503051758 -9.13408946990967\\
124.25 -9.1344518661499\\
124.262496948242 -9.13481330871582\\
124.275001525879 -9.13517093658447\\
124.287498474121 -9.13552379608154\\
124.300003051758 -9.13587665557861\\
124.3125 -9.13622951507568\\
124.324996948242 -9.13658046722412\\
124.337501525879 -9.13692760467529\\
124.349998474121 -9.1372709274292\\
124.362503051758 -9.13761425018311\\
124.375 -9.13795757293701\\
124.387496948242 -9.13829803466797\\
124.400001525879 -9.13863182067871\\
124.412498474121 -9.13896560668945\\
124.425003051758 -9.1392993927002\\
124.4375 -9.13963222503662\\
124.449996948242 -9.13995742797852\\
124.462501525879 -9.14028167724609\\
124.474998474121 -9.14060592651367\\
124.487503051758 -9.14093017578125\\
124.5 -9.14125442504883\\
124.512496948242 -9.14157867431641\\
124.525001525879 -9.14190196990967\\
124.537498474121 -9.14222431182861\\
124.550003051758 -9.14254760742188\\
124.5625 -9.14287090301514\\
124.574996948242 -9.14319515228271\\
124.587501525879 -9.14352416992188\\
124.599998474121 -9.14384937286377\\
124.612503051758 -9.14417362213135\\
124.625 -9.14449787139893\\
124.637496948242 -9.1448221206665\\
124.650001525879 -9.14514636993408\\
124.662498474121 -9.14547061920166\\
124.675003051758 -9.14579486846924\\
124.6875 -9.14611911773682\\
124.699996948242 -9.14644336700439\\
124.712501525879 -9.14676761627197\\
124.724998474121 -9.14709186553955\\
124.737503051758 -9.14741611480713\\
124.75 -9.14774036407471\\
124.762496948242 -9.14806461334229\\
124.775001525879 -9.14838886260986\\
124.787498474121 -9.14871311187744\\
124.800003051758 -9.14903736114502\\
124.8125 -9.1493616104126\\
124.824996948242 -9.14968585968018\\
124.837501525879 -9.15001392364502\\
124.849998474121 -9.15034580230713\\
124.862503051758 -9.15067958831787\\
124.875 -9.15101337432861\\
124.887496948242 -9.15134716033936\\
124.900001525879 -9.1516809463501\\
124.912498474121 -9.15201473236084\\
124.925003051758 -9.15234851837158\\
124.9375 -9.15268230438232\\
124.949996948242 -9.15301609039307\\
124.962501525879 -9.15334987640381\\
124.974998474121 -9.15368366241455\\
124.987503051758 -9.15401744842529\\
125 -9.15435123443604\\
125.012496948242 -9.15468502044678\\
125.025001525879 -9.15501880645752\\
125.037498474121 -9.15535259246826\\
125.050003051758 -9.155686378479\\
125.0625 -9.15602016448975\\
125.074996948242 -9.15635395050049\\
125.087501525879 -9.15668773651123\\
125.099998474121 -9.15702152252197\\
125.112503051758 -9.15735530853271\\
125.125 -9.15768909454346\\
125.137496948242 -9.1580228805542\\
125.150001525879 -9.15835666656494\\
125.162498474121 -9.15869045257568\\
125.175003051758 -9.15902328491211\\
125.1875 -9.15935325622559\\
125.199996948242 -9.15967750549316\\
125.212501525879 -9.16000175476074\\
125.224998474121 -9.16032600402832\\
125.237503051758 -9.1606502532959\\
125.25 -9.16097450256348\\
125.262496948242 -9.16129875183105\\
125.275001525879 -9.16162300109863\\
125.287498474121 -9.16194725036621\\
125.300003051758 -9.16227149963379\\
125.3125 -9.16259574890137\\
125.324996948242 -9.16291809082031\\
125.337501525879 -9.16323566436768\\
125.349998474121 -9.16355037689209\\
125.362503051758 -9.1638650894165\\
125.375 -9.16417980194092\\
125.387496948242 -9.16449069976807\\
125.400001525879 -9.16479587554932\\
125.412498474121 -9.16510105133057\\
125.425003051758 -9.16540622711182\\
125.4375 -9.16571140289307\\
125.449996948242 -9.16601371765137\\
125.462501525879 -9.16630935668945\\
125.474998474121 -9.16660499572754\\
125.487503051758 -9.16690063476563\\
125.5 -9.16719627380371\\
125.512496948242 -9.1674919128418\\
125.525001525879 -9.16778659820557\\
125.537498474121 -9.16808032989502\\
125.550003051758 -9.16836833953857\\
125.5625 -9.1686544418335\\
125.574996948242 -9.16894054412842\\
125.587501525879 -9.16922664642334\\
125.599998474121 -9.16951274871826\\
125.612503051758 -9.16979885101318\\
125.625 -9.17008495330811\\
125.637496948242 -9.17037105560303\\
125.650001525879 -9.17065715789795\\
125.662498474121 -9.17094326019287\\
125.675003051758 -9.17122936248779\\
125.6875 -9.17151546478271\\
125.699996948242 -9.1718053817749\\
125.712501525879 -9.17210006713867\\
125.724998474121 -9.17239570617676\\
125.737503051758 -9.17269134521484\\
125.75 -9.17298698425293\\
125.762496948242 -9.17328262329102\\
125.775001525879 -9.1735782623291\\
125.787498474121 -9.17387390136719\\
125.800003051758 -9.17416954040527\\
125.8125 -9.17446899414063\\
125.824996948242 -9.17477321624756\\
125.837501525879 -9.17507839202881\\
125.849998474121 -9.17538356781006\\
125.862503051758 -9.17568874359131\\
125.875 -9.17599391937256\\
125.887496948242 -9.17630004882813\\
125.900001525879 -9.17661380767822\\
125.912498474121 -9.17692756652832\\
125.925003051758 -9.17724227905273\\
125.9375 -9.17755699157715\\
125.949996948242 -9.17787170410156\\
125.962501525879 -9.17818641662598\\
125.974998474121 -9.17850112915039\\
125.987503051758 -9.1788158416748\\
126 -9.17913055419922\\
126.012496948242 -9.17944526672363\\
126.025001525879 -9.17975997924805\\
126.037498474121 -9.18007469177246\\
126.050003051758 -9.18038940429688\\
126.0625 -9.18070411682129\\
126.074996948242 -9.1810188293457\\
126.087501525879 -9.1813325881958\\
126.099998474121 -9.18163871765137\\
126.112503051758 -9.18194389343262\\
126.125 -9.18224906921387\\
126.137496948242 -9.1825532913208\\
126.150001525879 -9.18285655975342\\
126.162498474121 -9.1831521987915\\
126.175003051758 -9.18344783782959\\
126.1875 -9.18374347686768\\
126.199996948242 -9.18403911590576\\
126.212501525879 -9.18433380126953\\
126.224998474121 -9.18462467193604\\
126.237503051758 -9.18491077423096\\
126.25 -9.18519687652588\\
126.262496948242 -9.1854829788208\\
126.275001525879 -9.18576908111572\\
126.287498474121 -9.18605518341064\\
126.300003051758 -9.18634033203125\\
126.3125 -9.18661785125732\\
126.324996948242 -9.18689441680908\\
126.337501525879 -9.18717098236084\\
126.349998474121 -9.1874475479126\\
126.362503051758 -9.18772411346436\\
126.375 -9.18800067901611\\
126.387496948242 -9.18827724456787\\
126.400001525879 -9.18855381011963\\
126.412498474121 -9.18883037567139\\
126.425003051758 -9.18910694122314\\
126.4375 -9.1893835067749\\
126.449996948242 -9.18966007232666\\
126.462501525879 -9.18993663787842\\
126.474998474121 -9.19021129608154\\
126.487503051758 -9.19047832489014\\
126.5 -9.19074535369873\\
126.512496948242 -9.19101238250732\\
126.525001525879 -9.19127941131592\\
126.537498474121 -9.19154644012451\\
126.550003051758 -9.19181346893311\\
126.5625 -9.19207859039307\\
126.574996948242 -9.19234371185303\\
126.587501525879 -9.1926097869873\\
126.599998474121 -9.19287490844727\\
126.612503051758 -9.19314098358154\\
126.625 -9.19340705871582\\
126.637496948242 -9.19367408752441\\
126.650001525879 -9.19394111633301\\
126.662498474121 -9.1942081451416\\
126.675003051758 -9.1944751739502\\
126.6875 -9.19474220275879\\
126.699996948242 -9.19500923156738\\
126.712501525879 -9.19527626037598\\
126.724998474121 -9.1955451965332\\
126.737503051758 -9.19582080841064\\
126.75 -9.1960973739624\\
126.762496948242 -9.19637393951416\\
126.775001525879 -9.19665050506592\\
126.787498474121 -9.19692707061768\\
126.800003051758 -9.19720363616943\\
126.8125 -9.19748306274414\\
126.824996948242 -9.19776725769043\\
126.837501525879 -9.19805240631104\\
126.849998474121 -9.19833850860596\\
126.862503051758 -9.19862461090088\\
126.875 -9.1989107131958\\
126.887496948242 -9.19919681549072\\
126.900001525879 -9.19948291778564\\
126.912498474121 -9.19976902008057\\
126.925003051758 -9.20005512237549\\
126.9375 -9.20034599304199\\
126.949996948242 -9.20063972473145\\
126.962501525879 -9.20093441009521\\
126.974998474121 -9.20122814178467\\
126.987503051758 -9.20152282714844\\
127 -9.20181655883789\\
127.012496948242 -9.20210552215576\\
127.025001525879 -9.20239162445068\\
127.037498474121 -9.20267772674561\\
127.050003051758 -9.20296382904053\\
127.0625 -9.20324993133545\\
127.074996948242 -9.20353603363037\\
127.087501525879 -9.20382213592529\\
127.099998474121 -9.20410633087158\\
127.112503051758 -9.20438480377197\\
127.125 -9.20466136932373\\
127.137496948242 -9.20493793487549\\
127.150001525879 -9.20521450042725\\
127.162498474121 -9.205491065979\\
127.175003051758 -9.20576763153076\\
127.1875 -9.20604419708252\\
127.199996948242 -9.20632076263428\\
127.212501525879 -9.20659732818604\\
127.224998474121 -9.20687389373779\\
127.237503051758 -9.20714950561523\\
127.25 -9.20742416381836\\
127.262496948242 -9.20769119262695\\
127.275001525879 -9.20795822143555\\
127.287498474121 -9.20822525024414\\
127.300003051758 -9.20849132537842\\
127.3125 -9.20874881744385\\
127.324996948242 -9.20900630950928\\
127.337501525879 -9.20926380157471\\
127.349998474121 -9.20952129364014\\
127.362503051758 -9.20977878570557\\
127.375 -9.21003532409668\\
127.387496948242 -9.21029090881348\\
127.400001525879 -9.21053886413574\\
127.412498474121 -9.21078681945801\\
127.425003051758 -9.21103477478027\\
127.4375 -9.21128273010254\\
127.449996948242 -9.2115306854248\\
127.462501525879 -9.21177864074707\\
127.474998474121 -9.21202659606934\\
127.487503051758 -9.2122745513916\\
127.5 -9.21252250671387\\
127.512496948242 -9.21277046203613\\
127.525001525879 -9.2130184173584\\
127.537498474121 -9.21326637268066\\
127.550003051758 -9.21351432800293\\
127.5625 -9.2137622833252\\
127.574996948242 -9.21401214599609\\
127.587501525879 -9.21426868438721\\
127.599998474121 -9.21452617645264\\
127.612503051758 -9.21478366851807\\
127.625 -9.2150411605835\\
127.637496948242 -9.21530055999756\\
127.650001525879 -9.21556758880615\\
127.662498474121 -9.21584129333496\\
127.675003051758 -9.21612453460693\\
127.6875 -9.21642208099365\\
127.699996948242 -9.21673202514648\\
127.712501525879 -9.21705055236816\\
127.724998474121 -9.21736717224121\\
127.737503051758 -9.21767616271973\\
127.75 -9.21797370910645\\
127.762496948242 -9.21826267242432\\
127.775001525879 -9.21854877471924\\
127.787498474121 -9.218825340271\\
127.800003051758 -9.21910095214844\\
127.8125 -9.21936798095703\\
127.824996948242 -9.21962833404541\\
127.837501525879 -9.21988296508789\\
127.849998474121 -9.22013092041016\\
127.862503051758 -9.22037887573242\\
127.875 -9.22062683105469\\
127.887496948242 -9.22087287902832\\
127.900001525879 -9.22111797332764\\
127.912498474121 -9.22135639190674\\
127.925003051758 -9.22159481048584\\
127.9375 -9.22183322906494\\
127.949996948242 -9.22207164764404\\
127.962501525879 -9.22231006622314\\
127.974998474121 -9.22254848480225\\
127.987503051758 -9.22278690338135\\
128 -9.22302532196045\\
128.012496948242 -9.22326374053955\\
128.024993896484 -9.22350215911865\\
128.037506103516 -9.22374057769775\\
128.050003051758 -9.22397899627686\\
128.0625 -9.22421741485596\\
128.074996948242 -9.22445583343506\\
128.087493896484 -9.22469425201416\\
128.100006103516 -9.22493267059326\\
128.112503051758 -9.22517108917236\\
128.125 -9.22540950775146\\
128.137496948242 -9.22564792633057\\
128.149993896484 -9.22588634490967\\
128.162506103516 -9.22613048553467\\
128.175003051758 -9.22637748718262\\
128.1875 -9.22662544250488\\
128.199996948242 -9.22687339782715\\
128.212493896484 -9.22712135314941\\
128.225006103516 -9.22737693786621\\
128.237503051758 -9.22763347625732\\
128.25 -9.22789096832275\\
128.262496948242 -9.22815036773682\\
128.274993896484 -9.22841644287109\\
128.287506103516 -9.22868347167969\\
128.300003051758 -9.22895050048828\\
128.3125 -9.22921752929688\\
128.324996948242 -9.22949123382568\\
128.337493896484 -9.22976684570313\\
128.350006103516 -9.23004341125488\\
128.362503051758 -9.23031997680664\\
128.375 -9.2305965423584\\
128.387496948242 -9.23087310791016\\
128.399993896484 -9.23114967346191\\
128.412506103516 -9.23142623901367\\
128.425003051758 -9.23170280456543\\
128.4375 -9.23197937011719\\
128.449996948242 -9.23225498199463\\
128.462493896484 -9.23253059387207\\
128.475006103516 -9.2327995300293\\
128.487503051758 -9.23306655883789\\
128.5 -9.23333358764648\\
128.512496948242 -9.23360061645508\\
128.524993896484 -9.23386669158936\\
128.537506103516 -9.23412895202637\\
128.550003051758 -9.2343864440918\\
128.5625 -9.23464393615723\\
128.574996948242 -9.23490142822266\\
128.587493896484 -9.23515796661377\\
128.600006103516 -9.23540687561035\\
128.612503051758 -9.23565483093262\\
128.625 -9.23590278625488\\
128.637496948242 -9.23614978790283\\
128.649993896484 -9.23639678955078\\
128.662506103516 -9.23663520812988\\
128.675003051758 -9.23687362670898\\
128.6875 -9.23711204528809\\
128.699996948242 -9.23735046386719\\
128.712493896484 -9.23758792877197\\
128.725006103516 -9.23782634735107\\
128.737503051758 -9.23806190490723\\
128.75 -9.23829078674316\\
128.762496948242 -9.2385196685791\\
128.774993896484 -9.23874855041504\\
128.787506103516 -9.23897743225098\\
128.800003051758 -9.23920631408691\\
128.8125 -9.23943519592285\\
128.824996948242 -9.23966407775879\\
128.837493896484 -9.23989295959473\\
128.850006103516 -9.24012184143066\\
128.862503051758 -9.2403507232666\\
128.875 -9.24057960510254\\
128.887496948242 -9.24080848693848\\
128.899993896484 -9.24103736877441\\
128.912506103516 -9.24126625061035\\
128.925003051758 -9.24149513244629\\
128.9375 -9.24172401428223\\
128.949996948242 -9.2419605255127\\
128.962493896484 -9.24219799041748\\
128.975006103516 -9.24243640899658\\
128.987503051758 -9.24267482757568\\
129 -9.24291324615479\\
129.012496948242 -9.24315166473389\\
129.024993896484 -9.24339008331299\\
129.037506103516 -9.24363136291504\\
129.050003051758 -9.24387836456299\\
129.0625 -9.24412631988525\\
129.074996948242 -9.24437427520752\\
129.087493896484 -9.24462223052979\\
129.100006103516 -9.24487018585205\\
129.112503051758 -9.24512386322021\\
129.125 -9.24538040161133\\
129.137496948242 -9.24563694000244\\
129.149993896484 -9.24589443206787\\
129.162506103516 -9.2461519241333\\
129.175003051758 -9.24640941619873\\
129.1875 -9.24666690826416\\
129.199996948242 -9.24692440032959\\
129.212493896484 -9.24718189239502\\
129.225006103516 -9.24743938446045\\
129.237503051758 -9.24769687652588\\
129.25 -9.24795436859131\\
129.262496948242 -9.24821186065674\\
129.274993896484 -9.24846935272217\\
129.287506103516 -9.2487268447876\\
129.300003051758 -9.24898338317871\\
129.3125 -9.24923992156982\\
129.324996948242 -9.24949359893799\\
129.337493896484 -9.24974155426025\\
129.350006103516 -9.24998950958252\\
129.362503051758 -9.25023746490479\\
129.375 -9.25048542022705\\
129.387496948242 -9.250732421875\\
129.399993896484 -9.25097942352295\\
129.412506103516 -9.25122547149658\\
129.425003051758 -9.25146389007568\\
129.4375 -9.25170230865479\\
129.449996948242 -9.25194072723389\\
129.462493896484 -9.25217914581299\\
129.475006103516 -9.25241756439209\\
129.487503051758 -9.25265598297119\\
129.5 -9.25289344787598\\
129.512496948242 -9.25313091278076\\
129.524993896484 -9.25336456298828\\
129.537506103516 -9.25359344482422\\
129.550003051758 -9.25382232666016\\
129.5625 -9.25405120849609\\
129.574996948242 -9.25428009033203\\
129.587493896484 -9.25450897216797\\
129.600006103516 -9.25473785400391\\
129.612503051758 -9.25496673583984\\
129.625 -9.25519561767578\\
129.637496948242 -9.25542449951172\\
129.649993896484 -9.25565338134766\\
129.662506103516 -9.25588226318359\\
129.675003051758 -9.25611114501953\\
129.6875 -9.25634002685547\\
129.699996948242 -9.25656890869141\\
129.712493896484 -9.25679779052734\\
129.725006103516 -9.25702667236328\\
129.737503051758 -9.25725555419922\\
129.75 -9.25748729705811\\
129.762496948242 -9.25772476196289\\
129.774993896484 -9.25796222686768\\
129.787506103516 -9.25820064544678\\
129.800003051758 -9.25843811035156\\
129.8125 -9.25867652893066\\
129.824996948242 -9.25891494750977\\
129.837493896484 -9.25915336608887\\
129.850006103516 -9.25939178466797\\
129.862503051758 -9.25963020324707\\
129.875 -9.25987434387207\\
129.887496948242 -9.26012134552002\\
129.899993896484 -9.26036834716797\\
129.912506103516 -9.26061630249023\\
129.925003051758 -9.2608642578125\\
129.9375 -9.26111221313477\\
129.949996948242 -9.26136016845703\\
129.962493896484 -9.2616081237793\\
129.975006103516 -9.26185607910156\\
129.987503051758 -9.26210784912109\\
130 -9.26236438751221\\
130.012496948242 -9.26262092590332\\
130.024993896484 -9.26287078857422\\
130.037506103516 -9.26311874389648\\
130.050003051758 -9.26336860656738\\
130.0625 -9.26362609863281\\
130.074996948242 -9.26389122009277\\
130.087493896484 -9.26416110992432\\
130.100006103516 -9.26443767547607\\
130.112503051758 -9.26471424102783\\
130.125 -9.26499652862549\\
130.137496948242 -9.26528263092041\\
130.149993896484 -9.26556873321533\\
130.162506103516 -9.26585483551025\\
130.175003051758 -9.26614093780518\\
130.1875 -9.2664270401001\\
130.199996948242 -9.26671314239502\\
130.212493896484 -9.26699924468994\\
130.225006103516 -9.26728248596191\\
130.237503051758 -9.26755905151367\\
130.25 -9.26783561706543\\
130.262496948242 -9.26811218261719\\
130.274993896484 -9.26838874816895\\
130.287506103516 -9.26867008209229\\
130.300003051758 -9.26895523071289\\
130.3125 -9.2692403793335\\
130.324996948242 -9.2695255279541\\
130.337493896484 -9.26981163024902\\
130.350006103516 -9.27009773254395\\
130.362503051758 -9.2703914642334\\
130.375 -9.27069091796875\\
130.387496948242 -9.2710075378418\\
130.399993896484 -9.27136325836182\\
130.412506103516 -9.27195739746094\\
130.425003051758 -9.27400493621826\\
130.4375 -9.27865791320801\\
130.449996948242 -9.28772735595703\\
130.462493896484 -9.3040132522583\\
130.475006103516 -9.323073387146\\
130.487503051758 -9.34279823303223\\
130.5 -9.36238098144531\\
130.512496948242 -9.38151454925537\\
130.524993896484 -9.40002250671387\\
130.537506103516 -9.4177713394165\\
130.550003051758 -9.43457126617432\\
130.5625 -9.45029640197754\\
130.574996948242 -9.46486186981201\\
130.587493896484 -9.47838497161865\\
130.600006103516 -9.49124622344971\\
130.612503051758 -9.503830909729\\
130.625 -9.51605892181396\\
130.637496948242 -9.52770805358887\\
130.649993896484 -9.53872013092041\\
130.662506103516 -9.54906940460205\\
130.675003051758 -9.55861949920654\\
130.6875 -9.56728076934814\\
130.699996948242 -9.5750093460083\\
130.712493896484 -9.58178806304932\\
130.725006103516 -9.58769702911377\\
130.737503051758 -9.6046085357666\\
130.75 -9.64610004425049\\
130.762496948242 -9.68542957305908\\
130.774993896484 -9.72397804260254\\
130.787399291992 -9.75996398925781\\
130.799896240234 -9.79156398773193\\
130.812408447266 -9.81585597991943\\
130.82275390625 -9.82647705078125\\
130.834869384766 -9.83170890808105\\
130.847366333008 -9.83519172668457\\
130.85986328125 -9.83797836303711\\
130.872360229492 -9.84050369262695\\
130.884872436523 -9.84295463562012\\
130.897369384766 -9.84531307220459\\
130.909866333008 -9.84762001037598\\
130.92236328125 -9.84988880157471\\
130.934860229492 -9.85210609436035\\
130.947372436523 -9.85428428649902\\
130.959869384766 -9.85642528533936\\
130.972366333008 -9.85886192321777\\
130.98486328125 -9.86223602294922\\
130.997360229492 -9.88595104217529\\
131.009872436523 -9.93112373352051\\
131.022369384766 -9.98200035095215\\
131.034866333008 -10.0144786834717\\
131.04736328125 -10.0391664505005\\
131.059860229492 -10.0601329803467\\
131.072372436523 -10.0750322341919\\
131.084869384766 -10.0867557525635\\
131.097366333008 -10.0968208312988\\
131.10986328125 -10.1060838699341\\
131.122360229492 -10.1149396896362\\
131.134872436523 -10.1233901977539\\
131.147369384766 -10.1313762664795\\
131.159866333008 -10.1388845443726\\
131.17236328125 -10.1458930969238\\
131.184860229492 -10.1523599624634\\
131.197372436523 -10.158390045166\\
131.209869384766 -10.167950630188\\
131.222366333008 -10.2029151916504\\
131.23486328125 -10.2470664978027\\
131.247360229492 -10.2810049057007\\
131.259872436523 -10.3069839477539\\
131.272369384766 -10.3343620300293\\
131.284866333008 -10.3615341186523\\
131.29736328125 -10.3873023986816\\
131.309860229492 -10.4101581573486\\
131.322372436523 -10.4293594360352\\
131.334869384766 -10.444993019104\\
131.347366333008 -10.4571590423584\\
131.35986328125 -10.4644536972046\\
131.372360229492 -10.4679374694824\\
131.384872436523 -10.4695711135864\\
131.397369384766 -10.4707221984863\\
131.409866333008 -10.4745063781738\\
131.42236328125 -10.4987983703613\\
131.434860229492 -10.5350246429443\\
131.447372436523 -10.5719709396362\\
131.459869384766 -10.604100227356\\
131.472366333008 -10.627703666687\\
131.48486328125 -10.6410064697266\\
131.497360229492 -10.6457576751709\\
131.509872436523 -10.6471757888794\\
131.522369384766 -10.6478433609009\\
131.534866333008 -10.6483068466187\\
131.54736328125 -10.64866065979\\
131.559860229492 -10.6489667892456\\
131.572372436523 -10.6492366790771\\
131.584869384766 -10.6494922637939\\
131.597366333008 -10.6498212814331\\
131.60986328125 -10.6505975723267\\
131.622360229492 -10.6629781723022\\
131.634872436523 -10.697735786438\\
131.647369384766 -10.7372789382935\\
131.659866333008 -10.7732076644897\\
131.67236328125 -10.8029851913452\\
131.684860229492 -10.8252038955688\\
131.697372436523 -10.8397054672241\\
131.709869384766 -10.846851348877\\
131.722366333008 -10.8491716384888\\
131.73486328125 -10.8502635955811\\
131.747360229492 -10.8510980606079\\
131.759872436523 -10.8518352508545\\
131.772369384766 -10.8525018692017\\
131.784866333008 -10.8530960083008\\
131.79736328125 -10.8536062240601\\
131.809860229492 -10.8540239334106\\
131.822372436523 -10.8543319702148\\
131.834869384766 -10.8545951843262\\
131.847366333008 -10.8548650741577\\
131.85986328125 -10.8550958633423\\
131.872360229492 -10.8552989959717\\
131.884872436523 -10.8555297851563\\
131.897369384766 -10.855902671814\\
131.909866333008 -10.8562498092651\\
131.92236328125 -10.8565635681152\\
131.934860229492 -10.8568649291992\\
131.947372436523 -10.8571939468384\\
131.959869384766 -10.8575830459595\\
131.972366333008 -10.8580226898193\\
131.98486328125 -10.8584585189819\\
131.997360229492 -10.8588809967041\\
132.009872436523 -10.859278678894\\
132.022369384766 -10.8596525192261\\
132.034866333008 -10.8600158691406\\
132.04736328125 -10.8603782653809\\
132.059860229492 -10.8607406616211\\
132.072372436523 -10.8611030578613\\
132.084869384766 -10.8614654541016\\
132.097366333008 -10.8618240356445\\
132.10986328125 -10.8621740341187\\
132.122360229492 -10.8625087738037\\
132.134872436523 -10.8628244400024\\
132.147369384766 -10.8631229400635\\
132.159866333008 -10.8634948730469\\
132.17236328125 -10.8639192581177\\
132.184860229492 -10.8643932342529\\
132.197372436523 -10.8649435043335\\
132.209869384766 -10.8656272888184\\
132.222366333008 -10.8664922714233\\
132.23486328125 -10.8674917221069\\
132.247360229492 -10.8685140609741\\
132.259872436523 -10.8694610595703\\
132.272369384766 -10.870288848877\\
132.284866333008 -10.8709897994995\\
132.29736328125 -10.8717308044434\\
132.309860229492 -10.8729448318481\\
132.322372436523 -10.8741693496704\\
132.334869384766 -10.8750534057617\\
132.347366333008 -10.875542640686\\
132.35986328125 -10.8759469985962\\
132.372360229492 -10.876392364502\\
132.384872436523 -10.8769683837891\\
132.397369384766 -10.8778190612793\\
132.409866333008 -10.8793411254883\\
132.42236328125 -10.8826026916504\\
132.434860229492 -10.8905296325684\\
132.447372436523 -10.9107685089111\\
132.459869384766 -10.9447612762451\\
132.472366333008 -10.9613790512085\\
132.48486328125 -10.9663486480713\\
132.497360229492 -10.9683494567871\\
132.509872436523 -10.969762802124\\
132.522369384766 -10.9699697494507\\
132.534866333008 -10.9666509628296\\
132.54736328125 -10.9668216705322\\
132.559860229492 -10.9670610427856\\
132.572372436523 -10.9675321578979\\
132.584869384766 -10.9681367874146\\
132.597366333008 -10.9687786102295\\
132.60986328125 -10.9694137573242\\
132.622360229492 -10.9699764251709\\
132.634872436523 -10.9703578948975\\
132.647369384766 -10.9706325531006\\
132.659866333008 -10.9708366394043\\
132.67236328125 -10.9709806442261\\
132.684860229492 -10.9710569381714\\
132.697372436523 -10.9710779190063\\
132.709869384766 -10.9710597991943\\
132.722366333008 -10.9710035324097\\
132.73486328125 -10.9708766937256\\
132.747360229492 -10.9708032608032\\
132.759872436523 -10.9707880020142\\
132.772369384766 -10.9708185195923\\
132.784866333008 -10.9708757400513\\
132.79736328125 -10.970947265625\\
132.809860229492 -10.9710493087769\\
132.822372436523 -10.9712171554565\\
132.834869384766 -10.9714288711548\\
132.847366333008 -10.9716691970825\\
132.85986328125 -10.9719209671021\\
132.872360229492 -10.9721775054932\\
132.884872436523 -10.9724349975586\\
132.897369384766 -10.9726839065552\\
132.909866333008 -10.9729251861572\\
132.92236328125 -10.9731502532959\\
132.934860229492 -10.9733591079712\\
132.947372436523 -10.9735488891602\\
132.959869384766 -10.9737176895142\\
132.972366333008 -10.9738674163818\\
132.98486328125 -10.973991394043\\
132.997360229492 -10.9740905761719\\
133.009872436523 -10.974178314209\\
133.022369384766 -10.9742555618286\\
133.034866333008 -10.9743242263794\\
133.04736328125 -10.9743909835815\\
133.059860229492 -10.9744577407837\\
133.072372436523 -10.974534034729\\
133.084869384766 -10.9746170043945\\
133.097366333008 -10.9747114181519\\
133.10986328125 -10.9748191833496\\
133.122360229492 -10.9749355316162\\
133.134872436523 -10.975061416626\\
133.147369384766 -10.9751949310303\\
133.159866333008 -10.9753284454346\\
133.17236328125 -10.9754619598389\\
133.184860229492 -10.9755954742432\\
133.197372436523 -10.9757289886475\\
133.209869384766 -10.9758625030518\\
133.222366333008 -10.9759883880615\\
133.23486328125 -10.9761123657227\\
133.247360229492 -10.9762306213379\\
133.259872436523 -10.9763584136963\\
133.272369384766 -10.9765014648438\\
133.284866333008 -10.9766445159912\\
133.29736328125 -10.9767875671387\\
133.309860229492 -10.9769296646118\\
133.322372436523 -10.9770727157593\\
133.334869384766 -10.9772100448608\\
133.347366333008 -10.9773445129395\\
133.35986328125 -10.9774780273438\\
133.372360229492 -10.9776124954224\\
133.384872436523 -10.9777555465698\\
133.397369384766 -10.9778985977173\\
133.409866333008 -10.9780492782593\\
133.42236328125 -10.9782018661499\\
133.434860229492 -10.9783544540405\\
133.447372436523 -10.9785070419312\\
133.459869384766 -10.9786596298218\\
133.472366333008 -10.9788122177124\\
133.48486328125 -10.9789667129517\\
133.497360229492 -10.9791193008423\\
133.509872436523 -10.9792766571045\\
133.522369384766 -10.9794301986694\\
133.534866333008 -10.9795827865601\\
133.54736328125 -10.9797353744507\\
133.559860229492 -10.9798879623413\\
133.572372436523 -10.9800405502319\\
133.584869384766 -10.9801931381226\\
133.597366333008 -10.9803457260132\\
133.60986328125 -10.9804983139038\\
133.622360229492 -10.9806509017944\\
133.634872436523 -10.9808034896851\\
133.647369384766 -10.9809551239014\\
133.659866333008 -10.9810991287231\\
133.67236328125 -10.9812421798706\\
133.684860229492 -10.9813852310181\\
133.697372436523 -10.9815282821655\\
133.709869384766 -10.9816627502441\\
133.722366333008 -10.9817962646484\\
133.73486328125 -10.9819297790527\\
133.747360229492 -10.982063293457\\
133.759872436523 -10.9821968078613\\
133.772369384766 -10.9823236465454\\
133.784866333008 -10.9824476242065\\
133.79736328125 -10.9825716018677\\
133.809860229492 -10.9826955795288\\
133.822372436523 -10.9828195571899\\
133.834869384766 -10.9829435348511\\
133.847366333008 -10.9830675125122\\
133.85986328125 -10.9831914901733\\
133.872360229492 -10.9833154678345\\
133.884872436523 -10.9834394454956\\
133.897369384766 -10.9835605621338\\
133.909866333008 -10.983678817749\\
133.92236328125 -10.9838027954102\\
133.934860229492 -10.9839267730713\\
133.947372436523 -10.9840507507324\\
133.959869384766 -10.9841747283936\\
133.972366333008 -10.9843044281006\\
133.98486328125 -10.9844369888306\\
133.997360229492 -10.9845705032349\\
134.009872436523 -10.9847040176392\\
134.022369384766 -10.9848375320435\\
134.034866333008 -10.9849710464478\\
134.04736328125 -10.9851045608521\\
134.059860229492 -10.9852380752563\\
134.072372436523 -10.9853715896606\\
134.084869384766 -10.9855051040649\\
134.097366333008 -10.9856386184692\\
134.10986328125 -10.9857721328735\\
134.122360229492 -10.9859056472778\\
134.134872436523 -10.9860391616821\\
134.147369384766 -10.9861812591553\\
134.159866333008 -10.9863243103027\\
134.17236328125 -10.9864673614502\\
134.184860229492 -10.9866104125977\\
134.197372436523 -10.9867534637451\\
134.209869384766 -10.9869003295898\\
134.222366333008 -10.9870519638062\\
134.23486328125 -10.9872045516968\\
134.247360229492 -10.9873571395874\\
134.259872436523 -10.987509727478\\
134.272369384766 -10.9876623153687\\
134.284866333008 -10.9878149032593\\
134.29736328125 -10.9879751205444\\
134.309860229492 -10.9881372451782\\
134.322372436523 -10.988299369812\\
134.334869384766 -10.9884614944458\\
134.347366333008 -10.9886236190796\\
134.35986328125 -10.9887857437134\\
134.372360229492 -10.9889507293701\\
134.384872436523 -10.9891223907471\\
134.397369384766 -10.989294052124\\
134.409866333008 -10.989465713501\\
134.42236328125 -10.9896373748779\\
134.434860229492 -10.9898090362549\\
134.447372436523 -10.9899806976318\\
134.459869384766 -10.9901523590088\\
134.472366333008 -10.9903240203857\\
134.48486328125 -10.9904956817627\\
134.497360229492 -10.9906673431396\\
134.509872436523 -10.9908390045166\\
134.522369384766 -10.9910106658936\\
134.534866333008 -10.9911823272705\\
134.54736328125 -10.9913539886475\\
134.559860229492 -10.9915256500244\\
134.572372436523 -10.9916973114014\\
134.584869384766 -10.9918689727783\\
134.597366333008 -10.9920406341553\\
134.60986328125 -10.9922122955322\\
134.622360229492 -10.9923839569092\\
134.634872436523 -10.9925556182861\\
134.647369384766 -10.9927196502686\\
134.659866333008 -10.9928817749023\\
134.67236328125 -10.9930438995361\\
134.684860229492 -10.9932060241699\\
134.697372436523 -10.9933681488037\\
134.709869384766 -10.9935293197632\\
134.722366333008 -10.9936828613281\\
134.73486328125 -10.9938354492188\\
134.747360229492 -10.9939880371094\\
134.759872436523 -10.994140625\\
134.772369384766 -10.9942932128906\\
134.784866333008 -10.9944448471069\\
134.79736328125 -10.9945964813232\\
134.809860229492 -10.9947481155396\\
134.822372436523 -10.9949007034302\\
134.834869384766 -10.9950532913208\\
134.847366333008 -10.9952058792114\\
134.85986328125 -10.9953575134277\\
134.872360229492 -10.9955081939697\\
134.884872436523 -10.9956588745117\\
134.897369384766 -10.9958114624023\\
134.909866333008 -10.995964050293\\
134.92236328125 -10.9961166381836\\
134.934860229492 -10.9962692260742\\
134.947372436523 -10.9964160919189\\
134.959869384766 -10.9965553283691\\
134.972366333008 -10.9966802597046\\
134.98486328125 -10.9968042373657\\
134.997360229492 -10.9969301223755\\
135.009872436523 -10.9970684051514\\
135.022369384766 -10.997218132019\\
135.034866333008 -10.9973754882813\\
135.04736328125 -10.997537612915\\
135.059860229492 -10.9977006912231\\
135.072372436523 -10.9978723526001\\
135.084869384766 -10.9980440139771\\
135.097366333008 -10.9982175827026\\
135.10986328125 -10.9983987808228\\
135.122360229492 -10.9985799789429\\
135.134872436523 -10.998761177063\\
135.147369384766 -10.9989500045776\\
135.159866333008 -10.9991407394409\\
135.17236328125 -10.9993305206299\\
135.184860229492 -10.9995212554932\\
135.197372436523 -10.9997119903564\\
135.209869384766 -10.9999027252197\\
135.222366333008 -11.000093460083\\
135.23486328125 -11.0002841949463\\
135.247360229492 -11.0004749298096\\
135.259872436523 -11.0006656646729\\
135.272369384766 -11.0008563995361\\
135.284866333008 -11.0010471343994\\
135.29736328125 -11.0012378692627\\
135.309860229492 -11.001428604126\\
135.322372436523 -11.0016193389893\\
135.334869384766 -11.0018100738525\\
135.347366333008 -11.0020008087158\\
135.35986328125 -11.0022001266479\\
135.372360229492 -11.0024003982544\\
135.384872436523 -11.0026073455811\\
135.397369384766 -11.0028162002563\\
135.409866333008 -11.003026008606\\
135.42236328125 -11.0032453536987\\
135.434860229492 -11.0034761428833\\
135.447372436523 -11.0037212371826\\
135.459869384766 -11.0039768218994\\
135.472366333008 -11.0042419433594\\
135.48486328125 -11.0045175552368\\
135.497360229492 -11.0048046112061\\
135.509872436523 -11.0051012039185\\
135.522369384766 -11.0054063796997\\
135.534866333008 -11.0057210922241\\
135.54736328125 -11.0060453414917\\
135.559860229492 -11.0063724517822\\
135.572372436523 -11.006706237793\\
135.584869384766 -11.007043838501\\
135.597366333008 -11.0073804855347\\
135.60986328125 -11.0077428817749\\
135.622360229492 -11.0081233978271\\
135.634872436523 -11.0085048675537\\
135.647369384766 -11.0088787078857\\
135.659866333008 -11.0092439651489\\
135.67236328125 -11.0096101760864\\
135.684860229492 -11.0099725723267\\
135.697372436523 -11.0103349685669\\
135.709869384766 -11.0106973648071\\
135.722366333008 -11.0110597610474\\
135.73486328125 -11.0114221572876\\
135.747360229492 -11.0117874145508\\
135.759872436523 -11.0121593475342\\
135.772369384766 -11.0125312805176\\
135.784866333008 -11.0129070281982\\
135.79736328125 -11.0132894515991\\
135.809860229492 -11.0136623382568\\
135.822372436523 -11.0139751434326\\
135.834869384766 -11.0142841339111\\
135.847366333008 -11.0146226882935\\
135.85986328125 -11.014986038208\\
135.872360229492 -11.0153636932373\\
135.884872436523 -11.0157833099365\\
135.897369384766 -11.0162315368652\\
135.909866333008 -11.016695022583\\
135.92236328125 -11.0171585083008\\
135.934860229492 -11.0176124572754\\
135.947372436523 -11.0180568695068\\
135.959869384766 -11.0184946060181\\
135.972366333008 -11.0189237594604\\
135.98486328125 -11.0193519592285\\
135.997360229492 -11.0197744369507\\
136.009872436523 -11.0201940536499\\
136.022369384766 -11.0206136703491\\
136.034866333008 -11.0210390090942\\
136.04736328125 -11.0214672088623\\
136.059860229492 -11.0218954086304\\
136.072372436523 -11.0223245620728\\
136.084869384766 -11.0227479934692\\
136.097366333008 -11.0232076644897\\
136.10986328125 -11.0236778259277\\
136.122360229492 -11.0240850448608\\
136.134872436523 -11.0244693756104\\
136.147369384766 -11.0248498916626\\
136.159866333008 -11.0252132415771\\
136.17236328125 -11.0255718231201\\
136.184860229492 -11.0259418487549\\
136.197372436523 -11.0263223648071\\
136.209869384766 -11.0267124176025\\
136.222366333008 -11.0271034240723\\
136.23486328125 -11.0274934768677\\
136.247360229492 -11.0278797149658\\
136.259872436523 -11.0282611846924\\
136.272369384766 -11.0286331176758\\
136.284866333008 -11.0289945602417\\
136.29736328125 -11.0293455123901\\
136.309860229492 -11.0296878814697\\
136.322372436523 -11.0300216674805\\
136.334869384766 -11.030348777771\\
136.347366333008 -11.0306701660156\\
136.35986328125 -11.0309839248657\\
136.372360229492 -11.0312881469727\\
136.384872436523 -11.0315847396851\\
136.397369384766 -11.0318803787231\\
136.409866333008 -11.0321760177612\\
136.42236328125 -11.0324716567993\\
136.434860229492 -11.0327663421631\\
136.447372436523 -11.0330610275269\\
136.459869384766 -11.0333528518677\\
136.472366333008 -11.0336389541626\\
136.48486328125 -11.0339164733887\\
136.497360229492 -11.0341901779175\\
136.509872436523 -11.0344572067261\\
136.522369384766 -11.0347204208374\\
136.534866333008 -11.0349779129028\\
136.54736328125 -11.0352344512939\\
136.559860229492 -11.0354824066162\\
136.572372436523 -11.0357303619385\\
136.584869384766 -11.0359706878662\\
136.597366333008 -11.0362091064453\\
136.60986328125 -11.0364408493042\\
136.622360229492 -11.0366687774658\\
136.634872436523 -11.0368909835815\\
136.647369384766 -11.0371103286743\\
136.659866333008 -11.0373220443726\\
136.67236328125 -11.0375318527222\\
136.684860229492 -11.0377368927002\\
136.697372436523 -11.0379371643066\\
136.709869384766 -11.0381374359131\\
136.722366333008 -11.0383443832397\\
136.73486328125 -11.0385551452637\\
136.747360229492 -11.0387754440308\\
136.759872436523 -11.0390090942383\\
136.772369384766 -11.0392551422119\\
136.784866333008 -11.0395069122314\\
136.79736328125 -11.0397634506226\\
136.809860229492 -11.0400142669678\\
136.822372436523 -11.0402612686157\\
136.834869384766 -11.0404987335205\\
136.847366333008 -11.0407257080078\\
136.85986328125 -11.0409412384033\\
136.872360229492 -11.0411462783813\\
136.884872436523 -11.0413389205933\\
136.897369384766 -11.0415210723877\\
136.909866333008 -11.0417022705078\\
136.92236328125 -11.0418834686279\\
136.934860229492 -11.042064666748\\
136.947372436523 -11.0422458648682\\
136.959869384766 -11.0424270629883\\
136.972366333008 -11.042610168457\\
136.98486328125 -11.0428009033203\\
136.997360229492 -11.0429916381836\\
137.009872436523 -11.0431833267212\\
137.022369384766 -11.0433826446533\\
137.034866333008 -11.0435819625854\\
137.04736328125 -11.0437812805176\\
137.059860229492 -11.043981552124\\
137.072372436523 -11.0441818237305\\
137.084869384766 -11.0443820953369\\
137.097366333008 -11.0445833206177\\
137.10986328125 -11.044792175293\\
137.122360229492 -11.0450019836426\\
137.134872436523 -11.0452117919922\\
137.147369384766 -11.0454301834106\\
137.159866333008 -11.0456495285034\\
137.17236328125 -11.0458726882935\\
137.184860229492 -11.0461015701294\\
137.197372436523 -11.0463352203369\\
137.209869384766 -11.0465726852417\\
137.222366333008 -11.0468111038208\\
137.23486328125 -11.0470495223999\\
137.247360229492 -11.0472869873047\\
137.259872436523 -11.0475254058838\\
137.272369384766 -11.0477714538574\\
137.284866333008 -11.0480194091797\\
137.29736328125 -11.048267364502\\
137.309860229492 -11.0485172271729\\
137.322372436523 -11.0487747192383\\
137.334869384766 -11.0490312576294\\
137.347366333008 -11.0492877960205\\
137.35986328125 -11.0495452880859\\
137.372360229492 -11.0498027801514\\
137.384872436523 -11.0500640869141\\
137.397369384766 -11.0503301620483\\
137.409866333008 -11.0506010055542\\
137.42236328125 -11.050877571106\\
137.434860229492 -11.0511541366577\\
137.447372436523 -11.0514326095581\\
137.459869384766 -11.0517339706421\\
137.472366333008 -11.052077293396\\
137.48486328125 -11.0524673461914\\
137.497360229492 -11.0529050827026\\
137.509872436523 -11.0533838272095\\
137.522369384766 -11.0538921356201\\
137.534866333008 -11.05442237854\\
137.54736328125 -11.0549716949463\\
137.559860229492 -11.0555410385132\\
137.572372436523 -11.056131362915\\
137.584869384766 -11.0567398071289\\
137.597366333008 -11.0573663711548\\
137.60986328125 -11.0580101013184\\
137.622360229492 -11.0586719512939\\
137.634872436523 -11.0593509674072\\
137.647369384766 -11.0600442886353\\
137.659866333008 -11.0607538223267\\
137.67236328125 -11.0614776611328\\
137.684860229492 -11.062216758728\\
137.697372436523 -11.0629739761353\\
137.709869384766 -11.0637435913086\\
137.722366333008 -11.064528465271\\
137.73486328125 -11.0653266906738\\
137.747360229492 -11.0661392211914\\
137.759872436523 -11.0669651031494\\
137.772369384766 -11.0678024291992\\
137.784866333008 -11.0686521530151\\
137.79736328125 -11.0695123672485\\
137.809860229492 -11.0703840255737\\
137.822372436523 -11.0712661743164\\
137.834869384766 -11.0721569061279\\
137.847366333008 -11.0730562210083\\
137.85986328125 -11.0739631652832\\
137.872360229492 -11.0748777389526\\
137.884872436523 -11.0758008956909\\
137.897369384766 -11.0767259597778\\
137.909866333008 -11.0776596069336\\
137.92236328125 -11.078595161438\\
137.934860229492 -11.0795383453369\\
137.947372436523 -11.0804824829102\\
137.959869384766 -11.081428527832\\
137.972366333008 -11.0823802947998\\
137.98486328125 -11.0833330154419\\
137.997360229492 -11.084285736084\\
138.009872436523 -11.0852394104004\\
138.022369384766 -11.0861921310425\\
138.034866333008 -11.0871448516846\\
138.04736328125 -11.0880975723267\\
138.059860229492 -11.0890502929688\\
138.072372436523 -11.0900039672852\\
138.084869384766 -11.0909566879272\\
138.097366333008 -11.0919094085693\\
138.10986328125 -11.0928564071655\\
138.122360229492 -11.0937995910645\\
138.134872436523 -11.0947437286377\\
138.147369384766 -11.0956869125366\\
138.159866333008 -11.0966243743896\\
138.17236328125 -11.0975580215454\\
138.184860229492 -11.0984916687012\\
138.197372436523 -11.0994215011597\\
138.209869384766 -11.1003456115723\\
138.222366333008 -11.1012687683105\\
138.23486328125 -11.10218334198\\
138.247360229492 -11.1030979156494\\
138.259872436523 -11.1040105819702\\
138.272369384766 -11.1049156188965\\
138.284866333008 -11.1058197021484\\
138.29736328125 -11.1067152023315\\
138.309860229492 -11.1076078414917\\
138.322372436523 -11.108494758606\\
138.334869384766 -11.1093778610229\\
138.347366333008 -11.1102542877197\\
138.35986328125 -11.1111297607422\\
138.372360229492 -11.1119985580444\\
138.384872436523 -11.1128664016724\\
138.397369384766 -11.1137247085571\\
138.409866333008 -11.1145820617676\\
138.42236328125 -11.1154317855835\\
138.434860229492 -11.1162796020508\\
138.447372436523 -11.1171207427979\\
138.459869384766 -11.117959022522\\
138.472366333008 -11.1187906265259\\
138.48486328125 -11.1196193695068\\
138.497360229492 -11.1204433441162\\
138.509872436523 -11.1212635040283\\
138.522369384766 -11.1220779418945\\
138.534866333008 -11.1228876113892\\
138.54736328125 -11.1236963272095\\
138.559860229492 -11.1244964599609\\
138.572372436523 -11.1252946853638\\
138.584869384766 -11.1260852813721\\
138.597366333008 -11.1268758773804\\
138.60986328125 -11.1276578903198\\
138.622360229492 -11.128438949585\\
138.634872436523 -11.1292123794556\\
138.647369384766 -11.1299839019775\\
138.659866333008 -11.1307487487793\\
138.67236328125 -11.1315107345581\\
138.684860229492 -11.1322641372681\\
138.697372436523 -11.1330156326294\\
138.709869384766 -11.1337585449219\\
138.722366333008 -11.1344995498657\\
138.73486328125 -11.135232925415\\
138.747360229492 -11.1359615325928\\
138.759872436523 -11.1366863250732\\
138.772369384766 -11.1374025344849\\
138.784866333008 -11.1381149291992\\
138.79736328125 -11.138819694519\\
138.809860229492 -11.1395196914673\\
138.822372436523 -11.1402158737183\\
138.834869384766 -11.1409025192261\\
138.847366333008 -11.141583442688\\
138.85986328125 -11.1422595977783\\
138.872360229492 -11.1429262161255\\
138.884872436523 -11.1435890197754\\
138.897369384766 -11.1442461013794\\
138.909866333008 -11.1448936462402\\
138.92236328125 -11.1455354690552\\
138.934860229492 -11.1461725234985\\
138.947372436523 -11.1468019485474\\
138.959869384766 -11.147424697876\\
138.972366333008 -11.1480388641357\\
138.98486328125 -11.1486473083496\\
138.997360229492 -11.1492471694946\\
139.009872436523 -11.1498413085938\\
139.022369384766 -11.1504287719727\\
139.034866333008 -11.1510095596313\\
139.04736328125 -11.1515846252441\\
139.059860229492 -11.152153968811\\
139.072372436523 -11.1527166366577\\
139.084869384766 -11.1532716751099\\
139.097366333008 -11.1538219451904\\
139.10986328125 -11.1543645858765\\
139.122360229492 -11.1549024581909\\
139.134872436523 -11.1554365158081\\
139.147369384766 -11.1559610366821\\
139.159866333008 -11.1564846038818\\
139.17236328125 -11.1570024490356\\
139.184860229492 -11.1574745178223\\
139.197372436523 -11.1579723358154\\
139.209869384766 -11.1584939956665\\
139.222366333008 -11.1590270996094\\
139.23486328125 -11.1595602035522\\
139.247360229492 -11.160083770752\\
139.259872436523 -11.1605997085571\\
139.272369384766 -11.1611080169678\\
139.284866333008 -11.1616125106812\\
139.29736328125 -11.1621112823486\\
139.309860229492 -11.1626062393188\\
139.322372436523 -11.1631021499634\\
139.334869384766 -11.1635971069336\\
139.347366333008 -11.1640920639038\\
139.35986328125 -11.1645793914795\\
139.372360229492 -11.1650648117065\\
139.384872436523 -11.1655435562134\\
139.397369384766 -11.1660175323486\\
139.409866333008 -11.1664848327637\\
139.42236328125 -11.1669521331787\\
139.434860229492 -11.1674184799194\\
139.447372436523 -11.1678857803345\\
139.459869384766 -11.1683502197266\\
139.472366333008 -11.1688079833984\\
139.48486328125 -11.1692628860474\\
139.497360229492 -11.1697101593018\\
139.509872436523 -11.1701536178589\\
139.522369384766 -11.1705923080444\\
139.534866333008 -11.1710243225098\\
139.54736328125 -11.1714525222778\\
139.559860229492 -11.171875\\
139.572372436523 -11.1722946166992\\
139.584869384766 -11.1727066040039\\
139.597366333008 -11.1731157302856\\
139.60986328125 -11.1735191345215\\
139.622360229492 -11.1739187240601\\
139.634872436523 -11.1743106842041\\
139.647369384766 -11.1746959686279\\
139.659866333008 -11.1750736236572\\
139.67236328125 -11.175440788269\\
139.684860229492 -11.1757822036743\\
139.697372436523 -11.1761035919189\\
139.709869384766 -11.176365852356\\
139.722366333008 -11.1765747070313\\
139.73486328125 -11.1767473220825\\
139.747360229492 -11.1768970489502\\
139.759872436523 -11.1770401000977\\
139.772369384766 -11.1771860122681\\
139.784866333008 -11.1773433685303\\
139.79736328125 -11.1775102615356\\
139.809860229492 -11.1776819229126\\
139.822372436523 -11.1778535842896\\
139.834869384766 -11.1780319213867\\
139.847366333008 -11.1782131195068\\
139.85986328125 -11.1783981323242\\
139.872360229492 -11.1785888671875\\
139.884872436523 -11.1787796020508\\
139.897369384766 -11.178973197937\\
139.909866333008 -11.1791734695435\\
139.92236328125 -11.1793775558472\\
139.934860229492 -11.1795854568481\\
139.947372436523 -11.1797857284546\\
139.959869384766 -11.179986000061\\
139.972366333008 -11.1801853179932\\
139.98486328125 -11.1803770065308\\
139.997360229492 -11.1805772781372\\
140.009872436523 -11.1807775497437\\
140.022369384766 -11.1809864044189\\
140.034866333008 -11.1811962127686\\
140.04736328125 -11.1814117431641\\
140.059860229492 -11.1816310882568\\
140.072372436523 -11.1818504333496\\
140.084869384766 -11.1820774078369\\
140.097366333008 -11.1823053359985\\
140.10986328125 -11.1825284957886\\
140.122360229492 -11.1827478408813\\
140.134872436523 -11.1829671859741\\
140.147369384766 -11.1831865310669\\
140.159866333008 -11.1834058761597\\
140.17236328125 -11.1836252212524\\
140.184860229492 -11.1838436126709\\
140.197372436523 -11.1840629577637\\
140.209869384766 -11.1842823028564\\
140.222366333008 -11.1845016479492\\
140.23486328125 -11.1847286224365\\
140.247360229492 -11.1849575042725\\
140.259872436523 -11.1851863861084\\
140.272369384766 -11.1854238510132\\
140.284866333008 -11.1856622695923\\
140.29736328125 -11.1859006881714\\
140.309860229492 -11.1861295700073\\
140.322372436523 -11.1863298416138\\
140.334869384766 -11.1864748001099\\
140.347366333008 -11.1865901947021\\
140.35986328125 -11.186710357666\\
140.372360229492 -11.1868419647217\\
140.384872436523 -11.186975479126\\
140.397369384766 -11.1871109008789\\
140.409866333008 -11.1872529983521\\
140.42236328125 -11.1873960494995\\
140.434860229492 -11.1875410079956\\
140.447372436523 -11.1876964569092\\
140.459869384766 -11.1878709793091\\
140.472366333008 -11.1880683898926\\
140.48486328125 -11.1882791519165\\
140.497360229492 -11.1884984970093\\
140.509872436523 -11.1887235641479\\
140.522369384766 -11.1889524459839\\
140.534866333008 -11.1891822814941\\
140.54736328125 -11.1894197463989\\
140.559860229492 -11.189658164978\\
140.572372436523 -11.1898965835571\\
140.584869384766 -11.1901235580444\\
140.597366333008 -11.1903295516968\\
140.60986328125 -11.1904983520508\\
140.622360229492 -11.1906452178955\\
140.634872436523 -11.190788269043\\
140.647369384766 -11.1909313201904\\
140.659866333008 -11.1910820007324\\
140.67236328125 -11.191234588623\\
140.684860229492 -11.1913938522339\\
140.697372436523 -11.1915597915649\\
140.709869384766 -11.1917314529419\\
140.722366333008 -11.1919097900391\\
140.73486328125 -11.1920928955078\\
140.747360229492 -11.1922836303711\\
140.759872436523 -11.1924781799316\\
140.772369384766 -11.1926774978638\\
140.784866333008 -11.1928834915161\\
140.79736328125 -11.1930961608887\\
140.809860229492 -11.1933155059814\\
140.822372436523 -11.1935434341431\\
140.834869384766 -11.1937770843506\\
140.847366333008 -11.1940155029297\\
140.85986328125 -11.1942539215088\\
140.872360229492 -11.1944913864136\\
140.884872436523 -11.1947259902954\\
140.897369384766 -11.1949615478516\\
140.909866333008 -11.1951990127563\\
140.92236328125 -11.1954374313354\\
140.934860229492 -11.1956939697266\\
140.947372436523 -11.1959857940674\\
140.959869384766 -11.1963319778442\\
140.972366333008 -11.196738243103\\
140.98486328125 -11.1972103118896\\
140.997360229492 -11.1977481842041\\
141.009872436523 -11.1983480453491\\
141.022369384766 -11.1989984512329\\
141.034866333008 -11.1996736526489\\
141.04736328125 -11.2003107070923\\
141.059860229492 -11.2008295059204\\
141.072372436523 -11.2012252807617\\
141.084869384766 -11.2015504837036\\
141.097366333008 -11.2018346786499\\
141.10986328125 -11.2020807266235\\
141.122360229492 -11.2023086547852\\
141.134872436523 -11.2025375366211\\
141.147369384766 -11.2027730941772\\
141.159866333008 -11.2030200958252\\
141.17236328125 -11.2032861709595\\
141.184860229492 -11.203574180603\\
141.197372436523 -11.2038888931274\\
141.209869384766 -11.20423412323\\
141.222366333008 -11.2046098709106\\
141.23486328125 -11.2050180435181\\
141.247360229492 -11.2054586410522\\
141.259872436523 -11.2059354782104\\
141.272369384766 -11.206446647644\\
141.284866333008 -11.2069911956787\\
141.29736328125 -11.2075662612915\\
141.309860229492 -11.2081680297852\\
141.322372436523 -11.2087965011597\\
141.334869384766 -11.2094345092773\\
141.347366333008 -11.2100648880005\\
141.35986328125 -11.2106161117554\\
141.372360229492 -11.211051940918\\
141.384872436523 -11.2114496231079\\
141.397369384766 -11.2118577957153\\
141.409866333008 -11.2122898101807\\
141.42236328125 -11.2127485275269\\
141.434860229492 -11.2132339477539\\
141.447372436523 -11.2137441635132\\
141.459869384766 -11.2142763137817\\
141.472366333008 -11.2148294448853\\
141.48486328125 -11.2154026031494\\
141.497360229492 -11.2159986495972\\
141.509872436523 -11.2166185379028\\
141.522369384766 -11.2172613143921\\
141.534866333008 -11.2179298400879\\
141.54736328125 -11.2186231613159\\
141.559860229492 -11.2193412780762\\
141.572372436523 -11.2200965881348\\
141.584869384766 -11.2209062576294\\
141.597366333008 -11.221752166748\\
141.60986328125 -11.222617149353\\
141.622360229492 -11.2234907150269\\
141.634872436523 -11.2243356704712\\
141.647369384766 -11.2250270843506\\
141.659866333008 -11.2255029678345\\
141.67236328125 -11.2259006500244\\
141.684860229492 -11.2262945175171\\
141.697372436523 -11.226710319519\\
141.709869384766 -11.2271556854248\\
141.722366333008 -11.227635383606\\
141.73486328125 -11.2281494140625\\
141.747360229492 -11.2286968231201\\
141.759872436523 -11.2292747497559\\
141.772369384766 -11.2298793792725\\
141.784866333008 -11.2305088043213\\
141.79736328125 -11.2311601638794\\
141.809860229492 -11.2318305969238\\
141.822372436523 -11.2325067520142\\
141.834869384766 -11.2331476211548\\
141.847366333008 -11.2336463928223\\
141.85986328125 -11.2339019775391\\
141.872360229492 -11.2340488433838\\
141.884872436523 -11.2342042922974\\
141.897369384766 -11.2343769073486\\
141.909866333008 -11.234564781189\\
141.92236328125 -11.2347688674927\\
141.934860229492 -11.2349872589111\\
141.947372436523 -11.2352180480957\\
141.959869384766 -11.2354593276978\\
141.972366333008 -11.2357091903687\\
141.98486328125 -11.2359685897827\\
141.997360229492 -11.2362403869629\\
142.009872436523 -11.2365274429321\\
142.022369384766 -11.2368268966675\\
142.034866333008 -11.2371368408203\\
142.04736328125 -11.2374515533447\\
142.059860229492 -11.2377481460571\\
142.072372436523 -11.2379932403564\\
142.084869384766 -11.2382078170776\\
142.097366333008 -11.2384176254272\\
142.10986328125 -11.2386274337769\\
142.122360229492 -11.2388391494751\\
142.134872436523 -11.2390584945679\\
142.147369384766 -11.2392873764038\\
142.159866333008 -11.2395286560059\\
142.17236328125 -11.2397804260254\\
142.184860229492 -11.2400407791138\\
142.197372436523 -11.2403059005737\\
142.209869384766 -11.240550994873\\
142.222366333008 -11.2407474517822\\
142.23486328125 -11.2409076690674\\
142.247360229492 -11.2410545349121\\
142.259872436523 -11.2411975860596\\
142.272369384766 -11.2413311004639\\
142.284866333008 -11.2414646148682\\
142.29736328125 -11.2415981292725\\
142.309860229492 -11.2417316436768\\
142.322372436523 -11.2418651580811\\
142.334869384766 -11.2419986724854\\
142.347366333008 -11.2421321868896\\
142.35986328125 -11.2422657012939\\
142.372360229492 -11.2423992156982\\
142.384872436523 -11.2425327301025\\
142.397369384766 -11.2426652908325\\
142.409866333008 -11.2427959442139\\
142.42236328125 -11.2429437637329\\
142.434860229492 -11.2431087493896\\
142.447372436523 -11.2432928085327\\
142.459869384766 -11.2434978485107\\
142.472366333008 -11.2437314987183\\
142.48486328125 -11.2439975738525\\
142.497360229492 -11.2443017959595\\
142.509872436523 -11.2446451187134\\
142.522369384766 -11.2450246810913\\
142.534866333008 -11.2454385757446\\
142.54736328125 -11.2458820343018\\
142.559860229492 -11.2463464736938\\
142.572372436523 -11.2468128204346\\
142.584869384766 -11.2472543716431\\
142.597366333008 -11.2476644515991\\
142.60986328125 -11.248067855835\\
142.622360229492 -11.2484903335571\\
142.634872436523 -11.2489395141602\\
142.647369384766 -11.2494173049927\\
142.659866333008 -11.2499227523804\\
142.67236328125 -11.2504529953003\\
142.684860229492 -11.2510013580322\\
142.697372436523 -11.2515525817871\\
142.709869384766 -11.2520847320557\\
142.722366333008 -11.2525930404663\\
142.73486328125 -11.2531042098999\\
142.747360229492 -11.2536315917969\\
142.759872436523 -11.2541780471802\\
142.772369384766 -11.2547426223755\\
142.784866333008 -11.2553262710571\\
142.79736328125 -11.2559251785278\\
142.809860229492 -11.256534576416\\
142.822372436523 -11.2571430206299\\
142.834869384766 -11.2577352523804\\
142.847366333008 -11.2583103179932\\
142.85986328125 -11.258885383606\\
142.872360229492 -11.2594718933105\\
142.884872436523 -11.2600708007813\\
142.897369384766 -11.2606792449951\\
142.909866333008 -11.2612886428833\\
142.92236328125 -11.2618913650513\\
142.934860229492 -11.2624912261963\\
142.947372436523 -11.2630949020386\\
142.959869384766 -11.2637071609497\\
142.972366333008 -11.2643260955811\\
142.98486328125 -11.2649545669556\\
142.997360229492 -11.2655754089355\\
143.009872436523 -11.2661952972412\\
143.022369384766 -11.2668142318726\\
143.034866333008 -11.2674398422241\\
143.04736328125 -11.2680683135986\\
143.059860229492 -11.2686967849731\\
143.072372436523 -11.269326210022\\
143.084869384766 -11.2699546813965\\
143.097366333008 -11.270583152771\\
143.10986328125 -11.2712116241455\\
143.122360229492 -11.2718410491943\\
143.134872436523 -11.2724704742432\\
143.147369384766 -11.2731018066406\\
143.159866333008 -11.2737369537354\\
143.17236328125 -11.2743663787842\\
143.184860229492 -11.274995803833\\
143.197372436523 -11.2756252288818\\
143.209869384766 -11.2762537002563\\
143.222366333008 -11.2768821716309\\
143.23486328125 -11.2775106430054\\
143.247360229492 -11.2781391143799\\
143.259872436523 -11.2787685394287\\
143.272369384766 -11.2793941497803\\
143.284866333008 -11.2800140380859\\
143.29736328125 -11.2806329727173\\
143.309860229492 -11.2812519073486\\
143.322372436523 -11.2818717956543\\
143.334869384766 -11.2824831008911\\
143.347366333008 -11.2830924987793\\
143.35986328125 -11.2837018966675\\
143.372360229492 -11.2843103408813\\
143.384872436523 -11.2849111557007\\
143.397369384766 -11.28551197052\\
143.409866333008 -11.286111831665\\
143.42236328125 -11.2867116928101\\
143.434860229492 -11.2873039245605\\
143.447372436523 -11.2878952026367\\
143.459869384766 -11.2884855270386\\
143.472366333008 -11.2890758514404\\
143.48486328125 -11.2896585464478\\
143.497360229492 -11.2902393341064\\
143.509872436523 -11.2908210754395\\
143.522369384766 -11.2914009094238\\
143.534866333008 -11.2919731140137\\
143.54736328125 -11.2925443649292\\
143.559860229492 -11.2931156158447\\
143.572372436523 -11.2936878204346\\
143.584869384766 -11.2942543029785\\
143.597366333008 -11.2948169708252\\
143.60986328125 -11.2953786849976\\
143.622360229492 -11.2959403991699\\
143.634872436523 -11.2964973449707\\
143.647369384766 -11.2970504760742\\
143.659866333008 -11.2976026535034\\
143.67236328125 -11.2981548309326\\
143.684860229492 -11.2986993789673\\
143.697372436523 -11.2992429733276\\
143.709869384766 -11.2997856140137\\
143.722366333008 -11.3003253936768\\
143.73486328125 -11.3008584976196\\
143.747360229492 -11.3013916015625\\
143.759872436523 -11.3019227981567\\
143.772369384766 -11.3024473190308\\
143.784866333008 -11.3029708862305\\
143.79736328125 -11.3034896850586\\
143.809860229492 -11.3040037155151\\
143.822372436523 -11.304518699646\\
143.834869384766 -11.3050241470337\\
143.847366333008 -11.3055286407471\\
143.85986328125 -11.3060293197632\\
143.872360229492 -11.3065242767334\\
143.884872436523 -11.3070192337036\\
143.897369384766 -11.307505607605\\
143.909866333008 -11.307991027832\\
143.92236328125 -11.3084735870361\\
143.934860229492 -11.30894947052\\
143.947372436523 -11.3094263076782\\
143.959869384766 -11.3098955154419\\
143.972366333008 -11.3103618621826\\
143.98486328125 -11.3108253479004\\
143.997360229492 -11.3112821578979\\
144.009872436523 -11.3117361068726\\
144.022369384766 -11.3121843338013\\
144.034866333008 -11.3126316070557\\
144.04736328125 -11.3130702972412\\
144.059860229492 -11.3135080337524\\
144.072372436523 -11.3139429092407\\
144.084869384766 -11.3143720626831\\
144.097366333008 -11.3148002624512\\
144.10986328125 -11.3152256011963\\
144.122360229492 -11.3156452178955\\
144.134872436523 -11.3160648345947\\
144.147369384766 -11.3164834976196\\
144.159866333008 -11.3169021606445\\
144.17236328125 -11.3173122406006\\
144.184860229492 -11.3177213668823\\
144.197372436523 -11.3181314468384\\
144.209869384766 -11.3185377120972\\
144.222366333008 -11.3189382553101\\
144.23486328125 -11.3193378448486\\
144.247360229492 -11.3197374343872\\
144.259872436523 -11.3201341629028\\
144.272369384766 -11.3205251693726\\
144.284866333008 -11.3209161758423\\
144.29736328125 -11.3213062286377\\
144.309860229492 -11.3216896057129\\
144.322372436523 -11.3220710754395\\
144.334869384766 -11.3224515914917\\
144.347366333008 -11.322829246521\\
144.35986328125 -11.3232011795044\\
144.372360229492 -11.3235721588135\\
144.384872436523 -11.3239431381226\\
144.397369384766 -11.3243055343628\\
144.409866333008 -11.324667930603\\
144.42236328125 -11.3250293731689\\
144.434860229492 -11.325382232666\\
144.447372436523 -11.3257350921631\\
144.459869384766 -11.3260869979858\\
144.472366333008 -11.3264303207397\\
144.48486328125 -11.3267736434937\\
144.497360229492 -11.3271160125732\\
144.509872436523 -11.3274555206299\\
144.522369384766 -11.3277893066406\\
144.534866333008 -11.3281230926514\\
144.54736328125 -11.3284559249878\\
144.559860229492 -11.3287839889526\\
144.572372436523 -11.3291082382202\\
144.584869384766 -11.3294324874878\\
144.597366333008 -11.3297557830811\\
144.60986328125 -11.3300762176514\\
144.622360229492 -11.3303909301758\\
144.634872436523 -11.3307056427002\\
144.647369384766 -11.3310203552246\\
144.659866333008 -11.3313341140747\\
144.67236328125 -11.3316469192505\\
144.684860229492 -11.3319520950317\\
144.697372436523 -11.332257270813\\
144.709869384766 -11.3325624465942\\
144.722366333008 -11.3328676223755\\
144.73486328125 -11.3331718444824\\
144.747360229492 -11.3334760665894\\
144.759872436523 -11.3337812423706\\
144.772369384766 -11.3340826034546\\
144.784866333008 -11.3343782424927\\
144.79736328125 -11.3346738815308\\
144.809860229492 -11.3349695205688\\
144.822372436523 -11.3352651596069\\
144.834869384766 -11.335560798645\\
144.847366333008 -11.3358554840088\\
144.85986328125 -11.3361501693726\\
144.872360229492 -11.3364381790161\\
144.884872436523 -11.336724281311\\
144.897369384766 -11.337010383606\\
144.909866333008 -11.3372964859009\\
144.92236328125 -11.3375816345215\\
144.934860229492 -11.3378658294678\\
144.947372436523 -11.3381423950195\\
144.959869384766 -11.3384189605713\\
144.972366333008 -11.338695526123\\
144.98486328125 -11.3389711380005\\
144.997360229492 -11.339241027832\\
145.009872436523 -11.3395080566406\\
145.022369384766 -11.3397750854492\\
145.034866333008 -11.3400411605835\\
145.04736328125 -11.3403043746948\\
145.059860229492 -11.3405618667603\\
145.072372436523 -11.3408193588257\\
145.084869384766 -11.3410768508911\\
145.097366333008 -11.3413314819336\\
145.10986328125 -11.3415794372559\\
145.122360229492 -11.3418273925781\\
145.134872436523 -11.3420753479004\\
145.147369384766 -11.3423223495483\\
145.159866333008 -11.3425636291504\\
145.17236328125 -11.3428020477295\\
145.184860229492 -11.3430404663086\\
145.197372436523 -11.3432788848877\\
145.209869384766 -11.3435163497925\\
145.222366333008 -11.3437471389771\\
145.23486328125 -11.343976020813\\
145.247360229492 -11.3442049026489\\
145.259872436523 -11.3444337844849\\
145.272369384766 -11.3446626663208\\
145.284866333008 -11.3448905944824\\
145.29736328125 -11.3451099395752\\
145.309860229492 -11.345329284668\\
145.322372436523 -11.3455486297607\\
145.334869384766 -11.3457679748535\\
145.347366333008 -11.3459873199463\\
145.35986328125 -11.3462057113647\\
145.372360229492 -11.3464221954346\\
145.384872436523 -11.3466320037842\\
145.397369384766 -11.3468418121338\\
145.409866333008 -11.3470516204834\\
145.42236328125 -11.347261428833\\
145.434860229492 -11.3474702835083\\
145.447372436523 -11.3476781845093\\
145.459869384766 -11.3478784561157\\
145.472366333008 -11.3480787277222\\
145.48486328125 -11.3482789993286\\
145.497360229492 -11.3484792709351\\
145.509872436523 -11.3486795425415\\
145.522369384766 -11.3488712310791\\
145.534866333008 -11.3490619659424\\
145.54736328125 -11.3492527008057\\
145.559860229492 -11.3494434356689\\
145.572372436523 -11.3496341705322\\
145.584869384766 -11.3498191833496\\
145.597366333008 -11.3500003814697\\
145.60986328125 -11.3501815795898\\
145.622360229492 -11.35036277771\\
145.634872436523 -11.3505439758301\\
145.647369384766 -11.3507242202759\\
145.659866333008 -11.3508977890015\\
145.67236328125 -11.3510694503784\\
145.684860229492 -11.3512411117554\\
145.697372436523 -11.3514127731323\\
145.709869384766 -11.3515844345093\\
145.722366333008 -11.3517475128174\\
145.73486328125 -11.3519096374512\\
145.747360229492 -11.352071762085\\
145.759872436523 -11.3522338867188\\
145.772369384766 -11.3523960113525\\
145.784866333008 -11.352557182312\\
145.79736328125 -11.3527164459229\\
145.809860229492 -11.3528690338135\\
145.822372436523 -11.3530216217041\\
145.834869384766 -11.3531742095947\\
145.847366333008 -11.3533267974854\\
145.85986328125 -11.353479385376\\
145.872360229492 -11.3536310195923\\
145.884872436523 -11.35378074646\\
145.897369384766 -11.3539237976074\\
145.909866333008 -11.3540668487549\\
145.92236328125 -11.3542098999023\\
145.934860229492 -11.3543529510498\\
145.947372436523 -11.3544960021973\\
145.959869384766 -11.3546380996704\\
145.972366333008 -11.3547735214233\\
145.98486328125 -11.3549070358276\\
145.997360229492 -11.3550405502319\\
146.009872436523 -11.3551740646362\\
146.022369384766 -11.3553075790405\\
146.034866333008 -11.3554410934448\\
146.04736328125 -11.3555698394775\\
146.059860229492 -11.3556938171387\\
146.072372436523 -11.3558177947998\\
146.084869384766 -11.3559417724609\\
146.097366333008 -11.3560657501221\\
146.10986328125 -11.3561897277832\\
146.122360229492 -11.3563137054443\\
146.134872436523 -11.3564376831055\\
146.147369384766 -11.3565616607666\\
146.159866333008 -11.3566846847534\\
146.17236328125 -11.3568048477173\\
146.184860229492 -11.3569192886353\\
146.197372436523 -11.3570337295532\\
146.209869384766 -11.3571481704712\\
146.222366333008 -11.3572626113892\\
146.23486328125 -11.3573770523071\\
146.247360229492 -11.3574914932251\\
146.259872436523 -11.3576059341431\\
146.272369384766 -11.357720375061\\
146.284866333008 -11.357834815979\\
146.29736328125 -11.357949256897\\
146.309860229492 -11.3580627441406\\
146.322372436523 -11.35817527771\\
146.334869384766 -11.3582801818848\\
146.347366333008 -11.3583850860596\\
146.35986328125 -11.3584899902344\\
146.372360229492 -11.3585948944092\\
146.384872436523 -11.358699798584\\
146.397369384766 -11.3588047027588\\
146.409866333008 -11.3589096069336\\
146.42236328125 -11.3590145111084\\
146.434860229492 -11.3591165542603\\
146.447372436523 -11.3592119216919\\
146.459869384766 -11.3593072891235\\
146.472366333008 -11.3594026565552\\
146.48486328125 -11.3594980239868\\
146.497360229492 -11.3595933914185\\
146.509872436523 -11.3596887588501\\
146.522369384766 -11.3597841262817\\
146.534866333008 -11.3598709106445\\
146.54736328125 -11.3599576950073\\
146.559860229492 -11.3600511550903\\
146.572372436523 -11.3601369857788\\
146.584869384766 -11.3602228164673\\
146.597366333008 -11.3603000640869\\
146.60986328125 -11.3603763580322\\
146.622360229492 -11.3604526519775\\
146.634872436523 -11.3605289459229\\
146.647369384766 -11.3606052398682\\
146.659866333008 -11.3606815338135\\
146.67236328125 -11.3607616424561\\
146.684860229492 -11.3608474731445\\
146.697372436523 -11.360933303833\\
146.709869384766 -11.3610191345215\\
146.722366333008 -11.36110496521\\
146.73486328125 -11.3611907958984\\
146.747360229492 -11.3612718582153\\
146.759872436523 -11.3613481521606\\
146.772369384766 -11.361424446106\\
146.784866333008 -11.3615007400513\\
146.79736328125 -11.3615770339966\\
146.809860229492 -11.3616533279419\\
146.822372436523 -11.3617296218872\\
146.834869384766 -11.3618059158325\\
146.847366333008 -11.3618822097778\\
146.85986328125 -11.3619585037231\\
146.872360229492 -11.3620347976685\\
146.884872436523 -11.3621110916138\\
146.897369384766 -11.3621873855591\\
146.909866333008 -11.3622636795044\\
146.92236328125 -11.3623399734497\\
146.934860229492 -11.362416267395\\
146.947372436523 -11.3624992370605\\
146.959869384766 -11.3625888824463\\
146.972366333008 -11.3626880645752\\
146.98486328125 -11.3627958297729\\
146.997360229492 -11.3629112243652\\
147.009872436523 -11.3630352020264\\
147.022369384766 -11.3631591796875\\
147.034866333008 -11.3632764816284\\
147.04736328125 -11.3633871078491\\
147.059860229492 -11.3634872436523\\
147.072372436523 -11.363582611084\\
147.084869384766 -11.3636713027954\\
147.097366333008 -11.3637628555298\\
147.10986328125 -11.3638582229614\\
147.122360229492 -11.3639545440674\\
147.134872436523 -11.3640594482422\\
147.147369384766 -11.3641691207886\\
147.159866333008 -11.3642835617065\\
147.17236328125 -11.3643980026245\\
147.184860229492 -11.3645095825195\\
147.197372436523 -11.3646087646484\\
147.209869384766 -11.3646974563599\\
147.222366333008 -11.3647928237915\\
147.23486328125 -11.3648986816406\\
147.247360229492 -11.3650178909302\\
147.259872436523 -11.3651485443115\\
147.272369384766 -11.3652925491333\\
147.284866333008 -11.3654489517212\\
147.29736328125 -11.3656158447266\\
147.309860229492 -11.3657941818237\\
147.322372436523 -11.365984916687\\
147.334869384766 -11.3661851882935\\
147.347366333008 -11.3663911819458\\
147.35986328125 -11.3665914535522\\
147.372360229492 -11.3667869567871\\
147.384872436523 -11.3669729232788\\
147.397369384766 -11.3671493530273\\
147.409866333008 -11.3673210144043\\
147.42236328125 -11.3674850463867\\
147.434860229492 -11.3676519393921\\
147.447372436523 -11.367823600769\\
147.459869384766 -11.367995262146\\
147.472366333008 -11.3681726455688\\
147.48486328125 -11.368353843689\\
147.497360229492 -11.3685359954834\\
147.509872436523 -11.3687267303467\\
147.522369384766 -11.3689222335815\\
147.534866333008 -11.369122505188\\
147.54736328125 -11.3693294525146\\
147.559860229492 -11.3695449829102\\
147.572372436523 -11.3697834014893\\
147.584869384766 -11.3700389862061\\
147.597366333008 -11.3703050613403\\
147.60986328125 -11.3705720901489\\
147.622360229492 -11.3708381652832\\
147.634872436523 -11.3711004257202\\
147.647369384766 -11.3713579177856\\
147.659866333008 -11.3716077804565\\
147.67236328125 -11.3718519210815\\
147.684860229492 -11.372088432312\\
147.697372436523 -11.3723173141479\\
147.709869384766 -11.372537612915\\
147.722366333008 -11.3727512359619\\
147.73486328125 -11.3729572296143\\
147.747360229492 -11.3731565475464\\
147.759872436523 -11.3733491897583\\
147.772369384766 -11.3735399246216\\
147.784866333008 -11.3737230300903\\
147.79736328125 -11.3739042282104\\
147.809860229492 -11.3740844726563\\
147.822372436523 -11.3742580413818\\
147.834869384766 -11.3744297027588\\
147.847366333008 -11.3746013641357\\
147.85986328125 -11.3747730255127\\
147.872360229492 -11.3749446868896\\
147.884872436523 -11.3751163482666\\
147.897369384766 -11.3752880096436\\
147.909866333008 -11.3754596710205\\
147.92236328125 -11.3756313323975\\
147.934860229492 -11.3757925033569\\
147.947372436523 -11.3759479522705\\
147.959869384766 -11.3761110305786\\
147.972366333008 -11.3762836456299\\
147.98486328125 -11.37646484375\\
147.997360229492 -11.3766460418701\\
148.009872436523 -11.3768272399902\\
148.022369384766 -11.3770084381104\\
148.034866333008 -11.3771905899048\\
148.04736328125 -11.3773803710938\\
148.059860229492 -11.3775701522827\\
148.072372436523 -11.377760887146\\
148.084869384766 -11.3779516220093\\
148.097366333008 -11.3781423568726\\
148.10986328125 -11.3783397674561\\
148.122360229492 -11.3785381317139\\
148.134872436523 -11.3787269592285\\
148.147369384766 -11.37890625\\
148.159866333008 -11.3790826797485\\
148.17236328125 -11.3792676925659\\
148.184860229492 -11.3794641494751\\
148.197372436523 -11.3796644210815\\
148.209869384766 -11.3798656463623\\
148.222366333008 -11.3800745010376\\
148.23486328125 -11.3802843093872\\
148.247360229492 -11.3804941177368\\
148.259872436523 -11.3807039260864\\
148.272369384766 -11.380913734436\\
148.284866333008 -11.3811235427856\\
148.29736328125 -11.3813247680664\\
148.309860229492 -11.3815155029297\\
148.322372436523 -11.381706237793\\
148.334869384766 -11.3819036483765\\
148.347366333008 -11.3821039199829\\
148.35986328125 -11.3823041915894\\
148.372360229492 -11.3825044631958\\
148.384872436523 -11.3827047348022\\
148.397369384766 -11.3829050064087\\
148.409866333008 -11.3831052780151\\
148.42236328125 -11.3833055496216\\
148.434860229492 -11.3835020065308\\
148.447372436523 -11.3836879730225\\
148.459869384766 -11.3838586807251\\
148.472366333008 -11.3840208053589\\
148.48486328125 -11.3841857910156\\
148.497360229492 -11.3843584060669\\
148.509872436523 -11.384539604187\\
148.522369384766 -11.3847208023071\\
148.534866333008 -11.3849020004272\\
148.54736328125 -11.3850831985474\\
148.559860229492 -11.3852643966675\\
148.572372436523 -11.3854455947876\\
148.584869384766 -11.3856267929077\\
148.597366333008 -11.3858079910278\\
148.60986328125 -11.3859891891479\\
148.622360229492 -11.3861703872681\\
148.634872436523 -11.3863515853882\\
148.647369384766 -11.3865327835083\\
148.659866333008 -11.3867206573486\\
148.67236328125 -11.3869104385376\\
148.684860229492 -11.3871002197266\\
148.697372436523 -11.3872909545898\\
148.709869384766 -11.3874807357788\\
148.722366333008 -11.3876714706421\\
148.73486328125 -11.3878622055054\\
148.747360229492 -11.3880529403687\\
};
\addplot [
color=mycolor1,
solid
]
table[row sep=crcr]{
148.747360229492 -11.3880529403687\\
148.759872436523 -11.3882436752319\\
148.772369384766 -11.3884344100952\\
148.784866333008 -11.3886251449585\\
148.79736328125 -11.3888082504272\\
148.809860229492 -11.3889894485474\\
148.822372436523 -11.3891706466675\\
148.834869384766 -11.3893556594849\\
148.847366333008 -11.3895454406738\\
148.85986328125 -11.3897361755371\\
148.872360229492 -11.3899269104004\\
148.884872436523 -11.3901176452637\\
148.897369384766 -11.390308380127\\
148.909866333008 -11.3904991149902\\
148.92236328125 -11.3906898498535\\
148.934860229492 -11.3908805847168\\
148.947372436523 -11.3910713195801\\
148.959869384766 -11.3912582397461\\
148.972366333008 -11.3914480209351\\
148.98486328125 -11.3916387557983\\
148.997360229492 -11.3918294906616\\
149.009872436523 -11.3920202255249\\
149.022369384766 -11.392219543457\\
149.034866333008 -11.3924198150635\\
149.04736328125 -11.3926200866699\\
149.059860229492 -11.3928203582764\\
149.072372436523 -11.3930206298828\\
149.084869384766 -11.3932209014893\\
149.097366333008 -11.3934211730957\\
149.10986328125 -11.3936214447021\\
149.122360229492 -11.3938217163086\\
149.134872436523 -11.394021987915\\
149.147369384766 -11.3942222595215\\
149.159866333008 -11.3944225311279\\
149.17236328125 -11.3946228027344\\
149.184860229492 -11.3948230743408\\
149.197372436523 -11.3950233459473\\
149.209869384766 -11.3952236175537\\
149.222366333008 -11.3954229354858\\
149.23486328125 -11.395622253418\\
149.247360229492 -11.3958177566528\\
149.259872436523 -11.3960084915161\\
149.272369384766 -11.3961887359619\\
149.284866333008 -11.3963460922241\\
149.29736328125 -11.3964939117432\\
149.309860229492 -11.3966665267944\\
149.322372436523 -11.3968639373779\\
149.334869384766 -11.3970775604248\\
149.347366333008 -11.3973007202148\\
149.35986328125 -11.3975324630737\\
149.372360229492 -11.3977746963501\\
149.384872436523 -11.3980293273926\\
149.397369384766 -11.3982982635498\\
149.409866333008 -11.3985805511475\\
149.42236328125 -11.3988733291626\\
149.434860229492 -11.3991765975952\\
149.447372436523 -11.399489402771\\
149.459869384766 -11.3998117446899\\
149.472366333008 -11.4001436233521\\
149.48486328125 -11.4004859924316\\
149.497360229492 -11.4008378982544\\
149.509872436523 -11.4011974334717\\
149.522369384766 -11.4015645980835\\
149.534866333008 -11.4019365310669\\
149.54736328125 -11.4023141860962\\
149.559860229492 -11.4026947021484\\
149.572372436523 -11.4030752182007\\
149.584869384766 -11.4034442901611\\
149.597366333008 -11.4037847518921\\
149.60986328125 -11.4040613174438\\
149.622360229492 -11.404278755188\\
149.634872436523 -11.4044790267944\\
149.647369384766 -11.4046792984009\\
149.659866333008 -11.4048843383789\\
149.67236328125 -11.4050941467285\\
149.684860229492 -11.405309677124\\
149.697372436523 -11.4055328369141\\
149.709869384766 -11.40576171875\\
149.722366333008 -11.4059991836548\\
149.73486328125 -11.4062376022339\\
149.747360229492 -11.4064836502075\\
149.759872436523 -11.4067316055298\\
149.772369384766 -11.4069795608521\\
149.784866333008 -11.4072275161743\\
149.79736328125 -11.407470703125\\
149.809860229492 -11.4077043533325\\
149.822372436523 -11.4079113006592\\
149.834869384766 -11.4080848693848\\
149.847366333008 -11.408242225647\\
149.85986328125 -11.4084072113037\\
149.872360229492 -11.4085969924927\\
149.884872436523 -11.4088191986084\\
149.897369384766 -11.4090776443481\\
149.909866333008 -11.4093761444092\\
149.92236328125 -11.4097156524658\\
149.934860229492 -11.4100914001465\\
149.947372436523 -11.4104986190796\\
149.959869384766 -11.4109258651733\\
149.972366333008 -11.4113464355469\\
149.98486328125 -11.4117288589478\\
149.997360229492 -11.4120759963989\\
};
\addlegendentry{$\dtmax{} = \expneg{1.25}{3}{[s]}$};

\end{axis}
\end{tikzpicture}%
\caption{Condensation in upper head as calculated with domain decomposition algorithm.}
\label{fig:refillGammaDom}
\end{figure}

Running simulation with the domain decomposition active produces the condensation rates shown in \fig{fig:refillGammaDom}.
The two channels in section two that represent the core region and the four channels in section three above the core region comprise the nonlinear domain in this problem.
These results, while not as consistent as those produced by the global nonlinear solver, are both more consistent and more accurate than those produced by the linear solver.

\begin{figure}[t!hb]
\centering
% This file was created by matlab2tikz v0.4.3.
% Copyright (c) 2008--2013, Nico Schlömer <nico.schloemer@gmail.com>
% All rights reserved.
% 
\tikzsetnextfilename{plots/refillMaxGamma_pdf}
\begin{tikzpicture}

\begin{axis}[%
width=\mytikzpicwidth,
height=\mytikzpicheight,
scale only axis,
x dir=reverse,
%xmode=log,
xmin=0.00125,
xmax=0.01,
xminorticks=true,
xlabel={$\dtmax{} [\text{s}]$},
ymin=-30,
ymax=-10,
ylabel={$\displaystyle \int_{0}^{T}\dot{\Gamma}\, \mathrm{d}\tau \, [\,\lbm{}\,]$},
legend style={at={(0.97,0.03)},anchor=south east,draw=black,fill=white,legend cell align=left}
]
\addplot [
color=black,
solid,
mark=*,
mark options={solid}
]
table[row sep=crcr]{
0.01 -29.6177730560303\\
0.005 -22.8581886291504\\
0.0025 -20.1671562194824\\
0.00125 -20.2451515197754\\
};
\addlegendentry{Linear};

\addplot [
color=black,
solid,
mark=square,
mark options={solid}
]
table[row sep=crcr]{
0.01 -11.5944890975952\\
0.005 -11.6384372711182\\
0.0025 -11.5849742889404\\
0.00125 -11.6548013687134\\
};
\addlegendentry{Nonlinear};

\addplot [
color=black,
solid,
mark=diamond,
mark options={solid}
]
table[row sep=crcr]{
0.01 -12.8674726486206\\
0.005 -12.7719497680664\\
0.0025 -12.1526460647583\\
0.00125 -11.412148475647\\
};
\addlegendentry{Dual Domains};

\end{axis}
\end{tikzpicture}%
\caption{Integrated condensation in upper head for all three algorithms.}
\label{fig:refillMaxGamma}
\end{figure}

Next, the integrated condensation for each of the three solution techniques are shown in \fig{fig:refillMaxGamma}.
This figure shows the integrated condensate for each of the \dtmax{} for each solver.
Note that the linear solver does not produce results that are comparable with the nonlinear solver even at small timestep sizes.
By using the domain decomposition algorithm, a solution that is quantitatively and qualitatively more consistent with the nonlinear solution is obtained.

\begin{figure}[h!tb]
\centering
% This file was created by matlab2tikz v0.4.3.
% Copyright (c) 2008--2013, Nico Schlömer <nico.schloemer@gmail.com>
% All rights reserved.
% 
\tikzsetnextfilename{plots/refillRunTimeRatios_pdf}
\begin{tikzpicture}

\begin{axis}[%
width=\mytikzpicwidth,
height=\mytikzpicheight,
scale only axis,
x dir=reverse,
xmode=log,
xmin=0.001,
xmax=2,
xminorticks=true,
xlabel={$\dtmax{} [\text{s}]$},
ymin=0,
ymax=10,
ylabel={Runtime Ratio [-]},
legend style={at={(0.03,0.97)},anchor=north west,draw=black,fill=white,legend cell align=left}
]
\addplot [
color=black,
solid,
mark=square,
mark options={solid}
]
table[row sep=crcr]{
1.28 6.27895355224609\\
0.64 6.29074716567993\\
0.32 6.65997314453125\\
0.16 6.24629688262939\\
0.08 6.81604242324829\\
0.04 6.36056804656982\\
0.02 6.77778148651123\\
0.01 8.02093887329102\\
0.005 8.30350589752197\\
0.0025 8.6132984161377\\
0.00125 8.62719058990479\\
};
\addlegendentry{Nonlinear/Linear};

\addplot [
color=black,
solid,
mark=*,
mark options={solid}
]
table[row sep=crcr]{
1.28 2.6570348739624\\
0.64 2.63642430305481\\
0.32 2.79442429542542\\
0.16 2.67521476745605\\
0.08 2.8906033039093\\
0.04 2.71288633346558\\
0.02 2.81551170349121\\
0.01 3.08566451072693\\
0.005 3.11905431747437\\
0.0025 3.22687935829163\\
0.00125 3.19046020507813\\
};
\addlegendentry{Dual Domains/Linear};

\end{axis}
\end{tikzpicture}%
\caption{Ratios of nonlinear and dual-domain run times to linear run time.}
\label{fig:refillRunTimeRatios}
\end{figure}

Next, the computational cost of the nonlinear and the domain decomposition solution methods are compared in \fig{fig:refillRunTimeRatios}.
This figure shows the ratio of the full nonlinear and the domain decomposition run times to the linear solver's run time.
While the nonlinear solver is approximately eight times more computationally expensive than the linear solver, the domain decomposition is only three times more computationally expensive than the linear solver.
That is a consistent 60\% run time savings while using the domain decomposition algorithm as compared to the full nonlinear solver.      % Chapter: Numerical Experiments
\chapter{Summary of Findings}
\label{chap:end}
%A metric to quantify the nonlinear convergence of a timestep-size insensitive simulation was developed and implemented.

This dissertation has focused on developing a novel method for solving the nonlinear PDEs associated with thermal-hydraulic safety analysis software.
Traditional methods used in thermal-hydraulic safety analysis software involve solving large systems of nonlinear equations.
Some methods resolve those nonlinearities by using some form of iterative nonlinear refinement technique.
Other methods linearize the nonlinear equations and attempt to control the nonlinear truncation error through the use of timestep size refinement.
The insensitivity of the solution to timestep refinement is taken as an indicator of a converged solution. 
However, the error caused by the linearization can dominate that produced by the temporal discretization, even at smaller timesteps.
This nonlinear truncation error can a solution to remain unchanged with reduction in timestep size.
Until the nonlinear error is eliminated through the use of an iterative solver, temporal convergence cannot be assured.
However, the application of a nonlinear solver can incur excessive computationally expensive.
These two paradigms have stood at the opposite ends of a spectrum, while the middle ground has yet to be investigated.
This research has developed a means of providing a middle ground, a spatially-selective, nonlinear refinement algorithm.

During the course of this work, the two-fluid, three-field software \cobra{} was converted from a linearized semi-implicit solver to a nonlinearly convergent solver.
As part of that development, an operator-based scaling that provides a physically meaningful convergence measure was developed and implemented.
This operator-based approach allows for residual scaling that eliminate inherent bias due to the order of magnitude of the terms in the equations; as such, this localized scale factor normalizes its residual equation to between zero and one.
The use of this scale factor is integral to the effective use of the nonlinear solver in subsequent analysis.

The nonlinear solver utilizing the operator-based scale factor was used to evaluate the roles of nonlinearities in temporal convergence.
Two problem were used to determine the impact of nonlinearities upon the efficacy of the linear and the nonlinear solver.
It was found that not resolving the nonlinearities present in a simulation may result in situations where that timestep-size insensitivity is an artifact brought about degraded order of temporal accuracy caused by linearizing the discrete nonlinear equations.
Resolving the nonlinearities at every timestep not only provided a more consistent solution during temporal convergence studies, but also allowed for convergence to a different solution than that obtained by taking only a single Newton step during each timestep.
However, in simulations where nonlinearities were expected to be low, it was found that the linear solver provided as accurate a solution as that produced by the nonlinear solver.

A problem where nonlinearities were isolated to a give portion of the domain was run to determine the impact of resolving those localized nonlinearities upon the global solution.
By using a linear solver, the solution exhibited nonphysical behavior in all portion of the domain.
Use of the nonlinear solver eliminated the nonphysical behavior from the solution.
Through the use of the domain decomposition algorithm outlined in this work, it was found that eliminating the spatially isolable nonlinearity produced a global solution that was able to more accurately reflect the analytic solution.
When the nonlinearities were not part of the nonlinear subdomain, the entire domain still exhibited spurious nonphysical behavior at large timestep sizes.
These results emphasized the need to resolve nonlinearities and the usefulness of being able to resolve localized nonlinearities.

Resolving the local nonlinearities requires the use of the domain decomposition algorithm. 
To test the implementation of the domain decomposition algorithm, a geometrically complex problem was developed.
When the domain decomposition algorithm is used to subject the nonlinear subdomain to only a single linearization, the solution obtained should analytically match that from the traditional linear solver.
By running a sample of random simulation with various domain decompositions, it was shown that the obtained solutions matched those obtained by the linear solver to numerical round-off.
These results indicate that the mathematical formulation is accurate and that the implementation was carried out correctly.

As a final evaluation of the domain decomposition algorithm, a simple LWR model was developed.
This simulation modeled the refill portion of an accident scenario.
When observing engineering parameters of interest, such as condensation from the safety injection nozzle in the upper head, the nonlinear solver demonstrated a more temporally converged solution than the linear solver.
The linear solution was shown to converge to a different solution than that obtained by the nonlinear solver.
The domain decomposition algorithm was capable of generating a solution that more temporally consistent than that obtained by the linear solver's, and one that was more qualitatively in agreement with the nonlinear solver's solution, with only approximately one-third the computational effort.
This problem demonstrated that the domain decomposition algorithm may be useful in obtained nonlinearly and temporally converged safety simulations with less computational cost than traditional nonlinear solvers.

In summary, a nonlinear solver will assist in achieving a temporally converged simulation at larger timestep sizes during timestep sensitivity studies.
Unfortunately, nonlinear solvers are computationally expensive when multiple nonlinear iterates are required to resolve the nonlinearities.
However, in problem where the spatial location of the nonlinearities can be determined by engineering judgment, the use of the domain decomposition algorithm is warranted.
By selecting those areas of the domain where the nonlinearities are expected to be high and subjecting only them to multiple nonlinear iterations, the consistency of the nonlinear solver may be obtained at a lower computational cost than the full nonlinear solver.

\section{Areas of Future Work}
\label{sect:futureWork}
During this work several opportunities for follow-on research presented themselves.
These research opportunities include the development and implementation of a spatially and temporally adaptive version of the domain decomposition algorithm, the evaluation of theoretical computational costs for the domain decomposition, the investigation of the interaction of the nonlinear solver and the solid structures within \cobra{}, and the ability to dynamically switch governing equations for a given spatial location during the transient.
The following section will outline each of the above mentioned avenues of research in turn.

The current research produced a domain decomposition framework that requires engineering judgment to select portions of the domain where nonlinear convergence may require multiple Newton steps.
A possible extension of this work would be the development of an algorithm that identifies areas where additional Newton steps would be advantageous.
Since, over the course of a given simulation, the spatial location with the greatest nonlinearities may shift, being able to change the nonlinear domain as the transient progressed would provide a reduction in the computational cost.
Once a viable determination of when and where additional nonlinear iterates might be advantageous, a method for dynamically generating the pressure matrices would need to be developed.
Currently, the matrices are preallocated and of fixed size through the simulation.
If the two domains were to shift during the course of the transient, the pressure matrices and the corresponding ordinals of continuity volumes would need to be changed to reflect the new domains.

The current operator-based scaling method has conflicting implementations for determining the magnitude of the divergence operators.
For the momentum equations the net divergence is used as the operator for the scale factor, while the mass and energy equations consider each discrete portion of the surface integral to be an independent operator.
Studies looking at the efficiency of the two different interpretations are needed to clarify which is the better option.

The current implementation of the domain decomposition algorithm requires engineering judgment to determine which portion of the domain is subjected to subsequent nonlinearities.
However, there is a relationship between the number of surfaces connecting the linear domain to the nonlinear domain and the computational expense of solving the problem.
There is also additional computational overhead in setting up the infrastructure for the dual-domains.
There would need to be an investigation into the computational costs of solving the dual-domain problem as opposed to solving the entire domain using the nonlinear solver.

Additionally, the use of the nonlinear solver, in the presence of fluid-wall heat transfer produces non-convergent behavior in test problems.
This work was centered on the hydrodynamics of the reactor core, without consideration of the heat transfer.
A detailed study and analysis of the interaction of the explicit heat transfer and the iterative nonlinear solver for the hydrodynamic needs to be conducted.

The \cobra{} software possesses the ability to switch governing differential equations for the momentum equations if certain criteria are met.
The primary use of this feature is the implementation of a counter-current flow limit boundary condition, which switch governing equations based upon flow regimes.
During the course of this work, the interaction between the iterative solver and the decision on when to switch equations was found to produce excessive equation switching in some models.
The reason for this are as of yet unknown, and additional research is required to identify the root cause.   % Chapter: Conclusion and musings

%=======================================================================
% Bibliography
%=======================================================================
\bibliographystyle{plainnat}
\bibliography{bibliography/references}

%=======================================================================
% Appendices
%=======================================================================
%\noappendixtables
%\noappendixfigures
%\begin{appendices}
%\end{appendices}

%=======================================================================
% End Document
%=======================================================================
\end{document}