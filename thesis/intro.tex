\group{Introduction}

\subgroup{Motivation}
Of primary use in the field of nuclear reactor safety analysis is simulation.
The ability to predict the behaviour of reactors during off-normal events is the key to the licensing and the operation of nuclear power plants.
Within the United States, this simulation capacity is provided by a relatively small number of main stream software suites, among which are the RELAP variants, COBRA variants, and MELCOR.
While each of these software products varies in their models and implementations, the underlying numeric techniques and capabilities are similar.
Traditionally, these system codes utilize a semi-implicit discritization scheme.
In order to solve the resulting system of equations, the most common methodology is to take a single netwon step. 
The underlying numeric methods are first order methods.

% New Paragraph
The ability to use higher order methods is important.
Since the development of the semi-implicit method, whose form was motivated by the limited computer resources available at the time, there have been great advances in both the methodology used to solve linear algebra problems and computer capabilities.

\subgroup{Objectives}
The objective of this dissertation is the design, implementation, and evaluation of a practical non-linear solution framework for reactor safety systems codes.
Specifically, an efficient and reliable solution methodology to the two-phase, three-field, fluid-dynamics and the solid-structure heat transfer system of coupled non-linear partial differential equations is sought.
The specific methodology should be capable of obtaining a consistent solution to the system of PDEs while also possessing.

\begin{figure} %---------------------------------------------------------------
{\singlespace\tt\footnotesize}\caption{A Sample2 \LaTeX{} File}
\label{intro:sample1}
\end{figure}


\begin{figure} %---------------------------------------------------------------
{\singlespace\tt\footnotesize}\caption{A Sample2 \LaTeX{} File}
\label{intro:sample2}
\end{figure}