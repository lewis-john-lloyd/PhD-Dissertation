% This file was created with JabRef 2.8.1.
% Encoding: UTF8

@ARTICLE{Aktas1996,
  author = {B. Aktas and J. H. Mahaffy},
  title = {A two-phase level tracking method},
  journal = {Nuclear Engineering and Design},
  year = {1996},
  volume = {162},
  pages = {271 - 280},
  number = {2–3},
  abstract = {Interfacial closure models in most two-fluid system codes for reactor
	safety are usually tied to the flow regime map through the mean void
	fraction in a computational cell. When a void fraction discontinuity
	exists in a computational volume, neither heat nor momentum exchange
	at the phase interface for this particular cell can be properly represented
	in finite-difference equations governing the fluid flow. Moreover,
	finite-difference methods with a fixed, Eulerian grid will inaccurately
	predict the cell-to-cell convection of mass, momentum and energy
	when the mean cell macroscopic variables are convected from the cell
	containing the void fraction front. The adequate modeling of two-phase
	mixture levels requires the knowledge of front position and void
	fractions above and below the front. In order to obtain such information,
	an efficient and simple tracking method was implemented in the TRAC-BWR
	code (released April 1984). We have tested this method with a simple
	problem involving a moving two-phase air/water mixture level. The
	results revealed inconsistencies in the behavior of velocities, pressures
	and interfacial friction, and some bounded numerical oscillations.
	Following our numerical experiment, we developed a systematic approach
	to improve the two-phase level tracking method. We present this approach
	and the results of implementation in the TRAC-BWR code.},
  doi = {10.1016/0029-5493(95)01132-3},
  file = {Aktas1996.pdf:Aktas1996.pdf:PDF},
  issn = {0029-5493},
  owner = {llloyd},
  timestamp = {2012.10.20},
  url = {http://www.sciencedirect.com/science/article/pii/0029549395011323}
}

@PHDTHESIS{Ashwood2010,
  author = {Ashwood, Andrea C.},
  title = {Characterization of the Macroscopic and Microscopic Mechanics in
	Vertial and Horizontal Annular Flow},
  school = {University of Wisconsin - Madison},
  year = {2010},
  keywords = {Annular Flow, Two-Phase Flow, Droplet},
  owner = {lloydlj},
  timestamp = {2011.08.04}
}

@CONFERENCE{Aumiller2002,
  author = {D. L. Aumiller and E. T. Tomlinson and R. C. Bauer},
  title = {Incorporation of {COBRA-TF} in an Integrated Code System With {RELAP5-3D}
	Using Semi-Implicit Coupling},
  booktitle = {2002 RELAP5 International Users Seminar},
  year = {2002},
  month = {September},
  file = {Aumiller2002.pdf:Aumiller2002.pdf:PDF},
  owner = {llloyd},
  timestamp = {2012.11.12},
  url = {http://www.inl.gov/relap5/rius/parkcity/aumiller.pdf}
}

@ARTICLE{Aumiller2001,
  author = {D. L. Aumiller and E. T. Tomlinson and R. C. Bauer},
  title = {A coupled {RELAP5-3D/CFD} methodology with a proof-of-principle calculation},
  journal = {Nuclear Engineering and Design},
  year = {2001},
  volume = {205},
  pages = {83 - 90},
  number = {1-2},
  abstract = {The RELAP5-3D computer code was modified to make the explicit coupling
	capability in the code fully functional. As a test of the modified
	code, a coupled RELAP5/RELAP5 analysis of the Edwards–O'Brien blowdown
	problem was performed which showed no significant deviations from
	the standard RELAP5-3D predictions. In addition, a multiphase Computational
	Fluid Dynamics (CFD) code was modified to permit explicit coupling
	to RELAP5-3D. Several calculations were performed with this code.
	The first analysis used the experimental pressure history from a
	point just upstream of the break as a boundary condition. This analysis
	showed that a multiphase CFD code could calculate the thermodynamic
	and hydrodynamic conditions during a rapid blowdown transient. Finally,
	a coupled RELAP5/CFD analysis was performed. The results are presented
	in this paper.},
  doi = {10.1016/S0029-5493(00)00370-8},
  file = {Aumiller2001.pdf:Aumiller2001.pdf:PDF},
  issn = {0029-5493},
  owner = {llloyd},
  timestamp = {2012.11.12},
  url = {http://www.sciencedirect.com/science/article/pii/S0029549300003708}
}

@CONFERENCE{Aumiller2000,
  author = {Aumiller, D. L. and Tomlinson, E. T. and Clarke, W. G.},
  title = {A New Assessment of {RELAP5-3D} Using A General Electric Level Swell
	Problem},
  booktitle = {2000 RELAP5 Users Seminar},
  year = {2000},
  address = {Jackson Hole, Wyoming},
  month = {September},
  organization = {2000 RELAP5 Users Seminar},
  comment = {p},
  file = {Aumiller2000.pdf:Aumiller2000.pdf:PDF},
  keywords = {Validation, General Electric, Level Swell, RELAP5, Systems Codes,
	Two-Phase Flow},
  owner = {lloydlj},
  timestamp = {2011.08.03}
}

@ARTICLE{Auvinen2008,
  author = {A. Auvinen and G. Brillant and N. Davidovich and R. Dickson and G.
	Ducros and Y. Dutheillet and P. Giordano and M. Kunstar and T. Kärkelä
	and M. Mladin and Y. Pontillon and C. Séropian and N. Vér},
  title = {Progress on ruthenium release and transport under air ingress conditions},
  journal = {Nuclear Engineering and Design},
  year = {2008},
  volume = {238},
  pages = {3418 - 3428},
  number = {12},
  abstract = {A particular concern in the event of a hypothetical severe accident
	is the potential release of highly radiotoxic fission product (FP)
	isotopes of ruthenium. The highest risk for a large quantity of these
	isotopes to reach the containment arises from air ingress following
	vessel melt-through. One work package (WP) of the source term topic
	of the EU 6th Framework Network of Excellence project SARNET is producing
	and synthesizing information on ruthenium release and transport with
	the aim of validating or improving the corresponding modelling in
	the European ASTEC severe accident analysis code. The WP includes
	reactor scenario studies that can be used to define conditions for
	new experiments. The experimental database currently being reviewed
	includes the following programmes: - AECL experiments conducted on
	fission product release in air; results are relevant to CANDU loss
	of end-fitting accidents; - VERCORS tests on FP release and transport
	conducted by CEA in collaboration with IRSN and EDF; additional tests
	may potentially be conducted in more oxidizing conditions in the
	VERDON facility; - RUSET tests by AEKI investigating ruthenium transport
	with and without other FP simulants; - Experiments by VTT on ruthenium
	transport and speciation in highly oxidizing conditions. In addition
	to the above, at IRSN and at ENEA modelling of fission product release
	and of fuel oxidation is being pursued, the latter being an essential
	boundary condition influencing ruthenium release. Reactor scenario
	studies have been carried out at INR, EDF and IRSN: calculations
	of air ingress scenarios with respectively ICARE/CATHARE V2; SATURNE-MAAP;
	and ASTEC codes provided first insights of thermal-hydraulic conditions
	that the fuel may experience after lower head vessel failure. This
	paper summarizes the status of this work and plans for the future.},
  doi = {DOI: 10.1016/j.nucengdes.2008.07.010},
  file = {Auvinen2008.pdf:Auvinen2008.pdf:PDF},
  issn = {0029-5493},
  keywords = {htgr, unread},
  owner = {lewis.lloyd},
  timestamp = {2009.04.13},
  url = {http://www.sciencedirect.com/science/article/B6V4D-4TK2VJ2-1/2/ca5f7514f9b9a233f4391da0018e5943}
}

@CONFERENCE{Avramova2006,
  author = {Avramova, M. and Cuervo, D. and Ivanov, K.},
  title = {Improvements and Applications of {COBRA-TF} for Stand-Alone and Coupled
	{LWR} Safety Analyses},
  booktitle = {{PHYSOR}-2006, ANS Topical Meeting on Reactor Physics},
  year = {2006},
  month = {September},
  organization = {Canadian Nuclear Society},
  abstract = {The advanced thermal-hydraulic subchannel code COBRA-TF has been 
	
	recently improved and applied for stand-alone and coupled LWR core
	
	
	calculations at the Pennsylvania State University in cooperation with
	
	
	AREVA NP GmbH, Germany and the Technical University of Madrid. 
	
	To enable COBRA-TF for academic and industrial applications including
	
	
	safety margins evaluations and LWR core design analyses, the code
	programming, 
	
	numerics, and basic models were revised and substantially improved.
	The 
	
	code has undergone through an extensive validation, verification,
	and 
	
	qualification program.},
  file = {Avramova2006.pdf:Avramova2006.pdf:PDF},
  keywords = {COBRA},
  owner = {llloyd},
  timestamp = {2012.10.22}
}

@ARTICLE{Ball2006,
  author = {Syd Ball},
  title = {Sensitivity studies of modular high-temperature gas-cooled reactor
	postulated accidents},
  journal = {Nuclear Engineering and Design},
  year = {2006},
  volume = {236},
  pages = {454 - 462},
  number = {5-6},
  note = {HTR-2004},
  abstract = {The results of various accident scenario simulations for the two major
	modular high temperature gas-cooled reactor (HTGR) variants (prismatic
	and pebble bed cores) are presented. Sensitivity studies can help
	to quantify the uncertainty ranges of the predicted outcomes for
	variations in some of the more crucial system parameters, as well
	as for occurrences of equipment and/or operator failures or errors.
	In addition, sensitivity studies can guide further efforts in improving
	the design and determining where more (or less) R&D is appropriate.
	Both of the modular HTGR designs studied - the 400-MW(t) pebble bed
	modular reactor (PBMR, pebble) and the 600-MW(t) gas-turbine modular
	helium reactor (GT-MHR, prismatic) - show excellent accident prevention
	and mitigation capabilities because of their inherent passive safety
	features. The large thermal margins between operating and #potential##damage#
	temperatures, along with the typically very slow accident response
	times (approximate days to reach peak temperatures), tend to reduce
	concerns about uncertainties in the simulation models, the initiating
	events, and the equipment and operator responses.},
  doi = {DOI: 10.1016/j.nucengdes.2005.10.029},
  file = {Ball2006.pdf:Ball2006.pdf:PDF},
  issn = {0029-5493},
  keywords = {htgr, unread},
  owner = {lewis.lloyd},
  timestamp = {2009.04.13},
  url = {http://www.sciencedirect.com/science/article/B6V4D-4J625Y7-1/2/573c7624af4758d0882097554832762c}
}

@ARTICLE{Ball2008,
  author = {Syd Ball and Matt Richards and Sergey Shepelev},
  title = {Sensitivity studies of air ingress accidents in modular HTGRs},
  journal = {Nuclear Engineering and Design},
  year = {2008},
  volume = {238},
  pages = {2935 - 2942},
  number = {11},
  note = {HTR-2006: 3rd International Topical Meeting on High Temperature Reactor
	Technology},
  abstract = {Postulated air ingress accidents, while of very low probability in
	a modular high-temperature gas-cooled reactor (HTGR), are of considerable
	interest to the plant designer, operator, and regulator because of
	the possibility that the core could sustain significant damage under
	some circumstances. Sensitivity analyses are described that cover
	a wide spectrum of conditions affecting outcomes of the postulated
	accident sequences, for both prismatic and pebble-bed core designs.
	The major factors affecting potential core damage are the size and
	location of primary system leaks, flow path resistances, the core
	temperature distribution, and the long-term availability of oxygen
	in the incoming gas from a confinement building. Typically, all the
	incoming oxygen entering the core area is consumed within the reactor
	vessel, so it is more a matter of where, not whether, oxidation occurs.
	An air ingress model with example scenarios and means for mitigating
	damage are described. Representative designs of modular HTGRs included
	here are a 400-MW(th) pebble-bed reactor (PBR), and a 600-MW(th)
	prismatic-core modular reactor (PMR) design such as the gas-turbine
	modular helium reactor (GT-MHR).},
  doi = {DOI: 10.1016/j.nucengdes.2008.02.021},
  file = {Ball2008.pdf:Ball2008.pdf:PDF},
  issn = {0029-5493},
  keywords = {htgr, unread},
  owner = {lewis.lloyd},
  timestamp = {2009.04.13},
  url = {http://www.sciencedirect.com/science/article/B6V4D-4S9NGHS-1/2/9b1786e750c03179f13eb0fb5d2e2818}
}

@CONFERENCE{Bandini2002,
  author = {Bandini, B. R. and Aumiller, D. L. and Tomlinson, E. T.},
  title = {A New Assessment of the Large-Tank General Electric Swell Problem
	Using RELAP5-3D},
  booktitle = {2002 RELAP5 International Users Seminar},
  year = {2002},
  address = {Park City, Utah},
  month = {September},
  organization = {2002 RELAP5 International Users Seminar},
  file = {Bandini2002.pdf:Bandini2002.pdf:PDF},
  keywords = {Validation, General Electric, Level Swell, RELAP5, Systems Codes,
	Two-Phase Flow},
  owner = {lloydlj},
  timestamp = {2011.08.03}
}

@ARTICLE{Barre1990,
  author = {F. Barre and M. Bernard},
  title = {The {CATHARE} code strategy and assessment},
  journal = {Nuclear Engineering and Design},
  year = {1990},
  volume = {124},
  pages = {257 - 284},
  number = {3},
  abstract = {The French thermal-hydraulic code CATHARE is being developed in Grenoble
	by EDF, FRAMATOME and CEA. The best-estimate objectives of CATHARE
	require a very rigorous methodology for development and assessment.
	In particular, the validation process is performed in two steps:
	qualification of the models against separate effect tests and verification
	of the code on integral loop tests. The validation reports and the
	resulting evaluation document and user guidelines manual are delivered
	to the users. This paper illustrates the methodology for the case
	of CATHARE 1 version 1.3.
	
	The qualification process was conducted with reference to a matrix
	of 300 tests selected from experiments on critical flow, flow pattern
	determination, blowdown of heated or adiabatic test sections in various
	geometries, reflooding, boil-off, steam generator and pump behaviour,
	fuel thermomechanics. The results show the adequacy and the consistency
	of the models but indicate some items to be further studied: droplet
	diameter correlation, incipience of boiling model, critical flow
	in diaphragms, top-down rewet or reflood, interphase friction in
	rod bundle geometry.
	
	The verification process was performed on 21 tests coming from LOBI,
	LOFT, LSTF, PKL facilities. The natural circulation post-test calculations
	show good results for single phase or two-phase mass flowrate evaluation
	and a satisfactory prediction of the reflux condensor phase. The
	various phases of the small break or large break transients are generally
	well described by CATHARE. However, some quantitative discrepancies
	underline limited deficiencies in the domain of condensation at safety
	injection, CCFL at steam generator inlet or core upper tie plate,
	multidimensional effects in annular downcomer and in core region,
	modelling of steam generator secondary side.
	
	The consideration of these items led to the introduction of significant
	improvements and new capabilities in the CATHARE 2 version which
	consequently has to be validated following the same two-step process
	and with an extended test matrix.
	
	The efficiency of such a rigorous methodology has induced the French
	nuclear industry (EDF, FRAMATOME) to use CATHARE for safety research
	and development studies.},
  doi = {10.1016/0029-5493(90)90296-A},
  file = {Barre1990.pdf:Barre1990.pdf:PDF},
  issn = {0029-5493},
  owner = {llloyd},
  timestamp = {2012.11.14},
  url = {http://www.sciencedirect.com/science/article/pii/002954939090296A}
}

@ARTICLE{Barre1993,
  author = {F. Barre and M. Parent and B. Brun},
  title = {Advanced numerical methods for thermalhydraulics},
  journal = {Nuclear Engineering and Design},
  year = {1993},
  volume = {145},
  pages = {147 - 158},
  number = {1–2},
  abstract = {New applications have been found for advanced best estimate codes,
	due to the needs for industrial studies and safety studies, and due
	also to the appearance of new generations of computers, such as powerful
	work stations. Thus developments have been focused on the optimization
	of the numerical performance of thermal-hydraulics codes based on
	the two-fluid six equation model.
	
	One important goal is that of minimizing the computing time. Comparisons
	of existing methods demonstrate that a fully implicit scheme is well
	suitable for optimization.
	
	A second field of investigation is to develop powerful three-dimensional
	modules with advanced numerical methods, as multistep method, conjugate
	gradient method for the solver and, in future, non-structured meshing.},
  doi = {10.1016/0029-5493(93)90064-G},
  file = {Barre1993.pdf:Barre1993.pdf:PDF},
  issn = {0029-5493},
  owner = {llloyd},
  timestamp = {2012.10.28},
  url = {http://www.sciencedirect.com/science/article/pii/002954939390064G}
}

@ARTICLE{Benzi2002,
  author = {Benzi, Michele},
  title = {Preconditioning Techniques for Large Systems: A Survey},
  journal = {Journal of Computational Physics},
  year = {2002},
  volume = {182},
  pages = {418-477},
  month = {July},
  file = {Benzi2002.pdf:Benzi2002.pdf:PDF},
  keywords = {Newton-Krylov, Mathematics, Linear Algebra, Preconditioners},
  owner = {lloydlj},
  timestamp = {2011.09.15}
}

@ARTICLE{Bertodano1996,
  author = {M. Lopez de Bertodano and A. Becker and A. Sharon and R. Schnider},
  title = {DCH dispersal and entrainment experiment in a scaled annular cavity},
  journal = {Nuclear Engineering and Design},
  year = {1996},
  volume = {164},
  pages = {271 - 285},
  number = {1-3},
  abstract = {The objective of this experiment was to measure the amount of corium
	dispersal and the droplet size distribution during high pressure
	melt ejection from a CE reactor. The melt and the steam flowed to
	the containment through a narrow annular cavity. The experiment was
	carried out on a 1/20th scaled model of the cavity and the containment.
	The scaling was based on dimensionless numbers obtained from a two-phase
	flow model of the dispersal and entrainment mechanisms in the cavity.
	Furthermore, the model shows that the flow in the cavity was choked,
	so high levels of dispersal and entrainment were possible. The experiment
	consisted of air--water, air--helium, air--woods metal and helium--woods
	metal tests; the main result being that the level of dispersal was
	very high in all cases. The woods metal data supported a separated
	flow model in the cavity, implying that the gas choked velocity was
	very high and the droplets very small. In contrast, the measured
	drop sizes for the water tests were much larger than the separated
	flow model predictions. This discrepancy could not be resolved because
	the entrainment mechanism is not properly understood at the present
	time.},
  doi = {DOI: 10.1016/0029-5493(96)01223-X},
  file = {Bertodano1996.pdf:Bertodano1996.pdf:PDF},
  issn = {0029-5493},
  keywords = {liquid.metal.jet, unread},
  owner = {lewis.lloyd},
  timestamp = {2009.04.13},
  url = {http://www.sciencedirect.com/science/article/B6V4D-3VTJ9YC-F/2/115f8858bcb1889cfb89de99924dd62a}
}

@CONFERENCE{Bestion2000,
  author = {D. Bestion},
  title = {The Phase Appearance and Disapperance in the {CATHARE} Code},
  booktitle = {Trends in Numerical and Physical Modeling for Industrial Multiphase
	Flows},
  year = {2000},
  month = {September},
  file = {Bestion2000.pdf:Bestion2000.pdf:PDF},
  owner = {llloyd},
  timestamp = {2012.11.16}
}

@ARTICLE{Bestion1990,
  author = {D. Bestion},
  title = {The physical closure laws in the CATHARE code},
  journal = {Nuclear Engineering and Design},
  year = {1990},
  volume = {124},
  pages = {229 - 245},
  number = {3},
  abstract = {CATHARE is a 2-fluid thermal-hydraulic code capable of simulating
	thermal and mechanical phenomena occurring in the primary and secondary
	circuits of PWRs for a wide variety of accidental situations. The
	description of the flow is essentially 1-dimensional. Closure laws
	concerning mass, momentum and energy exchanges between phases and
	between each phase and the walls are required. A set of specifically
	designed separate effect experiments were performed and analysed.
	Having regard for some development principles, correlations are established
	on the basis of experimental data. The mechanical transfer laws are
	derived first from experiments where thermal non equilibrium is negligible.
	Using them as a basis for further interpretation of experimental
	data, interfacial heat transfer laws are then developed. Wall heat
	transfer correlations then have to be fixed. All these steps are
	presented with emphasis being placed on the most recent developments.
	These last investigations concern the direct contact condensation,
	stratification model, wall friction, droplet break up and the scale
	effect, geometrical effect and pressure effect on interfacial friction.},
  doi = {10.1016/0029-5493(90)90294-8},
  file = {Bestion1990.pdf:Bestion1990.pdf:PDF},
  issn = {0029-5493},
  owner = {llloyd},
  timestamp = {2012.11.14},
  url = {http://www.sciencedirect.com/science/article/pii/0029549390902948}
}

@ARTICLE{Binder1996,
  author = {J. L. Binder and B. W. Spencer},
  title = {Investigations into the physical phenomena and mechanisms that effect
	direct containment heating loads},
  journal = {Nuclear Engineering and Design},
  year = {1996},
  volume = {164},
  pages = {175 - 199},
  number = {1-3},
  abstract = {Integral direct containment heating (DCH) experiment results are presented.
	The results are analyzed and discussed for the insights they have
	given into understanding the important physical phenomena and mechanisms
	that effect DCH loads to the containment. Particular attention is
	paid to (1) debris dispersal from the cavity and containment structure
	trapping, (2) hydrogen production and combustion, (3) the importance
	of difference in corium simulants used in integral DCH experiments
	and (4) corium debris quenching by flooded cavities. It is found
	that much has been learned about DCH phenomena that can be used for
	modeling and assessing potential containment loads.},
  doi = {DOI: 10.1016/0029-5493(96)01219-8},
  file = {Binder1996.pdf:Binder1996.pdf:PDF},
  issn = {0029-5493},
  keywords = {dhc, unread},
  owner = {lewis.lloyd},
  timestamp = {2009.04.13},
  url = {http://www.sciencedirect.com/science/article/B6V4D-3VTJ9YC-8/2/7048422d4cb823d5ba11cacd20f8a644}
}

@ARTICLE{Blanchat1996,
  author = {Thomas K. Blanchat and Michael D. Allen},
  title = {Experiments to investigate DCH phenomena with large-scale models
	of the Zion and Surry nuclear power plants},
  journal = {Nuclear Engineering and Design},
  year = {1996},
  volume = {164},
  pages = {147 - 174},
  number = {1-3},
  abstract = {The Surtsey Test Facility and the Containment Technology Test Facility
	(CTTF) at Sandia National Laboratories (SNL) have been used to perform
	scaled experiments for the Nuclear Regulatory Commission (NRC) that
	simulate high-pressure melt ejection (HPME) accidents in a nuclear
	power plant (NPP). These experiments are designed to investigate
	the effects of direct containment heating (DCH) phenomena on the
	containment load. High-temperature, chemically reactive alumino (thermitic)
	melt is ejected by high-pressure steam into a scale model of either
	the Zion or Surry NPP. Integral effects tests under prototypic conditions
	have been performed to investigate the effects of dispersal of molten
	core materials on DCH loads, and to study the effects of Westinghouse
	plant configurations on DCH loads. In Westinghouse plants, there
	is (1) an intermediate compartment that is large compared with the
	reactor cavity but small compared with the main containment volume,
	and (2) no significant line-of-sight pathway for debris transport
	from the cavity to the main containment volume. Containment compartmentalization
	is the dominant mitigating feature for Zion, Surry, and most other
	pressurized water reactors. Experimental results will be used to
	further assess the applicability of existing DCH models to Westinghouse
	plants on DCH loads.},
  doi = {DOI: 10.1016/0029-5493(96)01218-6},
  file = {Blanchat1996.pdf:Blanchat1996.pdf:PDF},
  issn = {0029-5493},
  keywords = {dch, unread},
  owner = {lewis.lloyd},
  timestamp = {2009.04.13},
  url = {http://www.sciencedirect.com/science/article/B6V4D-3VTJ9YC-7/2/ef70efd03a29ad6d3421daae74935764}
}

@TECHREPORT{Blanchat1997,
  author = {Blanchat, Thomas K. and Pilch, M. M. and Allen, M. D.},
  title = {Experiments to investigate direct containment heating phenomena with
	scaled models of Calvert Cliffs Nuclear Power Plant.},
  institution = {Nuclear Regulatory Commission, Washington, DC (United States). Div.
	of Systems Technology; Sandia National Labs., Albuquerque, NM (United
	States)},
  year = {1997},
  number = {NUREG/CR--6469, SAND--96-2289, ON: TI97004139, TRN: 97:002148},
  doi = {DOI: 10.2172/453735},
  file = {Blanchat1997.pdf:Blanchat1997.pdf:PDF},
  keywords = {dch, unread},
  owner = {lewis.lloyd},
  timestamp = {2009.04.13},
  url = {http://www.osti.gov/bridge/servlets/purl/453735-xYZC6a/webviewable/453735.pdf}
}

@ARTICLE{Brinkmann2006,
  author = {G. Brinkmann and J. Pirson and S. Ehster and M.T. Dominguez and L.
	Mansani and I. Coe and R. Moormann and W. Van der Mheen},
  title = {Important viewpoints proposed for a safety approach of HTGR reactors
	in Europe: Final results of the EC-funded HTR-L project},
  journal = {Nuclear Engineering and Design},
  year = {2006},
  volume = {236},
  pages = {463 - 474},
  number = {5-6},
  note = {HTR-2004},
  abstract = {The inherent safety features of modular High Temperature Reactors
	(HTRs) make events leading to severe core damage highly unlikely
	and constitute the main differentiating aspects compared to LWRs.
	Furthermore, while a known and stable regulatory environment has
	long been established for Light Water Reactors (LWRs), different
	ways of thinking may help to develop a more appropriate licensing
	process for HTR-based power plants. The HTR-L project funded by the
	European Commission in the 5th Framework Programme was dedicated
	to the definition of a common and coherent European safety approach
	and the identification of the main licensing issues for the licensing
	framework of the modular HTRs. Several topics were developed during
	the course of this project. Due to the characteristics of the HTR
	design, it has been necessary to define specific defence-in-depth
	requirements which have then been evaluated for implementation in
	the safety approach. Safety-related functions appropriate for the
	HTR design have also had to be identified and listed. On one hand,
	the different possible solicitations of the fuel particles constituted
	the starting point for the identification of the accidental conditions
	(by means of the Master Logic Diagrams methodology); these accidental
	conditions were classified and the most appropriate methods to consider
	ultra low probability severe accidents were examined. On the other
	hand, the elements constituting the source term and, in particular,
	the requirements for the confinement of radioactive products and
	the conditions required to prevent the need for a #conventional#
	containment structure have been discussed. In the definition of the
	safety approach, attention has been paid to the need to maintain
	the potentially interesting economic perspectives of HTR reactors.
	Key issues to be addressed in the licensing process of the HTRs have
	also been identified. An innovative systems, structures and components
	classification method has been developed and rules that will govern
	equipment qualification proposed.},
  doi = {DOI: 10.1016/j.nucengdes.2005.11.017},
  file = {Brinkmann2006.pdf:Brinkmann2006.pdf:PDF},
  issn = {0029-5493},
  keywords = {htgr, unread},
  owner = {lewis.lloyd},
  timestamp = {2009.04.14},
  url = {http://www.sciencedirect.com/science/article/B6V4D-4J557GJ-4/2/012fd7a3d8446846c5a16c3ff2e4d84c}
}

@PHDTHESIS{Buschman2008,
  author = {Buschman, F. X.},
  title = {The Development of a Liquid Jet Model for Implementation in a 3-Dimensional
	Eualarian Analysis Tool},
  school = {The Pennsylvania State University},
  year = {2008},
  file = {Buschman2008.pdf:Buschman2008.pdf:PDF},
  owner = {llloyd},
  timestamp = {2012.11.03}
}

@INCOLLECTION{Cai2009,
  author = {X. C. Cai},
  title = {Nonlinear Overlapping Domain Decomposition Methods},
  booktitle = {Domain Decomposition Methods in Science and Engineering XVIII},
  publisher = {Springer Heidelberg},
  year = {2009},
  editor = {M. Bercovier and M. J. Gander and R. Kornhuber and O. Widlund},
  volume = {70},
  series = {Lecture Notes In Computational Science And Engineering},
  pages = {217-224},
  doi = {10.1007/978-3-642-02677-5},
  file = {Cai2009.pdf:Cai2009.pdf:PDF},
  owner = {llloyd},
  timestamp = {2012.11.12},
  url = {http://www.springerlink.com/content/qwv353/}
}

@ARTICLE{Cai2002,
  author = {X. C. Cai and D. Keyes},
  title = {Nonlinearly Preconditioned Inexact Newton Algorithms},
  journal = {SIAM Journal on Scientific Computing},
  year = {2002},
  volume = {24},
  pages = {183-200},
  number = {1},
  doi = {10.1137/S106482750037620X},
  eprint = {http://epubs.siam.org/doi/pdf/10.1137/S106482750037620X},
  file = {Cai2002.pdf:Cai2002.pdf:PDF},
  owner = {llloyd},
  timestamp = {2012.11.03},
  url = {http://epubs.siam.org/doi/abs/10.1137/S106482750037620X}
}

@ARTICLE{Cai2011,
  author = {X. C. Cai and X. Li},
  title = {Inexact Newton Methods with Restricted Additive Schwarz Based Nonlinear
	Elimination for Problems with High Local Nonlinearity},
  journal = {SIAM Journal on Scientific Computing},
  year = {2011},
  volume = {33},
  pages = {746-762},
  number = {2},
  doi = {10.1137/080736272},
  eprint = {http://epubs.siam.org/doi/pdf/10.1137/080736272},
  file = {Cai2011.pdf:Cai2011.pdf:PDF},
  owner = {llloyd},
  timestamp = {2012.11.03},
  url = {http://epubs.siam.org/doi/abs/10.1137/080736272}
}

@ARTICLE{Celnik2008,
  author = {Matthew Celnik and Abhijeet Raj and Richard West and Robert Patterson
	and Markus Kraft},
  title = {Aromatic site description of soot particles},
  journal = {Combustion and Flame},
  year = {2008},
  volume = {155},
  pages = {161 - 180},
  number = {1-2},
  abstract = {A new, advanced soot particle model is developed that describes soot
	particles by their aromatic structure, including functional site
	descriptions and a detailed surface chemistry mechanism. A methodology
	is presented for the description of polyaromatic hydrocarbon (PAH)
	structures by their functional sites. The model is based on statistics
	that describe aromatic structural information in the form of easily
	computed correlations, which were generated using a kinetic Monte
	Carlo algorithm to study the growth of single PAH molecules. A comprehensive
	surface reaction mechanism is presented to describe the growth and
	desorption of aromatic rings on PAHs. The model is capable of simulating
	whole particle ensembles which allows bulk properties such as soot
	volume fraction and number density to be found, as well as joint
	particle size and surface area distributions. The model is compared
	to the literature-standard soot model [J. Appel, H. Bockhorn, M.
	Frenklach, Combust. Flame 121 (2000) 122-136] in a plug-flow reactor
	and is shown to predict well the experimental results of soot mass,
	average particle size, and particle size distributions at different
	flow times. Finally, the carbon/hydrogen ratio and the distribution
	of average PAH sizes in the ensemble, as predicted by the model,
	are discussed.},
  doi = {DOI: 10.1016/j.combustflame.2008.04.011},
  file = {Celnik2008.pdf:Celnik2008.pdf:PDF},
  issn = {0010-2180},
  keywords = {particulate, unread},
  owner = {lewis.lloyd},
  timestamp = {2009.04.13},
  url = {http://www.sciencedirect.com/science/article/B6V2B-4SMGCC7-1/2/f6196aef3142f08dabb7d5ef6175fdff}
}

@ARTICLE{Chan1984,
  author = {T. Chan and K. Jackson},
  title = {Nonlinearly Preconditioned Krylov Subspace Methods for Discrete Newton
	Algorithms},
  journal = {SIAM Journal on Scientific and Statistical Computing},
  year = {1984},
  volume = {5},
  pages = {533-542},
  number = {3},
  doi = {10.1137/0905039},
  eprint = {http://epubs.siam.org/doi/pdf/10.1137/0905039},
  file = {Chan1984.pdf:Chan1984.pdf:PDF},
  owner = {llloyd},
  timestamp = {2012.11.03},
  url = {http://epubs.siam.org/doi/abs/10.1137/0905039}
}

@TECHREPORT{Christensen1961,
  author = {Christensen, H.},
  title = {Power-To-Void Transfer Functions},
  institution = {Argonne National Laboratory},
  year = {1961},
  number = {ANL-6385},
  address = {Argonne National Laboratory
	
	9700 South Case Avenue},
  month = {Argonne, Illinois},
  file = {Christensen1961.pdf:Christensen1961.pdf:PDF},
  keywords = {Christensen, Validation, Two-Phase Flow},
  owner = {lloydlj},
  timestamp = {2011.08.03}
}

@CONFERENCE{Chunhe1993,
  author = {Tang Chunhe and Guan Jie and Li Ende and Zhang Chun},
  title = {Study of reaction: Coated SiC coating on HTGR fuel elements},
  booktitle = {Technical committee meeting on response of fuel, fuel elements and
	gas cooled reactor cores under accidental air or water ingress conditions.},
  year = {1993},
  series = {IAEA-TECDOC--784},
  pages = {107-109},
  month = {October},
  organization = {International Atomic Energy Agency, Vienna (Austria)},
  abstract = {The graphite has excellent nuclear and high temperature performance,
	and is used in atomic reactors with enormous quantities. The oxidation
	resistance of graphite is very poor, therefore, the application of
	graphite is only in restricted environment. SiC presents excellent
	ability of the oxidation resistance. It is reported in this paper
	that the uniform and dense SiC coating is formed by the reaction
	of the graphite matrix and melted silicon. Oxidation experiments
	show that, when samples were heated to 1000 deg. C in air for 8 hours,
	the graphite was burnt off 68 wt. %, whereas the coated graphite
	only 1.7 wt.%, and 1.0wt% at. 1500 deg. C in air for 2 hours.},
  file = {Chunhe1993.pdf:Chunhe1993.pdf:PDF},
  keywords = {htgr, unread},
  owner = {lewis.j.lloyd},
  timestamp = {2009.06.17}
}

@ARTICLE{Cioni2006,
  author = {O. Cioni and M. Marchand and G. Geffraye and F. Ducros},
  title = {3D thermal-hydraulic calculations of a modular block-type HTR core},
  journal = {Nuclear Engineering and Design},
  year = {2006},
  volume = {236},
  pages = {565 - 573},
  number = {5-6},
  note = {HTR-2004},
  abstract = {This paper deals with 3D numerical simulation of assemblies of a high
	temperature helium cooled reactor core. The present aim is to investigate
	an emergency situation due to the blocking of few Helium channels
	in the core. In this situation, some Helium channels for coolant
	passage are blocked and the related issue is the temperature distribution
	within the assemblies, as well as the locations of the local extrema
	in graphite and fuel media, that must not exceed 1600 °C. Two configurations
	are investigated, depending on the situation of the blocked fuel
	assembly in the core. The blocked assembly is surrounded by six unblocked
	fuel assemblies in the first situation and by five unblocked fuel
	assemblies and one reflector assembly in the second situation. This
	numerical study is performed using the thermal-hydraulic CFD code
	Trio_U. Trio_U is developed at CEA Grenoble and especially designed
	to simulate unsteady turbulent flows developing in complex geometries.
	For the present work, an accurate description of 3D conduction problem
	(graphite and fuel), coupled with a simplified 1D thermal-hydraulic
	modelling for Helium is used. The reported simulations suggest that
	the main temperature increase is limited to the blocked fuel assembly
	and do not spread over surrounded assemblies. This seems to be due
	to the Helium flow rate in the gaps between blocks. Moreover, the
	present results show that the maximum temperature criterion is not
	respected for the blocking conditions at nominal power for the blocked
	assembly.},
  doi = {DOI: 10.1016/j.nucengdes.2005.10.024},
  file = {Cioni2006.pdf:Cioni2006.pdf:PDF},
  issn = {0029-5493},
  keywords = {htgr, unread},
  owner = {lewis.lloyd},
  timestamp = {2009.04.13},
  url = {http://www.sciencedirect.com/science/article/B6V4D-4J557GJ-3/2/76cc4f6ecbcc7bb29da622517254a2a3}
}

@ARTICLE{Dennis2006,
  author = {J.S. Dennis and A.N. Hayhurst and S.A. Scott},
  title = {The combustion of large particles of char in bubbling fluidized beds:
	The dependence of Sherwood number and the rate of burning on particle
	diameter},
  journal = {Combustion and Flame},
  year = {2006},
  volume = {147},
  pages = {185 - 194},
  number = {3},
  abstract = {Particles of char derived from a variety of fuels (e.g., biomass,
	sewage sludge, coal, or graphite), with diameters in excess of ,
	burn in fluidized bed combustors containing smaller particles of,
	e.g., sand, such that the rate is controlled by the diffusion both
	of O2 to the burning solid and of the products CO and CO2 away from
	it into the particulate phase. It is therefore important to characterize
	these mass transfer processes accurately. Measurements of the burning
	rate of char particles made from sewage sludge suggest that the Sherwood
	number, Sh, increases linearly with the diameter of the fuel particle,
	dchar (for ). This linear dependence of Sh on dchar is expected from
	the basic equation Sh=2[epsilon]mf(1+dchar/2[delta]diff)/[tau], provided
	the thickness of the boundary layer for mass transfer, [delta]diff,
	is constant in the region of interest (). Such a dependence is not
	seen in the empirical equations currently used and based on the Frössling
	expression. It is found here that for chars made from sewage sludge
	(for ), the thickness of the boundary layer for mass transfer in
	a fluidized bed, [delta]diff, is less than that predicted by empirical
	correlations based on the Frössling expression. In fact, [delta]diff
	is not more than the diameter of the fluidized sand particles. Finally,
	the experiments in this study indicate that models based on surface
	renewal theory should be rejected for a fluidized bed, because they
	give unrealistically short contact times for packets of fluidized
	particles at the surface of a burning sphere. The result is the new
	correlation for the dependence of Sh on dchar, the diameter of a
	burning char particle. This equation is based on there being a gas-cushion
	of fluidizing gas underneath a burning char particle; the implication
	of this correlation is that a completely new picture emerges for
	the combustion of a char particle in a hot fluidized bed.},
  doi = {DOI: 10.1016/j.combustflame.2006.08.007},
  file = {Dennis2006.pdf:Dennis2006.pdf:PDF},
  issn = {0010-2180},
  keywords = {particulate, unread},
  owner = {lewis.lloyd},
  timestamp = {2009.04.13},
  url = {http://www.sciencedirect.com/science/article/B6V2B-4M27X5V-2/2/04d858efbc11853f9f1e95c92a247204}
}

@BOOK{Dennis1996,
  title = {Numerical Methods for Unconstrained Optimization and Nonlinear Equations},
  publisher = {Society for Industrial and Applied Mathematics},
  year = {1996},
  author = {J. E. Dennis and R. B. Schnabel},
  series = {Classics in Applied Mathematics},
  doi = {10.1137/1.9781611971200},
  owner = {llloyd},
  timestamp = {2012.11.14},
  url = {http://epubs.siam.org/doi/book/10.1137/1.9781611971200}
}

@ARTICLE{Densmore2007,
  author = {Jeffery D. Densmore and Todd J. Urbatsch and Thomas M. Evans and
	Michael W. Buksas},
  title = {A hybrid transport-diffusion method for Monte Carlo radiative-transfer
	simulations},
  journal = {Journal of Computational Physics},
  year = {2007},
  volume = {222},
  pages = {485 - 503},
  number = {2},
  abstract = {Discrete Diffusion Monte Carlo (DDMC) is a technique for increasing
	the efficiency of Monte Carlo particle-transport simulations in diffusive
	media. If standard Monte Carlo is used in such media, particle histories
	will consist of many small steps, resulting in a computationally
	expensive calculation. In DDMC, particles take discrete steps between
	spatial cells according to a discretized diffusion equation. Each
	discrete step replaces many small Monte Carlo steps, thus increasing
	the efficiency of the simulation. In addition, given that DDMC is
	based on a diffusion equation, it should produce accurate solutions
	if used judiciously. In practice, DDMC is combined with standard
	Monte Carlo to form a hybrid transport-diffusion method that can
	accurately simulate problems with both diffusive and non-diffusive
	regions. In this paper, we extend previously developed DDMC techniques
	in several ways that improve the accuracy and utility of DDMC for
	nonlinear, time-dependent, radiative-transfer calculations. The use
	of DDMC in these types of problems is advantageous since, due to
	the underlying linearizations, optically thick regions appear to
	be diffusive. First, we employ a diffusion equation that is discretized
	in space but is continuous in time. Not only is this methodology
	theoretically more accurate than temporally discretized DDMC techniques,
	but it also has the benefit that a particle's time is always known.
	Thus, there is no ambiguity regarding what time to assign a particle
	that leaves an optically thick region (where DDMC is used) and begins
	transporting by standard Monte Carlo in an optically thin region.
	Also, we treat the interface between optically thick and optically
	thin regions with an improved method, based on the asymptotic diffusion-limit
	boundary condition, that can produce accurate results regardless
	of the angular distribution of the incident Monte Carlo particles.
	Finally, we develop a technique for estimating radiation momentum
	deposition during the DDMC simulation, a quantity that is required
	to calculate correct fluid motion in coupled radiation-hydrodynamics
	problems. With a set of numerical examples, we demonstrate that our
	improved DDMC method is accurate and can provide efficiency gains
	of several orders of magnitude over standard Monte Carlo.},
  doi = {DOI: 10.1016/j.jcp.2006.07.031},
  file = {Densmore2007.pdf:Densmore2007.pdf:PDF},
  issn = {0021-9991},
  keywords = {radiative.transport, unread},
  owner = {llloyd},
  timestamp = {2012.10.20},
  url = {http://www.sciencedirect.com/science/article/B6WHY-4KVXPX7-1/2/626de28c555c6ec8dc3fcae92403c0ca}
}

@BOOK{Deuflhard2004,
  title = {Newton Methods for Nonlinear Problems: Affine Invariance and Adaptive
	Algorithms},
  publisher = {Springer-Verlag Berlin Heidelberg},
  year = {2006},
  editor = {R. Bank and R. L. Graham and J. Stoer and R. Varga and H. Yserentant},
  author = {P. Deuflhard},
  volume = {35},
  series = {Springer Series in Computational Mathematics},
  edition = {2nd},
  owner = {llloyd},
  timestamp = {2012.11.09}
}

@ARTICLE{Dhir2006,
  author = {Vijay K. Dhir},
  title = {Mechanistic Prediction of Nucleate Boiling Heat Transfer--Achievable
	or a Hopeless Task?},
  journal = {Journal of Heat Transfer},
  year = {2006},
  volume = {128},
  pages = {1-12},
  number = {1},
  doi = {10.1115/1.2136366},
  file = {Dhir2006.pdf:Dhir2006.pdf:PDF},
  keywords = {heat transfer; boiling; two-phase flow; gravity waves; bubbles; flow
	simulation; critical.heat.flux; unread},
  owner = {lewis.j.lloyd},
  publisher = {ASME},
  timestamp = {2009.10.06},
  url = {http://link.aip.org/link/?JHR/128/1/1}
}

@ARTICLE{Downar2001,
  author = {T. J. Downar and H. G. Joo},
  title = {A preconditioned Krylov method for solution of the multi-dimensional,
	two fluid hydrodynamics equations},
  journal = {Annals of Nuclear Energy},
  year = {2001},
  volume = {28},
  pages = {1251 - 1267},
  number = {12},
  abstract = {A preconditioned Krylov method is introduced for reducing the computational
	burden in the solution of the multi-dimensional, two fluid hydrodynamics
	equations. The BILU3D preconditioned BICGSTAB method was applied
	to the linearized continuity equation in the inner iteration of the
	fully implicit solution of the mass, energy, and momentum equations
	in the EPRI code VIPRE-02. A nuclear reactor thermal-hydraulics test
	problem was performed with 104 multi-dimensional flow channels in
	the vessel. For the simulation of a typical pressurized water Reactor
	steam line break transient, the overall execution time was reduced
	by more than 50% compared to the existing solution techniques which
	utilize stationary alternate direction implicit (ADI) iterative methods
	for the inner iteration.},
  doi = {10.1016/S0306-4549(00)00124-9},
  file = {Downar2001.pdf:Downar2001.pdf:PDF},
  issn = {0306-4549},
  owner = {llloyd},
  timestamp = {2012.10.22},
  url = {http://www.sciencedirect.com/science/article/pii/S0306454900001249}
}

@BOOK{Drew1998,
  title = {Theory of Multicomponent Fluids},
  publisher = {Springer-Verlag New York, Inc.
	
	175 Fifth Avenue
	
	New York, NY 10010
	
	USA},
  year = {1998},
  editor = {J.E. Marsden and L. Sirovich},
  author = {D. A. Drew and S. L. Passman},
  volume = {135},
  series = {Applied Mathematical Sciences},
  edition = {1st},
  owner = {llloyd},
  timestamp = {2012.10.29}
}

@ARTICLE{Dryja1997,
  author = {M. Dryja and W. Hackbusch},
  title = {On the Nonlinear Domain Decomposition Method},
  journal = {BIT},
  year = {1997},
  volume = {37},
  pages = {296-311},
  number = {2},
  file = {Dryja1997.pdf:Dryja1997.pdf:PDF},
  owner = {llloyd},
  timestamp = {2012.11.07}
}

@ARTICLE{Emonot2011,
  author = {P. Emonot and A. Souyri and J. L. Gandrille and F. Barre},
  title = {{CATHARE-3}: A new system code for thermal-hydraulics in the context
	of the NEPTUNE project},
  journal = {Nuclear Engineering and Design},
  year = {2011},
  volume = {241},
  pages = {4476 - 4481},
  number = {11},
  note = {<ce:title>13th International Topical Meeting on Nuclear Reactor Thermal
	Hydraulics (NURETH-13)</ce:title>},
  abstract = {After a thorough analysis of the industrial needs and of the limitations
	of current simulation tools, EDF and CEA (Commissariat à l’Energie
	Atomique) launched the NEPTUNE Project in 2001 (see Guelfi et al.,
	2007) with the support of AREVA-NP and IRSN. The NEPTUNE activities
	include software development, research in physical modeling and numerical
	methods, development of advanced instrumentation techniques and new
	experimental programs. Four different simulation scales were addressed
	including DNS (Direct Numerical Simulation), CFD in open medium (Computational
	Fluid Dynamics), component (subchannel-type analysis) and system
	(reactor modeling) scales.
	
	In 2006 CEA, EDF, AREVA-NP and IRSN defined the strategy for the system
	scale of NEPTUNE and the CATHARE-3 development was launched. The
	main objectives are:• advanced physical modeling of two-phases
	flows, mainly by using multi-field and turbulence models, • improved
	3D modeling by the use of fine and non conforming structured meshes,
	• generalized coupling capabilities with other thermal-hydraulic
	scales and with other disciplines (core physics, structural mechanics,
	…), • extension of the applicability to new Gen IV reactors (Sodium
	Cooled Fast Breeder Reactors, Gas Cooled Reactors, Supercritical
	Light Water Reactors), • a true object-oriented code architecture.
	
	
	At the same time CATHARE-3 is in continuity with the CATHARE-2 code
	which is the current industrial version of CATHARE and internationally
	used for nuclear power plant safety analysis, in simulators and in
	coupled simulation tools. The road map of these two codes will allow
	a smooth transition from CATHARE-2 to CATHARE-3 for all users.
	
	This paper gives an overview of the choices made for the development
	of CATHARE-3 including new physical models, validation strategy and
	experimental programs, numerical improvements, enhanced coupling
	capability and software architecture evolution. The current status
	of the project as well as the overall schedule will be presented.},
  doi = {10.1016/j.nucengdes.2011.04.049},
  file = {Emonot2011.pdf:Emonot2011.pdf:PDF},
  issn = {0029-5493},
  owner = {llloyd},
  timestamp = {2012.11.14},
  url = {http://www.sciencedirect.com/science/article/pii/S0029549311004018}
}

@PHDTHESIS{Fatenejad2010,
  author = {Milad Fatenejad},
  title = {Improved Computational Methods for Simulating Intertial Confinement
	Fusion},
  school = {University of Wisconsin - Madison},
  year = {2010},
  file = {Fatenejad2010.pdf:Fatenejad2010.pdf:PDF},
  owner = {llloyd},
  timestamp = {2012.11.03}
}

@TECHREPORT{Findlay1981,
  author = {Findlay, J. A. and Sozzi, G. L.},
  title = {BWR Refill-Reflood Program - Model Qualification Task Plan: Description
	of Void Fraction Distribution and Level Swell During Vessel Vlowdown
	Transients},
  institution = {General Electric Company},
  year = {1981},
  type = {Interim Report},
  number = {NP-1527
	
	Research Project 1377-1
	
	NUREG/CR-1899
	
	GEAP-24898},
  address = {Genereal Electric Company
	
	175 Curtner Avenue
	
	San Jose, California 95125},
  month = {October},
  note = {Prepared for:
	
	US NRC},
  file = {Findlay1981.PDF:Findlay1981.PDF:PDF},
  keywords = {General Electric, Level Swell, Validation},
  owner = {lloydlj},
  timestamp = {2011.08.04}
}

@ARTICLE{Fleck1971,
  author = {J.A. Fleck and Jr. and J.D. Cummings and Jr.},
  title = {An implicit Monte Carlo scheme for calculating time and frequency
	dependent nonlinear radiation transport},
  journal = {Journal of Computational Physics},
  year = {1971},
  volume = {8},
  pages = {313 - 342},
  number = {3},
  abstract = {A flexible and accurate method for solving nonlinear, frequency-dependent
	radiative transfer problems by a Monte Carlo technique is developed.
	The method is based upon the concept of effective scattering, wherein
	a fraction of the radiative energy absorbed is instantaneously and
	isotropically reradiated in a manner analogous to a scattering process.
	The method appears to be unconditionally stable, conserves energy
	exactly, and is suitable for handling either transparent or optically
	thick media.},
  doi = {DOI: 10.1016/0021-9991(71)90015-5},
  file = {Fleck1971.pdf:Fleck1971.pdf:PDF},
  issn = {0021-9991},
  keywords = {radiative.transport; unread},
  owner = {llloyd},
  timestamp = {2012.10.20},
  url = {http://www.sciencedirect.com/science/article/B6WHY-4DD1VFD-11B/2/77ba2487ceb9f075824f64dff2884281}
}

@ARTICLE{Frepoli2003,
  author = {C. Frepoli and J. H. Mahaffy and K. Ohkawa},
  title = {Notes on the implementation of a fully-implicit numerical scheme
	for a two-phase three-field flow model},
  journal = {Nuclear Engineering and Design},
  year = {2003},
  volume = {225},
  pages = {191 - 217},
  number = {2–3},
  abstract = {The development of a fully-implicit scheme to model the two-phase
	three-field flow and heat transfer problem is presented here. The
	model was originally developed to simulate the complex phenomena
	occurring in proximity of the quench front of a nuclear reactor core
	during the reflood phase of a postulated LOCA. The fully-implicit
	method allows relative large time steps to be used even on very fine
	spatial grids which can not be considered when a semi-implicit scheme
	is applied to solve the conservation equation. The objective of this
	paper is to capture as much as possible, the lessons learned during
	the development and coding of the fully-implicit two-phase three-field
	model. The implementation of the model is one of the most time consuming
	and a challenging task. The literature on numerical models generally
	concentrates on the theoretical aspects of the numerical method but
	available information on the problems encountered during the implementation
	of such methods for real applications is scarce. The reason is that
	many of these methods are tailored to specific applications and sometimes
	are rather empirical. These techniques are the result of a long and
	tedious trial and error process from the developer. The article presented
	here attempts to provide some insights and guidelines for future
	development of this or similar models.},
  doi = {10.1016/S0029-5493(03)00159-6},
  file = {Frepoli2003.pdf:Frepoli2003.pdf:PDF},
  issn = {0029-5493},
  owner = {llloyd},
  timestamp = {2012.10.20},
  url = {http://www.sciencedirect.com/science/article/pii/S0029549303001596}
}

@CONFERENCE{Froehling1993,
  author = {W. Froehling and H. Hohn and M. Kugeler and Y. Sun and Z. Zhang},
  title = {The NACOK experiment on natural convection of air through the core},
  booktitle = {Technical committee meeting on response of fuel, fuel elements and
	gas cooled reactor cores under accidental air or water ingress conditions.},
  year = {1993},
  series = {IAEA-TECDOC--784},
  pages = {55 - 61},
  month = {October},
  organization = {International Atomic Energy Agency, Vienna (Austria)},
  abstract = {KFA is setting up an experiment to study the ingress of air into the
	core, with the subsequent formation of an airflow driven by natural
	convection, for the hypothetical accident of a complete rupture of
	the coaxial hot gas duct. The experiment will have an active height
	for the natural convection of 7,50 m and a maximum temperature of
	1,200 deg. C. The aim aims are to determine the onset time of the
	convection depending on various parameters, the mass flow and the
	locally dependent corrosion phenomena caused by the draft. By this
	means the theoretical models can be validated. The layout of the
	experiment as well as theoretical investigations to predict the expected
	performance are reported.},
  file = {Froehling1993.pdf:Froehling1993.pdf:PDF},
  keywords = {Air Ingress, Dust, htgr, TRISO, unread},
  owner = {lewis.j.lloyd},
  timestamp = {2009.06.17}
}

@ARTICLE{Gabler2006,
  author = {Dorothea Gabler and Jürgen Henniger and Uwe Reichelt},
  title = {AMOS - An effective tool for adjoint Monte Carlo photon transport},
  journal = {Nuclear Instruments and Methods in Physics Research Section B: Beam
	Interactions with Materials and Atoms},
  year = {2006},
  volume = {251},
  pages = {326 - 332},
  number = {2},
  abstract = {In order to expand the photon version of the Monte Carlo transport
	program AMOS to an adjoint photon version, AMOS Pt, the theory of
	adjoint radiation transport is reviewed and evaluated in this regard.
	All relevant photon interactions, photoelectric effect, coherent
	scattering, incoherent scattering and pair production, are taken
	into account as proposed in the EPDL 97. In order to simulate pair
	production and to realise physical source terms with discrete energy
	levels, an energy point detector is used. To demonstrate the qualification
	of AMOS Pt a simple air-over-ground problem is simulated by both
	the forward and the adjoint programs. The results are compared and
	total agreement is shown.},
  doi = {DOI: 10.1016/j.nimb.2006.07.005},
  file = {Gabler2006.pdf:Gabler2006.pdf:PDF},
  issn = {0168-583X},
  keywords = {radiative.transport, unread},
  owner = {lewis.j.lloyd},
  timestamp = {2009.06.18},
  url = {http://www.sciencedirect.com/science/article/B6TJN-4KTVNT1-1/2/c58dcf29cc66fd47e626f2e66fddaa4f}
}

@MANUAL{Geist1994,
  title = {{PVM}: Parallel Virtual Machine
	
	A User's Guide and Tutorial for Networked Parallel Computing},
  author = {A. Geist and A. Beguelin and J. Dongarra and W. Jiang and R. Manchek
	and V. Sunderam},
  organization = {Massachusetts Institute of Technology},
  address = {Cambridge, Massachusetts},
  year = {1994},
  file = {Geist1994.pdf:Geist1994.pdf:PDF},
  owner = {llloyd},
  timestamp = {2012.11.09}
}

@ARTICLE{Gibou2007,
  author = {Frédéric Gibou and Liguo Chen and Duc Nguyen and Sanjoy Banerjee},
  title = {A level set based sharp interface method for the multiphase incompressible
	Navier-Stokes equations with phase change},
  journal = {Journal of Computational Physics},
  year = {2007},
  volume = {222},
  pages = {536 - 555},
  number = {2},
  abstract = {In this paper, we describe a sharp interface capturing method for
	the study of incompressible multiphase flows with phase change. We
	use the level set method to keep track of the interface between the
	two phases and a ghost fluid approach to impose the jump conditions
	at the interface. This work builds on the work of Gibou et al. for
	the study of Stefan problems [F. Gibou, R. Fedkiw, L.-T. Cheng, M.
	Kang, A second-order-accurate symmetric discretization of the Poisson
	equation on irregular domains, J. Comput. Phys. 176 (2002) 205-227]
	and the work of Nguyen et al. for the simulation of incompressible
	flames [D. Nguyen, R. Fedkiw, M. Kang, A boundary condition capturing
	method for incompressible flame discontinuities, J. Comput. Phys.
	172 (2001) 71-98]. We compare our numerical results to exact solutions
	in one spatial dimension and apply this algorithm to the simulation
	of film boiling in two spatial dimensions.},
  doi = {DOI: 10.1016/j.jcp.2006.07.035},
  file = {Gibou2007.pdf:Gibou2007.pdf:PDF},
  issn = {0021-9991},
  keywords = {critical.heat.flux; unread},
  owner = {lewis.j.lloyd},
  timestamp = {2009.10.06},
  url = {http://www.sciencedirect.com/science/article/B6WHY-4M4KR9C-1/2/089e13d1a00c31db65e99d93c1907ce7}
}

@ARTICLE{Golobic2004,
  author = {I. Golobic and E. Pavlovic and J. Von Hardenberg and M. Berry and
	R.A. Nelson and D.B.R. Kenning and L.A. Smith},
  title = {Comparison of a Mechanistic Model for Nucleate Boiling with Experimental
	Spatio-Temporal Data},
  journal = {Chemical Engineering Research and Design},
  year = {2004},
  volume = {82},
  pages = {435 - 444},
  number = {4},
  note = {8th UK National Heat Transfer Conference},
  abstract = {Mechanistic numerical simulations have been developed for pool nucleate
	boiling involving large groups of nucleation sites that are non-uniformly
	distributed spatially and have different activation superheats. The
	simulations model the temperature field in the heated wall accurately
	and use approximations for events in the liquid-vapour space. This
	paper describes the first attempt to compare the numerical simulations
	with spatio-temporal experimental data at a similar level of detail.
	The experimental data were obtained during pool boiling of water
	at atmospheric pressure on a horizontal, electrically heated stainless
	steel plate 0.13 mm thick. They consist of wall temperature fields
	measured on the back of the plate by liquid crystal thermography
	at a sampling rate of 200 Hz over a period of 30 s. Methods of image
	analysis have been developed to deduce the time, position, nucleation
	superheat and size of the cooled area for every bubble nucleation
	event during this period. The paper discusses the methodology of
	using some of the experimental data as input for the simulations
	and the remainder for validation. Because of the high-dimensional
	dynamics and possibly chaotic nature of nucleate boiling, the validation
	must be based on statistical properties over a large area and a long
	period. This preliminary study is restricted to a single heat flux.},
  doi = {DOI: 10.1205/026387604323050146},
  file = {Golobic2004.pdf:Golobic2004.pdf:PDF},
  issn = {0263-8762},
  keywords = {nucleate boiling; critical.heat.flux; unread},
  owner = {lewis.j.lloyd},
  timestamp = {2009.10.06},
  url = {http://www.sciencedirect.com/science/article/B8JGF-4RV2DFX-4/2/00366b8a7a8ce4acd1b9ff57113ad3d0}
}

@ARTICLE{Hammersley1996,
  author = {Robert J. Hammersley and Cruz Cirauqui and Jose Faig and Robert E.
	Henry},
  title = {Direct containment heating experiments for Vandellos and ASCo nuclear
	power plants},
  journal = {Nuclear Engineering and Design},
  year = {1996},
  volume = {164},
  pages = {287 - 296},
  number = {1-3},
  abstract = {The ongoing IPE studies for the Vandellos and ASCo nuclear power plants
	require evaluation of accident phenomena that have been perceived
	to potentially challenge containment integrity including direct containment
	heating (DCH). Analyses and scaled experiments performed to date
	indicated that the lower containment structures play a substantial
	role in mitigating the extent of DCH given a high pressure melt ejection.
	Since the geometry is judged to be of major importance, linearly
	scaled experiments were conceived and conducted to evaluate the role
	of such structures in the Vandellos and ASCo specific configurations.
	The Vandellos test configuration with an initally dry cavity and
	significant exhaust area for the instrument tunnel resulted in the
	dispersal of a majority of the debris from the instrument tunnel
	into the lower compartment. The test of the ASCo configuration with
	an initially wet reactor cavity and limited exhaust area from the
	instrument tunnel exhibited the retention of the majority of the
	debris within the instrument tunnel and reactor cavity. The observed
	pressure responses in these scaled experiments for the seal table
	room, lower containment vessel, and upper containment vessel were
	all less than the containment design basis pressure. These test results
	contribute to the existing technical basis for concluding that direct
	containment heating would not represent a challenge to the integrity
	of these containments.},
  doi = {DOI: 10.1016/0029-5493(96)01225-3},
  file = {Hammersley1996.pdf:Hammersley1996.pdf:PDF},
  issn = {0029-5493},
  keywords = {dch, unread},
  owner = {lewis.lloyd},
  timestamp = {2009.04.13},
  url = {http://www.sciencedirect.com/science/article/B6V4D-3VTJ9YC-G/2/b1744cb9950f64aab3646eeed42febe8}
}

@ARTICLE{Haque2008,
  author = {H. Haque},
  title = {Consequences of delayed air ingress following a depressurization
	accident in a high temperature reactor},
  journal = {Nuclear Engineering and Design},
  year = {2008},
  volume = {238},
  pages = {3041 - 3046},
  number = {11},
  note = {HTR-2006: 3rd International Topical Meeting on High Temperature Reactor
	Technology},
  abstract = {Air ingress is a specific event in a high temperature reactor (HTR).
	The potential threat posed by air ingress lies in the chemical reaction
	of oxygen with hot graphite at a temperature above 500 °C leading
	to reaction heat and graphite corrosion. In order to assess the consequence
	of air ingress into the reactor, it is postulated that breaks are
	present above and below the reactor core and that unobstructed ingress
	of air through them is possible. It is obvious that the air ingress
	incident has to be preceded by a depressurization accident. For this
	hypothetical scenario the maximum possible air flow rate through
	the core resulting solely from the pressure losses in the core is
	estimated as a function of the break cross-sections exposed above
	and below the core. In this paper, the thermal behavior of an HTR
	with prismatic fuel (operating inlet/outlet temperatures 450/850 °C)
	during air ingress accident conditions has been investigated. In
	particular, maximum temperatures and burn-off of the fuel and bottom
	graphite reflector for various air flow rates for the postulated
	hypothetical scenario have been analyzed. It also indicates the limiting
	time at which the graphite layer of fuel will be completely burnt-off
	and the fuel compacts exposed. In addition, the consequences of delayed
	air ingress after a core heat up following depressurization have
	been investigated. This paper, thus, throws light on the safety aspects
	of the new generation HTRs with prismatic fuels (e.g. NGNP/ANTARES)
	conceived for power generation and process heat application.},
  doi = {DOI: 10.1016/j.nucengdes.2007.12.030},
  file = {Haque2008.pdf:Haque2008.pdf:PDF},
  issn = {0029-5493},
  keywords = {htgr, unread},
  owner = {lewis.lloyd},
  timestamp = {2009.04.13},
  url = {http://www.sciencedirect.com/science/article/B6V4D-4S9FHC2-1/2/19fd0225b0d9309a21a1385c4e70f02c}
}

@ARTICLE{Haque2006,
  author = {H. Haque and W. Feltes and G. Brinkmann},
  title = {Thermal response of a modular high temperature reactor during passive
	cooldown under pressurized and depressurized conditions},
  journal = {Nuclear Engineering and Design},
  year = {2006},
  volume = {236},
  pages = {475 - 484},
  number = {5-6},
  note = {HTR-2004},
  abstract = {The concept of inherent safety features of the modular HTR design
	with respect to passive decay heat removal through conduction, radiation
	and natural convection was first introduced in the German HTR-module
	(pebble fuel) design and subsequently extended to other modular HTR
	design in recent years, e.g. PBMR (pebble fuel), GT-MHR (prismatic
	fuel) and the new generation reactor V/HTR (prismatic fuel). This
	paper presents the numerical simulations of the V/HTR using the thermal-hydraulic
	code THERMIX which was initially developed for the analysis of HTRs
	with pebble fuels, verified by experiments, subsequently adopted
	for applications in the HTRs with prismatic fuels and checked against
	the results of CRP-3 benchmark problem analyzed by various countries
	with diverse codes. In this paper, the thermal response of the V/HTR
	(operating inlet/outlet temperatures 490/1000 °C) during post shutdown
	passive cooling under pressurized and depressurized primary system
	conditions has been investigated. Additional investigations have
	also been carried out to determine the influence of other inlet/outlet
	operating temperatures (e.g. 490/850, 350/850 or 350/1000 °C) on
	the maximum fuel and pressure vessel temperature during depressurized
	cooldown condition. In addition, some sensitivity analyses have also
	been performed to evaluate the effect of varying the parameters,
	i.e. decay heat, graphite conductivity, surface emissivity, etc.,
	on the maximum fuel and pressure vessel temperature. The results
	show that the nominal peak fuel temperatures remain below 1600 °C
	for all these cases, which is the limiting temperature relating to
	radioactivity release from the fuel. The analyses presented in this
	paper demonstrate that the code THERMIX can be successfully applied
	for the thermal calculation of HTRs with prismatic fuel. The results
	also provide some fundamental information for the design optimization
	of V/HTR with respect to its maximum thermal power, operating temperatures,
	etc.},
  doi = {DOI: 10.1016/j.nucengdes.2005.10.027},
  file = {Haque2006.pdf:Haque2006.pdf:PDF},
  issn = {0029-5493},
  keywords = {htgr, DLOFC, unread},
  owner = {lewis.lloyd},
  timestamp = {2009.04.13},
  url = {http://www.sciencedirect.com/science/article/B6V4D-4J557GJ-8/2/059e12c5bdf0b59f2c9e2bccc207679d}
}

@ARTICLE{He2001,
  author = {Ying He and Masahiro Shoji and Shigeo Maruyama},
  title = {Numerical study of high heat flux pool boiling heat transfer},
  journal = {International Journal of Heat and Mass Transfer},
  year = {2001},
  volume = {44},
  pages = {2357 - 2373},
  number = {12},
  doi = {DOI: 10.1016/S0017-9310(00)00269-6},
  file = {He2001.pdf:He2001.pdf:PDF},
  issn = {0017-9310},
  keywords = {critical.heat.flux; unread},
  owner = {lewis.j.lloyd},
  timestamp = {2009.10.06},
  url = {http://www.sciencedirect.com/science/article/B6V3H-42DP021-G/2/6e24086236cfece4dac8486bda600daa}
}

@ARTICLE{Hinssen2008,
  author = {Hans-Klemens Hinssen and Kerstin Kühn and Rainer Moormann and Bärbel
	Schlögl and Mary Fechter and Mark Mitchell},
  title = {Oxidation experiments and theoretical examinations on graphite materials
	relevant for the PBMR},
  journal = {Nuclear Engineering and Design},
  year = {2008},
  volume = {238},
  pages = {3018 - 3025},
  number = {11},
  note = {HTR-2006: 3rd International Topical Meeting on High Temperature Reactor
	Technology},
  abstract = {Graphite oxidation due to gas impurities in normal operation and ingressing
	oxidants in accidents plays a key role in the material and safety
	behaviour of HTRs. An overview is presented of the theoretical background
	concerning graphite oxidation, mainly in regimes I and II. Some differences
	between the classical oxidation model, based on effective diffusivity
	and chemical reaction on the inner graphite surface, are discussed.
	These differences may be due to the complex pore system in graphite,
	which cannot be approximated by one single diffusivity. Based on
	these theoretical results, a procedure for measurements on candidate
	graphites to be used in PBMR is proposed. Regime I measurements are
	selected for material characterization because of the strong sensitivity
	to chemical influences. First results measured in air at 650-750 °C
	at the Graphite Oxidation Laboratory, GOLab, Research Centre Jülich,
	are outlined. Graphites examined so far are the SGL grades NBG-10
	and NBG-18. Whereas NBG-10 is significantly more oxidation resistant
	for all specimens and at all temperatures than the former German
	nuclear graphite V483T5, taken as a standard, the scatter of oxidation
	rates of NBG-18 is even larger, but is on average also satisfactory.
	In contrast to the classical model, preliminary low-temperature oxidation
	experiments on NBG-10 reveal a significant rate dependence on specimen
	size. Additional experiments in regime I and in regime II are proposed
	for PBMR graphites, as those for clarification of the deviations
	to the classical oxidation model. The latter probably requires a
	broader discussion in the graphite community.},
  doi = {DOI: 10.1016/j.nucengdes.2008.02.013},
  file = {Hinssen2008.pdf:Hinssen2008.pdf:PDF},
  issn = {0029-5493},
  keywords = {htgr, unread},
  owner = {lewis.j.lloyd},
  timestamp = {2009.05.26},
  url = {http://www.sciencedirect.com/science/article/B6V4D-4S8TWCS-2/2/03751dea3d44fea506939cfb1cbbb585}
}

@MASTERSTHESIS{Hogan2006,
  author = {Hogan, Kevin J.},
  title = {Pebble bed modular reactor analysis with MELCOR},
  school = {Purdue University},
  year = {2006},
  type = {Master's Thesis},
  address = {Purdue University, West Lafayette, Indiana},
  month = {August},
  file = {Hogan2006.pdf:Hogan2006.pdf:PDF},
  keywords = {htgr, unread},
  owner = {lewis.lloyd},
  timestamp = {2009.04.14}
}

@ARTICLE{Hwang2005,
  author = {F. N. Hwang and X. C. Cai},
  title = {A parallel nonlinear additive Schwarz preconditioned inexact Newton
	algorithm for incompressible Navier–Stokes equations},
  journal = {Journal of Computational Physics},
  year = {2005},
  volume = {204},
  pages = {666 - 691},
  number = {2},
  abstract = {A nonlinear additive Schwarz preconditioned inexact Newton method
	(ASPIN) was introduced recently for solving large sparse highly nonlinear
	systems of equations obtained from the discretization of nonlinear
	partial differential equations. In this paper, we discuss some extensions
	of ASPIN for solving steady-state incompressible Navier–Stokes
	equations with high Reynolds numbers in the velocity–pressure formulation.
	The key idea of ASPIN is to find the solution of the original system
	by solving a nonlinearly preconditioned system that has the same
	solution as the original system, but with more balanced nonlinearities.
	Our parallel nonlinear preconditioner is constructed using a nonlinear
	overlapping additive Schwarz method. To show the robustness and scalability
	of the algorithm, we present some numerical results obtained on a
	parallel computer for two benchmark problems: a driven cavity flow
	problem and a backward-facing step problem with high Reynolds numbers.
	The sparse nonlinear system is obtained by applying a Q1&#xa0;−&#xa0;Q1
	Galerkin least squares finite element discretization on two-dimensional
	unstructured meshes. We compare our approach with an inexact Newton
	method using different choices of forcing terms. Our numerical results
	show that ASPIN has good convergence and is more robust than the
	traditional inexact Newton method with respect to certain parameters
	such as the Reynolds number, the mesh size, and the number of processors.},
  doi = {10.1016/j.jcp.2004.10.025},
  file = {Hwang2005.pdf:Hwang2005.pdf:PDF},
  issn = {0021-9991},
  keywords = {Incompressible Navier–Stokes equations},
  owner = {llloyd},
  timestamp = {2012.11.03},
  url = {http://www.sciencedirect.com/science/article/pii/S0021999104004322}
}

@ARTICLE{Ishii1984,
  author = {M. Ishii and K. Mishima},
  title = {Two-fluid model and hydrodynamic constitutive relations},
  journal = {Nuclear Engineering and Design},
  year = {1984},
  volume = {82},
  pages = {107 - 126},
  number = {2-3},
  abstract = {Two-fluid formulation for two-phase flow analyses is presented. A
	fully three-dimensional model is obtained from the time averaging,
	whereas the one-dimensional model is developed from the area averaging.
	The constitutive equations for the interfacial terms are the weakest
	link in a two-fluid model because of considerable difficulties in
	terms of experimentation and modeling. However, these are of supreme
	importance in determining phase interactions. In view of this, the
	interfacial transfer terms have been studied in great detail both
	for the three- and one-dimensional models. New interfacial area,
	drag, virtual mass, droplet size and entrainment correlations are
	presented. In the one-dimensional model, a number of serious shortcomings
	of the conventional model have been pointed out and new formulations
	to eliminate them are presented. These shortcomings mainly arose
	due to the improper consideration of phase distributions in the transverse
	direction.},
  doi = {10.1016/0029-5493(84)90207-3},
  file = {Ishii1984.pdf:Ishii1984.pdf:PDF},
  issn = {0029-5493},
  owner = {llloyd},
  timestamp = {2012.11.17},
  url = {http://www.sciencedirect.com/science/article/pii/0029549384902073}
}

@CONFERENCE{Iyoku1993,
  author = {T. Iyoku and S. Maruyama and M. Ishihara and S. Shiozawa and K. Ohashi
	and F. Okamoto and H. Hayakawa},
  title = {Evaluating on graphite oxidation during and air ingress accident
	in HTTR},
  booktitle = {Technical committee meeting on response of fuel, fuel elements and
	gas cooled reactor cores under accidental air or water ingress conditions.},
  year = {1993},
  series = {IAEA-TECDOC--784},
  pages = {72-78},
  month = {October},
  organization = {International Atomic Energy Agency, Vienna (Austria)},
  abstract = {The High Temperature Engineering Test Reactor (HTTR) is a graphite-moderated
	and helium gas-cooled reactor with prismatic fuel elements of hexagonal
	blocks. Each fuel element consists of graphite sleeves containing
	fuel compacts and a graphite block holding graphite sleeves. A support
	post is a cylindrical component and the most essential element to
	support the core. There are two kinds of accidents classified as
	an air ingress accident in the safety design of the HTTR. One is
	the rupture of the primary concentric hot gas duct at the reactor
	pressure vessel (RPV) inlet nozzle, the other the rupture of a stand
	pipe attached to the top head closure of the RPV. The graphite structures
	will be oxidized during the air ingress accidents. It was found from
	this study that the air ingress accident of the guillotine rupture
	of the primary concentric hot gas duct was the severest from the
	graphite oxidation point of view and the oxidized graphite structures
	maintained the structural integrity after the accident. There is
	also no possibility of detonation induced by the produced carbon
	monoxide in the containment vessel.},
  file = {Iyoku1993.pdf:Iyoku1993.pdf:PDF},
  keywords = {Air Ingress, Dust, htgr, TRISO, unread, DLOFC},
  owner = {lewis.j.lloyd},
  timestamp = {2009.06.17}
}

@ARTICLE{Jeong1999,
  author = {J.-J. Jeong and K.S. Ha and B.D. Chung and W.J. Lee},
  title = {Development of a multi-dimensional thermal-hydraulic system code,
	MARS 1.3.1},
  journal = {Annals of Nuclear Energy},
  year = {1999},
  volume = {26},
  pages = {1611 - 1642},
  number = {18},
  __markedentry = {[llloyd:]},
  abstract = {A multi-dimensional thermal-hydraulic system code MARS has been developed
	by consolidating and restructuring the RELAP5/MOD3.2.1.2 and COBRA-TF
	codes. The two codes were adopted to take advantage of the very general,
	versatile features of RELAP5 and the realistic three-dimensional
	hydrodynamic module of COBRA-TF. In the course of code development,
	major features of each code were consolidated into a single code
	first. The resulting source programs were rewritten in standard fortran
	90, and then were restructured using modular data structures based
	on “derived type variables� and a new “dynamic
	memory allocation� scheme. In addition, the Windows graphics
	features were implemented for user friendliness. This paper presents
	the developmental activities up to mars version 1.3.1 including the
	code consolidation, the code restructuring and modernization, and
	the results of the developmental assessment.},
  doi = {10.1016/S0306-4549(99)00039-0},
  file = {Jeong1999.pdf:Jeong1999.pdf:PDF},
  issn = {0306-4549},
  owner = {llloyd},
  timestamp = {2012.11.17},
  url = {http://www.sciencedirect.com/science/article/pii/S0306454999000390}
}

@ARTICLE{Jeong2008,
  author = {J. J. Jeong and H. Y. Yoon and I. K. Park and H. K. Cho and J. Kim},
  title = {A semi-implicit numerical scheme for transient two-phase flows on
	unstructured grids},
  journal = {Nuclear Engineering and Design},
  year = {2008},
  volume = {238},
  pages = {3403 - 3412},
  number = {12},
  abstract = {A component-scale two-phase analysis code, CUPID, has been developed
	for a realistic simulation of transient two-phase flows in light
	water nuclear reactor components. In the CUPID code, a two-fluid
	three-field model is adopted and the governing equations are solved
	on an unstructured grid to make CUPID very useful for flow analysis
	in complicated geometries. For the numerical solution scheme, the
	semi-implicit method of the RELAP5 code, which has been proved to
	be very stable and accurate for most practical applications, was
	used with some modifications for an application to an unstructured
	non-staggered grid. This paper presents the modified semi-implicit
	numerical method for an unstructured grid and the preliminary results
	of the calculations.},
  doi = {10.1016/j.nucengdes.2008.08.017},
  file = {Jeong2008.pdf:Jeong2008.pdf:PDF},
  issn = {0029-5493},
  keywords = {semi-implicit},
  owner = {llloyd},
  timestamp = {2012.10.20},
  url = {http://www.sciencedirect.com/science/article/pii/S0029549308004615}
}

@ARTICLE{Kadak2006,
  author = {Andrew C. Kadak and Tieliang Zhai},
  title = {Air ingress benchmarking with computational fluid dynamics analysis},
  journal = {Nuclear Engineering and Design},
  year = {2006},
  volume = {236},
  pages = {587 - 602},
  number = {5-6},
  note = {HTR-2004},
  abstract = {The air ingress accident is a complicated accident scenario that may
	limit the deployment of high-temperature gas reactors. The complexity
	of this accident scenario is compounded by multiple physical phenomena
	that are involved in the air ingress event. These include diffusion,
	natural circulation, and complex chemical reactions with graphite
	and oxygen. In an attempt to better understand the phenomenon, the
	FLUENT-6 computational fluid dynamics code was used to assess two
	air ingress experiments. The first was the Japanese series of tests
	performed in the early 1990s by Takeda and Hishida. These separate
	effects tests were conducted to understand and model a multi-component
	experiment in which all three processes were included with the introduction
	of air in a heated graphite column. MIT used the FLUENT code to benchmark
	these series of tests with quite good results. These tests are generically
	applicable to prismatic reactors and the lower reflector regions
	of pebble-bed reactors. The second series of tests were performed
	at the NACOK facility for pebble bed reactors as reported by Kuhlmann
	[Kuhlmann, M.B., 1999. Experiments to investigate flow transfer and
	graphite corrosion in case of air ingress accidents in a high-temperature
	reactor]. These tests were aimed at understanding natural circulation
	of pebble bed reactors by simulating hot and cold legs of these reactors.
	The FLUENT code was also successfully used to simulate these tests.
	The results of these benchmarks and the findings will be presented.},
  doi = {DOI: 10.1016/j.nucengdes.2005.11.019},
  file = {Kadak2006.pdf:Kadak2006.pdf:PDF},
  issn = {0029-5493},
  keywords = {Air Ingress, CFD, DLOFC, htgr, PBMR, reviewed},
  owner = {lewis.lloyd},
  timestamp = {2009.04.13},
  url = {http://www.sciencedirect.com/science/article/B6V4D-4J5C849-2/2/aadb82df1a2a1169e2863afed366a54b}
}

@ARTICLE{Kadioglu2010,
  author = {S. Y. Kadioglu and D. A. Knoll},
  title = {A fully second order implicit/explicit time integration technique
	for hydrodynamics plus nonlinear heat conduction problems},
  journal = {Journal of Computational Physics},
  year = {2010},
  volume = {229},
  pages = {3237 - 3249},
  number = {9},
  abstract = {We present a fully second order implicit/explicit time integration
	technique for solving hydrodynamics coupled with nonlinear heat conduction
	problems. The idea is to hybridize an implicit and an explicit discretization
	in such a way to achieve second order time convergent calculations.
	In this scope, the hydrodynamics equations are discretized explicitly
	making use of the capability of well-understood explicit schemes.
	On the other hand, the nonlinear heat conduction is solved implicitly.
	Such methods are often referred to as IMEX methods [2,1,3]. The Jacobian-Free
	Newton Krylov (JFNK) method (e.g. [10,9]) is applied to the problem
	in such a way as to render a nonlinearly iterated IMEX method. We
	solve three test problems in order to validate the numerical order
	of the scheme. For each test, we established second order time convergence.
	We support these numerical results with a modified equation analysis
	(MEA) [21,20]. The set of equations studied here constitute a base
	model for radiation hydrodynamics.},
  doi = {10.1016/j.jcp.2009.12.039},
  issn = {0021-9991},
  keywords = {Hydrodynamics},
  owner = {llloyd},
  timestamp = {2012.10.20},
  url = {http://www.sciencedirect.com/science/article/pii/S0021999110000069}
}

@ARTICLE{Katanishi2007,
  author = {Shoji Katanishi and Kazuhiko Kunitomi},
  title = {Safety evaluation on the depressurization accident in the gas turbine
	high temperature reactor (GTHTR300)},
  journal = {Nuclear Engineering and Design},
  year = {2007},
  volume = {237},
  pages = {1372 - 1380},
  number = {12-13},
  note = {18th International Conference on Structural Mechanics in Nuclear
	Engineering, 18th International Conference on Structural Mechanics
	in Nuclear Engineering},
  abstract = {Japan Atomic Energy Agency has been developing a gas turbine high
	temperature reactor (GTHTR300) with electric power of approximately
	300 MW. One of the unique safety design concepts of this system is
	that events with frequency of occurrence of higher than 10-8/reactor-year
	are evaluated as design basis events in order to show that the frequency
	of large amount of FP release is less than 10-8/reactor-year. According
	to this concept, a depressurization accident by a large break of
	helium piping is postulated as a design basis event. This accident
	is one of the most serious accidents in the high-temperature gas-cooled
	reactors from the viewpoint of loss of coolability. The safety evaluation
	on the accident was conducted based on the actual design of the system.
	The short-term and long-term behaviors of fuel temperature after
	occurrence of the accident, internal pressure of the reactor building,
	oxidation behavior of fuels and graphite structures were evaluated
	and exposure dose of general public was also estimated using the
	results of evaluation of fuel temperature and fuel failure by oxidation.
	All of the evaluation results meet the safety criteria and feasibility
	of the GTHTR300 was shown by this study.},
  doi = {DOI: 10.1016/j.nucengdes.2006.09.037},
  file = {Katanishi2007.pdf:Katanishi2007.pdf:PDF},
  issn = {0029-5493},
  keywords = {htgr, unread},
  owner = {lewis.j.lloyd},
  timestamp = {2009.05.26},
  url = {http://www.sciencedirect.com/science/article/B6V4D-4MFCWDN-3/2/877d16e8838d94075123b583994c9e08}
}

@ARTICLE{Kauffman1992,
  author = {C. W. Kauffman and M. Sichel and P. Wolanski},
  title = {Research on dust explosions at the University of Michigan},
  journal = {Powder Technology},
  year = {1992},
  volume = {71},
  pages = {119 - 134},
  number = {2},
  abstract = {Dust explosion research carried out at the University of Michigan
	during the last two decades has been summarized. Significant results
	are presented on the smoldering combustion of dust heaps, turbulent
	combustion of premixed dust clouds, entrainment and combustion of
	layered dust, and on shock wave ignition of particles and shock wave
	initiated detonative combustion. Also, information on the detonation
	of hybrid mixtures and gaseous mixtures containing nonreactive particles
	is given.},
  doi = {DOI: 10.1016/0032-5910(92)80002-E},
  file = {Kauffman1992.pdf:Kauffman1992.pdf:PDF},
  issn = {0032-5910},
  keywords = {parituclate, unread},
  owner = {lewis.j.lloyd},
  timestamp = {2009.05.26},
  url = {http://www.sciencedirect.com/science/article/B6TH9-440YXRJ-FG/2/0c9aa34b2a8af0db4af3870648031f69}
}

@TECHREPORT{Kazimi1980,
  author = {M. Kazimi and M. Massoud},
  title = {A Condensed Review of Nuclear Reactor Thermal-Hydraulic Computer
	Codes for Two-Phase Flow Analysis},
  institution = {Massachusetts Institute of Technology},
  year = {1980},
  type = {Energy Labratory Report},
  number = {MIT-EL 19-018},
  address = {Cambridge, Massachusetts 02139},
  month = {February},
  file = {Kazimi1980.pdf:Kazimi1980.pdf:PDF},
  owner = {llloyd},
  timestamp = {2012.11.28}
}

@TECHREPORT{Keeys1971,
  author = {Keeys, R. K. F. and Ralph, J. C. and Roberts, D. N.},
  title = {Post Burnout Heat Transfer In High Pressure Steam Water Mixtures
	In a Tube with Cosine Heat Flux Distribution},
  institution = {Atomic Energy Research Estabilishment},
  year = {1971},
  type = {United Kingdom Atomic Energy Authority Report},
  number = {AERE - R 6411},
  address = {Harwell, Berkshire},
  file = {Keeys1971.PDF:Keeys1971.PDF:PDF},
  keywords = {HARWELL, Validation, Heat Transfer},
  owner = {lloydlj},
  timestamp = {2011.08.04}
}

@BOOK{Kelley1995,
  title = {Iterative Methods for Linear and Nonlinear Equations},
  publisher = {Society for Industrial and Applied Mathematics},
  year = {1995},
  author = {C. T. Kelley},
  file = {Kelley1995.pdf:Kelley1995.pdf:PDF},
  owner = {llloyd},
  timestamp = {2012.11.03}
}

@ARTICLE{Knoll2004,
  author = {Knoll, D. A. and Keyes, D. E.},
  title = {Review: Jacobian-free Newton-Krylov methods: a survey of approaches
	and applications},
  journal = {Journal of Computational Physics},
  year = {2004},
  volume = {193},
  pages = {357-397},
  comment = {pR},
  file = {Knoll2004.pdf:Knoll2004.pdf:PDF},
  keywords = {Newton Methods, Krylov Solvers, JFNK},
  owner = {lloydlj},
  timestamp = {2011.08.03}
}

@ARTICLE{Knoll2000,
  author = {Knoll, D. A. and Mousseau, Vincent},
  title = {On Newton-Krylov Multigrid Methods for the Incompressible Navier-Stokes
	Equations},
  journal = {Journal of Computational Physics},
  year = {2000},
  volume = {163},
  pages = {262-267},
  file = {Knoll2000.pdf:Knoll2000.pdf:PDF},
  owner = {lloydlj},
  timestamp = {2011.09.15}
}

@ARTICLE{Knoll2001,
  author = {D. A. Knoll and W. J. Rider and G. L. Olson},
  title = {Nonlinear convergence, accuracy, and time step control in nonequilibrium
	radiation diffusion},
  journal = {Journal of Quantitative Spectroscopy and Radiative Transfer},
  year = {2001},
  volume = {70},
  pages = {25 - 36},
  number = {1},
  abstract = {We study the interaction between converging the nonlinearities within
	a time step and time step control, on the accuracy of nonequilibrium
	radiation diffusion calculations. Typically, this type of calculation
	is performed using operator-splitting where the nonlinearities are
	lagged one time step. This method of integrating the nonlinear system
	results in an “effective� time-step constraint to obtain accuracy.
	A time-step control that limits the change in dependent variables
	(usually energy) per time step is used. We investigate the possibility
	that converging the nonlinearities within a time step may allow significantly
	larger time-step sizes and improved accuracy as well. The previously
	described Jacobian-free Newton–Krylov method (JQSRT 63 (1999) 15)
	is used to converge all nonlinearities within a time step. In addition,
	a new time-step control method, based on the hyperbolic model of
	a thermal wave (J. Comput. Phys. 152 (1999) 790), is employed. The
	benefits and cost of a second-order accurate time step are considered.
	It is demonstrated that for a chosen accuracy, significant increases
	in solution efficiency can be obtained by converging nonlinearities
	within a time step.},
  doi = {10.1016/S0022-4073(00)00112-6},
  file = {Knoll2001.pdf:Knoll2001.pdf:PDF},
  issn = {0022-4073},
  keywords = {Radiation diffusion},
  owner = {llloyd},
  timestamp = {2012.11.16},
  url = {http://www.sciencedirect.com/science/article/pii/S0022407300001126}
}

@NUREG{Koontz1983,
  author = {Koontz, A. S. and Cuta, J. M.},
  booktitle = {COBRA/TRAC - A Thermal-Hydraulics Code for Transient Analysis of
	Nuclear Reactor Vessels and Primary Coolant Systems},
  volume = {5},
  title = {Programmers' Manual},
  institution = {Pacific Northwest Laboratory},
  number = {NUREG/CR-3046
	
	PNL-4385
	
	R4},
  year = {1983},
  keywords = {COBRA, Systems Codes, Two-Phase Flow, Mathematics},
  file = {Koontz1983.pdf:Koontz1983.pdf:PDF},
  owner = {lloydlj},
  timestamp = {2011.09.15},
  type = {NUREG}
}

@ARTICLE{Kunitomi2004,
  author = {Kazuhiko Kunitomi and Shusaku Shiozawa},
  title = {Safety design},
  journal = {Nuclear Engineering and Design},
  year = {2004},
  volume = {233},
  pages = {45 - 58},
  number = {1-3},
  note = {Japan's HTTR},
  abstract = {JAERI established the safety design philosophy of the HTTR based on
	that of current reactors such as LWR in Japan, considering inherent
	safety features of the HTTR. The strategy of defense in depth was
	implemented so that the safety engineering functions such as control
	of reactivity, removal of residual heat and confinement of fission
	products shall be well performed to ensure safety. However, unlike
	the LWR, the inherent design features of the high-temperature gas-cooled
	reactor (HTGR) enables the HTTR meet stringent regulatory criteria
	without much dependence on active safety systems. On the other hand,
	the safety in an accident typical to the HTGR such as the depressurization
	accident initiated by a primary pipe rupture shall be ensured. The
	safety design philosophy of the HTTR considers these unique features
	appropriately and is expected to be the basis for future Japanese
	HTGRs. This paper describes the safety design philosophy and safety
	evaluation procedure of the HTTR especially focusing on unique considerations
	to the HTTR. Also, experiences obtained from an HTTR safety review
	and R&D needs for establishing the safety philosophy for the future
	HTGRs are reported.},
  doi = {DOI: 10.1016/j.nucengdes.2004.07.010},
  file = {Kunitomi2004.pdf:Kunitomi2004.pdf:PDF},
  issn = {0029-5493},
  keywords = {htgr, unread},
  owner = {lewis.lloyd},
  timestamp = {2009.04.13},
  url = {http://www.sciencedirect.com/science/article/B6V4D-4DNB1TN-1/2/4a18eec91d1db7db9201d7618d24aed2}
}

@ARTICLE{Kuran2006,
  author = {S. Kuran and Y. Xu and X. Sun and L. Cheng and H.J. Yoon and S.T.
	Revankar and M. Ishii and W. Wang},
  title = {Startup transient simulation for natural circulation boiling water
	reactors in PUMA facility},
  journal = {Nuclear Engineering and Design},
  year = {2006},
  volume = {236},
  pages = {2365 - 2375},
  number = {22},
  abstract = {In view of the importance of instabilities that may occur at low-pressure
	and -flow conditions during the startup of natural circulation boiling
	water reactors, startup simulation experiments were performed in
	the Purdue University Multi-Dimensional Integral Test Assembly (PUMA)
	facility. The simulations used pressure scaling and followed the
	startup procedure of a typical natural circulation boiling water
	reactor. Two simulation experiments were performed for the reactor
	dome pressures ranging from 55[thin space]kPa to 1[thin space]MPa,
	where the instabilities may occur. The experimental results show
	the signature of condensation-induced oscillations during the single-phase-to-two-phase
	natural circulation transition. The results also suggest that a rational
	startup procedure is needed to overcome the startup instabilities
	in natural circulation boiling water reactor designs.},
  doi = {DOI: 10.1016/j.nucengdes.2005.11.002},
  file = {Kuran2006.pdf:Kuran2006.pdf:PDF},
  issn = {0029-5493},
  owner = {llloyd},
  timestamp = {2010.12.08},
  url = {http://www.sciencedirect.com/science/article/B6V4D-4KFV3RX-1/2/51a2f870847f43e053e270bab6747405}
}

@PHDTHESIS{Lane2009,
  author = {Lane, J. W.},
  title = {The Development of a Comprehensive Annular Flow Modeling Package
	For Two-Phase Three-Field Transient Safety Analysis Codes},
  school = {The Pennsylvania State University},
  year = {2009},
  file = {Lane2009.pdf:Lane2009.pdf:PDF},
  owner = {llloyd},
  timestamp = {2012.11.03}
}

@ARTICLE{Lanzkron1996,
  author = {P. Lanzkron and D. Rose and J. Wilkes},
  title = {An Analysis of Approximate Nonlinear Elimination},
  journal = {SIAM Journal on Scientific Computing},
  year = {1996},
  volume = {17},
  pages = {538-559},
  number = {2},
  doi = {10.1137/S106482759325154X},
  eprint = {http://epubs.siam.org/doi/pdf/10.1137/S106482759325154X},
  file = {Lanzkron1996.pdf:Lanzkron1996.pdf:PDF},
  owner = {llloyd},
  timestamp = {2012.11.03},
  url = {http://epubs.siam.org/doi/abs/10.1137/S106482759325154X}
}

@ARTICLE{Lee1992,
  author = {J. H. S. Lee and F. Zhang and R. Knystautas},
  title = {Propagation mechanisms of combustion waves in dust-air mixtures},
  journal = {Powder Technology},
  year = {1992},
  volume = {71},
  pages = {153 - 162},
  number = {2},
  abstract = {The current status of understanding of flame structure and mechanisms
	of propagation in dust-air mixtures is reviewed. The equilibrium
	properties, which are those that depend on the energetics of the
	medium, are well described by existing computer codes. However, measurements
	of the dynamic parameters, which are those that depend on the rate
	of reaction (e.g. flame thickness, minimum ignition energy, burning
	velocity, etc.) are virtually non-existent for dust combustion. The
	limited measurements of flame structure that exist suggest that dust
	flames are thin and propagate predominantly by thermal and molecular
	diffusion rather than by radiative preheating. However, the available
	results are too few and far between to be able to draw firm conclusions.
	The more convenient experimental measurement appears to be the quenching
	distance. The present paper reviews existing measurements of the
	quenching distance and describes a novel experimental approach to
	achieving a homogeneously suspended dust-air medium for laminar dust
	combustion studies.},
  doi = {DOI: 10.1016/0032-5910(92)80004-G},
  file = {Lee1992.pdf:Lee1992.pdf:PDF},
  issn = {0032-5910},
  keywords = {particulate; unread},
  owner = {lewis.j.lloyd},
  timestamp = {2009.05.26},
  url = {http://www.sciencedirect.com/science/article/B6TH9-440YXRJ-FJ/2/d0f711821541f329688928eaca6bfaf1}
}

@BOOK{LeVeque2007,
  title = {Finite Difference Methods for Ordinary and Partial Differential Equations},
  publisher = {Society for Industrial and Applied Mathematics},
  year = {2007},
  author = {R. J. LeVeque},
  owner = {llloyd},
  timestamp = {2012.11.06}
}

@BOOK{LeVeque2002,
  title = {Finite Volume Methods for Hyperbolic Problems},
  publisher = {Cambridge University Press},
  year = {2002},
  author = {R. J. LeVeque},
  series = {Cambridge Texts in Applied Mathematics},
  address = {32 Avenue of the Americas
	
	New York, NY 10013-2473
	
	USA},
  owner = {llloyd},
  timestamp = {2012.11.11}
}

@TECHREPORT{Li1999,
  author = {X. S. Li and J. W. Demmel and J. R. Gilbert and L. Grigori and M.
	Shao and I. Yamazaki},
  title = {{SuperLU} User's Guide},
  institution = {Ernest Orlando Lawrence Berkeley National Laboratory},
  year = {1999},
  type = {LBNL},
  number = {44289},
  month = {September},
  file = {Li1999.pdf:Li1999.pdf:PDF},
  owner = {llloyd},
  timestamp = {2012.11.14}
}

@ARTICLE{Lignell2007,
  author = {David O. Lignell and Jacqueline H. Chen and Philip J. Smith and Tianfeng
	Lu and Chung K. Law},
  title = {The effect of flame structure on soot formation and transport in
	turbulent nonpremixed flames using direct numerical simulation},
  journal = {Combustion and Flame},
  year = {2007},
  volume = {151},
  pages = {2 - 28},
  number = {1-2},
  abstract = {Direct numerical simulations of a two-dimensional, nonpremixed, sooting
	ethylene flame are performed to examine the effects of soot-flame
	interactions and transport in an unsteady configuration. A 15-step,
	19-species (with 10 quasi-steady species) chemical mechanism was
	used for gas chemistry, with a two-moment, four-step, semiempirical
	soot model. Flame curvature is shown to result in flames that move,
	relative to the fluid, either toward or away from rich soot formation
	regions, resulting in soot being essentially convected into or away
	from the flame. This relative motion of flame and soot results in
	a wide spread of soot in the mixture fraction coordinate. In regions
	where the center of curvature of the flame is in the fuel stream,
	the flame motion is toward the fuel and soot is located near the
	flame at high temperature and hence has higher reaction rates and
	radiative heat fluxes. Soot-flame breakthrough is also observed in
	these regions. Fluid convection and flame displacement velocity relative
	to fluid convection are of similar magnitudes while thermophoretic
	diffusion is 5-10 times lower. These results emphasize the importance
	of both unsteady and multidimensional effects on soot formation and
	transport in turbulent flames.},
  doi = {DOI: 10.1016/j.combustflame.2007.05.013},
  file = {Lignell2007.pdf:Lignell2007.pdf:PDF},
  issn = {0010-2180},
  keywords = {particulate, unread},
  owner = {lewis.lloyd},
  timestamp = {2009.04.13},
  url = {http://www.sciencedirect.com/science/article/B6V2B-4PDK4DY-2/2/847081f79112a531bce3018af1dd41ba}
}

@ARTICLE{Liles1978,
  author = {D. R. Liles and Wm. H. Reed},
  title = {A semi-implicit method for two-phase fluid dynamics},
  journal = {Journal of Computational Physics},
  year = {1978},
  volume = {26},
  pages = {390 - 407},
  number = {3},
  abstract = {A new technique is developed for solving the equations of two-phase
	fluid dynamics. This technique involves a semi-implicit differencing
	of the field equations and a variation of the Newton Gauss Seidel
	iterative method for solving at each time level the resulting system
	of algebraic equations. Although the technique can be applied to
	any of several sets of equations representing two-phase flow, including
	the two-fluid equations, numerical results are presented here for
	the drift-flux approximation in one dimension. Significant advantages
	of the method are its stability, ease of programming for complicated
	flow networks, and ease of extension to problems in two or three
	dimensions.},
  doi = {10.1016/0021-9991(78)90077-3},
  file = {Liles1978.pdf:Liles1978.pdf:PDF},
  issn = {0021-9991},
  keywords = {two-phase flow, semi-implicit},
  owner = {llloyd},
  timestamp = {2012.05.08},
  url = {http://www.sciencedirect.com/science/article/pii/0021999178900773}
}

@ARTICLE{Ma2009,
  author = {Weimin Ma and Aram Karbojian and Bal Raj Sehgal and Truc-Nam Dinh},
  title = {Thermal-hydraulic performance of heavy liquid metal in straight-tube
	and U-tube heat exchangers},
  journal = {Nuclear Engineering and Design},
  year = {2009},
  volume = {239},
  pages = {1323 - 1330},
  number = {7},
  abstract = {Motivated by an increased interest in heavy liquid metal (lead or
	lead alloy) cooled fast reactors (LFR) and accelerator-driven system
	(ADS), the present paper presents a study on resistance characteristics
	and heat transfer performance of liquid lead bismuth eutectic (LBE)
	flow through a straight-tube heat exchanger and a U-tube heat exchanger.
	The investigation is performed on the TALL test facility at KTH.
	The heat exchangers have counter-current flow arrangement, and are
	made from a pair of 1-m-long concentric ducts, with the LBE flowing
	in the inner tube of 10 mm I.D. and the secondary coolant flowing
	in the annulus. The inlet temperature of LBE into the heat exchangers
	is from 200 °C to 450 °C with temperature drops from 0 °C to 100 °C
	within the LBE flow range of Re = 104-105. Analysis of the experimental
	results obtained provides a basic understanding and quantification
	of the regimes of lead-bismuth flow and heat transfer through a straight
	tube and a U-shaped tube. The unique data base also serves as benchmark
	and improvement for system thermal-hydraulic codes (e.g. RELAP, TRAC/AAA)
	whose development and testing were dominantly driven by applications
	in water-cooled systems. Lessons and insights learnt from the study
	and recommendations for the heat exchanger selection are discussed.},
  doi = {DOI: 10.1016/j.nucengdes.2009.03.014},
  file = {Ma2009.pdf:Ma2009.pdf:PDF},
  issn = {0029-5493},
  keywords = {fbr, unread},
  owner = {lewis.j.lloyd},
  timestamp = {2009.06.19},
  url = {http://www.sciencedirect.com/science/article/B6V4D-4W50K7X-2/2/164dd30217c09e1f1d63a3d8933f9082}
}

@ARTICLE{Mahaffy1993,
  author = {J. H. Mahaffy},
  title = {Numerics of codes: stability, diffusion, and convergence},
  journal = {Nuclear Engineering and Design},
  year = {1993},
  volume = {145},
  pages = {131 - 145},
  number = {1–2},
  abstract = {The numerical methods used in the primary US reactor safety codes
	are summarized. The basic Courant-type stability limits for these
	codes are reviewed, and more subtle stability problems arising from
	the explicit evaluation of various friction and heat-transfer coefficients
	are discussed. Much of the stability and robustness of these codes
	has come at the expense of high numerical diffusion. The impact of
	numerical diffusion is illustrated. The question of convergence of
	solutions of the difference equations to those of the original differential
	equations is also addressed.},
  doi = {10.1016/0029-5493(93)90063-F},
  file = {Mahaffy1993.pdf:Mahaffy1993.pdf:PDF},
  issn = {0029-5493},
  keywords = {semi-implicit},
  owner = {llloyd},
  timestamp = {2012.10.20},
  url = {http://www.sciencedirect.com/science/article/pii/002954939390063F}
}

@ARTICLE{Mahaffy1982,
  author = {J. H. Mahaffy},
  title = {A stability-enhancing two-step method for fluid flow calculations},
  journal = {Journal of Computational Physics},
  year = {1982},
  volume = {46},
  pages = {329 - 341},
  number = {3},
  abstract = {An extension is presented of semi-implicit methods such as the implicit
	continuous Eulerian (ICE) technique for modeling fluid flow. This
	new approach eliminates the material Courant stability limit associated
	with semi-implicit methods at little additional computational cost.},
  doi = {10.1016/0021-9991(82)90019-5},
  file = {Mahaffy1982.pdf:Mahaffy1982.pdf:PDF},
  issn = {0021-9991},
  keywords = {SETS, two-phase numerics},
  owner = {llloyd},
  timestamp = {2012.10.20},
  url = {http://www.sciencedirect.com/science/article/pii/0021999182900195}
}

@TECHREPORT{Makihara2003,
  author = {Y. Makihara},
  title = {Use and development of coupled computer codes for the analysis of
	accidents at nuclear power plants: proceedings of a technical meeting
	held in {Vienna}, 26-28 {November} 2003},
  institution = {{International Atomic Energy Agency}},
  year = {2003},
  file = {Makihara2003.pdf:Makihara2003.pdf:PDF},
  owner = {llloyd},
  timestamp = {2012.11.17}
}

@TECHREPORT{McHugh1995,
  author = {McHugh, P. R.},
  title = {An Investigation of Newton-Krylov Algorithms for Solving Incompressible
	and Low Mach Number Compressible Fluid Flow and Heat Transfer Problems
	Using Finite Volume Discretization},
  institution = {Idaho National Engineering Laboratory},
  year = {1995},
  number = {INEL-95/0118},
  abstract = {Fully coupled, Newton-Krylov algorithms are investigated for solving
	sttongly coupled, nonlinear systems of partial differential equations
	arising in the field of computational fluid dynamics. Primitive variable
	forms of the steady incompressible and compressible Navier-Stokes
	and energy equations that describe the flow of a laminar Newtonian
	fluid in two-dimensions are specifically considered. Numerical solutions
	are obtained by first integrating over discrete finite volumes that
	compose the computational mesh. The resulting system of nonlinear
	algebraic equations are linearized using Newton's method. Preconditioned
	Krylov subspace based iterative algorithms then solve these linear
	systems on each Newton iteration. Selected Krylov algorithms include
	the Amoldi-based Generalized Minimal RESidual (GMRES) algorithm,
	and the Lanczos-based Conjugate Gradient Squared (CGS), Bi-CGSTAB,
	and Transpose-Free Quasi-Minimal Residual (TFQMR) algorithms. Both
	Incomplete Lower-Upper @.U) factorization and domain-based additive
	and multiplicative Schwan preconditioning strategies are studied.
	Numerical techniques such as mesh sequencing, adaptive damping, pseudo-transient
	relaxation, and parameter continuation are used to improve the solution
	efficiency, while algorithm implementation is simplified using a
	numerical Jacobian evaluation.
	
	
	The capabilities of standard Newton-Krylov algorithms are demonstrated
	via solutions to both incompressible and compressible flow problems.
	Incompressible flow problems include natural convection in an enclosed
	cavity, and mixdforced convection past a backward facing step. Additionally,
	matrix-free Newton-Krylov implementations are constructed by approximating
	the Jacobian-vector products appearing in the Krylov algorithms with
	finite difference approximations. Performance of the matrix-fiee
	implementation is found to depend upon problem size, problem nonlinearity,
	and Krylov algorithm selection. Higher order accurate solutions for
	mixed convection flow are obtained using the third order cubic upwind
	interpolation scheme (CUI) convection scheme with a defect correction
	procedure and several CPU performance enhancement techniques. Solution
	efficiency for high Reynolds number forced convection flow is investigated
	using a discrete pressure equation formulation, alternative cell
	ordering strategies, and pseudo-transient relaxation. Solutions to
	the compressible flow problem, consisting of low Mach number subsonic
	flow past a backstep, found ILU preconditioners to be strongly sensitive
	to cell ordering and less effective than domain based preconditioners
	at low Mach numbers.},
  file = {McHugh1995.pdf:McHugh1995.pdf:PDF},
  keywords = {Newton-Krylov},
  owner = {lloydlj},
  timestamp = {2011.08.03}
}

@PHDTHESIS{Meholic2011,
  author = {Meholic, M. J.},
  title = {The Development of a Non-Equilibrium Dispersed Flow Film Boiling
	Heat Transfer Modeling Package},
  school = {The Pennsylvania State University},
  year = {2011},
  file = {Meholic2011.pdf:Meholic2011.pdf:PDF},
  owner = {llloyd},
  timestamp = {2012.11.03}
}

@ARTICLE{Menck2002,
  author = {J. Menck},
  title = {An Approximate Newton-Like Coupling of Subsystems},
  journal = {ZAMM - Journal of Applied Mathematics and Mechanics / Zeitschrift
	für Angewandte Mathematik und Mechanik},
  year = {2002},
  volume = {82},
  pages = {101--114},
  number = {2},
  abstract = {Complex technical systems are often assembled from well-studied subsystems.
	Here, we elaborate on the obvious idea of coupling existing subsystem
	solvers to solve a coupled system. More specifically, we present
	a matrix-free iterative method that is inspired by a Newton type
	coupling of the subsystems but aims at efficiently controlled linear
	convergence. We prove its convergence and propose a control mechanism
	to optimize its efficiency. We illustrate the method's properties
	with the help of a numerical example.Note: (As yet) we only deal
	with the stationary case; mathematically speaking, we are looking
	for roots of systems of nonlinear equations.},
  doi = {10.1002/1521-4001(200202)82:2<101::AID-ZAMM101>3.0.CO;2-O},
  file = {Menck2002.pdf:Menck2002.pdf:PDF},
  issn = {1521-4001},
  keywords = {Newton's method, approximate Newton's method, block-structured Newton's
	method, coupled systems, stationary process simulation, work control},
  owner = {llloyd},
  publisher = {WILEY-VCH Verlag Berlin GmbH},
  timestamp = {2012.11.03},
  url = {http://dx.doi.org/10.1002/1521-4001(200202)82:2<101::AID-ZAMM101>3.0.CO;2-O}
}

@ARTICLE{Mousseau2000,
  author = {Mousseau, Vincent and Knoll, D. A. and Rider, W. J.},
  title = {Physics-Based Preconditioning and the Newton-Krylov Method for Non-Equilibrium
	Radiation Diffusion},
  journal = {Journal of Computational Physics},
  year = {2000},
  volume = {160},
  pages = {743-765},
  file = {Mousseau2000.pdf:Mousseau2000.pdf:PDF},
  keywords = {Newton-Krylov, Linear Algebra, Preconditioners},
  owner = {lloydlj},
  timestamp = {2011.09.15}
}

@TECHREPORT{Oh2004,
  author = {Oh, Chang and Davis, Cliff and Moore, Richard and No, Hee C. and
	Kim, Jong and Park, Goon C. and Lee, John and Martin, W. and Holloway,
	James},
  title = {Development of Safety Analysis Codes and Experimental Validation
	for a Very High Temperature Gas Cooled Reactor},
  institution = {Idaho National Engineering and Environmental Laboratory},
  year = {2004},
  type = {Report},
  number = {INEEL/EXT-04-02459},
  address = {Idaho Falls, ID 83415},
  month = {November},
  abstract = {The very high temperature gas-cooled reactors (VHTGRs) are those concepts
	that have average coolant temperatures above 900 degrees C or operational
	fuel temperatures above 1250 degrees C. These concepts provide the
	potential for increased energy conversion efficiency and for high-temperature
	process heat application in addition to power generation and nuclear
	hydrogen generation. While all the High Temperature Gas Cooled Reactor
	(HTGR) concepts have sufficiently high temperatures to support process
	heat applications, such as desalination and cogeneration, the VHTGR's
	higher temperatures are suitable for particular applications such
	as thermochemical hydrogen production. However, the high temperature
	operation can be detrimental to safety following a loss-of-coolant
	accident (LOCA) initiated by pipe breaks caused by seismic or other
	events. Following the loss of coolant through the break and coolant
	depressurization, air from the containment will enter the core by
	molecular diffusion and ultimately by natural convection, leading
	to oxidation of the in-core graphite structures and fuel. The oxidation
	will release heat and accelerate the heatup of the reactor core.
	Thus, without any effective countermeasures, a pipe break may lead
	to significant fuel damage and fission product release. The Idaho
	National Engineering and Environmental Laboratory (INEEL) has investigated
	this event for the past three years for the HTGR. However, the computer
	codes used, and in fact none of the world's computer codes, have
	been sufficiently developed and validated to reliably predict this
	event. New code development, improvement of the existing codes, and
	experimental validation are imperative to narrow the uncertaninty
	in the predictions of this type of accident. The objectives of this
	Korean/United States collaboration are to develop advanced computational
	methods for VHTGR safety analysis codes and to validate these computer
	codes.},
  doi = {DOI: 10.2172/911001},
  file = {Oh2004.pdf:Oh2004.pdf:PDF},
  keywords = {htgr, unread},
  owner = {lewis.lloyd},
  timestamp = {2009.04.14},
  url = {http://www.inl.gov/technicalpublications/Documents/3028244.pdf}
}

@INPROCEEDINGS{Paraschivoiu1999,
  author = {M. Paraschivoiu and X. Cai and M. Sarkis and D. P. Young and D. E.
	Keyes},
  title = {Multi-Domain Multi-Model Formulation for Compressible Flows: Conservative
	Interface Coupling and Parallel Implicit Solvers for {3D} Unstructured
	Meshes},
  booktitle = {Unstructured Meshes, AIAA Paper 99-0784, American Institute of Aeronautics
	and Astronautics},
  year = {1999},
  pages = {99--0784},
  __markedentry = {[llloyd:]},
  file = {Paraschivoiu1999.pdf:Paraschivoiu1999.pdf:PDF},
  owner = {llloyd},
  timestamp = {2012.11.25}
}

@ARTICLE{Park2009,
  author = {Park, Hyeong Kae and Nourgaliev, Robert R. and Martineau, Richard
	C. and Knoll, D. A.},
  title = {On physics-based preconditioning of the Navier-Stokes equations},
  journal = {Journal of Computational Physics},
  year = {2009},
  volume = {228},
  pages = {9131-9146},
  comment = {pR},
  file = {Park2009.pdf:Park2009.pdf:PDF},
  keywords = {Fluid Dynamics, Mathematics, Newton-Krylov},
  owner = {lloydlj},
  timestamp = {2011.08.04}
}

@ARTICLE{Park2009a,
  author = {I.K. Park and H.K. Cho and H.Y. Yoon and J.J. Jeong},
  title = {Numerical effects of the semi-conservative form of momentum equations
	for multi-dimensional two-phase flows},
  journal = {Nuclear Engineering and Design},
  year = {2009},
  volume = {239},
  pages = {2365 - 2371},
  number = {11},
  abstract = {Some of the thermal hydraulics codes for multi-dimensional two-phase
	flow analysis use the non-conservative form of momentum equations
	for numerical convenience. From a mathematical point of view, these
	equations are equal to those in the conservative form. But, numerical
	integration of the non-conservative momentum equations over a control
	volume results in different solution characteristics, which may cause
	inaccurate solutions under some two-phase flow conditions. In this
	paper, a semi-conservative form of the momentum equations is suggested
	which is close to the conservative form but still maintains the feature
	of the non-conservative form. The numerical results of the semi-conservative
	and the non-conservative forms are compared against analytical solutions
	and the solutions of the FLUENT code that uses the conservative form.
	The results clearly showed that the semi-conservative form of the
	momentum equations provides better solutions than the non-conservative
	form, especially for heterogeneous two-phase flows.},
  doi = {DOI: 10.1016/j.nucengdes.2009.06.011},
  file = {Park2009.pdf:Park2009.pdf:PDF},
  issn = {0029-5493},
  keywords = {critical.heat.flux; unread, two-phase flow},
  owner = {lewis.j.lloyd},
  timestamp = {2009.10.06},
  url = {http://www.sciencedirect.com/science/article/B6V4D-4WPS9RX-3/2/c8cb1549a58dc524a4041e9b6b1cad3b}
}

@ARTICLE{Pilch1996c,
  author = {Martin M. Pilch},
  title = {Hydrogen combustion during direct containment heating events},
  journal = {Nuclear Engineering and Design},
  year = {1996},
  volume = {164},
  pages = {117 - 136},
  number = {1-3},
  abstract = {Direct containment heating (DCH) has the potential to cause short-term
	overpressure failure of the containment in a nuclear power plant
	that has experienced a core melt accident. This paper addresses the
	possible modes of hydrogen combustion in the dome region of a pressurized
	water reactor containment building during a DCH event. The combustion
	modes considered are: combustion of hot hydrogen jets, deflagrations,
	volumetric combustion, and mixing limited combustion. The limits
	for each combustion mode are defined and interpreted in the context
	of both experiment and reactor applications. With the exception of
	jet combustion, heat transfer to structures can significantly limit
	the contribution of hydrogen combustion to peak containment pressures
	on the DCH time scale. Implementation of these hydrogen combustion
	models into the two-cell equilibrium model is discussed.},
  doi = {DOI: 10.1016/0029-5493(96)01228-9},
  file = {Pilch1996c.pdf:Pilch1996c.pdf:PDF},
  issn = {0029-5493},
  keywords = {dch, unread},
  owner = {lewis.lloyd},
  timestamp = {2009.04.13},
  url = {http://www.sciencedirect.com/science/article/B6V4D-3VTJ9YC-5/2/cb4689210d8d8096f9e8dcfc4990f57f}
}

@ARTICLE{Pilch1996d,
  author = {Martin M. Pilch},
  title = {A two-cell equilibrium model for predicting direct containment heating},
  journal = {Nuclear Engineering and Design},
  year = {1996},
  volume = {164},
  pages = {61 - 94},
  number = {1-3},
  abstract = {This paper discusses two adiabatic equilibrium models. Assessment
	and validation of the separate effects (kinetic) models and the parameters
	(i.e. particle size) that control them are not required. The first,
	a single-cell equilibrium model, places a true upper bound on direct
	containment heating (DCH) loads. This upper bound, when compared
	with the entire DCH database, often far exceeds experiment observations
	by a margin too large to be useful in reactor analyses. The single-cell
	model is used as a conceptual seed for a two-cell model. A two-cell
	equilibrium (TCE) model is developed that captures the dominant mitigating
	features of containment compartmentalization and the noncoherence
	of the entrainment and blowdown processes. The existing DCH database
	has been used to extensively validate the TCE model. DCH loads are
	shown to be insensitive to physical scale and details of the subcompartment
	geometry. A simple model is developed to predict the coherence of
	debris dispersal and reactor coolant system blowdown. The coherence
	ratio is independent of physical scale and only weakly dependent
	on cavity design.},
  doi = {DOI: 10.1016/0029-5493(96)01230-7},
  file = {Pilch1996b.pdf:Pilch1996b.pdf:PDF},
  issn = {0029-5493},
  keywords = {dch, unread},
  owner = {lewis.lloyd},
  timestamp = {2009.04.13},
  url = {http://www.sciencedirect.com/science/article/B6V4D-3VTJ9YC-3/2/eb8f2de0e27078963c72b659d090538d}
}

@ARTICLE{Pilch1996b,
  author = {M. M. Pilch and M. D. Allen},
  title = {Closure of the direct containment heating issue for Zion},
  journal = {Nuclear Engineering and Design},
  year = {1996},
  volume = {164},
  pages = {37 - 60},
  number = {1-3},
  abstract = {This paper, which was originally published in more detail (M.M. Pilch,
	M.D. Allen, D.L. Knudsen, D.W. Stamps and E.L. Tadios, Rep. NUREG/CR-6075,
	Supplement 1, 1994b (Sandia National Laboratories, Albuquerque, NM)),
	provides closure of the direct containment heating (DCH) issue for
	the Zion plant. It incorporates the comments and suggestions of the
	peer reviewers of NUREG/CR-6075 (M.M. Pilch, H. Yan, and T.G. Theofanous,
	Rep. NUREG/CR-6075, SAND93-1535, 1994a (Sandia National Laboratories,
	Albuquerque, NM)) and specifically includes assessments of four new
	splinter scenarios defined in working group meetings and modeling
	enhancements recommended by the working groups. In the four new scenarios,
	consistency of the initial conditions has been implemented by using
	insights from systems-level codes. was used to analyze three short-term
	station blackout cases with different leak rates. In all three cases,
	the hot leg or surge line failed well before the lower head and thus
	the primary system depressurized to a point where DCH was no longer
	considered a threat. However, these calculations were continued to
	lower head failure in order to gain insights that were useful in
	establishing the initial and boundary conditions. The most useful
	insights are that the reactor coolant system pressure is low at vessel
	breach, metallic blockages in the core region do not melt and relocate
	into the lower plenum, and melting of upper plenum steel is correlated
	with hot leg failure. The output was used as input to to assess the
	containment conditions at vessel breach. The containment-side conditions
	predicted by are similar to those originally specified in NUREG/CR-6075.
	The methodology originally developed in NUREG/CR-6075 (M.M. Pilch,
	H. Yan, and T.G. Theofanous, Rep. NUREG/CR-6075, SAND93-1535, 1994a
	(Sandia National Laboratories, Albuquerque, NM)) was used to analyze
	the new splinter scenarios. Some modeling enhancements in response
	to working group discussions were implemented for these analyses.
	The entrainment of hydrogen pre-existing in the atmosphere into a
	burning jet was examined more carefully. In addition, the impact
	of DCH-induced deflagrations on DCH loads was quantified. A new computational
	tool--the two-cell equilibrium--Latin hypercube sampling (TCE-LHS)
	code--was developed for this effort to perform Monte Carlo sampling
	of the scenario distributions. The TCE-LHS code was benchmarked against
	the original Scenario I calculations in NUREG/CR-6075 performed using
	the code, which is based on the method of discrete probability distributions.
	The results were in excellent agreement. The analyses of the new
	scenarios showed no intersection of the load distributions and the
	containment fragility curves, and thus the containment failure probability
	was negligible for each scenario. These supplemental analyses complete
	closure of the DCH issue for Zion.},
  doi = {DOI: 10.1016/0029-5493(96)01229-0},
  file = {Pilch1996a.pdf:Pilch1996a.pdf:PDF},
  issn = {0029-5493},
  keywords = {dch, unread},
  owner = {lewis.lloyd},
  timestamp = {2009.04.13},
  url = {http://www.sciencedirect.com/science/article/B6V4D-3VTJ9YC-2/2/5c4767ca167549567a1fae2653d98f21}
}

@TECHREPORT{Pilch1995,
  author = {Martin M. Pilch and Allen, M. D. and Bergerson, K.D. and Tadios,
	E. L. and Stamps, D. W. and Spencer, B. W. and Quick, K.S. and Knudson,
	D.L.},
  title = {The probability of containment failure by direct containment heating
	in Surry.},
  institution = {Nuclear Regulatory Commission, Washington, DC (United States). Div.
	of Systems Technology; Sandia National Labs., Albuquerque, NM (United
	States)},
  year = {1995},
  number = {NUREG/CR--6109, SAND--93-2078, ON: TI95012121},
  doi = {DOI: 10.2172/67788},
  file = {Pilch1995.pdf:Pilch1995.pdf:PDF},
  keywords = {dch, unread},
  owner = {lewis.lloyd},
  timestamp = {2009.04.13},
  url = {http://www.osti.gov/bridge/servlets/purl/67788-TE2yYc/webviewable/67788.pdf}
}

@TECHREPORT{Pilch1996a,
  author = {Pilch, M. M. and Allen, M. D. and Klamerus, E.W.},
  title = {Resolution of the direct containment heating issue for all Westinghouse
	plants with large dry containments or subatmospheric containments},
  institution = {Nuclear Regulatory Commission, Washington, DC (United States). Div.
	of Systems Technology; Sandia National Labs., Albuquerque, NM (United
	States)},
  year = {1996},
  number = {NUREG/CR--6338, SAND--95-2381,ON: TI96007153; TRN: TRN: AHC29607%%125},
  doi = {DOI: 10.2172/206606},
  file = {Pilch1996d.pdf:Pilch1996d.pdf:PDF},
  keywords = {dch, unread},
  owner = {lewis.lloyd},
  timestamp = {2009.04.13},
  url = {http://www.osti.gov/bridge/servlets/purl/206606-eFqJl7/webviewable/206606.pdf}
}

@TECHREPORT{Pilch1994,
  author = {Martin M. Pilch and H. Yan and T.G. Theofanous},
  title = {The probability of containment failure by direct containment heating
	in Zion.},
  institution = {Nuclear Regulatory Commission, Washington, DC (United States). Div.
	of Systems Technology; Sandia National Labs., Albuquerque, NM (United
	States)},
  year = {1994},
  number = {NUREG/CR--6075, SAND--93-1535, ON: TI95005275, BR: GB0103012},
  abstract = {This report is the first step in the resolution of the Direct Containment
	Heating (DCH) issue for the Zion Nuclear Power Plant using the Risk
	Oriented Accident Analysis Methodology (ROAAM). This report includes
	the definition of a probabilistic framework that decomposes the DCH
	problem into three probability density functions that reflect the
	most uncertain initial conditions (UO{sub 2} mass, zirconium oxidation
	fraction, and steel mass). Uncertainties in the initial conditions
	are significant, but our quantification approach is based on establishing
	reasonable bounds that are not unnecessarily conservative. To this
	end, we also make use of the ROAAM ideas of enveloping scenarios
	and ``splintering.`` Two causal relations (CRs) are used in this
	framework: CR1 is a model that calculates the peak pressure in the
	containment as a function of the initial conditions, and CR2 is a
	model that returns the frequency of containment failure as a function
	of pressure within the containment. Uncertainty in CR1 is accounted
	for by the use of two independently developed phenomenological models,
	the Convection Limited Containment Heating (CLCH) model and the Two-Cell
	Equilibrium (TCE) model, and by probabilistically distributing the
	key parameter in both, which is the ratio of the melt entrainment
	time to the system blowdown time constant. The two phenomenological
	models have been compared with an extensive database including recent
	integral simulations at two different physical scales. The containment
	load distributions do not intersect the containment strength (fragility)
	curve in any significant way, resulting in containment failure probabilities
	less than 10{sup {minus}3} for all scenarios considered. Sensitivity
	analyses did not show any areas of large sensitivity.},
  doi = {DOI: 10.2172/10106618},
  file = {Pilch1994.pdf:Pilch1994.pdf:PDF},
  keywords = {dch, unread},
  owner = {lewis.lloyd},
  timestamp = {2009.04.13},
  url = {http://www.osti.gov/servlets/purl/10106618-Cs15Q1/webviewable/}
}

@ARTICLE{Pilch1996,
  author = {M. M. Pilch and H. Yan and T. G. Theofanous},
  title = {The probability of containment failure by direct containment heating
	in Zion},
  journal = {Nuclear Engineering and Design},
  year = {1996},
  volume = {164},
  pages = {1 - 36},
  number = {1-3},
  abstract = {This paper is the first step in the resolution of the direct containment
	heating (DCH) issue for the Zion nuclear power plant using the risk
	oriented accident analysis methodology (ROAAM). This paper includes
	the definition of a probabilistic framework that decomposes the DCH
	problem into three probability density functions that reflect the
	most uncertain initial conditions (UO2 mass, zirconium oxidation
	fraction, and steel mass). Uncertainties in the initial conditions
	are significant, but our quantification approach is based on establishing
	reasonable bounds that are not unnecessarily conservative. To this
	end, we also make use of the ROAAM ideas of enveloping scenarios
	and [`]splintering'. Two causal relations (CRs) are used in this
	framework: CR1 is a model that calculates the peak pressure in the
	containment as a function of the initial conditions, and CR2 is a
	model that returns the frequency of containment failure as a function
	of pressure within the containment. Uncertainty in CR1 is accounted
	for by the use of two independently developed phenomenological models,
	the convection-limited containment heating model and the two-cell
	equilibrium model, and by probabilistically distributing the key
	parameter in both, which is the ratio of the melt entrainment time
	to the system blowdown time constant. The two phenomenological models
	have been compared with an extensive database including recent integral
	simulations at two different physical scales (1:10-scale in the Surtsey
	facility at Sandia National Laboratories and 1:40-scale in the COREXIT
	facility at Argonne National Laboratory). The loads predicted by
	these models were significantly lower than those from previous parametric
	calculations. The containment load distributions do not intersect
	the containment strength (fragility) curve in any significant way,
	resulting in containment failure probabilities less than 10-3 for
	all scenarios considered. Sensitivity analyses did not show any areas
	of large sensitivity. The feasibility of extrapolating containment
	loads distributions to most other pressurized water reactors is explored.},
  doi = {DOI: 10.1016/0029-5493(96)01227-7},
  file = {Pilch1996.pdf:Pilch1996.pdf:PDF},
  issn = {0029-5493},
  keywords = {dch, unread},
  owner = {lewis.lloyd},
  timestamp = {2009.04.13},
  url = {http://www.sciencedirect.com/science/article/B6V4D-3VTJ9YC-1/2/ede7908aa6a991cff5e2e5b2a7472b2c}
}

@ARTICLE{Pilkhwal2007,
  author = {D.S. Pilkhwal and W. Ambrosini and N. Forgione and P.K. Vijayan and
	D. Saha and J.C. Ferreri},
  title = {Analysis of the unstable behaviour of a single-phase natural circulation
	loop with one-dimensional and computational fluid-dynamic models},
  journal = {Annals of Nuclear Energy},
  year = {2007},
  volume = {34},
  pages = {339 - 355},
  number = {5},
  abstract = {This paper discusses the results obtained using one-dimensional and
	three-dimensional computational fluid-dynamic codes for the prediction
	of the dynamic behaviour observed in experiments carried out in a
	single-phase natural circulation apparatus. The loop is made of glass
	and is equipped with vertical and horizontal heaters and coolers
	that can be separately operated, thus obtaining different working
	configurations. An in-house program, capable of linear and non-linear
	stability analysis of one-dimensional loops, with arbitrary configurations
	of heat sources and sinks, and a transient thermal-hydraulics system
	code were adopted for the purpose of highlighting capabilities and
	limitations of one-dimensional models in predicting the involved
	phenomena. Both linear stability maps and transient flow evolutions
	have been calculated by the programs, obtaining information on both
	the linear and the non-linear dynamic behaviour of the addressed
	system, as predicted by one-dimensional, cross-section averaged balance
	equations. A computational fluid-dynamics code has been also adopted
	for simulating the system dynamics in the different loop configurations.
	In particular, the CFD code was effective in showing the origin of
	pulsating instabilities observed with horizontal heater and cooler,
	which could in no way be predicted by the one-dimensional models.},
  doi = {DOI: 10.1016/j.anucene.2007.01.012},
  file = {Pilkhwal2007.pdf:Pilkhwal2007.pdf:PDF},
  issn = {0306-4549},
  keywords = {critical.heat.flux; unread},
  owner = {lewis.j.lloyd},
  timestamp = {2009.10.06},
  url = {http://www.sciencedirect.com/science/article/B6V1R-4NC5V97-1/2/50e7aba28c592a7f1dc72bbc8fe89db7}
}

@PHDTHESIS{Prakash2007,
  author = {Arun Prakash},
  title = {Multi-Time-Step Domain Decomposition and Coupling Methods For Non-Linear
	Structural Dynamics},
  school = {University of Illinois at Urbana-Champaign},
  year = {2007},
  file = {Prakash2007.pdf:Prakash2007.pdf:PDF},
  owner = {llloyd},
  timestamp = {2012.11.03}
}

@ARTICLE{Ragusa2009,
  author = {J. C. Ragusa and V. S. Mahadevan},
  title = {Consistent and accurate schemes for coupled neutronics thermal-hydraulics
	reactor analysis},
  journal = {Nuclear Engineering and Design},
  year = {2009},
  volume = {239},
  pages = {566 - 579},
  number = {3},
  abstract = {Conventional coupling paradigms currently used to couple different
	physics components in reactor analysis problems can be inconsistent
	in their treatment of the nonlinear terms due to the operator-split
	(OS) strategies employed. This leads to the usage of small time steps
	to maintain accuracy requirements, thereby increasing the overall
	computational time. This paper proposes some remedies to OS techniques
	that can restore consistency in the coupling of the nonlinear terms
	and explores high-order mono-block nonlinearly consistent techniques
	with time step control. The performance of the methods was studied
	for several transient scenarios using a 0D point-kinetics/thermal-hydraulics
	lumped model and a 1D neutronics/heat conduction/enthalpy balance
	model. The results prove that consistent approximations can be made
	to enhance the overall accuracy in conventional codes with simple
	nonintrusive techniques. Additionally, an analysis of a mono-block
	coupling strategy (without having recourse to an OS strategy) is
	carried out to assess automated time stepping control using higher
	order Implicit Runge-Kutta (IRK) schemes. The conclusions from these
	results indicate that nonlinearly consistent adaptive time stepping
	methods can provide better accuracy and reliability in the solution
	fields than constant time stepping methods, even for transients with
	rapid and discontinuous variations.},
  doi = {DOI: 10.1016/j.nucengdes.2008.11.006},
  file = {Ragusa2009.pdf:Ragusa2009.pdf:PDF},
  issn = {0029-5493},
  owner = {Lewis John Lloyd},
  timestamp = {2009.12.16},
  url = {http://www.sciencedirect.com/science/article/B6V4D-4V9RHXT-1/2/6d28d0bb106fb63054efb25ab93319ca}
}

@ARTICLE{Ransom1984,
  author = {V. H. Ransom and D. L. Hicks},
  title = {Hyperbolic two-pressure models for two-phase flow},
  journal = {Journal of Computational Physics},
  year = {1984},
  volume = {53},
  pages = {124 - 151},
  number = {1},
  abstract = {For some time it has been known that many of the two-phase flow models
	lead to ill-posed Cauchy problems because they have complex characteristic
	values. A necessary condition (at least in the linear case) for the
	Cauchy problem to be well-posed is that it be stable in the sense
	of von Neumann. For systems of partial differential equations of
	first order, stability in the sense of von Neumann is essentially
	equivalent to the condition that the model be hyperbolic (all real
	characteristic values and complete set of characteristic vectors).
	Herein models are developed which have real characteristic values
	for all physically acceptable states (state space) and except for
	a set of measure zero have a complete set of characteristic vectors
	in state space. Therefore, these models are hyperbolic a.e. (almost
	everywhere) in state space. Also, they are stable in the sense of
	von Neumann a.e. in state space even without inclusion of viscosity
	terms. The models discussed herein are developed for the case of
	two-phase separated planar flow and include transverse momentum considerations.
	These models are referred to as “two-pressure� models because
	each phase is assumed to exist at an average pressure different from
	the average pressure in the other phase; the pressure fields are
	related through momentum considerations. Numerical results on a steady-state
	problem show good agreement with existing steady-state results. Numerical
	results on a transient problem agree with a single-pressure model
	until the onset of numerical instability in the single-pressure model.
	Compared to the single-pressure (hydrostatic) model, the two-pressure
	model approximates additional physical features and is shown to be
	a viable approach for the case of separated flow.},
  doi = {10.1016/0021-9991(84)90056-1},
  file = {Ransom1984.pdf:Ransom1984.pdf:PDF},
  issn = {0021-9991},
  owner = {llloyd},
  timestamp = {2012.10.20},
  url = {http://www.sciencedirect.com/science/article/pii/0021999184900561}
}

@ARTICLE{Ravenswaay2006,
  author = {Jan P. van Ravenswaay and Gideon P. Greyvenstein and Willem M.K.
	van Niekerk and Johan T. Labuschagne},
  title = {Verification and validation of the HTGR systems CFD code Flownex},
  journal = {Nuclear Engineering and Design},
  year = {2006},
  volume = {236},
  pages = {491 - 501},
  number = {5-6},
  note = {HTR-2004},
  abstract = {Regulatory requirements prescribe extensive verification and validation
	(V&V) of computer codes that are used in the design and analysis
	of accident conditions in nuclear plants. Flownex is a dynamic systems
	CFD code used as the primary thermal-fluid simulation code by the
	Pebble Bed Modular Reactor Company (PBMR). Stringent quality assurance
	processes have been implemented to ensure that the code conforms
	to the set standards. These processes include the comparison of Flownex
	with analytical results as well as with experimental data. The results
	of this process are summarized in this paper. Analytical solutions
	are used to verify Flownex's element models so as to ensure that
	the basic theory is correctly implemented in the computer code. As
	part of the analytical V&V effort various well-defined problems are
	solved using numerical methods implemented in independent computer
	codes. Comparison with experimental and plant data is a very important
	feature of the V&V program to validate that the chosen theory is
	fit for purpose. For this, validation data from the pebble bed micro
	model (PBMM) is used. In addition to the PBMM experimental data Flownex
	is compared to a number of small thermal-fluid experiments in which
	certain specific component phenomena is validated. These experiments
	were developed in collaboration with North-West University (previously
	Potchefstroom University).},
  doi = {DOI: 10.1016/j.nucengdes.2005.11.025},
  file = {Ravenswaay2006.pdf:Ravenswaay2006.pdf:PDF},
  issn = {0029-5493},
  keywords = {htgr, PBMR, CFD, unread},
  owner = {lewis.lloyd},
  timestamp = {2009.04.13},
  url = {http://www.sciencedirect.com/science/article/B6V4D-4J557GJ-9/2/63bd4b3e6da523b75847cb811cb3f283}
}

@TECHREPORT{Rodriguez2002,
  author = {S. B. Rodriguez},
  title = {Using the Coupled {MELCOR-RELAP5} Codes for Simulation of the {Edward’s
	Pipe}},
  institution = {Sandia National Laboratories},
  year = {2002},
  file = {Rodriguez2002.pdf:Rodriguez2002.pdf:PDF},
  owner = {llloyd},
  timestamp = {2012.11.17}
}

@BOOK{Saad2011,
  title = {Numerical Methods for Large Eigenvalue Problems},
  publisher = {Society for Industrial and Applied Mathematics},
  year = {2011},
  author = {Saad, Y.},
  edition = {2nd},
  file = {Saad2011.pdf:Saad2011.pdf:PDF},
  keywords = {Linear Algebra, Newton-Krylov, Mathematics},
  owner = {lloydlj},
  timestamp = {2011.08.03}
}

@BOOK{Saad2003,
  title = {Iterative Methods for Sparse Linear Systems},
  publisher = {Society for Industrial and Applied Mathematics},
  year = {2003},
  author = {Saad, Y.},
  edition = {2nd},
  comment = {pR},
  file = {Saad2003.pdf:Saad2003.pdf:PDF},
  keywords = {Newton-Krylov, Mathematics, Linear Algebra},
  owner = {lloydlj},
  timestamp = {2011.08.03}
}

@PHDTHESIS{Schubring2009,
  author = {Schubring, D.},
  title = {Behavior Innterrelationships in Annular Flow},
  school = {University of Wisconsin - Madison},
  year = {2009},
  type = {Dissertation},
  keywords = {Annular Flow, Two-Phase Flow},
  owner = {lloydlj},
  timestamp = {2011.08.03}
}

@TECHREPORT{Schultz2007,
  author = {Schultz, Richard R and Nigg, David W. and Johnson, Richard W. and
	Oh, Chang H. and Johnsen, Gary W. and Yoon, Woo F. and Herring, J.
	Steve and Yang, Won S. and Khalil, Hussein S. and Ougouag, Abderrafi
	M. and Gougar, Hans D. and Terry, William K. and McEligot, Donald
	M. and McCreery, Glenn E. and Sterbentz, James W. and Taiwo, Temitope
	A. and Pointer, William D. and Farmer. Mitchell T. and Feltus, Madeline
	A.},
  title = {Next Generation Nuclear Plant Methods Technical Program Plan},
  institution = {Idaho National Laboratory},
  year = {2007},
  type = {Methods Technical Program Plan},
  number = {INL/EXT-06-11804},
  address = {NGNP Methods Program, Idaho National Laboratory, P. O. Box 1625 2525
	Fremon, Idaho Falls, ID 83415-3890},
  month = {January},
  file = {Schultz2007.pdf:Schultz2007.pdf:PDF},
  keywords = {htgr, unread},
  owner = {lewis.lloyd},
  timestamp = {2009.04.14},
  url = {http://www.inl.gov/technicalpublications/Documents/3644017.pdf}
}

@TECHREPORT{Seban1979,
  author = {Seban, R. and Grief, R. and Abdollahian, D. and Peake, W.},
  title = {Comparison of Experimental and Predicted Heat Transfer for the Data
	of the UC-B Reflood Experiment},
  institution = {University of California at Berkeley},
  year = {1979},
  type = {Topical Report},
  number = {NP-1290, Research Project 248-1},
  month = {December},
  file = {Seban1979.pdf:Seban1979.pdf:PDF},
  keywords = {Reflooding, Boiling, Quenching, Droplet, Dispersed Flow, Quality,
	Validation, UCB, Two-Phase Flow},
  owner = {lloydlj},
  timestamp = {2011.08.03}
}

@TECHREPORT{Seban1978,
  author = {Seban, R. and Grief, R. and Yadigaroglu, G. and Elias, E. and Yu,
	K. and Abdollahian, D. and Peake, W.},
  title = {UC-B Reflod Program: Experimental Data Report},
  institution = {University of California at Berkeley},
  year = {1978},
  type = {Interim Report},
  number = {NP-743, Research Project 248-1},
  month = {April},
  file = {Seban1978.pdf:Seban1978.pdf:PDF},
  keywords = {Reflooding, Quenching, Flow Rates, Quality, Heat Transfer Coefficient,
	Pressure, UCB, Validation, Two-Phase Flow},
  owner = {lloydlj},
  timestamp = {2011.08.03}
}

@ARTICLE{Shin2005,
  author = {Seungwon Shin and S.I. Abdel-Khalik and Damir Juric},
  title = {Direct three-dimensional numerical simulation of nucleate boiling
	using the level contour reconstruction method},
  journal = {International Journal of Multiphase Flow},
  year = {2005},
  volume = {31},
  pages = {1231 - 1242},
  number = {10-11},
  doi = {DOI: 10.1016/j.ijmultiphaseflow.2005.06.005},
  file = {Shin2005.pdf:Shin2005.pdf:PDF},
  issn = {0301-9322},
  keywords = {Numerical simulation, critical.heat.flux; unread},
  owner = {lewis.j.lloyd},
  timestamp = {2009.10.06},
  url = {http://www.sciencedirect.com/science/article/B6V45-4GWJ8K9-2/2/ed2ab0853e82448c9dd04afe105c741a}
}

@ARTICLE{Shoji2004,
  author = {Masahiro Shoji},
  title = {Studies of boiling chaos: a review},
  journal = {International Journal of Heat and Mass Transfer},
  year = {2004},
  volume = {47},
  pages = {1105 - 1128},
  number = {6-7},
  abstract = {Research into boiling has been carried out over the last several decades
	and extensive data have accumulated from experimental studies carried
	out under various conditions and configurations, leading to the development
	of the currently available empirical and phenomenological correlations.
	However, most correlations apply to situations under a relatively
	narrow range of conditions and exhibit a considerable error band,
	even for the data sets on which they are based. In contrast to this
	multitude of correlations, the development of mechanistic models
	based on the underlying physical processes has been sporadic and
	limited. Thus, it is said that the use of boiling is not yet a #mature#
	technology, though it is one of our important technologies. This
	paper summarizes recent developments in research addressing the nonlinear
	dynamical behaviors of pool nucleate boiling for the purpose of suggesting
	some potential reasons why we have had limited success in mechanistic
	modeling, and to provide a promising perspective on future boiling
	research. In the introductory section, the problems facing current
	boiling research are described. In the following two sections, recent
	nonlinear experimental as well as theoretical research achievements
	are overviewed, while in the last section, the problems that remain
	unsolved are discussed, along with some concluding remarks.},
  doi = {DOI: 10.1016/j.ijheatmasstransfer.2003.09.024},
  file = {Shoji2004.pdf:Shoji2004.pdf:PDF},
  issn = {0017-9310},
  keywords = {Pool nucleate boiling, critical.heat.flux; unread},
  owner = {lewis.j.lloyd},
  timestamp = {2009.10.06},
  url = {http://www.sciencedirect.com/science/article/B6V3H-49XPT5W-7/2/71c89f268089bbe2875ba58e877eb0ac}
}

@TECHREPORT{Sjoberg1986,
  author = {Sjoberg, A. and Caraher, D.},
  title = {Assessment of RELAP5/MOD2 Against 25 Dryout Experiments Conducted
	at the Royal Institute of Technology},
  institution = {U.S. Nuclear Regulatory Commission},
  year = {1986},
  type = {International Agreement Report},
  number = {NUREG/IA-0009},
  address = {Swedish Nuclear Power Inspectorate
	
	S-102 52 Stockholm, Sweden},
  month = {October},
  file = {Sjoberg1986.pdf:Sjoberg1986.pdf:PDF},
  keywords = {RELAP, post chf heat transfer,},
  owner = {lloydlj},
  timestamp = {2011.08.03}
}

@PHDTHESIS{Soler-Martinez2011,
  author = {M. D. Soler-Martinez},
  title = {Semi-Implicit Thermal-Hydraulic Coupling of Advanced Subchannel and
	System Codes for Pressurized Water Reactor Transient Applications},
  school = {The Pennsylvania State University},
  year = {2011},
  file = {Soler-Martinez2011.pdf:Soler-Martinez2011.pdf:PDF},
  owner = {llloyd},
  timestamp = {2012.11.05}
}

@ARTICLE{Song2001,
  author = {J. H. Song and M. Ishii},
  title = {On the stability of a one-dimensional two-fluid model},
  journal = {Nuclear Engineering and Design},
  year = {2001},
  volume = {204},
  pages = {101 - 115},
  number = {1–3},
  abstract = {An analysis on the stability of the governing differential equations
	for area averaged one-dimensional two-fluid model is presented. The
	momentum flux parameters for gas and liquid are introduced to incorporate
	the effect of void fraction profiles and velocity profiles. The stability
	of the governing differential equations is determined in terms of
	gas and liquid momentum flux parameters. It is shown that the two-fluid
	model is well posed with certain restrictions on the liquid and gas
	momentum flux parameters. Simplified flow configurations for bubbly
	flow, slug flow, and annular flow are constructed to test the validity
	of proposed stability criteria. The momentum flux parameters are
	calculated for these flow configurations by assuming a power-law
	profile for both velocity and void fraction. Existing correlation
	for volumetric distribution parameter Co is used. By employing simplified
	velocity profiles, the void fraction profile is determined from Co
	correlation. It is found that the void fraction is wall-peaked at
	low void fraction and it becomes center-peaked as the void fraction
	increases. A simplified annular flow is also constructed. With these
	flow configurations, the momentum flux parameters are determined.
	It is shown that the calculated momentum flux parameters are located
	in the stable region above the analytically determined stability
	boundary. The analyses results indicate that the use of momentum
	flux parameter is promising, since they reflect flow structure and
	help to stabilize the governing differential equations.},
  doi = {10.1016/S0029-5493(00)00253-3},
  file = {Song2001.pdf:Song2001.pdf:PDF},
  issn = {0029-5493},
  keywords = {two-phase flow, two-phase numerics},
  owner = {llloyd},
  timestamp = {2012.10.20},
  url = {http://www.sciencedirect.com/science/article/pii/S0029549300002533}
}

@ARTICLE{Song2001a,
  author = {J. H. Song and M. Ishii},
  title = {The one-dimensional two-fluid model with momentum flux parameters},
  journal = {Nuclear Engineering and Design},
  year = {2001},
  volume = {205},
  pages = {145 - 158},
  number = {1–2},
  abstract = {A characteristic analysis for the governing differential equations
	for the compressible one-dimensional two-fluid model is presented.
	The momentum flux parameters for the gas and liquid phase are introduced
	to incorporate the effect of void fraction and velocity profiles.
	A characteristic equation is derived. Two roots of the equation determine
	the choked flow condition. The other two roots determine the stability
	of the differential equations in response to the short wave length
	disturbances. It is shown that the compressible one-dimensional two-fluid
	model is stable for the whole range of flow regime by use of appropriate
	momentum flux parameters. The choked flow condition is calculated
	and is compared with that of the conventional model.},
  doi = {10.1016/S0029-5493(00)00351-4},
  file = {Song2001b.pdf:Song2001b.pdf:PDF},
  issn = {0029-5493},
  keywords = {two-phase flow, two-phase numerics},
  owner = {llloyd},
  timestamp = {2012.10.20},
  url = {http://www.sciencedirect.com/science/article/pii/S0029549300003514}
}

@ARTICLE{Stempniewicz2008,
  author = {M.M. Stempniewicz and E.M.J. Komen and A. de With},
  title = {Model of particle resuspension in turbulent flows},
  journal = {Nuclear Engineering and Design},
  year = {2008},
  volume = {238},
  pages = {2943 - 2959},
  number = {11},
  note = {HTR-2006: 3rd International Topical Meeting on High Temperature Reactor
	Technology},
  abstract = {The graphite dust generated in an HTR/PBMR during normal reactor operation
	is deposited inside the primary system and becomes radioactive due
	to sorption of fission products. A significant amount of radioactive
	dust may be resuspended and released to the environment in case of
	LOCA. Therefore accurate particle resuspension models are required
	for HTR/PBMR safety analyses. Thermal-hydraulic safety analyses of
	HTR/PBMR type reactors are typically performed using computer codes
	such as FLOWNEX, MELCOR, or SPECTRA. None of these codes currently
	includes a well-tested mechanistic resuspension model. The resuspension
	model based on the Vainshtein model has been developed and implemented
	into the SPECTRA thermal-hydraulic system code. The resuspension
	model formulation has been extended in such way that other formulations,
	for example the Rock'n Roll model, may easily be defined and used
	within the general model framework. Several test calculations were
	performed, including comparisons of the numerical SPECTRA results
	with the analytical solutions obtained by means of MathCAD. Furthermore,
	comparisons with the experimental results of the Reeks and Hall,
	and STORM experiments were made. It was concluded that the model
	gives satisfactory results for a number of tests.},
  doi = {DOI: 10.1016/j.nucengdes.2007.11.024},
  file = {Stempniewicz2008.pdf:Stempniewicz2008.pdf:PDF},
  issn = {0029-5493},
  keywords = {particulate, unread},
  owner = {lewis.lloyd},
  timestamp = {2009.04.14},
  url = {http://www.sciencedirect.com/science/article/B6V4D-4S97JJS-2/2/d65468730d622fc072f80da3eb9a8177}
}

@ARTICLE{Stewart1981,
  author = {H. B. Stewart},
  title = {Fractional step methods for thermohydraulic calculation},
  journal = {Journal of Computational Physics},
  year = {1981},
  volume = {40},
  pages = {77 - 90},
  number = {1},
  abstract = {A method of fractional steps is used to extend a semi-implicit finite-difference
	technique for multidimensional two-phase flow calculations. This
	extension permits time step size to be chosen independent of convection
	velocities in one space direction, and still resolves multidimensional
	coupled sonic and phase exchange effects implicitly by forming a
	simple Poisson problem for pressure. Because pure time-splitting
	by physical phenomena was found unsuited to thermal hydraulics problems,
	a stabilizing corrections method was chosen. An application in nuclear
	reactor safety analysis is demonstrated.},
  doi = {10.1016/0021-9991(81)90200-X},
  file = {Stewart1981.pdf:Stewart1981.pdf:PDF},
  issn = {0021-9991},
  owner = {llloyd},
  timestamp = {2012.11.14},
  url = {http://www.sciencedirect.com/science/article/pii/002199918190200X}
}

@ARTICLE{Stewart1979,
  author = {H. B. Stewart},
  title = {Stability of two-phase flow calculation using two-fluid models},
  journal = {Journal of Computational Physics},
  year = {1979},
  volume = {33},
  pages = {259 - 270},
  number = {2},
  abstract = {Two-fluid modeling of two-phase flow may yield a system of partial
	differential equations having complex characteristics; this results
	in a mathematically ill-posed initial-value problem. Despite the
	fact that no finite-difference method for solving such a problem
	can be stable in the usual sense, finite-difference solution of two-fluid
	models is in widespread use. We investigate the numerical behavior
	of one such set of difference equations, derive conditions under
	which solutions appear to be well behaved, and offer a physical interpretation.},
  doi = {10.1016/0021-9991(79)90020-2},
  file = {Stewart1979.pdf:Stewart1979.pdf:PDF},
  issn = {0021-9991},
  keywords = {two-phase flow, two-phase numerics},
  owner = {llloyd},
  timestamp = {2012.10.20},
  url = {http://www.sciencedirect.com/science/article/pii/0021999179900202}
}

@ARTICLE{Stewart1984,
  author = {H. B. Stewart and B. Wendroff},
  title = {Two-phase flow: Models and methods},
  journal = {Journal of Computational Physics},
  year = {1984},
  volume = {56},
  pages = {363 - 409},
  number = {3},
  abstract = {A variety of two-phase flow models can be derived following a few
	basic principles, which are here illustrated which no more generality
	than is essential. Among the models derived is one already widely
	used in applications, even though it is ill-posed in the sense of
	Hadamard. Final assessment of such models remains a distant goal,
	but will clearly involve numerical solutions; several methods in
	current use are discussed with a guide to selecting the one appropriate
	to a particular problem.},
  doi = {10.1016/0021-9991(84)90103-7},
  file = {Stewart1984.pdf:Stewart1984.pdf:PDF},
  issn = {0021-9991},
  keywords = {two-phase flow, two-phase numerics},
  owner = {llloyd},
  timestamp = {2012.10.20},
  url = {http://www.sciencedirect.com/science/article/pii/0021999184901037}
}

@ARTICLE{Stoecker1999,
  author = {Bernd Stöcker and Antonio Hurtado and Stephan Struth},
  title = {Component exposure in hypothetical accidents with very fast depressurization
	in a HTR module reactor},
  journal = {Nuclear Engineering and Design},
  year = {1999},
  volume = {190},
  pages = {297 - 302},
  number = {3},
  abstract = {The starting event of the massive air ingress into the core of the
	HTR module reactor, classified as hypothetical incident, is the very
	fast depressurization of the primary circuit. Provided that the integrity
	of the reactor pressure vessel is not in question, a rupture of the
	connecting pressure vessel between reactor pressure vessel and steam
	generator vessel is the maximum possible leak cross-section. In this
	work it is investigated whether the components of the reactor pressure
	vessel are exposed by the depressurization process to mechanical
	loads which exceed the load limits. These loads are caused by two
	different events, the strong momentum change of the fluid and the
	local pressure differences, respectively. Due to the momentum change
	the bottom reflector receives the maximum load, whereby only 2% of
	the compressive strength of the graphite quality used there are reached.
	However, the load by local pressure differences is between passed
	volumes and in normal operation, not-passed volumes lead to high
	load values. A maximum pressure difference of 44.5 bar was calculated
	at the thermal top shield.},
  doi = {DOI: 10.1016/S0029-5493(99)00081-3},
  file = {Stoecker1999.pdf:Stoecker1999.pdf:PDF},
  issn = {0029-5493},
  keywords = {DLOFC, htgr, MHTGR, unread},
  owner = {lewis.lloyd},
  timestamp = {2009.04.13},
  url = {http://www.sciencedirect.com/science/article/B6V4D-3WNMTR4-4/2/7c2133573d313fa4812cf755c4242bcc}
}

@NUREG{Summers1994,
  author = {R. M. Summers and R. K. Cole and R. C. Smith and D.S. Stuart and
	S. L. Thompson and S. A. Hodge and C. R. Hyman and R. L. Sanders},
  booktitle = {Reference Manuals},
  volume = {2},
  title = {{MELCOR} Computer Code Manuals},
  institution = {Sandia National Laboratories},
  number = {NUREG/CR-6119
	
	SAND93-2185},
  year = {1994},
  organization = {U.S. Nuclear Regulatory Commission},
  file = {Summers1994.pdf:Summers1994.pdf:PDF},
  owner = {llloyd},
  timestamp = {2012.10.31}
}

@ARTICLE{Sussman2007,
  author = {M. Sussman and K.M. Smith and M.Y. Hussaini and M. Ohta and R. Zhi-Wei},
  title = {A sharp interface method for incompressible two-phase flows},
  journal = {Journal of Computational Physics},
  year = {2007},
  volume = {221},
  pages = {469 - 505},
  number = {2},
  abstract = {We present a sharp interface method for computing incompressible immiscible
	two-phase flows. It couples the level-set and volume-of-fluid techniques
	and retains their advantages while overcoming their weaknesses. It
	is stable and robust even for large density and viscosity ratios
	on the order of 1000 to 1. The numerical method is an extension of
	the second-order method presented by Sussman [M. Sussman, A second
	order coupled levelset and volume of fluid method for computing growth
	and collapse of vapor bubbles, Journal of Computational Physics 187
	(2003) 110-136] in which the previous method treated the gas pressure
	as spatially constant and the present method treats the gas as a
	second incompressible fluid. The new method yields solutions in the
	zero gas density limit which are comparable in accuracy to the method
	in which the gas pressure was treated as spatially constant. This
	improvement in accuracy allows one to compute accurate solutions
	on relatively coarse grids, thereby providing a speed-up over continuum
	or #ghost-fluid# methods.},
  doi = {DOI: 10.1016/j.jcp.2006.06.020},
  file = {Sussman2007.pdf:Sussman2007.pdf:PDF},
  issn = {0021-9991},
  keywords = {Incompressible flow, critical.heat.flux; unread},
  owner = {lewis.j.lloyd},
  timestamp = {2009.10.06},
  url = {http://www.sciencedirect.com/science/article/B6WHY-4KH4815-1/2/3b11bd605108db1573258994a3f5ab0e}
}

@ARTICLE{Syred2007,
  author = {Nick Syred and Katon Kurniawan and Tony Griffiths and Tom Gralton
	and Ruby Ray},
  title = {Development of fragmentation models for solid fuel combustion and
	gasification as subroutines for inclusion in CFD codes},
  journal = {Fuel},
  year = {2007},
  volume = {86},
  pages = {2221 - 2231},
  number = {14},
  note = {The 6th European Conference on Coal Research and its Applications},
  abstract = {This work arises from substantial problems found in the modelling
	of the gasification and combustion of solid fuel, both coal and biomass
	in the following different systems: - Coal fired non-slagging cyclone
	combustors. - Pre-calciners on cement plant. - Fuel rich inverted
	cyclone combustors used to simulate the time temperature history
	of large utility boilers. - Cyclonic gasifiers for sawdust and direct
	firing of small gas turbines. - Deposition studies for slagging and
	fouling in large utility boilers. - Prediction of final carbon in
	ash from pulverised coal systems. Commercial CFD codes such as Fluent
	are well developed and have well proven routines for lagrangian tracking
	of burning particles through complex flow fields. However what has
	become apparent in numerous studies is that existing models for solid
	fuel combustion can be adjusted to predict the initial flow field
	aerodynamics, sometimes the temperature, but fall down when particles
	have to be followed completely through a system. This is manifested
	with cyclone combustors and gasifiers via enhanced retention of burning
	particles in centrifugal force fields, which can only be resolved
	by changes in the particle size distribution and thus fragmentation
	as the particle gasify or burn. This problem also becomes apparent
	in studies of processes in pre-calciners and in deposition in large
	utility boilers and furnaces. The paper will review the literature
	of the fragmentation of pulverised coal and biomass during gasification,
	devolatilisation and combustion and relate it to observed phenomena
	in the type of system under consideration. The difficulties of incorporating
	models of fragmentation in CFD codes such as Fluent are discussed.
	Then the implementation of such a model in Fluent is described, together
	with results from a number of different systems. It is concluded
	that the model so implemented shows improved prediction in many difficult
	areas, but still needs development to better reflect actual fragmentation
	conditions in different experimental systems. It is concluded that
	the model needs to be further extended beyond the single step fragmentation
	at present used, especially to include many of the fine particles
	known to be generated in such systems.},
  doi = {DOI: 10.1016/j.fuel.2007.05.060},
  file = {Syred2007.pdf:Syred2007.pdf:PDF},
  issn = {0016-2361},
  keywords = {CFD, Dust, Graphite Combustion, Oxidation, unread, particulate},
  owner = {lewis.j.lloyd},
  timestamp = {2009.06.15},
  url = {http://www.sciencedirect.com/science/article/B6V3B-4P71D82-3/2/9ae602f8dd3889ce1478355ad7424f8b}
}

@ARTICLE{Takeda2004,
  author = {Tetsuaki Takeda},
  title = {Research and development on prevention of air ingress during the
	primary-pipe rupture accident in the HTTR},
  journal = {Nuclear Engineering and Design},
  year = {2004},
  volume = {233},
  pages = {197 - 209},
  number = {1-3},
  note = {Japan's HTTR},
  abstract = {A primary-pipe rupture accident is one of the design-basis accidents
	of a high-temperature gas-cooled reactor (HTGR). When the primary-pipe
	rupture accident occurs, air is expected to enter the reactor core
	from the breach and oxidize in-core graphite structures. In order
	to predict or analyze the process of air ingress during the primary-pipe
	rupture accident of HTGRs, it is very important to develop computer
	programs and to validate them by experiments. This paper describes
	a computer program, which is named FLOWGR, developed to analyze the
	process of air ingress during the first stage of the primary-pipe
	rupture accident, in the case of the primary-pipe attached to the
	bottom part of the reactor core. An overview of the method is given
	for the prevention of air ingress into the reactor during the primary-pipe
	rupture accident. The numerical results are in good agreement with
	the experimental ones regarding the density of the gas mixture, the
	concentration of each gas species produced by the graphite-oxidation
	reaction and the onset time of the natural circulation of air. The
	experimental results show that the natural circulation flow of air
	during the accident can be controlled by the method of helium gas
	injection into the reactor pressure vessel.},
  doi = {DOI: 10.1016/j.nucengdes.2004.08.009},
  file = {Takeda2004.pdf:Takeda2004.pdf:PDF},
  issn = {0029-5493},
  keywords = {htgr, unread},
  owner = {lewis.lloyd},
  timestamp = {2009.04.14},
  url = {http://www.sciencedirect.com/science/article/B6V4D-4DF4C5W-1/2/d802bdeac17778b0c2cbfdba1bcfec2a}
}

@ARTICLE{Takeda2000,
  author = {Tetsuaki Takeda and Makoto Hishida},
  title = {Study on the passive safe technology for the prevention of air ingress
	during the primary-pipe rupture accident of HTGR},
  journal = {Nuclear Engineering and Design},
  year = {2000},
  volume = {200},
  pages = {251 - 259},
  number = {1-2},
  abstract = {A primary-pipe rupture accident is one of the design-basis accidents
	of a high-temperature gas-cooled reactor (HTGR). When the primary-pipe
	rupture accident occurs, air is expected to enter the reactor core
	from the breach and oxidize in-core graphite structures. This study
	is to investigate the air ingress phenomena and to develop the passive
	safe technology for the prevention of air ingress and of graphite
	corrosion. This paper describes the method for the prevention of
	air ingress into the reactor during the primary-pipe rupture accident.
	It is found that a safe cooling rate of the reactor core exists for
	the prevention of air ingress. The experimental results show that
	the natural circulation flow of air during the accident can be controlled
	by the method of helium gas injection into the reactor pressure vessel.},
  doi = {DOI: 10.1016/S0029-5493(99)00338-6},
  file = {Takeda2000.pdf:Takeda2000.pdf:PDF},
  issn = {0029-5493},
  keywords = {Air Ingress, htgr, DLOFC, unread},
  owner = {lewis.lloyd},
  timestamp = {2009.04.13},
  url = {http://www.sciencedirect.com/science/article/B6V4D-40TR3C9-N/2/8d76331dbfdc884959c61ea2d9de9307}
}

@ARTICLE{Takeda1996,
  author = {Tetsuaki Takeda and Makoto Hishida},
  title = {Studies on molecular diffusion and natural convection in a multicomponent
	gas system},
  journal = {International Journal of Heat and Mass Transfer},
  year = {1996},
  volume = {39},
  pages = {527 - 536},
  number = {3},
  abstract = {Experimental and numerical studies have been carried out on the combined
	phenomena of molecular diffusion and natural convection with a graphite
	oxidation reaction in a multicomponent gas system to investigate
	the process of air ingress into a reverse U-shaped tube consisting
	of one side heated and the other side cooled pipes. The range of
	the Grashof number based on the height of the tube was about 3.7
	× 109 < GrL < 4.7 × 1011. One-dimensional basic equations for continuity,
	momentum conservation, energy conservation of the gas mixture, and
	mass conservation of the gas species are numerically solved to obtain
	concentration changes in the gas species and the onset time of the
	natural circulation of air. The experimental results showed that
	air entered the tube due to molecular diffusion and a very weak natural
	convection of the multicomponent gas mixed prior to the onset of
	the natural circulation of air. The calculated results are in good
	agreement with the experimental ones regarding the concentration
	changes in the gas species and the onset time of the natural circulation
	of air.},
  doi = {DOI: 10.1016/0017-9310(95)00148-3},
  file = {Takeda1996.pdf:Takeda1996.pdf:PDF},
  issn = {0017-9310},
  keywords = {htgr, Air Ingress, unread},
  owner = {lewis.lloyd},
  timestamp = {2009.04.13},
  url = {http://www.sciencedirect.com/science/article/B6V3H-3Y0RRX1-8/2/780078fee66531feadbbcda0102807a5}
}

@BOOK{Tannehill1997,
  title = {Computational Fluid Mechanics and Heat Transfer},
  publisher = {Taylor \& Francis Group},
  year = {1997},
  author = {J. C. Tannehill and D. A. Anderson and R. H. Pletcher},
  edition = {2nd},
  owner = {llloyd},
  timestamp = {2012.11.09}
}

@ARTICLE{Tauveron2008,
  author = {Nicolas Tauveron},
  title = {Plant control to avoid surge development in the case of a pipe rupture
	in a direct cycle HTGR},
  journal = {Nuclear Engineering and Design},
  year = {2008},
  volume = {238},
  pages = {2925 - 2934},
  number = {11},
  note = {HTR-2006: 3rd International Topical Meeting on High Temperature Reactor
	Technology},
  abstract = {This work concerns the design and safety analysis of a direct cycle
	gas cooled reactor and is concentrated on surge occurrence, development,
	consequences, and control in case of a pipe rupture. To describe
	these phenomena, a specific turbomachine model is used. The dynamic
	behaviour of the compressor and different anti-surge configurations
	are tested. The tendencies (role of actuators, sensors, gain, bandwidth,
	plenum size) are in a good agreement with bibliographical data. The
	application to transient simulations of gas cooled nuclear reactors
	(hypothetical 10-in. break accident) shows that deep surge has to
	be avoided to protect compressor, heat exchangers and other pipes,
	whenever possible. The effects of using the bypass line are pointed
	out. Different strategies for instrumentation and control deployment
	are studied and different levels of efficiency are obtained. The
	complete avoidance of any surge oscillation would require the use
	of multiple very fast valves with a dedicated control command.},
  doi = {DOI: 10.1016/j.nucengdes.2007.12.025},
  file = {Tauveron2008.pdf:Tauveron2008.pdf:PDF},
  issn = {0029-5493},
  keywords = {htgr, Thermal Hydraulics, unread},
  owner = {lewis.lloyd},
  timestamp = {2009.04.13},
  url = {http://www.sciencedirect.com/science/article/B6V4D-4SFHMHY-1/2/9f1f44d8a3a11ad34489c49574e0b1ee}
}

@TECHREPORT{RELAP,
  author = {{The RELAP5-3D Code Development Team}},
  title = {{RELAP5-3D} Code Manual Volume {I}: Code Structure, System Models,
	and Solution Methods},
  institution = {Idaho National Laboratory},
  year = {2012},
  type = {INEEL-EXT},
  number = {98-00834},
  address = {Idaho Falls, Idaho},
  month = {June},
  file = {RELAP.pdf:RELAP.pdf:PDF},
  keywords = {RELAP, two-phase numerics,},
  owner = {llloyd},
  timestamp = {2012.10.22}
}

@NUREG{Thurgood1983,
  author = {Thurgood, M. J. and Cuta, J. M. and Koontz, A. S. and Kelly, J.},
  booktitle = {COBRA/TRAC - A Thermal-Hydraulics Code for Transient Analysis of
	Nuclear Reactor Vessels and Primary Coolant Systems},
  volume = {3},
  title = {User's Manual},
  institution = {Pacific Northwest Laboratory},
  number = {NUREG/CR-3046
	
	PNL-4385
	
	R4},
  year = {1983},
  keywords = {COBRA, Systems Codes, Two-Phase Flow, Mathematics, two-phase numerics},
  file = {Thurgood1983b.pdf:Thurgood1983b.pdf:PDF},
  owner = {lloydlj},
  timestamp = {2011.09.15},
  type = {NUREG}
}

@NUREG{Thurgood1983a,
  author = {Thurgood, M. J. and George, T.L.},
  booktitle = {COBRA/TRAC - A Thermal-Hydraulics Code for Transient Analysis of
	Nuclear Reactor Vessels and Primary Coolant Systems},
  volume = {2},
  title = {COBRA/TRAC Numerical Solution Methods},
  institution = {Pacific Northwest Laboratory},
  number = {NUREG/CR-3046
	
	PNL-4385
	
	R4},
  year = {1983},
  keywords = {COBRA, Systems Codes, Two-Phase Flow, Mathematics, two-phase numerics},
  file = {Thurgood1983a.pdf:Thurgood1983a.pdf:PDF},
  owner = {lloydlj},
  timestamp = {2011.09.15},
  type = {NUREG}
}

@NUREG{Thurgood1983b,
  author = {Thurgood, M. J. and Guidotti, T. E. and Sly, G. A. and Kelly, J.
	and Kohrt, R. J. and Crowell, K. R. and Wilkins, C. A. and Cuta,
	J. M. and Bian, S. H.},
  booktitle = {Developmental Assessment and Data Comparisons},
  volume = {4},
  title = {{COBRA/TRAC} - A Thermal-Hydraulics Code for Transient Analysis of
	Nuclear Reactor Vessels and Primary Coolant Systems},
  institution = {Pacific Northwest Laboratory},
  number = {NUREG/CR-3046
	
	PNL-4385
	
	R4},
  year = {1983},
  keywords = {COBRA, Systems Codes, Two-Phase Flow, Mathematics},
  file = {Thurgood1983c.pdf:Thurgood1983c.pdf:PDF},
  owner = {lloydlj},
  timestamp = {2011.09.15},
  type = {NUREG}
}

@NUREG{Thurgood1983c,
  author = {M. J. Thurgood and J. Kelly and T. E. Guidotti and R. J. Kohrt and
	K. R. Crowell},
  booktitle = {Equations and Constitutive Models},
  volume = {1},
  title = {{COBRA/TRAC} - A Thermal-Hydraulics Code for Transient Analysis of
	Nuclear Reactor Vessels and Primary Coolant Systems},
  institution = {Pacific Northwest Laboratory},
  number = {NUREG/CR-3046
	
	PNL-4385
	
	R4},
  year = {1983},
  keywords = {COBRA, Systems Codes, Two-Phase Flow, Mathematics, two-phase numerics},
  file = {Thurgood1983.pdf:Thurgood1983.pdf:PDF},
  owner = {lloydlj},
  timestamp = {2011.09.15},
  type = {NUREG}
}

@BOOK{Todreas2011,
  title = {Nuclear Systems Volume I: Thermal Hydraulic Fundamentals},
  publisher = {Taylor \& Francis Group},
  year = {2011},
  author = {N. E. Todreas and M. S. Kazimi},
  address = {6000 Broken Sound Parkway NW, Suite 300
	
	Boca Raton, FL 33487-2742},
  edition = {2nd},
  owner = {llloyd},
  timestamp = {2012.10.30}
}

@ARTICLE{Torres2006,
  author = {David J. Torres and Mario F. Trujillo},
  title = {KIVA-4: An unstructured ALE code for compressible gas flow with sprays},
  journal = {Journal of Computational Physics},
  year = {2006},
  volume = {219},
  pages = {943 - 975},
  number = {2},
  abstract = {The KIVA family of codes was developed to simulate the thermal and
	fluid processes taking place inside an internal combustion engine.
	In this latest version of this open source code, KIVA-4, the numerics
	have been generalized to unstructrured meshes. This change required
	modifications to the Lagrangian phase of the computations, the pressure
	solution and fundamental changes in the fluxing schemes of the rezoning
	phase. This newest version of the code inherits all the droplet phase
	capabilities and physical sub-models of previous versions. The integration
	of the gas phase equations with moving solid boundaries continues
	to employ the successful arbitrary Lagrangian-Eulerian (ALE) methodology.
	Its new unstructured capability facilitates grid construction in
	complicated geometries and affords a higher degree of flexibility.
	The numerics of the code, emphasizing the new additions, are described.
	Various computational examples are performed demonstrating the new
	capabilities of the code.},
  doi = {DOI: 10.1016/j.jcp.2006.07.006},
  file = {Torres2006.pdf:Torres2006.pdf:PDF},
  issn = {0021-9991},
  keywords = {Unstructured meshes, KIVA, particulate, unread},
  owner = {lewis.lloyd},
  timestamp = {2009.04.28},
  url = {http://www.sciencedirect.com/science/article/B6WHY-4KSD8DG-1/2/b01cdb2f03f9fa5ce00f1f0a484f1e90}
}

@ARTICLE{Trapp1986,
  author = {J. A. Trapp and R. A. Riemke},
  title = {A nearly-implicit hydrodynamic numerical scheme for two-phase flows},
  journal = {Journal of Computational Physics},
  year = {1986},
  volume = {66},
  pages = {62 - 82},
  number = {1},
  abstract = {This article presents an extension of the semi-implicit numerical
	scheme for two-phase flow simulation. The extension uses an implicit
	evaluation of the convective fluxes and thus eliminates the material
	Courant stability restriction. The new algorithm involves a two-step
	approach for the mass and energy equations and a single fully-implicit
	step for the momentum evaluations. Some analysis and accuracy considerations
	for the scheme are also presented.},
  doi = {10.1016/0021-9991(86)90054-9},
  file = {Trapp1986.pdf:Trapp1986.pdf:PDF},
  issn = {0021-9991},
  keywords = {nearly-implicit, SETS, two-phase numerics},
  owner = {llloyd},
  timestamp = {2012.10.20},
  url = {http://www.sciencedirect.com/science/article/pii/0021999186900549}
}

@TECHREPORT{Tzanos2007,
  author = {Tzanos, C.P. and Farmer, M.T.},
  title = {Feasibility study for use of the natural convection shutdown heat
	removal test facility (NSTF) for VHTR water-cooled RCCS shutdown.},
  institution = {Argonne National Laboratory (ANL)},
  year = {2007},
  number = {ANL-GENIV-079, OSTI ID: 932941},
  doi = {DOI: 10.2172/932941},
  file = {Tzanos2007.pdf:Tzanos2007.pdf:PDF},
  keywords = {htgr, unread},
  owner = {lewis.lloyd},
  timestamp = {2009.04.13},
  url = {http://www.osti.gov/bridge/servlets/purl/932941-qm2qj2/932941.pdf}
}

@ARTICLE{Washington1996,
  author = {K. E. Washington and D. S. Stuart},
  title = {Comparison of and TCE calculations for direct containment heating
	of Surry},
  journal = {Nuclear Engineering and Design},
  year = {1996},
  volume = {164},
  pages = {201 - 210},
  number = {1-3},
  abstract = {This paper presents the results of several CONTAIN code calculations
	used to model direct containment heating (DCH) loads for the Surry
	plant. The results of these calculations are compared with the results
	obtained using the two-cell equilibrium (TCE) model for the same
	set of initial and boundary conditions. This comparison is important
	because both models have been favorably validated against the available
	DCH database, yet there are potentially important modeling differences.
	The comparisons are to quantitatively assess the impact of these
	differences. A major conclusion of this study is that, for the accident
	conditions studied and for a broad range of sensitivity cases, the
	peak pressures predicted by both TCE and are well below the failure
	pressure for the Surry containment.},
  doi = {DOI: 10.1016/0029-5493(96)01220-4},
  file = {Washington1996.pdf:Washington1996.pdf:PDF},
  issn = {0029-5493},
  keywords = {dch, unread},
  owner = {lewis.lloyd},
  timestamp = {2009.04.13},
  url = {http://www.sciencedirect.com/science/article/B6V4D-3VTJ9YC-9/2/3b2c46063827c9e13604ce5fe5e737f0}
}

@ARTICLE{Watanabe1996,
  author = {Takanobu Watanabe and Goung Jin Lee and Takashi Iseki and Mamoru
	Ishii},
  title = {A mechanistic model for the analysis of flashing phenomena},
  journal = {Annals of Nuclear Energy},
  year = {1996},
  volume = {23},
  pages = {801 - 811},
  number = {10},
  abstract = {A mechanistic model is developed to analyze the swelling phenomena
	caused by flashing in a tank. Up to now, in the case of flashing
	in a tank, there is no reliable mechanistic model available to predict
	the swelling level and the void fraction when the flashing occurs.
	In this paper, a mechanistic model is developed to predict the swelling
	level, the average void fraction, and the pressure transients in
	the case of blow down of steam from a tank. Both the equilibrium
	model and the non-equilibrium model are developed to analyze the
	flashing phenomena in a tank. Thus, in the equilibrium model, detailed
	knowledge of the gas bubble generation and growth is not needed.
	In the case of the non-equilibrium model, liquid can be superheated
	by depressurization, and the bubble generation rate is calculated
	as a function of superheat. By applying a Lagrangian approach, the
	bubble growth rate and the total volume of gas can be calculated.
	Computer programs for the equilibrium and the non-equilibrium models
	are developed. Results are compared with experimental data. For pressure
	and venting rate, the results of both the equilibrium model and the
	non-equilibrium model agree very well with the experimental data.
	The equilibrium model is not accurate in predicting swelling level
	and void fraction because the model does not use detailed information
	about bubble generation and growth. For swelling level and void fraction,
	results of the proposed non-equilibrium model agree well with the
	experimental data.},
  doi = {DOI: 10.1016/0306-4549(95)00063-1},
  file = {Watanabe1996.pdf:Watanabe1996.pdf:PDF},
  issn = {0306-4549},
  keywords = {flashing, unread},
  owner = {lewis.lloyd},
  timestamp = {2009.04.13},
  url = {http://www.sciencedirect.com/science/article/B6V1R-3VVTWWV-H/2/83ff7e4ec4e7ace36ba40c7fbfdbbd40}
}

@PHDTHESIS{Watson2010,
  author = {J. K. Watson},
  title = {Implicit Time-Integration Method for Simultaneous Solution of a Coupled
	Non-Linear System},
  school = {The Pennsylvania State University},
  year = {2010},
  type = {Dissertation},
  file = {Watson2010.pdf:Watson2010.pdf:PDF},
  owner = {llloyd},
  timestamp = {2012.11.12}
}

@ARTICLE{Weaver2002,
  author = {W. L. Weaver and D. L. Aumiller and E. T. Tomlinson},
  title = {A generic semi-implicit coupling methodology for use in {RELAP5-3D}},
  journal = {Nuclear Engineering and Design},
  year = {2002},
  volume = {211},
  pages = {13 - 26},
  number = {1},
  abstract = {A generic semi-implicit coupling methodology has been developed and
	implemented in the RELAP5-3D© computer program. This methodology
	allows RELAP5-3D© to be used with other computer programs to perform
	integrated analyses of nuclear power reactor systems and related
	experimental facilities. The coupling methodology potentially allows
	different programs to be used to model different portions of the
	system. The programs are chosen based on their capability to model
	the phenomena that are important in the simulation in the various
	portions of the system being considered and may use different numbers
	of conservation equations to model fluid flow in their respective
	solution domains. The methodology was demonstrated using a test case
	in which the test geometry was divided into two parts, each of which
	was solved as a RELAP5-3D© simulation. This test problem exercised
	all of the semi-implicit coupling features that were implemented
	in RELAP5-3D© The results of this verification test case show that
	the semi-implicit coupling methodology produces the same answer as
	the simulation of the test system as a single process.},
  doi = {10.1016/S0029-5493(01)00422-8},
  file = {Weaver2002.pdf:Weaver2002.pdf:PDF},
  issn = {0029-5493},
  keywords = {semi-implicit, two-phase numerics, coupling},
  owner = {llloyd},
  timestamp = {2012.10.20},
  url = {http://www.sciencedirect.com/science/article/pii/S0029549301004228}
}

@ARTICLE{Welch1998,
  author = {Samuel W. J. Welch},
  title = {Direct simulation of vapor bubble growth},
  journal = {International Journal of Heat and Mass Transfer},
  year = {1998},
  volume = {41},
  pages = {1655 - 1666},
  number = {12},
  abstract = {This paper presents a numerical method directed towards the local
	simulation of axisymmetric vapor bubble growth. We use an interface
	tracking method in conjunction with a finite volume method on a moving
	unstructured mesh. We allow metastable bulk states and assume the
	interface exists in thermal and chemical equilibrium. The bulk fluids
	are viscous, conducting, and compressible. The control volume continuity,
	momentum and energy equations are modified in the presence of a phase
	interface to include surface tension and discontinuous pressure and
	velocity. A solid wall model is included to allow for conjugate heat
	transfer modes.},
  doi = {DOI: 10.1016/S0017-9310(97)00285-8},
  file = {Welch1998.pdf:Welch1998.pdf:PDF},
  issn = {0017-9310},
  keywords = {critical.heat.flux; unread},
  owner = {lewis.j.lloyd},
  timestamp = {2009.10.06},
  url = {http://www.sciencedirect.com/science/article/B6V3H-3SYYKWW-4/2/9c960bbca7c8b5547937150effc945f7}
}

@PHDTHESIS{Wilson1999,
  author = {P. P. H. Wilson},
  title = {Analytic and Laplacian Adaptive Radioactivity Analysis},
  school = {University of Wisconsin - Madison},
  year = {1999},
  file = {Wilson1999.pdf:Wilson1999.pdf:PDF},
  owner = {llloyd},
  timestamp = {2012.11.28},
  url = {http://fti.neep.wisc.edu/pdf/fdm1098.pdf}
}

@ARTICLE{Wu1996a,
  author = {Q. Wu and S. Kim and M. Ishii and S. T. Revankar and R. Y. Lee},
  title = {High pressure simulation experiment on corium dispersion in direct
	containment heating},
  journal = {Nuclear Engineering and Design},
  year = {1996},
  volume = {164},
  pages = {257 - 269},
  number = {1-3},
  abstract = {Purdue 1/10 scale direct containment heating separate effects experiments
	under a reactor vessel pressure up to 14.2 MPa are presented. With
	the test facility scaled to the Zion PWR geometry, these tests are
	mainly focused on the corium dispersion phenomenon in order to obtain
	a better understanding of the dominant driving mechanisms. Water
	and woods metal have been used separately to simulate the core melt,
	the reactor vessel being pressurized with nitrogen gas analogous
	to the steam in the prototypic case. The entire test transient lasted
	for a few seconds, and the liquid dispersion in the test cavity occurred
	within only 0.5 s. To synchronize the data acquisition and blowdown
	transient, the test initiation was triggered by breaking two rupture
	discs in the liquid/gas delivery system. Parameters characterizing
	the liquid transport were obtained via various instruments. Important
	information about the mean size and size distribution of the dispersed
	droplets in the test cavity, the liquid film flow transient, the
	subcompartment trapping, and the liquid carry-over to the containment
	has been obtained. These results, along with data from a previous
	low pressure (1.4 MPa) experiment carried out at Purdue University,
	form a solid database for further theoretical analysis.},
  doi = {DOI: 10.1016/0029-5493(96)01222-8},
  file = {Wu1996a.pdf:Wu1996a.pdf:PDF},
  issn = {0029-5493},
  keywords = {dch, unread},
  owner = {lewis.lloyd},
  timestamp = {2009.04.13},
  url = {http://www.sciencedirect.com/science/article/B6V4D-3VTJ9YC-D/2/9325368e6c8a157e550584e6ce59923b}
}

@ARTICLE{Wu1996,
  author = {Q. Wu and G. Zhang and M. Ishii and R. Lee},
  title = {Modeling of corium dispersion in DCH accidents},
  journal = {Nuclear Engineering and Design},
  year = {1996},
  volume = {164},
  pages = {211 - 235},
  number = {1-3},
  abstract = {A model that governs the dispersion process in the direct containment
	heating (DCH) reactor accident scenario is developed by a stepwise
	approach. In this model, the whole transient is subdivided into four
	phases with an isothermal assumption. These are the liquid and gas
	discharge, the liquid film flow in the cavity before gas blowdown,
	the liquid and gas flow in the cavity with droplet entrainment, and
	the liquid transport and re-entrainment in the subcompartment. In
	each step, the dominant driving mechanisms are identified to construct
	the governing equations. By combining all the steps together, the
	corium dispersion information is obtained in detail. The key parameters
	are predicted quantitatively. These include the fraction of liquid
	that flows out of the cavity before gas blowdown, the dispersion
	fraction and the mean droplet diameter in the cavity, the cavity
	pressure rise due to the liquid friction force, and the dispersion
	fractions in the containment via different paths. Compared with the
	data of the 1:10 scale experiments carried out at Purdue University,
	fairly good agreement is obtained. A stand-alone prediction of the
	corium dispersion under prototypic Zion reactor conditions is carried
	out by assuming an isothermal process without chemical reactions.},
  doi = {DOI: 10.1016/0029-5493(96)01224-1},
  file = {Wu1996.pdf:Wu1996.pdf:PDF},
  issn = {0029-5493},
  keywords = {dch, unread},
  owner = {lewis.lloyd},
  timestamp = {2009.04.13},
  url = {http://www.sciencedirect.com/science/article/B6V4D-3VTJ9YC-B/2/d2f3de836b09d678927ae6b3b303ec21}
}

@ARTICLE{Xiaowei2004,
  author = {Luo Xiaowei and Robin Jean-Charles and Yu Suyuan},
  title = {Effect of temperature on graphite oxidation behavior},
  journal = {Nuclear Engineering and Design},
  year = {2004},
  volume = {227},
  pages = {273 - 280},
  number = {3},
  abstract = {The temperature dependence of oxidation behavior for the graphite
	IG-11, used in the HTR-10, was investigated by thermogravimetric
	analysis in the temperature range of 400-1200 °C. The oxidant was
	dry air (water content <2 ppm) with a flow rate of 20 ml/min. The
	oxidation time was 4 h. The oxidation results exhibited three regimes:
	in the 400-600 °C range, the activation energy was 158.56 kJ/mol
	and oxidation was controlled by chemical reaction; in the 600-800 °C
	range, the activation energy was 72.01 kJ/mol and oxidation kinetics
	were controlled by in-pore diffusion; when the temperature was over
	800 °C, the activation energy was very low and oxidation was controlled
	by the boundary layer. Due to CO production, the oxidation rate increased
	at high temperatures. The effect of burn-off on activation energy
	was also investigated. In the 600-800 °C range, the activation energy
	decreased with burn-off. Results of low temperature tests were very
	dispersible because the oxidation behavior at low temperatures is
	sensitive to inhomogeneous distribution of any impurity, and some
	impurities can catalyse graphite oxidation.},
  doi = {DOI: 10.1016/j.nucengdes.2003.11.004},
  file = {Xiaowei2004.pdf:Xiaowei2004.pdf:PDF},
  issn = {0029-5493},
  keywords = {graphite, unread},
  owner = {lewis.j.lloyd},
  timestamp = {2009.05.26},
  url = {http://www.sciencedirect.com/science/article/B6V4D-4BFVPJM-1/2/c6c4b12f30acb47ecb519c644d9c71f0}
}

@ARTICLE{Yadigaroglu1993,
  author = {G. Yadigaroglu and R.A. Nelson and V. Teschendorff and Y. Murao and
	J. Kelly and D. Bestion},
  title = {Modeling of reflooding},
  journal = {Nuclear Engineering and Design},
  year = {1993},
  volume = {145},
  pages = {1 - 35},
  number = {1–2},
  abstract = {The state of the art in modeling reflooding situations, mainly with
	the two-fluid system analysis codes, is reviewed; certain related
	general code development issues are included. Our current modeling
	of reflooding is reasonable and can be made sufficiently conservative
	for safety assessments, but it is not outstanding. Fundamental understanding
	of the detailed two-phase flow and heat transfer mechanisms has not
	progressed significantly over the state already available several
	years ago.
	
	The better understanding of system behavior achieved by the coordinated
	program of large-scale experiments is summarized and its impact on
	the modeling work discussed. In the future, factors such as the additional
	accident scenaria now considered, the new and advanced reactor types
	being analyzed, and the geometric growth of computing capacity are
	likely to drive our efforts. The new requirements and challenges
	can be met best by building into the codes pieces of understanding
	of the actual physical processes at the most fundamental level practicable.
	
	The discussion focuses on the existing codes and their successes and
	shortcomings; both certain specialized and the more complex general-purpose
	system codes are considered. The aim is not to conduct an exhaustive
	review of all aspects of the problem, but rather to reach consensus
	on certain issues.},
  doi = {10.1016/0029-5493(93)90056-F},
  file = {Yadigaroglu1993.pdf:Yadigaroglu1993.pdf:PDF},
  issn = {0029-5493},
  owner = {llloyd},
  timestamp = {2012.10.20},
  url = {http://www.sciencedirect.com/science/article/pii/002954939390056F}
}

@ARTICLE{Yan1996,
  author = {H. Yan and T. G. Theofanous},
  title = {The prediction of direct containment heating},
  journal = {Nuclear Engineering and Design},
  year = {1996},
  volume = {164},
  pages = {95 - 116},
  number = {1-3},
  abstract = {A simple analytical model is proposed and shown to capture the essence
	of the direct containment heating phenomenon. The model is based
	on assuming thermal/chemical equilibrium in the melt dispersal (flow)
	process, and separation of the melt out of this [`]equilibrium steam'
	in the intermediate compartment. The model reveals a natural scale
	(hence named the [`]DCH scale') for the DCH phenomenon, and the results
	are in very good agreement with the Integral Effects Tests series.
	On this basis, reactor predictions can be made quite simply, provided
	that the DCH scale for the particular condition of interest is known.
	This prediction of DCH scale is also addressed by a scaling approach
	that is shown to be consistent with the experimental data. Finally,
	reactor predictions (of DCH loads) are also included in generalized
	terms convenient for use under a wide variety of conditions. In general,
	the results appear to be well within the structural capability of
	large dry containments.},
  doi = {DOI: 10.1016/0029-5493(96)01226-5},
  file = {Yan1996.pdf:Yan1996.pdf:PDF},
  issn = {0029-5493},
  keywords = {dch, unread},
  owner = {lewis.lloyd},
  timestamp = {2009.04.13},
  url = {http://www.sciencedirect.com/science/article/B6V4D-3VTJ9YC-4/2/4398b65e1401def2c69dd0cf84a53d40}
}

@ARTICLE{Yang2003,
  author = {Jing Yang and Liejin Guo and Ximin Zhang},
  title = {A numerical simulation of pool boiling using CAS model},
  journal = {International Journal of Heat and Mass Transfer},
  year = {2003},
  volume = {46},
  pages = {4789 - 4797},
  number = {25},
  abstract = {This paper presents a new numerical model, called the CAS model, for
	boiling heat transfer. The CAS model is based on the cellular automata
	(CA) technique that is integrated into the popular SIMPLER algorithm
	for CFD problems. In the model, the CA technique deals with the microscopic
	nonlinear dynamic interactions of bubbles while the traditional CFD
	algorithm is used to determine macroscopic system parameters such
	as pressure and temperature. The popular SIMPLER algorithm is employed
	for the CFD treatment. The model is then employed to simulate a pool
	boiling process. The computational results show that the CAS model
	can reproduce most of the basic features of boiling and capture the
	fundamental characteristics of boiling phenomena. The heat transfer
	coefficient predicted by the CAS model is in excellent agreement
	with the experimental data and existing empirical correlations.},
  doi = {DOI: 10.1016/S0017-9310(03)00353-3},
  file = {Yang2003.pdf:Yang2003.pdf:PDF},
  issn = {0017-9310},
  keywords = {Numerical simulation, critical.heat.flux; unread},
  owner = {lewis.j.lloyd},
  timestamp = {2009.10.06},
  url = {http://www.sciencedirect.com/science/article/B6V3H-498TRFF-3/2/7bb12f7b3a70861862cc48d29f862dfe}
}

@ARTICLE{Yang2001,
  author = {Z. L. Yang and T. N. Dinh and R. R. Nourgaliev and B. R. Sehgal},
  title = {Numerical investigation of bubble growth and detachment by the lattice-Boltzmann
	method},
  journal = {International Journal of Heat and Mass Transfer},
  year = {2001},
  volume = {44},
  pages = {195 - 206},
  number = {1},
  abstract = {A numerical study has been performed to investigate the characteristics
	of bubble growth on, and detachment from, an orifice. The FlowLab
	code, which is based on a lattice-Boltzmann model of two-phase flows,
	was employed. Macroscopic properties, such as surface tension ([sigma])
	and contact angle ([beta]), were implemented through the fluid-fluid
	(G[sigma]) and fluid-solid (Gt) interaction potentials. The model
	was found to possess a linear relation between the macroscopic properties
	([sigma], [beta]) and microscopic parameters (G[sigma], Gt). The
	separate effects of the body force (gravity), gas injection rate,
	surface tension, and wettability were analyzed for both horizontal
	and vertical surfaces. It is shown that results of the lattice-Boltzmann
	modeling exhibit correct parametric dependencies of the departure
	diameter of bubbles generated on the horizontal surface on the above
	factors as previously established in experiments. For the case of
	bubble growth and departure on the vertical surface, the different
	effects of hydrodynamic parameters, except gas generation rate, were
	predicted.},
  doi = {DOI: 10.1016/S0017-9310(00)00101-0},
  file = {Yang2001.pdf:Yang2001.pdf:PDF},
  issn = {0017-9310},
  owner = {Lewis John Lloyd},
  timestamp = {2010.04.05},
  url = {http://www.sciencedirect.com/science/article/B6V3H-40RTK05-K/2/af0ef407d3813f9f83408e4b9d09902e}
}

@ARTICLE{Yoon2009,
  author = {H. Y. Yoon and I. K. Park and Y. J. Lee and J. J. Jeong},
  title = {An unstructured SMAC algorithm for thermal non-equilibrium two-phase
	flows},
  journal = {International Communications in Heat and Mass Transfer},
  year = {2009},
  volume = {36},
  pages = {16 - 24},
  number = {1},
  abstract = {The SMAC (Simplified Marker And Cell) algorithm is extended for an
	application to thermal non-equilibrium two-phase flows in light water
	nuclear reactors (LWRs). A two-fluid three-field model is adopted
	and a multi-dimensional unstructured grid is used for complicated
	geometries. The phase change and the time derivative terms appearing
	in the continuity equations are implemented implicitly in the pressure
	correction step. The energy equations are decoupled from the momentum
	equations for faster convergence. The verification of the present
	numerical method was carried out against a set of test problems which
	includes the single and the two-phase flows. The results are also
	compared to those of the semi-implicit ICE method, where the energy
	equations are coupled with the momentum equation for pressure correction.},
  doi = {DOI: 10.1016/j.icheatmasstransfer.2008.08.015},
  file = {Yoon2009.pdf:Yoon2009.pdf:PDF},
  issn = {0735-1933},
  keywords = {Numerical method, critical.heat.flux; unread},
  owner = {lewis.j.lloyd},
  timestamp = {2009.10.06},
  url = {http://www.sciencedirect.com/science/article/B6V3J-4TMRNW7-4/2/520979ec19acc53977b1baa900dd3b3a}
}

@ARTICLE{Zhang2009,
  author = {Zuoyi Zhang and Zongxin Wu and Dazhong Wang and Yuanhui Xu and Yuliang
	Sun and Fu Li and Yujie Dong},
  title = {Current status and technical description of Chinese 2 × 250 MWth
	HTR-PM demonstration plant},
  journal = {Nuclear Engineering and Design},
  year = {2009},
  volume = {239},
  pages = {1212 - 1219},
  number = {7},
  abstract = {After the nuclear accidents of Three Mile Island and Chernobyl the
	world nuclear community made great efforts to increase research on
	nuclear reactors and to develop advanced nuclear power plants with
	much improved safety features. Following the successful construction
	and a most gratifying operation of the 10 MWth high-temperature gas-cooled
	test reactor (HTR-10), the Institute of Nuclear and New Energy Technology
	(INET) of Tsinghua University has developed and designed an HTR demonstration
	plant, called the HTR-PM (high-temperature-reactor pebble-bed module).
	The design, having jointly been carried out with industry partners
	from China and in collaboration of experts worldwide, closely follows
	the design principles of the HTR-10. Due to intensive engineering
	and R&D efforts since 2001, the basic design of the HTR-PM has been
	finished while all main technical features have been fixed. A Preliminary
	Safety Analysis Report (PSAR) has been compiled. The HTR-PM plant
	will consist of two nuclear steam supply system (NSSS), so called
	modules, each one comprising of a single zone 250 MWth pebble-bed
	modular reactor and a steam generator. The two NSSS modules feed
	one steam turbine and generate an electric power of 210 MW. A pilot
	fuel production line will be built to fabricate 300,000 pebble fuel
	elements per year. This line is closely based on the technology of
	the HTR-10 fuel production line. The main goals of the project are
	two-fold. Firstly, the economic competitiveness of commercial HTR-PM
	plants shall be demonstrated. Secondly, it shall be shown that HTR-PM
	plants do not need accident management procedures and will not require
	any need for offsite emergency measures. According to the current
	schedule of the project the completion date of the demonstration
	plant will be around 2013. The reactor site has been evaluated and
	approved; the procurement of long-lead components has already been
	started. After the successful operation of the demonstration plant,
	commercial HTR-PM plants are expected to be built at the same site.
	These plants will comprise many NSSS modules and, correspondingly,
	a larger turbine.},
  doi = {DOI: 10.1016/j.nucengdes.2009.02.023},
  file = {Zhang2009.pdf:Zhang2009.pdf:PDF},
  issn = {0029-5493},
  keywords = {htgr, reviewed},
  owner = {lewis.j.lloyd},
  timestamp = {2009.06.18},
  url = {http://www.sciencedirect.com/science/article/B6V4D-4W1BFWS-1/2/14b98ff8bad804f6b9dc1a6c057f2fea}
}

@CONFERENCE{Zuying1993a,
  author = {Gao Zuying and Wang Chunyun},
  title = {Transient analysis of water ingress into the HTR-10 high temperature
	gas cooled test reactor},
  booktitle = {Technical committee meeting on response of fuel, fuel elements and
	gas cooled reactor cores under accidental air or water ingress conditions.},
  year = {1993},
  series = {IAEA-TECDOC--784},
  pages = {92 - 96},
  month = {October},
  organization = {International Atomic Energy Agency, Vienna (Austria)},
  abstract = {The high temperature gas-cooled test reactor HTR-10 is a graphite
	moderated and helium cooled reactor with a pebble bed core of spherical
	fuel elements. The primary system consists of reactor pressure vessel,
	steam generator vessel and hot gas duct vessel. The water ingress
	into HTR-10 reactor pressure vessel can occur for the accident of
	the rupture of the heat transfer tube of the steam generator. The
	mixture of helium and steam will enter the reactor with the stream.
	The transient behaviour of reactivity increase and graphite corrosion
	of both fuel elements and graphite structure caused by chemical reaction
	between graphite and steam are analyzed in order to evaluate the
	safety feature. The water ingress rate, reactivity increase, pressure
	and temperature transients in the primary system and graphite oxidation
	are analyzed by making use of the RETRAN code, THERMIX code and OXID-RE
	code. With the analytical results that the fuel element and graphite
	structure corrosion is not significant, the overpressurization of
	the reactor vessel is only for hypothetical accident.},
  file = {Zuying1993.pdf:Zuying1993.pdf:PDF},
  keywords = {Air Ingress, Dust, htgr, TRISO, DLOFC, reviewed},
  owner = {lewis.j.lloyd},
  timestamp = {2009.06.17}
}

@CONFERENCE{Zuying1993,
  author = {Gao Zuying and Wang Chunyun and Jiang Zhiqiang},
  title = {Transient analysis of air ingress from broken pipe into the HTR-10
	reactor pressure vessel},
  booktitle = {Technical committee meeting on response of fuel, fuel elements and
	gas cooled reactor cores under accidental air or water ingress conditions.},
  year = {1993},
  series = {IAEA-TECDOC--784},
  pages = {69 - 72},
  month = {October},
  organization = {International Atomic Energy Agency, Vienna (Austria)},
  abstract = {The high temperature gas-cooled test reactor HTR-10 is a modular pebble
	bed core reactor. The spherical fuel elements are loaded and discharged
	continuously from the top to the bottom of the reactor pressure vessel(RPV)
	in operation. The rupture of a fuel handling tube is classified as
	one of the design basis accidents of HTR-10 reactor. After the accident
	of the rupture of a fuel handling tube, the RPV depressurizes speedily
	to the atmosphere in 3 minutes. Following the cooled helium of the
	reactor, air enters into the RPV and a natural convection will be
	built up. A numerical simulation of graphite oxidation is carried
	out by the OXIDE-RE code to analyze the consequences of the accident.
	The graphite oxidate reaction produces carbon monoxide and carbon
	dioxide. The concentration distributions of carbon monoxide and carbon
	dioxide and the graphite structure in the reactor are analyzed. Because
	of the relative lower temperature in the core throughout this accident,
	the amount of dioxide and oxygen production can meet the safety requirements.
	It is impossible to initiate denotation in HTR-10 reactor.},
  file = {Zuying1993a.pdf:Zuying1993a.pdf:PDF},
  keywords = {Air Ingress, Dust, htgr, TRISO, DLOFC, reviewed},
  owner = {lewis.j.lloyd},
  timestamp = {2009.06.17}
}

@ARTICLE{Zuying2002,
  author = {Gao Zuying and Shi Lei},
  title = {Thermal hydraulic transient analysis of the HTR-10},
  journal = {Nuclear Engineering and Design},
  year = {2002},
  volume = {218},
  pages = {65 - 80},
  number = {1-3},
  abstract = {The HTR-10 is the first high temperature gas-cooled test module reactor
	built in China. In the accident analysis, typical design basis accident
	and beyond design basis accident (BDBA), including the reactivity
	accident, the loss of external power ATWS, the air ingress accident
	and water ingress accident into the primary system are selected and
	detailed analyzed. The results show that the HTR-10 has inherent
	safety properties. The maximum fuel element temperature will not
	exceed the limit value 1230 °C. The total amount of graphite corrosion
	maintains no more than 320 kg, the exposure ratio of first coated
	particles is less than 2.4% in the BDBA. The released radioactivity
	is limited at a low level and the ability of fuel particles to retain
	fission products is not corrupted. Even the consequence of the severest
	hypothetical accident has no safety danger to the reactor. The reactor
	can shut itself down via its negative temperature coefficient of
	reactivity in heat-up conditions.},
  doi = {DOI: 10.1016/S0029-5493(02)00199-1},
  file = {Zuying2002.pdf:Zuying2002.pdf:PDF},
  issn = {0029-5493},
  keywords = {Thermal Hydraulics, MHTGR, htgr, unread},
  owner = {lewis.lloyd},
  timestamp = {2009.04.13},
  url = {http://www.sciencedirect.com/science/article/B6V4D-4603WNN-2/2/405b1bf0c65c7f83a5b7a59308fbc4c2}
}

@MANUAL{CFR10,
  title = {{10 C. F. R. \S 50.46}},
  owner = {llloyd},
  timestamp = {2012.11.17}
}

@MANUAL{TRACE,
  title = {{TRACE V5.0} Theory Manual
	
	Field Equations, Solution Manual, and Physical Models},
  organization = {United States Nuclear Regulatory Commission},
  address = {Washington, DC},
  file = {TRACE.pdf:TRACE.pdf:PDF},
  keywords = {TRACE, two-phase numerics,},
  owner = {llloyd},
  publisher = {United States Nuclear Regulatory Commission},
  timestamp = {2012.10.22}
}

@OTHER{2001,
  address = {Nuclear Power Technology Development Section, International Atomic
	Energy Agency, Wagramer Strasse 5, P.O. Box 100, A-1400 Vienna, Austria},
  file = {IAEA_1198.pdf:IAEA_1198.pdf:PDF},
  howpublished = {Electronic Document},
  institution = {International Atomic Energy Agency},
  keywords = {htgr, unread},
  language = {English},
  month = {February},
  number = {1198},
  organization = {Nuclear Power Technology Development Section},
  owner = {lewis.lloyd},
  timestamp = {2009.04.14},
  title = {Current status and future development of modular high temperature
	gas cooled reactor technology},
  type = {Technical Document},
  url = {http://www-pub.iaea.org/MTCD/publications/PDF/te_1198_prn.pdf},
  year = {2001}
}

@comment{jabref-meta: selector_publisher:}

@comment{jabref-meta: protectedFlag:true;}

@comment{jabref-meta: selector_author:}

@comment{jabref-meta: fileDirectory:W:\\thesis\\references;}

@comment{jabref-meta: selector_journal:}

@comment{jabref-meta: selector_keywords:Air Ingress;CFD;DLOFC;Dust;Gra
phite Combustion;HTGR;MHTGR;Oxidation;PBMR;Reviewed;TRISO;Unread;}

@comment{jabref-meta: groupsversion:3;}

@comment{jabref-meta: groupstree:
0 AllEntriesGroup:;
1 ExplicitGroup:All\;0\;Benzi2002\;Knoll2000\;Knoll2004\;Kuran2006\;Mc
Hugh1995\;Park2009\;Saad2003\;Saad2011\;Watanabe1996\;Yadigaroglu1993\
;;
2 ExplicitGroup:Prelim\;0\;Aktas1996\;Aumiller2000\;Bandini2002\;Barre
1993\;Bestion1990\;Buschman2008\;Christensen1961\;Downar2001\;Dryja199
7\;Fatenejad2010\;Findlay1981\;Hwang2005\;Jeong2008\;Kadioglu2010\;Kaz
imi1980\;Keeys1971\;Kelley1995\;Lane2009\;Meholic2011\;Menck2002\;Prak
ash2007\;Ransom1984\;Seban1978\;Seban1979\;Sjoberg1986\;Soler-Martinez
2011\;Song2001\;Song2001a\;Stewart1979\;Watson2010\;Wilson1999\;Yoon20
09\;;
3 ExplicitGroup:Cited\;2\;Aumiller2001\;Aumiller2002\;Avramova2006\;Ba
rre1990\;Bestion2000\;CFR10\;Cai2002\;Cai2009\;Cai2011\;Chan1984\;Denn
is1996\;Deuflhard2004\;Drew1998\;Emonot2011\;Frepoli2003\;Geist1994\;I
shii1984\;Jeong1999\;Knoll2001\;Lanzkron1996\;LeVeque2002\;LeVeque2007
\;Li1999\;Liles1978\;Mahaffy1982\;Mahaffy1993\;Makihara2003\;Paraschiv
oiu1999\;Park2009a\;RELAP\;Ragusa2009\;Rodriguez2002\;Stewart1981\;Ste
wart1984\;Summers1994\;TRACE\;Tannehill1997\;Thurgood1983c\;Todreas201
1\;Trapp1986\;Weaver2002\;;
3 ExplicitGroup:Will not be cited\;2\;Koontz1983\;Thurgood1983\;Thurgo
od1983a\;Thurgood1983b\;;
2 ExplicitGroup:Unrelated\;0\;2001\;Ashwood2010\;Auvinen2008\;Ball2006
\;Ball2008\;Bertodano1996\;Binder1996\;Blanchat1996\;Blanchat1997\;Bri
nkmann2006\;Celnik2008\;Chunhe1993\;Cioni2006\;Dennis2006\;Densmore200
7\;Dhir2006\;Fleck1971\;Froehling1993\;Gabler2006\;Gibou2007\;Golobic2
004\;Hammersley1996\;Haque2006\;Haque2008\;He2001\;Hinssen2008\;Hogan2
006\;Iyoku1993\;Kadak2006\;Katanishi2007\;Kauffman1992\;Kunitomi2004\;
Lee1992\;Lignell2007\;Ma2009\;Mousseau2000\;Oh2004\;Pilch1994\;Pilch19
95\;Pilch1996\;Pilch1996a\;Pilch1996b\;Pilch1996c\;Pilch1996d\;Pilkhwa
l2007\;Ravenswaay2006\;Schubring2009\;Schultz2007\;Shin2005\;Shoji2004
\;Stempniewicz2008\;Stoecker1999\;Sussman2007\;Syred2007\;Takeda1996\;
Takeda2000\;Takeda2004\;Tauveron2008\;Torres2006\;Tzanos2007\;Washingt
on1996\;Welch1998\;Wu1996\;Wu1996a\;Xiaowei2004\;Yan1996\;Yang2001\;Ya
ng2003\;Zhang2009\;Zuying1993\;Zuying1993a\;Zuying2002\;;
}

@comment{jabref-entrytype: Nureg: req[author;booktitle;volume;title;institution;number;year] opt[organization;comment;keywords;month;series]}

