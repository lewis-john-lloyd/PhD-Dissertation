%=====================================================================
% main.tex
%=====================================================================
% This file contains:
%	- Document Class
%	- Packages
%	- Format Information
%	- Custom Commands
%	- Chapters
%	- Bibliography
%	- Appendices
%	- Curriculum Vita

%=====================================================================
% Document Style
%=====================================================================
% The margincheck option flags lines which overflow their hbox with a black
%  box at the end of the line.  This usually (but not always) indicates a
%  margin violation on the right margin.  Left margin violations aren't
%  indicated and if the margin violation is large enough, there isn't room
%  for the black box to be visiable.  

% A 12 Point UW PhD Thesis.
\documentclass[margincheck]{withesis}

%=====================================================================
% Packageges
%=====================================================================
% 	- inputenc
%		Purpos: fonts
%\usepackage[latin1]{inputenc}

% ifthen:
\usepackage{ifthen}
\newcommand{\StandAlone}{false}

% epsfig: The package epsfig is used to bring in the Encapsulated PostScript figures into the document.
\usepackage{epsfig}

% verbatim:
\usepackage{verbatim}

% pfg/TikZ: The package tikz is used to create graphics.
\usepackage{tikz}
\usetikzlibrary{shapes,arrows}

% times: The package times is used to change the fonts to Times Roman; however because the times typewriter font looks odd, the original LaTeX Computer Modern font is kept for the typewriter font using \renewcommand{\ttdefault}{cmtt}.
%	Times Roman is a PostScript font and therefore, the document cannot be correctly viewed from the *.dvi file. 
%	It should be converted to a *.ps file first and then viewed with a PostScript previewer.
\usepackage{times}
\renewcommand{\ttdefault}{cmtt}

% syntonly: The package syntonly checks the syntax of document without actually generating files.
\usepackage{syntonly}
% Activated by the \syntaxonly flag.
%\syntaxonly

% amsmath: The package amsmath includes most useful math stuff.
\usepackage{amsmath}

% IEEEtrantools: This package provides the IEEEeqnarray environment.
\usepackage{IEEEtrantools}

% algorithm: This package provides the algorithm environment.
\usepackage[ruled,chapter]{algorithm}

% algpseudocode: This package provides the pseudocode algorithm environment.
\usepackage{algpseudocode}

% pdflscape: This package allows for pages to be rotated within the document.
\usepackage{pdflscape}

% hyperref: The package hyperref allows for hyperlinks within the document.
\usepackage[bookmarks,pdfauthor={Lewis John Lloyd},colorlinks,citecolor=black,filecolor=black,linkcolor=black,urlcolor=black]{hyperref}

%========================================================================
%  Draft Control Commands
%========================================================================
%	- \psdraft
%		Purpose: 	Causes the \psfig or \epsfig commands to draw a box and label the box with the postscript file name instead of reading in the full postscript figure.  
%					This can save time and toner when printing drafts.
%\psdraft


% 	- \psfull
%		Purpose: Causes the inclusion of the postscript figures
%\psfull


%	- \pagestyle{thesisdraft}:
%		Purpose: Changes the header and footer
%		Options:
%			{thesisdraft} 	Casues the header and footer to be: DRAFT: Do Not Distribute        <time><Date>        <input file name>
%			{thesis}		Casues the header and footer to be the correct format
\pagestyle{thesis}

%  	- \draftmargins:
%		Purpose:	The page margins can be marked with a post-script box using the \draftmargins command.
%					This command uses dvips's end-of-page hook.
%					This is only visible in the *.ps file (NOT the *.dvi file)!
%\draftmargins


%	- \draftscreen:
%		Purpose:	The word ``DRAFT'' can be diagonally printed across the page using the \draftscreen command.
%					This command uses dvip's beginning-of-page hook.
%					This is only visible in the *.ps file (NOT the *.dvi file)!
%\draftscreen

%=======================================================================
% Include custom math commands
%=======================================================================
%============================================================================
% commands.tex
%============================================================================
% This file contains:
% 	- Defined Variables
%	- Redefined math shorthand
%	- Defined math shorthand

%============================================================================
% Defined Variables
%============================================================================
% 	- \abstractType:
%		Use: Toggles the type of abstract to be used
%		Default Value: abstract
%		Options: abstract, umiabstract
\newcommand{\abstractType}{abstract}

%============================================================================
% Redefined Math Commands
%============================================================================
% 	- \Vec{1} or \vec{1}
%		Long Name: Vector
%		Arguements[1]: bold and overbar arg1	
\DeclareRobustCommand{\Vec}[1]{%
    \ifmmode
        \mathbf{#1}\,%
    \else
        $\displaystyle \mathbf{#1}\,$%
    \fi
}
\DeclareRobustCommand{\vec}[1]{\Vec{#1}}

\DeclareRobustCommand{\lbm}{%
    \ifmmode
        \text{lb}_{\text{m}}
    \else
        $\displaystyle \text{lb}_{\text{m}}$%
    \fi
}
\DeclareRobustCommand{\lbf}{%
    \ifmmode
        \text{lb}_{\text{f}}
    \else
        $\displaystyle \text{lb}_{\text{f}}$
    \fi
}
\DeclareRobustCommand{\dt}{%
	\ifmmode
		\Delta t
	\else
		$Delta t$
	\fi
}
\DeclareRobustCommand{\dtmax}{%
	\ifmmode
		\Delta t_{\text{MAX}}
	\else
		$\Delta t_{\text{MAX}}$
	\fi
}
\DeclareRobustCommand{\dx}{%
	\ifmmode
		\Delta x
	\else
		$\Delta x$
	\fi
}

\delimitershortfall-1sp
\newcommand\abs[1]{\left|#1\right|}


%=======================================================================
% Include custom tikz commands
%=======================================================================
\tikzstyle{Decision} = [diamond, draw, text width=4.5em, text badly centered, node distance=3cm, inner sep=0pt]
\tikzstyle{Action} = [rectangle, draw,text width=5em, text centered, node distance=3cm, rounded corners, minimum height=0em]
\tikzstyle{NodePoint} = [circle, draw, minimum height = 0 em, node distance = 3 cm]
\tikzstyle{BlackBox} = [rectangle, draw, text centered, node distance=1cm, fill=black!10]
\tikzstyle{line} = [draw, -latex']
    

%=======================================================================
% Include custom variables based on ifthen commands
%=======================================================================
\DeclareRobustCommand{\BlackBox}{\State \textbf{Black Box: }}
\DeclareRobustCommand{\Test}{\State \textbf{Test: }}
\DeclareRobustCommand{\Define}{\State \textbf{Define: }}
\DeclareRobustCommand{\Update}{\State \textbf{Update: }}
\DeclareRobustCommand{\Set}{\State \textbf{Set: }}
\DeclareRobustCommand{\Calculate}{\State \textbf{Calculate: }}
%\newcommand{\algorithmicset}{\textbf{Set:}}
%\algnewcommand\Solve{\item[\algorithmicset]}

%=======================================================================
% Start Document
%=======================================================================
\begin{document}

%=======================================================================
% Bibliography Style
%=======================================================================
\bibliographystyle{plain}

%=======================================================================
% Chapters
%=======================================================================
%============================================================================
% Make the page numbers Roman (i, ii, etc)
%============================================================================
\clearpage\pagenumbering{roman}  

% ============================================================================
% Title Page
% ============================================================================
\title{Selective Spatial-Temporal Nonlinear Refinement for Thermal-Hydraulic Safety Analysis Codes}
\author{Lewis John Lloyd}
\date{2012}
\prelim
\department{Nuclear Engineering and Engineering Physics}
\advisorname{Michael Corradini}
\advisortitle{Professor}
\maketitle

%============================================================================
% Copyright Page
%============================================================================
\copyrightpage

%============================================================================
% Abstract
%============================================================================
\begin{\abstractType}
  %============================================================================
% abstract.tex
%============================================================================
The methods used to simulate the thermal-hydraulic behavior in the core of a nuclear power plant during postulated accidents can be characterized by the manner in which the governing conservation equations are discretized and the manner in which any nonlinearities present in those fully discrete equations are resolved.
While each method has a different way of approximating the governing equations, they all require that a discrete nonlinear problem be approximately solved at each timestep.
This is done either through a single Newton step or through an iterative Newton procedure.

The primary advantage of using a single Newton step is the low computational cost; however, the accuracy of a linear approximation in regions of highly non-linear physics may be suspect.
This has traditionally been mitigated by limitations placed upon the maximum change in independent parameters within a timestep.
Alternatively, by resolving the nonlinearities within a timestep through an iterative Newton solver, the errors from the linear approximation are reduced; however, the computational cost of global Newton methods is high.

For spatially isolable nonlinearities the computational expenditure of iteratively solving the global nonlinear problem may be unnecessary.
The objectives of this research include the design, implementation, and evaluation of a novel, spatially selective, nonlinear solution method for nuclear thermal-hydraulic safety analysis.
Isolation of subdomains where nonlinearities are high will be achieved by domain decomposition.
The method of decomposition chosen enables feedback across the subdomain boundaries. 
Upon isolation, the nonlinear subdomain will be subjected to a globalized Newton method to resolve the local nonlinearities.
The nonlinearly converged solution from the subdomain will then be communicated via coupling coefficients to the rest of the problem domain for use in calculating its single Newton step.
This unique use of selective nonlinear refinement via domain coupling may provide a route to nonlinearly converged timestep size insensitive solutions for traditional two-phase flow methods at a lower computational cost.
\end{\abstractType}

%============================================================================
% Acknowledgement Page
%============================================================================
\begin{acknowledgments}
  This research was performed under appointment to the Rickover Fellowship 
Program in Nuclear Engineering sponsored by Naval Reactors Division of the 
U.S. Department of Energy. 
\end{acknowledgments}

%============================================================================
% Auto-Generated Pages
%============================================================================
\tableofcontents
\listoftables
\listoffigures
\listofalgorithm

% ============================================================================
% Nomenclature Page
%============================================================================
%============================================================================
% nomenclature.tex
%============================================================================
% This file contains:
% 	- List of defined nomenclature items

%============================================================================
% Nomenclature List
%============================================================================

\begin{nomenclature}
\begin{longtable}{p{.13\textwidth} p{.67\textwidth} c} 
\NomItem{NPP}{Nuclear Power Plan}{}
%
\NomItem{NRC}{Nuclear Regulatory Commission}{}
%
\NomItem{10CFR}{Title 10 of the Code of Federal Regulations}{}
%
\NomItem{10CFR50}{Part 50 of 10CFR}{}
%
\NomItem{SAR}{Safety Analysis Report}{}
%
\NomItem{LWR}{Light Water Reactor}{}
%
\NomItem{ECCS}{Emergency Core Cooling System}{}
%
\NomItem{LOCA}{Loss-of-Coolant Accident}{}
%
\NomItem{PWR}{Pressurized Water Reactor}{}
%
\NomItem{BWR}{Boiling Water Reactors}{}
%
\NomItem{$L$}{Axial length of the problem}{
$\displaystyle \left[\, \text{ft} \,\right]$
}
%
\NomItem{$M$}{Number of sections in problem}{}
%
\NomItem{$S_{i}$}{Indexed section}{}
%
\NomItem{$J_{i}$}{Number of axial continuity volumes in section $i$}{}
%
\NomItem{$\Delta x_{i,j}$}{Indexed axial continuity volume length}{
$\displaystyle \left[\, \text{ft} \,\right]$
}
%
\NomItem{$T(i)$}{Number of channels in section $i$}{}
%
\NomItem{K}{Total number of channels in a problem}{}
%
\NomItem{$A_{c_{k,j}}$}{Cross-sectional area of indexed continuity volume}{
$\displaystyle \left[\, \text{ft}^2 \,\right]$
}
%
\NomItem{$A_{m_{k,j}}$}{Cross-sectional area of indexed momentum volume}{
$\displaystyle \left[\, \text{ft}^2 \,\right]$
}
%
\NomItem{$\omega_{k}$}{Volume of a geometric sub-component}{
$\displaystyle \left[\, \text{ft}^3 \,\right]$
}
%
\NomItem{$\Omega$}{Volume of whole domain}{
$\displaystyle \left[\, \text{ft}^3 \,\right]$
}
%
\NomItem{$\tilde{a}$}{Cross-sectionally averaged variable}{}
%
\NomItem{A}{Cross-sectional flow area}{
$\displaystyle \left[\, \text{ft}^2 \,\right]$
}
%
\NomItem{$S^{'''}$}{Volumetric inter-field source or sink of mass}{
$\displaystyle \left[\, \frac{ \lbm{} }{ \text{ft}^3 \text{s} } \,\right]$
}
%
\NomItem{$\Gamma^{'''}$}{Volumetric inter-phase source or sink of mass}{
$\displaystyle \left[\, \frac{ \lbm{} }{\text{ft}^3 \text{s} } \,\right]$
}
%
\NomItem{$s_{m,k}$}{Volumetric source or sink of mass for a given field or phase k}{
$\displaystyle \left[\, \frac{ \lbm{} }{\text{ft}^3 \text{s} } \,\right]$
}
%
\NomItem{$\eta$}{Apportionment factor}{
$\displaystyle \left[\, - \,\right]$
}
%
\NomItem{$\vec{u}_{k}$}{Velocity vector for field or phase k}{
$\displaystyle \left[\, \frac{\text{ft}}{\text{s}} \,\right]$
}
%
\NomItem{$\rho_{k}$}{Density of field or phase k}{
$\displaystyle \left[\, \frac{ \lbm{} }{\text{ft}^3} \,\right]$
}
%
\NomItem{$\alpha_{k}$}{Volume fraction of field or phase k}{
$\displaystyle \left[\, - \,\right]$
}
%
\NomItem{P}{Pressure}{
$\displaystyle \left[\, \text{psia} \, \right]$
}
%
\NomItem{$h_{k}$}{Enthalpy of field or phase k}{
$\displaystyle \left[\, \frac{\text{BTU}}{ \lbm } \,\right]$
}
%
\NomItem{$\Gamma^{'''} h^{'}_{k}$}{Volumetric energy transfer due to aqueous phase change}{
$\displaystyle \left[\, \frac{\text{BTU}}{\text{ft}^3 \text{s}} \,\right]$
}
%
\NomItem{$q^{'''}_{i,k}$}{Volumetric energy transfer between the saturated interface and any given field or phase k}{
$\displaystyle \left[\, \frac{\text{BTU}}{\text{ft}^3 \text{s}} \,\right]$
}
%
\NomItem{$q^{'''}_{g,l}$}{Volumetric energy transfer between the \ncg{} field and the liquid water phase}{
$\displaystyle \left[\, \frac{\text{BTU}}{\text{ft}^3 \text{s}} \,\right]$
}
%
\NomItem{$q^{'''}_{w,k}$}{Volumetric energy transfer between a solid structure and any given field or phase k}{
$\displaystyle \left[\, \frac{\text{BTU}}{\text{ft}^3 \text{s}} \,\right]$
}
%
\NomItem{$\alpha_k \frac{\delta P}{\delta t}$}{Pressure work done by a given field or phase k}{
$\displaystyle \left[\, \frac{\text{BTU}}{\text{ft}^3 \text{s}} \,\right]$
}
%
\NomItem{$s_{e,k}$}{External source or sink of volumetric energy for a given field or phase k}{
$\displaystyle \left[\, \frac{\text{BTU}}{\text{ft}^3 \text{s}} \,\right]$
}
%
\NomItem{$\vec{g}$}{Gravitational acceleration vector}{
$\displaystyle \left[\, \frac{\text{ft}}{\text{s}^2} \,\right]$
}
%
\NomItem{$\tau^{'}_{w,k}$}{Shear force from contact between the channel wall and field or phase k}{
$\displaystyle \left[\, \frac{ \lbf{} }{\text{ft}^3} \,\right]$
}
%
\NomItem{$\tau^{'}_{i,k_{1} k_{2}}$}{Shear force from contact between the field or phase $k_{1}$ and $k_{2}$}{
$\displaystyle \left[\, \frac{ \lbf{} }{ \text{ft}^3 } \,\right]$
}
%
\NomItem{$\Gamma^{'''} \vec{u}^{'}$}{Momentum transfer due to the exchange of mass between the aqueous phases}{
$\displaystyle \left[\, \frac{ \lbf{} }{ \text{ft}^3 } \,\right] $
}
%
\NomItem{$S^{'''} \vec{u}^{'}$}{Momentum transfer due to the exchange of mass between the two liquid fields}{
$\displaystyle \left[\, \frac{ \lbf{} }{ \text{ft}^3 } \,\right] $
}
%
\NomItem{$s_{p,k}$}{External source or sink of momentum for a given field or phase k}{
$\displaystyle \left[\, \frac{ \lbf{} }{ \text{ft}^3 } \,\right] $
}
%
\NomItem{$\vec{e}$}{Vector of conservation equations, except the temporal derivatives of the conserved quantities}{

}
%
\NomItem{$\vec{y}$}{Vector of conserved quantities}{

}
%
\NomItem{$V_{j}$}{Indexed volume}{
$\displaystyle \left[\, \text{ft}^3 \,\right]$
}
%
\NomItem{$\don{a}_{d,j \pm \onehalf }$}{Donored quantity}{

}
%
\NomItem{$\ave{a}_{a,j \pm \onehalf }$}{Average quantity}{

}
%
\NomItem{$\vec{E}$}{Vector of spatially discrete conservation equations}{

}
%
\NomItem{$\vec{x}$}{Vector of nine independent variables used by \cobra{}}{

}
%
\NomItem{$\dot{m}_{k}$}{Momentum of a given phase or field flowing through a cross-sectional area}{
$\displaystyle \left[\, \frac{ \lbm }{\text{s}} \,\right]$
}
%
\NomItem{$\vec{E}^{*}$}{Approximation of temporal integral of $\vec{E}(\vec{y}(\vec{x}))$}{

}
%
\NomItem{$\delta \vec{x}^{k}$}{Newton update vector}{

}
%
\NomItem{$\lambda_j $}{Linesearch scaling parameter}{
$\displaystyle \left[\, - \,\right]$
}
%
\NomItem{$\alpha $}{Linesearch termination parameter}{
$\displaystyle \left[\, - \,\right]$
}
%
\NomItem{$\vec{S} $}{Vector of scaling parameters}{

}
%
\NomItem{$\vec{F} $}{Vector of nonlinear residuals}{

}
\NomItem{$R $}{Integral of nonlinear residuals}{
$\displaystyle \left[\, - \,\right]$
}
\NomItem{$\tilde{R} $}{Time-averaged integral of nonlinear residuals}{
$\displaystyle \left[\, - \,\right]$
}
%
\NomItem{$\tilde{R}_{M} $}{Time-moment integral of nonlinear residuals}{
$\displaystyle \left[\, - \,\right]$
}
%
\NomItem{\dtmax{}}{Maximum timestep allowed for a simulation}{
$\displaystyle \left[\, \text{s} \,\right]$
}
%
\NomItem{\dt{}}{Timestep between time $t^{n}$ and $t^{n+1}$ for a simulation}{
$\displaystyle \left[\, \text{s} \,\right]$
}
%
\NomItem{$n^{n+1}_{g,j}$}{Flux of \ncg{} mass in RELAP5-3D}{
$\displaystyle \left[\, \frac{ \lbm{} }{\text{s}} \,\right]$
}
%
\NomItem{$m^{n+1}_{k,j}$}{Flux of mass for phase k in RELAP5-3D}{
$\displaystyle \left[\, \frac{ \lbm{} }{\text{s}} \,\right]$
}
%
\NomItem{$w^{n+1}_{k,j}$}{Volumetric flux for phase k in RELAP5-3D}{
$\displaystyle \left[\, \frac{ \text{ft}^3 }{\text{s}} \,\right]$
}
%
\NomItem{$u^{n+1}_{k,j}$}{Flux of internal energy for phase k in RELAP5-3D}{
$\displaystyle \left[\, \frac{ \text{BTU} }{\text{s}} \,\right]$
}
%
\LastNomItem{$t^{i}$}{Discrete points in time}{
$\displaystyle \left[\, \text{s} \,\right]$
}
\end{longtable}
\end{nomenclature}

%============================================================================
% Make the page numbers Arabic (1, 2, etc)
%============================================================================
\clearpage\pagenumbering{arabic}

\group{Introduction}

\subgroup{Motivation}
Of primary use in the field of nuclear reactor safety analysis is simulation.
The ability to predict the behaviour of reactors during off-normal events is the key to the licensing and the operation of nuclear power plants.
Within the United States, this simulation capacity is provided by a relatively small number of main stream software suites, among which are the RELAP variants, COBRA variants, and MELCOR.
While each of these software products varies in their models and implementations, the underlying numeric techniques and capabilities are similar.
Traditionally, these system codes utilize a semi-implicit discritization scheme.
In order to solve the resulting system of equations, the most common methodology is to take a single netwon step. 
The underlying numeric methods are first order methods.

% New Paragraph
The ability to use higher order methods is important.
Since the development of the semi-implicit method, whose form was motivated by the limited computer resources available at the time, there have been great advances in both the methodology used to solve linear algebra problems and computer capabilities.

\subgroup{Objectives}
The objective of this dissertation is the design, implementation, and evaluation of a practical non-linear solution framework for reactor safety systems codes.
Specifically, an efficient and reliable solution methodology to the two-phase, three-field, fluid-dynamics and the solid-structure heat transfer system of coupled non-linear partial differential equations is sought.
The specific methodology should be capable of obtaining a consistent solution to the system of PDEs while also possessing.

\begin{figure} %---------------------------------------------------------------
{\singlespace\tt\footnotesize}\caption{A Sample2 \LaTeX{} File}
\label{intro:sample1}
\end{figure}


\begin{figure} %---------------------------------------------------------------
{\singlespace\tt\footnotesize}\caption{A Sample2 \LaTeX{} File}
\label{intro:sample2}
\end{figure}         % Chapter: Introduction
\group{Fluid Mechanics}

Of primary concern in reactor safety analysis is thermal-hydraulic behaviour of the nuclear core during off-normal conditions. In order to evaluate 

\subgroup{Two Phase Flow}
The physical behaviour of fluids within COBRA are represented by a set of differential equations.
These governing equations are a collection of balance laws.
Balance laws are statements of conservation that include external sources and sinks.
The physical quantities being tracked by COBRA are mass, momentum, and energy.
Since COBRA models two phase behaviour, the


  % Chapter: Background
%============================================================================
% mathematics.tex
%============================================================================
% This file contains:
% 	- Hydrodynamic Conservation Equations
%	- Structural Thermal Energy Equations
%	- Newton's Method
%	- Jacobian-Free Methodology

%============================================================================
\group{Mathematical Formulation}
% Begin Chapter
%============================================================================
As mentioned earlier, most reactor safety-analysis codes depend upon three discrete sets of physics.
These include, but are not necessarily limited to, two-phase hydrodynamics, heat-transfer between the fluid and a solid structure, and a nuclear power source.
The different physical phenomena are represented by a system of PDEs and ODEs that constitute a set of balance laws for mass, momentum, and energy.
This system encompasses different scales in both space and time.
\pagebreak
%============================================================================
\subgroup{Hydrodynamic Conservation Equations}
% Hydro. Cons. Eqns.
%============================================================================
\begin{minipage}{0.42\textwidth}
\begin{center}\textbf{Conserved Variables}\end{center}
\begin{align}
\Vec{q} & = 
\begin{bmatrix} 
\alpha \rho_g  \\
\alpha \left( \rho_v + \rho_g \right)\\
\left(1-\alpha\right) \rho_l \\
\alpha_e \rho_l \\
\alpha \left( \rho_v + \rho_g \right) \Vec{U}_v \\
\left( 1-\alpha \right) \rho_l \Vec{U}_l \\
\alpha_e \rho_l \Vec{U}_e \\
\alpha \left( \rho_v H_v+ \rho_g H_g \right)\\
\left( 1 -\alpha \right) \rho_l H_l
\end{bmatrix} 
\end{align}
\end{minipage}
\begin{minipage}{0.42\textwidth}
\begin{center}\textbf{Independent Variables}\end{center}
\begin{align}
\Vec{x} & = \begin{bmatrix}
\alpha \\
\alpha_e \\
P \\
\alpha P_g \\
\alpha \left( \rho_v + \rho_g \right) \Vec{U}_v \\
\left( 1-\alpha \right) \rho_l \Vec{U}_l \\
\alpha_e  \rho_l \Vec{U}_e \\
\alpha H_v \\
\left( 1-\alpha \right) H_l
\end{bmatrix}
\end{align}
\end{minipage}

Note that $\displaystyle \Vec{q}=f(\Vec{x})$.

\subsubgroup{Mass}
\begin{align}
\Ddt{\alpha   \rho_v}{\Vec{U}_v} & =  \Gamma   + \DivOne{\Vec{G}_v^{T}}\\
\Ddt{\alpha   \rho_g}{\Vec{U}_v} & =  \Gamma   + \DivOne{\Vec{G}_g^{T}}\\
\Ddt{\alpha_l \rho_l}{\Vec{U}_l} & = -\Gamma_l + \DivOne{\Vec{G}_l^{T}} - S''' \\
\Ddt{\alpha_e \rho_l}{\Vec{U}_e} & = -\Gamma_e + S'''
\end{align}

\subsubgroup{Momentum}

\begin{align}
\Ddt{\alpha \rho_g \Vec{U}_g }{\Vec{U}_g} & = \nonumber \\
-\alpha\;\nabla P + \alpha \rho_g g - \tau^{'''}_{wv}-\tau^{'''}_{I_{lv}}-\tau^{'''}_{I_{ev}}+\Gamma_e U^{'}+\Div{\alpha T_g^{T}} & \\
\Ddt{\alpha_e \rho_l \Vec{U}_e}{\Vec{U}_e} & = \nonumber \\
-\alpha_e\;\nabla P + \alpha_e \rho_l g - \tau^{'''}_{wl}+\tau^{'''}_{ev}+\Gamma_e U^{'}+S^{'''}U^{'} & \\
\Ddt{\alpha_l \rho_l \Vec{U}_l}{\Vec{U}_l} & = \nonumber \\
-\alpha_l\;\nabla P + \alpha_l \rho_l g - \tau^{'''}_{wl}+\tau^{'''}_{lv}-\Gamma_l U^{'}-S^{'''}U^{'}+\Div{\alpha_l T_l^{T}} &
\end{align}

\begin{align}
\DisDdtOne{\dot{\Vec{m}}_{i,j}} & = \min(A_{j},A_{j+1})\frac{1}{2}\left(\frac{\dot{\Vec{m}}_{i,j}^{n}}{A_{j}}+\frac{\dot{\Vec{m}}_{i,j+1}^{n}}{A_{j+1}}\right)\hat{\Vec{U}}_{i,j-\frac{1}{2}}^{n} 
\end{align}

\subsubgroup{Energy}

\begin{align}
\Ddt{\alpha \rho_g H_g}{\Vec{U}_g} & \nonumber \\
= \Gamma H^{'}_{v}+q_{iv}+q_{gl}+Q_g^{'''}-\Div{\alpha \Vec{q}_g^{T}} &\\
\DerivPar{\left(1-\alpha \right)\rho_l H_l}{t}+\Div{\alpha_l \rho_l H_l \Vec{U}_l} +\Div{\alpha_e \rho_l H_l \Vec{U}_e} & = \nonumber \\
-\Gamma H^{'}_l + q_{il} - q_{gl} + Q_l^{'''}-\Div{\alpha_l \Vec{q}_l^{T}} &
\end{align}
%============================================================================
\subgroup{Discrete Hydrodynamic Equations}
% Hydro. Cons. Eqns.
\pagebreak
%============================================================================
\subsubgroup{Axial Momentum}

\begin{IEEEeqnarray}{rCl}
F_g(\Vec{x}^{n+1}) & = & \overbrace{E_g({\Vec{x}^{n}})}^{\text{Purely explicit terms}}-\overbrace{A_{mom,j}\Delta z_j\left[ \alpha_g^n\frac{P_{J+1}^{n+1}-P_{J}^{n+1}}{\Delta z_j}\right]}^{\text{Semi-Implicit Pressure Term}}  \\
&&-\overbrace{A_{mom,j}\Delta z_j\left[ \frac{dP}{dz}\bigg|_{w,g}^{*}+\frac{dP}{dz}\bigg|_{i,lg}^{*}+\frac{dP}{dz}\bigg|_{i,eg}^{*}\right]}^{\text{Semi-Implicit Drag Terms}}\nonumber \\
&& +\overbrace{\mathcal{S}p^{n+1}_{g,j}}^{\text{Implicit Source Term}} -\overbrace{\frac{\left(M_{g,j}^{n+1}-M_{g,j}^{n}\right)}{\Delta t}\Delta z}^{\text{Time rate of change}}\nonumber \\
& = &  0 \nonumber
\end{IEEEeqnarray}

The nonlinear functional associated with this equation is as follows:

\begin{IEEEeqnarray}{rCl}
\overbrace{\FVS{F}{\Vec{M}^{n+1},\;P^{n+1}}}^{\text{Nonlinear Functional}} & = & 0 \nonumber \\
& = & \underbrace{\FVS{E}{\Vec{M}^{n}}}_{\text{Explict Terms}} +\underbrace{\FVS{I}{\Vec{M}^{n+1},\;P^{n+1}}}_{\text{Implicit Terms}}\nonumber
\end{IEEEeqnarray}
\begin{IEEEeqnarray}{rCl}
\Vec{x}^{n+1} & = & \begin{bmatrix} M_l^{n+1}\\M_g^{n+1}\\M_e^{n+1}\\P^{n+1}\end{bmatrix} \nonumber \\
\FVS{F}{\Vec{x}^{n+1}} & = & \FVS{E}{\Vec{x}^n}+\FVS{I}{\Vec{x}^{n+1}} \nonumber
\end{IEEEeqnarray}

Now, this nonlinear functional is solved using a Newton Step.

\begin{IEEEeqnarray}{rCl}
\Vec{F}(\Vec{x}_{k+1}^{n+1}) = \Vec{F}(\Vec{x}_{k}^{n+1}) + \Mat{J}\cdot\Vec{\delta x}_{k} & = & 0\nonumber \\
\Vec{\delta x}_k & =&  \Vec{x}^{n+1}_{k+1}-\Vec{x}^{n+1}_{k} \nonumber \\
\Vec{F}(\Vec{x}_{k}^{n+1}) + \Mat{J} \cdot\Vec{\delta x}_{k} & = & 0 \nonumber \\
\Mat{J} \cdot\Vec{\delta x}_{k} & = & -\Vec{F}(\Vec{x}_{k}^{n+1}) \nonumber \\
\Mat{J}_{[:,1:3]} \cdot\Vec{\delta M}_{k}+\Vec{J}_{[:,4]}\cdot \Vec{\delta P}_{k} & = & -\Vec{F}(\Vec{x}_{k}^{n+1}) \nonumber
\end{IEEEeqnarray}

\begin{IEEEeqnarray}{rCl}
\Mat{J}_{[:,1:3]}\cdot [\Vec{M}^{n+1}_{k+1}-\Vec{M}^{n+1}_{k}] & = & -\Vec{F}(\Vec{x}^{n+1}_{k})-\Vec{J}_{[:,4]}\cdot \Vec{\delta P}_{k} \nonumber \\
\Mat{J}_{[:,1:3]}\cdot \Vec{M}^{n+1}_{k+1} & = & \Mat{J}_{[:,1:3]}\cdot \Vec{M}^{n+1}_{k} - \Vec{F}(\Vec{x}^{n+1}_{k})-\Vec{J}_{[:,4]}\cdot \Vec{\delta P}_{k}\nonumber \\
\Vec{M}^{n+1}_{k+1} & = & \Vec{M}^{n+1}_{k} - [\Mat{J}_{[:,1:3]}]^{-1}\cdot\left[\Vec{F}(\Vec{x}^{n+1}_{k})+\Vec{J}_{[:,4]}\cdot \Vec{\delta P}_{k}\right]\nonumber \\
\end{IEEEeqnarray}



\begin{align}
\Mat{J}_{[1:3,1:3]} & \equiv -\frac{2 A_{mom}\Delta t}{\Delta z}\cdot \nonumber \\
&\cdot \begin{bmatrix} 
K^{n}_{w,l} + \frac{K^{n}_{i,lg}}{\Ave{\alpha\rho}^{n}_{l}} +\frac{1}{2} &  -\frac{K^{n}_{i,lg}}{\Ave{\alpha\rho}^{n}_{g}} & 0\\
-\frac{K^{n}_{i,lg}}{\Ave{\alpha\rho}^{n}_{l}} &  K^{n}_{w,g} + \frac{K^{n}_{i,lg}}{\Ave{\alpha\rho}^{n}_{g}}+\frac{K^{n}_{i,eg}}{\Ave{\alpha\rho}^{n}_{g}} +\frac{1}{2} & -\frac{K^{n}_{i,eg}}{\Ave{\alpha\rho}^{n}_{e}}\\
0 & -\frac{K^{n}_{i,eg}}{\Ave{\alpha\rho}^{n}_{g}} &  K^{k}_{w,e} + \frac{K^{n}_{i,eg}}{\Ave{\alpha\rho}^{n}_{e}}+\frac{1}{2}\\
\end{bmatrix}
\end{align}

\begin{align}
\Mat{J}(\Vec{x}^{k})_{[1:3,4]} & \equiv \frac{A_{mom}\Delta t}{\Delta z} \begin{bmatrix} 
\Ens{\alpha^{n}_{l}}\\
\Ens{\alpha^{n}_{g}}\\
\Ens{\alpha^{n}_{e}}
\end{bmatrix}
\end{align}



\subsubgroup{Linearization}
The semi-implicit method, with a single Newton Step linearizes about the old time value and solves for a single delta. This has several ramifications. 
\subsubsubgroup{Wall Drag}
The beginning equation to formulate the wall drag is shown in Eqn.

\begin{IEEEeqnarray}{rCl}
\frac{dP}{dz}\bigg\vert_{\text{Fric},k} & = & \frac{f}{D_h}\frac{\Ave{\rho}_k U^2_k}{2}\\
\frac{dP}{dz}\bigg\vert_{Fric,k} & = & \frac{f}{D_h} \frac{M^2_k}{A^2_{mom}}\frac{1}{2\Ave{\rho}_k}\frac{1}{\Ave{\alpha}^2}\\
\frac{dP}{dz}\bigg\vert_{k} & = & \frac{f}{D_h}\frac{M^2_k}{A^2_{mom}}\frac{1}{2\Ave{\rho}_k}\frac{\alpha^2_b}{\Ave{\alpha}^2}\\ 
K_{wf,k} & = & \frac{A_{mom}}{M_k}\frac{dP}{dz}\bigg\vert_{k}\\
K_{wf,k} & = & \frac{A_{mom}}{M_k}\frac{f}{D_h}\frac{M^2_k}{A^2_{mom}}\frac{1}{2\Ave{\rho}_k}\frac{\alpha^2_b}{\Ave{\alpha}^2}\\
K_{wf,k} & = & \frac{f}{D_h}\frac{M_k}{A_{mom}}\frac{1}{2\Ave{\rho}_k}\frac{\alpha^2_b}{\Ave{\alpha}^2}
\end{IEEEeqnarray}

\begin{IEEEeqnarray}{rCl}
\frac{dP}{dz}\bigg\vert_{\text{Form},k} & = & \frac{f}{D_h}\frac{\Ave{\rho}_k U^2_k}{2}\\
\frac{dP}{dz}\bigg\vert_{w,k} & = & \frac{f}{D_h} \frac{M^2_k}{A^2_{mom}}\frac{1}{2\Ave{\alpha\rho}_k}\\
\frac{dP}{dz}\bigg\vert_{k} & = & \frac{f}{D_h}\frac{M^2_k}{A^2_{mom}}\frac{1}{2\Ave{\rho}_k}\frac{\alpha^2_b}{\Ave{\alpha}^2}\\ 
K_{wf,k} & = & \frac{A_{mom}}{M_k}\frac{dP}{dz}\bigg\vert_{k}\\
K_{wf,k} & = & \frac{A_{mom}}{M_k}\frac{f}{D_h}\frac{M^2_k}{A^2_{mom}}\frac{1}{2\Ave{\rho}_k}\frac{\alpha^2_b}{\Ave{\alpha}^2}\\
K_{wf,k} & = & \frac{f}{D_h}\frac{M_k}{A_{mom}}\frac{1}{2\Ave{\rho}_k}\frac{\alpha^2_b}{\Ave{\alpha}^2}
\end{IEEEeqnarray}

The Momentum Equations, outlined in Section \ref{blarp} contain an implicit wall drag. This means that we need to evaluate the wall drag at the future time value. To do this in the semi-implicit methodology requires that we linearize the future value using Newton's Method. The current implementation uses the following derivation.

\begin{eqnarray}
\frac{dP}{dz}\bigg\vert^{*}_{w,k} & = & \frac{dP}{dz}\bigg\vert^{n}_{w,k}+\frac{d (\frac{dP}{dz}\big\vert_{w,k})}{d M_k}\bigg\vert^{n}\;\delta M_k\\
\frac{d (\frac{dP}{dz}\big\vert_{w,k})}{d M_k} & = & \frac{d}{d M_k}\left[\left(\frac{f}{D_h}+\frac{K_{form}}{\Delta z_j}\right)\left(\frac{M^2_k}{A^2_{mom}}\right)\left(\frac{1}{2\rho_k}\right) \right]\\
& = & 2 \left(\frac{f}{D_h}+\frac{K_{form}}{\Delta z_j}\right)\left(\frac{M_k}{A_{mom}}\right)\left(\frac{1}{2\rho_k}\right)\\
& = & \left(\frac{2}{M_k}\right)\left(\frac{dP}{dz}\bigg\vert_{w,k}\right) \\
\frac{dP}{dz}\bigg\vert^{n+1}_{w,k} & = & \frac{dP}{dz}\bigg\vert^{n}_{w,k}\left(1+\frac{2 \delta M_k}{M_k}\right) \\
\frac{dP}{dz}\bigg\vert^{n+1}_{w,k} & = & \frac{1}{M_k}\frac{dP}{dz}\bigg\vert^{n}_{w,k}\left(M_k+2\delta M_k\right) \\
\frac{dP}{dz}\bigg\vert^{n+1}_{w,k} & = & K^{n}_{wz,k}\left(M_k+2\delta M_k\right) \\
\end{eqnarray}

Modifying this derivation to take into account the multiple Newton Steps results in the equations shown below. For brevity, the phase index, $_k$, is dropped from this derivation. Once the linearization starts, the index, $_k$, will refer to Newton iteration. 
\begin{eqnarray}
\label{NewPressureLoss}
\frac{dP}{dz}\bigg\vert_{w}& = & \left(\frac{f}{D_h}+\frac{K_{form}}{\Delta z_j}\right)\left(\frac{M^2}{A^2_{mom}}\right)\left(\frac{1}{2\rho}\right)\\
\frac{dP}{dz}\bigg\vert_{w} & = & f(M) \\
f|_{M^{n+1}_{k}} & = & f|_{M^{n+1}_{k-1}}+\frac{d\; f }{d M}\bigg\vert_{M_{k-1}^{n+1}}\;\delta M^{n+1}_{k-1}\\
\frac{d\; f }{d M}\bigg\vert_{M_{k-1}^{n+1}} & = & \frac{2}{M^{n+1}_{k-1}}\; f|_{M^{n+1}_{k-1}}\\
K^{n+1}_{wz,k-1} & = & \frac{f|_{M^{n+1}_{k-1}}}{M^{n+1}_{k-1}} \\
f|_{M^{n+1}_{k}} & = & [K^{n+1}_{wz,k-1}] M_{k-1}^{n+1}+2 \left[K^{n+1}_{wz,k-1} \right]\delta M^{n+1}_{k-1}
\end{eqnarray}

\begin{IEEEeqnarray}{rCl}
M^* & = & M_k \frac{\alpha_b}{\Ave{\alpha}}\\
G_k & = & \frac{M^*}{A_{mom}}=\frac{M_k \alpha_b}{A_{mom}\Ave{\alpha}}\\
\frac{dP}{dz}\bigg\vert_{k} & = & f \frac{1}{D_h}\frac{G_k^2}{2 \Ave{\rho}_k}\\ 
\frac{dP}{dz}\bigg\vert_{k} & = & f \frac{1}{D_h}\frac{1}{2 \Ave{\rho}_k}\frac{M^2_k \alpha^2_b}{A^2_{mom}\Ave{\alpha}^2}\\ 
K_i & = & \frac{1}{2} fi_{ia}\cdot coefd\cdot rvp\cdot urvl\cdot aintfh\cdot dxi\cdot \frac{1}{(\alpha ^*)^3}\\
alrl & = & A_{mom}\Ave{\alpha\rho}\\
C[2] & = & \frac{\Delta t}{\Delta x}\frac{K_i \Delta x}{A_{mom}\Ave{\alpha\rho}}
\end{IEEEeqnarray}



%============================================================================
\subgroup{Structural Thermal Energy Equations}
% Heat Structures
%============================================================================

%============================================================================
\subgroup{Newton's Method}
% Newton's Method
%============================================================================

COBRA currently uses a single linearized Newton step to solve the hydrodynamic equations.
This method has been shown to be adequate \ref{blarp} given a small enough time step.
One potential improvement that can be made to the current method is multiple Newton steps being taken with a frozen Jacobian.
The Jacobian will be evaluated at the old time value and not updated during the Newton process.


%============================================================================
\subgroup{Krylov Solvers}
% Krylov Solvers
%============================================================================
To solve a give newton iteraete, a Krylov subspace based method is used.
This methodology eliminates the requirement of analytically forming the Jacobian.


\begin{align}
\begin{bmatrix} 
h  \\
hu \\
\end{bmatrix}_t +\begin{bmatrix} 
uh  \\
h u^2 + \frac{1}{2}g h^2 \\
\end{bmatrix}_x & = 0
\end{align}


   % Chapter: Mathematical formulation
\group{Jacobian-Free Newton Krylov Overview}
In the Jacobian-Free Newton Krylov framework, the following procedure is taken:

%%---------------------------------------------------------------------------------------------------
%%            Discretization of Governing Equations
%%---------------------------------------------------------------------------------------------------
%
%\subgroup{Discretization of Governing Equations}
%The governing partial differential equations all have the following general form:
%
%\begin{equation}
%\DerivParOne{\Vec{u}}{t} = \mathcal{F}\left(\Vec{u},\;\DerivParOne{\Vec{u}}{\Vec{x}},\;\DerivParOne{^2\Vec{u}}{\Vec{x}^2}\right)
%\end{equation}
%
%The time derivative of the unknown vector function, \Vec{u}, is a functional of both itself and its derivatives.
%In practice, \Vec{u} is not solved for directly.
%Instead, \Vec{u} is parametrized by a different set of unknowns, \Vec{q}.
%By using either a discrete approximation to both the temporal and the spatial derivatives of \Vec{u}, a discretized form of the governing equations is obtained. 
%
%\begin{equation}
%\Vec{T}(\Vec{u}(\Vec{q})) = \Vec{S}(\Vec{u}(\Vec{q}))
%\end{equation}
%
%Where \Vec{T} and \Vec{S} represent the discrete temporal and the spatial approximation schemes, respectively.
%Any constants in the PDEs are group with \Vec{S}.

%---------------------------------------------------------------------------------------------------
%            Nonlinear Function
%---------------------------------------------------------------------------------------------------
\begin{algorithm}
\setlength{\baselineskip}{0.625\baselineskip}
\label{TransientLoop}
\caption{Transient Loop}
\begin{algorithmic}[1]
\Require $\Vec{x}^{0}$ and $t^{0}$
\Set $n = 0$
\Loop \; Take a Time Step
    \Set $\vec{x}^{n}$        
    \Calculate $\Delta t$ 
    \State $t^{n+1} : = t^{n} + \Delta t$
    \BlackBox Solve for $\vec{x}^{n+1}$ 
    \Test CCFL \Comment{Time-step Failure Mechanism (ccfl\_fail) }
    \BlackBox Interfacial Area Transport Equation
    \Calculate Courant Numbers 
\EndLoop{\;$n = n+1$}
\end{algorithmic}
\end{algorithm}

\begin{algorithm}
\setlength{\baselineskip}{0.625\baselineskip}
\label{ModifiedNewtonsAlgorithm}
\caption{Modified Newton's Method - Frozen Jacobian}
\begin{algorithmic}[1]
\Define $\vec{x}^{n}$
\Calculate Material Properties @ $\vec{x}^{n}$.
\Calculate Split/Merge Values @ $\vec{x}^{n}$.
\Calculate Velocities @ $\vec{x}^{n}$
\Calculate $\vec{E}_{mom}$
\Calculate $\vec{E}_{con}$
\Set $k = 0$
\BlackBox Predict $\vec{x}_{k} \equiv \vec{x}^{n+1}_{k}$ / Apply Implicit Boundary Conditions
\Loop \quad Take a Newton Step
  \Calculate $\vec{I}(\vec{x}_{k})_{mom}$
  \Calculate $\vec{I}(\vec{x}_{k})_{con}$
  \BlackBox Scale $\vec{F}(\vec{x}_k)$ and $\vec{J}(\vec{x}_{k})$.
  \BlackBox $\Vec{\delta x}_k = -\mathbf{J}^{-1}_0\cdot \mathbf{F}_k$ \Comment{Solver Choice}
  \BlackBox $\Vec{x}^{n+1}_{k+1}$ \Comment{Line-search or Trust Region}
  \Update Material Properties \Comment{UpdateVariables}
  \Test Material Properties
  \Update Junction Values \Comment{UpdateJunctions}
  \Update $\vec{u}^{n+1}_{k+1}$ \Comment{CalcAxialVelocity and CalcTransVelocity}
  \Update $\Vec{u}^{*}_{k+1}$ \Comment{CalcModAxialVelocity and CalcModTransVelocity}
  \Calculate Convergence Norms
  \If{ Converged}
    \State \bf{exit loop}
  \EndIf
\EndLoop \quad $k = k+1$
\end{algorithmic}
\end{algorithm}


\subgroup{Non-linear Function}

The vector function, \Vec{F}(\vec{x}), is the nonlinear residual of the discrete version of the governing PDEs.
For the physics of interest in this work, \Vec{F}(\Vec{x}) is a non-linear function.
The degree of non-linearity of \Vec{F}(\Vec{x}) depends upon the choice of numerical method used to distretize the governing PDEs.
%Since the independent parameters of \Vec{F} are the aforementioned $\Vec{x}_{n+1}$, the subscript $_{n+1}$ will be dropped.
For implementational reasons, the non-linear function is broken into two parts: the explicit and the implicit portions.
As shown in equation \eqref{ImplicitAndExplicit}, the explicit component, \Vec{E}, is independent of \Vec{x}, while the implicit component, \Vec{I}(\Vec{x}), is a function of \Vec{x}.

\begin{equation}
\label{ImplicitAndExplicit}
\Vec{F}(\Vec{x}) = \Vec{E} + \Vec{I}(\Vec{x})
\end{equation}

Solving the non-linear function depends upon finding a \vec{x} such that equation \eqref{NonlinearFunction} is satisfied.

\begin{equation}
\label{NonlinearFunction}
\Vec{F}(\Vec{x}) = 0
\end{equation}

%---------------------------------------------------------------------------------------------------
%            Newton's Method
%---------------------------------------------------------------------------------------------------

\subgroup{Newton's Method}
Newton's Method for solving the nonlinear system, Equation \eqref{NonlinearFunction}, is an iterative process involving a multidimensional Taylor Series expansion.
A first order Taylor series expansion of \Vec{F}(\Vec{x}) near $\Vec{x}_0$ is shown in equation \eqref{TaylorSeriesExpansion}.

\begin{equation}
\label{TaylorSeriesExpansion}
\Vec{F}(\Vec{x}) = \Vec{F}(\Vec{x}_0+\Vec{\delta x})= \Vec{F}(\Vec{x}_0)+\Mat{J}(\Vec{x}_0)\cdot\Vec{\delta x}+\mathcal{O}(\Vec{\delta x}^2)
\end{equation}

Let $\Vec{x}_n$ represent a vector of known values of independent parameters, with $\Vec{x}_0$ being the initial conditions of the problem.
Let $\Vec{\delta x}_n$ be the changes in $\Vec{x}_n$ that will bring you from $\Vec{x}_n$ to $\Vec{x}_{n+1}$, where $\Vec{x}_{n+1}$ is the future time value of independent parameters.

\begin{equation}
\label{DeltaX}
\Vec{x}_{n+1} = \Vec{x}_n + \Vec{\delta x}_n
\end{equation}

Note that \Mat{J}($\Vec{x}_0$) is the Jacobian of \Vec{F}(\Vec{x}) evaluated at $\Vec{x}_0$. 
Using \eqref{TaylorSeriesExpansion}, equation \eqref{NonlinearFunction} can be recast as a linear system where the unknown is \Vec{\delta x}.

\begin{equation}
\label{NewtonsMethod}
\Mat{J}(\Vec{x})\cdot\Vec{\delta x} = -\Vec{F}(\Vec{x})
\end{equation}

Solving for \Vec{\delta x} in \eqref{NewtonsMethod} constitutes one Newton step.

\begin{algorithm}
\setlength{\baselineskip}{0.625\baselineskip}
\caption{Newton's Method}
\begin{algorithmic}[1]
\Define $x_0$
\Define tolerance
\State $\gamma := 1$
\State $k := 0$
\While{$ \gamma \leq \text{tolerance}$}\Comment{ Test convergence of non-linear step.}
    \State $F^k := F(x^k)$ \Comment{Evaluate non-linear function at $x^k$.}
    \State $J^k := J(x^k)$ \Comment{Evaluate Jacobian matrix at $x^k$.}
    \State $\delta x^k := J^{-k} \cdot F^k$ \Comment{Solve for $\delta x$ by applying the inverse of $J^k$ to $F^k$.}
    \State $x^{k+1} := x^k + \delta x^k$ \Comment{Calculate $x^{k+1}$.}
    \State $\gamma := \text{min}\left(\frac{\norm{J \delta x}{2}}{\norm{F}{2}},\frac{\norm{\delta x}{2}}{\norm{x}{2}}\right)$ \Comment{Calculate convergence criteria.}
    \State $k := k + 1$ \Comment{Increment index.}
\EndWhile
\end{algorithmic}
\end{algorithm}
%---------------------------------------------------------------------------------------------------
%            Krylov Solver
%---------------------------------------------------------------------------------------------------

\subsubgroup{Krylov Solver}

Once \eqref{NewtonsMethod} is solved for $\Vec{\delta x}$, a new \Vec{x} is obtain by equation \eqref{DeltaX}. in order to update $\Vec{x}_k$ to $\Vec{x}_{k+1}$. 
To solve this linear system, a Krylov subspace method is used, in particular GMRES.
We will first drop the $_k$ and $_{k+1}$ subscripts because we will be using $_j$ and $_{j+1}$ to denote the Krylov (inner) iteration count.

First, an initial guess for $\Vec{\delta x}$ is chosen, $\Vec{\delta x}_0$.
In the literature, this initial guess is often zero for a transient simulation; the assertion is that $\Vec{\delta x}$ should be small within a time step.
Allowing for a nonzero initial guess, define the residual:

\begin{equation}
\Vec{r}_0 = -\Vec{F}(\Vec{x}_k)-\Mat{J}(\Vec{x}_k)\cdot\Vec{\delta x}_0
\end{equation}

We now normalize the residual, $\displaystyle \Vec{v}_1 = \frac{\Vec{r}_0}{\norm{\Vec{r}_0}{2}}$.
This vector will serve as our first basis vector (we need to start somewhere).
We then enter into an iterative process starting with j = 1.
Now compute the Jacobian-vector product, $\Mat{J}(\Vec{x}_k)\cdot\Vec{v}_j$.
We do this using a variant of Equation \eqref{TaylorSeriesExpansion}.

\begin{eqnarray}
\label{JacobianFreeExpansion}
\Vec{F}(\Vec{x}_k + \epsilon \Vec{v}_j) & = & \Vec{F}(\Vec{x}_k + \epsilon \; \Mat{J}(\Vec{x}_k) \cdot \Vec{v}_j + \mathcal{O}((\epsilon \Vec{v}_j)^2) \\
\label{JacobianFreeApproximation}
\Mat{J} (\Vec{x}_k) \cdot \Vec{v}_j & = & \frac{\Vec{F}(\Vec{x}_k + \epsilon \Vec{v}_j) - \Vec{F}(\Vec{x}_k)}{\epsilon}
\end{eqnarray}

It is apparent that each Jacobian-vector product requires the evaluation of \Vec{F}(\Vec{x}) at a slightly perturbed value from the base state. 
The epsilon is a small value that is one of the tweaks that can affect convergence performance.
However, traditionally this is near machine round-off.


The vector resulting from the Jacobian-Vector product, $\Vec{w}_j$, is then put through a Gram-Schmidt orthogonalization with all previous Krylov vectors $\Vec{v}_1,\;\Vec{v}_2,\;\dots,\;\Vec{v}_{j-1},\;\Vec{v}_j$.
There are alternative options for this point in the algorithm (such as the Householder variation of the Arnoldi process) - these are additional options that you can change in PETSc.
Part of this process (the inner-product with previous Kyrlov vectors) generates entries for the $j^{th}$ column of a matrix, \textbf{H}.
The projections of each previous Krylov vector onto $\Vec{w}_j$ are subtracted from $\Vec{w}_j$ (this is Gram-Schmidt). 
This new Krylov vector is determined by normalizing $\Vec{w}_j$, $\displaystyle \Vec{v}_{j+1} = \frac{\Vec{w}_j}{\norm{\Vec{w}_j}{2}}$.
\textbf{Side Note:} the norm of $\Vec{w}_j$ is the $h_{j+1,j}$ entry in the matrix $\mathbf{H}$.

The stopping criteria for this process is related to the norm of the latest residual, Equation \eqref{KrylovResidual}, and the norm of $-\Vec{F}(\Vec{x}_k)$.
The exact relation is determined by three tunable knobs in PETSc.
\begin{equation}
\label{KrylovResidual}
\norm{\Vec{r}_j}{2} = \norm{-\Vec{F}-\Vec{J}\cdot\Vec{\delta x}_j}{2}
\end{equation}
\textbf{NOTE:} The residual in Equation \eqref{KrylovResidual} is generated independently of constructing $\Vec{\delta x}_j$ (the final answer), which is only done at the end of the GMRES process.

After the process has stopped, a least square process is applied using $\Vec{H}$ to determine the coefficients for a linear combination of $\Vec{v}$ that will result in a $\Vec{\delta x}$ that minimizes $\norm{-\Vec{F}(\Vec{x}_k)-\Mat{J}(\Vec{x_k})\cdot\Vec{\delta x}}{2}$.
That concludes the Krlov (inner) iteration.

This $\Vec{\delta x}_k$ then used to computer $\Vec{x}_{k+1}$, which is then used in a line search (or trust region) globalization step.
If good enough (PETSc options), then accept $\Vec{\delta x}_k$ and $\Vec{x}_{k+1}$ as $\Vec{\delta x}_n$ and $\Vec{x}_{n+1}$.
This ends the Newton (outer) iteration.

Take next time step.


%---------------------------------------------------------------------------------------------------
%            Globalization Strategies
%---------------------------------------------------------------------------------------------------

\subsubgroup{Globalization Strategies}

\subsubsubgroup{Line Search}

\subsubsubgroup{Trust Region}
          % Chapter: Jacobian Free Newton Krylov Overview

%=======================================================================
% Bibliography
%=======================================================================
%\bibliography{references}    

%=======================================================================
% Appendices
%=======================================================================
\appendix
\begin{appendices}
% code.tex
% this file is part of the example UW-Madison Thesis document
% It demonstrates one method for incorporating program listings
% into a document.

\group{Model stuff} \label{merp} 
This is an example of a Matlab m-

				% Appendix: Verification Work
% code.tex
% this file is part of the example UW-Madison Thesis document
% It demonstrates one method for incorporating program listings
% into a document.
\chapter{BLARPER}

Test Test Test 
\pagebreak


% code.tex
% this file is part of the example UW-Madison Thesis document
% It demonstrates one method for incorporating program listings
% into a document.
\chapter{Model stuff}

Test Test Test 

\begin{landscape}
\begin{IEEEeqnarray}{lCr}
\begin{bmatrix}
  \begin{matrix} 
    x & x & x & 0 & 0 & x \\ 0 & x & 0 & x & x & x \\ x & x & x & 0 & 0 & x \\ 0 & x & 0 & x & 0 & x \\ 0 & x & 0 & x & x & x \\ x & x & x & 0 & 0 & x \\
    0 & 0 & 0 & 0 & 0 & x \\ 0 & 0 & 0 & 0 & 0 & x \\ 0 & 0 & 0 & 0 & 0 & x \\
    0 & 0 & 0 & 0 & 0 & 0 \\ 0 & 0 & 0 & 0 & 0 & 0 \\ 0 & 0 & 0 & 0 & 0 & 0 \\ 0 & 0 & 0 & 0 & 0 & 0 \\ 0 & 0 & 0 & 0 & 0 & 0 \\ 0 & 0 & 0 & 0 & 0 & 0
  \end{matrix} 
& \begin{matrix} 
    0 & \frac{\partial F_{J  ,1}}{\partial M_{g}} & 0 	\\ \frac{\partial F_{J,2}}{\partial M_{l}} & 0 & 0 	\\ 0 & \frac{\partial F_{J,3}}{\partial M_{g}} 	 & 0 	\\ \frac{\partial F_{J,4}}{\partial M_{l}} & 0 & \frac{\partial F_{J,4}}{\partial M_{e}} 	\\ 0 & 0 & \frac{\partial F_{J,5}}{\partial M_{e}} 	\\ \frac{\partial F_{J,6}}{\partial M_{l}} & 0 & 0 \\
    x & x 					  & 0 	\\ x 					   & x & x	\\ 0 & x 					 & x	\\ 
    0 & \frac{\partial F_{J+1,1}}{\partial M_{g}} & 0 	\\ \frac{\partial F_{J+1,2}}{\partial M_{l}} & 0 & 0 	\\ 0 & \frac{\partial F_{J+1,3}}{\partial M_{g}} & 0 	\\ \frac{\partial F_{J+1,4}}{\partial M_{l}} & 0 & \frac{\partial F_{J+1,4}}{\partial M_{e}} 	\\ 0 & 0 & \frac{\partial F_{J+1,5}}{\partial M_{e}} 	\\ \frac{\partial F_{J+1,6}}{\partial M_{l}} & 0 & 0 
  \end{matrix} 
& \begin{matrix}
    0 & 0 & 0 & 0 & 0 & 0 \\ 0 & 0 & 0 & 0 & 0 & 0 \\ 0 & 0 & 0 & 0 & 0 & 0 \\ 0 & 0 & 0 & 0 & 0 & 0 \\ 0 & 0 & 0 & 0 & 0 & 0 \\ 0 & 0 & 0 & 0 & 0 & 0 \\
    0 & 0 & 0 & 0 & 0 & x \\ 0 & 0 & 0 & 0 & 0 & x \\ 0 & 0 & 0 & 0 & 0 & x \\
    x & x & x & 0 & 0 & x \\ 0 & x & 0 & x & x & x \\ x & x & x & 0 & 0 & x \\ 0 & x & 0 & x & 0 & x \\ 0 & x & 0 & x & x & x \\ x & x & x & 0 & 0 & x
  \end{matrix}\\
\end{bmatrix}
  \begin{bmatrix}
    \delta(\alpha_g P_{nc}) \\ \delta \alpha_{g} \\ \delta(\alpha_{g} H_v) \\ \delta((1-\alpha_{g}) H_l) \\ \delta \alpha_e \\ \delta P \\ \delta M_l \\ \delta M_g \\ \delta M_e \\ \delta (\alpha_g P_{nc}) \\ \delta \alpha_{g} \\ \delta(\alpha_{g} H_v) \\ \delta((1-\alpha_{g}) H_l) \\ \delta \alpha_e \\ \delta P
  \end{bmatrix}
  & = & 
 -\begin{bmatrix}
    F_{J,1} \\ F_{J,2} \\ F_{J,3} \\ F_{J,4} \\ F_{J,5} \\ F_{J,6} \\ F_{j,1} \\ F_{j,2} \\ F_{j,3} \\ F_{J+1,1} \\ F_{J+1,2} \\ F_{J+1,3} \\ F_{J+1,4} \\ F_{J+1,5} \\ F_{J+1,6}
  \end{bmatrix}
\end{IEEEeqnarray}

\begin{IEEEeqnarray}{lCr}
\begin{bmatrix}
%
   \begin{matrix} x & x & x & 0 & 0 & x \\ 0 & x & 0 & x & x & x \\ x & x & x & 0 & 0 & x \\ 0 & x & 0 & x & 0 & x \\ 0 & x & 0 & x & x & x \\ x & x & x & 0 & 0 & x \end{matrix}
 & \begin{matrix} 0 & x & 0             \\ x & 0 & 0             \\ 0 & x & 0             \\ x & 0 & x             \\ 0 & 0 & x             \\ x & 0 & 0             \end{matrix}
 & \mat{0} \\
%
   \begin{matrix} 0 & 0 & 0 & 0 & 0 & x \\ 0 & 0 & 0 & 0 & 0 & x \\ 0 & 0 & 0 & 0 & 0 & x \end{matrix}
 & \mat{J}^{*}
 & \begin{matrix} 0 & 0 & 0 & 0 & 0 & x \\ 0 & 0 & 0 & 0 & 0 & x \\ 0 & 0 & 0 & 0 & 0 & x \end{matrix}\\
%
   \mat{0}
 & \begin{matrix} 0 & x & 0             \\ x & 0 & 0             \\ 0 & x & 0             \\ x & 0 & x             \\ 0 & 0 & x             \\ x & 0 & 0             \end{matrix}
 & \begin{matrix} x & x & x & 0 & 0 & x \\ 0 & x & 0 & x & x & x \\ x & x & x & 0 & 0 & x \\ 0 & x & 0 & x & 0 & x \\ 0 & x & 0 & x & x & x \\ x & x & x & 0 & 0 & x \end{matrix}\\
\end{bmatrix}
\begin{bmatrix}\delta(\alpha_g P_{nc}) \\ \delta \alpha_{g} \\ \delta(\alpha_{g} H_v) \\ \delta((1-\alpha_{g}) H_l) \\ \delta \alpha_e \\ \delta P \\ \delta M_l \\ \delta M_g \\ \delta M_e \\ \delta (\alpha_g P_{nc}) \\ \delta \alpha_{g} \\ \delta(\alpha_{g} H_v) \\ \delta((1-\alpha_{g}) H_l) \\ \delta \alpha_e \\ \delta P\end{bmatrix}
& = &
-\begin{bmatrix} F_{J,1} \\ F_{J,2} \\ F_{J,3} \\ F_{J,4} \\ F_{J,5} \\ F_{J,6} \\ F_{j,1} \\ F_{j,2} \\ F_{j,3} \\ F_{J+1,1} \\ F_{J+1,2} \\ F_{J+1,3} \\ F_{J+1,4} \\ F_{J+1,5} \\ F_{J+1,6}\end{bmatrix}
\end{IEEEeqnarray}

\begin{IEEEeqnarray}{lCr}
\begin{bmatrix}
%
\begin{matrix} x & x & x & 0 & 0 & x \\ 0 & x & 0 & x & x & x \\ x & x & x & 0 & 0 & x \\ 0 & x & 0 & x & 0 & x \\ 0 & x & 0 & x & x & x \\ x & x & x & 0 & 0 & x \end{matrix}
 & \begin{matrix} 0 & x & 0 \\ x & 0 & 0 \\ 0 & x & 0 \\ x & 0 & x \\ 0 & 0 & x \\ x & 0 & 0 \end{matrix}
 & \begin{matrix} 0 \end{matrix}\\
%
\begin{matrix} 0 & 0 & 0 & 0 & 0 & x \\ 0 & 0 & 0 & 0 & 0 & x \\ 0 & 0 & 0 & 0 & 0 & x \end{matrix}
 & \mat{J}^{*}
 & \begin{matrix} 0 & 0 & 0 & 0 & 0 & x \\ 0 & 0 & 0 & 0 & 0 & x \\ 0 & 0 & 0 & 0 & 0 & x \end{matrix}\\
%
\begin{matrix} 0 \end{matrix}
 & \begin{matrix} 0 & x & 0 \\ x & 0 & 0 \\ 0 & x & 0 \\ x & 0 & x \\ 0 & 0 & x \\ x & 0 & 0 \end{matrix}
 & \begin{matrix} x & x & x & 0 & 0 & x \\ 0 & x & 0 & x & x & x \\ x & x & x & 0 & 0 & x \\ 0 & x & 0 & x & 0 & x \\ 0 & x & 0 & x & x & x \\ x & x & x & 0 & 0 & x \end{matrix}\\
\end{bmatrix} \begin{bmatrix}\delta(\alpha_g P_{nc}) \\ \delta \alpha_{g} \\ \delta(\alpha_{g} H_v) \\ \delta((1-\alpha_{g}) H_l) \\ \delta \alpha_e \\ \delta P \\ \delta M_l \\ \delta M_g \\ \delta M_e \\ \delta (\alpha_g P_{nc}) \\ \delta \alpha_{g} \\ \delta(\alpha_{g} H_v) \\ \delta((1-\alpha_{g}) H_l) \\ \delta \alpha_e \\ \delta P\end{bmatrix}& = & -\begin{bmatrix} F_{J,1} \\ F_{J,2} \\ F_{J,3} \\ F_{J,4} \\ F_{J,5} \\ F_{J,6} \\ F_{j,1} \\ F_{j,2} \\ F_{j,3} \\ F_{J+1,1} \\ F_{J+1,2} \\ F_{J+1,3} \\ F_{J+1,4} \\ F_{J+1,5} \\ F_{J+1,6}\end{bmatrix}
\end{IEEEeqnarray}

\begin{IEEEeqnarray}{lCr}
\begin{bmatrix}
\begin{matrix} x & x & x & 0 & 0 & x \\ 0 & x & 0 & x & x & x \\ x & x & x & 0 & 0 & x \\ 0 & x & 0 & x & 0 & x \\ 0 & x & 0 & x & x & x \\ x & x & x & 0 & 0 & x \end{matrix} & \begin{matrix} 0 & x & 0 \\ x & 0 & 0 \\ 0 & x & 0 \\ x & 0 & x \\ 0 & 0 & x \\ x & 0 & 0 \end{matrix} & \begin{matrix} 0 \end{matrix}\\
\begin{matrix} 0 & 0 & 0 & 0 & 0 & -\frac{d M_l}{d P} \\ 0 & 0 & 0 & 0 & 0 & -\frac{d M_g}{d P} \\ 0 & 0 & 0 & 0 & 0 & -\frac{d M_e}{d P} \end{matrix} & \mat{I} & \begin{matrix} 0 & 0 & 0 & 0 & 0 & \frac{d M_l}{d P} \\ 0 & 0 & 0 & 0 & 0 & \frac{d M_g}{d P} \\ 0 & 0 & 0 & 0 & 0 & \frac{d M_e}{d P} \end{matrix}\\
\begin{matrix} 0 \end{matrix} & \begin{matrix} 0 & x & 0 \\ x & 0 & 0 \\ 0 & x & 0 \\ x & 0 & x \\ 0 & 0 & x \\ x & 0 & 0 \end{matrix} & \begin{matrix} x & x & x & 0 & 0 & x \\ 0 & x & 0 & x & x & x \\ x & x & x & 0 & 0 & x \\ 0 & x & 0 & x & 0 & x \\ 0 & x & 0 & x & x & x \\ x & x & x & 0 & 0 & x \end{matrix}\\
\end{bmatrix} \begin{bmatrix}\delta(\alpha_g P_{nc}) \\ \delta \alpha_{g} \\ \delta(\alpha_{g} H_v) \\ \delta((1-\alpha_{g}) H_l) \\ \delta \alpha_e \\ \delta P \\ \delta M_l \\ \delta M_g \\ \delta M_e \\ \delta (\alpha_g P_{nc}) \\ \delta \alpha_{g} \\ \delta(\alpha_{g} H_v) \\ \delta((1-\alpha_{g}) H_l) \\ \delta \alpha_e \\ \delta P\end{bmatrix}& = & -\begin{bmatrix} F_{J,1} \\ F_{J,2} \\ F_{J,3} \\ F_{J,4} \\ F_{J,5} \\ F_{J,6} \\ F_{j,1} \\ F_{j,2} \\ F_{j,3} \\ F_{J+1,1} \\ F_{J+1,2} \\ F_{J+1,3} \\ F_{J+1,4} \\ F_{J+1,5} \\ F_{J+1,6}\end{bmatrix}
\end{IEEEeqnarray}

\begin{IEEEeqnarray}{lCr}
\begin{bmatrix}
\begin{matrix} x & x & x & 0 & 0 & x-\frac{d M_g}{d P} \\ 0 & x & 0 & x & x & x-\frac{d M_l}{d P} \\ x & x & x & 0 & 0 & x-\frac{d M_g}{d P} \\ 0 & x & 0 & x & 0 & x-\frac{d M_l}{d P}-\frac{d M_e}{d P} \\ 0 & x & 0 & x & x & x-\frac{d M_e}{d P} \\ x & x & x & 0 & 0 & x-\frac{d M_l}{d P} \\ 0 & 0 & 0 & 0 & 0 & \frac{d M_g}{d P} \\ 0 & 0 & 0 & 0 & 0 & \frac{d M_l}{d P} \\ 0 & 0 & 0 & 0 & 0 & \frac{d M_g}{d P} \\ 0 & 0 & 0 & 0 & 0 & \frac{d M_g}{d P} + \frac{d M_g}{d P} \\ 0 & 0 & 0 & 0 & 0 & \frac{d M_e}{d P} \\ 0 & 0 & 0 & 0 & 0 & \frac{d M_l}{d P}\end{matrix}  & \begin{matrix} 0 & 0 & 0 & 0 & 0 & \frac{d M_g}{d P} \\ 0 & 0 & 0 & 0 & 0 & \frac{d M_l}{d P} \\ 0 & 0 & 0 & 0 & 0 & \frac{d M_g}{d P} \\ 0 & 0 & 0 & 0 & 0 & \frac{d M_g}{d P} + \frac{d M_g}{d P} \\ 0 & 0 & 0 & 0 & 0 & \frac{d M_e}{d P} \\ 0 & 0 & 0 & 0 & 0 & \frac{d M_l}{d P} \\ x & x & x & 0 & 0 & x-\frac{d M_g}{d P} \\ 0 & x & 0 & x & x & x-\frac{d M_l}{d P} \\ x & x & x & 0 & 0 & x-\frac{d M_g}{d P} \\ 0 & x & 0 & x & 0 & x-\frac{d M_l}{d P}-\frac{d M_e}{d P} \\ 0 & x & 0 & x & x & x-\frac{d M_e}{d P} \\ x & x & x & 0 & 0 & x-\frac{d M_l}{d P}\end{matrix}\\
\end{bmatrix}\begin{bmatrix}\delta(\alpha_g P_{nc}) \\ \delta \alpha_{g} \\ \delta(\alpha_{g} H_v) \\ \delta((1-\alpha_{g}) H_l) \\ \delta \alpha_e \\ \delta P \\ \delta (\alpha_g P_{nc}) \\ \delta \alpha_{g} \\ \delta(\alpha_{g} H_v) \\ \delta((1-\alpha_{g}) H_l) \\ \delta \alpha_e \\ \delta P\end{bmatrix}
\end{IEEEeqnarray}
\end{landscape}
\pagebreak


\end{appendices}

%=======================================================================
% End Document
%=======================================================================
\end{document}